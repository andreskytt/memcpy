\index[ppl]{Loit, Kaspar}
\index[ppl]{B'Knows}
\index[ppl]{B'Knows|see{Loit, Kaspar}}

\question{Kes sa oled?}

Mina olen Kaspar. Ja kunagi, kuna see võtame selle teema, et me peame tagasi kerima mingisugune miljard aastat, siis toona see aka oli B'Knows. 

\question{Aga kust sa said sihukese aka?}

Seda ei mäleta enam keegi. Seal on nagu kaks komponenti. Üks on nagu \enquote{B} ja siis on nagu \enquote{knows}, ehk siis see B peaks  midagi nagu teadma. Pronto\index[ppl]{Pronto} alati kutsus mind Buttknows.

\question{Kuidas sina arvutite juurde said või arvutid said sinu juurde?}

Mul on selge mälupilt, et mu tädi, kes on superkuul ja minust mõnevõrra vanem, töötas Tartus
vist Bioloogia Instituudis. Ja  talle oli kuidagi jäänud mulje, et mind võivad huvitada sellised asjad. Ma arvan, et ma olin mingi, ma ei tea, kaheksa-üheksa-kümme, \emph{something like that}. Ja siis, kui ma tal ükskord Tartus külas käisin, ta viis mind instituuti. Muidugi peale tööd, kõik oli juba pime ja  seal oli mingi kabinet lahti ja seal seisis laua peal mingisugune masin, mille nimi oli Apple II Europlus\index{Arvutid!Apple II}\sidenote{Apple II Europlus oli Apple Euroopa turule kohandatud versioon. Muu hulgas erines toiteblokk aga ka video osa tuli ümber teha, sest Steve Wozniaki trikid NCTS signaali genereerimisel keerukama PAL süstemi puhul enam ei toiminud.}. See oli \emph{freaking awesome}. Ta oli seal mingi laborandi käest küsinud, et kuidas seal midagi käib, ja sai laadida paar üli superägedat mängu, mis tekstiekraanil jooksid. Üks oli vist \emph{Train robbery}\index{Mängud!Train Robbery}, ma mäletan, mul see pilt on täitsa silme ees. Ja sellest hetkest ma arvan, ma olin müüdud ka. Ma ei oska  meenutada, kas ma enne olin kokku puutunud ka arvutitega aga tõenäoliselt mitte, see oli ikkagi liiga  vara. 

Kuna mulle tohutult meeldisid koolis Nintendo väiksed Game \& Watch\index{Nintendo Game \& Wathc} mängud, võib-olla mäletad. Ühesõnaga sihukesed, tänapäeva telefoni suurused umbes. Neil oli LCD ekraan, millesse oli ette joonistatud mingisugused tegelased ja siis nuppudega said mängu mängida seal ekraanil\sidenote{Vaata ka märkust \ref{sidenote!gameandwatch} leheküljel \pageref{sidenote!gameandwatch}.}. Ja  ma kuidagi mõtlesin, et oh, kui lahe oleks, kui saaks ise niisuguseid teha, aga no ma sain aru, et seal taga on mingi tootmine ja see ei ole nagu reaalne. Ja nüüd järsku saada aru, et selliseid asju on võimalik sinna masina sisse programmeerida ilma  et sa pead mingit elektroonikaskeemi tootma, et sul ei pea üldse mingit tehast olema. See oli üks niisugune \emph{revalation}, on ju. See muidugi  tõenäoliselt viis mind kunagi ka selleni, et me vaikselt tegime mänge. 

Aga niisugune päris toimetamine hakkas ilmselt  tööle kuskil seoses Jaak Loondega\index[ppl]{Loonde, Jaak}. Ta oli kindlasti ülioluline tegelane, sest sel ajal oli arvutile ligipääs oluline asi. Ja ma mäletan, et ma tegelikult olin kaardistanud omale kõik kohad, kus üldse  tõenäoliselt Eesti Vabariigis arvutitele ligi pääses. Nõo oli liiga kaugel selgelt. Aga Tallinnas neid kohti ikka oli. 

Aga Jaak Loonde\index[ppl]{Loonde, Jaak} oli selles mõttes lahe tegelane, et minu meelest vist tema kaudu ma esimest korda sain midagi progeda. 

\question{Kas ta siis andis selle võimaluse või ikka õpetas ka?}

Ta ikka õpetas loomulikult. Mingil põhjusel, ma ei tea, kuidas ma sattusin  3. Keskkooli\index{Koolid!Tallinna 3. Keskkool}, kus oli sihuke suur klass mingisuguste masinatega, tõenäoliselt ka need olid MSXid\index{Arvutid!Yamaha MSX}. Seal BASIC\index{Keeled!BASIC} oli ees, sa võisid seal midagi hakata klõpsima. MSX selle selle pärast  üldse oli pull masin, et  ta tegelikult vist mõeldi välja, tagantjärgi tarkusega võin öelda, selleks, et ühtlustada koduarvutite standardit ja BASICuid, mida nad jookseksid.  See initsiatiiv oli isegi tegelikult  mingisuguse Jaapani Microsofti \emph{executiv}'i poolt.

\question{Ta buutis BASICusse otse, eks?}

Jah. Sa põhimõtteliselt hakkasidki peale niimoodi, et ta tõmbab selle  selle ekraani ette ja sa võid seal kirjutada \verb|10| ja siis kirjutada ühe rea. Kirjutad \verb|20|, kirjutad teise rea, ütled \verb|list|, siis ta  näitab, mis sul on. Kirjutad uuesti \verb|20|, kirjutad selle rea üle ja kirjutad \verb|run| ja \emph{that's that}, ta sul kohe käib. 

Seal oli kari tegelasi, paari tükki ma tundsin ja nende kaudu ma vist kuidagi sain sellest klassist teada. Ma mäletan, et üks mu koolikaaslane nägi sihukest masinat esimest korda ja meile öeldigi, et hakake midagi tegema. No ja siis too koolikaaslane kirjutas \verb|please draw me a circle|. Ütleme, et NLP\sidenote{\emph{Natural Language Processing - NLP}} ei olnud veel nii kaugel ja sealt midagi ei tulnud.

Seal üldse oli palju rahvast, Karel Kannel\index[ppl]{Karel Kannel} seal kuidagi toimetas ja kõik need väiksed tattnokad olid seal kõik koos. Aga palju huvitavam oli tegelikult see, et Jaak Loondel\index[ppl]{Loonde, Jaak} oli ka üks masin, nimi oli MIR-2\index{Arvutid!MIR-2}\sidenote{\begin{russian}МИР\end{russian} oli varane Nõukogude miniarvutite sari, mille kolm generatsiooni (MIR, MIR-1 ja MIR-2) töötati välja aastatel 1965-1969. Sarja nimi oli lühend pikemast nimest \begin{russian}Машина для Инженерных Расчётов\end{russian} (Inseneriarvutuste Masin).}, mingisugune nõukogude aparaat. See masin oli põhimõtteliselt sihuke pikk kapp, ikka pikem kui viis meetrit, poisikesest oli ta kõrgem, Jaagule võib-olla ninani. Masin tegi meeletut häält, põhiline osa ilmselt oli jahutusel. Ta tegi sihukest villa kraasimise masina häält, ilge müra. Aeg-ajalt, kui inimestel viskas nagu kopa ette, siis nad lülitasid masina jahutuse välja ja siis Jack\index[ppl]{Jack}\index[ppl]{Jack|see{Loonde, Jaak}} tuli muidugi ja tohutult röökis, sest oleks seal kõik kokku küpsenud. 

Selles mõttes oli ta ka nagu geniaalne vana, et kust ta selle masina üldse kätte oli saanud, see oli \emph{advanced} masin. Tal oli ikkagi võimalik klaviatuurist sisestada käske, kusjuures klaviatuur oli elektronkirjutusmasin, mis  põhimõtteliselt oli nagu klaver ja printer ühekorraga. Nii et \emph{hard copy} tuli ka samast masinast. Ja temaga oli ühendatud mustvalge telekas aga tal oli valguspliiats,  ehk siis sa said nagu ekraanil tabada mingisuguseid punkte ja  masin tundis selle ära. Ilmselt ta luges seda  kineskoobi kiirt ja selle järgi pani asukoha kokku. Ja ta suutis ka mingisugust rudimentaarset graafikat kuvada, ehk et tal ei olnud ainult teksti ekraan, vaid ta suutis ekraanile kuvada mingit punkti. See oli arvuti jaoks tohutu ülesanne, see punkti ei püsinud hästi paigal, ta õrnalt ujus, aga ta sai hakkama. 

See programmeerimiskeel oli vene keeles, vene tähtedega, kõik olid mingid lühendid, superluks, onju. Mu esimene programm, ma mäletan, oli mingisugune graafiline nelja-tipuline täht. Ma arvan, et ta koosnes ütleme siis  umbes kuueteistkümnest punktist võibolla ja see ikka tõmbas selle arvuti ikka täiesti kooma. Kõik see ekraan ujus seal aga väga uhke oli. 

Aga mida Jack\index[ppl]{Loonde, Jaak} meile veel õpetas, oli näiteks perfolindi lugemist. Masin sõi kahte moodi meediat. Üks oli paber, perfolint, siuke õhukene, mis lasti vurinal masinast läbi. Teoorias teravamad vennad suutsid torkida nõela või augurauaga  perfolindi peale proge, ma arvan. Mul on selline tunne, et äkki oskasin seda kunagi. Aga aga siis olid seal veel mingisugused vahvad magnetkaardid. 

\question{Magnetkaart? Perfokaarti ma tean aga magnetkaardist ei ole kuulnud}

Magnetkaart oli selline  tänapäeva  telefonist suurem, ma arvan, et mingi kaheksa senti korda mingi viisteist senti sihukene pruun latakas, mis põhimõtteliselt meenutada oma materjalilt seda, mis flopi diski sees on tegelikult. Ja sa põhimõtteliselt panid selle mingist \emph{slot}'ist sisse, ta tõmbas selle surtsti läbi ja luges sealt midagi mingisuguses koguses. Aga see oli nagu sihuke müstika, seda enam niisama lugeda ei saadud. 

\question{Noorel nagamannil peab olema ikka päris änksa tahtmine, et ekraani peale tähe joonistamine huvitav oleks?}

See oli nagu müstiliselt äge. Tõesti sa nagu kirjutasid midagi ja see pilt tekkis sulle  sinna ekraanile.

\question{Ja siis see huvitav asi oli just see, et mina andsin käsu ja masin tegi midagi?}

Noh, täpselt, et kui see oleks nii lihtne, et \verb|please draw me a circle|, siis  ilmselt oleks kaotanud huvi, aga see oli ikkagi \emph{complicated} värk. Selles oli mingi alkeemiline element, see oli ülikõva. Väikestele poistele meeldivad igasugused salakeeled ja igasugused koodid ja ma ei tea lipukirjad ja igsusugsed niisugused asjad. See  oli nagu kõike seda  ja veel midagi,  see oli ikkagi super noh.

Selge oli see, et too masin oli  meeletult piiratud, et kaua sa seal ikka jändad. Ja lisaks see, et seda MIRi oli ainult uuks, õnneks MSXi klass oli suurem. Aga klassiga olid vist jälle mingisugused piirangud, kuna see oli keskkooli all. Ja ilmselt sellepärast Jack\index[ppl]{Loonde, Jaak} sebiski Roopa tänavale selle ÕTK\index{Tallinna Oktoorbriajooni Õppetootmiskombinaat}\sidenote{Täpsemalt Tallinna Oktoorbrirajooni Õppetootmiskombinaat. Sellest asutusest on natuke rohkem juttu leheküljel \pageref{content!OTK}.}, kus oli siis ka terve klass. ÕTKs oli  üks niisugune nagu juhtarvuti, millel oli mingisugune draiv (ma eeldan, et see oli mingi flopidraiv) ja terve klassitäis arvuteid, mis said sellest peaarvutist omale asju alla laadida. Või sa võisid ka seal lihtsalt kirjutada oma neid programme. Aga kuna kuna draive oli ainult üks, siis kui sa tahtsid salvestada või midagi,  siis sa pidid selle juhtarvutisse saatma. 

Tavaliselt see tegevus oli üsna nagu lihtne, et. Kuna Jackil\index[ppl]{Loonde, Jaak} ei olnud aega seal klassis väga nagu hängida, siis oli seal kogu aeg mingi poistekari ja tal oli paar nutikamat vend pandud seda vedama. Üks legendaarne tegelane oli Mukats\index[ppl]{Mukats|see{Edesi, Linnar}} ehk Linnar Edesi\index[ppl]{Edesi, Linnar}, kes täna vist kuskil Soomes toimetab. Tema oli selgelt minu esimene guru, keda ma nägin. Ta oli, ma arvan, umbes minuvanune, aga ta oli omandanud kõik need peenemad alged. Põhiline, mida ta oskas oli see, et ta oskas kahest programmijupist panna kokku ühe terviku ja  paketeerida selle nii, et seda sai laadida. Point oli selles, et väga suur osa softi levis tavalistel magnetofoni kassettidel. Ja vist oli see kuidagi nii, et šeffimad mängud olid  32 kilobaiti umbes pikad, noh see oli ikka \emph{massive}. Aga selleks, et nad kasseti peale ära mahuks, olid nad tehtud pooleks: 16+16 kilo, sa pidid vahepeal kasseti ümber keerama. Ühesõnaga kogu see kassetimajandus oli keeruline. Aga kui sul oli juba nii kõva asi nagu flopidraiv, siis sa said selle kasseti pealt lugeda selle kuusteist kilo sisse, tõsta ta kuskil mälus mujale ja siis lugeda teise kuusteist kilo ja esimese jupiga kokku panna. Tekkis  tervik, mida sai \emph{launch}'ida kuidagimoodi. Ühesõnaga see kõik oli täielik supermaagia. 

\question{Järelikult tekkis sul niisugune teadmistel põhinev eeskuju. Keegi inimene, kellele sa vaatasid alt üles, sest ta teadis rohkem, kui sina?}

Oo jaa, selliseid tegelasi oli veel. Üks vend, kelle nimi oli Kont\index[ppl]{Kont} (ma ei mäleta, mis ta eesnimi oli), tal oli väikene metallkohvrikene, mille sees oli kogu MSX'i manual. See oli fotokopeeritud sihuke \emph{stack of paper}. Ta käis sellega ringi väga uhkelt, aegajalt tegi lahti ja siis midagi selle alusel kirjutas, see nagu superluks. Sihukesi tegelasi oli veel seal. 

Kuna sa istusid seal ja sul otsest \emph{access}'i sellele draivile ei olnud, võib-olla  mängisid, siis mingi hetk tüütas see ära, siis kirjutasid oma BASICut\index{Keeled!BASIC}. Sellele tegevusele tulid mingid piirid ette. Loomulikult mind huvitas graafika pool, ma üritasin mingisuguseid pilte ekraanile manada. MSX'il\index{Arvutid!Yamaha MSX}  tegelikult ei olnud graafika ekraani,  seda emuleeriti tekstiekraaniga. Ehk siis  põhimõtteliselt iga pilt, iga mäng, mis MSX'il toona jooksis, oli tegelikult otse mälus tähegeneraatorei ümber programmeerimine. 

\question{Jukuga\index{Arvutid!Juku} oli sama lugu, et sa said kuskile mällu mingisuguse oma nii-öelda fondi laadida. Iga tähe asemele panid bitmapi ja nendest sai mida tahes kokku laduda}

Just, põhimõtteliselt sama laks. Ekraan põhimõtteliselt MSX'il oli otse aadresseeritav. Kui sa teadsid, et selle režiimi ekraan algas aadressil heksas \emph{whatever-whatever} onju, siis sa said sinna järjest kirjutada, iga bait oli üks rida, ja võis olla kas  läbipaistev, taustavärv või esivesivärv ja mingites režiimides sai rida-realt neid värve vahetada, see oli eriti \emph{advanced}. Ehk siis põhimõtteliselt niisugune mõiste, ma ei tea, kas sa oled kokku puutunud, et mängudes on sihuksed tegelikult nagu \enquote{spraidid}, need on need, mis tausta ees liiguvad, eks ole. MSX'il spaite ka emuleeriti sellesama tähegenega, ehk siis kohalik tähegene programmeeriti jooksvalt ringi. Ekraan kirjutati sümboleid täis, mis  pähe tulid, ja siis need kogu aeg adresseeriti ja kirjutajati ringi. Ekraan vist oli jagatud kolmeks osaks, igas osas sa said eri tähestiku väänata.

Minu jaoks oligi see võlu, kui ma sain selgeks, et on olemas Assembler\index{Keeled!Assembler}, Assembleris saab laadida ühele mälu aadressile ühe baidi ja siis ma põhimõtteliselt veetsin suure osa oma ärkveloleku ajast millimeetripaberile joonistades mingisuguseid tegelasi, neid heksaks tõlkides ja kuskile mällu laadides. 

\question{Kui ma nüüd tagasi peegeldan, siis ega tänapäevane arvutigraafika ei ole ka lihtne aga keerukus tundub olevat teises kohas. Sa pead 3D geomeetriast aru saama ja spetsiifilisi APIsid tundma jne.}

No täna ikkagi \emph{layers of stuff} on sinu ja pildi vahel, aga MSX'il oli just see, et sa põhimõtteliselt sa toorelt toppisid otse ekraanile midagi. 

\question{Sa pidid ikkagi välja mõtlema selle, mida sinna mällu toppida, ja hoolitsema, et see värskendatud saaks ja nii edasi}

Ütleme, et seal oli igasugused trikid, et ta töötaks. Aga kuna keelestik ja kõik see asi oli nii lihtne, oligi tulemus super elegantne, super  lihtne.  Ma arvan, et ega minu progemise aeg jäigi sinna kaheksakümnendate keskele. Pärast ma olen võib-olla natuke HTMLi ja võib-olla CSSi nokkinud, aga see võlu läks nagu üle kohe, kui asjad läksid keerulisemaks. Aga õnneks tulid tasemele igasugused graafikapaketid ja ja muud asjad. 

\question{Aga millal see oli? Juba keskkooli ajal?}

Olles enda jaoks kaardistanud ära kõik kohad, kus sai midagi arvutitega näppida, jõudsin ma läbi

Väga tähtis on kusjuures see, et TPI's\index{Tallinna Tehnikaülikool} oli ka üks klass, kus olid MSX'id\index{Arvutid!Yamaha MSX}. Aga seal oli igal masinal juba drive taga. See oli ka juba super \emph{advanced} ja seal see guru staatus oli ka nagu juba järgmisele levelil. Seal olid  laborandid,
Aare Tali\index[ppl]{Tali, Aare} nimi tuleb kuidagi ette aga ma ei ole kindel, kas see on õige nimi õige näo juures\sidenote{Aare tegutses TTÜs küll ja temast on ka varasemalt juttu olnud (vt. lk. \pageref{sisu!aare_tali})}. Aga igal juhul olid seal mingisugust juba üliõpilased,  palju kõvemad vennad või isegi juba \emph{post-graduate} või mis iganes ja loomulikult see nagade jada seal ukse taga neid selgelt tüütas. Nad tegid siis oma mingisuguseid reegleid, olid suured jumalused. Näiteks mingi hetk oli, kui info levis ja järjest rohkem kutte tekkis sinna värava taha,  oli vaja reglementeerida, et kes saab ligi. Siis nad võtsid ühe kõige popima mängu, mis seal parasjagu oli, Kings Valley\index{Mängud!King's Valley}, trükkisid välja kogu selle \emph{source}koodi. See oli \emph{stack of paper}, ilusti perforeeritud ja niimoodi. Ja siis nad lugesid seda, punase pastakaga tõmbasid ringe ja progesid selle mängu ringi. Minu jaoks oli see nagu jumaluse tase! Nad progresid selle niimoodi ringi, et nad said väikeste ise-ehitatud \emph{joystick}'idega juhtida selle mängu kolle. Reegel oli põhimõtteliselt see, et kui sa said nende jumalate vastu ühe leveli läbi siis sa said ühe päeva klassis käia. Iseenesest see mängufaktor ja see kõik oli põnev aga just see, et nad tõesti võtsid selle mängu, mis minu jaoks tundus superkeeruline ja lihtsalt kirjutasid \emph{binary} ringi. Nad mitte lihtsalt ei teinud seal mingile tegelasele mütsi pähe,  vaid nad lihtsalt nagu tegidki kõik ringid, käitumine muutus. Tänapäeval muidugi tagasi vaadates tundub, et see kõik oli tegelikult väga lihtne.

\question{Eks see oli ju \emph{gamification}, mis praegugi populaarne on ja üksiti kindlustas veelgi lugupeetavate jumala-staatust poistekamba silmis}

Aga igal juhul lõppkokkuvõttes ma lõpetasin kuskil Kullos\index{Kullo}, kus oli ka üks klass, kus olid vist juba natukene kõvemad MSX'id. Seal oli juba  mingi graafikarežiim ja igasugused muud asjad, ehkki, kui sa midagi kiiresti liigutada tahtsid, pidid ikkagi kasutama tekstiekraani. Toda klassi majandas selline legend nagu Räni Meister\index[ppl]{Meister, Räni}, kestoona oli selgelt sihuke tore punkar, kes oli tulnud kuskilt Valga Gaasianalüsaatorite tehasest umbes ja seal viitsis poistega jahmerdada. Aga taa vaikselt seal hakkas tegelema Commodore Amigadega\index{Arvutid!Amiga}\sidenote{Amiga oli Commodore poolt 1985. aastal turule toodud personaalarvutite sari. Teistest põlvkonnakaaslastest eristas seda perekonda spetsiaalse graafika- ja heliriistvara lisamine ning väljatõrjuva mitmetegumilisuse realiseerinud AmigaOS}, mis oli juba sihukene \emph{super advanced} raud. 

Kuidagi mahtus see kõik videotootmise ja selliste asjade tähe alla. Ja tänu sellele ta oli ka loomulikult siis on välja kaardistanud, et kus, kus niisugune asi veel toimub. Ja Eesti Televisioon\index{Eesti Rahvusringhääling!Eesti Televisioon} oli selgelt üks ja veel oli mingisuguste vene metalliärikate turundusharu. Ilmselt keegi vend oli piisavalt palju lobi teinud ja tal ei ole muud teha. Ta oli kuskil Kristiines  keldris püsti pannud väikse nii-öelda reklaamistuudio kus ta siis tootis värki ja tal oli seal ka üks Amiga\index{Arvutid!Amiga}. 

\question{See pidi siis olema üheksakümnendate algus juba, eks?}

Jah, kuskil sealkandis. Kullos me hakkasime ka juba mingeid mänge tegema ja nii. Markus Klessman\index[ppl]{Klessman, Margus} toimetas seal näiteks. Raul Keller\index[ppl]{Keller, Raul}, kelle aka oli \enquote{Killer}\index[ppl]{Killer|see{Keller, Raul}}, üritas MSXi mänge vist kuidagi publitseerida, aga see (vähemalt mulle ja toona) tundus kuidagi väga kahtlane ja naiivne tegevus.

Aga siis juba Räni, kuidagi nähes minus potentsiaali, meelitas mu Eesti Televisiooni\index{Eesti Rahvusringhääling!Eesti Televisioon} ja siis põhimõtteliselt ma ei olnud isegi veel keska viimases klassis, kui ma töötasin juba Aktuaalses Kaameras. Uudistetoimetuse kõrval oli sihuke väike kubrik, kus me siis tegime Aktuaalse Kaamera infonurki, mis olid  diktori taga seina peal. Ja kuna Amiga\index{Arvutid!Amiga} oli selline tore masin, et  sinna sai lasta videosignaali sisse ja sealt tuli videosignaal, sai seal teha digimiksi vist juba toona. 

\question{Ossa, see oli PC peal jõhkralt kallis riistvara toona}

Ongi. Miks need Amingad siinkandis selles vallas levisid, oli just see, et PC jaoks selliste võimalustega videokaart oli Hollywoodi tasemega asi. Ja PC'del oli enamasti, mingisugune CGA ja neli värvi onju, samas kui Amiga\index{Arvutid!Amiga} oli \emph{full video}. Põhimõtteliselt polnud sul vaja isegi arvuti monitori, sa võisid talle teleka järele panna ja see toimis. Ja see oligi ilmselt see point, miks tal oli videosignaal, et ta oli nagu kodukodutarbimisest arenenud selliseks. 

Seal tegime oma ilmakaarte ja panime videopilte sinna ja põhilise osa ajast muidugi mängisime arvutimänge, sest  Amigal olid super šefid mängud. 

\question{Aga mis tarkvaraga te tegite seda kõike? Ega te ju nullist ei kirjutanud kogu seda kraami?}

Olid olemas täitsa viisakad graafikapaketid, Delux Paint\index{Delux Paint}\sidenote{Deluxe Paint on rastergraafika redaktorite sari, mille lõi Electronic Arts'i jaoks Dan Silva. Programm alustas elu majasisese graafikaprogrammina kuid sai pärast avaldamist \emph{de facto} standardiks Amiga platvormil} on üks šefimaid graafikasofte, mis oli igasugustest Photoshoppidest ja kurat teab millest ikka kümme aastat ees. 

Me olime nagu sellised \emph{in-the-know} amiga-vennad ja vaatasime  kõikide PC'de  ja muude vendade peale ikka väga ülevalt alla, sest et nad ikka ei teadnud, milles nad seal sorkisid, onju. Paraku see Amiga bisnes oli kehva ja ta lõpuks läks nurja aga iseenesest see tehnika oli superäge. 

Meil tekkis mingi väikese   punt tegelastest, kellel kas oli kodus Amiga või kuidagi tegelesid. Näiteks Martin Rinne\index[ppl]{Rinne, Martin}, kes täna teeb Directot\index{Directo} tema juba kuidagi tekkis   sinna telesse ja siis Margus Kliimask\index[ppl]{Kliimask, Margus}, kes tegeles Eesti Videos\index{Eesti Video} Siilatsi mingite asjadega ja Mati Veermets\index[ppl]{Veermets, Mati} kellest pärast sai Tallinna linna  disainer  Kõigil oli nagu mingisugune \emph{access} Amigate juurde. 

Jällegi loomulikult seal ka võlus mind pigem see, et sa midagi seal nikerdasid ja sa tekitasid mingisuguse elava pildi, sul ei  pidanud olema kaamerad ja näitlejad ja mingi asi, vaid sa võisid teha mingeid väikseid  animatsioone otse arvutis teha.

\question{Isegi animatsiooni ta veads välja?}

Noh, selles mõttes, et sa said seda teha põhimõtteliselt \emph{stop motion}'iga. Ütleme,  reaalajas animatsioon kippus ikka nõgisema juba, kuigi me tegelikult ikka telepäid tegime reaalajas ka, sest keegi ei viitsinud \emph{stop motion}'iga lasta. Aga minu meelest ikka enamus sellised asju käisid ikkagi reaalajas, et Deluxe Paint'is\index{Deluxe Paint} olid sisse ehitatud igasugused nutikad asjad. Näiteks  liikumise aeglustamine või kiirendamine. Sa andsid talle põhimõtteliset ette, et siin on sul see kast, nüüd see kast peab liikuma mingisuguse viiekümne kaadriga siia, siis ta automaatselt täitis need viiskümmend kaadrit ära. Ja vajadusel, kui sa ütlesid, et \emph{ease in}, siis ta tõmbas lõpus hoo maha ja kõik oli väga \emph{fine}.  

Ma mäletan, kui ma alles läksin sinna telesse (selle järgi võib muidugi aasta paika panna), me tegime Öölaulupeole\index{Öölaulupidu}\sidenote{Esimene Öölaulupidu toimus 1987. aasta juunis Tallinna Vanalinna päevade ajal, aga toona meediakajastus puudus ja üritus toimus spontaanselt. 1988. aastal oli Öölaulupidu juba ametlikult Vanalinna päevade programmi lülitatud.}  mingisuguseid valgusklipple, see oli jällegi \emph{super advanced}.

\question{Võru poisina ma Öölaulupidudest ei tea midagi, aga Öötelevisioonil\sidenote{1990. aastal aset leidnud omas ajas mitmes mõttes innovatiivne teleprojekt, mille käigus Eesti Televisioon\index{Eesti Rahvusringhääling!Eesti Televisioon} öö läbi katkematult otse-eetris oli.} oli väga äge graafika}

Jajah, see oli ka meie tehtud. Tegelikult oli kogu tele  sihuksesed asjad meie rida, sest  alternatiiv oli tiitrimasin, mis oli mingi räme pool-analoog pult ja mis ikka eriti koledat jälge tootis. Meil oli ikka \emph{super-advanced} animatsioonidega ja värviline, sai teha mida iganes. Vahest tegime mingit haltuurat mingite reklaamide jaoks ja igasugu lollusi sai tehtud.

\question{Kas see oli puhas ise-õppimise värk või hakkas kusagilt mingit informatsiooni ka juba tulema?}

Ei, see oli ikka puhas iseõppimise teema. Need vahendid olid suhteliselt piiratud ja ega seal midagi väga keerulist ei olnud. Kunagi hiljem tulid ka esimesed 3D paketid, nendega sai pusserdatud. Tase oli nendega ikka hoopis teine, sa pidid ikkagi punkt punkti haaval mingisuguseid pindu konstrueerima ja siis nendega kuidagi opereerima. Tänapäeval vaatad, kuidas väänatakse mingeid \emph{bump mapping}'uid\sidenote{Arvutigraafika tehnika, mille abil kolmemõõtmelise objekti pinnale simuleeritakse kühme ja kortse. Lihtsalt öeldes, oranžist kerast tehakse usutava väljanägemisega apelsin.}  ja mingisuguseid asju \emph{layer}'ite kaupa ja see kõik annab kuidagi tulemuse,  see on täiesti müstika. 

\question{Mis sa siis tegid, kui sa Eesti Televisioonis\index{Eesti Rahvusringhääling!Eesti Televisioon} enam ei olnud?}

Kuidagi tundus, et see videograafika oli  väga põnev, aga hakkas tekkima mingisugune \emph{business}. Sõbrad, kes kuidagi olid rohkem sattunud trükigraafika peale, kes seal kujundas Eesti Ekspressi\index{Eesti Ekspress} ja kes seal tegi mida, see tundus kuidagi nagu rohkem \emph{business}. Kuidagi ma sain aru, et, ahah, videot teeme Amigaga aga selle selle \emph{business}'i tarvis peaks ennast kuidagi PC'de peale  sebima. Sealsamas telemajas kuidagi tekkisid potensiaalsed kliendid ja ma pidin hakkama tootma mingisugust kujundust, mis on  trükikõlbulik. Ma ei olnud  kunagi näinud sellist programmi nagu Corel Draw\index{Corel Draw} aga mul oli see töö vaja  ära teha ja ma istusin öö läbi ja tegin ta endale selgeks. Mis oli tohutult frustreeriv, sest ta oli täiesti teine maailm. Tänapäeval on ikka see, et sa joonistad ja siis see pilt on  ekraanil, mida sa joonistad. Siis oli niimoodi, et sa  konstrueerisid mustvalgelt mingisuguse \emph{vector mesh}'i, panid sinna mingid värvid peale, vaatasid \emph{preview}'d ja siis ta joonistas sulle selle pildi aeglaselt ette. Ja alles siis sa läksid uuesti selle pildi kallale.

\question{Kuidas sul Corel Draw õppimine välja tuli, sest minu mälestuste järgi ta ei olnud kuigi töökindel: aegajalt tegi faile katki ja nii?}

Olles kasvanud nende arvutitega üles, sa arvestasid ju, et nad aeg-ajalt jooksid kokku ja aeg-ajalt nad tegid rumalusi. Aga võib-olla siin mängis natuke rolli ka see  poisikesepõlves ÕTK's\index{Tallinna Oktoorbriajooni Õppetootmiskombinaat} õpitud arvutist üleolek läbi ühe lihtsa fakti. MSX'il\index{Arvutid!Yamaha MSX} oli paremas nurgas port, mille sisse käis kas siis kettaseade või mingi mälu \emph{cartridge}. See oli sihuke päris suur sahtel. Selleks, et mitte seal midagi tuksi keerata, \emph{cartridge} sisse lükkamise hetkel, seal sees oli üks väike lüliti, mis tegi masinale reseti. Ja loomulikult õpiti kiirelt ära, et kui sa oled midagi tuksi keeranud, näiteks olid kirjutanud programmi, mis jäi loopima, siis selle asemel, et voolu välja võtta, panid kohe nagu näpud sinna auku ja masin oli surnud. Alati sa teadsid, et mingi valemiga sa saad sa temast jagu. See teadmine on olnud minuga siiani, et ma alati tean, et kui ma kuskilt seinast ikka lõpuks juhtme kätte saan, siis on ta surnud. Ma ei pea teda pelgama.

\question{Selle koha peal ma pean järgi andma kihule ja ära küsima küsimused, mida ma väga tahan küsida. Me jõuame Microlinki\index{Microlink} ja .EXE'ni\index{.EXE}. Kuidas sa nende juurde jõudsid?}

Kui ma olin juba selle prindiga alustanud, siis ma vahepeal kuidagi sattusin mingisse niisugusesse maailma, kus print oligi niisugune asi, millega ma tegelesin. Ja kuna ma olin Margusega\index[ppl]{Kliimask, Margus}\sidenote{Tanel peab silmas Margus Kliimaskit} varem suhelnud  televisioonis ja tema omakorda suhtles sellise tegelastega nagu Lõvi\index[ppl]{Lõvi}. Lõvi on muidugi kõige olulisem tegelane üldse ja kelle juurest ilmselt algab kogu Eesti arvuti \emph{business}. Kui Jaak Loondest\index[ppl]{Loonde, Jaak} algab kogu Eesti arvutiteadvus, siis ma arvan, et Lõvist algab kogu arvuti-\emph{business} kuigi ta ise pole vist binest kunagi teinud. No ja Rainer Nõlvak\index[ppl]{Nõlvak, Rainer} ja kõik see nagu klikkis kokku. Ma saan aru, et Rainer oli Margusele teinud ettepaneku toimetada mingisugust ajakirja. Tema siis võttis mul varrukast kinni ja ütles, et davai, nüüd on vaja ajakirja teha. Mina muidugi pigem oleks mänginud arvutimänge, nagu ma olin teles harjunud, kus ikkagi üheksakümmend protsenti meie tegevusest oli arvutimängude mängimine. Aga noh, ma sain omale väga korrektse 486'e, ma arvan, ja selle peal jooksis Ultima Underworld\index{Mängud!Ultima Underworld}\sidenote{1992. aastal Blue Sky Productions'i poolt üllitatud Ultima Underworld oli väga mitmes mõttes (kolmemõõtmeline keskkond, simuleeritud mittelineaarne mängu käik jne.) teedrajav rollimäng.} ja oli täitsa tore. 

Toimetustegevusega mitte tuttava inimesena ma mõtlesin, et millest peaks alustama, peaks alustama ikkagi ajakirja esikaanest. Ja siis ma sellest \emph{Corel Draws} seda esikaant  hiirega joonistasin. Praegu tagantjärgi mõeldes mulle tundub, et ma joonistasin seda kuude kaupa. Aga tõenäoliselt see nii ei olnud aga sinna läks tohutu aru. Pronto\index[ppl]{Pronto} luges kokku, et neid numbreid nii väga palju ei olnud ja nendega läks suhteliselt palju aega. Ja kuna tegu polnud nagu otseselt ka äriline ettevõtmisega vaid .EXE oligi pigem promo, siis keegi nagu väga ei survestanud seda aja poolt ka. Meil ei olnud nagu kohustust, meil ei olnud tellijaid, et ta peab nüüd iga kuu ilmuma.

\question{Aga kuskil Võrus istus üks nohik, kes kurvastas, et \enquote{miks ei ole tunlnud veel .EXE't}!}

No vot, me ei adunud, et meil on \emph{impact}.

\question{Mõju oli kindlasti olemas. Võin omal näitel kinnitada ja et ta Pronto panduna praegu niimoodi internetis on\sidenote{\url{punktexe.ee}}, on ka selgesti märk mõjukusest. Seetõttu ma ka küsin.}

Ta oli oluline igas plaanis. Olles selles asjas sees, siis minu jaoks ei olnud  küsimus, et kas arvutid tulevad muutma maailma. Ma isegi ei mõelnud sellele, nendega oli lihtsalt hea asju teha ja  tõenäoliselt inimesed, kes ei teinud, olid ikka täiesti rumalad. Kõrvalt vaadates ma isegi ei saanud aru, kuivõrd vähe tegelikult arvuteid toona kasutati, sest me istusime MicroLinki peakontoris, seal käis kogu aeg mingisugune sebilung. Telemajas ja igal pool, mul oli \emph{access} arvutitele päris hea. Aga ma mäletan, et .EXE  esimeses numbris  oli arhitekt Kalle Rõõmuse\index[ppl]{Rõõmus, Kalle} büroo niisugune väikene tutvustus  läbi selle, et nad hakkasid kasutama arvuteid projekteerimisel. See oli see midagi täiesti epohhiloovat. Ja ma isegi toona ei saanud sellest aru, kuivõrd imelik see  üldse on, et keegi teeb  paberil midagi. Ega  ma üldiselt  laksisin artiklid paika ja panin pildid külge ja mind võib olla väga ei huvitanudki, mis seal kirjas oli, välja arvatud need, mis ma ise kirjutasin. Seal artiklis siis kirjutati, et üks arhitekt käis  kuskil Kanadas stažeerimas ja seal tegeleti just sellega, et osteti personaalarvutid ja see  muutis  töö efektiivsust võrreldes sellega, kui arhitektid ja konstruktorid päevad läbi joonistasid kalka peale midagi.  Järsku panid selle kõik arvutisse ja kõik on nagu hästi. See oli ka nagu väga põnev mõte. Tegelikult on huvitav vaadata seda, mis täna toimub, et meil on see nii-öelda BIM-modelleerimine\index{\emph{Building Information Modeling - BIM.}} ja siis sa kuuled, mis on need  väljakutsed. Mul üks sõber töötab startupis, mis tegeleb BIM-mudelite konfliktide analüüsiga. Et  kuidagi üritada aru saada, et näiteks ventilatsioonitoru ei tohi läbi akna minna. Ja siis sa mõtled, et \enquote{issand jumal, millega need inimesed on tegelenud, miks nad seda arvutit pole varem kasutusele võtnud?}. Kui palju on aega raisatud!

.EXE\index{.EXE}, olgugi, et temast jäi mulje, et ta on ikka \emph{super advanced} ja mingi häkkerite värk, üritas anda pilti sellest, mis tegelikult toimub. Et arvuti ei ole ainult raamatupidaja kalkulaator. 

\question{Kuidas sa joonistamise juurest kirjutamise juurde jõudsid?}

Oli vaja ju \emph{content}'i toota ja ega keegi toona ei olnud arvutiajakirjanik. Ja  mulle meeldis arvutimänge mängida  ja ma arvan, et kirjutamine on iseenesest tore tegevus. 

\question{Kas sul juba kooli ajal lõi kirjutamise ja kirjandi soon kuidagi välja?}

Ei, ma olen võimeline kirjutama okeilt. Mulle joonistada meeldib võib-olla rohkem, sest kirjutamine on selline, raske asi, et sa pead laused läbi mõtlema ja siis sulle tundub, et nad ei ole head. On nagu liiga
konkreetne formaat.

\question{Teema jätkuks veel üks oluline küsimus. Mõni aeg tagasi Tõnis Kahu (keda tuleb ilmselt uskuda)\index[ppl]{Kahu, Tõnis} lükkas ümber mu arusaama sellest, misasi on küberpunk. Aga minu vastav arusaam tuleb ühest konkreetsest .EXE artiklist, kus on sinu ja Pronto\index[ppl]{Pronto} nimed all. Räägi nüüd ära, kuidas te tolle sisu produtseerisite}

Väga raske öelda tagantjärgi. Aga eks meil oli mingi ettekujutus. Ega küberpunk ei ole mingisugune geneetiline mingisugune organism, mis on välja arenenud ja siis pärast on hea klassifitseerida, et  pool on hüljes ja pool on mingisugune gepard. Ma eeldan, et me toona juba teadsime juba Gibsoni\index{} \enquote{Neuromancer}'it\sidenote[][-2.2cm]{\enquote{Neuromancer} on William Gibsoni 1984. aastal ilmunud romaan, esimene tema \emph{sprawl}'i triloogiast. Romaani loetakse \v{z}anri üheks mõjukamaks ning on ainus romaan, mis on võitnud nii Nebula, Hugo kui ka Philip K. Dick'i auhinna.}. Kui kõik räägivad Hichiker Guide'st\sidenote{Douglas Adams, \enquote{The Hitchhiker's Guide to the Galaxy}. 1978. aastal algselt raadiokuuldemänguna alustanud komöödia, mis  avaldati viieosalise raamatutriloogiana ning millele kuuenda osa lisas pärast autori surma avaldamata materjali põhjal Eoin Colfer. Sarja raamatud levisid tekstifailidena laialt BBSide ja interneti vahendusel olles seega ka siinmail kergesti kättesaadavad.}, siis see oli väga oluline teos. Aga noh, minu jaoks Gibsoni \enquote{Burning Chrome}\sidenote{\enquote{Burning Chrome} on William Gibsoni 1982. aastal ilmunud novell, kus tutvustatakse \emph{sprawl}'i maailma ja mille sündmusi ning tegelasi mainitakse triloogias korduvalt.} ja \enquote{Neuromancer} lasid ikkagi aju täiesti välja.

\question{Ma siiamaani loen neid asju regulaarselt üle. Härra Gibson kirjutas need raamatud trükimasinaga paberi peale ja aastal 2019 täpselt nii ongi}

Oled sa tema uuemaid raamatuid ka lugenud? Need lähevad veel hirmuäratavamalt tõepärasemaks  ja ajahorisont tuleb üha lähemale.

\question{Siit siis loogiline küsimus, et pidi ju olema mingi allikas, te ei mõelnud ju küberpungi mõistet (mida Gibson ei maini) ise välja? Olid teil välismaa BBSid, internet?}

Ma arvan, et kõik see nimetatu klikkis kuidagi kokku, tõenäoliselt. Ma ei oska Pronto\index[ppl]{Pronto} eest rääkida, aga \enquote{Blade Runner}\sidenote{\enquote{Blade Runner} on 1982. aastal linastunud Ridley Scott'i film, milles peaosa mängib Harrison Ford ja unustamatu lõpumonoloogi esitab Rutger Hauer. Film toetub lõdvalt  Philip K. Dick'i 1968. aasta, samuti klassikaks peetavale, novellile \enquote{Do Androids Dream of Electric Sheep?}.} on, eksole, eepiline nurgakivi, Sid Meier\sidenote{Sid Meier on küll legendaarne arvutimängude autor, kuid mitte futuroloog. Ei ole selge, keda Kaspar silmas peab.} oli  see futuroloog, kes joonistas ilusaid düstoopilisi pilte .See kõik kujundas meil välja mingisuguse düstoopilise pildi tehnilisest maailmast, mis kõik on külge ühendatav. Kasvõi \enquote{Battle Angel Alita}\sidenote{\enquote{Alita: Battle Angel} on 2019. aastal linastunud  Robert Rodriguez'i film, mis tugineb Jaapani mangakunstniku Yukito Kishiro 1990. aastate sarjal \enquote{Battle Angel Alita}}, mis täitsa juhuslikult praegu kinno jõudis, onju. Ma arvan, et vähesed inimesed Eestis teavad seda originaallugu. Ja mina olin selle toonane totaalfänn. Ma käisin aeg-ajalt Helsingis Akadeemilises Kirjakauppa's\sidenote{
Akateeminen Kirjakauppa on Helsingi kesklinnas asuv kuulsusrikas raamatupood, mis just kõikvõimalike servapealsete huvidega eestlasi pikki aastaid raamatutega varustas. Palverännak sellesse poodi oli ka minu Helsingi-käikude lahutamatu osa.} ja seal kogu aeg vaatasin, et kas uus osa on tulnud. Üheksa raamatut, mul on nad kõik olemas. Kuidas mingid metalltorud lähevad su silmamuna sisse ja ajust on selle järgi ainult mingisugused \emph{chip}'id ja natukene pudru. See kuidagi kujundas meid, me elasime selle asja sees, ma arvan. Mängumaailmas tõenäoliselt olid paar mängu jälle, mis kuidagi sinna kontributeerisid. 

Pluss on see, et me muidugi üritasime  siis ka mänge teha. Enne veel, kui me Bluemoon'iga\index{Bluemoon} \sidenote{Kaspar peab ilmsesti silmas Bluemoon Interactive nimelist ettevõtet, millest on juttu leheküljel \pageref{sisu!bluemoon} , ja mitte samanimelist Londoni ööklubi (mille ees Ahtit ja Jaani korduvalt pildistatud on)} midagi tegime. Selles Amiga maailmas me üritasime koos Ott Aaloe\index[ppl]{Aaloe, Ott} ja Juhan Soonetsaga\index[ppl]{Soomets, Juhan} ja midagi teha. Me tegime Rocketsi-nimelise\index{Mängud!Rockets} mängu, mille Bluemoon pärast keeras PC peale palju ägedama, aga võib-olla vähem ägeda. Selle mängu intro, ma mäletan, oli väga selgelt kantud kõigest sellest vee kaheksatest ja rakettidest ja nii edasi. Väga tähtis oli kindlasti see, et päikseprillid olid õige kujuga. Andrus Aaslaid\index[ppl]{Aaslaid, Andrus} sinna kõrvale rääkis või kirjutas lugusid, kuidas plinkiva valgusega saab sul aju ümber programmeerida. See kõik nagu absorbeerus ja tekitas mingisuguse omaette alternatiivse reaalsuse ja ma arvan, et see on see meie arusaam küberpungist. 

\question{Kui teie tuleviku ettekujutuses oli ajust järel natuke putru ja palju kiipe ja tulevik oli muidu ka düstoopiline, siis miks te sellele vaatamata pika sammuga tolle tuleviku suunas astusite?}

No aga seda tagasi hoida on tõenäoliselt mõttetu, sest et ludiidid ka üritasid midagi. Aga parem on olla seal enne teisi. Et sa paned juba õiged \emph{chipid} omale õigesse kohta  ära ja võtad selle pudru osakaalu väiksemaks. 

\question{Et üheksakümnendatel selliseid mõtteid mõelda, oli ikka korralikku visiooni tarvis. Aga räägime Bluemoonist: kuidas sa Ahti\index[ppl]{Heinla, Ahti}, Jaani\index[ppl]{Tallinn, Jaan} ja tolle pundiga kokku sattusid?}

Toona igaüks mõtles, et ta on super-häkker. Nii see, kellel oli kastis see NSXi manual kui see, kes oskas neid faile kokku panna. Ma olin kogu selle niiöelda super-häkkerite seltskonnas üks väheseid tegelasi, kes joonistas pilte. Ma  tegelikult oskan ka ilma arvutita päris hästi joonistada, arvutis tundus see lihtsalt  kuidagi nagu lahedam, et seda sai salvestada. Ja  seal sai \emph{undo} teha, see on nagu põhiline. Kui sa lihtsalt joonistad, siis \emph{undo} teha on väga raske, isegi võiks öelda, et peaaegu võimatu. Jällegi, see seltskond ei olnud nagu nii suur ja nad kõik nagu \emph{connect}'isid kuidagi. 

Teen korra kiire kõrvalehüpe. Tuli meelde, kuidas BBSide ja värkidega suheldi, et meil sealsamas Telemajas oli täpselt samasugune ambitsioon. Meil olid Amigad aga mänge ju ei olnud, siis pidi neid mänge kuskilt pirama. Ma loodan, et tagantjärgi mingid pidamisasjatundjad ei hakka peale lendama. Aga igal juhul üks viis neid mänge saada oligi see, et sa pidid jõudma kuskile mingisugusesse BBSi ja kuidagi sinna sisse pääsema. Ega seal ei olnud niimoodi, et \enquote{astu sisse}. Seal tavaliselt istusid mingid vennad, kes monitoorisid tegevust,  Eesti tundus eksklusiivne, sihukene veider koht, sama hea kui eskimod. Ja mingi hetk meil isegi tekkis mingisugune \emph{trading capacity}, et meil juba oli midagi, mida  vastu pakkuda. Aga tavaliselt me ikka mängisime sellist vaest sugulast ja siis me isegi nagu \emph{bluebox}'isime\sidenote{Ennevanasti liikusid instruktsioonid telefonikeskjaamadele konkreetse kõne kohta samas kanalis, kui kõne ise. Seega, tõstes toru ja vilistades  õigeid signaale, võis nuuta kõne teekonda keskjaamade vahel ja, mis kõige olulisem, saada mööda kõnetasudest. Kõvemad spetsialistid, nagu Joe \enquote{Joybubbles} Engressia suutsid kaugekõneliini lähtesgamiseks vajalikku 2600Hz signaali suuga vilistada. Natuke nõrgemad, nagu Jogn \enquote{Captain Crunch} Draper, vajasid tehnilisi vahendeid. Lihtsurelikud aga kasutasid elektroonilisi seadmeid. Neist esimene, mille ehitas 1960. aastal Robert Barclay, oli pakendatud sinisesse kesta, sealt ka mõiste \enquote{blueboxing}.} ennast sinna sisse. 

\question{Oot, räägime nüüd sellest lähemalt. Kas see tähendas seda, et sa pidid toonasele telefonikeskjaamale kõrva vilisama midagi, mida too tingimata kuulda ei tahtnud?}

Kusjuures nüüd, kui ma hakkan mõtlema, siis Margus\index[ppl]{Kliimask, Margus} oli põhiline \emph{bluebox}'i spetsialist, aga kas me ka reaalselt \emph{bluebox}'imiseni ka jõudsime, see on nüüd hea küsimus.

\question{Ma mäletan, .EXE's ilmus manuaal selle kohta\sidenote{\enquote{Blueboxing parimates peredes. I ja II osa}. .EXE esimene ja teine number}, mis oli jube huvitav lugeda}

Jaa. Põhimõtteliselt see on ju iseenesest üsna lihtne, kuna toona nood keskjaamad olid suhteliselt rumalad.  Võtame kasvõi selle, kui kiired olid modemid. Ma mäletan, et oli kolmesajana Hayes. Ja seda, et minu meelest ma ei mäleta, kas Mast\index[ppl]{Kaal, Madis} või keegi, oli väidetavasti suuteline \emph{handshake}'i ära vilistama sellele modemile. Ta  oli  piisavalt aeglane, talle sai põhimõtteliselt suusõnaliselt selgeks teha, mis sa tahtsid.

Võib-olla oli see, et igaühel meist oli oma fookus on ju. Kes tahtis rohkem seal mingit \emph{network}'i häkkida, kes tahtis rohkem lihtsalt häkkida, kes tahtis progeda. Mind huvitasid selgelt mängud, liikuvad pildid, värvilised pildid, kuidas neid ise teha, 3D kõik niisugune värk. Ma pigem nagu otsisin neid võimalusi. Eks see viis meid ka tegelikult kokku siis lõpuks Bluemoon'i\index{Bluemoon} pundiga, kellel oli  kindel soov, et nad tahavad teha mängu. Ja kuna minu jaoks oli see lihtsalt natukene niisugune nõme ülesanne, kuna mul oli Amiga\index{Arvutid!Amiga}, seal olid miljonid värvid ja neil oli mingisugune EGA (alguses vist isegi CGA) ja sinna pidi mingi nelja värviga midagi valmis nikerdama. \emph{Why not}, teeme ära. Sellest sündis siis Kosmonaut\index{Mängud!Kosmonaut}. 

See, kuidas sa turustasid ja toimetasid, ma nagu lihtsalt vaatasin ja imestasin. Jube lahe oli. Mul on meeles kuivõrd \emph{dedicated} need vennad toona olid ja on vist siiamaani. Nad olid nagu tõesti nagu \emph{focused}, et teeme seda asja. 

Aga see graafiline pool oli super lihtne. Nokkisin selle valmis, siis nad tegid oma selle musa editori, sinna ma nokkisin ka mingisugused ikoonid, mingid kitarred ja  trummid ja väga lahe oli.

\question{Seda ma mäletan küll, et inimesi, kes oskasid arvutiga joonistada, oli vähe. Kas sul kõigepealt tuli arvutiga joonistamine ja siis joonistamine või oli sul enne ka joonistamise huvi?}

Ma ikka enne joonistasin ka. Kes ikka ennast kiidab, kui mitte ise, eks ole. Akadeemilist joonistamist ma valdan  suhteliselt väga hästi. Selles mõttes mul ei olnud nagu keeruline omandada neid oskusi. Toona ei joonistatud tabletitega, vaid oli seesama munaga hiir, mille muna aeg-ajalt jooksis mingit pahna täis, siis sa pidid jälle küünega puhastama. Ma ikkagi endale otsisin mingi hiire, mis nagu enam-vähem jooksis. Selles mõttes minu jaoks ei ole vahet, kas see on pliiats või hiir või tablet või mis iganes. 

\question{Arvuti oli sinu kunsti tegemise nii-öelda laiendus}

Jah, ta oli lihtsalt nagu mingi teistsugune tehnika ja tunduvalt andeksandvam, kui näiteks akvarell. Täna sa vaatad, et  kõik kunstnikud kasutavad mingit Cintiq'u tabletti, neil on kõik super ägedad toolid. Ma siiamaani aeg-ajalt, kui mul on vaja nikeradada midagi kasutan oma läppari \emph{touchpad}'i ja kõik vaatavad, et ma olen peast soe. See on mugav ja käe järgi \emph{tool} tegelikult, kui sa ta ära omandad ja keerad kursori piisavalt kiireks.

\question{Üks hetk sul tekkis mõte, et võiks hakata veebi tegema?}
 
See oli pigem läbi selle, et ma olin aru saanud, et ma ei ole piisavalt järjepidev ja see progemise osa  tundus toona liiga kuiv. Mul olid sõbrad, kes sellega tegelesid ja see tulemus ei olnud nagu seksikas. Veeb oli alguses ka super \emph{boring}, mis seal oli see Mosaci või mis see esimene brauser oli, ikka \emph{ugly as hell}. Aga siis kui ma sain aru, et kui sa said juba tabelitel keerata ported maha ja sinna mingite üksikute ühepiksliste tükkidega  hakata mingit \emph{layouot}'i tegema, siis ma olin müüdud mees. Ma töötasin ühes reklaamibüroos ja midagi  katsetasin, nokkisin. Mindworks\index{Mindworks} oli juba olemas ja selle asutajad siis vaatasid, et ma jagan natuke sellest reklaamibisnesist ka.  Et paneme seljad kokku. See, et ma sain mitte lihtsalt enam selle pildi oma käe seest ekraanile, vaid ma ise sain selle pildi nagu pauh kõigile nina ette, eks ole. Ja toona poisikesed, mis ma tegelikult ikka nii väga poisikesed olime, aasta oli 1996 või 1997, me olime umbes kakskümmend viis. Siis sai juba bisnest teha. Lõikasid neid piksleid \ldots Ma mäletan, meil oli selline klient nagu Reval Hotel Group, mingid nagad tulid ja võtsid sihukesed kliendid ja tegid neile ägedaid asju.

\question{Ja jätkuvalt oli sind liigutav faktor see, et sa said oma pildi inimestele silma ette?}

Tegelikult mul ei oleks isegi vahet, kas inimestele silma ette vaid just nimelt see, et sul oli mingisugune distsipliin, HTML. Ja sa teadsid, kuidas optimeerida gif'e, sul oli mingisugune \emph{toolset} ja sa valdasid seda suhteliselt hästi. See tekitas rõõmu, et sellega sai teha mingisuguseid asju, mida võib-olla teised ei osanud teha. Sihuke \emph{job satisfaction}'i värk. Ma kujutan ette, et muru niitmine on ka selles mõttes lahe, et sa näed, kuidas niidetud muru jääb su taga maha. Sihuke suhteliselt \emph{instant gratification}.

\question{Mis sa praegu teed?}

Ma olen kuidagi liigselt distantseerunud sellest disaineri rollist aga samas mitte. Ajapikku ma olen aru saanud, et selle pildi tegemine on mõnes mõttes niisugune käsitöölise töö. Tegelikult need lõikelauad, mida minu lapsepõlves turul müüdi, kus  põletiga oli tehtud \begin{russian}ну погоди\end{russian} peale, on ju nagu veits sarnane. Palju on šeffim on tegelikult võtta ja aru saada mingitest äriprotsessidest või mingisugusest inimeste mõttemallidest ja disainida neist midagi. See progemine minu jaoks on see, et kui asi õigesti sõnastada, siis mingisugused vennad teevad selle valmis ja see, see muutub nagu päriseks, see  protsessi toetav  asi seal masina sees toimetab täpselt nii nagu sa oled talle  öelnud, et toimeta. 

\question{Nüüd mitte ei tule käe seest pilt vaid tuleb sinu pea seest mõte, kuidas see kupatus võiks käia, programmeerijad teevad selle valmis ja siis käibki nii}

Just. Lapsepõlves kuidagi, ma mäletan selgelt, mind võlus mõte, et tehas on tore asi, sest et ta võtab mingisugused toorme ja detailid ja siis neist pannakse kokku mingi asi. Ja jällegi me jõuame sedasama  Nintendo \emph{device} juurde, et füüsilisel kujul seda toota on jõle tüütu. Palju lihtsam oleks teha seesama asi nii, et oleks bittide jada, mis kõik grupeeruvad, moodustavaid mustreid ja sellest peaaegu nagu võluväel tekivad mingisugused asjad, mis inimestele tegelikult on tänaseks sama reaalsed tööriistad kui haamer ja höövel.
