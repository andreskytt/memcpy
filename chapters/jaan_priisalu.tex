\index[ppl]{Priisalu, Jaan}

\ldots ütles eestlaste kohta väga hästi. Et eestlased otsused saunas 
teevad, eks ole, on \emph{no-brainer}. Aga kuidas seda tehakse? 
Inimesed käivad saunas, on paljad, räägivad midagi. Ja kui ära 
lähevad, kõik nagu teavad, mis otsus on. Seda ei hääldatud mitte ühtegi korda 
välja 
ja ei ole aru saada, kes on liider, kes selle \emph{move}-ga välja tuli. 
Sul on pikka aega olnud mingid võõrad sellid peal, 
kelle eesmärk on mingi muu, kui kohaliku rahva eesmärk. Ja sa tead, et võimu 
peale loota ei saa. Aga kui sa teed otsuseid sellise \emph{mode}-ga, et sa oma 
liidreid välja ei näita, on see tegelikult liidrite kaitsmise süsteem. Mis 
muuhulgas tähendab, et me oleme projektirahvas. Kui sa Vabadussõda vaatad, 
siis see oli ka selgelt projekt.

\ldots Mulle on seda küsimust mitu korda esitatud. Ameeriklased tulevad ja 
küsivad, et kui me ringi vaatame, siis need \emph{challenge}'d, mis te 
välja käite, on nagu pooltel maailma riikidel. Aga miks siin välja tuleb?

\bigskip
\noindent\rule{.3\textwidth}{.7pt}
\bigskip

\question{Kuidas sina sattusid arvutite juurde?}

Arvutite juurde sattusin neljandas või viiendas klassis. Meil oli pioneerisalk, kellega tegime igasuguseid asju, ja mind pandi seda juhtima. Salgas oli ka klassivend Kermo 
Jaaksoo\index[ppl]{Jaaksoo, Kermo}, kes pakkus välja, et võiksime 
minna tema isa töö juurde. Ülo\sidenote{
Kermo isa, akadeemik Ülo Jaaksoo\index[ppl]{Jaaksoo, Ülo}} töötas sel ajal 
Estonia puiesteel ja tal oli seal ES-1010\index{ES EVM!ES-1010}\sidenote{1010 oli ES EVMi esimese alaseeria esimene mudel.}. Ega me seal arvutiga midagi muud teha ei 
osanud kui Kuule maandumise mängu mängida. 

Ma käisin 1. keskkoolis\index{Tallinna 1. Keskkool}. Kui lastelt küsitakse, kelleks nad saada tahavad, siis tavaliselt 
öeldakse, et tuletõrjujaks, politseinikuks, autojuhiks. Mu vanemad 
väidavad, et kui minu käest seda esimest korda küsiti, siis mina tahtsin saada
inseneriks. Seepeale otsustasid
vanemad, et poiss tuleks panna matemaatikat ja füüsikat õppima. Isa leidis, et 1. keskkool on õige koht, see 
oli elukohajärgne kool ka --- me elasime vanglahoovis, Suur-Patarei 29. 
Läksime kooli katsetele, tegin katsed ära ja siis direktor küsis isalt, 
miks nad tahavad mind sinna panna. Isa rääkis inseneriloo ära ja ütles, et poisil on vaja 
matemaatikat ja füüsikat teada. Direktor mõtles natuke ja ütles, et see kõik on ju väga tore ja tõepoolest, poiss tegi katsed 
edukalt, aga meil on see häda, et matemaatika-füüsika klass hakkab üheksandast 
klassist, mis ta seni teeb? Seepeale pandi mind prantsuse keelt õppima.

\question{See on ka ilmselt väga tarviline olnud!}

Prantsuse keel on paarikümne aasta pärast
kõige kõneldum keel maailmas. Kui mõelda, kui levinud see on Aafrikas, siis on sellel seal sama funktsioon, mis Indias inglise keelel. Ja 
kui 
vaadata, missugune rahvastikuplahvatus neil on ja palju neil maad käes on, siis 
Aafrikal ei ole India inimeste tiheduse probleemi, vaid paisumisruumi on
palju.

\question{Kas pioneerirühmas kohtusite esimest korda essukesega\sidenote{ES EVM seeria masinate levinud hellitusnimi.}, maandusite 
Kuule ja mis veel?}

Põnev oli vaadata, kuidas lindid käivad ringi. Sel ajal olid juba 
vahetatavad kõvakettad. Nad näitasid meile ka trummelsalvestit, kuigi see 
ei olnud käigus, vaid oli lahti ühendatud.

\question{Mis asutus see oli Estonia puiesteel?}

Üks Teaduste Akadeemia asutus, ma täpselt ei mäleta. Ilmselt 
praegune Küber\index{Küber}, sest Ülo oli kunagi Küberi 
direktor.

\question{Kas tollest ajast jäidki seal oma rühmaga käima?}

Ei, mitte päris. Järgmine kord oli siis, kui meil olid arvutiõppe tunnid 
ja õpetaja Loonde\index[ppl]{Loonde, Jaak} viis meid Pedasse\index{Tallinna Pedagoogikaülikool}. 
Teine arvuti mu elus oli sealne Nairi-2\index{Nairi!Nairi-2}. Nairi on 
transistorite peal arvuti, perfolint on viierealine. Kui 
matemaatika-füüsika klassi läksime, siis Jako Bergson\index[ppl]{Bergson, Jako} tõi 
Kirovi kalurikolhoosist\index{Kirovi Kalurikolhoos} MIR-2\index{MIR-2} 
ära. Mir-2 on tegelikult maailma esimene personaalarvuti, mis oli tehtud inimese aitamiseks. Selle protsessor kaalus pool 
tonni, aga ideoloogia oli selles, et inimene saaks 
oma rehkendused tehtud. Muu hulgas olid integreerimine ja diferentseerimine 
rauas\sidenote{St. riistvaraliselt.} realiseeritud.

\question{Sest inimesel oli ju vaja integreerida ja diferentseerida, mille 
jaoks talle muidu üldse arvuti!}

See oli Ukrainas tehtud arvuti, seal arvutati gaasiturbiine, 
rakette ja muud säärast. Programmil oli 
Algoliga sarnane keel ja see algas käsuga \verb|RAZR|, mis tähendas 
\begin{russian}разрядность\end{russian} ehk kui pikad arvud on. 

\question{Kui pikaks arve sai keerata?}

Me keerasime kas 300 või 400 peale, kümnendkohtades. Panime programmi piid arvutama ja arve ritta ajama ning see lõppes sellega, et 
kuigi oli talveaeg ja me tegime aknad lahti, kuumenes arvuti ikkagi üle. 
Pooleteist tundi saigi sellega tegutseda.

\question{Kust teil selline mõte, et võiks piid arvutada?}

See lihtsalt tundus lahe.

\question{Ja kust te matemaatika saite?}

Mul oli üks paks venekeelne matemaatika õpik või 
entsüklopeedia. Seal olid igasugused read ja pii rida oli 
üks nendest.

\question{Järelikult oli teil koolis tolleks hetkeks matemaatika juba pihta 
hakanud?}

Jah. Meie koolis oli nii, et kui käisid 
matemaatika-füüsika eriklassis, siis esimese asjana jagas õpetaja 
Uudelepp\index[ppl]{Uudelepp, Helgi}\sidenote{Gustav Adolfi 
Gümnaasiumi legendaarne matemaatikaõpetaja Helgi Uudelepp.} klassi pooleks. Pool klassi hakkas õppima tavalise 
keskkooliprogrammi järgi ja ülejäänud kambale anti olümpiaadi ülesandeid. 
Kamp oli väga kõva (näiteks Mati 
Pentus näiteks\index[ppl]{Pentus, Mati}\sidenote{Mati Pentus on Eesti 
matemaatik, alates 2003. aastast Moskva Riikliku Ülikooli professor.}), probleem oli koolist üldse välja jõuda. Rajoonist sai 
niikuinii vabariiklikule edasi.\sidenote{Toonased olümpiaadid olid organiseeritud kooli-, rajooni- ja 
vabariiklikeks olümpiaadideks. Vabariiklikult olümpiaadilt oli võimalik pääseda ka 
üleliidulisele olümpiaadile.}. 

\question{Kas teil oli koolis arvutitund ka ja kasutasite MIR-2?}

Jah. See asus küll kooli spordihoones. 

Selge see, et 1. keskkool seisis direktor Viikholmi\sidenote{Helmi 
Viikholm\index[ppl]{Viikholm, Helmi} oli kooli direktor aastatel 1962---1982.} najal ja kui 
Viikholm läks pensionile, juhtus nagu ikka organisatsioonidega juhtub, et 
hakatakse rootsi keelt või muud sellist õpetama.

\question{Kas selleks ajaks olid sina sealt koolist juba läinud?}

Ma käisin siis veel koolis, kui Viikholm ära läks. Organisatsioon toimis vana energiaga 
veel natuke aega edasi, tavaliselt võtab lagunemine 
kaks-kolm aastat aega.

\question{Kas keskkooli ajal pusisite Jaaguga MIRi peal või oli juba muid 
võimalusi ka?}

Mina sain oma esimese palga progemise eest aastal 1984, 
üks suvi enne keskkooli lõpetamist. Paldiski maantee 1 asus termo- ja 
elektrofüüsika instituut\sidenote{Eesti NSV Teaduste 
Akadeemia Termofüüsika ja Elektrofüüsika Instituut 
(TEFI).\index{Teaduste Akadeemia!Termofüüsika ja Elektrofüüsika Instituut}}, nende käes oli näiteks Arnold Veimeri nimeline laev.
Mina olin akadeemik Krummi\sidenote{Akadeemik Lembit Krumm\index[ppl]{Krumm, Lembit} (1928---2016).} juures, 
kes arvutas elektrivõrkude staatilisi režiime. Neil olid ka 
arvutid ja mina tegin oma esimese töö arvutil 
Iskra-226\index{Iskra!Iskra-226}, mis on sisu poolest Wang 2200 koopia ja millel on muu hulgas videoprotsessor
8080. Üks vend tegeles sellega, et 
pani selle videoprotsessori peale käima CP/Mi. 

\question{Kust nad su leidsid?}

Ma ei mäleta, aga ilmselt tutvuste 
kaudu või siis lihtsalt läksin ise sinna.

\question{Ja seal arvutati elektrivõrke?}

Esimese asjana pidin tegema rehkenduse ühe
ministeeriumi aruandluse jaoks --- tabelarvutuse Basicus\index{BASIC}. Tegin programmi valmis ja nägin esimest korda 
päris \emph{user}'i probleemi ka. Korraldasime piduliku üleandmise, 
komisjon tuli 
kokku ja naisterahvale, kes pidi seda 
programmi kasutama hakkama, öeldi, et istu arvuti taha ja proovi 
midagi toksida. Naine keeldus. Keegi ei saanud aru, 
mis juhtus --- kas programm ei meeldi või milles asi. \enquote{Ei, ma kardan 
elektrit!} vastas tema. Ma ei mäletagi, kuidas see tsirkus lõppes. 
Siis tuli juba järgmine töö. Neil oli programm, millega nad suuri jakobiaane 
arvutasid, ja see käis essukese\index{ES EVM} peal, mis oli vist 1055. Neid 
oli Küberis kaks tükki: üks oli 360 ja teine 370 koopia\sidenote{ES EVMi 
esimene alaseeria oli IBM System/360 ning teine ja kolmas System/370 koopiad. 
Mudel 1055 kuulus teise ja 1066 kolmandasse seeriasse.}. Nende terminal ei olnud 
mitte VT100 nagu mujal maailmas, vaid IBMi VT52, mis näeb üsna 
hüperteksti moodi välja. Kirjeldad ära sisendi ja väljundi väljad ning terve 
ekraan saadetakse korraga, töötab nagu veebileht. Sinna peale 
ma tegin ühe programmi, mis võimaldas sisendandmeid mõistlikult sisestada ja tulemusi vaadata. Kuna essuke oli \emph{patch}-arvuti, siis pidi 
vahele tegema programmi, mis \emph{patch}'i tulemused 
sisse söödaks või välja võtaks ning 
terminaliga suhtleks. Tolle vahejupi tegi vist Tarmo Mere\index[ppl]{Mere, 
Tarmo}. See kõik oli üsna aeglane, ringitõstmine võttis
aega ja käis läbi ketta.

Teine Küberi essuke oli 1066. Selle peal nad andsid mulle ühe
assembleri makro, et paneksin asja käima, aga protsessor oli juba 32bitine. Olin Inteli assemblerit vaadanud, aga no üldse ei olnud 
sarnane. Kõiki registreid sai kõikides funktsioonides kasutada ja kõige 
hullem, millest ma ei saanud aru, oli see, et kõiki aadresse võeti 
baasregistri suhtes. Sealjuures eeldati, et sa lihtsalt 
tead seda. Aga kuidas teha kindlaks, milline on baasregister ja kust see 
algab? Leidsin baasregistri valimise käsu üles, kuid 
midagi oli ikka puudu. Tuli välja, et seal, kust baasregister hakkas aadresse 
lugema, oli vaja lihtsalt kõikide käsuridade ette panna üks 
tärn.

\question{Programmeerimisoskus oli sul järelikult olemas. Kas õppisid seda Jaagult, 
pusisid ise või kust see tuli?}

Jaak\index[ppl]{Loonde, Jaak} õpetas jah, MIR-2 peal me mingit nalja 
tegime. Kui juba piid arvutad, siis peaksid ka paar rida oskama kirjutada. Järgmisena tuli Basic TEFIs. VT52 puhul ma ei mäleta, 
milles ma programmi tegin, võibolla oli Fortran. 

\question{Kuidas VT52 puhul progemine käis, kui ekraanil olid lihtsalt 
mingisugused väljad?}

Ei mäleta, interaktsiooni kirjeldus on selline, et saadad 
terve 
andmepaketi korraga minema ja saad terve andmepaketi korraga tagasi. Andmete 
töötlus ja esitus on eraldatud. 

\question{Naljakaid aparaate on olemas!}

Mäletan, kuidas vennad presenteerisid modemit, millega me Küberisse 
helistasime. 1200boodine modem oli külmkapisuurune. Kui tuli 2400boodine modem, siis see oli tükk maad väiksem, pool külmkappi.

Ja millega vennikesed veel tegelesid?! 1984. aastal olid olümpiamängud LAs. Mängude 
ajal nad jälgisid elektrivõrgu parameetreid, peamiselt sagedust, ja selle põhjal 
ütlesid, palju tootmist seisab ja mitu inimest vaatab olümpiamänge. 

\question{Kas nad ütlesid seda ka ametlikult kellelegi?}

Ei, nad vaatasid oma lõbuks. Neil oli kihlveokontor --- vedasid kihla, palju järgmisel päeval 
vaatajaid on.

\question{Sa olid matemaatikas ja füüsikas tugev, aga arvutiasja 
pidid suuresti ise pusima. Mis sind selle juures tõmbas?}

Äge on see, kui saab oma kätega midagi teha. Üks liik inimesi armastab teooriaid 
välja mõelda, teised asju kokku 
ja käima panna. Täna ma olen 
mõtlemise ja teooria poole peal.

\question{Eks asjad ole maailmas tasakaalus. Nii et tol ajal meeldis sulle 
vajutada arvutiklahve ja arvuti muudkui tegi?}

Automatiseerimine --- see, et asjad ise juhtusid --- oli väga äge. Näiteks et
auto sõidab ise. Sel ajal oli natukenegi targem 
juhtimisalgoritm haruldus.

\question{Ja see hoidiski sind nii palju arvuti taga, et õppisid programmeerima 
ja õigetesse kohtadesse tärne panema?}

Jah.

\question{Kas selle juurde käis ka mõni laiem valdkondlik huvi? Mõni on 
rääkinud, et raamatud, muusika ja muu selline suunas arvutite poole.}

See oli sügav Vene aeg, mis mõttes \enquote{raamatud ja muusika}? 
Loomulikult lugesin ma Asimovit, robotivärgist räägiti seal päris palju. Lugesin kõiki 
raamatuid, mida kätte sain. Neid, mida ei saanud, lugesin 
juba Prantsusmaal, kui läksin Toulouse'i õppima. Ma panin 
kooliminekuga natuke puusse --- kui ma poole septembri pealt kohale läksin, 
polnud koolis veel kedagi. Nii ma siis 
istusin raamatukogus ja lugesin matemaatikat ning Asimovi jutte, et keeleoskust parandada.

\question{Enne kui Toulouse'i juurde jõuame, küsin, kas 
töölkäimine keskkoolis õppimist ei seganud?}

Ei seganud, see oli eluviisi osa, ma olen alati kooli kõrvalt tööl 
käinud.

\question{Mida sa ülikooli õppima läksid?}

Automaatikat, automaatjuhtimissüsteeme tehnikaülikoolis\index{Tallinna 
Tehnikaülikool!Automaatikateaduskond}.

\question{Kuidas see otsus sündis? Oli see loomulik valik?}

Oli küll loomulik valik. Kuulsin oma sugulaselt Jaan 
Võrgult\index[ppl]{Võrk, Jaan}, mis 
see automaatika üldse on. Meie rühm oli väga äge, klassivendi ja -õdesid
oli umbes kümme. 

Gibbs\index[ppl]{Kübbar, Heiki} ütles mulle hiljem, et seda oli 
vastik vaadata --- ise higistad matemaatika- ja füüsikaloengutes, aga 
need vennad tulevad kuskilt, teevad pulli, lähevad eksamile ja saavad 
kõik viied. 

\question{Kas ta oli su kursavend?}

Me oleme kindlasti koos loengus käinud. Ta on aasta noorem, aga ilmselt läksid meil loengud minu sõjaväest tagasitulekuga kuidagi sünki --- ma istusin oma kaks aastat sõjaväes ära.

\question{Kas sind võeti enne ülikooli kroonusse?}

Ülikooli esimeselt kursuselt. Arvasin, et ma ei pea minema, aga tolle aasta viimane võtmine oli detsembris ning mind pandi rongi peale ja läksin.

\question{Kus sa need kaks aastat veetsid?}

Põhikoht oli Rostov Doni ääres sisevägedes ehk siis vangivalvurid. 
Alguses olin mingisuguses isolaatoris, kus oli kolm varianti, mida saab teha 
veest ja hapukapsastest. Esimene oli hapukapsasupp ehk vesi hapukapsastega. 
Teine oli praad ehk hapukapsad ilma veeta. Kolmas oli kissell ehk 
vesi ilma hapukapsasteta. Ma vaatasin, et suren sinna ära, kui pean seal väga pikalt 
olema, ja munsterdasin ennast \emph{utšebka}'sse\sidenote{Õppeväeosa.}, 
natuke luuletasin 
ka. Seepeale saadeti mind Galatši valveseadmete inseneriks õppima. Galatš on see koht, kus Stalingradi kott kokku murti. Nii et ma olen
Volgogradi Venemaa Ema Mamajevi kurgaani peal päriselt lähedalt näinud. 
Hirmus roostes kolakas oli --- kaugelt vaadates ilus, aga lähedale minnes roostes.

\question{Mina küll Vene kroonusse napilt ei jõudnud aga, nohik nagu ma olin, 
kartsin, kuidas ma seal füüsilise koormuse ja keelega hakkama saaksin. Kas sul seda 
hirmu ei olnud?}

See, et üritad ellu jääda, oli igal juhul. Ja selge see, 
et vene keelt koolis ära ei õppinud. Seal aga ei olnud valikut.

\question{Klassikalise haridusega vene proua ju kolme- ja neljatähelisi sõnu
ei õpeta.}

Jah, need on sõnad, mille abil õpid tõepoolest kõiki asju ära ütlema. Seal 
tehti naftast viina. Läksin kord brigaadi töökotta, et juhendada 
järgmist venda, üht Leedu poolakat, kes teadis elektroonikast 
tegelikult rohkem kui mina. Prapporid\sidenote{Praportšik ehk lepinguline 
allohvitser Nõukogude armees.} tulid oma viinapudeliga sinna ja tahtsid, et me selle 
treipingis ära tsentrifuugiksime. Panin pudelile rätiku ümber, treipinki ja 
pöörded peale. Seesama major, kes mind \emph{utšebka}'sse vajas, vaatas kõrvalt 
ja ütles: \enquote{\begin{russian}ты уважай русский язык, ты хот \ldots\ 
скажи!\end{russian}}. See oli päris hull keel, kindlasti mitte tavaline.

\question{Mõned inimesed on rääkinud, et neil oli kolmetäheliste maailmast
keeruline tagasi teadusmaailma tulla. Kas sul seda probleemi ei olnud?}

Kindlasti oli. Sõjaväest tagasi 
tulnud loobiti kõik eraldi kursusele, neid ei lastud puutumatute 
inimestega kokkugi.

\question{Kas sa läksid Tehnikaülikooli tagasi sama asja õppima?}

Jah, sain isegi sama töökoha tagasi, aga siis ühel hetkel läksin
EKTAsse\index{EKTA}\sidenote{Arvutustehnika Erikonstrueerimisbüroo oli 
Eesti NSV Teaduste Akadeemia Küberneetika Instituudi autonoomne osakond}. 
Ektaco\index{Ektaco} on EKTA \emph{spin-off} ja EKTA direktor oli 
Märtin\sidenote{Kaarel Märtin\index[ppl]{Märtin, Kaarel} oli siiski EKTA 
tarkvaraosakonna pealik, tema alluvuses Jaan ilmselt töötaski. EKTA direktor 
oli Kalju Leppik\index[ppl]{Leppik, Kalju}, Ektaco oma Rein 
Haavel\index[ppl]{Haavel, Rein}.}. Ma hakkasin seal FoxPros andmebaase kirjutama.

Üks huvitav kogemus oli käia putši ajal Moskvas. Tegime sealsele 
juveelitehasele 
väärismetallide arvestusprogrammi, aga sel ajal pidi softile autor kaasa 
minema, sest need asjad ei olnud väga töökindlad. Tehase 
osakonna juhataja oli tiba juudi verd. Kui ta kuulis, et 
erakorraline komitee on võimu üle võtnud, ütles ta mulle kohe, et see kõlab 
halvasti, \enquote{vedur teise otsa ja kohe koju tagasi!}. Mina aga olen eluaeg lollustega 
maha saanud või hirmus otse öelnud. Arvasin vastu: 
\enquote{Mis sa jamad, vaata kui hästi teevad
Levitani\sidenote{Juri Borissovitš Levitan oli 
Nõukogude diktor, kelle kanda oli peamiste oluliste uudiste edastamine Teise 
maailmasõja ajal, tema iseloomulikku häält tunti hästi.} järele, 
nagu oleks sõjaajast pärit.} Läksime tänavatele ja need olidki 
BTRe\sidenote{Nõukogude Liidus valmistatud soomustransportöör.} täis ning madin käis. Seal olid 
suured seitsmerealised tänavad, mis olid kõik autodest tühjad. 
Inimesed korjasid sillutisekive ja ehitasid nendest barrikaade. Vennikesed 
hüüdsid mulle veel uhkelt, et vaata, kui kõvad mehed me oleme, ehitame nii kõrgeid barrikaade. Barrikaadid olid aga põlvekõrgused.
Ütlesin neile, et tank T-72 tehnilises 
spetsifikatsioonis on kirjas, et see sõidab 70 kilomeetrit tunnis, kui maapinna 
ebatasasus ei ületa meetrit. 
Juveelimessil rääkis üks korralik proua, kuidas see kõik on nii 
kohutav, ja küsis, mis mina sellest 
arvan. Ma ütlesin, et kõik on ju hästi. Proua imestas: \enquote{Mis mõttes hästi?} 
Ma siis seletasin: \enquote{Vaadake, seni on venelased tapnud kõiki teisi rahvaid, nüüd 
tapavad venelased venelasi.} Populaarsust ma sellega muidugi ei võitnud.

\question{Kas sul igav ei olnud andmebaase treida, tulles matemaatiliselt keeruliste 
asjade juurest? See on ju rutiinne töö.}

Ei olnud, seal oli tegelikult sisendit ja väljundit palju ning pusimist piisavalt, et erinevatele 
inimestele vaated teha. Ja mul olid väga lahedad 
töökaaslased Jüri Freiberg\index[ppl]{Freiberg, Jüri} ja Ülle 
Heinla\index[ppl]{Heinla, Ülle} --- Ahti\index[ppl]{Heinla, Ahti} ema, kes näitas uhkusega poja
tehtud mängu. 

\question{Kaua sa neid andmebaase tegid?}

Ma ei mäleta. Kui ma läksin Ektacosse\index{Ektaco}, tegin seal 
baase edasi. Olin siis just Prantsusmaalt tagasi tulnud ning tegin juba 
niisuguseid baase, kus olid füüsilised asjad ka taga, nagu lukud ja kassad.

\question{Kuidas sa Prantsusmaale sattusid?}

Nõukogude Liidule oli eraldatud 300 stipendiumit ja kui liit 
lagunes, 
siis kuus stipendiumit tuli Eestisse. Kuna sel ajal oli prantsuse keele 
oskajaid suhteliselt vähe, korraldati 
avalik konkurss. TPIst korjati ka inimesi ja sealne prantsuse keele õpetaja 
pani mu naise (kes oli ka 1. keskkoolist) kirja. Aga dekanaat tõmbas ta 
maha, et naine on rase ja kuidas ta sinna läheb. Naine oli suhteliselt kõva 
iseloomuga, et mis see dekanaadi asi on, kas ta on rase või mitte. Otsustasime
saatkonda minna, aga tee peal jäi tal samm järsku 
aeglasemaks ja ta ütles, et kuule, ma olen tõesti rase, mine sina.

Saatkond teatas, et stipendiumi saamiseks 
peab paar tingimust täitma. Esiteks, võimalikult kõrgel õppima. Kuna mul oli 
neli aastat ülikooli seljataga, siis soovitati kohe magistrisse minna. Teiseks soovitati mitte 
Pariisi minna. Kuna Leo Mõtusel\index[ppl]{Mõtus, Leo} oli Toulouse'is 
üks tehisintellektiga tegelev tuttavtegelane, siis läksin 
sinna.

\question{Üheksakümnendate algus oli huvitav aeg tehisintellektiga tegelda, 
see oli ju enne riistvara läbimurret.}

Selle aja peale oli juba igasuguseid asju tehtud: 
produktsioonisüsteemid\sidenote{Produktsioonisüsteem on tüüpiliselt tehisintellekti pakkumiseks rakendatud arvutiprogramm, 
mis koosneb formaalsetest reeglitest, mehhanismidest nende reeglite järgimiseks 
ning süsteemi olekut säilitavast andmebaasist.}, esimesed teoreemitõestajad, otsustuspuud ja muud asjad. Ma ei mäleta, millal 
Rete algoritm\sidenote{Charles L. Forgy poolt 1974. aastal maailmale tutvustatud 
algoritm efektiivseks formaalsete reeglite rakendamiseks.},
produktsioonisüsteemide indeks tehti. Tolleks ajaks oli 
selliseid põhialuseid laotud juba päris palju. Masinõpet vist väga 
ei tehtud ega osatud, nii palju jõudu käes ei olnud.

\question{Kui kaua sa olid Prantsusmaal?}

Aasta. 

\question{Kas sealt hakkas su võrguhuvi tekkima?}

Jah, see oli esimene koht, kus ma internetti nägin. Naine oli veel Tallinnas, 
tema käis Küberis\index{Küber} interneti küljes. Oli niisugune programm nagu talk: 
ühel pool Unixi masinas kirjutad sina ja teisel pool teine. Kuna sel ajal pidi
kaugekõnesid tellima ja see oli keeruline protseduur, siis
talk võimaldas paremini suhelda.

\question{Kas sul sellist mõtet ei olnud, et hakkaks teadust tegema?}

Oli. Aga naine käis mul Prantsusmaal külas ja siis sündis meil teine 
laps ka ning tulin koju tagasi.

\question{Eestis saab ju ka teadust teha.}

Sel ajal ei saanud. Oli üheksakümnendate algus ja lihtsalt raha ei olnud. Pere jaoks oli vaja raha teenida ja kuidagi korter saada. 
Korterihinnad olid naeruväärselt madalad. Sain 
Prantsusmaal kõrget magistristippi ja pool sellest hoidsin kokku 
ning ostsime korteri.

\question{Mida sa Prantsusmaalt tagasi tulles tegema hakkasid? 
Kas programmeerisid jätkuvalt?}

Jah, ikka. Läksin Ektacosse\index{Ektaco} ja tegin lukkude juhtimist. 
Muu hulgas tuli seda teha ühes pullis kohas, Viimsi 
Talveaias\sidenote{Viimsi Talveaed asub Pringi külas ja valmis 1973. aastal 
Kirovi-nimelisele näidiskalurikolhoosile. See kolhoos (ja 
sealsed kolhoosnikud) olid tolle aja mõistes põhjatult rikkad ning Talveaiast 
kujunes Tallinna 
peenema rahva peokoht. Hulludel üheksakümnendatel oli tegu populaarseima 
paigaga, kus 
kiiresti ja kõikvõimalikel viisidel rikastunud inimesed käisid oma rikkust 
demonstreerimas.}. Seal garderoobis oli püramiid, kuhu korjati numbri vastu 
relvad ära. Avamispeo ajal oli lukkudega mingi jama, need ei töötanud. Läksin sinna ja saunapõrandal oli kiht paljaid purjus 
naisi. Väga imelik koht.
 
\question{Mis lukuprogrammeerimises huvitavat oli?}
 
Seal on pusimist, et kõik asjad paika saada. Mul oli ka see 
probleem, et kui Prantsusmaal õpitu kokku võtta, siis oli põhimõtteliselt tegu diskreetse matemaatikaga. Õppisin, kuidas kompilaatoreid tehakse, 
kategooriate teooriat, eri liiki semantikat (loogiline, denotatsiooniline ja 
operatsiooniline). Aga kus seda vaja 
läheb? Tuled tagasi ja raha saad ikka selle eest, kui kellelgi mõne päris 
probleemi lahendad. See lukuprojekt läks hulluks kätte. Kõigepealt 
pidime tegema kassasüsteemi, mille külge tulid lukud, ja nii see pintsaku 
nööbi ümber õmblemine käis. Tellija tahtis hästi palju muutusi, aga ma suutsin andmemudeli kohe niimoodi paika panna, et ei pidanud seda 
pärast enam muutma, ainult juurde tuli panna. Seepeale sain järsku 
aru, et olen midagi õppinud ka.

\question{See vajab päris head rakendusvõimet, et nii abstraktset 
teemat kohe baasi mudelis kasutada. Kategooriate teooria ju ei ütle sulle, 
millised tabelid olema peavad.}

Jah ja ei. Kategooriate teooria õpetab seda, kuidas maailmas 
asjad on korrastatud. Matemaatika point on selles, et see korrastab 
mõtlemist.

\question{Üheksakümnendate lõpus tegelesid juba infoturbe ja -riskidega, 
kuidas sa lukkude juurest selleni jõudsid?}

Mul hakkas igav. Enn Lakspere\index[ppl]{Lakspere, Enn}
läks Küberisse\index{Küber} tööle. Monika\sidenote{Monika 
Oit\index[ppl]{Oit, Monika}} ja Ülo Jaaksoo\index[ppl]{Jaaksoo, Ülo} olid 
teinud turvaseltskonna enne, kui Eesti Vabariigi iseseisvus paistma 
hakkas, sest nad arvasid, et see on strateegiline oskus, mida on igal riigil 
vaja. Ja neil on selles suhtes õigus.

\question{Kas neil oli selline visioon juba tol ajal?}

Jah, neil oli enne iseseisvumist visioon olemas, et iseseisvus 
ühel hetkel tuleb ja selleks ajaks peab kompetentsi olema. Riigi 
infoturve on riigi jaoks strateegiline asi ja see tuleb korda saada.

\question{Mõnes kohas ei ole sellest siiamaani aru saadud!}

Ilmselt ei saadagi. Need olid kindlasti väga suure visiooniga inimesed. 

\question{Ja sa läksid nende juurde tööle?}

Enn Lakspere viis mind sinna. Kokkulepe oli, et mina teen uurimusi ja tema otsib 
tööd. Minu eriala olid kiipkaardid ja teda kaardid huvitasid, kuna ta tuli 
Ektacost, kus kassasüsteemide külge käisid ka kaardid. Mina pidin kiipkaarte 
uurima. Kirjutasingi raha eest uurimusi, kolm lehekülge puhast teksti 
päevas, seda on päris palju.

Vello Hanson\index[ppl]{Hanson, Vello} õpetas mind kirjutama. Osa tema 
õpetustest olen tänaseks küll ära unustanud, aga Vello Hanson on tõsiselt kõva 
vend. 

Kirjutasin näiteks Pankade Kaardikeskuse\index{Pankade 
Kaardikeskus} arhitektuuri. Keskpank tahtis sellele asjale litsentsi anda ja 
menetleda. Aga ma kirjutasin sinna ühe asja, mis oli 
Sildmäe\index[ppl]{Sildmäe, Tõnis} jaoks uudis. Ütlesin, et ärge \emph{settlement}'i ja 
raha liigutamist üldse sinna 
keskusse pange. Võtke lihtsalt info, kes kellele kui palju võlgu on, ja tehke 
bilateraalne \emph{settlement}. Ta käis üle küsimas, 
kas nii saab. Nad hoidsid sedasi paar aastat 
puhast regulatsiooni kokku. Grupivend Margus Aun\index[ppl]{Aun, 
Margus} 
läks seda värki juhendama. 

Ühel hetkel küsis Ülo Jaaksoo minult, kas
kaartidega mässamisest ühiskonnale ka midagi kasulikku teha saab. Kõrval 
oli Ahto Buldas\index[ppl]{Buldas, Ahto}, kes rääkis mulle asümmeetrilisest 
krüptost ja et sellega saab digiallkirja teha. Lugesin selle kohta veel 
kuskilt juurde ja ühel siseseminaril 1994. aastal pakkusin välja, et 
kiipkaardid võiks inimestele kätte anda ja nendega digiallkirja teha. 
Esimene avalik esinemine sel teemal oli Küberis 1995. aastal. Lõpuks müüs
Tarvi\index[ppl]{Martens, Tarvi} selle idee
riigile maha ja nii see läkski. Tarvi oli tegelikult selle asja juurutaja 
ja innovaator.

\question{Kust tuleb legend, et tegu on soomlaste tehnoloogiaga? Või on tehnoloogia
soomlastelt ja idee teilt?}

See ei ole tõsi, soomlaste tehnoloogia ei ole ka originaalne. Kui 
digiallkirja seadust hakati tegema, tellis Tarvi minu käest profiili, missugune 
see kaart peaks olema. Vaatasin ringi, mis kuskil tehtud on, ja rootslastel oli 
kaardi profiil kirjeldatud, nad tegid kolm võtmepaari. Soomlased kopeerisid 
rootslasi ja panid kaks võtmepaari kokku. Eri võtmepaare on vaja 
seetõttu, et neil on täiesti erinev poliitika. Autentimise ja 
krüpteerimise võtmeid ei tohiks kokku panna, sest krüpteerimise võtmel peaks 
olema taaste, kui tahad seda pikaajaliseks säilitamiseks kasutada. Allkirja 
võtmel aga ei tohi olla taastet. 
Autentimisvõtme jaoks ei ole mingit põhjust taastet tahta. Need on erinevad 
poliitikad, mis tegelikult ei sobi hästi kokku, aga nii on lihtsam inimesi 
õpetada. Turbe põhimõte on see, et ainult lihtsad asjad töötavad. 
Jaapanlased võib-olla saavad keerulise asjaga ka hakkama, aga meie ei saa. Ja 
kuna ükski inimene krüpteerida ei oska, siis talle tuleb anda arvuti, mis teeb
seda tema eest. Kiipkaardist lihtsamat arvutit ei ole olemas.

\question{Ja nii jõudsidki infoturbeni?}

Jah.

\question{Infoturve tundub olevat sinu juttu kuulates ideaalne kombinatsioon: 
sai asju ära teha, oli palju matemaatikat ja arvuteid, kõik kenasti koos.}

Küber oli üsna selge teadusasutus: väga palju 
tehti teooriat ja natuke kirjutati ka programmi. Arne\sidenote{Arne 
Ansper\index[ppl]{Ansper, Arne}} oli juba sel ajal kodeerimises kibe käsi.

Ma läksin sealt ära sellepärast, et tekkis tunne, et kirjutan 
igasuguseid plaane ja siis teised mehed plaanide põhjal ehitavad. 
Ühispank\index{Ühispank} oli viimane pank, kellel ei olnud oma 
kaardiserverit, nii et ma läksin seda 
tegema. Ja kuna turvainimesi ka ei olnud, siis pidin olema turbe ja 
maksekaartide peal. Edasi läksin Hansasse\index{Hansapank}. 

Tore oli pangasüsteemi ringitõstmine, kui tuli ühendada kõik 
maapangad\sidenote{Ühispanga asutasid 15. detsembril 1992 kaheksa 
maapanka, Viljandi kommertspank ja Nordpank}, ning selleks oli vaja süsteemi. 
IT-direktor oli 
Novelli-mees. Ma rehkendasin talle \emph{roundtrip} aegadega, mitu 
transaktsiooni ta tänu lukustamistele jõuab üldse päevas teha, see oli kas 30 
000 või 40 000. See tähendas, et tal Tallinnas oleva 
panga jaoks jätkus, aga terve Eesti peale oli vähe. Siis sai Unix sinna alla 
valitud, et lukustamine käiks ühes masinas ära. Tema valis millegipärast HP, aga
tegelikult oli HP-UX\index{HP-UX} väga äge Unix. Inimesed arvavad, mis 
nad arvavad, aga väga robustne riist. Solaris, millest tavaliselt 
räägitakse, oli tükk maad hellem. 

Toona oli Windowsiga selline õnnetus, et TCP \emph{stack}'i eest 
pidi eraldi raha maksma, Linuxi käimapanek oli kaks korda odavam. Seepärast ühendatigi
kõik maapangad niimoodi ära, et pandi Linux \emph{front}'i ja 
ühe ööga keerati kontor ringi, süsteemi vahetus ja \emph{front}'i vahetus. 

\question{Mida sa praegu teed?}

Praegu uurin kriitilisi sõltuvusi, see teema on pärit 
RIA\index{Riigi Infosüsteemi Amet}\sidenote{Jaan oli aastatel 2011---2015 Riigi 
Infosüsteemi Ameti peadirektor} ajast. 
Kui pead vastutama niisuguse asja eest nagu massiivne küberrünnak, siis 
selle juhtimiseks lükkad kokku staabi. Ja esimene küsimus on muidugi, kas see, mida sa näed, on õige. Niisugust infosüsteemist 
sisse tulevat müra, kus pead süsteemi enda käitumist rünnakuks, on 
päris palju. Teine küsimus on, mis edasi 
juhtub ja kust rünnak peale hakkas. Tavaliselt ei oska inimesed kummalegi 
vastata. Minu algne mõte oli, et äkki nendel sõltuvustel on mingi 
võrestruktuur, võre moodi osaline järjestus. Kui leiad
selles osalises järjestuses miinimumi, siis võib see olla algpõhjus. Ja kui võtad transitiivse 
sulundi, saad kõik tuleviku asjad kätte. Nii tekkis mõte hakata pakkuma 
planeerimisabi. Selleks aga on vaja 
need sõltuvused kuidagi kirja panna. Nüüd olen saanud nii palju targemaks, 
et mu arvamus, et seal ei ole tsükleid, ei ole õige. Tsüklid on ja neid on väga 
palju ning väga lühikesi. Kogu majandus on tegelikult tsükleid ja tagasisidet 
täis ning ma ise olen dünaamilisi süsteeme õppinud ja näinud, mida sellised asjad teevad. Ühesõnaga, nüüd üritan neist asjust aru saada.

\question{Kõlab, nagu oleksid asjad jätkuvalt huvitavad ja see on üks väheseid 
olulisi asju. Või eelistad sa igavaid?}

Ei, seda kindlasti mitte, aga mõnikord võivad need liiga huvitavaks minna. 
Keerukus kipub kasvama ja seda peab jõudma
jõuga maha võtta --- \emph{refactoring} on alati töö.
