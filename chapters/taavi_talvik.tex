\index[ppl]{Talvik, Taavi}

\question{Kuidas ja umbes millal sa jõudsid arvutite juurde?}


Tere, see siin on Memkopi ja meie teine hooaeg nagu ka eelmisel hooajal on plaanis viisteist episoodi, kuid erinevalt eelmisest on selle hooaja alguseks kuus episoodi juba purgis, seetõttu loodetavasti vähem viimase minuti rabelemist. Samuti olen natukene parandanud ristitarkvara olukorda ja audio kvaliteediprobleeme, rohkem ei tohiks esineda. Aga külas on meil täna Taavi Talvik mees, kes nagu ka meie teised külalised üldiselt tutvustamist ei vaja. Kuid kelle käsi on olnud mängus eesti internetiseerimise juures tegevuse tagajärgi ühel või teisel viisil kõik tajunud olema, sest tema tegi niisuguse asja nagu uninet. Ja see on suur asi Eesti internetis. Head kuulamist.
Tere, tere. Sinu nimi on minu nimi, on Taavi Talvik.
Oleme kogunenud siia suurepärasesse kontorisse, kus me oleme juba korra varem käinud rääkimaks sellest, millest ei ole võib-olla ammu hästi räägitud. Ehk siis sellest, kuidas asjad alguse said. Aga hakkame sellest pihta, kuidas asjad sinu jaoks alguse said. Kuidas arvutite juurde jõudsid?
Arvutite juurde jõudmine iseenesest on väga lihtne, et kodus sattusid olema paar põnev põnevat draamat, Totteta ise isegi enam ei mäleta, mis see täpselt oli, kas see oli mingi Ustus, Aguri abakusestraalini või või, või mingisugune Norbert Wiener'i mingi küberneetika või? Igal juhul mingi niisugune raamat oli, tundus jube põnev, et ja noh, siis kuna esivanemad olid tööl Tartu ülikooli juures keemikutena ja, ja.
Olid kuulujutud Ülikoolis ikka mõni arvuti on, et siis ma kohe hakkasin neile pinda käima, et kuulge, et ma tahaks panna ja näha, missugune see arvuti päriselt välja näeb.
Aga kas ühesõnaga saad siit kohe järeldama, et sa oled Tartust?
Jah, ma olen Tartust, et selles mõttes tort oli jumala okei, niisugune väike väike linnakene Elva lähedal ja lapsepõlves mulle seal väga meeldis väike puust linn. Kui vana sa olid, kui sa vanematega hakkasin, ma arvan, et see oli kuskil ütleme niimoodi, üheksas kümnes klass pigem üheksas ja, ja tõepoolest neil seal ülikoolis arvutid oli, siukseid, väljamaa omi isegi. See oli aasta umbes kaheksakümmend viis ja, ja siis välja oma arvuti oli suht niisugune haruldus, aga, aga kuna nad tegid mingisuguseid imelik elliptiliste kilede mõõtmisi, siis selle kilede mõõtmismasinaga oli kogemata ta koos ostetud mingisugune arvuti, mille nimi oli Hewlett-Packard kaheksakümmend viis. Oli selline lauaarvuti, ei olnud, see oli lauaarvuti, kus oli sees pisikene. Ma ei tea, viietolline ekraan, klaviatuur, kassetid jaa, jaa, jaa. Termoprinter ning taga oli hunnik juhtmeid, mis ühendasid teda siis selle mõõtmismõõtmisseadmetega, niisugune mudel ei ole küll läbi kellelegi jutust, et kõlab täitsa nagu eksootilisi, see on iseenesest väga eksootiline mudel. Ja tal oli mingisugune oma Hewlett Packardiprotsessor, mis omas omas ajas, oli isegi täitsa innovaatiline ja tore. Kuigi protsessor, protsessor, eks millega see välja paistis, oli tavaline peesik ja see, et ekraani peal sai jutte joonistada ja, ja ütleme kui ise jutte joonistada ei osanud, siis sai mingisugust pingpong pingpongi või kosmonautide maandumist mängida, aga siin tõesti kohe siis sinna juurde. Näe, poiss, võta läbi. Jah. Teie täiesti niimoodi ja noh, eks nad, eks nad sealt, et vanemad inimesed, kui ma õieti mäletan, siis tema nimi oli Zirk õpetasid ka, mis, et kaua sa siin mängid, et proovi kokku liitorve ühest kümneni või midagi sihukest, noh sealt need asjad pihl pihta hakkasid. Okei, koolis ei olnud mingisugust nihukest, just kuulsin Tartu kümnes keskkool, mis on tänapäeval siis Mart Reiniku gümnaasium. Koolis ei olnud see aeg veel mitte midagi, täitsa täitsa täitsa tühi maa. Tõenäoliselt samal ajal noos midagi oli, aga, aga nõo Tartust nii kaugel ja noh, selleks peab ikkagi nõos tutvusi olema. Tseme seid, nagu keegi kutsuks.
Kui ma mõtlen siukse üheksanda ja kümnenda klassini peale, siis seal kipuvad igasugused muud põnevad hobid olema, selle asemel, et lugeda härra viineri küberneetikat või ka agulis
Teost, miks sa lugesid seda? Tore oli, et huvitav oli ja, ja noh, võib-olla vanemad sokutasid ka midagi, et loe, poiss, et, et järsku saad targemaks või midagi nihukest. Noh, eks see tagantjärgi tarkusena sekrilt enam ei mäleta, mis see täpselt see vajanud oli.
Seepärast küsin, et kas sul oli, oligi populaarteaduslike asjade huvi.
Kui oli hea, võiks selles mõttes hooli populaarteaduslike ostjate huvi oli ulmehuvi ja kuna nagu nupp selles reaalteadustes jagas alates igasugustest nendest olümpioodidesti asjadest, siis see tundus nagu naturaalne
Oleks nagu loogiline ja sealt edasi. Kuidas sul see reaalteaduste jagamine nagu, nagu esile kerkis, kas sa kohe nagu esimesest klassist alates tundsin ennast selles osas mugavalt, tegeles keegi su arendamisega spetsiifiliselt? Ma arvan, et põhikool või midagi.
Tundsin suht mugavalt tänu sellele, et isa-ema olid ülikoolis õppejõud ja aeg-ajalt nad ikka keemikutega midagi midagi rääkisid ja, ja ja alati nende käest on ju alati võimalik küsida, et kui ma kuskil füüsikas, keemias, matemaatikas etajain ja noh, kui sa hädast üle saad, keegi sind hädast üle aitab, siis siis endal tekib mugav tunne ka, et ja, ja ei saa nagu vastane väsis.
See on hea turvaline toimetele täpselt. Aga kõlab ju, mis sa siis keemiku teed ei läinud?
Arvutid olid põneva. Peale seda, kui üks etapp oli antud, siis siis see põnevus järjest järjest lihtsalt kasvas.
Seal nagu noh, selle ühe HP selle kaheksakümne viiega või mis ta oli, et ega see ei saanud ju nagu väga kauaks põnevaks jääda.
Eks seal oli, olid omad asjad, et natukene mingid trips traps, trulli laadseid mänge kirjutada ja sai selle esimese hea edukogemuse kätte ja see sealt siis edasi sai kuhugile järgmistesse kohtadesse kus olid siis veidi ägedamad ja võimsamad arvutid.
Et keemiahoone kõrval oli Tartus olemas just kõrval, aga, aga lisaks oli ka füüsikahoone, kus, kus oli see selline inimene nagu Alo raidaru, kellel oli päriselt kuskilt saadud PC-laadsed arvutid sama tähe tänaval läenäosutaja neli kuskil seal keldrikorrusel Tauli ja pissilaadsete arvutitega sai juba teha väga palju rohkemat, kui selle väikse õnnetu happe kaheksakümne viiega. Ja noh, lõpptulemusena siis umbes kümnendast klassist alates füüsikahoones võeti mind nii-öelda laborandina tööle ise endisi, nii isegi niimoodi ja siis siis tööülesandeks oli üht või teist või kolmandat või neljandat programmeerida. Kümnenda klassi poisina.
Oh, kus sul see programmeerimisoskust
Tuli siis no ma ei tea, tuli järjest kõik, kõik, kõik aitasid kõrvalt ja õpetasid ja noh, kuidagi sisse naturaalselt kasvas.
Okei see kõlab päris nagu kiire normaalniukene, normaalne kasvamine, et tühja koha pealt nagu laborandiks.
Aga ei noh, seal selles mõttes ma loodan, et ma tegin seal isegi midagi kasulikku ja, ja, ja ma sain sellest palka ja sain palka ikka täitsa kõvasti laborandi palk oli mingisugune viiskümmend rubla kuus, mis kümnenda klassi poisile oli sihuke sihuke Kresuse tunde tekitas, et see oli väga palju. Ja, ja eks see noh, laborandi pank natuke toetas kõiki neid huvisid ja värke. Tulemus oli see, et enamasti peale koolitundide lõppu oli mitte koju minek, vaid sinna füüsikahoonesse minek.
See on ju, see on ju arusaadav? Ikkagi mind, mind ei jäta rahule see, et, et kui sa niimoodi ise arened, siis peab olema mingisugune niukene huvi, mis siit viib selle asja juurde.
Absoluutselt noh, mis tegi põnevaks. Põnev põnevaks tegi ikkagi see, et kui sa arvutile mingisuguse programmilaadse asja selgeks teed siis ta teebki midagi, mida sa arvasid, et ta võiks teha. Tihtipeale ei tee, aga väga tihti ta tegi ka ja see oli jube kihvt, kui, kui midagi juhtus. Mingi asi ollus sinu kurjusel, mingi asi allus minu korraldusele, noh, täiesti niisugune unikaalne situatsioon maailmas.
Jah, eriti Tiina üles ja kontrolli puudumine võib olla päris absoluutselt suur probleem. Ja oskad sa mõnda näidet tuua, et mis sa seal laborandina prognoosisid?
Laborandina progesin. No üks asi, mis kindlasti meelde tuleb, on antiviiruse antiviiruse kaheksakümmend millegagi kaheksakümmend kuus umbes selles mõttes, et siis sihuke
Avastushetk, et maailm hakkas vaikselt lahti minema ja vaikselt liikuv liikusid, ilmusid viirused Eestis sega ja, ja siis tekki, tekkis see jama, et viirus jõudis sinna meie juurde ka ja kui sellest oli koja kuidagi lahti saada, kuna arvutid hakkasid imelikult käituma. Viiruse nimi oli, kui ma õigesti mäletan, jänki Tuudel mis tegi siis piikse ja ekraani peal vist hakkasite?
Kukkuma jah, tähele hakkasid kukkuma ja siis ta mängis jänki Tuudeid ka, kui ta riistvara võimalik
No ja ja niisugune viir viirus oli ja siis sai uuritud, kuidas see käitub ja siis tehtud pisike programmikene, et sellest lahti saada oli täitsa võimalik. Aga noh, see on niisugune lihtsalt tore mälestus. Põhiline, mida seal Aloiduru laboris tehti, oli tehti elektroonikat füüsikutele ehk mingisuguseid lisasid nendele mõõteseadmetele, katseeksperimentidega ja nii edasi ja seoses sellega nad tegid ise trükkplaate ja trükk plaatide tegemiseks olid esimesed need Smart käädi või laadsed programmid, millega õnnestus elektroonikaskeemi joonistada, trükkplaat joonistada ja taga siis kas välja printida või siis Alo ehit, kuidas arvuti külge freespingijuhtimise interfeissimis free siis selle trükkplaadi välja. Ja, ja aga noh, lisaks välja freesitud trükk, padi radadele oli vaja seda, et Need läbiviigu augud ka puuritakse ja see näiteks läbiviigu aukude puurimise programm oli see, mis usaldati mulle.
Aga nüüd, kui ma nüüd siis moodsasse terminoloogiasse kohe niimoodi tõlgin, siis sa tegelesid kohe esimese hooga iiluateega.
Jah, see, seda võib tänapäeva Liioteega viot teeks nimetada, aga noh, tegelikult oli see trükkplaatide aukude puurimine ugri nimetame Rubootikaks. Ikkagi nimetama õige asja nimedega puurpingi puuri, õigele kohaleviimise ka siis käsuandmised, vajumine olla ja tule üles tagasi.
Ja see pidi siis lahendusena olema, kuna seal tõenäoliselt mingisuguseid valmisid, teeki või valmisliideseid ei olnud, see oli siis nagu sinu programmist kuni restorani välja.
Põhimõtteliselt küll, et sellest samast käädi käädi programmis sai aukude koordinaadid. Jaa jaa. Nende koordinaatide peale lihtsalt tuli see augud puurida sena seal vahepeale sai mingi puuride liigutav puuri liigutamise keel, et vastu sada sammu siiapoole, mine olla vastu, sadas hamba sinnapoole tule üles. Ei. Kohe tuleb üles tulla ikkagi. Ja nii edasi.
Ja selle keele mõtlesid välja sina.
Ross, eks need vanemad inimesed kõrvalt ikka aitasid, et et nii kõige kinni jooksid, õige tuli keegi ikkagi appi.
See on hästi turvale jällegi absoluutselt hästi turvaline. Kasvukeskkond kasvatab, teised on rääkinud, et, et nad üsna varakult saidingi teiste omasugustega kaar ninapidi kokku vahetasin infot ja tekkis mingisugune niukene kogukonna moodi asi.
Kooliajal ei olnud aga pärast kooli ülikooli astudes see tekkis üsna üsna kohe tekki tekkis kogukonnatunne ülikooli esimesel kursusel.
Aga ärme veel ülikond ülikooli lähme, et see keskkoolis õppimist segama ei hakka.
Ei otseselt ei hakanud.
Kuna nupp natukene lõikas, siis võis mõne mõne koha pealt üle nurga lasta, et ei ole. Et noh, selles mõttes koolis jah, lõputult pingutada ei ole vaja, võib-olla seal eesti keele kontrolltööd läksid kehvemaks, aga aga noh, ütleme üldtase sinna nelja juurde jäi. Täitsa okei, ütleme kooli lõpetamisel oli, oli tunnistus umbes selline, et kõik olid neljad, välja arvatud üks, viis, üks, kolm sest keskmise küll ütleksid neljaks mis see kull enam ei mäleta, lihtsalt ei mäleta, või ma võimalik, et oli vene keel, aga ma hetkel enam ei mäleta.
Vaat nüüd me jõuame ülikooli juurde, et see, mis koolis läksime
Tartu Ülikooli füüsika füüsikateaduskond.
Ja seepärast, sest seal sa juba olid laborandina.
Laborandina oli käsi sees ja, ja noh, tegelikult füüsikaga nähtus huvitas ka ja, ja selles mõttes tegelikult füüsikaga nähtuse huvitas oluliselt rohkem kui matemaatika. Et füüsikas oli, oli nagu see
Keerukusaste väiksem selles mõttes, et matemaatika, matemaatikud, need läksid mingid teise tuletise või seitsmenda tulebki sinna välja ja ja, ja samas kui füüsikud ütles, et, et teine tuletis, see on nii ebaoluline juba, et seda efekti sellel kursusel ei aruta, see jääb kolmanda kursuse materjaliks ja see mulle jumalast sobiks.
Muidugi väga huvitav, sest mina just mõtlesin vastupidi, et, et füüsikas on ikka päris maailm ja see on nagu messi ja keeruline ja seal on mingisugune.
Ei, vastupidi vastupidi, et seal on mingisugused, suhteliselt lihtsad rusikareeglit, kui kui nendest suhteliselt lihtsatest rusikareeglitest aru saada, siis need noh, peenhäälestamine, see tuleb peale ja nagu ma ütlesin, see tuleb nagu järgmise kursuse materjalist esialgu kõrvale jätta.
Okei, nii võttes küll jah. See, aga nagu huvitav, olidki seal mingi spetsialiseerumine ka seal tekkis.
Vot spetsialiseerumisega läks natukene natukene sandisti, sellepärast et kohe peale tuli Vene sõjavägi ja, ja, ja peale Vene sõjaväkke ma küll täitsa jätkasin füüsikas kaks pool aastat, aga, aga noh, siis tuli ka muu elu kõrvale ja siis nii-öelda õppimisvaimustus vaikselt. Ütleme kuidas seda viisakalt öelda, siis laius mõjus väga viisakalt. Kus sa teenisid Valgevenes selline koht nagu Borissov kolmteist on niisugune super koht.
Ei ütle mitte midagi, ilmselt mitte kellelegi.
Tõenäoliselt mitte kellelegi alla kõigele, kes oldi. Aga Valgevene ja, ja eks ta oli sihuke sihuke mõnes mõttes ajaraiskamine, teisalt siuke, et sa nägid maailma, et kui palju erinevaid inimesi tegelikult olemas on.
No see arvutiinimesed on üsna niukene nagu silmi avav.
Ilmselt jah, on selles mõttes, et kui palju on palju, palju meist ütleme kaheksakümne kaheksandal, kui ma sõjaväe läheksin, palju meist reisinud, tegelikult olime võib-olla Nõukogude liidu piires siin-seal kuskil käinud, aga see, ega see reisimine ei olnud niisugune nagu teema, mida kõik on teinud ja see, see uute inimeste nägemine tegelikult oli sele sõjav seda päris kasulik kogemus tagantjärgi.
Nojah, sest seda ma just pidasin seda silmas, et arvuti inimesel on tõenäoliselt nagu eri tüüpi arvutite eri programmeerimiskeelte niisuguste asjadega nagu suurusjärk, rohkem kogemust kui eri tüüpi inimeste ja seda kindlasti kultuuride siukse asjaga. Vot siis sa tõid tagasi aastani.
Tagasi tulin, aasta oli kaheksakümmend üheksa, mul õnnestus Vene sõjaväest pääseda ühe aastaga, kuna Gorbatšov ütles, et üliõpilased nemad on meie sotsialistliku riigi tulevik ja minge õppige ülikoolis parem edasi, mitte ei jookske püssiga, ärge jookske püssiga ringi.
Ja siis tulevik tuli Tartu ülikooli edasi.
Tulevik tuli Tartu Ülikoolis edasi füüsikat õppima ja noh, siis ma proovisin ka spetsialiseeruda astronoomia peale. Joosta, surume jälle niisugune niisugune juhus, et,
Tõenäoliselt sa tead sihukest ulmekirjanikku nagu Vaisaka sõimub, olen kuulnud ja suure tõenäosusega oled kuulnud, et lisaks sellele, et ta oli ulme, kirjalik, ta oli jube hea teaduse popule populariseerija ja, ja jälle kodusest raamaturiiulist oli mingisugune raamat, mille, mis, mille pealkiri oli siis universum. See oli küll tõsi küll, venekeelne Psylemm ja aga ta kirjutas niivõrd fantastiliselt seda, kuidas toimib. Et see nagu jäi kuklas kripeldama, et noh, et järsku peaks edasi uurima seda teemat ja, ja see tundub nii kihvt olema ja siis ma proovisin sinna spetsialiseeruda.
Mis, mis see tähendab, proovisid?
Kas see välja ei, otseselt otseselt ei tulnud välja, kuna sellest koolieelsest tööelust kasvas välja järgmine tööelu, mis hakkas natuke õppimist segama. Et seesama Palo, Raido sokutas mind tööle ajalehte edasi.
Ahaa et siis ei olnud ju vaja.
Edasi ei olnud auke puurida vaja, aga aga ka kaheksakümne üheksas aasta oli juba see aasta, kus reaalselt tekkisid ka ettevõtetesse esimesed arvutid. Jaa jaa.
Tekkis niiöelda esimene desktop abishing. Ja kuna üks selle selle olo hobidest oli ülikooliteatmiku väljaandmine, siis tema käest küsiti nõu, et kuule, et meie saime Edasis ühe arvuti, et kas seda saaks kasutada kuidagi ajalehe väljaandmise abiks. Nii, ja siis mind, mind sokutati sinna, et kuule, Taavi, sina mine aita neid, see neid, neid edasi tegelasi siis Maitasingi näiteks neli-viis aastat.
Oh, see, see aitamine pidi siis olema puhtalt ainult ju selle Paulisingu programmi käimaajamine.
Ei, mitte ainult pobishingu programmi käimaajamine, vaid seal on võimalik. Reaalsuses on mingisugused töö rutiinid, et kui need lähevad lihtsamaks kevadel, kiiremini. Sulle nüüd omakorda küsimused, mis sa pakud, et mis oli esimene asi Edasis, mida arvutiga automatiseerita.
Aastal kaheksakümmend üheksa eesti keele spellerit ei ole ju veel.
Eesti keele speller oli ka olemas juba, aga see selleks, mis võiks olla see teema, mida automentiseeriti ei tea, ei oska. Väga lihtne see, see oli see teema, kust raha tuli ajalehte, surmakuulutused surma kuulasid. Et noh, tänavu tänaselgi päeval on Postimehes populaarne paar viimast lehekülge, kus on Need surma surmakuulutused, nende nendele on see hea omadus, et nad on suhteliselt standardses formaadis, seal on mingisugune noh, neli-viis, võib-olla kümmekond erinevat kujundust meile ja siis, kui need kümmekond erinevat kujundust kuidagi mallidena ära implanteerida, siis see surmakuulutuse nii-öelda Väljaandmise publitseerimise aeg kukkus drastiliselt. Nojah, ja, ja kuna see oli sisuliselt ainuke allikas, kust Lisat tellimusele raha tuli, siis selle vast oli lehe juhtkonnas ikka päris normaalne.
Okei aga mis see, mis see väljund?
Väljund oli kile peale trükitud. Selline.
Lehekülg, mis läks siis ofsettrükki?
Ühesõnaga sinu tarkvara, mis automantiseeris seal asja otsekilele.
Jah, otse kylal, laserprinteriga lased paberi abil läbi kile ja siis see asi, mis sealt välja tuleb, see on enam-vähem see, mida saab trükkalitele kätte anda, et kleepige siis.
Nendesse õigesse kohta südamesse produtseeris Postgüti.
See on juba päris keeruline.
No eks seal oli, seda pabistasin tarkvara. Kui ma õigesti mäletan, siis Ventura publis mis tegi põhitöö ära ja seda otseselt puskrüpti tasemele, oli väga mõne mõned üksikud asjad.
Aga see saab jälle kõik teadmist saadet, kus sa seda teadmist juurde hankisid.
No.
Istud ja nokid ja nii ta on, kuid leida tuleb lõpuks küll ta tuleb. Kus ta pääseb?
Kus ta pääseb? Sa ütlesid enne, et sul tekkisid seal juba mingisugused kogukonna moodi asjad.
Ja, ja kogu kogukonna moodi asjad, et üliõpilastena sa käid ikkagi seltskonnaga ringi. Proovid Ühes arvutiklassis proovid teises arvutis, D-klassis, see hetk olid juba tekkinud Tartu Ülikooli matemaatikateaduskonda ka arvutiklassid ja seal sealsete inimestega suheldes mingi kogukond vaikselt tekkis ja samamoodi tekkis kogukond ta nendest, kes olid mul kursusekaaslased. Füüsikas.
Luke tol ajal mingeid arvutiside
Tol tol ajal veel arvutisidet ei olnud. Aga eks see arvutiside tuli ka suhteliselt kiiresti, et ühtedel meest, et meestel oli ühte laadi arvuti, teistel teistlaadi arvuti ja flopikettaheide vahel ei lugenud, siis pandi kaks või kolm traati omavahel kokku ja prooviti neid jälle kuskilt saadud programme teisele mehele ka üle kanda.
Okei.
Ma küsin korra selle edasi kohta veel, et kuidas see töövoog välja nägi. Ikka tahad küsida, et see ajakirjanik kuidagi kirjutas, millega?
Siis edasi aegadel ajakirjanik kirjutas ikkagi kirjutusmasinaga ja siis oli tinaladu. Aga kui tekkis nagu rohkem arvuteid, siis siis läks dinoloolt üle ka ütlema selle kile peale trükitud väljundini. Seal oli terve hunnik etappe veel vahel, et.
Et ajakirjanikele arvutit saada, arvuti oli tol ajal suhteliselt kallis asi ja, ja terminalid olid natukene odavamad. Siis sai Postimehe see, kes oli siis juba erastatud ja postimeheks muutumas ja sai pandud üles üks juuniksi server, kus oli küljes kuusteist terminali mis siis ajakirjanikele maja peale laiali sai veetud terminalid ühenduseks. Need vajalikud kaardid sai Tõraverest, seal oli mingi uraani nimeline firma või millest kasvas välja Astordaata ja, ja.
Teksti sisestamiseks, ajakirjanikul oli terminal piisavalt lihtne, et, et seal ei olnud vaja tal seda ajada, ilusaks peasid, tekst olemas on. Ja seesama tekst siis võeti ja pandi mingi pabistamine, tarkvara ja jälle kile peale välja. Kleeplindiga kleebiti küljeks kokku, mis läks siis öösel trükikotta kunagi ja mis juuliks seal serveris juhtus. Seal serveris jooksis BSD juuniks mis sai täiesti ausalt ostetud soos koodiga kõigega kõige värgiga. Ja.
Nii ta oli tol ajal ju tol tol ajal oli ju lausa mingi embargo asi oli.
Embargot ja asjad olid ka, aga need kolm kaheksa kuuelaadsed arvuti demorga olla ei kukkunud. Ülemine ots, nagu pyydipi oli embargo all.
Ja sisse UNIXi purk käis ka sinna, mitte embargo segas mitte embargo alla. Siis ikkagi ta ka mingit erinevi jaksas vedada.
Jaak rahulikult jaksas need kuusteist terminali välja vedada ja, ja noh, eks arvuta hankimine tol ajal oli nagu sihuke keerukas tegevus, et a la kui telli Postimees oma tellimise rublad kätte, siis, siis veeti need kohvriga. Minge oskuslik ärimeeste juurde, kes siis kuskilt oskuslikult Moskvast said mingi arvutit. Et noh, see oli, too aeg oli nagu niisugune vorsti kaubaaeg. Ja, aga lõpuks oli need vajalikud arvutid võimalik välja ajada, vihikud on, me oleme sellest rääkinud just ja, ja mõned kohvrid jõudsid tema juurde ka ja tema siis oli.
Põhiline Postimehel arvuti tanki.
Jällegi mind pani imestama, kui sujuvalt läheb, nagu skoop läheb nagu laiemaks. Et kui ma sellest programmeerimise asjast veel saad aru siis sellest, et sul see huvi läks nagu laiemaks, nagu mingite uniksite serverite püstipanek, Pablishing ja nii edasi. Et mis sind hoidis nagu laiendamas seda, et oleks olnud lihtne nagu keskendutaks programmeerimisele või.
See ma arvan, et siin oli just siin oli just see, et see seltskond ümberringi
Postimehe ajakirjanikud või edasi ajakirjanikud, neil läks ka silm särama, et näed, niisugune võimalus on seda oma oma tööd paremini teha ja see tekitas nagu surve eelkõige selleks, et neid kuidagi aidata. Et kui, kui inimesel silm särab midagi tehes, kui sa teda aitad, läheb sinul endal ka silm särama ja noh, põhimõtteliselt nii lihtne see ongi.
Aga siis see eeldab ikkagi mingisugust elementaarset asja üldse sedastada mingit elementaarset huvi inimeste vastu ka arvutiinimeste hulgas. Arvutid.
No sellel absoluutselt, aga kui, kui sa igapäevaselt kellelegi kõrval istud, tahes-tahtmata huvi tekib ju, et ei ole võimalik, et ei tekiks ja, ja kui sa istud veel intrigeerivate inimeste kõrval, kes nii-öelda hoiavad kätt elu pulsil, räägivad sulle, oh, seal Tallinnas Toompeal räägiti seda ja toda ja okas paneme selle lehte või parema, hakkab kõrv liikuma küll ja, ja tahad selle, selle selle melu sees olla?
See isegi aega olla küll, sest ilmusid Nelli teatajale hakkasid ilmuma Nelly hiljem, see oli bossistoli siis.
Tartlasena ma neid kõiki Tallinna asju ei tea, hakkas ilmuma see Eesti Ekspress, Liivimaa kuller Kalle Mülleri ja Väino Koorberg vedamisel siis kroonika alustuseks Kalle Mülleri, Sindi, Ingrid Veidembergi vedamisel. Kõikide nende juures olid mingisugused momendid, mis olid nii-öelda mega kihvtid huvitavad, et alguses oli mustvalge, siis tekkis värviline logo siis ja nii edasi ja nii edasi.
Nii suur asi, kooli värviline logo, see ja kõigi nende juures oli mingisugune niisugune Pablisshingu või trükivõi niukene inimene, nagu ametis kas või Peeter Marvet juba kuskil tõmbutes tõenäosust.
Et tõenäoliselt Tallinnas tembutas, aga noh, Tallinn ja Tartu on siuksed erinevad asjad täiesti.
Aga seepärast ma tahtsin küsida, et kas nendel Paulistusega inimestel mingit oma üks kuu on tekkinud ikka vahetasin programmiga asju.
Kindlasti oli, aga vot see on see teema, mida enam ei mäleta, lihtsalt enam ei mäleta, et pärast oli neid asju nii palju peale, et.
Okei, aga ühel hetkel sai edasi asi otsatmise eesti
No edasi asi enne kui ta otsa sai, on kindlasti üks selline huvitav moment, millest ma tahaksin rääkida. Ta edasi ka seoses. Ma sain umbes aastal kaheksakümmend üheksa või üheksakümmend nüüd.
Ei vastuta kummagi numbriõigusest, sain emaili endale. Saidi emaili, jah, siukse emaili meil mille aadress selle umbes niimoodi, et Taavi jäät EVS-i.
E-tabeli punkt SO suunagu Soviet Soviet Union Eesti vabariigis ja niisugune domeen on olemas, jah, niisugune domeen oli olemas ja kusjuures domeen nagu SO on endiselt olemas.
Ja Siimile sõidumale. Jälle mõnes mõttes tänu ülikoolile, kuna ülikool oli ülikooli psühholoogia teaduskonda aasta Tiit mugavam oli käima ajanud Tallinnas.
Küberisse või KBFIsse seda enam ei mäleta, modemega uudsebee meilisideme ja, ja sealt siis tulise meiliaadress. Ja, ja siis sellest ajast ma veel mäletan, esi esimees niisugune suuremamahulist internetioste, interneti või meili umbeajamise intsidenti. Et noh, siis siis olid siuksed asjad nagu lingvistid, mida sai tellida ja, ja noh, siis ma kogemata tellisin mingi siukse aktiivsema kirjavahetusega meelingvisti ja siis kui meile muudkui tuli ja tuli, modem ei pannud toru hargile ja ei pannud ja pannud, pannud läheb tund ja teine ja kolmas täitsa pekkis on kogu see maailm umbe läheb katki ja siis ma võtsin jalad selga üle Toome, kõndisin sinna Tiidu juurde, kuule, aita mul see meilivoo kärakat Kestova. Et noh, selles mõttes siis oli umbes moodeme helistamine minu juures tema juurde ja tema juurest kuhugi Tallinnasse ja Tallinnast võlgu Soome ja mis iganes.
Kusjuures see, see on ju tänapäeval täitsa unustatud, kui oluline asi oli nii meili, mis kõik asjad käisid üle emaili, oli olemas niisugune asi nagu FTP üle emaili ja mingisugused failid, keerati õige sobiva pikkusega tükkideks ja lasti PS kuuekümne neljaga kokku.
Sa oled täpselt, eks ole, et siis oleks see sellise, sellisel kujul oli võimalik kuskilt list serveritest või arhiiviserveritest endale nii-öelda tellida vaba tarkvara lähtekoodi.
Ja isegi minu meelest isegi mingisuguseid pilte levitati.
Pilte levitati, aga seal see kõige huvitavam tol ajal oli. Meie jaoks oli see just see vaba tarkvara lähtetekstide kättesaamine üks huvitav selles mõttes, et siis sa said jälle mingi uut uut võimsamat asja teha selle sammu edasi juures sellesama edasi juures. Noh, eks seal järjest neid automatiseerimis asju tuli peale, et pinge tehti oma kojukanne. Et ma kojukande jaoks oli jubedalt abiks, et postiljoni tele, Need pakid, jagataks siht sihtrajoonide järgi, õiged kleepsud oleks peal õigesse hunnikusse, õige õige kogus ajalehti saadaks, oleksid neil oleks nimekirjad, mille mille järgi viia ja, ja sihuke kujukond infosüsteem sai näiteks tehtud ja siis on jälle jälle, mida tänapäeval ei, täna see kindlasti on olemas Eesti Eesti postis või kommunimas on kojukandeinfosüsteem raudselt olemas.
Just, aga see on asi, mida tänapäeval sagedasti ei juhtu, et sa astud uksest sisse, hakkad nullist programmeerima mingisugust kujuga impostimine, nagu tüüpilist võetakse mingi asja alla.
No see oli see aeg, kus ei olnud võimalik võtta olla lihtsalt ei olnud aga, aga, aga äriliselt Postimehel oli ainuke võimalus teha ise kujukondades süsteem, kuna see riiklik kujuga neid toiminud ta ei saanu. Ta viis ajalehe kätte mingiks lõunaks. Postimees tahtis, et hommikuse kohvijoomise ajaks oleks olemas ja ehitas nullist üles oma kohukojukandesüsteemi, mis pärast rist liitus siis Express Posti omaga. Ja praegusel hetkel vist alternatiivsena kojukandesüsteemina mingi mingis ulatuses isegi toimiv
Sest ma mäletan seda küll sellepärast et see Postimees, ta oli hommikul vara, oli postkastis, oli mingeid täiesti nagu läänes, see oli nagu päris. See oli väga kõva sõna. Et aga missa peale listide lugemises emailiga veel tegid.
Üks üks asi, mis selle emaili teemal on ere õigupoolest kaks asja, mis on eredalt meeles, on see, et üheksakümne esimesel aastal oli seal Moskvas sihuke putšilaadne asi, kust vahetati valitsusi ja, ja tankitankid olid siin igal pool nende teletornide ees ja mis iganes ja siis oli jube infoauk oli, et mis toimub, kus toimub, et siis oli, sai sobivate Moskva, venelaste arvutihäkkerite ka kokku lepitud, et teeme mingi otseliini, et paneme mingisugused info listid käima ja, ja see oli jube põnev, et sa said selle toimuva info ja ajakirjanikele kasuliku info emaili teel kätte natuke ennem kui ta. Kuulge kurat teab, kust tekkis selline kontaktid, olid olemas. Need kontaktid olid olemas kuskilt mingist konverentsil käimisest või midagi sihukest. Ta sai, juuniks ei kasuta, et konverentsil käidud, Moskvad, aga Vladimiri siis mingis siukses linnas. Kusjuures, kus oli kohal esinejana peeti ülikoolist. Mäletan Keit postiku ei usu nii. Ühesõnaga, kaks pööki, juuniksi loojat või guru ja selles mõttes ei tasu naerda, et venelased suutsid need sinu enda juurde meelitada, vedada, tõenäoliselt neil oligi huvitav ja see konverents oli megakihvt.
No muidugi seda ma seda ma imestan, et see võis olla väga kõva sõna.
Ja oli ja noh, tagantjärgi,
See tekitas jälle jälle tunda, et see arvuti teema on nagu hea teema, et sa hoiad nagu näppu pulsil, et oled suhteliselt lähedal sellega, mis maailmas toimub. Ägedat.
Ja see äge ägeda toimumise juurest. Sa ikkagi ühel hetkel liik.
Müüsid ära sealt edasist oota veel natuke natuke selles mõttes, seal ülikooli juures tekkis gaase, esimene moment, kus emelist nähti, et see on päris kihvt asi lisaks email on olemas ka mingi sihuke värk nagu püsiühendused ja asjad ja ja üheksakümne teisel aastal oli see moment, kus Jaak Lippmaa pani Tallinnas püsti taldriku rootsi teleyksiga ja tartus teletorni siis püsiühenduse, mis oli kuuskümmend neli kilobitti ja selleks, et see sealt tähetornist toomel kuhugile mujale Calebi leviks, siis Postimehe eestvedamisel sai üle katuste veetud.
Tähetornist toomel kõigepealt keemiahoonesse, sealt sealt ülikooli peahoonesse ja sealt Postimehe majja, siis siis tort, esimene pühi püsiühendus, mis oli jälle vene sõjaväelaste käest viiesaja rubla eest ostetud. Ülejäänud koobi rullist ehitatud.
Ja ja noh, siis tekkis reaalselt see nii-öelda internetti kommuun sinna ümber. Sellepärast et need, kes sinna vahele jäid, said ju ka endale interlinnu absoluutselt. Ja, ja see oli onlain, see oli täiesti Online, et aga Lott prints pidi kõva protsent, latents oli kuskil kuussada millisekundit ja nii hull ei olnudki ja, ja noh, ega see kuuskümmend neli kilobitti täna tänapäeva mõistes ei ole nagu mingisugune superkiirus, aga noh, tollel ajal, kui internet oli veel tühi igasugustest kassipiltidest, siis oli see täitsa okei, kiirus.
Sai sai asju ajada küll. Kes see kogukond seal siis oli?
Kogukond oli, oli Eeennetti inimesed, Enok Sein on Villems, Richard Villems, Tiit Mokuma, Marek Tiits, pall teeuuringute instituudist. Ja tõenäoliselt neid inimesi on veel kelle, keda praegu mälu mälu meelde ei tule kindlasti sellele reageerinud juba olemas, siis ei veel veel ei olnud neid, palun, aga inimesed, seda videol see tuumik, inimesed, millest teen, et tekkis, olid needsamad kes, nagu selle kogukonna moodustasid. Kes nägid vaeva selle nimel, et see interneti püsiühendus oleks olemas ja ja noh, siis kogunes sinna mudeliga sisse helistajaid ja vaikselt hakkas kasvama.
Mõtlesin loogiliselt järeldan, siis käis Tartu ja Tallinna sidesatelliidi.
Jah alguses käis üle satelliiti, mingi aastapäevad, iili tuli ka maapealne püsiühendus. Meil eestvedajaks oli siis küberja, noh, see on see koht, kus ilmselt Eesti nii-öelda selles internetimaailmas
Tekkis kaks nii öelda rististe suuskadega kommuuni üks oli sele Küberi kommuun ja teine oli siis nagu KBFI kommuunides, mitte siis Aktsiaselt küberneetika instituut, Küberneetika Instituut ja aktsiaselts tekkis ilmselt sellele kunagi hiljem veel. Ants Võrk.
Miks need suusad riskid risti läksid? Vaat seda mina ei tea selles mõttes, et tortu inimesena siukseid tollina siukseid probleemidest aru lihtsalt ei saanud. Tõenäoline tõenäoliselt tõenäoliselt oli kui ressursiga kitsas, mida ta Eesti vabariigi alguses oli siis võimalik, et seal olid lihtsalt mingisugust rahade jagamise või teaduste või finantseerimise mured, et kes oli, kes said oskuslikumalt finantseerimisele ligi või või noh, see oli olelusvõistlus, tegelikult tundus mõlemal pool, kuhu tulevad inimesed. Aga no ma arvan, et see olelusvõitlus, mis oli Eesti vabariigi alguses ja jättis kõigile jäise suli
Aga mis edasi sai? Edasi sai see postime, periood sai läbi, siis mingisuguse aastakese töötasin Tartu Ülikooli raamatukogus kus kuu sai rootsi kunni abiga muretsetud esimene serveri asija otsast hakatud kirjutama raamatu infosüsteemi.
Ma olen niisugune pisikene projekt, sinu ajalisi sattusid vanasti, Tartu ülikoolil olid legendaarsed serveri nimed, kalanimedega, Serbia ei tööta. Kas sinu Eestis?
Enam ei mäleta, lihtsalt enam ei mäleta kilu, LIVE juust. Kilu, n-i p oli mingisugune. Ma arvan, et Spark Station kaks.
Aga kus, kus, kusjuures selle põhiline funktsioon oli ikkagi see, et sinna tekkis esimene elektroolil ülikooli raamatukataloog, mida kohapealsed inimesed ise programmeerisid.
Ja selle jaoks on isegi terminali, seda sai kuidagi ülikooli raamatukogudest kasutada. See oli tolle aja kohta, oli väga niukene, funktsionaalne, tore asi.
Ta ei olnud küll see oli jube töö ja, ja, ja, ja noh, eks andmesisestamine kõik nullist tehes on ikka väga raske
Vaja, sest seal on meeletu kogu selle taga. Ja mitte ainult raamatud, vaid vaid muusika ja mingid käsikirjad ja.
Aga kahjuks kahjuks või õnneks tagantjärgi ei oska öelda, see raamatukoguperiood jäi suhteliselt lühikeseks kuna, kuna Jaak Lippmaa kutsus mind Tallinnasse ja see tundus veel põnevam ju kutsustasid, mida tegema. Tegelikult ta kutsus mind sellises riigiasutusse nagu valitsusside. Et umbes sellise mõttega, et kuule, Taavi, sina oled nüüd selle internetiga side ka natuke kokku puutunud, oskad ühest arvutist teise sümboleid saata, et kuule, et et siin mingisugune KGB sidekeskus tuleb Eesti riigil üle üle võtta ja tule aita. Ja noh, siis tulingi.
See kõlab hästi küll. Siis võtsite valitsusside selle üle.
Põhimõtteliselt küll ja noh, ütleme niimoodi, et,
KGB-st võib rääkida mida iganes, aga selle tehnoloogilise poole pealt nägi asi välja ikka suhteliselt õnnetu, et hädala tänava majas olid suured saalid täis mittetöötavaid, telefonijaamu ja selle ütleme ma võin nüüd natukene valetada, aga suurusjärgus kakssada töötavat telefoni oli ja mis oli siis ettenähtud nende riigi riigi kui nähtuse ja jaoks ja nende igasuguste organite jaoks, aga noh, ütleme kakssada töötavatele uni on ikkagi suhteliselt madin number kogu valitsuse peale kogu valitsuse omavalitsusaparaadiga. Ja, ja järgmine projekt seal Siimetsi telefoni jaamatega reaalselt. Mis siis on?
Toompeal Kadriorus, välisministeeriumis, mis, kes nad seal kõik pikal tänaval olid? Triaalse telesisetelefonijaamade võrgu käima panemine kokku mingi paar tuhat numbrit mis tuli õnneks, tuli täitsa edukalt välja ja, ja oli reaalselt ka mingisugune kasu olemas, see oli oroloogioon veel, eks ole, ei ole, see, see on digijaam, simestop, komandigi jaam võrgustatav ja kõik ja puha. See oli päris, see oli päris sidevõrk.
Äge, aga kust see visioon tuli, et siukest asja ehitada või see pidi kallis ka olema. Et oleks odavam olnud, võta lihtsalt analoogjaamad üritada kuidagi hakkama saada, aga tehti investeerinud.
Tehti investeering, kus investeeringu lükke tuli kasse tuli jaagu isiklikust initsiatiivist. Mida ta siis tehnoloogilise poole pealt konsulteeris, Paavo Picofiga, kes oli Tallinna telefonivõrgus seal tõenäoliselt sinna taha, tulid tänu jaagu ja Endel Lippmaa tutvustele ka riigi riigi funktsioonid. Et ainult seda on tõepoolest vaja ja, ja, ja tisse, visioon osteti ära ja finantseeriti ära. Bioriaalse oli reaalselt reaalselt töötavatel sisetelefonivõrk, aga, aga noh, eks ta mingi aja pärast jäi ajale jalgu.
Roll on kindlasti füüsiline kaabeldus oli Eesti Telefon.
Füüsi füüsi füüsiline kaabeldus oli, osa oli Eesti telefoni käest ja osa oli seda nii-öelda vana KGB kaablivõrku, mida oli tegelikult Tallinnas mõnisada kilomeetrit täitsa.
Aasias, nojah, keskjaame võis, võis katki olla, eks ole, aga kaabelkaardil kaablil.
Olid olid olemas ja olid siuksed korraliku tinakestaga kaablid, mis oli noh, umbes nagu tuumasõja üleelamiseks ettenähtud ja meenutasid rohkem tanki kui kaablit.
Tulidki tuumasõja üleelamiseks ette nähtud ka praegu sihukese projektina, mille käigus kohtumine huvitavate inimestega
Absoluutselt et telefonijaama installeerimine ja käima panemine Kadrioru lossis, kus Lennart Meri vaatab sult üle õla ja õpetab, kuidas telefoni ühe pistikuid ühendada, noh see võib tagantjärgi tarkusena muigama panna, aga noh, see oli niimoodi seal Lennarti moodi absoluutselt, et ta oskas kõike kõikides asjades nõu anda. Et vaadake, poisid, et te teete nii või naa.
Ja kaua sa neid kauneid vedasid seal valitsusse?
Valitsussides ma vedasin kaableid kukuks kolm aastat ja, ja ütleme peale seda, seda esimest edukogemust telefonijaamadega noh, loomulikult me tahtsime seal järgmisi siukseid edukogemusi veel, et see interneti-nimeline asi kogu aeg ronis uksest ja aknast sisse. Kirjutasime järgmiseid pabereid, et nüüd oleks mõttekas selle jaoks kuidagi investeerida. Siia-sinna ma ei võtnud väga vedu. Kuigi, kuigi see hetk juba ükskõik kelle käest, kes natukenegi internetiga tegeles nuutiat kuule, mina tahan ka inimesi tuli uksest ja aknast sisse, et kuulge, tehke midagi, noh, mul läks internetti vaja. Aga, aga.
Paberi- kirjutamin, paberite kirjutamine ei olnud edukas ja, ja siis.
Hakkasin kõikide oma tuttavate juurest rääkige läbi sõitma, et kuulge, niimoodi asi ei toimi, et interneti kõik tahavad, võiks ju teha, aga välja ei tule, et kuulge võtaks pundi kokku, hakkaks nagu normaalselt ise tegema.
Harva, et ma käisin, kõik inimesed, kes vähegi internetiga tegelesid läbi, et kuule, et võtaks pundi kokku, teeks mingisugune ettevõte, mis teeks, teeks midagi ja no ütleme, reaalsusena tartlased vedu ei võtnud, noh, Tartus on nii mugav olla ülikooli juures, et Pirogovi ja kõik kõik Pirogovi ja kõik asjad. Kuigi puht praktilised manectiits aitas päris palju ideed formuleerida ja nii edasi. Ja reaalsusena võttis kõige rohkem vedu KBFIs tundmis Bauman. Ja noh, see siis Andresega koos me tegimegi siukse internetiettevõte käisin poes, ostsin riiulifirma, mille nimi oli Nesper. Mis on, mis on praegu siis praeguseks siis Elisa seesamasugune riiulifirmad peavad vastu ostetud riiulifirma registreerimiskoodi, Elise registreerimiskood on samad.
Jätkame juriidilist järjepidevust, täiesti juriidiline järjepidevus olemas.
Mis aastal see oli? See oli aastal üheksakümmend viis, üheksakümmend neli, üheksakümmend viis algus.
See oli ikkagi üsna niukene, karm aeg, sul ei olnud üldse mitte midagi saada ja kõik asjad tuli nagu nullist ehitada.
No oli küll, aga, aga noh, teisalt inimesed olid ka siis leidlikud, seal tato Telekom, Neeme takis eestvedamisel ehita siis modemeid verest neli, kaks kohe elektrilise ühenduse peal. Kui ei olnud, siis tead ise.
Kes kohandas mingeid asju? Jaa, jaa. Midagi oli võimalik igal juhul teha.
Just nimelt, et ma sind kuulan, siis oligi nagu võimalik, nagu ise teha just peamiselt midagi saada ei ole, sa pead ise tegema.
No ütle, ütleme mingisugused,
Lisaboonused olid ka, et kui tänapäeval inimesed arvavad, et mis ta siis tuleb, et mingi kümme eurot kuus peaks tulema nii ja me intervjuu, kui maailmas üldse olemas on, ehk mis ta siis kolm kohvitassi kuus siis too aeg oli internet, nii nagu kui vastu öelda seksikas või mis iganes, või uus või edev, et siis oldi veel interneti eest nõus maksma. Halo sellised asjad, et ettevõte ostab ise seadmat välja, maksab kinni paigalduse ja siis hakkab kuutasuga veel maksma. Niisugune see oli sihuke helge aeg ja ütleme tänu tänu sellele oli võimalik need esimesed seadme investeeringut teha. Sest et samasuguse mudeliga nagu täna on, et kõik on kõik on kümme või viis kohvitassi kuus. Sellise mudeli ka interneti ei oleks Eestisse jõudnud mitte kunagi.
Ei saa hakkama, siis kui ma õieti mäletan, siis mina hakkasin toimetama teie edasimüüjana avastan mingi üheksakümmend kuus võib olla. Mis tõenäoliselt üsna varakult, teil oli ikkagi üsna niuke laivõrk, mingid juba nagu teenuseid ja mingid all.
Olla põhi põhipõhiline, mis siis oli õli sissehelistamine, ma arvan küll. Et ei usu, et seal. Ma ei mäleta, mis, mis süsteemiga SAIS füüsile, modemi puul telefoninumbreid ja, ja, ja sellesse tegelikult pihta hakkas. Tõeline, tõenäoliselt telefoninumbritega aitas. Kuna need olid Kadevko, aitas Jaak Pippa isa Endel Lippmaa, et, et noh, saaks kuidagi seal mingi sobiva järjekorra mingisuguse sobivasse punkti. On lõpuks need said.
Ja sealt sealt hakkas internetiäri vaikselt kasvama, kuna kuna inimesed tahtsid huvi, huvi, kerg, kes siis üheksakümne seitsmendal aastal või üheksa, ma arvan, et üheksakümne seitsmenda aasta alguses oli see koht, kus kus nii-öelda üldkasutatav interneti välisühendus oli igasuguste puukide poolt, nagu oli meie puukfirma seal köiest nii täis aetud, et teen, et tegi otsuse, et nii nüüd ärikasutajat, muretsege endale oma välisühendus.
Seegi hilja.
Selles mõttes, et üsna kaua saite toimetada nagu akadeemilise kraadiga.
Pea topelt täpselt samamoodi. Noh, ta ei olnud veel päris akadeemilisest maailmast välja pääsenud, see internet ja eks me, eks seal kõik toimetasid akadeemilise traadi peal. Jaa, jaa ja senimaani kolmi Eeennet jalga maha ei pannud, senise osas oli see täiesti täiesti loomulik nähtus. Uuesti siis sõitis Soome kuhugi Soome kuhugile. Mul oli mingist ajast jäänud kontakt Helsingi telefonivõrguga ja Helsingi telefonivõrgu inimeste ka koos sai siis nii-öelda kanal tellitud ühendus hangitud seadmed hangitud ja, ja meil oli juba siis sedavõrd palju käivet, et me enam-vähem suutsime isegi selle väliskanali kuutasu kinni maksta. Mis oli siuke soliidsed pea sada tuhat krooni kuus. Oioi oi Juhk summa tulla, see oli väga jõhker summa ja, ja noh, selles mõttes noh, kogu see internetitehnoloogia või ruuteri, see kogu see värk maksis ikka suhteliselt hingehinda, et niuke Kuue Wordiga Cisco ruuter marki enam ei mäleta, see maksis mingi mingi kakssada viiskümmend tuhat ehk ehk siukse noh.
Ma ei tead S-klassi Mersu hinna õudu, aga sellest ma järeldan, et mingi füüsiline kaabel oli olemas Eesti ja Eesti füüsi füüsi. Füüsiline kaabel oli olemas. Peale seda, kui see Telekom sai kontsessioonilepingu Eesti riigilt suhteliselt kiiresti, vedas Eesti Telekom ka kaks valguskaablit. Eesti-Soome vahele. Ma ei mäleta, mis aastal see oli, aga ma arvan, et see oli ka kuskil üheksakümmend seitse, üheksakümmend q. Päris Voro. Selles mõttes, et tellijal oli see kogemus olemas ja investeerimisvajadus kogu kogu selle analoogvõrgu kõige väljavahetamiseks oli väga selgelt olemas. Lihtsalt investeeriti ja Telial tegelikult investeerimisjõudu vali. Te tegite mingisugust hostingut, kao mingeid servereid ja sai servereid pidada, et ei peaks seda traadi raha maksma. Mis, mis oli, ma arvan, et täiesti normaalne teema.
Kuidas see käis selles mõttes, et kui ma praegu ostan mingisuguse virtuka endale, siis tol ajal ma sain lihtsalt minge kaundia parooli kuskile.
Siis said juuniks Jacoundi endale ja oligi kõik.
Ja mäletan, kuidas sai kuidas veebiprogrammeerimine käis niimoodi, et mul oli Windowsi masin siis ma tegin Berli skripti naalsusele FTP-ga ülesse siis ei töötanud siis ma kellelegi teie süstlad, Minni käest sain väljundi, mis logidesse kirjutati, see saadeti mulle kuidagi. Siis ma vaatasin, ahah, näe, Perli skript ei tööta.
Parandasin ära, saatsin uue versiooni. Vaat tunnistan ausalt seda, seda, seda poolt arendusprotsessist ma ei näinud, kuna, kuna ta olles pumba juures sa ise vaadata koev kas töötab või ei tööta.
Ja ilmselt ilmselt oled selle võrra ka mõnevõrra ajurakkusid säilitama. Et see läks üsna ruttu, enne, kui te teie admin mind välja viskas, selle sellest protsessist tuli mind või seal peal käima ajada. See selleks, et see on, see oli üks legendaarne ettevõtmine, legendaarne aega, aga mis sa praegu teed?
Praegu olen sellises huvitavas ettevõttes nagu rebel rõõm ja meie tegeleme transpordiettevõtetele wifi teenuste pakkumisega ja nende optimeerimiseks. Selles mõttes kliendibaasiks on mööda Euroopat, Ameerikat Inglismaalt sõitvad bussid kus siis on bussis wifi, mida saavad bussireisijad kasutada, või jõelaevad kus siis kruiisilaevad, kus pensionärid ostavad mingi noh, kolme nelja tuhande dollarise nädalase reisipaketi ja kui nad Pariisis Eiffeli torni pildistavad, siis nad peavad õhtul saat saama saata oma selle pildi lastelastele, et muidu on see reisikogemus natukene nõrk, puudulikuks. Ja siis teie teete nii, et Ifi turni Pilsood, Viiust.
Et selles selles mõttes, see võib tunduda imelikuna eestlasele, et meil on kogu aeg internet igal pool vabalt kättesaadav, aga näidetena ei, Prantsusmaal, Ameerikas, Inglismaal ei ole, see ei neligee viiske nii levinud, et see kvaliteet oleks kõigile piisav ja seal on see, ütleme teenuse optimeerimise vajadus täitsa olemas.
Mis tähendab, et need kõik need inimesed, kes Ameerika kiirteel kreondi bussi järel sõidavad, et saaks vihvikus.
Siis nad tegelikult kosutavat teed jokk Krihhound on meie klient. Ma ei tea ka, kas kõikides bussiliinides või, või oli, oli seal rohkem selle lääneranniku pool ma lisada lihtsalt ei tea. See laagri koondan, tuttav nimi küll, et on kliendi läbi jooksnud ka tehnomeeste poole pealt.
Ega sellest sellest ei tahagi rääkida, ma vaatan, et et see ring on jällegi kuidagi nagu tavapäraselt tundub nagu täis saanud. Et kui sa alguses seal edasi ütlesid, et inimestel nagu silm särab, siis siis praegu kaaslased, sa tõid nagu esimese näite selle pensionäri oma Eiffeli torniga, eks. Et lõpuks on ikka inimeste rõõmsaks tegemine absoluutselt läbiv joon, eks.
Ja, ja noh kui sa teda rõõmsaks ei tee ja, ja sinu teenust nagu viha koostadega ta raha sulle ka ei maksa, kui teed rõõmsaks, siis saad tõenäoliselt mingisugune raha enda, panga palga kontoleb sobivatel päevadel ja on ja on rõõm endal ka midagi paremaks teha, uut ja, ja noh, nii ta on koodi veel kirjutatud. Võrku konfilise võrku ei konfise maa, ma arvan, et ma korralikult enam ei oska seda teha, et see võrk on tänapäeval ikkagi niivõrd Äraviltaliseeritud juba. Ja et enam ei oskaks seda konfigureerida.
Mina küll märkasin Facebookis, kuidas keegi küsis ja siis sa mitte ei vastanud, et kuidas saab teha, vaid vastasid if konfigi käsureaga, mis võtmest töötas.
Et ei, noh, selles mõttes seal oma arvuti konfimine mitte ei ole selle võrgu konf võrgu konfrimise all, ma mõtlesin ikkagi seda, et päris võrku, et kus on sul mingisugused jämedad ruuterit, kus vilkuvad sinised LEDid? Jaa, jaa.
Pool Eesti interneti trehvikust, noh see oleks nagu võrgu konfimine, silki.
Lisa annab, tõmbab päris hästi meie jutule joone alla, ma arvan, et,
Selline mastaap on, on päris päris värskendav. Ega ma siin selle koha peal ei oskagi muud soovida, kui et need leed ikka vilguks.
Inimesed naerataks mõne, mõlemad teevad vähemalt miinimum minul tuju heaks, kui LEDid vilguvad nii, nagu ma tahan ja inimeste inimesed on rahul, siis on jumala super ju väga hästi, aitäh sulle.
