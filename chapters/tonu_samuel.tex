\index[ppl]{Samuel, Tõnu}
\label{sisu:tonu}

\question{Kuidas jõudsid arvutid sinu juurde ja sina arvutite juurde?}

Arvutite juurde sain ma kolmeteistaastaselt, aastal 1985, kui mul tekkis 
esimest korda ligipääs ühele programmeeritavale 
taskuarvutile. Aga see oli ikka mäekõrguselt rohkem kui 
MK-51\sidenote{Elektronika MK-51 oli alates 1982. aastast 
Zelenogradis, Billuris ja Rodonis toodetud Nõukogude 
taskuarvuti, mis oli modelleeritud Casio FX-2500 alusel.}, mis oli tollal 
tavalisel koolijütsil unistuste arvuti. 

\question{MK-51 oli ju Nõukogude kalkulaator?}

MK-51 oli jah Nõukogude kalkulaator, mis oskas trigonomeetriat, aga mina sain 
ligi välismaa Casio PB-100-le\sidenote{Casio üks esimesi ja 
lihtsamaid samme taskuarvuti juurest päris arvuti poole. See toodi PB-100 nime
all turule 1982. aastal ja ka 1983. aastal kui TRS-80 PC-4 (Tandy Radio 
Shack) ja OP-544 (Olympia). Tegu oli QWERTY klaviatuuriga päriselt 
programmeeritava arvutiga, kuigi üherealine ekraan tegi Casio 
BASICus\index{BASIC!Casio BASIC} programmeerimise küllaltki vaevarikkaks.}. 

\question{See on ju klassika!}

Mõnes mõttes jah. See on taskuarvuti, mälu on pool kilobaiti, mis 
tänapäeval tundub ulmeliselt vähe, ja oskab ainult BASICut. Aga tollal võttis mul 
silme eest kirjuks, sest klaviatuuril oli terve tähestik peal. 
Tundus ikka täielik kosmos. 

\question{Kust niisugune asi õndsasse Nõukogude vabariiki sattus?}

Nägin seda kõigepealt kuskil komisjonipoes 
ja kuna sellel kalkulaatoril olid tähed, siis jättis see mulle
nii sügava mulje, et rääkisin sellest igale sõbrale. Paar 
päeva hiljem tuli üks tuttav, kes juhtus olema suhteliselt rahakast perest, sellesama asjaga mulle nina alla liputama.
Siis sain seda natukene veel näppida. Läks mööda nädal või kuu ja juhtus selline ime, mis juhtub ainult filmides. Igatahes NSV Liidu 
ajal oli see täpselt sihukest laadi sündmus. Tuli info, et mul on välismaal, Šveitsis 
rikas vanatädi. Selle \enquote{rikas} mõtlesin sinna juurde vist ise, sest tollal 
tundus kõik rikas, mis oli välismaal. Juhuslikult saatis ta meile veel kirja ka, 
et äkki tahate midagi siit. Ja mina teadsin täpselt, et tahan 
samasugust asja, ja kirjutasin sellest talle kohe.

Ma olin kolmteist ja tagantjärele mõeldes tundsin ennast vist
väga täiskasvanuna, aga kui praegu vaatan kolmeteistaastaseid, siis saan 
aru küll, miks ma sellise kirja kirjutasin. Väga lakooniliselt, et kui sa juba 
küsisid, siis üht arvutit oleks mulle vaja küll. Ja tuligi, küll 
pool aastat hiljem. Tollal käis niimoodi, et kiri läks siit Šveitsi kaks kuud ja 
kaks kuud hiljem tuli vastus, et vabanda, ma oleksin sulle hea meelega saatnud, aga sa unustasid tüübi kirjutada. Siis 
ma kirjutasin jälle, ootasin veel mitu kuud ja ühel hetkel tuli 
tollist kiri, et saatke viiskümmend rubla tollimaksuks. See oli tollal 
kosmiliselt suur raha,\sidenote[][-2mm]{Pudel viina maksis 10 rubla, pudel {\v Z}iguli õlut aga 33 kopikat.} aga
ma kuskilt laenasin selle. 

\question{Tollimaks tol ajal?}

Ma olin väga vaesest perest, viiskümmend rubla oli minu jaoks 
ulmeliselt suur raha. Ja kui lõpuks nägin tolli hinnakirja, tuli välja, et 
välismaine kalkulaator on viiskümmend rubla. Aga selle karbile oli kirjutatud 
\emph{personal computer} ja selle puhul oleks maks olnud vist 
kolmsada või kolm tuhat, mida ma kohe 
kindlasti ei oleks suutnud kuskilt leida. 

Sellest ajast alates hakkas asi minema. Kui olin viisteist, õnnestus mul esimest 
korda Normasse tööle saada. Ja kui olin kuusteist, siis selleks ajaks 
olin täiesti veendunud, et tahan arvutitega tööd teha. Haridust mul ei 
olnud, teadsin, et keegi mind tööle ei võta, aga oli unistus, et 
lähen kuhugi arvutuskeskusse, kus on tohutu suur arvuti, koristajaks.

\question{Mida sa selle personaalse kompuutriga tegid?}

Näiteks lahendasin ruutvõrrandit koolis. Kolm numbrit sisse, kaks välja, 
väga kiiresti. Teised kõik tagusid oma kalkulaatorit, aga vigaselt ja said tihti 
valesid vastuseid. Seitsmes klass oli selles suhtes imelik, et
tegime matemaatikas vist pool aastat ruutvõrrandeid. See viskas kõigil
väga ära ja mina olin ainuke, kes kunagi ei eksinud, sest mul oli 
kolm numbrit sisse, kaks välja. Kõige naljakam oli see, et 
matemaatikaõpetaja lõpetas kodus asjade lahendamise, sest tunnis käis 
nii, et keegi pidi püsti tõusma ja oma vastused ette lugema ning 
pärast vaatas õpetaja küsivalt minu poole: kui noogutasin, 
siis järelikult oli õige. Suutsin enam-vähem reaalajas neid numbreid 
toksida, sel ajal kui teine vastuseid luges. 

\question{Õpetaja tegi ju ka käsitsi, temal ei olnud sellist 
arvutit!}

Ei olnud jah, ta pidi selle nimel tööd tegema ja eksis samuti. Ükskord sain õpetaja vea kätte ja pärast seda ta vist 
loobuski. 

Lisaks trollisin inglise keele õpetajat. Küsisin talt, kas võin eksamil arvutit kasutada. 
Vanemapoolne daam oli tõsiselt hämmastunud ja ütles, et kui 
sul sest kasu on, eks kasuta. Ja siis käis laua alt klaviatuur mürtsuga laua 
peale, terve sõnastik sees. Seejärel oli tal väga piinlik öelda, et ikkagi ei tohi. 

\question{Mis sind arvuti juures paelus? Kas
klaviatuur või et sai programmeerida?}

Midagi seal oli. Olin
selline laps, kes absoluutselt kõik asjad ära lõhkus. Näiteks kodus oli raadio 
ja minule ei mahtunud pähe, kuidas selle kasti sees saab olla inimene. 
Selge see, et inimest seal ei ole, aga sa ju kuuled, et midagi on. Lammutasin kõlari ära ja sain selle eest 
peaaegu peksa. Meil oli suur lampraadio ja ma uuristasin 
ennast seal ees olevast võrgust läbi, et teada saada. Raadio oli tollal asi, 
mis osteti üks kord elus. Kui selle ära lõhkusid, siis oli pahandust
palju. Kõik äratuskellad võtsin ka lahti, aga kokku panna enam ei osanud.

Oma taskuarvuti lõhkusin ka lõpuks ära. Üritasin seda lahti võtta, aga 
see oli kleepsuga kokku pandud. Venitasin seda lahti, 
kleeps muudkui venis ja venis ning ma kasutasin kääre, et natukene 
kaasa aidata, aga seal oli üks lintkaabel sees, mida ma ei märganud, ja suutsin 
selle ka läbi hammustada.

Nii et mul oli jah meeletu huvi, kuidas asjad töötavad. Mul on üks väga oluline mentor ka elus olnud, naabrimees, kes viitsis 
mulle seletada. Lapsest saati olen igasugu asju ehitanud, 
väikseid raadiosaatjad ja muid vidinaid. Ühelt poolt olid sel naabrimehel 
närvid läbi, aga teiselt poolt viitsis ta aeg-ajalt õpetada. 

\question{Kas sa oled Tallinna poiss?}

Jah, ma olen kogu aeg Tallinnas olnud. 

\question{Kas juppe ja pudinaid, millest raadiosaatjaid teha, siis ikka 
liikus?}

Tegelikult oli väga raske. Tavaliselt käis asi nii, et sain kuskilt 
mingi skeemi, aga 
kui jooksin sellega poodi, et nüüd hakkan ehitama, siis selgus, et 
kõige põhilisemat asja ei ole. Näiteks käis tollal 
raadioajakirjast\sidenote{Tõenäoliselt peab Tõnu silmas ka mujal jutuks olnud 
ajakirja \begin{russian}Радио\end{russian}.\index{Radio}} läbi arvutiskeem ja mõtlesin, et nüüd teen ise arvuti. Jooksin ringi, et asju saada, ja selgus, et mikroskeeme lihtsalt ei ole. 
Videokontroller, CPU ja muu oli defitsiit, ma ei omanud ühtegi kanalit, kust 
midagi saab. Tagantjärele tean, et tuttavad, kelle vanemad töötasid 
sõjatööstuses, suutsid kõike hankida. Ühe tuttava vanemad olid näiteks 
keemikud ja kui tahtsime poisikestena plahvatavaid asju teha, 
siis käisime nende juures ükshaaval aineid pinnimas, 
varjates muidugi nende tegelikku otstarvet. Tagantjärele mõeldes 
pidid nad muidugi aru saama, 
milleks me näiteks salpeetrit vajame. Või siis oli tegu nii tarkade 
inimestega, et teadsid veel sadat otstarvet ega suutnud seal vahel enam 
ohtu näha. 

\question{Või mõtlesid, et kui poiss oskab juba salpeetrit küsida, siis vast 
näppe päris küljest ei lase?}

Vot ei tea. Meil keemiaõpetaja läks ükskord väga närvi sellepärast, et 
hakkasime nitroglütseriini valmistama.\sidenote{1956. 
aastal ilmus Eesti Riikliku Kirjastuse sarjas \enquote{Seiklusjutte maalt ja 
merelt} Jules Verne'i \enquote{Saladuslik saar} ja seal oli juttu nii 
nitroglütseriini plahvatusjõust kui ka selle valmistamisprotsessi 
detailidest.} Keemiaõpetaja veel seletas meile 
innustunult, et see kodustes tingimustes ei õnnestu. Tagantjärele
saan aru, et seda ta täpselt üritaski öelda, et ärge tehke, 
sest teisiti ei õnnestunud meid veenda. 

\question{Nüüd on ju teistpidi. Kui keegi läheb eetrisse ja teatab, et nende 
süsteemid ei ole häkitavad, siis kohe palju inimesi proovib!}

Tavaliselt käib see veel sellise ülbusenoodiga, et meie oleme 
paremad kui nemad. Ja kui väidad end kellestki parem olevat, siis 
enamasti see keegi solvub. Igatahes tavaliselt on mingi motivaator, miks 
selline väide toob sellise tulemuse. 

\question{Kust sul see arvutuskeskuse mõte tuli?}

Ma ei mäleta, kust ma selle info sain, et selline asi üldse olemas on. Vist 
üks tuttav käis kuskil arvutuskeskuses mingi onu juures, mul õnnestus kord ennast 
sinna kaasa nihverdada ja see tundus nii põnev. Tuba oli arvutit täis. Igatahes 
teadsin, et ma tahan arvutuskeskusesse, ja kuna ma midagi ei osanud, siis 
käisin mööda uksetaguseid. Tagantjärele saan aru, et ääretult naiivne 
oli käia viieteistaastaselt, ilma igasuguse 
hariduseta, uksi kulutamas ja mõelda, et lähen sinna tööle. Aga võeti tööle. Olin siis kuusteist.

\question{Kellena sa seal tööd said?}

Arvutioperaatorina. 

\question{Kas keskkool jäi sul tegemata?}

Üheksanda, tollal kaheksanda klassi tegin lõpuks hiljem ära, aga edasi ei 
teinud midagi. 

\question{Mida arvutioperaatori töö endast kujutas?}

Oi, see oli väga ülbe töö --- mul oli lubatud isiklikult 
arvuti käima panna ja klahve vajutada. See ületas mu ootusi tugevalt, olin sel 
hetkel endaga tõsiselt rahul. 

Algne ülesanne oli mingit teksti sisse panna. Vene keelt oskasin ma hästi ja 
üks osa keskuse tööst oli tarkvara tegemine Venemaa 
asutustele. Meil oli asutusetäis tädisid --- tollal oli mul tunne, et täiesti 
surmalähedased vanamammid, sellised neljakümnesed --- ja 
nemad programmeerisid midagi, millest me (paar 
poissi tuli veel) väga aru ei saanud. Kui nad kirjutasid kasutajajuhendi käsitsi paberile, siis meie pidime selle 
arvutisse sisestama ja välja printima. See tõi paratamatult õiguse ka
printerit näppida.

\question{Mis süsteemi too arvutuskeskus kuulus?}

Sideministeeriumi alla. Asutusel oli edev nimi 
Sideministeeriumi Info- ja Arvutuskeskus\index{Sideministeeriumi Info- ja 
Arvutuskeskus}, mis oli Siseministeeriumi endaga äravahetamiseni sarnane. 
Töötõendil olid punased kaaned, mis tegid nii mõnegi vajaliku 
töö ära, kui oli vaja midagi kuskilt läbi suruda. Lõid oma kaaned lauale (ma ei mäleta, kas Lenin oli ka peal)
ja jäi mulje, nagu töötaksid Siseministeeriumis. 

\question{Sina sisestasid tekste, aga mida need programmid tegid?}

Palgaarvestust ja põhivara 
arvestust, lõpupoole pidin neid ise tegema. Mul oli tollal poisikesena väga raske aru saada, mis asi on 
põhivara ja mis väikevara, ja siis üks mammi seletas, et sina 
istud väikevara peal, aga mina istun põhivara peal. Kuskil viiskümmend 
rubla oli see piir ja temal oli viiekümne viie rublane tool. 

Mäletan, et kirjutasin ka postitoodangu arvestuse programmi. Näiteks löödi sisse, et kirja kohaletassimine kellelegi koju on 0,01 kopikat, ja 
postkontor arvutas niimoodi oma toodangumahtu. 

\question{Nii et põhimõtteliselt jooksis seal arvutuskeskuses kellegi ERP?}

Jah, midagi niisugust, kuigi tollal olid kõik need sõnad võõrad. Mõnes mõttes oli see muidugi väga nõme, mis tollal sai tehtud --- tänapäeva mõistes väga lihtsaid asju. 

\question{Tekstide toksimine võis olla üsna nüri tegevus. Kas see huvi ära 
ei võtnud või said nende masinatega omi asju ka teha?}

Mul oli huvi nii suur, et ma istusin seal nii kaua, kui üldse kannatas olla. Sain venekeelse tekstiga hästi hakkama, harjusin vene keele klaviatuuriga ära. Eesmärk oli saada töö kaelast ära, et teha enda jaoks huvitavaid asju. Kui ma sinna tööle läksin, siis ei osanud ma midagi. Pool aastat hiljem teadsin kõikidest tädidest rohkem. Ja kuna poisse oli peale minu veel neli, siis hakkasime omavahel infot jagama. Tollal internetti ei olnud, mitte midagi ei olnud kuskilt võtta, nii et enamik asju käis kuulujuttude põhjal ja katsetamise teel. 

Me õppisime näiteks residentseid programme kirjutama. Tänapäeval võiks seda 
isegi viiruseks nimetada. See oli tollal väga \emph{high tech}\ldots

\question{Kuidas te sellise asja välja uurisite? Teil olid ju Nõukogude arvutid.}

Meil oli erinevaid. Arvutuskeskuses oli kaks suurt arvutit: ES-1022\index{ES EVM!ES-1022} ja ES-1045\index{ES EVM!ES-1045}. Nende jaoks oli terve korrus. Mälu oli kuus megabaiti --- ferriitmälu, bitid olid ükshaaval 
silmaga näha. Aeg-ajalt läks mõni bitt tuksi ja insenerid parandasid neid. 
Kokku oli 24 inseneri, kes seda kõike lappima pidid. Ja kuna mälu 
pidi kord nädalas piiritusega puhastama, siis firmapeod möödusid ilma muu 
alkoholita.

Aga meil oli teine osakond, kõigil personaalarvutid --- 
kaheksabitised Robotron 1715d\index{Robotron!Robotron 1715}. Siis tulid 
Iskra 1030d\index{Iskra!Iskra 1030}, mis olid PC kloonid, kohutavalt halvad 
arvutid. Kuskilt tuli üks DVK-2\index{DVK!DVK-2}, mis oli IBMi kloon. 

Igatahes asi muudkui arenes ja näiteks DVK-2 on naljaka arvutina 
meelde jäänud. Nimelt oli arvutuskeskus suhteliselt külm maja --- Endla 16, 
tänapäeval Eesti Telefoni maja\sidenote{Selles majas asus tõesti kunagi Eesti 
Telekom, praeguseks on maja põhjalikult renoveeritud.} ---, mis ei pidanud sooja. Hommikul oli tubades viisteist kraadi või 
vähemgi ja arvuti ei läinud selle külmaga käima. Flopidraivi rihmad olid 
jäigad, mootor käis rihma sees ringi, aga flopi ringi ei läinud. 
DVK-l oli ees flopi jaoks nii suur auk, et sinna mahtus käsi sisse. 
Pistsid käe sisse ja tõmbasid kogu selle asja ringi käima, et see saaks piisava hoo, 
ja lükkasid flopi ruttu järele ning luugi kinni. Kui seda piisavalt kärmelt 
teha, sai arvuti käima. Hiljem, päeva peale, ei olnud enam probleemi. 

\question{Kust te ikkagi infot saite? Manuaalidest?}

Manuaalid olid täiesti kasutud. Kui Iskrad\index{Iskra!Iskra 1030} 
tulid, siis me veel eraldi mõnitasime neid, sest peale oli kirjutatud 
\begin{russian}электронная персональная вычислительная профессиональная 
машина\end{russian} --- personaalne professionaalne elektrooniline arvutuslik 
masin. 

\question{Kuidas te siis oskasite? Ei ole ju nii, et paned aga suvalisi käske 
ja äkki jääb programm residentseks!}

Tollal, kui mina arvutuskeskuses olin, seal võrku ei 
olnud. Enamik asju toimus nii, et keegi käis teises 
arvutuskeskuses külas, kus keegi oli kuskilt midagi välja nuuskinud, kas siis välismaalt kuulnud või mujal käinud, aga tavaliselt liikus info koos 
inimesega. Keegi käis kuskil ja õppis seal, kuidas teha 
mingit asja, mida me olime juba pool aastat mõelnud. Kusjuures probleem oli 
tavaliselt väike. Näiteks meil ei olnud printeril täpitähti ja siis tuli keegi Tervishoiuministeeriumi arvutuskeskusest 
väga kavala trikiga, kuidas \emph{map}'ida klaviatuuril kandilised sulud 
täpitähtedeks.

Teine häda oli see, et klaviatuurid olid aeglased. Meil oli tihti vaja teha
mingeid jooni niimoodi, et pidime hästi palju miinuseid panema. Hoidsid seda miinust 
nagu ma ei tea mida, aga see ei jooksnud. Keegi õppis seda 
kiirendama, selleks tuli kuhugi porti kirjutada mingi number. Aga selline info 
liikus ainult suust suhu. Assemblerit\index{Assembler} ja muid sääraseid asju oskasime juba 
kõik ise teha, aga muu oli müstika. Mingid pordid ja mida sinna 
kirjutada oli dokumenteerimata. Kuskilt aga tuli info, et kui kirjutad porti, läheb see klaviatuurile ja klaver on kiirem. 

\question{Kas te seda folkloori kuidagi üles ka kirjutasite või jäi see lihtsalt 
inimeste pähe?}

Ei, see oli täiesti suuline. Seda ei olnud tollal nii palju, et oleks tulnud üldse pähe üles 
kirjutada. Pealegi oli see kõik selline info, mida sa niisama ära ei 
andnud, sest see tõstis sind teistest kõrgemale. Ega me 
tädidele ei rääkinud, kuidas residentseid programme kirjutada. Esiteks
pidasime neid madalamaks kassiks, kes nagunii aru ei saaks. Teiseks andis see 
meile võimaluse teha arvutis mida iganes, ilma et nad oleksid aru
saanud. Oleks tollal tahtnud mingit kräkki jooksutada, küll me oleks seda siis
jooksutanud \emph{background}'is.

\question{Kas te seda infot teiste arvutuskeskuste tüüpidega 
jagasite?}

Tavaliselt käis see jah vorst vorsti vastu. Sul pidi olema mingi mõju 
inimese üle. Tavaliselt professionaalid hindavad üksteist ja jagavad sellist 
infot, aga kui tuleb mingi jobu küsima, siis ega sa talle ei ütle. Sina oled palju 
vaeva näinud, et asi ära lahendada --- ringi jooksnud, küsinud ja mõelnud ---, ja 
sa ei anna seda infot niisama ära. Vahest antakse, aga see oli osa 
käitumismustrist. 

Fidonet oli tõenäoliselt üks esimesi arvutivõrke, millest ma kuulsin ja kuskil 
nägin. Sellest võis rääkida Tarmo Mamers\index[ppl]{Mamers, Tarmo}, aga ma ei ole kindel. Igatahes hakkas see selliseks asjaks muutuma, et 
küsisin töö juures modemit ja võrku. Meie 
arvutuskeskus ei pidanud seda kuidagi vajalikuks ja ma ei osanud kuhugi 
õigesse kohta vajutada ka. Kord vist insenerid kuskilt tagatoast 
pakkusid üht modemit, mis oli kingakarbisuurune ja mida ei saanud telefoniliini otsa panna. Oli mingisugune 
\emph{leased line} modem, mis ei osanud helistada. Igatahes 
minu jaoks oli see täiesti tarbetu, sest mul oli vaja modemit, mis käib 
telefoniliini külge. Telefoniliin oli ka tol ajal väga suur ressurss. Meil oli 
asutuse peale piiratud arv telefone, kusjuures me olime Sideministeeriumi 
arvutuskeskus! Jaam oli meie enda majas, aga meie osakonnas oli kakskümmend 
inimest ja kaks numbrit. 

\question{Küsin inseneride kohta. Kui mõtlen, kas lasta noor inimene 
tarkvara või riistvara juurde, siis mina julgeksin teda pigem riistvara kallale lubada. 
Miks sind tarkvara juurde lasti?}

Riistvara ei olnud meil tollal midagi erilist.

\question{Näiteks oleks pandud piiritusega ferriitrõngaid 
nühkima?}

Vastus, miks ma sinna tööle sain, on mul hiljem 
tulnud. Tollal uskusin, et jätsin tõsise mulje, nagu oleksin oma 
taskuarvutiga mingeid programme teinud. Tagantjärele saan aru, et tegelikult oli tollal probleem selles, et IT eriala ei olnud üldse 
populaarne. Keegi ei tahtnud seal töötada, palgadki olid vist madalad. Tõsine 
mees tegi kuskil haltuurat, näiteks kui töötasid viinapoes, siis said 
sealt midagi müüa. Arvutuskeskuses sai ametlikku 
palka ja midagi varastada ei olnud. Reputatsiooniga oli häda --- sinna läksid 
matemaatikaharidusega naisterahvad, samal ajal kui kõik teravamad 
tüübid läksid traktoristiks, sest seal sai kütust varastada. 
Igatahes sinna tööle ei mindud. 

\question{Aga siis tuli üks, kes tahtis!}

Jah. Kuna osakonnas oli kakskümmend naist ja osakonna 
juhataja oli mees, siis olen enda jaoks selle dekodeerinud niimoodi, 
et ta oli andnud kaadriosakonnale käsu, et kui tuleb ükskõik mis meesterahvas, 
tuleb ta kinni võtta ja temale anda. Nende paari aasta jooksul, mis 
ma seal olin, nägin, millised kismad seal käisid --- naised kraapisid vaat et
üksteise silmad verele. Midagi füüsilist muidugi ei olnud, aga ussitamist oli 
korralikult. Hiljemgi olen tajunud, et kollektiivis peavad 
mehed-naised tasakaalus olema, vastasel juhul tekib mõlemas suunas
jama. Ja mul on tunne, et mina olin esimene meesterahvas, kes talle ukse taha 
tuli; teda ei huvitanud ükski muu asi peale selle, et olin mees. 

\question{BBSi aeg sattus arvutuskeskuses olemise aja tagumisse 
otsa. Sa rääkisid, et tahtsid sinna modemit saada.}

Tahtsin, aga ei saanud. Tollal, vist 1989. aastal, hakkasid tekkima
kooperatiivid ja kooperatiivinduse tüüpidel oli raha 
paksult käes. Kui midagi väga tahtsid, siis nad sulle ostsid. Mul 
õnnestus saada modemi ligi niimoodi, et see ja arvuti olid ainult minu kasutada ning suutsin ennast Fidonetti ajada.

Fidonet oli 
kullaauk, täpselt nagu tänapäeval internet. Kus seal info jooksis! 
Kui kellelgi midagi oli, siis sellest räägiti, ja see seltskond tundus 
kohutavalt suur võrreldes paari arvutuskeskuse inimesega, kellega ma 
tavaliselt lävisin ja keda vääristasin endaga võrdseks. Fidonetis aga oli 
kümnete viisi inimesi. Tänapäeva internetis ei kujuta seda enam ette, aga Fidonetis oli Eestis 
50---100 aktiivset inimest, mitte rohkem, ja need inimesed 
olid targad. Jooksin hommikul arvuti juurde, et panna see käima, tõmmata 
kõik viimased kirjad ära ja vaadata, mida nad räägivad. Oli terve hulk 
legendaarseid mehi, kelle iga kiri oli kulda täis.

\question{Näiteks?}

Mulle jättis mulje näiteks Sulo Kallas\index[ppl]{Kallas, Sulo}. Tollal teadsime, et CD-plaat
on olemas, aga oma silmaga polnud näinud. Seda üksnes reklaamiti, et nüüd on lõpuks 
ometi puhas heli. Sulo Kallas oli audiofriik, kes sai endale Sony CD-mängija sel ajal, kui teised igatsesid endale Vene oma. Ta tunnistas helikvaliteedi ebakõlblikuks, pildus selle kasti sisust tühjaks ja tegi sinna uue 
elektroonika. Minule jättis see kustumatu mulje, mäletan seda mitukümmend 
aastat hiljemgi. Ja nüüd, kus ma olen selle härraga ise koostööd teinud, austan teda endiselt.

\question{Kui sa alguses olid Fidos klient, siis ühel hetkel hakkasid sa ka oma 
\emph{node}'i pidama. Millal see tuli?}

See tuli üsna ruttu. Ma olin alguses kellelegi \emph{point} ja ühel 
hetkel \emph{node} numbriga 25. Minu jaoks on Tarmo Mamers\index[ppl]{Mamers, 
Tarmo} alati olnud see vaimne isa, kelle käest olen väga palju vastuseid ja abi
saanud, ja tõenäoliselt tõmbasin tema juurest alguses ka kogu 
oma meili. Hiljem muutusin ise nii suureks, et vahendasin näiteks kogu 
Venemaa meili Eesti vahel. 

\question{Kas Venemaal olid BBSid ja Fido vähem levinud?}

Millegipärast jah. Eestis käis päris 
algusaegadel (mina olen sellest ilma jäänud) kõik Soome küljes. 
Eestit ei tunnustatud välismaal üldse, meil puudus oma aadressruum, kõik olid 
Soome \emph{node}'id. Millalgi aga hakkas see asi Eestis kasvama nagu seen pärmi 
peal ja siis Vene omad olid kõik Eesti \emph{node}'id. Venemaa võrk oli minu 
meelest umbes sama suur kui Eesti oma. 

\question{Olen kuulnud legende sellest, kuidas kaugelt Venemaa 
avarustest käidi lennukiga Eestisse Fidosse, kohver flopisid kaasas, ja muudkui 
kopeeriti öö läbi.}

Tollal oli see vist kiirem jah, sest üle telefoni läksid asjad nii 
aeglaselt, et oli odavam kohale lennata. Minu meelest maksis
Moskva lend umbes üksteist rubla, mis oli üsna väike 
raha.

\question{Kuidas Eesti oma tsooni saamine käis? Siis oli ju veel Nõukogude Liit
või ei olnud enam?}

Vaat ei oska öelda, mina jäin sellest otsast ilma. Äkki Sulo 
Kallas\index[ppl]{Kallas, Sulo} või keegi teine juurguru vanematest aegadest 
oskab rääkida. 

\question{Fido \emph{node}'i sa panid püsti, kas sul BBS on ka olnud?}

Mul ei ole kunagi otseselt BBSi olnud. Kindlasti olen midagi mänginud ja paar tükki äriinimestele kommertsasjade jaoks püsti pannud. Neid 
\emph{impress}'is kohutavalt see, et mingid tüübid olid neilt paar 
aastat raha küsinud, et midagi programmeerida, ja nad ei olnud kunagi näinud, 
mis tulu sellest sai. Siis tuli Tõnu, küsis mingi mõttetu 
rahasumma ja kaks tundi hiljem asi töötas. Tollal jättis see
kustumatu mulje, et lõpuks ometi keegi, kes aitas. Aga sealt edasi 
ei ole ma midagi teinud. 

\question{Kas kuskil olid inimesed, kellel oli äriline põhjus BBS püsti 
panna?}

Tollal oli lootus, et äkki nüüd hakkab äri minema, sest tulid inimesed, kes 
ütlesid, et nüüd läheb kogu äri internetti. Toon samast ajast 
võrdluseks sellise pisiasja, et tol ajal kõige populaarsemal tarkvaral, 
Maximusel, mida kõik BBSid kasutasid, oli \emph{user ID} ühebaidine. Tänapäeval ei kujuta ettegi sellise tarkvara tegemist. 
Keegi hoidis ruumi kokku ja tegi ühebaidise \emph{user ID}! Enamik BBSidel ei 
olnud nii palju kasutajaid, et neil oleks sellest puudu jäänud. 

\question{Järelikult tehti disainis õige otsus!}

Eestis oli üks või kaks BBSi, keda see hakkas ühel hetkel tõsiselt häirima. Ma 
tahan öelda, et see kogukond ei olnud üldse nii suur. Mina tunnetan seda siiamaani 
kui väga elitaarset seltskonda, sest kõik, kes sinna 
suutsid tulla, olid targad inimesed. Näiteks Fidonetti saamiseks oli vaja kolm erinevat tarkvara koos tööle 
panna, et sinna üldse ligi saada, see ei olnud üldse nii lihtne. 

\question{Kas sul hakkas siis Fidos tekkima seltskond, kellega 
juttu rääkida?}

Jah, ja need suhted kestavad üle mitmekümne aasta. 
Näiteks Venemaaga hakkasime äri tegema nendesamade Fido \emph{node}'idega, 
kes sealpool olid. Hüperinflatsioon oli selline kummaline asi, et kuna 
Nõukogude Liit oli suur, siis ühes otsas liikusid hinnad kiiremini kui teises, 
Eesti oli Moskvast kuni nädal aega hindadega maas. 

\question{Ja sul oli infot ning said seda vahendada!}

Enamik inimesi toimetas ajalehekuulutuste kaudu, aga mina rääkisin tuttavaga Peterburis või Moskvas, et 
tahan printereid saada. Tema ütles, et hind on selline. 
Ostsin selsamal õhtul pileti, hommikuks olin juba seal, ladusin asjad 
peale, ülejärgmisel ööl tulin Eestisse ja müüsin need 
Kinexisse\index{Kinex}\sidenote{Üks varaseid Eesti arvutifirmasid, mis hiljem 
tegeles äritarkvara ja sellega seotud konsultatsioonidega.} maha. Kinex oli 
nii õnnelik --- nad ostsid mult kõik kakskümmend printerit korraga ära. Raha oli tollal
päris palju ja vaheltkasuga sai Venemaalt järgmise kuhja tuua.

\question{Kas siis tulidki arvutuskeskusest ära ja hakkasid äri tegema?}

Arvutuskeskuses ei olnud enam mõtet olla, sain seal 110 rubla miinus maksud. 
Eraäris sain juba päris alguses kaheksasada rubla päevas. Neil lihtsalt ei 
olnud mind enam mitte millegagi motiveerida, ainult 
sellega, et \enquote{Tõnu, sinu programmid näevad paremad välja kui minu omad} --- 
kirjutasin neile jubinaid, mis käivitasid nende programme. 
Nad tegid oma moodulid igaüks eraldi binaarina ja minul oli üks
akendega asi, mis neid käivitas. Akendel olid varjud taga, ega seda ka igaüks 
teha ei osanud. 

\question{Kas see oli tekstipõhine värviline terminal?}

Puhas tekstivärk. Meie arvutuskeskus oli selle poolest teistest halvem, et kõigil teistel olid mingid graafilised asjad. Meil oli neid vähe 
või ei olnud üldse. Ma tundsin ennast maru halvasti, sest teised mängisid 
värvilisi mänge ja mina ei saanud. 

\question{Mis mänge sa mängisid?}

Näiteks \enquote{Diggerit}\index{Digger}\sidenote{1983. aastast pärit, 
suhteliselt lihtsa graafikaga arvutimäng, mis oli omal ajal väga levinud.}, mis tollal 
oli täisvärviline. Aga meil olid Iskrad\index{Iskra}, millel olid rohelised 
ekraanid, ja kõik veel null ja üks, polnud pooltoonegi. 
Häkkisime jootekolviga, et saaks vähemalt pooltoonid kätte. 

\question{Järelikult oli sul tol ajal jootekolvi- ja insenerihuvi olemas?}

Oli, aga oskused olid vähesed. See info tuli jälle mõnest teisest 
arvutuskeskusest, et vot sinna kohta tuleb panna kaks takistit. Praegu 
suudaksin ka ise viie minuti jooksul välja 
mõelda, et teeks monitori sellise \emph{fix}'i, aga tollal istusime
pool aastat monokroomsete üks-null monitoride taga. 
Alles siis tuli keegi ülimalt hea infoga, et kui monitor lahti teha, seal 
kaks juhet lahti võtta ja takistid vahele panna, siis tekivad pooltoonid. Meil käed 
värisesid, kui me seda tegime. 

\question{Muidugi, monitor oli ju kallis!}

Sellel ei olnudki hinda. Kui tegid katki, siis lihtsalt rohkem ei saanud kasutada.

\question{Aga oli piisavalt julgust, et kaas maha võtta?}

Kuidagi oli. Kui viis poissi koos on, küll see julgus tekib. Ei ole ilus 
öelda, aga me aeg-ajalt jõime seal ka koos. Tollal tekkis 
arusaam, et kui natuke peale võtta, siis edeneb programmeerimine 
kiiremini. 

Huvitav, et kui hommikul iseenda kirjutatud 
tarkvara vaatasid, siis oli tunne, et üks väga tark inimene on kirjutanud. Aru ei 
saanud, töötas, aga kui puutusid, läks katki. See oli kõrvalefekt. 

\question{Sa ütlesid, et Fidost hakkas kohe infot tulema. Mis infot? Manuaale?}

Digitaalseid manuaale kui selliseid tollal ei olnudki, kõik liikus prinditud info peal 
ja nende digiteerimiseks ei olnud mingeid lahendusi. Kui keegi midagi teada sai, siis ta seda levitas. Teine asi oli igasugune 
ostan-müün-vahetan. Nõukogude Liidus oli kõigest puudu ja sellepärast oli väga oluline 
teada, et keegi midagi müüb. Kasvõi oma vana tooli, üks jalg 
alt ära, aga see oli NSV Liidus väärt info. Ostsin oma 
esimese auto Indrek Sauli\index[ppl]{Saul, Indrek} käest, kes oli tollal 
aktiivne Fidokas. Tal oli Žiguli eksportvariant. Eksportvariant oli tavaliselt 
1500se mootoriga, aga temal oli 1600ne! Enam kõvemat autot ei andnud 
ette kujutada ja ma ostsin selle ära. Ja kuna ma olin Fidonetis, kus ta seda reklaamis 
ja ainult paarkümmend inimest nägi, siis oli mul eelis.

\question{Kas mänge, muusikat, graafikat ja muud sellist kraami ka liikus?}

BBSides liikus väga palju, aga tollal oli arvutivõrk niivõrd aeglane, et ei
tulnud isegi mõtet saata binaare meiliga. Tollal vaatasid, et 
oi, siin on kümme kilobaiti suur asi, ja panid modemi ööseks tõmbama. 
Poolel rahval olid 1200boodised 
modemid. Peter Marvet\index[ppl]{Marvet, Peeter} kirjutas kord suhteliselt ülbes 
toonis meili, et alles 9600 bps modemiga tunned, et tegemist on 
\emph{communication}'iga. Nii et 9600 bps-i üle võis uhke 
olla. Mina ei jõudnud seda osta, aga temal oli selline kuskilt 
saadud, ta oligi minust kõrgemal. 

Kui midagi väga otsisid, siis keegi teadis öelda, et 
vot seal BBSis ma nägin seda, sest enamik asju taandus ikkagi 
piraattarkvarale. Muusika tuli veel hiljem. Mäletan siiamaani, 
et MP3 korralikuks mängimiseks oli vaja 100 Mhz 486, mis oli täpselt selle 
piiri peal, et kui hiirt liigutasid, oli muusika kinni. Ja kui tulid 120 Mhz 
486d, siis võisid hiirt ka liigutada.

\question{Demoskene asjad ju liikusid.}

Oi, demod liikusid, see oli ilus! Igatsen siiamaani neid vidinaid, mis olid 
imeväikesed, aga kui käima tõmbasid, siis oli tuba muusikat ja ekraan 
graafikat täis. Lausa kolmemõõtmelist ja see tundus mulle 
kosmiliselt ilus --- CGA graafika\sidenote{Võimaldas 320 x 200 
ekraanilahutusega kuvada nelja ja 640 x 200 lahutusega kahte värvi.}, mida 
tänapäeval keegi ei vaata. 

Mäletan seda hetke, kui nägin esimest korda \enquote{Diggerit}\index{Digger}. Meil polnud arvutuskeskuses ühtegi helikaarti ja ma käisin 
Tervishoiuministeeriumi Arvutuskeskuses\index{Tervishoiuministeeriumi 
Arvutuskeskus} asju ajamas ja 
keegi mängis seal \enquote{Diggerit} Olivetti arvuti peal. See heli ja värvid tundusid nii 
võimsad! 

Muide, Tervishoiuministeeriumi Arvutuskeskus asus 
surnukuuri kõrval, vist Tervise tänaval. 

\question{Mõnikord käib arvutihuviga kaasas ulmehuvi, kas sinul ka?}

Ma ei mäleta. Olen kõik seiklusjutte vee alt ja kuu pealt sarjad läbi 
lugenud, hakkasin väga vara lugema ja lugesin ulmeliselt palju. Aga 
selleks ajaks huvitas mind meeletult reaalne, mitte väljamõeldud maailm, sest see
on alati välja mõeldud. Ma olen täiskasvanueas kogu aeg fakte 
otsinud, mind huvitavad faktipõhised asjad, näiteks ajalugu. 

\question{Ajaloo kohta ütleb mõni, et see ei ole ju fakt, vaid puhas 
arvamus.}

NSV Liidust tulnud inimesel on see hea omadus, et suudad päris palju 
filtreerida. Ma kuulan Vene propagandat hea meelega, kuna saan üsna hästi aru, mida nad üritavad näidata ja mis on 
tegelikkus. Aeg-ajalt just see, 
kuhu nad propagandat suunavad, annab vastuseid. 

\question{Kas pärast arvutuskeskust toimetasid iseseisvalt või oli sul 
mõni kooperatiiv?}

Ma sattusin eraärisse niimoodi, et hakkasin tooma Venemaalt arvuteid ja 
teenisin selle eest tolle aja kohta meeletut raha. Kuigi see oli imelik aeg, sest
see raha oli ikkagi väiksem kui kellegi teise meeletu raha. Kuna 
praktiliselt kõiki Eesti arvutipoed olid mu kliendid, 
siis üks, kellele ma kogu aeg asju vedasin, ütles, et kuule, hakkame 
parem koos tegema. Tal oli
idee, et müüb kogu mu kolu maha, aga mina toon ainult talle. Mind see huvitas, sest muidu otsisin mööda linna, kellele oma 
kolu lükata. 

\question{Järelikult pidi sul olema päris korralik suhtevõrgustik nii 
Venemaal kui ka Eestis.}

Arv oli väike, aga võrgustik korralik ning suhted kestavad 
siiamaani. Näiteks Venemaal olen püüdnud igasuguseid asju ajada ja olen alati 
lõpuks petta saanud, aga seal on üks inimene, keda usaldan siiamaani 
siiralt. 

\question{Kas see võrgustik tekkis ainult tänu Fidole?}

Jah. Vanad võrgustikud ongi erakordselt usaldusväärsed. Inimesed, kes on üksteist
kolmkümmend aastat tundnud, ei keera üksteisele jama kokku. Kui oled
kommuunis endale pleki külge saanud, siis ei ole enam kuhugi taganeda. 

\question{Ma ei kujuta hästi ette, et Fido seltskond oleks ideaalsetest 
inimestest koosnenud. Kindlasti visati keegi välja ka?}

Mind visati ka välja. Ma ei mäleta, mida ma halvasti ütlesin, aga Tarmo 
Mamers\index[ppl]{Mamers, Tarmo} viskas mu väga kiirelt välja. See oli väga 
hea õppetund, et Fidos ei ole demokraatiat. Fidos on igaühel oma kuningriik ja 
sa oled alati kellegi kuningriigis, pead tema reeglite järgi mängima ja 
kõik. Kui tahad, võid oma kuningriigi luua ja hakata sinna rahvast 
meelitama, aga tavaliselt istud seal üksi. 

\question{See seab selle esimese adminnide saunaõhtu ju hoopis teise 
valgusse!}

Neil inimestel oli reaalne võim sind informatsioonist ära lõigata, aga seda 
ei kuritarvitatud. Kes sai kinga, sai asja eest. Ja mina 
sain ka asja eest. Aga sain ka väga kiirelt aru, kus need 
piirid on. Kakskümmend neli tundi hiljem olin tagasi, sest käisin kenasti 
õige inimese juures vabandamas ja rohkem ei teinud. 

Kloune oli seal üht- ja teistsuguseid, ja võimuga inimesi oli erinevaid. Näiteks 
oli teada, et üks mees oskab karatet, ja kui oli vaja kellelegi peksa anda, siis 
räägiti pigem temaga. Kui oli vaja elektroonikat teha, 
teadsid teise inimesega juttu rääkida, näiteks Madis Kaaluga\index[ppl]{Kaal, 
Madis}, kes hiljem töötas Skype'is. Tema oli see vend, kes julges nii 
kallist asja nagu arvuti häkkida. Enamik meist ei julgenud, sest arvuti oli 
sul üks elu jooksul. Aga tema kraapis vaibanoaga mingid rajad lahti ja 
panin relee vahele, et modemit lahti ühendada, kui see lolliks läks. See tundus 
nii riskantne tegevus, et isegi kui teadsin, mida teha, siis mina ei julgenud. Nii et
kui oli riistvara probleem, räägiti temaga. 

Mingid tegelased kogu aeg müüsid midagi ja oli teada, kelle käest 
mida saab. Näiteks igaüks teadis kedagi Microlinkist või mujalt ja kui tahtsid allahindlusega asja 
saada, siis tuli temaga suhelda. Fidokad tegid tavaliselt omavahel 
allahindlusi. 

\question{Mis selle võrgustiku nii tihedaks tegi? Kas vastastikune respekt kõrge sisenemisbarjääri tõttu või veel midagi?}

Respekt kindlasti, sest alternatiivi ei olnud. See oli oma tsunft: sa kas 
olid seal või ei olnud. Need olid targad inimesed. Tänapäeva internetiga võrreldes on 
tohutu vahe --- kuskil aastast 2000 edasi on internetti tulnud lollid. Ma ei 
mõtle muidugi kõiki. Kui varem lugesid midagi võrgust, siis see oli 
kuld. Üheksakümnendate lõpus, kui Eestis pandi 
püsti Delfi ja igaüks sai endale koju
neti, siis hakkasid horoskoobid ja muu jama nii hullusti 
levima, et enam ei teadnud, mis internetis on tõsi. 

\question{Targad inimesed oleksid ju võinud oma kogukonna kolida teise, 
kõrgema barjääri taha.}

Fido on eksisteerinud tükk aega ja eksisteerib mingil kujul vist siiani. 
Seesama barjäär on ka väga tarkadel inimestel lihtsalt jalus, nad ei viitsi 
seda teha. Tollased piirangud olid ka tülikad. Näiteks asjad ei 
käinud reaalajas, vaid pidid kuhugi helistama, saatma oma kirjad ära, panema toru 
hargile. Keegi teine pidi helistama sama numbri peale, kui sina olid toru 
hargile pannud (muidu ta ei saanud helistada), ja tõmbama kirjad ära. Aga ta ei 
teadnud, et peab just sel päeval helistama, kuna sina oled kirja saatnud. Kui ta otsustas 
sulle vastata, siis tavaliselt käis see kahekümne nelja tunnise 
tsükliga. Ma kirjutasin oma mure ära ja sain sama päeva jooksul kuidagi oma 
vastused kätte. 

\question{Mina tean sind rohkem infoturbe inimesena. Mis hetkel sa 
hakkasid arvutite toomise asemel arvutitega seotud probleeme lahendama?}

Ma ei tea, aga mul on 
alati olnud sügav huvi asjade vastu ja millegipärast mõte töötab ka alati 
tagurpidi, et mida selle asjaga veel teha saab. Sel ajal müüsin automaatvastajaid palju, 
enamik Eestis olnud Panasonicu automaatvastajatest tulid minu käest, 
samuti Citizeni ja Casio kalkulaatorid. Eesti 
Pank\index{Eesti Pank} kasutas valuutakursside teatamiseks Panasonicu 
automaatvastajat, see oli ainuke ametlik kanal, kust sai valuutakursse teada, 
ja seda muudeti vist kord päevas või kord nädalas. Igatahes oli 
see väga tõsine infokanal: helistasid numbrile ja sealt loeti maha, 
et näiteks Ameerika dollar on nii palju. Neil 
oli Panasonicu automaatvastaja vaikeparool muutmata jäänud, seal oli 
kolmekohaline number, vist 555. Ja kui 
valisid rääkimise ajal 555, siis tegi automaatvastaja piiksu, ja kui vajutasid 7, võisid sinna uue teate peale lugeda. Nägin kohe, 
et põhimõtteliselt saaks nii teha. 

\question{Kas tegid?}

Ei. Lihtsalt ütlen, et see oli võimalik. 

\question{Kas sa neile ütlesid, et vahetage oma kood ära?}

Ei, sest tollal ei olnud selleks kanalit, maailm töötas teistmoodi. Siis ei olnud ju isegi 
telefoni, vaid pidi minema telefoniputkasse ja otsima telefoniraamatust numbri. 
Asjad ei käinud nii nagu praegu. Kui tahtsid sõbrale helistada, siis mõtlesid, 
et homme helistan, kuna homme lähen inimese juurde, kellel on 
telefon. 

\question{Ometi ei saanud sinust riistvaraärimeest, vaid huvi asjade 
toimimise vastu sai nii tugevaks, et hakkasid hoopis sellega tegelema.}

Olen vist kogu aeg jooksnud huvi ja raha kombinatsiooni järgi. Näiteks 
sattusin haltuura tegemise ja spekuleerimise järel Kinexi\index{Kinex} 
direktoriks, mis oli tollal Eesti tuntuim ja küllaltki 
tõsiseltvõetav arvutifirma. See tegeles kõigega ja tarkvara osa oli päris oluline. 

\question{Sa pidid siis ju hakkama inimesi juhtima!}

Jah, aga ma olin seda tegelikult kogu aeg teinud, näiteks enda 
erafirmas, mida me kahekesi tegime. Alguses seisime kordamööda letis: kui tema 
jooksis kauba järele, seisin mina leti taga, ja kui mina olin Venemaal, siis seisis 
tema leti taga. Ühel hetkel oli raha nii palju, et mõtlesime, mida me siin 
seisame, võtame kellegi tööle. Võtsimegi. Siis istusime kodus ja 
vaatasime, kuidas see keegi müüb. Ja mina kirjutasin poele nullist tarkvara. Pikapeale läks
seltskond päris suureks, meil oli Tallinnas palju poode ja 
päris arvestatav hulgiäri. Enamik Eesti kontoritehnikast tuli meilt. 

\question{Kas inimeste juhtimine tuli sul loomulikult välja?}

Jah, see oli sõpruskond, kes usaldas üksteist ilma suuremate probleemideta. Ma ei ole kunagi konfliktidesse 
sattunud. Seda olen küll täheldanud, et kui ma ära lähen, siis on aeg-ajalt 
hakatud üksteisele jalga taha panema ja konfliktid
eskaleerunud.

\question{Mida sa praegu teed?}

Praegu on mul selline firma nagu Tochimo Lab\index{Tochimo Lab}. See on nii uus, et keegi ei ole sellest veel kuulnudki, aga mõte on teha
Planet Way Corporationi all Skunkworksi moodi moodustis, kus 
teeme uusi projekte. Planet Way on ise muutunud selliseks, et meil 
on väga tõsised kliendid ja tootmine peab igapäevaselt jooksma, seal ei tohi 
mitte midagi katsetada --- kõik peab käima nagu kellavärk. 

\question{Aga katsetada sulle meeldib!}

Jah, mul on vaja teha just asju, mida veel ei ole olemas. 

\question{Miks?}

Uute asjade tegemine on ääretult põnev ja olen viimasel ajal saanud
aru, et mida võimatum ülesanne, seda rohkem see mulle meeldib, ja see on osaliselt muutunud mu tugevuseks. Pen-testide\sidenote[][-4mm]{Penetratsioonitest --- autoriseeritud küberrünnak, 
mille käigus testija üritab süsteemi siseneda nii, nagu päris häkker seda 
teeks.} tegemine on andnud sellise mõttemaailma, et lammutad süsteeme, mis 
on ehitatud kindlaks. Ma ei taha pen-teste enam teha, sest see on 
tõsiselt depressiivne töö, aga see on andnud lihtsa eelise, et kui 
kaks nädalat ei tule ühtegi ideed ja tekib 
totaalne depressioon, siis aeg-ajalt tuleb pärast seda läbimurre. 
Inseneriteadustel on üldiselt see hea omadus, et kui paned piisavalt 
ressursse alla, siis hakkab iga asi juhtuma. 

\question{Kas nüüd tegeledki uute asjade leiutamisega?}

On asju, mille järele on olemas ärilised vajadused (päris udu ei tee), aga 
millele ei ole päris selgeid vastuseid. 

Pooltel inseneridel on see häda, et kui anda neile väga udune 
ülesanne, siis nad ei suuda seda teha.

\question{Just, sest kuidas arvutada midagi, mille kohta ei tea, mida see 
tegema hakkab!}

See ongi Skunkworks. Kui 
SR-71\sidenote{Lockheed SR-71 Blackbird on strateegiline pikamaa
luurelennuk, mille arendas välja Lockheedi Skunk Worksi osakond. 
Lennuki loomisest Clarence \enquote{Kelly} Johnsoni käe all on tema toonane 
alluv Ben R. Rich kirjutanud inseneride hulgas populaarse raamatu, mida 
loetakse nii innovatsiooniõpikuna kui ka inspiratsiooni saamiseks. } 
tehti, siis eesmärk ei olnud teha mitte kolme-machine-lennuk\sidenote{Lennuk, mis suudaks kolmekordselt helikiiruse ületada.}, nagu lõpuks välja 
kukkus, vaid Venemaa kohal luurata. Keegi ei 
öelnud, mida tegema pidi. Tollal ei 
tundunud ükski asi reaalne, sest kõike, mis lendab, saab raketiga alla lasta. 

Peab olema sellise peakujuga inimene, kes mõtleb nii kaua, kuni saab selle 
vastuse. Tehti nii kiire lennuk, et selleks ajaks, kui rakett õhku tõuseb, on 
lennuk taevast kadunud. Praegu tundub see vastus elementaarne, aga 
tollal oli see võimatu. Selleks peavad olema 
inimesed, kes ei mõtle esimese asjana, et ma ei võta ette, see ei ole 
tehtav, vaid tarbivad kuu aega raha ja muid ressursse ning 
tulevad välja kõige hullumeelsemate mõtetega, millele pannakse hind külge. 
Siis on juba kliendi asi, kas ta tahab seda või ei taha. Näiteks SR-71 
ehitamisel oli see hind, et läks vaja suuremas koguses titaani, kui terve 
läänemaailm toota suutis. See tähendas ärioperatsiooni kuskil Venemaa kõhus, 
kust sai titaani osta. CIA tegi erioperatsiooni, et varjata, milleks ostetavat titaani vaja läheb. Kusjuures titaan on 
nii haruldane materjal, et selle otstarvet on raske varjata. 

\question{Titaanist kelli ju sellises koguses ei tee!}

Just, see oli väga tõsine ressursiprobleem. Ja olla siis see hull, kes ütleb, et 
kuulge, teeme ülikalli asja, me suudame teha! Kusjuures need 
vennad, kes lennukit ehitasid, ei teadnud, kas nad suudavad. 

\question{Aga nad ütlesid ja uskusid, et suudavad.}

Jah, ja kui loed nende inimeste kirjutatud raamatuid, 
siis \ldots Lennuk oli pooleldi ehitatud ja siis ilmnesid mingid hädad, näiteks 
et lennuk venib kuumenedes kolmkümmend sentimeetrit. Selgus, et kütusevoolikud, mis mootorisse jooksevad, 
peavad ka venima, aga kuna temperatuur tõuseb kuuesaja kraadini, 
siis ei saa voolikuid teha ühestki mittemetallist, aga metall ju ei veni. Tekkisid 
probleemid, mida ei olnud võimalik lahendada. Nad tegid torud 
üksteise sisse. Kui lennuk on maa peal, siis lekib kohutavalt. Lennuk 
tangitakse minimaalselt täis, lendab üles, teeb paar ülehelikiiruselist 
tõmmet, kuumeneb mõnisada kraadi ülespoole ja siis pannakse 
tankurlennuki pealt paagid täis. 

\question{Kas sinul on see usk olemas, et mõtled välja ja ongi võimalik?}

No vot, selleks peab täiesti hull olema, et mitte tagasi 
põrkuda. Pen-testide tegemine on andnud selle, et ma enam ei pelga hullusti 
probleeme. 

\question{Kas seda on ka juhtunud, et ei tule välja?}

Oi, kindlasti. See oli üks 
põhjus, miks ma pen-testimise maha jätsin: iial ei tea, millal ja kas jõuad
tulemuseni, ja see on meeletult depressiivne. Teine häda pen-testidega on see, et igal juhul saad peksa. Kui sa seda ära ei lõhu, mida 
ette antakse, siis oled nõrk, ja kui lõhud ära, siis on kõik su peale 
solvunud. Tavaliselt on lõhkumine nii lihtne ka ja siis
öeldakse, et nojah, nii me oleks isegi osanud. See
on tõsiselt ebameeldiv töö, ei soovita kellelegi. 

Uute asjade ehitamine on selles mõttes lahe, et kui olen sinna aega 
investeerinud, siis olen tavaliselt sealt ise midagi saanud ja 
kommuun samuti. See on see koht, kus mulle meeldib 
inseneride hulgas näidata, mida ma tegin. 

\question{Ja sa ju ei ehita triviaalseid asju! Kuidas sa oskad? Lihtsalt 
kogemusest?}

Vist jah, ma isegi ei tea, mida sa silmas pead. 

\question{Näiteks see sõrmus, mille kallal sa töötasid.\sidenote{Tõnu on 
ehitanud sõrmuse, mis toimib žestikontrolleri, võtme, NFC maksevahendi 
ja märguandjana, olemata palju suurem tavalisest gümnaasiumi lõpusõrmusest.}}

Sõrmusega oli lihtne: sellel on Bluetoothi saatja, 
patarei, väike mikroprotsessor ja andurid. Sellist asja suudab 
igaüks ehitada. Ainuke asi, et ei tea, kui suurt. Esimese asjana mõtlesingi 
välja, et kui tahaksin midagi sellist ehitada, siis kui suur see umbes 
tuleks. Vaadates, mida poes müüakse, Arduino näiteks, siis tegelikult suudab igaüks 
selle väga lihtsalt ehitada. 

Seejärel mõtlesin, kas suudan selle 
väiksemaks teha. Kuna ma elan Jaapanis, 
siis võtsin esimeseks reegliks, et peksame igast suurfirmast välja lahenduse, mis on väiksem kui 
turult saadav. Kui tean, et mingi asi on näiteks 
kümnemillimeetrine, siis lähen nende ukse taha ja ütlen, et enne ma ära ei 
lähe, kui saan üheksamillimeetrise. Ma tahtsin konkurentsieelist. 

Ja tuli lihtsalt ehitama hakata, sest siis tulid ka avastused. Esimene avastus oli 
see, et teatud asjad, mida pidasin oluliseks, polnud üldse olulised. Näiteks kui
palju Bluetooth voolu tarbib. Selgus, et see ei loe üldse, vaid hoopis 
see, kui palju Bluetooth magades voolu sööb, sest selle saatmise hetk on 
niivõrd lühike, et see võib tarbida, palju tahab, mind see ei sega. Aga kui see
paari päevaga magades tühjaks jookseb, siis see häirib mind. Me suutsime 
sõrmuse ajada nii kaugele, et see suudab viis aastat karbis voolu sees hoida. 
Kui klient saab karbi kätte ja teeb lahti, siis sõrmus ärkab ellu, kui 
see on toodetud viimase viie aasta sees. 

\question{Tundub, et sul on teatud printsiibid, millest lähtudes on võimalik ehitada mida 
iganes.}

Tavaliselt on jah mõni väga lihtne läbiv idee. Tuleb teha endale lihtne rusikareegel. Näiteks auto 
on mootor, rool ja pidurid ning sa pead hakkama seda selle põhjal kuidagi tükeldama, minimaalse asja valmis tegema. Siis tekib arusaam, milline on 
probleem, mida sa tegelikult lahendad. 

\question{Praktiline käegakatsutav lahendus?}

Jah, ma tegelikult ei saa vist üldse 
keerulistest asjadest aru. Minu esimene samm on asja lihtsustada. 

\question{Keegi ei saa keerulistest asjadest aru, seepärast need ongi 
keerulised!}

Jah, aga mul on tunne, et see on vist see tee, kuidas ma asju teen. 

Toon teise näite. Mind hakkas millalgi huvitama arvutiga 
nägemine: kuidas arvuti näha saab? Võtsin raamatu ja hakkasin otsast lugema, 
mis on see OpenCV teek, mis on \emph{computer vision}'i ehk arvutiga 
nägemise puhul kõige levinum teek. Kui olin poole raamatu peal, siis
mul juba näpud sügelesid kohutavalt, sest olin kõik 
ideed kätte saanud, mida see teeb, ja printsiibid olid lihtsad. 

Kõik teavad, et arvutiga saab nägusid otsida --- tänapäeval teeb seda juba iga 
telefon. Aga mind hakkas huvitama, kas suudan kokku panna, et kui 
inimene on pooleks lõigatud, siis milline on alumine ja milline ülemine ots. 
Nägusid me leiame, aga kas ka jalgu või midagi 
muud? Kas seesama printsiip on rakendatav? Üritasin midagi kokku käkerdada ja 
sain tulemuseks, et võid frankensteine ehitada nii palju, kui tahad, 
arvuti leiab inimesele täiesti sobiva alakeha. See näeb maru naljakas välja, 
aga on ülimalt loogiline, selline inimene võiks isegi 
olemas olla. Ainult et ta ei ole õige. 

Tähtis on proovida. Tollal jõudsin selle projektiga 
nii kaugele, et sain aru, et tegelikult on taust palju olulisem kui inimene. 
Kas sa oled kuulnud sellest projektist, kus ma tõmbasin terve 
rate.ee\index{rate.ee}\sidenote{rate.ee oli Eesti esimene tõeliselt 
populaarseks saanud sotsiaalvõrgulaadne teenus, mille sisu seisnes peamiselt 
üksteise piltidele hinnangute andmises.} alla? See oli hästi lihtsasti 
kopeeritav ja seal oli praktiliselt terve elanikkond sees, igaühest 
hulk pilte. Rate.ee sai alla tõmmatud ja avastasin, et Eestis on
selline sait nagu sexinestonia.com --- väike kommuun, umbes tuhat kasutajat. Kui 
Eestist on tuhat kasutajat, kes kasutab pornosaiti iseenda reklaamiks --- üks on su õpetaja, teine kolleeg, kolmas ülemus ---, 
siis on see väga sensitiivne asi. Üritasin leida paralleele, millised 
profiilid kattuvad rate.ee omadega. 

Mind huvitas tehniliselt, kas olen võimeline neid kokku viima, ja 
avastasin, et neid ei ole väga palju, kes kasutab mõlemas teenuses sama 
telefoninumbrit. Selle järgi kokku viia oli elementaarne, aga oli 
muidki asju. Ma vaatasin \emph{computer vision}'iga taustamustreid. Tuli välja, et tapeedimuster on üsna unikaalne asi. Kui 
pildil on kolm erinevat mustrit, näiteks tapeedi-, vaiba- ja mööblimuster, ja kui nende kombinatsioon on unikaalne, 
siis see ongi unikaalne ruum. Ja kui leiad juba ruumil samasuse, siis 
tavaliselt on ka profiil kohe arusaadav. Saad täpselt aru, et see 
rate.ee-s olev tore väikeste lastega inimene on sama, kes sexinestonias 
kõiki neid muid asju teeb. 

\question{Need on ju küsimused, millele inimesed ei taha üldjuhul vastust 
saada. Miks sina tahad?}

Mind huvitas, kas see on võimalik, ja see oli võimalik. See oli väga 
vapustav avastus, et nii saabki. See tuli ka IT-inimestele üllatuseks, 
isegi turvaala omadele. Kui käisin seda pankuritele 
näitamas, siis neile tuli see selles mõttes ebameeldiva üllatusena, et 
pangas on tuhandeid tellereid ja kui nad hoiavad endast kuskil alasti pilti, siis
muutuvad nad santažeeritavaks. Panga jaoks on see probleem, kui häkker 
teab seda, aga pank ei tea. 

Tekkis mõtlemine, et tuleb iseenda käitumist korrigeerida. 
Keegi teab sust alati rohkem, kui sa ise arvad. 
Tavaliselt on see lihtsalt vandenõuteooria, aga see on see koht, kus saad ise 
tunda, et mina tegingi selle süsteemi, mina tean. 

Ma kasutasin seda spämmerite püüdmisel ära. Suutsin nende 
kohta nii mõnegi pildi leida.

\question{Läksime küll BBSide juurest kaugele, aga sellest ei ole 
lugu, sest su jutt on väga huvitav! Aitäh!}

Sa küsid küsimusi, mida ma ei ole iseendaltki korralikult küsinud. Ma ei tea, miks ma 
midagi teen või kuidas ma teen! 

\question{Ega ei peagi teadma!}

Põhiline reegel on see, et üritan olla ise ja iseenda vastu aus.
