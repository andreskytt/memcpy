\index[ppl]{Mamers, Tarmo}
\index[ppl]{MomraT|see{Mamers, Tarmo}}


Loodetavasti tekib siia jutt Oktroobrirajooni ÕTK kohta\label{content!OTK}, B'Knows viitab.

\question{Kes sa oled?}
Tere. Sina oled.
Tarmo soetamata.
Tarmo, sina oled see mees, kelle kohta sa oled üks väga auväärses kohas veel nii-öelda nimekirjas kõige üleval. Kaks esimest lõngopi episoodi välja näeksid siis inimesteks, koeri toodavad oma. Ja peaaegu kõik, kes nagu otse sõnumeid saad sellest rääkida, sest esiteks on ta väga oluline inimene olnud kõiges selles loos üheteistkümnendal nagu korjandusega. Folkloor ise ka. Et.
Et räägi temaga ja logisime siin, nüüd siis ole. Aga hakkame pihta.
Päris nagu algusest, et kuidas, ja arvutite juurde said juures arvutit sinna saiti.
Mul oli üks klassikaaslane, kelle isa töötas küberneetika instituudis.
Ja ma arvan, et see võis olla kuskil keskkooliaastate alguses, ilmselt siis ehk tol ajal mingi seitsmes-kaheksas-üheksas klass jah, põhikooli ja keskkooli algus, kus sai hakatud või kus sai käidud päris mitu korda järjest siis tutvumas sellise asjaga nagu läheb veel kaks. Ja see oli küll mõnevõrra keeruline, sest et see oli üsna koormatud kuna seal Küberneetika Instituudis teda kasutati mingisuguseks teadus ja uurimistööks. Paha ma küll ei tea täpselt mil moel. Ja kuna selle Apple kahe monitori asemel oli tavaline telekas siis ma mäletan väga hästi seda, et kui ma esimesi kordi sinna sattusin mingil talvisel perioodil, siis põhiliselt seda telekat kasutati vist suusahüppe MMi. Lisaks vastasel ehk siis päris alati ei saanud seda küll kahte nagu näperdada või vähemalt saia või siis ilma pildita hüppasid suunduda hoopis ja hüppasid hoopis suusatades suhtlusele. See oli siis tõenäoliselt umbes aasta kaheksakümmend viis või natukene seda. Kaheksakümmend kolm kuni kaheksakümmend viis mingi selline vahelduvad. Ja Marlen Vassili. Ja mis oli seal olnud nüüd?
No mina kõigepealt vaatasin, mis teised teevad, ja siis kui ma ise hakkasin seda näperdama, siis tähendab teiste, põhiliselt mängisid igasuguseid huvitavaid mänge, mis seal äkki veel tol ajal oli. Aga mind pigem hakkas huvitama see
Et nojah, et kui need arvutimängud on mingi teatud hulga eludega ja teatud hulka relvadega ja teatud hulga mingit abivahenditele, mida saab kasutada kas neid kuidagi nagunii-öelda ära häkkida saaks, et noh, et võib-olla oleks rohkem elusid või. Võib-olla oleks koptimendis, kui seda abivahendeid või kuidagimoodi saaks rohkem maad ja punkte näiteks? Millegipärast minu huvi oli nagu miski selline, et mitte see mängimine kui, kui tegevus, vaid pigem kas millegi ümbertegemine, millegi kuidagi teistmoodi tegemine. Võib-olla, et mingi tegelase
Müts ei, oleks mitte punane, vaid oleks vaheline kuskil mängus. No midagi sellist oleks kuidagi nagu nikerdada sealt midagi, midagi muutuks ja saaks teistsuguseks. Ja enne seda noh, ma ei teadnud midagi programmeerimisest ega ka eriti arvutite tööpõhimõttest. Et see oli nagu see esimene aiendav tegur mis siis tõi kokkupuute esiteks teisiku keelega ja järgmisena sisetüli Assembler ega ja teisalt noh, kuna ma natukene tundsin huvi tol ajal ka elektroonika ja eriti siis digitaalelektroonika vastu
Noh, siis juhtus nii, et meile jäid ette ka selle Apple'i arvuti on valikus olid ka tollaste noh siis Apple'i, tollest või muude arvutitega olid kaasas maalides alati nende elektroonikaskeemide supp ja noh, põhimõtteliselt minu huvi oli see sealt siis nagu näpuga järge ajada, et okei, et kuidas need bitid liiguvad, kui midagi printida või nuppu vajutada. Või kuidas, kuidas ekraani peale pilt tekitatakse bittidest.
See oli nagu.
See oli, see oli armas. Selle hoolduse kohta küsinud, et kas need mängud olid enamasti kuskilt välismaalt või väljaspoolt tulnud või liiklus ka mingid isetehtud asju.
Apple'i peal olevad mängud olid välismaalt tulnud koos siis noh, kas sama kanalit pidi, kus täpselt kunagi olid tol ajal siis Nõukogude Liitu tulnud. Või siis osad nendest noh, eeskätt need Apple'i kasutajaid, kes olid kuskil tartus näos kellel oli kuuldavasti mäletatavasti ka mingisuguseid kontakte või kuskilt mujalt maailmast teiste Apple'i kasutajatega on emaseid, mänge kuskilt mujalt. Jälja.
Sest mingisuguseid mänge ja tehti ise ka mingi, mäletan, mingeid mängus ei mõisa majandada, luku peale, mingit niisugust kraami teha.
Sellest Apple'i ajast me mäletame väga palju mingisuguseid kodukootud mängida küll mõnevõrra hilisemast ajast siis kui siis kui mul oli kasutada nõukogude päritolu selline arvuti nagu visk kaks, kaks, kuus seal oli küll kodu kodumänge, mis olid siis noh, kas idee võetud kuskilt mujalt ja tehtud mäng valmis siis või oli, oli nagu nullist mingisugune mõttemänguks vormistatud ja siis, kui jõudsid meile, venelastele oligi selline liigloo nimi oli aga Ott ja kui lõpevad, kui Eestisse näide kotte tuli kuskilt Venemaalt, siis noh, seal oli mingisuguseid venelannad ärkaks.
Paljude pordid, kui palju?
Ja ja osa olid jah, selgelt Apple'i pealt maha lükatud ja saalid nagunii-öelda, originaalid aga jah, sellest ajast ma ei mäleta, et oleks nagu väga palju mingisuguseid kodumaist päritolu mängija eriti olnud.
Aga kui sul need inimesed, kes seda arvutit kasutasid, lasid sul ka seal niimoodi kõhu alla vaadata ja kantpead maha võtta.
No ega seal Küberneetika Instituudis väga palju ei lastud, sest et seal oli oluline ikkagi see, et see oleks nagu töökorras hoida, vett, kasutada nende mingite uurimistöö jaoks.
Aga hiljem, siis kui ma mäletan, et kas jah, see oli ka ikka keskkooli arvel siis ma sattusin ka tollases ehk siis praegune Tehnikaülikool kus oli raadiotehnikakateedris oli ka üks või kaks ja noh, seal olid siis juba nii-öelda Raadio tehnikut noh, kelle, kelle igapäevane leib ongi see, eks, et vaadata pigem seda, mis seal tõhusam jõudis ka vilets. Et seal see siis nagu sisse vaadata, seal oli masinaga on olemas, kui jootekolb, millega sai pädevam seltskond teo ise Apple'ile mingisuguseid perifeeria kohta kus kivi, ja selleks, et siis noh, tüüpi majas mingisuguseid juhtimise teha või mingeid mõõtmisi teha või asjaks. Et, et niimoodi näitab neid, seal kaotati.
Aga tol ajal, siis ma saan ju aru, et neid ühesõnaga oligi üsna hästi teada, kus kohas on mingisugune konkreetne Apple kaks ja kuskohas? Neid oli Canitud.
Tulitaks ja see seltskond, kes kes oli mingi nii öelda arvutihuviliste seltskond, noh, need inimesed tundsid, teadsid üksteist üsna hästi, noh, võib-olla on tegu olnud, aga nad teadsid, ehk siis ma teadsin, et kellega kuskil arvutiringis see nagu iga nädal kokku puutudes, noh, mingisugune seltskond võib olla kolmkümmend inimest nelikümmend inimest. Ja neid arvutiringe õli millega mina kokku puutusin, selle ajani oli, oli kolm tüütu, põhiliselt neli-viis, siis tähendab oli selline koht nagu Oktoobri rajooni õppe-tootmiskombinaat.
Kus oli arvutiklass ja tegelikult seal oli jama emmessiksid.
Mis kindlasti on tal hästi meeles, nii on seal oli päris suur seltskond noori huvilisi, kes seal koos käis. Ja siis oli tippis oli arvutiring, mida juhendamas Vladimir kõiges. Ja mingi hulk muid õppejõude, kes seal olid, ja sealkandis, et serverit robotrott ma küll ei mäleta, mis see täpne tüüp oli. Aga mingeid selliselt sellise kaubamärgiga masinat seal olid. Ja noh, tippis olid siis ka need Iskra kaks kaks kuueta millega ma siis kokku puutusime tegelikult sülearvuti ringi käigus esimest korda ja siis oli
Selline kool Tallinnas nagu kolmas keskkool, kus oli üks selline matemaatikaõpetaja nagu Jaak londe kes oli nagu haridussüsteemis selline omaette fanatt ja vastiku poleer populariseerimis tol ajal veel koolidesse ikkagi jõuaks. Arvuti, nii riistvara alates kui arvutiõpe ja Jaan kloon teil mingist hetkest alates oli üks bait kasutada. Ma ei tea, kustkaudu nendepoolsele siis sai sealt Venemaalt ilmselt. Ja siis oli olemas selline koht nagu kolmekümne neljanda kooli tehnikaring mille juhendaja oli Ants Kaili. Ja, ja seal käis ka päris mitu sellist poissi, kelle noh, kes see oli nagu otseselt sellise üldise tehnika või siis vähemalt elektroonika huvi väga, veel rohkem siis just nagu faktsiooniga.
Et need vist olid nagu need põhilised seltskonnad, kus siis noh, igas ühes oli.
Oli ka mingisugune katoorseks, et noh, mina teadsin kõiki neid nelja seltskonda, võib-olla oli veel mingisuguseid seltskondi või nii-öelda arvutiringe või huvilisi. Tallinnas, eks ole, Jaan Edward, Tallinnas ja Tartus oli hulk inimesi, kes koondusülikooli juurde seal oli. Anne Villems. Ja seal oli vist mingi hulk, ma ei tea, kui palju oli appleid. Aga võib-olla ei olnud kaeblehmannata, kus oli tartus, äppel olemas, oli füüsika instituut. Seal oli selline mees nagu jaatovormann kellegagi, nagu ütlesin ka tol ajal siis kokku, kui kui ma ükskord seal seda kohta käisin vaatamas ja lihtsalt kannatama sattusin mingil põhjusel Tartusse mingisuguste koolikaaslastega või, või ringikaaslastega, siis, siis me mõtlesime, et lähme sinna Füüsika Instituuti. Küll ta seal esiteks saab sooja, sest et ilmad olid talved väga külmad, mäletab ja äkki saab arvutis ka midagi teha. Ja siis me kuulma nimetatakse. Aga ma ei tea, kuidas need asjad toimisid selles mõttes noh, et üks hommik, mingi seltskond võtab vähe koolivõistlusi, täpselt. Lähme ja sõidame. Ja kellelegi mingeid kontakte ei oleks, mingeid eelnevaid kokkuleppeid ei ole, kui helistaja ei saada meili all ja aga noh, kõik nagu kõik nagu tuleb välja, lõppkokkuvõttes eks.
Mis ma seal koolis saan aru ja ülikoolis on ka aru, et arvuti ja võib-olla arvutirikka, aga mis selle Oktoobri rajooni, mis iganes asutusega huvi oli mingisugused arutureszemmeranki
Siis olid need õppetootmiskombinaadid, olidki mingisuguseid selliseid kuulide tähendab mitme kooli peale eks siis noh, nagu rajooni kauplex neli rajooni oli Tallinnast oli tol ajal. Ja noh, nimi oli õppetootmiskombinaat, eks, mis ilmselt siis pidi viitama nagu sellega, et seal siis annab mingit praktilist asja proovida teha ja, ja proovida ma ei tea, siis nendega ometi kogemusi saada. Ja noh, see arvutiring oli, see on nagu koht selline huviring, et seal ei olnud nagu mingit sellist tootmisväljunditeks, nagu ses õppedofessioon
Minagi seda ma just seda ma imestan, et mis need, miks nad hankisid selles ja pidin keeruline olema.
No igal juhul teleri ja arvutiga asja lugema, neid arvuteid seal oli mingi tosina jagu. Ja, ja muidugi see võis ka olla, kuna see oli otsapidi seotud sellesama young Kloomudega, et tema kuidagi sinna võis vabalt selle arvutiklassi sebida siis õppetootmiskombinaat oli lihtsalt selline noh, nii-öelda katusekontsert kuhu siis õnnestus artiklis tekitada, et see ei olnud ühes konkreetses koolis kus oleks olnud nagu mingisugused nii öelda poliitilise kindel, et näe, neljakümne, aga meil ei ole käes. Ja maitse oli siis nagu reaalne inimene. Võib-olla seda oli lihtsam organiseerida. Ja väga hea, et selline koht olemas või surnud.
Kas kui sa seal hakkasin arvutitega toimetama ja ja siis, ega sa ju puhta koha pealt ju ei hakka, ei avaetapil kahekümne elektroonikas keemia ja ütlete Haasitiiv bit sinna, eks ole. Mille, mille peal sa seda tegid söögil raamatud või kuidas kuidas oskasid?
No elektroonikavabad tausta mul oli nii palju, et et noh, seda ma teadsin, spordi bitid liiguvad, eks, ja mismoodi mingi loogikatehted toimivad ja kuidasmoodi nagu asju tööle panna. Ja kuidas näiteks teha?
LCD displeiga elektronkellaks Ustides ühelt poolt selle kolmekümne neljanda kooli tehnikaringi teadmiste baasi tõenäoliselt või õpitubasid. Ja noh, teisalt oma lunisin vanematelt endale välja küllaltki palju kirjandust, mis oli põhiliselt Vene ja saksa keeles saksakeelsest kirjandusest, inglisekeelse kirjandus tol ajal meil ei olnud lihtsalt saada kuskilt või noh, kui oli, siis see oli siis see oli ilukirjandus ja niisugune Aime ja ulme, eks, aga noh, ta ju mingisugune teadus või tegelik või tehnikakirjanduseks tehnika kirjeldus, kui veel mitte nõukogude päritolu, siis ta oli ikkagi saksa keeles. Ja ega ma ausalt öeldes neid raamatuid ühtegi otsast lõpuni läbi lugenud, aga need ikkagi natukene selle pisi ja ja lugesin võib-olla mõned olulisemad peatükid läbi. Et sealt tasapisi ilmselt kogemus või, või teab minu nagu tekkis.
Millest me siis saame järeldada, et saksa ja vene keel tehnilise sisulise teksti lugemiseks ei olnud probleem tolleks hetkeks?
No minu jaoks Saksa GP-l oli siiski tahaksin koolis inglise keelt süvendatult, kui et selles mõttes, et inglise keel oli noh, minu jaoks nagu nagu eesti keele kõrval teine emakeel, pedagoog aga saksa keelt jah, ma ei skurssinud nagu eriti üldse tol ajal isegi tänapäeval on küll nii, et võtad raamatu lahti, eks, mis sest et ma saksa keelt ei oska. Aga noh, inglise keelega palju sõnu samad mingite muude tuntud keeltega on palju, see, sul seal on samad, nii et et mingist üldisest mõttest saab, aga muidugi noh, ega selliseid konkreetseid juhiseid või mingit praktilist infot saksa keelest näidati. Loe. Mis kujul. Neljakümne neljas keskkool, mis on tänapäeva Mustamäe Gümnaasium ja see on ainult inglise keelt, aga kooli ajal inglise keele väljund oligi eriti Lauriito maailmas oli nagu üsna noh, see ei olnud mingi asi, mida oleks saanud väga palju rakendada peale selle, et jaa, okei esikus on, kas, kas Pentax torusid ja muud ei on küll inglise keeles teaks, aga aga noh, ega seal teisi, kus ei ole neid käske nii väga palju.
Ja nii nad kõiki õnneks ka ei ole, et kõik nagu puhast inglise keeleluga ja sõna. Aga noh, neid ei ole keeruline ka pähe õppida, juhul kui sa inglise keelt ei valdaks. Kas selle tehnikakirjanduse juurde käis ka mingisugune niisugune mingisugune muu kirjanduse või, või ulme huvi, näiteks filmid, mingisugused raamatud, mõnikord käib, inimesed, uurivad masinaid, ise, loevad masinatest.
No kuule, ma lugesin üsna palju ulmet inglise keeles. Ja kooliajal minuni sattus või mulle sattus kätte aga Douglas Adams ja tema Highway keri raamatut, mida ta tol hetkel
Viis tükki kloori ja viieosaline triloogia valmis ja ma mäletan seda vitsa, Hitlerit meil oli, noh, kuna see inglise keel oli kooli ajal veel süvaõppe siis meil oli üks selline tund inglise keeles nagu inglise keele kodulugemine. Kus siis kodus pidi mingit ilukirjandust lugema inglise keeles ja tunnis pidi jutustamas. Ja mina läksin raamatupoodi ja ma nägin kuskil üleval najal riiuli peamisega lits väikeriigi kõige esimest osa. Ja ma ütlesin, et see on huvitav ja pealkiri ostsin selle raamatu ära. Ja hakkasin siis, mõtlesin, et võtame siis selle inglise keele kodu lugemiseks. Aga see on nagu väga mõistlik mõte, sellepärast et noh, inglise keeles, kui mõelda nende sõnade või häda ja mida seal kasutatakse, eks me kõik välja mõeldud sõnad välja mõeldud liigi nimedeks, seadmete nimed ja nii edasi, mis annab eesti keelde tegelikult noh, üsna raskesti tõlgitava tekkis selleks peab väga hea fantaasiaga tõlkijale. Aga mina hakkasin seda raamatut lugema ja siis õpetajale jutustavad, ma küll ei tea, kui palju õpetaja tol ajal sellest aru sai, mis ma talle jutustage hinnaga. Aga vähemasti rohule
See oli ju narratiivi, seal on ju mingisugune niisugune keeruline, sõlmis viienda raamatu lõpuks nagu umbselt.
No ega mina, jah, Ega mina ei saanud ka sellest eriti väga palju aru, kui ma võltsida esimest osa kooliajal lugesin. Ma hiljem lugesin, jah, need ülejäänud osad läbi siis nagu Dogsusse kihtpaikadeks.
Aga oli siis saada niukest ingliskeelset kirjandust ja ingliskeelset ilu.
Siis oli küll ja mis oli? Seda oli igalpool, mina, isegi Tallinna sealt tähendab noh, mina käisin küll see oligi, oli siis pärast kooli lõppu, kui ma töötasin tippis siis ma käisin palju Moskvas ja Leningradis komandeeringus ja noh, sealt sai osta, seal oli valik, on üsna lai ingliskeelset ilukirjandust. Ja Siimo oli ka selline, kelle ma mäletasin, mis tegelikult asumi asumi esimesed raamatud veel, need olid, mingid muud saab jõud või üksiklood, mida maha sattusin, Asimovi tugevam. Et noh, need olid nagu kaks põhilist sellist ulmekirjanikku, kellega, kellega me esimest Liibanonis kokku puutusin. Aga vene klassikud küll võiks küll Strugatski aastal Hans käima olin lugenud, vanem, sest noh, need olid tõlgitud eesti keelde, võiks näha, et purpurpunaste pilvede maha vist oli amfiib inimene, mis telligast ratsutama. Ja noh, siis ma neelasin võimalust mööda igasugust popteadusliku ja aimekirjandust, eks, mis siis oli mosaiik selline raamatusari oli Narvas ega suurt midagi muud ei olnud, sellist aimekirjandust ja seekord võtta seda oli isegi isegi päris palju.
Ja noh, ma ei teagi, nii lollid olid teine, mis teine valdkond, mis tol ajal peale sellise ulmekirjanduse nagu ilukirjandusest või pop. Et nüüd ma tulen, neelasin. Ja noh, tempoga Isa tahtele niimoodi ma mäletan, see vist oli asumi mingisugune nagu kolmas või neljas osa mille maha vist lugesin Rootsis töötades läbi. Ma ei tea, kui palju lehekülgi sellistes olla. Veider värk. Kolm või nelisada lehekülge lehe kõige tõenäoliselt. Seega sada jah, selliseid asju juhtus, neid sai lubada tol ajal, kui kui kuul oli äsja lõpetatud ja Tulast tööde kõrvalt ja tollase elu kõrvalt, et jah, selle asemel et öösel magada ja puhata, eks. Siis võib-olla võtsid järgmise raamaturiiulist.
Aga siis lõpetasin keskkooli ära, siis läksid klipi kohe tööle või õppima.
Võib-olla noh, enam-vähem kohe ma läksin pärast keskkooli artipi tööle ja ja see tipi töökoht tegelikult sattuski mulle kätte tänu sellele Vladimir Viies juhendatud Arnold ringile. Ja meie tibismarieltsin selles samas kateedris, kus viiesaja tegelikult see oli siis elektronarvutite kateeder ja noh, aitasin seal igasuguseid arvuti hooldustöid mingeid laborite häälestamise ettevalmistamise mis erinevatel õppejõududele vaja oli. Ja mõnevõrra hiljem siis
Siis sai kaasa löödud juba mingisugustes, noh aru Tiit hõlmuvates projektides, et kui oli vaja midagi programmeerida ja õliga ja mingi sisend-väljundseadme jaoks mingisugune draiver kirjutada arvutile. Aga kas sa läksid õppima ka õppima, ma läksin mõnevõrra hiljem, kui ma sinna tööle läksin, sest mind see tipp või.
Punased ained ei morjendanud küll meelitanud eriti noh, keda nad oleks meelde andeks, aga mina tundsin nende vastu naguniivõrd suurt vastumeelsust, et et ma leidsin, et ma ei taha nagu üldse õppima minna, aga noh, mingit tehnilist asja, eks, et kui seal on need punased ained juures mis oli siis muidugi NLKP ajalugu ja mingid sellised asjad, eks. Aga mingil hetkel ma, jah, ikkagi paar aastat hiljem läksin õhtusesse osakonda õppima. Olin küll üks nendest tähendab eramu üks enamusest, kes ei lõpetanud. Sest meie kursusele ühe astus sisse vist umbes kakskümmend viis inimest, kellest lõpetas kaks. No mis ained, mis erialases elektronarvutid, aeglevatsu. Aga õhtuses osakonnas senine lõpetanute protsent oli minu arust tol ajal üsna nagu noh, tavapärane võib olla sellise üheksakümnendate algul ja see oli üheksakümnendate alguses, ma arvan, et olin vist üheksakümmend sügis, kui ma astusin siis õppima tippi ja sealt sealt edasi noh, ongi tegelikult kõik need tööd ja meie samamoodi mingisuguseid vaba aja tegemised on üsna palju, nagu on seotud siis alguses ja programmeerimisega ja sellise arvuti tehnilise või pyksdora poolega.
Ja kui kui tips sai töötatud, siis noh, mina olin, elektronarvutite kateeder asus teisel korrusel. Ja samas korpuses neljandal korrusel asus siis raadiotehnika, kus oli, saab veel kaks. Ja siis meil tekkis mastiga ehk Madis kool, ühel hetkel kuidagimoodi mõte, et võiks proovida IBMi vissi arvuteid, mis siis olid meil teisel korrusel kasutada ja mis oli minu igapäevane tööriist kokku ühendada siis Apple'i kahega, mis oli neljandal korrusel masti igapäevane tööriist. Siis me leidsime karm klubi sinna vahele või siis kana kruubi, no see on RS kaks koma kaks pöördelised eksanud natuke teistsuguse, elektrilise signaalina. Ja siis meil oli siis meil oli nagu selline film olemas, et et viisises väljak, kapid, andmeid Apple'i, kahe siis seostamine jääks, et noh nagu tina, ikkagi elektri, kellegi teise arvutisse, kus on selge, miks sa küll muudab ja ja üheksakümnes aasta oli minu arust ka see, kui, kui Eestis sees, mis jõudis, mingisugune info sellest, et on olemas peresid ja tipi majandus oli, siis mast oligi see entusiast, kes pani, pani sealkandis siis esimesel EBSi jooksma. Ja mina esialgu vaatasin seda lihtsalt kõrvalt, mul ei olnud selle kohta nagu mingit arvamust. Ja, ja ma ei tundnud nagu väga palju huvi selle vastu mis siis konkreetselt puudutas, nagu seda ei peetud. Ja ka esindus kui sellist noh, seal seal mingeid faile vahetada, aga aga ma ei ole kunagi mingi eriline. Esiteks ma jälle näppu fanatt olnud, sest ja noh, siis mind ei huvita nagu mingisugused mängud, mida saab miskit sätebesside muudu tõmmataks. Ja see, kus ma leidsin, et neid VVS-i võivad olla kuidagi kasulikud. Vist oli see moment, kui tuli välja, et seal Debessides on olemas mingeid tekstifaile, mis on mingisugused referentsdokumendid nii-öelda, noh, mingisugused maalimas, mingit standardit, mingit programmeerimisõpikud. Noh, kas siis adjemide jaoks vajab minu jaoks?
Kas nende mingisugused lihtsalt teinud tekstifailid või lapiga?
Mis tulid, olid läinud tekstifailid, aga nad olid natuke formatitud ikkagi, et neil olid tabulatsioonid sees. Siis oli lehekülje vahelise ees, et noh, täitsa välja trükkida, need tulid ikka killustiku matud paberi peal.
Ja seda siis selle sümbolprinteriga.
Ja noh, mäletate selline eesti keeles oli Tärk printeris teatrit häid maksumus või noh, Maardus?
Sest noh, maa püksid olid, olid nagu kättesaadava hinnaga ja need olid enamusele arvutite taga sest noh, mingid suurelt suured arvutid ehk siis nagu jeeessil või Essemmid, mis olid tibi su Küberneetika Instituudis, seal olid need laiad mida siis, kuidas nende printerite kui pikk rida võimetele ja lahingintervjuud inglise keeles, aga seal oli mingisugune eestikeelne sõna ka mille Ustus, Agur, mis mõttes välja. Aga ma ei mäleta, mis see sõna võis olla. Aga ühesõnaga mingit koledat koledat häält ja Värnatagevad printerit.
Ja siis siis kui sa said aru, et sealt saab igasuguseid speke alla võtta, siis saaksite, võisid tulla.
Jah, ma arvan, et see oli see, see hetk ja see ajanud või kui ma leidsin, et sealt nagu peale mängude ja mingi tilulilu sa midagi mõistlikku ka
Ja siis mingi hetk, mina panin oma Webessiga püsti ja selleks ajaks oli pidanet jõudnud ka otsapidi Eestisse. Noh, väga paljud, kes, kes nagu ajalooliselt on tagasi vaadanud ja, ja rääkinud võib-olla sellest ajastu, nad ei pruugi eri olla vahet teenopeeveeessildusel ja pidumetiga, tegelikult need olid kaks eraldi maailma täiesti eraldi moment, mis loetud. No vahe oli see, et GPS oli lihtsalt mingi süsteem, kuhu saab modemiga sisse istuda ja siis saab seal süsteemis sees toimetab. Ja mingeid andmeid noh, failide näol siis talle tõmmata või siis mingisuguseid sõnumeid vahetada. Aga kogu see info ja need sõnumid on salvestatud sinna, ühte konkreetsesse VVS-i süsteemi.
Ja sydame.
Noh, ühe otsaga ta sai alguse nendest samadest Eeveeessidest, aga Idoniti eesmärk oli siis see sõnum näit. Peebeeesside, mingite muude Sidoniti liikmete süsteemide vahel edasi-tagasi toimetatud.
Pidanetis arvan, et kui need kohad, kuhu see sisse helistada, helistasid üksteisele sisse, paistsid mingid.
Ja see oli siis nagu automatiseeritud süsteem juba, kus siis olid automaatvahendid selleks, et neid sõnumeid vahetada, ehk siis meile valetada. Ja, ja meile oli, oli siis kahte liiki, olid privaatmeilid.
Ja ja olid konverents, meilid, mis siis on, aga noh, nii-öelda tänapäeva mõistes meiligrupid või, või meililistis
Kas juust, juust, nüüd tekkis ka?
Kui ostjaks oli varemgi Osvet on, on hästi vana asi ja ja Youznel ja juhini uusi piiriprotokollid sellega seotuna.
Noh, on põhilisel linnuga juuniksi maailmapäritolu ehk siis see oli konkreetselt juuniks arvutitevahelise meilivahetuse protokoll. Ja see Youznet, mis siis sinna ümber tekkis, noh see oli siis ka nagu selline konverentsile või vestlusringiga süsteem, kus seda peegeleti pilustusega. Ja seal olid Päsid olemas. Et Youznetist sai konvertida ümber kirju Sidonenuti nendesse rühmadesse või neli vahenentsidesse ega muuhulgas ka faile, sest et juust netis vahetati ka väga palju faile. Ja neid oli siis võimalik ka juustitist konvertida tavalisteks valjemaks, mis siis kuskil Debessis salvestati ühest.
Kas kas need eestlased, toimetused seal Youznetis mingites oma gruppides või möllati nendes olemasolevates?
Arvutijoomise ja uznetis. Ma mäletan küll, et ei olnud mingeid erilisi Eesti spetsiifilisi gruppe või, või regionaalseid gruppe. Erinevalt pidamatust, sest idanemis oli küll noh, mingi viisteist või alaealisi oli võib-olla kakskümmend lokaalset sellist vestlust struktuur gruppi, et infot.
Siis noh, mingi kaks-kolm gruppi olid üsna populaarsed
Liikmeskonna mõttes, siis sa pead viiskümmend, sada, sada viiskümmend, viissada inimest. No ma arvan, et lugejaid võis, seal oli väga palju, sest et pidevalt tuleb välja inimesi, kellega sa ei ole, noh, mina ei ole kunagi kokku puutunud mai tea neid nimepidi, aga nad räägivad, et nad on kunagisel Tehvast midagi lugenud. Sest noh, tegelikult selleks, et neid ehmasid või konverents lugeda, vaba ei pidanud sa ise oma Peebeeessiga mingi südametu süsteem. Vaat see säilib helistada, tõekspidamisi sisse, sealses välja lugeda, kui sa tahtsid ja, ja kirjutada. Ja kellel süda netisüsteem, siis oli püsti pandud, noh, eelis oli selles, et siis talle need kirjad tulid automaatselt koju kätte ja tal ei olnud vaja kuskile kaugele. Ise helistatakse. Et lugeda-kirjutada, et tegelikult siin rinde ilmselt neid inimesi, kes võib-olla ainult luges see rind, kui sulle tegelikult päris suur. Kui püüda hinnata seda, kes seal aktiivselt suhtlesid ja kirjutasid ka siis, noh võib-olla see on mingi kakssada inimest ikkagi päris palju. Noh, see on väga laastavalt.
Ja, aga noh, see on suurusjärgus mõttes ei olnud kolm ja ei olnud kolm miljonit, eks järgmisena on selle oma selle VVS-i paikimisele. Dido nõudi panite püsti selle jaoks, et asjad tuleksid koju kätte, ei oleks ma ise vaeva näha või mis asi nimi on üldse.
Ma arvan, et eesmärk oli jah, see, et asjad oleks automatiseeritud, piisavalt, et mulle tuleks nagu endal vaja mingeid liigutusi teha ja ja aega viita sellepärast, et kuskile VVS-i nii-öelda löögile saada, sest kui GPSi küljes välismaailmaga suhtlemiseks oli üks muude, siis see tähendab, et üks inimene igal ajahetkel korraga sai seda VVS-i või teenust kasutada nooli, vedesse, millel oli mitu modemeid küljes.
Siis see mitu inimest paralleelselt seda kasutada, aga nüüd see tähendaski seda, et et helistasid modemiga telefoni kinni Eesti viie minuti pärast kinni. Ja noh, miks ma pean niimoodi vaeva nägema ja pidevalt helistama, noodis külge? Modem valis ise automaatselt, tegi kordusvalimisteks. Et siis kui lõpuks löögile suitsust andis mingi signaali. Aga, aga ma leidsin, et parem on seda asja lasta sellel Vidoneti automaatikal teha. Ja siis see rahumeeli saab. Hetkel, kui sa tahad avada meililugemise programmi ja lugeda seda veini, mis vahepeal suri, sinna masinasse, siis ära tuhandeteks. Mis sul uudi nimi oli? Minu nõude nimi oli mämbox. Memm, paks.
See on, tähendab ainult, ma ei mäleta, mis hetkel see eesliide, mis on siis minu perekonnanime algusest tekkinud, millal, millal see nagu hakkas kuskile mingisuguste asjade külge tekkima. Aga noh, tol hetkel oli jah, nii et kui ma tegin Debessi, siis ta oli mänguks, kui ma tegin, kirjutasin mingisugust programmi nii-öelda oma lõbuks, siis siis ma kirjutasin sinna kopi Rait narris oht. Et see oli, see oli tol ajal selline kaubamärk ta, mida mina kasutasin siis sellist ühesugust ees liidritega. Ja üsna tüüpiline oli see, et, et kellel oli vedes, kui tal alguses öelnud, siis ta mingil hetkel lisas sinna GPS-i ülesele Fidoniti funktsionaalsus. Ja väga palju oli ka teistsuguseid suundumusi ilusti, et kui sul oli mingil põhjusel tekkinud pidanud jõud siis väga palju nendest naudi omanikest mingil ajal leidsin, et võiks nagu ka PGS-i püsti panna. Noh, muidugi väga palju ka ida netinõude kelle omanikud või siis säsopid. Nende eesmärk oligi see, et lihtsalt ja see nõud on, selle jaoks võib eeliseks lugeda-kirjutada ja automaatselt laste vahetada. Et nende huvi ei olnud mingisugust GPS-i üleval pidada, seal on.
Ehk siis see, et, et kui mõni PPS sai populaarseks, et sinna ei saa nagunii-öelda löögile siis see võis olla nagu kahel põhjusel. Esimene põhjus oli see, et seal oli mingisugune kogukond, kes omavahel mingeid sõnumeid vahetasid, õnnetuse mingeid faile jagasid ja asetas. Ja teine põhjus oli see, et millegipärast leiti, et just nimelt sealt on hea siis Gido pidurdada pääseda.
Noh, kus ida juurde pääses kõikidest VVS-i käest, kes olid sigareti liikmed, sest eks noh, kõigis oli ühesugune koopia, põhimõtteliselt nendes konverents kirjedestaks. Iseasi oli, ei vaatlejaid, sest noh, siis oli põhimõtteliselt vaja pidaliTignaudi numbrit teada, kuhu saab kellelegi inimesele kirjesata, et, et iga inimene oli siis mingisuguse piduneti noodiga seotut. Et seda privaatmeili vahetada. Aga mis konverentsile infosid siis puudutas jäljed olid ühtmoodi kas või täis siis tegelikult saadaval.
Aga ega muidugi ei olnud eriti mõnus ka see, et tänane patsient, kaamerad hoopis teises teie käis siis seda meili, sest et noh, seal on viited või kontori, täpselt, kui palju sul on loetud meile, kus lugemisjärjekord on. Kas sa oled millelegi vastanud või ei ole, eks, et et see läheb sassi, kui sul ei ole oma sellist nii öelda udubedes. Ja ka jah, see, et kes, kus oli väga populaarne välja käia tõmbamas noh, oli oli selge, et jalad, nende püssid on nagu üsna hõivatud ja tihtipeale kinni nende failide tõmbamise pärast ekstsesse võitluslikku jalgpalli kätte saada. Sest noh, alguses, kui terrassid Eestisse tekkisid ilmselt rida ametinõuded, siis noh, ütleme niimoodi, et neliteist tuhat nelisada voodi siis noh, ümmargused võib seda teisendada ca neliteist tuhat nelisada bitti sekundis. Või siis neliteist kilobitti sekundis. Andmevahetuskiirus oli üsna tüüpiline nagu algajatel nende Debesside juures mäletan, üheksa tuhande kuuesaja millegipärast jääks tuhat kuussada oli jah, selline lihtne, odav iga mehe tehnoloogia, aga ütleme nii viisteist ja neliteist koma neli olid sellised, kuhu poole kõik nagu püüdlesid. Ja sealt edasi tuli siis üheksateist koma kaks, kakskümmend kuus koma kuus, mingid sellised numbrid.
Minul ühel hetkel oli kasutada selliselt üsna nii-öelda härnat modemit, mille töö kiirus oli kolmkümmend kolm tuhat kuussada poodi. Aga sellisel juhul, kui teisele poole sideliini otsas on vastas täpselt sama tootja madalam robotiks, siis see oli Beck õppureid ja selline firma nagu bet mudeli nimi oli drill, pleier. Paneb aeglase nimeliseks läks ja robotiks siduvas motiksid olid need, mis töötasid. Peebeeesside põhirajas ajab nii-öelda või põhiajastul kõige kiiremini vist kakskümmend kolmkümmend neli koma neli töötasid robotiks kõige kiiremini.
Kas ja kes selle beibe essi selle vahetusega seoses, kus on arvutivõrkude vastu huvi tekkima, sa rääkisid, mis noh, kuidas te mastiga Apple'it ja pisipaaritasitajaga
Ma arvan, et see Reawishibavitamine oligi see, mis selle võrguduse kui sellise pisiku nagu tekkis sest ega tippis ega ka kuskil mujal, kus arvutitega sai kokku puututud, ei olnud nagu mingisuguseid erilisi nii-öelda koht, võrgutamise tehnoloogiaid kasutusele jõuludest jõulud, mis neil võrkude väga, et, et ainukene olid ise uusi filmis käis siis juhitsite vahel. No see oli rohkem nagu selline tõsisemate suuremate arvutite sidepidamine siis ja rohkem nagu teadus akadeemilistes ringkondades, eks. Ja teisalt siis just see idaneb, kus siis oli selline asjaarmastaja lik ja pärast tipp minu järgmises töökohas põhimõtteliselt ma kuulsin siis esimest korda kokku parknetiga, kus oli, mis vastuolus. See oli siis aasta üheksakümmend üks. See oli selline ettevõte nagu skriining, mis eksisteerib tänapäeval. Ja springus ma puutusin kokku Sizarknetiga, mis, mis on, mis jooksis tol ajal kahe ja poole megabitti illuse peal. See oli koaksiaalkaablivõrk. Nii et noh, peaaegu nagu esimesed Ethernetivõrgu taga, siis jah, ütleme neli korda aeglasemaks. Minu arust see kaabel oli ka vist seitsmekümne viie ohvide, ma arvan, oli koaksiaalkaabel, mis argnetisel kasutusel.
Versus etherneti viiekümne ma minema veel. Aga noh, see ahnete oli, oli nagu üsna lühiajaline selles mõttes, et Daphne tiga oli fokud tänu sellele, et see oli see armastus Soomest seljakotiga toodi mingeid kraavi. Ja, ja kuskilt mujalt lähivälismaalt, kus väga palju kraami, mis siis jah, Soomes tuli oli selline kraam, mis Soomes oli nüüd maha pandud sest seda ei tahetud seal ära visata, sest see maksis, eks elu utiliseerimine siis anti mitu kasutage tekib midagi. Ja siis püsti ja siis tehti ja midagi selle ametiga. Ja narknetiga. Ma ei mäleta, et jah midagi väga tõsist oleks tehtud, aga mingeid kokkupuuteid sellega ikkagi olid. Ja selle peale tuli siis Ethernet, mis oli tol ajal kuks Jakob Ethernet kümme megabitti sekundis. Mis, mis oli siis selline asi, mis hakkas päris reaalselt nagu ettevõtetesse jõudma. Ja mille peal siis hakati kohtvõrk tegelikult üsna-üsna palju ehitama.
Räägi korra sellest skriiningu sel aastal üheksakümmend üks, mil Eesti Vabariigi iseseisvaks sai uuesti vigastada üheksakümmend üks. Et aastal üheksakümmend üks teha arvutifirmad, siis sellest järeldub vaieldamatult kaks asja peab olema kuskil arvuti peab olema kuskil mingi ärimees, kes nagu esimene-teine pool, aga et aastal üheksakümmend üks oleks olnud kumbagi, et see tundub nagu natuke uskumatu.
Noh, arvutid olidki sellised, mis alguses tulid seljakotis piiri tagant. Ja, ja siis järgmine faas oli see, kus kus nad olid endised seljakotiga piiri tagant, aga selleks, et neid saada, selleks oli vaja sinna piiri taha seljakotiga kõigepealt sularaha voolama. Siis kakskümmend tuhat rubla, võib-olla arvuti eest, ma arvan, oli selline keskmine arti madja, mina ei muutunud hindadega kokku, sest ma ei tea küll müügitööga. Nii et ma juhin, ma ei mäleta, ma ei kujuta ette, kui palju neid arvuti tol ajal nagu numbriliselt maksid, aga aga arvutustehnika ja oli veel meeletult kallis.
Kuidas, kuidas kütet tekib, üldse, et ei olnud ju nagu CV Online Eesti saanud võtsid, et tulge meile tööle, teeme uuesti.
No ma ei tea, IT-maailmas inimesed liikusid ilmselt tutvuste kaudu ühest kohast teise tööle. Ja mina siiascreeningusse jõudsin ka tutvuste kaudu, sest et üks inimene, kes varem oli olnud minu kolleeg tippis sattus tööülestreeningusse ja kutsus paar aastat hiljem mind ka sinna. Skriiningu nii-öelda Bertikalt või kliendisegment oli ja on ka tänapäeval meditsiiniasutused ja meditsiiniasutustele võrgud, arvutibaas ja infosüsteemid, nende kirjutamine ja hooldamine. Ma arvan, et see on ka üks põhjus, miks kriinil tänapäeval elus endiselt ja ja ilmselt elab väga hästi pole just et tal on oma üsna kitsas kliendisegment ja, ja kindlad ja väljakujunenud noh, kliendisuhteks.
See tähendab see seda, et seal mõnusalt kuulates tundub, et see esialgu esimesed, nagu need esimesed Hartkema Tsygisid, siis need olid ikka siuksed noh, sõpruskonna või vähemalt tutvuskonna nagu põhised.
Kuid hinnad on jäänud, sest menüüs noh, Mart, kes enne mind sinna läks ja kes mind hiljem kutsus.
Oli ainukene inimene, keda ma seal tundsin, aga teisi ei tundnud, aga jah, sellised arvutifirmad olid skinny, kolimad ja viis-kuus inimest, tõenäoliselt mitte rohkem. Tol ajal kõik tegid kõike enam-vähem noh, võib-olla mõni ja programmeerimis rohkem, võib-olla mõni nagu mina näiteks edasrohkem kaablit või käis seal mingeid kruvisid keerama seks või Dimimas mingeid asju seal arvuti kaane all. Et ja mingid eelistused olid kindlasti inimestel olemas. Aga üldjoontes võib öelda, et, et kõik käisid nagu mingil määral vähemalt üle kõikidest nendest süsteemidest, mis, mis maja siis nii-öelda, kui seal IT-firma sees kasutusel olid või millega see firma tegeles.
Kas su sel ajal veel oma peresse pidasid või?
Seadus TPS, mul oli üleval päris pikka aega, ehk siis noh, ma olen nii-öelda kaasa vedanud seda ühest eurost teised. Sest ega tol ajal kodus ei saanud. Noh, esiteks kodu ei olnud kellelgi eriti võimalik arvuti tankida, sest see oli kalliseks. Ja kui oli ka võimalik hankida, siis võib-olla mingi niru arvuti, mille peal VVS-i püsti ei pane, hästi, võib-olla. Ja teisalt tol ajal kodus telefoni ka väljahelistamine noh, ei olnud just mitte võib-olla kõige odavam lõbu. Pealegi, kui mõelda pidaneti peale ja vaikselt idanenud oli ülemaailmne süsteem, siis see hõlmas ka mingit hulka rahvusvahelisi kõnesid siis kodust ei olnud võimalik otse helistada välismaale. Kokollimine toimis noh, läbi inimoperaatoriks. Ja ega ma kõikidest ettevõtetest ka ei olnud võimalik välismaale istuda siis tihtipeale oli ettevõttes oli üks telefoninumber, võib olla mingi kümne või saja telefoni peale, kus siis sai otse välismaale helistada ja siis püüti endale ära rääkida, et sinna taha saaks PPS ühendada. Ja siis tihtipeale olid ka VVS-i omanikel kokkulepped, et nendega vestelda öösiti. Siis nad saavad seda telefoniliini kasutada sealt eristada välja, kui on vaja, eks, ja päeval saab seda liini siis kasutada kontoritööks ja laseb seaduspäraseks inimkonna teenimiseks. Ja noh, sellised ajad tekkisid hiljem, kus kus GPSi jooksu oli mõnedes firmades võimalik siis kakskümmend neli tundi saada mingisugune telefonil ja eriti hästi, kui sealt sai, siis ka välismaale helistada. Aga selliseid kohti oli. Ja noh, tol ajal oli nii, et kui mina liikusin ühest ettevõttest teise, siis siis ma uut ettevõtet muuhulgas nendes ikka selle järgi, et kas mul on võimalik neid asju kaasa võtta. Ja kas mul on seal võimalik selle GPS-i jaoks saada siis kaubavalimisega telefoniliin ja veel parem, kui see nii oleks siis kakskümmend neli tundi kasutada.
Aga miks siis mida see sulle pakkusid sa seda niimoodi seal ikka vahel see oli huvi lihtsalt, mis seal oli huvitav.
No see, et see info tuleb üsna lihtsalt kätte, mida on võimalik, sealt võib essidest saada. Noh, esiteks on teda varem lihtne otsida, kui sul on, kui sul on juba piduneti, nokk püsti ja, ja siis automatiseerida just nimelt seda noh, veel kord neli vahetust, teiselt poolt ka failivahetust, kui ma tahan saada ka kätte kuskilt kaugelt EBS-ist mingit faili, ma tean selle faili nimeks siis mul ei ole vaja endal käsitsi jällegi sinna veebis sisse logida, vaid seda failised siis tõmmata, vaid ma saan selle Pidoneti Automatic puudude
Selles Fida maailmas neid inimesi ei olnud nagu noh, ei olnud sadu, võib-olla seal selles selle tuumiku juures, aga nüüd ikkagi natuke oli. Aga ometi on see, et sinu nimi otseselt päris nagu oluliselt läbi. Miks, mis sa arvad, miks see nii on?
Seal ei ole väga palju midagi arvata, sest et sest, et noh, tegelikult edesi pidajaid on, on palju nimekamaid olemas, kes seda teeb essi Maarjamaa Eestis põhimõtteliselt siis alustasid ja kes on VVS-i kontekstis palju tuntumad. Aga kas see oli üsna pea tegelikult, kui need tegessid ja rida need olid Eestis levima hakanud ja kasutusele võetud üsna väga hästi, kui kui meil mingis seltskonnas
Pidanud inimeste seltskonnas tekkis, tekkis selline äratundmine, et nojah, et meil on siin küll mingi suur sada kuni kakssada inimest, kes igapäevaselt nagu suhtlevat Heido neti kaudu ja ja vahetavad kirju ja teevad nalju ja vahetevahel sõidunud üksteist ja mida iganes. Et aga noh, meie jazzi nagu näeme kümme inimest, võib-olla, keda me teame, päevas läheb, eks nime ja nägupidi, aga teisi meedia. Et peaks nagu mingisuguse lahenduse sellele probleemile otsima. Ja tegelikult oli juba üheksakümne esimene aasta, kui kui see probleem muutus niivõrd teravaks et enam-vähem siis seesama mingisugune umbes kümne inimene, kümne inimeseline seltskond, mõtlesin, et võiks teha mingisuguse kokkutulekul, ma küll ei mäleta, kuidas need mõtted käisid või liikusid või kes mingisuguseid ideid välja käis. Kui palju me kuskil pidali teie kodus neid asju arutasime või mõtlesime. Eelnevalt, kui me selle mõtte väljareisile, et davai, teeme mingisuguse kokkutuleku. Aga siis oli.
Üheksakümne esimese aasta augusti. Ma ei mäleta täpselt, mis kuupäevad olid augusti esimene pool kus me olime siis paika pannud, et nüüd jääme siis ühel nädalavahetusel kokkutuleku, saame, väänas ühes ürituse kohas kokku. Mingi osavõtumaks oli ka mäleta, viskad kui ka võib-olla näiteks viissada rubla ja viiskümmend rubla, pigem võib-olla vähem. Ja siis räägime sel ajal ja suhtleme ja nägime midagi mingeid IT-kalduvustega bändina, mitte arvutimänge, mis aga noh, flopi heidan ja kõvakettaheide ja mingid sellised asjad on olnud nende ürituste üheksakümne esimesel aastal kõvaketas ning kandis heita, nagu kõvaketas tol aastal tõenäoliselt ei olnud kavas. Küll oli.
Jah, oli kõvakettaheide ilmselt, aga see ei olnud päris tänapäevane kõvaketas, vaid siis olid sellised suured üheksateist hulgaliselt läbivat vist või kakskümmend üks tolli sellised plaadid, mis moodustasid kõvakettale, mis ei olnud kuskil energeetilises varguses nagu tänapäevased pöörlevad kettad, eks kas need, kes kuidagi väita esimest lugemist jäänud sest lumest välja nii palju kordi, kui neid oli, ma ei tea, ütleme, kaheksa või kümme või mingi hulk oli kokku pandud ühe sellise oreli, midagi mingisuguse käekirjaga, siis noh, tõsteti noh, niimoodi välja sealt seest sööb, siis see oli nende vaaditonoomsi. Et neid sealt lahti lammutatud kettaid lennutasime küll sellel esimesel kokkutulekul. Me mõtlesime selle kokkutuleku nimeks Väljaveebe summer. Ehk siis Napeedeesi kõlas või lähendisbedes. Ja summad sinna taha. Üks aasta varem oli toimunud esimene Rock Summer. Aga nii palju, kui me oleme erinevate inimestega meenutanud, see Rock Summer ei olnud kuidagimoodi enamuse summeri osa algataja või põhjustada sinna nime, siis sealt öeldi. Et meil olid sõltumatud kaubamärgid. Ja siis, kui see üheksakümne esimese aasta kõige summer toimus, siis, kes teab natukene rohkem ajalugu või on ise siis tol ajal noor olnud huvi ka natukene vanem siis siis põhimõtteliselt sel nädalal, noh, ütleme siis viis päeva, enne, kui beebi Summer oleks pidanud toimima.
Oli see aeg, kui vist Vilnius oli see, kus tulid tankid tänavale. Ja mina selle peale ütlesin, et noh, meie teeme oma tebe summeri ära ka juhul, kui kui ei ole mingisuguseid liiklust, segavaid, tanke tänavatele. Sest noh, olukord oli üsna pingeline ja keeruline. Ja sellele esimesele summeril oli vist viiskümmend kuus osalejat, pluss-miinus. Aga umbes selline number oli ja suur hulk oli neid siis muidugi, kes, kes seal olid puhtalt nagu sellepärast, et noh, nad olid pidaniti liikmed nii-öelda või siis ametisopid. Aga ma arvan, et et üle poole ilmselt oli seda rahvast, kes siis olid mingid VVS-i lihtkasutajaliides, kes lihtsalt tõmbas faile igapäevaselt vahetus meile igapäevaselt, eks, ilma et tal endal
See on kas tasuta või selles üsna niisugune kõva suhe, see tähendab, et seal see seltskond pidi olema ikkagi üsna niisugune tehniline, sest ega selle Peebeeessi ju Gidon uudi pidamise barjäär oli päris kõrge. Seda igaüks pidanud.
No ta oli mõõdukalt ilmselt tehniline, igaüks ei pidanud tõenäoliselt seda, kes tundis ja tehnika võib-olla temast kuskilt üle pea.
Ja selleks ähetserides välja häälestada, korralikult tööle panna ja võib-olla sinna see Fidoniti automaatika programm käima panna. Noh, see ei olnud päris triviaalne oleks. Aga kelle jutt Youtube'ist videote saanud vaateid ja YouTube'ist videot ei saanud vaadata, teisest EBS-ist ka videot ei saanud vaadata. Küll sai mingisuguseid tekstifaile ja selle kohta tõmba see softi ja tõmbas suhted, aga see soft ja siis pane nad kõik niimoodi kokku. Ja siis tee sellised ja sellised kontrollid ja siis läheb asi käima.
Siis olid ju siis sündis selline asi nagu effi kium, mis tänapäeval on lihtsalt veebisaidi mingisugune sanktsioon effi q, liigagi konkreetne fail, mida siis nagu levitati, millest olid erinevad versioonid sulid, Kiili kui, kui sedalaadi küsimused ja vastused.
Eks sealt, kus olid küsimused-vastused, kuidas asi käima panna, kui midagi ei tööta, siis mida tuleks vaadata või on sellised sümptomid? Ja nii ja nii edasi?
Siis neil Eestis ka levitati, tekitati, täiendati. Sa mõtled seoses messidel. Iraanil on igasuguseid siukseid juhendeid kus logistika seitsmekümne tarbiti sisu või toodet uut ka.
Ja kui, kui nii-öelda toodeti mingit sisu, ehk siis, kui keegi kirjutas mingit programmi
Ja kui see ei olnud nagu päris selline asi, mida ta ainult ise oma jaoks kasutab, vaid vaid kui see oli ikkagi mingi asi, mida rohkem saab kasutada ja püsivalt mitte mingi mäng vaid kui see oli näiteks mingi funktsioonide või alaprogrammide teek. Ehk siis Kay Berry mäletan mast tol ajal Cyrites graffiti hoidiste tegemiseks. Ei, mitte päris Kraft, kes oli tekstiliideste tegemiseks ühe funktsioonide keegi noh, millega sassis teha, menüüsid ja kaste ja igasuguseid asju ekraani, pühale tekstile, sisest ole. Hiir oli abiks, millega siis klikata sealt menüüdest midagi välja oli, siis oli see nii aeglaselt kogu, mida sai siis oma programmi sisse kompileerida. Jah, selliste asjade jaoks ikka oli mingisugused pakud või või mingid lihtsad juhendid olemas. Iga ma arvan, et igale vähegi mõistlikum on Operi
Sest noh, mängud olid küll sellised jah, et võtad flopi ja installida, ravi lasermängu käima ja siis vaatad, et kuidas kuidas nagu tööle hakkab ja mida, nagu mingi nukk tee, et ma arvan, et ega mängude monumendid ilmselt eriti lugenud.
Kuidas sealt siis kliinikusse läksin? Ma vaatasin, et ja Uninet figureerib sul ka kuskilt üheksakümnendatest. Ma saan õigesti aru.
Jah, mingisugusel ajal on muidu need olnud minu tööandja on küll päris alguses või natuke hiljem, kui see vist oli umbes neljas koht, ehk siis pärast treeningut ma sattusin Baltic Computer süsteem sisse ka legendaarne ettevõte, mis ka tänapäeval eksisteerib. Ja petšessis, ma tegelesin siis nüüd üsna sihitult juba üsna süütuna selles File siit ära arvutivõrkudega, ehk siis meil oli arvutivõrkude osakond ja me siis ühelt poolt tegelesime kaubedusega ja teiselt poolt siis ka serverite ja noh, mingil määral või sellist tarkvaraga, mis siis oli vaja võrgus käima panna näiteks andmebaasid, mis võib-olla oligi mõeldud algselt ühes arvutis kasutamiseks. Aga mis siis kuskil ettevõtetes oli väga niimoodi käima panna, et töötaks võrgus. Ja pärast
Ehk siis lihtsalt just vahemärkusena tol ajal enamus andmebaas ja olid mõeldud käima ühes masinas fox, proodia ja mis nad kõik olid. Ja teeb elu hästi teise kordne käiakse teades nii lihtsalt näited, mis tähendas, et näiteks mingit jagatud transaktsiooni või sellist lugu ei pidanud tagasi või siis ehitab seda otseselt.
Ei olnud olemas.
Aga noh, oli mingisuguseid viise, kuidas motiivi mindi iiliti ja mööda sellest, et et kui andmebaas arvutis lahti teha, et siis arvutibaas selleks mitte lukus võrgus kõikide kasutajate jaoks sotsiaalseks ikka midagi võib-olla teha. Või kuidagimoodi siis skriptide andmebaase, aga mis sa praegu teed? Vahepeal olen teinud igasuguseid muid asju, mis ei ole olnud väga sellise võrgu tehnilise ülesehitusega seotud või mis on rohkem rakenduste turbaga seotud. Ja nüüd paar kuud ma olen uuesti jälle mõnes mõttes sattunud tagasi sellise tegevuse peale, mis on seotud taas kord selliste võrgubaasprotokollidega ühelt poolt ehk siis ma peaks nagu une pealt teadma aegu, kuidasmoodi, erinevad IP ja distsipliini, UDP protokolli töötavad. Ja nii palju on minu praegusel tegevusel ka endisel seos muidugi mingite rakendusprogrammidega ja ja äppidega siis mobiilide sees. Et otsapidi minu töö on ka teada ja vastavalt sellele seadmisele siis toimetada sellega, kuidas kuidas neid läbi põrkus, käituvad ja mis, mismoodi nende liiklus on. Või andmevahetus on üles ehitatud võrgus kuidas seda andmevahetust ohjes hoida, kuidas seda juhtida, kuidas hakkama saada sellega, mida Google pidevalt uute protokollide näol välja pakub? Ja mille eesmärk Google'i loomulikult on see, et kasutajal oleks Internetis turvalisem kui minu. Minu eesmärk siin on see, et Lisaks sellele kasutajale peab seal Internetis ka olema mugav. Ehk siis amet peaks jõudma ühest punktist teise. Nii kiiresti, et kasutaja teadvustavad kuskil seal vahepeal. Niisugune Internet on, mis võib olla evanitel või aeglema.
Siis on ju selles mõttes väga toredasti kuidagi, kui ma su juttu kuulan, siis see on kõik üks üks lugu sellest, kuidas sa rääkisid res algusesse Apple kahe peale tundsid huvi selle vastu, kuidas selle tüübi asemel mängu sees nagu müts punane pähe panna, et mis seal kapoti allakäik siis on ju tegelikult ju täpselt sama asi on ju tänapäeval lihtsalt see tegelane seal arvutis on täiesti teistsuguse arvuti sees ja natuke tehnoloogia, teine, aga põhimõte ja huvi on ju sama.
Täpselt nii, et minu jaoks on oluline see, mis on karu kõhus kuidasmoodi siis seal kõhus töötab. Kui ta ei tööta Eestis, kas ja mida siis saab paremaks teha. Ja noh, kui ta töötab hästi, siis, siis sellest hoolimata kindlasti seal midagi teistmoodi.
Ja kui hästi suudetakse, on see huvitav. Muudab seda, sest et.
Ja kuidas töötab ja miks ta, miks ta nii hästi tead miks?
Ma olen väga tänulik selle selle aja eest, mis sa, mis sa juttu rääkisid ja küsimustele vastasid, hästi valgustama olnud ja mina arvan, et ma sain sellele oma küsimusele, et miks sa selles nimekirjas nii kõrgele kohale täitsa täitsa hea vastane ja mulle meeldib see vastus, aitäh.
Igatahes me abi.
