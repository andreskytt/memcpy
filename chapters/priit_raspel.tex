\index[ppl]{Raspel, Priit}


Minul on selline Jaapani ajaarvamine, mille järgi ma asju paika panen. Tean
mingeid sündmusi, mille suhtes ma teisi määratlen. Näiteks 
1993. aastal käisin maailmameistrivõistlustel. 

\question{Mille maailmameistrivõistlustel?}

4GL\sidenote{Neljanda põlvkonna programmeerimiskeel.} 
programmeerimise maailmameistrivõistlustel. Käisime seal kolmekesi: mina, Veiko 
Herne\index[ppl]{Herne, Veiko} ja Tiiu Lumberg\index[ppl]{Lumberg, Tiiu}. 
Tulime viiendaks ja saime veel eripreemia kõige elegantsema lahenduse eest. 
Mõtlesime välja Amazoni, mida tol hetkel ei olnud veel olemas. 

Ja selle järgi teangi, et samal ajal hakkasin ära tulema 
Innovatsioonipangast\index{Innovatsioonipank}. 

See jaapani \emph{native} ajaarvamine käib suurte sündmuste vahel, näiteks 
keisri võimule tulemisest kolmas aasta. Ja kui tuli suur maavärin, siis 
see keiser unustati ära. Teati küll, et maavärin oli keisri võimuletulekust nii mitu aastat hiljem, aga edaspidi arvutati aega maavärisemise järgi. 
Sedasi on jube hea, sest muidu ei pane asju enam liini peale. 

\question{Kuidas sina arvutite juurde said?}

Mina olen juhuslik inimene. Keskkoolis olin 
täiesti veendunud, et see on viimane roppus, mida ma õppima lähen. 

\question{Mis aastal see oli?}

Keskkooli lõpetasin 1979. aastal. Ma käisin Kusti-koolis ehk \mbox{1.
keskkoolis}\index{Tallinna 1. Keskkool}, praeguses Gustav Adolfi Gümnaasiumis. Tol 
ajal oli seal matemaatika-füüsika eriklass ning arvutid fakultatiivselt õppekavas sees. 

\question{Kas juba 1970ndate lõpus?!}

Jah. Meile õpetati programmeerimist Fortranis\index{Fortran} ja pagan 
teab milles. Arvutis käisime Teaduste Akadeemias\index{Teaduste Akadeemia} 
Lenini puiesteel (praegu Rävala puiestee), seal nurga peal, kus on ka
raamatukogu. Arvuti oli Minsk-32\index{Minsk!Minsk-32}, mis 
koliti siis just välja ja toodi asemele vist ES-1022\index{ES EVM!ES-1022}. Mäletan, kuidas seda Minsk-32 välja koliti -- terve alumine klaasfuajee oli 
tükke täis. Ühe klassivenna onu, kes oli seal programmeerija, ütles, et 
arvuti visatakse prügimäele. Me käisime sealt plaate välja koukimas, sest 
seal oli P13 transistor, mis oli kõige defkam 
ja igasuguste asjade tegemisel kasulik.

\question{Nii et sul oli juba siis elektroonikahuvi?}

Jaa, esimese raadio panin kokku vist kuueaastaselt. Isa oli ju raadiotehnik. Kodus vedeles jube palju
juppe ja raamaturiiulis oli raamat \enquote{Noor 
raadioamatöör}\sidenote[][-2mm]{Noor raadioamatöör, 
Viktor Borissov, tõlkinud Arnold Isotamm, Tallinn: Eesti Riiklik Kirjastus, 
1953.}, kus oli kirjas, kuidas detektorvastuvõtjat teha. Eks ma siis 
nihverdasin isa sahtlist mõned dioodid, tegin pooli, panin kõrvaklapi külge 
ja sain mingi Majaki\sidenote{1964. aastal käivitatud üleliiduline raadiojaam, mis
tegutseb siiani.} kätte. Majak oli muidugi nii võimsa signaaliga, et see oleks tulnud ka 
pliidiraua pealt. Kui pliit oleks pilti näidanud, oleks pilti ka tulnud. 

Õppimisega oli selline värk, et meile õpetati jube kehvasti. Õpetajaid ei 
olnud saada ja igal poolaastal luges erinev õpetaja mingisugust täiesti erinevat 
asja, mida ta parasjagu ise just oskas. Metoodika puudus, aga vähemalt saime käia 
Teaduste Akadeemia arvutis. Seal olid perfokaardi 
\enquote{kivipurustajad}, mis lõid auke läbi. Saime oma pakid ühte kappi panna ja nii palju ma ikka tegin, et see tundus päris huvitav. 

Olin täitsa kindel, et lähen elektroonikat õppima. Tegin poistele 
raadiosaatjaid ja ükskord isa küsis: \enquote{Poiss, kas see on sinu töö, et
peilingaator sõidab akna all?} Muidugi oli minu töö. \enquote{Näita, mis sa 
tegid? Kurat, sul on sihukesed lõpptransid, me peame võimsuse tagasi tõmbama!} 
Elasime Tõnismäel ja segajad olid sealsamas\sidenote{Nõukogude Liidus oli 
komme välismaiseid raadiojaamu sihipäraselt segada. Üks selleotstarbelistest 
raadiojaamadest (need allusid sideministeeriumile), nr 602, asus Tõnismäe 
ligidal Luha tänaval.}. Vennad püüdsid signaali kinni ja tulid otsima, kus see on. 
Lõpuks rehkendasime 300 meetri peale, kust lähemale ei tohtinud minna ja 
kui läks, siis tuli ruttu ära kaduda. 

Nii et ma tinutasin igasuguseid asju kokku. Siis juhtus selline jama, et mul tuli 
üks üsna raske haigus, mis viskas mul korra aastas 
teadvuse ära. Ravi 
oli keeruline: pool aastat üpris kangeid tablette täpse 
režiimi järgi ja kui ei mõjunud, siis pidi pool aastat pausi pidama. Haiguse äraminek võttis kolm aastat aega, nii et veel keskkooli lõpus olin haige 
ja elektroonika õppimaminek oli vastunäidustatud, kuigi ma tahtsin just seda teha, sest olin lapsest peale aparaate 
ehitanud. 

Olin noor vihane inimene ja mõtlesin, et ei lähe kuskile. Ei taha, 
mul pole vaja, ma lähen tööle elektroonikuks! Kolb püsib käes, skeemist saan 
aru, oskan isegi skeemi koostada, montaažplaate teha, telekat 
käsikaudu parandada (selles mõttes, et ilma igasuguste mõõteriistadeta leian 
vea üles ja parandan ära), raadioga saan hakkama. Aga isa rääkis augu pähe. 

\question{Mida su ema tegi?}

Ema oli Eesti Raadio\index{Eesti Rahvusringhääling!Eesti Raadio} 
majandusjuhataja, tema viis mind sealse tehnokeskuse meestega kokku. Poes 
polnud ju suurt midagi. Koostasin listi kõigest, mida ma poest ei saanud, 
läksin tehnokeskusse ja sealt leiti mulle. Ükshaaval küsisin juppe ja kuskilt 
karbist need mulle ka leiti. Ja kui nad teinekord midagi maha kandsid, siis 
andsid need tükid ka mulle. Mul on praegugi üks karbitäis asju alles.

\question{Kas sa kokkuvõttes leidsid, et pead \emph{midagi} ikkagi õppima minema?}

Jah, isa arvas, et võiksin ikkagi õppima minna. Ütles, et ole nüüd kaval, esimesel 
aastal on üldained -- mine õpi midagi sellist, kus eksamid on samad. 
Läksin ITsse, mis oli siis majandusliku informatsiooni 
mehhaniseeritud töötlemise organiseerimine -- Eesti kõige pikema nimega eriala 
üldse. Kusjuures selles suhtes vahva nimega, et sõjavägi ei saanud 
aru, keda me koolitame. Terve selle eriala jooksul ei läinud mitte ükski poiss 
ohvitserina pärast kooli sõjaväkke, sest nad ei taibanud, et seal koolitatakse
puhtaverelisi programmeerijaid.

\question{Nii et sa pole Vene kroonus käinud?}

Ei ole. 

Meil oli lahe grupp, kus oli ka kuus kooliõde-venda, ja seltsielu läks kohe käima. Kokku oli meid 25, poisse ainult kuus, sest meil oli 
majandusteaduskonna grupp ja sinna teaduskonda tuli palju ilusaid ja tarku tüdrukuid, kes programmeerisid ka päris kõvasti. 

Läks poolteist kuud ja ma olin müüdud mees.

\question{Mille peale see sul juhtus?}

Ma nägin selle maailma ilu -- neid võimalusi, 
kuidas kõik, mis kõrvade vahel olemas, on võimalik ka päriselt. Ega see 
mul lihtsalt ei läinud, sest Gustav Adolfi Gümnaasiumis õpetati 
neid asju väga hüplikult. Mulle on eluaeg meeldinud süsteemne lähenemine, 
aga seal räägitut ei pannud keegi minu jaoks süsteemi ja 
tol ajal ju ei olnud kohta ka, kust lugeda. Internetti polnud ja ka
raamatuid ei olnud eriti võimalik saada. 

Üks sõber, Lembit Sammel\index[ppl]{Sammel, Lembit} vedas mu Leo 
Võhandu\index[ppl]{Võhandu, Leo} jutu peale kolmanda ühika alla, kus olid Nairi-2\index{Nairi!Nairi-2}, AP keel\index{AP keel} ja 
elektriline kirjutusmasin Konsul. Hakkasime APs kirjutama: tegime biorütme ja 
silusime neid ning tegin ka oma esimese mängu, mis oli tiku äravõtmine -- masin mängis vastu selle peale, kes võtab 
viimase tiku, ja sain sellega päris ilusti hakkama. Vaat 
sellesama mängu kirjutamise ajal see asi haaraski mind, sest ma ei 
saanud sellest päris lõpuni aru. Mulle on jäänud see eluks ajaks meelde, kuidas istusin ühel vihmasel oktoobriõhtul laua 
taga, paber ees, lamp põlemas, ja kirjutasin blokkskeemi. 
Pusisin ja pusisin, aga ei tulnud, ja ühel hetkel käis täiesti kuuldav nips ja sain 
aru, mida ma pean tegema. Kirjutasin üsna kiiresti algoritmi valmis ja 
pärast seda ei ole mul algoritmide kirjutamisega mitte mingisugust probleemi 
olnud. 

\question{Kas sa sattusid üsna varsti tööle ka?}

Teisel kursusel, sest esimesel kursusel kammisin ikkagi ühika vahet. Meil oli esmaspäev vaba ja hommikul 
läksin nii vara kohale, kui ühikaalune klass lahti tehti, ning tulin alles siis tulema, kui mind jõuga 
välja visati. Lembit Sammel, hüüdnimega Sass, tegi täpselt samamoodi. 
Panime nagu paaris härjad. Ei olnud päeva, kus me mõnda masinat nässu ei 
keeranud, sest kui neid pidevalt piinata, siis need põlesid läbi. Arvutiklassis sain tuttavaks ka Lindre Reinuga\index[ppl]{Lindre, Rein}, kes oli 
seal inseneriks ja kellega me hiljem tegime koos ühe vahva asja. 

Teisel kursusel läksin tööle EPTsse ehk Eesti 
Põllumajandustehnikasse\index{Eesti Põllumajandustehnika}. Keskkontor asus Salve 
tänaval, kus olid Minsk-32\index{Minsk!Minsk-32} ja 
ES-1022\index{ES EVM!ES-1022}. ESi numbriga ma võin eksida, aga 
Minsk-32 oli küll. Suured saalid olid mürisevaid seadmeid täis. 

Sain seal operaatoritega hästi läbi ega pidanud enam TPI arvutuskeskuses päev otsa perfokaardiga jamama, et see 
järgmisel päeval kuskilt kapist kätte saada. Läksin tüdrukute 
juurde ja ütlesin, et kuulge, laske mu pakk läbi. 

\question{Kas sind võeti sinna mõnd konkreetset asja programmeerima?}

Ei, lihtsalt otsiti inimest, kes oleks noor ja avatud ning keda nad saaksid ise
oma käe järgi välja õpetada. Oleg Kase\index[ppl]{Kase, Oleg} oli selle tiimi 
juht, kus tegutses näiteks väga geniaalne programmeerija Tõnu Toomus\index[ppl]{Toomus, Tõnu}, kes läks kahjuks 
Estoniaga minema. Nad 
hakkasid mind õpetama ja alustasin lihtsamatest asjadest. Esialgu tegin
alltöid, mingeid funktsioone ja värke, mida neil vaja oli. Aga üsna pea 
jõudsin selleni, et tahaksin ise midagi teha. Selle peale öeldi, et vali 
ise. Kuna seal oli parasjagu igasuguste 
kasutajaliideste tegemine, laoarvestuse ja kõige muu viimine suurest ESist suurde 
SMi\index{SM EVM!SM-4}, siis ma tegin vormi 
generaatori: joonistasin ekraani, sidusin andmebaasiga ära ja siis
MUMPSi\sidenote[][]{Massachusetts General Hospital Utility 
Multi-Programming System -- transaktsiooniline võtme-väärtuse andmebaas, 
millega on integreeritud ka programmeerimiskeel. Selle süsteemi puhul oli 
fookus jõudlusel (selle kaudu käib tänini rohkem kui poolte USA patsientide 
terviseinfo), mitte loetavusel; kõiki käske võis lühendada ja reavahetus ei 
olnud oluline. Tulemuseks oli sageli raskesti loetav kood.} süsteemipuuga ära ja valmis.

\question{Ma olen MUMPSist lugusid kuulnud. Kui tänapäeval oled õpetatud 
programmeerija, siis MUMPS on hoopis teisest maailmast!}

Ja, see on endiselt täitsa olemas, jäi mulle ükspäev kogemata 
internetis jalgu. 

MUMPSis ei ole indekseid, need peab ise tegema üle inverteeritud 
immituste. Kuna seal on puu, siis peab puu võtme, \emph{path}'i kirja 
panema ja registreerima, et nüüd on selle kohta võti olemas. Võtmega
võib \emph{path}'i järgi objektile otse peale minna. 

MUMPSis on näiteks niimoodi, et kui tahad mõnd seadet kasutusele võtta, 
siis pead teadma seadme numbrit. Näiteks printer oli 
vist 80. Kirjutad \verb|U:80|, mis tähendab, et nüüd läheb kõik ülejäänud jama, mida 
väljundisse paned, printeri peale. Kui tahtsid kuvarit saada, siis 
igal kuvaril oli oma number. Kuna need olid füüsilised masinad, siis tegid 
andmebaasi kõigepealt loendi olemasolevatest kuvaritest ehk nimetasid need
ära. Seejärel öeldes näiteks \verb|U:1|, sattusid 
esimese kuvari, konsooli peale. 

Lisaks oli võimalik anda \emph{wait}-aegu ehk \enquote{oota nii kaua ja siis mine edasi}, aga ma ei mäleta, mis sümbol 
seal vahel oli. Üldiselt oli selle programmeerimiskeel enam-vähem loogiline. Kuna sel oli 
hierarhiline, puukujuline andmebaas all, siis see pani muidugi omaette 
pitseri, sest \emph{pre-} ja \emph{post-order} ning muu säärane 
pidi hästi käpas olema. 

\question{Kas EPT-l oli igal pool kontoreid?}

Jah, näiteks Sauel, Paides, Tartus, igal pool 
tugevad inimesed eesotsas. Tiimid olid väikesed, neli inimest. Muidugi tehti siis kontoris suitsu 
ja joodi kohvi. Kuvarid olid eraldi ruumis, arvuti teises ruumis, 
igaühel oli ühel pool kuvarit tuhatoos ja teisel pool kohvitass. Kohv oli täiesti 
must ja suhkruta, sest piim läheb ju hapuks, seda ei saa kuskil 
kapis hoida, ja suhkurgi saab otsa, mis sest ikka osta. Nii et seal ma õppisingi musta 
kohvi jooma ja suitsu tegema, mis on mind tükati mitmeid aastaid 
saatnud. Sõltuvust mul ei ole, võin jätta suitsetamise 
maha niimoodi, et panen paki lauaservale, tikutopsi peale ja seal see 
seisab, mind see ei häiri. 

\question{Kuidas kontorite vahel side käis?}

Side käis kahtemoodi: kas ümbrikuga\sidenote{Tõenäoliselt peab Priit silmas tavalist postiteenust.} või läbi teletaibi kanali. Igas EPT 
kontoris oli teletaibi aparaat. Kirjutusmasinaga teletaip oli teksti edastamiseks ja selle küljes oli ka perfolindi lugeja. Meie 
masinast lasti perfolint välja ja söödeti teletaipi ning teisel pool, näiteks 
Paides, lasti lint välja ja söödeti sealsesse masinasse. Aga ega see nii 
lihtsalt ei käinud, seal oli protokoll ka, sest side ei olnud püsiv 
ja kippus ära kukkuma. Siis tegi teletaip piiksu. Selle 
peale tõstis inimene telefoni, helistas teisele osapoolele ja ütles: 
\enquote{Kuule, tõstan kümme kirjet tagasi.} Teisel pool tõstis operaator õla 
üles ja sättis perfolindi tagasi, vastuvõtja tõmbas lindile poole pastakaga 
joone. Seda võis mitu korda juhtuda. Ja kui lint oli lõpuni jõudnud, siis 
vastuvõtja lõikas lindid märgitud kohast katki, võrdles, kus asi kokku langeb, 
liimis otsad kokku (spetsiaalne rakis oli, millega augud läbi torgati, et need 
puhtad oleksid) ja söötis lindi masinasse. 

\question{Ühesõnaga elektrooniline seade muutis andmed kõigepealt 
pabermeediasse, siis elektroonilisse meediasse, siis uuesti pabermeediasse ja 
lõpuks tagasi elektroonilise meediasse. Ja veaparandus oli manuaalne!}

Jah. Tekkis neli koopiat linte: üks, mis lasti siitpoolt välja ja võeti sealpool 
vastu, ning teine, mis lasti seal sisse ja võeti siin vastu. 
Töötas suurepäraselt. Aga siis õpetas Leo Võhandu\index[ppl]{Võhandu, 
Leo} mulle andmeedastust ja protokolle. Läksin meie keskuse peainseneri Rolandi 
juurde ja ütlesin: \enquote{Roland, kas saad mulle 
sellise seadme teha, mis paneb selle arvuti ja selle teletaibi kokku, ja 
teisele poole samasuguse vastuvõtmiseks?} Ma ei teadnud, et see on modem, siis 
ei olnud sellist asja olemas. Tema ütles: \enquote{Oot, ma vaatan, ma just Radio ajakirjas (tolleaegne tehniliste nikerdajate ajakiri) nägin ühte skeemi.} Ja tegigi valmis. Montaažplaat oli 50 x 50 
cm, kuna see oli SMi sisemine plaat, läks nagu riiul 
sisse. Aga skeem oli nurga peal 10 x 10 cm. Ta pani selle 
käima ja mina otsisin vahepeal mööda opsüsteemi, mis võimalused on. Leidsin 
ühe struktureerimata ala -- eraldad lihtsalt mälu ja 
struktureerid ära. Ehitasin sinna peale kataloogisüsteemi ja kirjutasin linte väljastavad
programmid niimoodi ringi, et need kirjutasid 
kataloogisüsteemi, mitte ei saatnud, ja ka seda, kellele saata. 
Esialgu ei olnud mujale saata kui Paidesse, kuigi perfektsionist kirjutab ikkagi adressaadi ka juurde, 
sest mine tea, kellele on veel vaja saata. 
Üks programm vaatas aeg-ajalt, kas on linte tekkinud, ja helistas teise poole välja. Kui side 
kukkus, siis tõstis 10 kirjet tagasi, teine teadis seda ja hakkas uuesti saatma. Probleem oli selles, et sa ei teadnud, millisel 
hetkel side kukkus ehk ei olnud võimalik määrata, mis kirjed olid ära läinud. 
Üks saatis minema, teine võttis vastu ja võrdles, kus hakkasid samasugused kirjed 
tulema, ning loksutas paika. Asjad läksid täitsa ilusti üle. Teinekord kui side ei taastunud, siis võis seanss olla terve päev katki.

\question{Mis tempos see andmeside toimus?}

See võis olla 100--300 bitti sekundis, kiirem küll olla ei saanud. 

Side kiirenes siis, kui läksime elektroonseks. Kõik läks hästi niikaua, 
kuni Tartu ütles, et nemad tahavad ka. Ka nüüd läks kõik esialgu 
hästi, kuni juhtus niimoodi, et Paide helistas mulle peale, side kukkus ja 
siis helistas Tartu peale. Aga ma ei teadnud, et see on Tartu. Üritasin vastu 
võtma hakata, aga sealt ei tulnud midagi tuttavat. Siis mõtlesime välja sessioonivõtme ehk ühe \emph{hash}'i. Enne seansi
tekitamist saatis programm \emph{hash}'i ette ja nüüd oli teada, et 
kui see uuesti tuleb, siis sellesama \emph{hash}'iga, nii et võis ka segamini 
vastu võtta. 

\question{TCP\sidenote{Üks interneti alusprotokolle.} käib põhimõtteliselt samamoodi.}

See on jah vahva, et tunned pidevalt asju ära. 

Ühesõnaga, ma töötasin seal kuni TPI lõpuni, kokku neli aastat ja täitsa huvitav oli.

\question{Kas sa lõpetasid ülikooli töö kõrvalt nominaalajaga? Töö koolis käimist ei seganud?}

Ei! Tööst oli palju kasu, ma olin teistest kogu aeg peajagu üle, sest olin 
saanud kõike seda, mis meile õpetati, elus katsetada. EPT seltskond lubas 
mul väga vabalt toimetada ja ütles, et kasuta kõike, mida saad, peaasi 
et on hea. 

Me tegime Eesti esimese \emph{online}-messi 
arvutitega ja vedasime ise pool kilomeetrit kaablit posti otsas Saue mõisa. Tol 
ajal müüdi hooaja lõpus kõik varuosad maha ja see käis tavaliselt 
niimoodi, et kaubatundjad istusid suurte paberite taga ja tõmbasid maha, mis 
müüdud sai, aga meie vedasime Saue mõisa side ja panime viis kuvarit üles. Peainsenerid ja kolhoosiesimehed said ise 
arvutist valida ja asju selekteerida. Töötas suurepäraselt. Meil oli väga innovatiivne 
kamp.

EPTs oli üks tuntud mees, Tõnu Lume\index[ppl]{Lume, Tõnu}, kes mängis filmis 
Lurichit\sidenote{Tallinnfilmis 1984. aastal valminud film \enquote{Lurich}.} ja 
oli EPT arvutuskeskuse juhataja asetäitja. Juhataja oli Jaak Raja\index[ppl]{Raja, Jaak}, karm mees, aga minusse suhtus hästi. 

\question{Miks sa EPTsse pikemaks ei jäänud?}

Tuli kooli lõpp ja suunamine. Tänapäeval keegi ei teagi, mis on suunamine. 
Tol ajal tehti hinnete põhjal pingejärjekord ja said valida, kuhu lähed. Aga nimekirjas oli kaks kohta, mis 
olid spetsiaalselt mulle -- kui mina neid ei valinud, siis keegi teine neid 
ka valida ei saanud. Üks oli EPT ja teine TTÜ, mis oli sel 
ajal muutunud põnevaks kohaks, kuna sinna hakkas välismaist
tehnikalt tulema, sealhulgas personaalarvuteid. 

Käisime koos Sven 
Jürgensoniga\index[ppl]{Jürgenson, Sven} esimesi personaalarvuteid rongiga Moskvast toomas. Need olid kaheksabitised Yamahad, 
Z80 prosega.

\question{Mis ametikohta sulle pakuti?}

Juhtivinseneri kohta, see oli 
teadusliku uurimise sektori ametikoht infotehnoloogia kateedri 
all\index{Tallinna Tehnikaülikool!Infotehnoloogia kateeder}. Lisaks pakuti head 
palka, kuigi ega EPTs ka kehv palk ei olnud. Tollal oli hea kuupalk 
120 rubla, mina teenisin 155 rubla. Tegelikult aga läks elu 
kehvemaks, sest õppimise ajal elasin ma ikka väga priskelt: sain lisaks
60 rubla kõrgendatud stippi, EPTst poole koha eest 60 rubla ja teadusliku uurimise sektorile\index{Tallinna 
Tehnikaülikool!Teadusliku Uurimise Sektor} tehtud tööde eest 40 rubla. Lisaks maksis EPT 
60, vahel isegi 100 rubla kvartalipreemiat. Kokku mingi kakssada rubla kuus! Aga nüüd oli mu palk 155 
rubla. Mul läks tükk aega, enne kui kõik liinid tööle sain ja teadusliku uurimise 
sektor hakkas mulle lisa maksma. 

\question{Mida see teadusliku uurimise sektor endast kujutas?}

TPIs tehti kõik lepingulised tööd teadusliku uurimise 
sektori alt, kes sõlmis lepinguid ja võttis vahelt oma obroki. 

\question{Mida sealt telliti?}

Igasuguseid asju, näiteks kriminalistika infosüsteeme. 

Ma valisin TTÜ ja läksin Raja Jaagule\index[ppl]{Raja, Jaak} 
lahkumisavaldust viima. Tema ütles: \enquote{Priit, jää ikka
poole kohaga tööle. Sa ei pea kogu aeg käima, astu vahel läbi ja 
ütle, mis arvad.} Nii ma töötasingi kaks-kolm 
aastat veel seal. Käisin ikkagi kohal, sest niisama raha vastu võtta
südametunnistus ei lubanud. 
Täitsa huvitavaid asju sai veel tehtud. Aastaid hiljem, kui ma seal enam ei töötanud, olid SMi 
matused viina ja kartulisalatiga. Masin ise oli maha müüdud, aga 
protsessorikast maeti kuskile Sauele maha. 

\question{Sa oled ainus inimene, keda ma tean, kes on päriselt laulu sisse pandud.\sidenote[][]{Ansambli Folkmill 1996. aasta albumi \enquote{Paksult 
rahul} populaarses avaloos \enquote{Madis Mäekalle valss} on salm: \\
Üks talv oli see, jube libe oli tee,\\
Madis mütaki istuli kukkus.\\
Aga igav oli maas, seltsiks vaid kaevukaas,\\
Madis ohkas ja tudile tukkus.\\
Siis ühmatas Raspeli Priidu,\\
kes kunagi ei kiskund riidu:\\
\enquote{Sa aja end, Madis, nüüd püsti\\
ja tunne end pagana hästi.}} Kuidas sa sinna sattusid?}

Lauri Saatpalu\index[ppl]{Saatpalu, Lauri}\sidenote{Folkmilli 
laulja ja käilakuju.} on minu hea sõber ja tal on niisugune komme, et kui tal 
millestki muust enam laule pole kirjutada, siis ta hakkab sõpradest kirjutama. 

\question{Kust sa teda tead?}

Käisime Lauriga EÜEs\index{Eesti Üliõpilaste 
Ehitusmalev}\sidenote[][]{Tagantjärele vaadates nõukogude aega oma vaimsuse, 
suhtumise ja ärimudeliga hämmastavalt halvasti sobitunud, tudengite jaoks 
organiseeritud suvise töö tegemise vorm. EÜE organisaatoritest, legendaarsetest 
trubaduuridest, sõpruskondadest ja suhetest on hiljem nii mitmeski 
valdkonnas suuri asju võrsunud.} ja oleme koos mitmeid laule teinud. 
Üldiselt tulevad tal sõnad hästi, aga on ka juhtunud, et ei tule, ja siis ma olen 
katalüsaatorina töötanud. Olen ise ka maleva jaoks laulusõnu teinud. 

Lauriga kohtusime esimesel Tiirimetsa suvel, aastat ei mäleta. Hakkasime kohe hästi läbi saama, 
ta on vaimukas inimene. Serbati Tom ja 
Mäekalle\sidenote{Tegelased viidatud laulus.} on kõik reaalsed 
inimesed.

\question{Nii et sa oled muusikamees ka?}

Jah. Ma olen õppinud muusikat päris palju, alustasin kuueaastaselt ja 
õppisin suisa neli aastat muusikakeskkoolis\index{Tallinna Muusikakeskkool}, 
aga siis sain aru, et ma ei peaks seal olema. Seal midagi muud ei õpetatud, aga mul olid muud asjad ka tähtsad. 
Pealegi olen ma natuke rutiinitalumatu nagu infotehnoloogid ikka -- kogu aeg peab \emph{action} käima, sama pala kaheksakümnendat 
korda mängida oli piinav. Tegelikult ma ei tahtnud seda katki jätta, aga seal oli üks solfiõpetaja, kes mind terroriseeris. Ta on 
kõiki terroriseerinud, aga mina olin tal eriline lemmik ja see lõi mu lukku, ma 
ei saanud solfiga hakkama. Ja tulingi ära. Ütlesin emale, et lähen hoopis
laste muusikakooli\index{Tallinna Lastemuusikakool} klarnetit 
õppima. Seal sattusin Aleksander Rjabovi\index[ppl]{Rjabov, Aleksander} 
juurde, kes on Eesti džässi suurkuju ja väga hea õpetaja. Solfiõpetaja
Porrason oli ka kuldne inimene. 
Selgus, et mul on kõik oskused olemas, ainult et need olid lukus. Mul 
on absoluutne kuulmine, mitte küll kõige kõrgemal, aga täiesti arvestataval 
tasemel. See on elus ka veidi piinarikas -- nii kui midagi 
valesti kõlab, siis kohe kratsib. 

Õppisin ka laulmist, nii et poistekoor pluss eraldi ansamblitunnid andsid kõva laulmiskooli. Hiljem õppisin ise saksi ja kitarri 
juurde, klaverit ka natuke. Ma ei ole ammu mänginud, aga klarnet ja saks on 
nii käes, et need tuleb ainult kastist välja võtta. Mul on 
kapis üks siinkandi paremaid klarneteid. Iseenesest on kahju, et see minu käes on, aga kuna see on kingitud pill, 
siis ei saa seda ära anda. Selle kinkis üks 
Eesti välishelilooja sünninimega Elmar Rossman\index[ppl]{Rossman, 
Elmar}, kes on Priit Ardna\index[ppl]{Ardna, 
Priit|see{Rossman, Elmar}} nime all kirjutanud \enquote{Kuldrannakese}. Käisime nädalavahetusel just Ugalas, kus ajaloomuuseumis
on tema ooperi reklaam seina peal. Väärt inimesed on elust läbi 
käinud.

\question{Tuleme korraks tagasi tehnikaülikooli juurde.}
 
Tehnikaülikoolis sattusin 
Toomas Mikli\index[ppl]{Mikli, Toomas} juurde, kellega saime väga hästi 
läbi. Ta oli väga keeruline tüüp, temaga oli raske rääkida. 
Seda suutsid suhteliselt vähesed inimesed, sest ta jättis umbes kolm loogilist 
taset vahele ja alustas neljandalt ning sa pidid ise puuduvad kihid vahele 
ehitama ja mina suutsin seda. 
Tema suutis panna mind andmebaasidest innustuma ja oli 
mu diplomitöö juhendaja. Diplomitöö oli meil muide 300 lehekülge. 

Natuke uhkustan ka, et aastal 1984, kui keegi selle peale veel ei mõelnud, oli mul üks osa
diplomitööst pühendatud konsultatiivinfole ehk \emph{help}-tekstidele. 
Tegelesin tööl metoodilise palgaarvestusega, kus muu 
hulgas õpetasin ja juhendasin kasutajat, mismoodi süsteem töötab. 

\question{Täitsa innovatiivne mõte tol ajal!}

Tollal jah keegi sellest veel suurt ei rääkinud. Korjasin selle teema
Tomiga vestluse käigus üles ja tegin ära. 
Diplomitöö kirjutamise käigus sain veel ühe asjaga hakkama. TTÜs kasutati 
SETORi\sidenote{Varastel kaheksakümnendatel liikvele läinud TOTALi 
andmebaasisüsteemi kloon ESide jaoks.} ehk andmebaasi, mida ülejäänud 
maailm tunneb nimega Total\sidenote{Ka TOTAL. 1968. aastal asutatud Cincom 
Systems Inci andmebaasimootor, mis oli esimene omasuguste seas.}. Arvutitel 
oli mälu vähe, 256 KB, millest 16 kilo jäi puhvrisse, kui kuvarid taga 
olid, ja nüüd ei mahtunud enam kompilaatorid ja linkurid mällu ära. Diplomitöö 
käigus kirjutasin skripti, mis vaatas programmis järele, milliseid teeke 
vaja on, ja linkis ainult need teegi osad sinna külge, mida tõepoolest vaja 
oli. Sedasi oli võimalik 16 KBga hakkama saada. Veel mõtlesin välja puhverdamissüsteemi, kuidas läbi puhvri erinevaid mooduleid siduda, sest 
suur tükk ei mahtunud korraga mällu. 

Tom pani kokku grupi, kuhu kuulusin mina, Mart Roost\index[ppl]{Roost, Mart} 
(praegu tunnustatud õppejõud), Lea 
Elmik\index[ppl]{Elmik, Lea} ja Tiiu Lumberg\index[ppl]{Lumberg, Tiiu}. Meid hakati kutsuma \enquote{Mikli noorteks 
ekstremistideks}. Me kõik kirjutasime oma teadustööd, aga me ei teinud 
kunagi midagi nii nagu teised. 

Moskvas oli üks kaval juut Tjomov, kes istus kuskil instituudis 
Iskra-226\index{Iskra!Iskra-226} peal, mis oli laetava BASICuga arvuti, ja
kirjutas opsüsteemi Skoropis ehk kiirkiri. See oli esimene viitadega keel, 
mida ma nägin. Tal oli \emph{time sharing} ilusti sisse ehitatud. Programmi täitmine 
käis nii, et tõmbasid programmi stringi, panid viida peale ja ütlesid, et 
selle viida järgi hakkad nüüd täitma. Mälu oli jälle vähe, 64 KB, aga meil 
tekkis Lindre Reinuga\index[ppl]{Lindre, Rein}, kellega me 
arvutisaalis tuttavaks saime, mõte panna Iskrale veel kaks kuvarit külge. Mina 
kirjutasin opsüsteemi ringi, tema tegi kaks videokaarti, panime Videotoni kuvarid 
taha ja vaatasime, kas hakkab tööle. Selleks ma tegin 
\emph{overlapping}'u -- kui tundsin ära, et programm on 
juba mälus, siis lisasin teise viida ja panin selle veel kord tööle. Programm visati 
välja alles siis, kui viitasid enam ei olnud. 

\question{Kas mälukaitse või turve ei olnud probleem?}

Ei. Kogu infoturve seisnes selles, et masinat ei saanud käimagi, flopi oli välja 
võetud ja tuba käis lukku. Kuigi oli olemas ka kahemegane ketas, mis nägi välja nagu
suur valge \emph{baraban} -- mul on praegugi kapi otsas kaks tükki, üks 
Iskra ja teine SMi oma.

Mul on seal kapi otsas terve muuseum, näiteks lint ja kolmesajane 
modem (nimega Nightingale, laksutab nagu ööbik) ja üks 
esimesi läpakaid, mis Eestisse tuli ja mis oli Siim Kallase\index[ppl]{Kallas, Siim} oma, 
kui ta oli Eesti Panga president. Lisaks arvelaud, lükati, kaheksa-, viie- ja kolmetollised 
flopid, magnetoptilised kettad -- 
ühesõnaga kõik, mis elus ette on jäänud. Kõige vanem eksemplar, taskukalkulaator, on pärit 
aastast 1936.

\question{Kas Feliks?\sidenote{Nõukogude Liidus aastatel 1920--1970 toodetud 
mehaaniliste kalkulaatorite sari, mille tootmise algatas Nõukogude 
julgeolekuteenistuse asutaja Feliks Edmundovit{\v s} Dzer{\v z}inski, mistõttu laienes 
tema hüüdnimi Raudne Feliks ka kalkulaatoritele.}}

Feliks on ka. Aga see kõige vanem on sakslaste tehtud mehaaniline numbrinäiduga taskukalkulaator, mis liidab ja 
lahutab ning on umbes 3 mm paks ja 6 x 10 cm suur. Vanaisa kinkis selle isale kuuendaks 
sünnipäevaks ja isa kinkis mulle. 

\question{Kas sa tehnikaülikoolis teadust ka tegid?}

Jaa, ma hakkasin tegelema andmeedastusega, aga tulid segased ajad, raha sai 
otsa ja see jäi seisma. Tegelesin sünkronisatsioonimudelitega, millega olen 
elus hiljemgi väga palju tegelenud, ja praegu võiks nendest kirjutada sellise
töö, mida keegi pole kunagi välja mõelnud. Aga nüüd on mul muud huvid tekkinud\ldots 

\question{Mis on sünkronisatsioonimudelid?}

Nende põhimõte seisneb selles, et kui on kaks infosüsteemi, siis millist mudelit 
kasutada, et kõige odavamalt välja tulla, ja mismoodi see automaatselt käima 
saada, et nad süngis oleksid. Tollal ma mõtlesin välja ühe termini, mida on hakatud minu suureks rõõmuks tänapäeval kasutama -- 
\enquote{automaagiline}. Kasutasin seda kunagi ühel konverentsil ja Jaak Tepandi\index[ppl]{Tepandi, Jaak} küsis, mida ma selle all mõtlen. See on asi, mis muutub automaatselt, aga ma ei 
tea täpselt, mismoodi. Ja nüüd kasutatakse seda reklaamideski. 

Meil tekkis 
Reinuga\index[ppl]{Lindre, Rein} oma rühm, sest Žiguli autotehas AutoVAZ soovis meilt süsteemi väikejaamade jaoks -- 
ladu, remont ja muu säärane. Ütlesime, et teeme küll, aga omamoodi, 
meil peab teadus sees olema. Kirjutasimegi 
neljakesi nullist süsteemi, mille loogikat poleks tänagi häbi näidata. Mart\index[ppl]{Roost, Mart} kirjutas 
andmebaasi mootori, Tiiu\index[ppl]{Lumberg, Tiiu} vormi generaatori, Lea\index[ppl]{Elmik, Lea} raporti generaatori ja 
mina süsteemi arhitektuuri kirjelduse ning mõtlesin välja ka
tolle aja mõistes XMLi. Suurem-väiksem märgi asemel olid kandilised sulud ja 
\emph{slash}'i asemel sõna \enquote{END}, aga keel oli sama. 
Tõestus on olemas ühes TPI kogumikus, kuhu ma kirjutasin selle kohta artikli. 

Mul oli väga lihtne tõsta asjad seal keeles ringi ja süsteem hakkaski 
teistmoodi menüüsid ehitama ning igasuguseid küsimusi küsima.

\question{Kas nad võtsid süsteemi kasutusele ka?}

Jah, me kasutasime seda AutoVAZi jaoks ja hiljem mujalgi.

\question{Kas ühel hetkel sukeldusid pangandusse?}

Selleni läks aega, enne toimus see
maailmameistrivõistlus. 

Ühel hetkel sai raha otsa ja palka sai 
TPIst\index{Tallinna Tehnikaülikool} nii palju, et kui auto oli olemas, 
jaksasid autoga tööl käimiseks bensiini osta. Sain tuttavaks niisuguse huvitava mehega nagu Veiko 
Herne\index[ppl]{Herne, Veiko}, kes praegu elab Euroopas nii-öelda kodutuna. Ta tahabki seda ja see 
ei tähenda, et ta halvasti elaks, vaid ta rändab ringi. Tema eluunistus oli olla vaba. Ja nüüd ta kirjutabki mobiiliäppe pargis. Kui kuskil on 
põllumajandusperiood, siis läheb põllumajandusse tööle ja aeg-ajalt paneb 
feissarisse, kus ta käinud on. Välimuselt on ta minu täielik vastand: 
pisike, kõhetu ja ümmarguste prillidega. Geniaalne programmeerija ja orgunnimeister. Ta kutsus mind tarkvara tegema oma loodud firmasse
OÜ Tarkvara ja andis kolmandiku osakuid mulle.

\question{Kui OÜ, siis pidi ajahetk olema 1991.}

Millalgi siis jah. Ühel päeval ütles ta: \enquote{Kuule, hakkame tõsiselt 
tegema -- ma leidsin Microsoft Magazini sabast ühe süsteemi nimega 
Gupta\index{Gupta} SQLBase\sidenote{Tegu oli esimese relatsioonilise kliendi-serveri 
andmebaasiga, mis jooksis PC platvormil, mitte mini{\-}arvutitel.}. 
Nad pakuvad, et hakkaksime nende esindajaks, ma käin korra Inglismaal.} Mina 
olin TPIst selleks ajaks otsad juba lahti võtnud. Üks asi, millega me raha 
teenisime, oli Robotroni nõelprinterite ümberprogrammeerimine eesti 
tähestiku peale. Sellest tekkis natukene 
algkapitali. Inglismaa-sõidu ja lansseerimise 
peale läks ilge raha, 100 000 rubla, aga kuidagi me selle kokku 
kraapisime. Igal juhul Veiks tuli Inglismaalt tagasi ja olimegi esimese kliendi-serveri süsteemi ametlikud esindajad Eestis. Siis ei 
olnud veel Oracle'it, Cybase'i ega kedagi. Korraldasime seminari ja tuli ainult 
vilistada -- terve Küberi amfiteater oli inimesti puupüsti täis. 

Tegime lepingu Põlva Piimaga ja Võrus oli 
eksperimentaalne õmblustootmiskoondis, kellega sõlmisime süsteemide 
arenduslepingud. Uurisime süsteemid välja ja panime andmebaasid käima. Põlva 
Piim oli väga suur projekt, seda me ei hallanud enam kolmekesi ära, nii et võtsime 
Andres Lombi\index[ppl]{Lomp, Andres} ja IE-tarkvara\index{IE-Tarkvara} appi 
programmeerima. 

Mul on õudselt hea nina igasuguste vigade peale. Leidsin SQLBase'ist ühe laheda vea, et kui tingimused
\verb|IN| ja \verb|NOT IN| olid järjest, siis täitusid suvalised 
tingimused. Ja kui panid sinna vahele \verb|1=1 AND|, siis hakkas tööle. 
Vennad ei uskunud seda ja kaks tükki tulid suisa kohale. Korraldasime ruttu seminari ja panime nad esinema. Näitasin neile enda tehtud asju ja kuidas me oleme nende 
süsteemi kasutanud. Saime nendega täitsa \enquote{kuuma liini}. Ükspäev teatas Veiko\index[ppl]{Herne, Veiko}, et 
Gupta\index{Gupta} otsib endale esindajat 
maailmameistrivõistlustele 4GL programmeerimises ja kas lähme. Guptast öeldi, et te 
olete nii kõvad vennad küll, minge, aga ise peate oma arvutitega Rootsi jõudma. 

Oli selline väga tark soome poiss nagu Pauli Visuri\index[ppl]{Visuri, 
Pauli}, kes müüs Olivettisid. Tänu talle tõi
üks Rootsi Olivetti esindaja meile messiboksi tuttuued masinad, meie 
lihtsalt sõitsime lennukiga kohale. Tahtsime ööbida mingis tagasihoidlikus 
kohakeses, aga Gupta ütles, et ei, meie meeskond ööbib ainult Kung 
Carlis, maksame selle teile kinni. 

Läksime sinna ja nägime esimest korda elus 66 MHz Suprema masinaid, millel oli peal Plug-n-Play, Windows 3.11. Hakkasime installima, aga ei õnnestunud, hiir ei läinud külge. Arvasin, 
et seal on Microsofti meeskond, panin käed puusa ja teatasin: 
\enquote{See teie opsüsteem on igavene pask! \emph{Plug and play} küll, aga hiired külge ei 
lähe!} Kaks venda istusid meie masina taha ja hea oli vaadata, kuidas 
ini-failid\sidenote{.INI laiendiga failides hoiti Windowsi platvormil 
tavakohaselt programmide konfiguratsiooni.} lendasid näppude alt 
välja. Lasid-lasid ja üks masin läks käima. Ajasin nad minema, 
kopeerisin ini-failid kõikidesse masinatesse ja oligi korras. Veiko hakkas 
proovima häältuvastust, mis oli just välja tulnud, aga kuna messihalli helifoon oli 
väga kõva, siis ta karjus oma arvuti peale: \enquote{Õupen, õupen, õupen, 
klõus, klõus, klõus, ran!} Järsku kostis teiselt poolt seina: 
\enquote{\emph{Clear all!}} Küllap ta käis kellelegi närvidele. 

\question{Mis ülesannet te lahendasite?}

Ülesanne oli vahva, umbes selline, et kass ärkas, sirutas, hüppas ja sattus klaviatuurile. Arvuti tegi 
piiks, kass tegi näu ja selle peale ärkas üles tema perenaine Celia, kes 
mõtles, et täna on kolmapäev -- mida ma olen tellinud, mis kaubad peaksid täna 
tulema ja mis mul veel tellida oleks vaja? Siis räägiti Peterist, kes istub 
kesklaos ja paneb kaupu liini peale, ning Larryst, kes sõidab 
\emph{lorry}'ga ringi ja veab kaupu laiali. Klassikaline 
veebikaubanduse logistika, mida tollal ajal veel ei olnud. Meile anti 
ette kaart ja GPS-signaal ning pidime programmeerime auto armatuurlaua 
koos GPSi liigutamisega ja tsentrumi. Meie lahendus erines teiste omast, sest ma ei 
viitsinud seda igavat ladu programmeerida ja tegin keskele 
logistikakeskuse, kus laod olid eraldi. Ladusid imiteerisime omaette failidest 
ja Peter võttis lihtsalt tellimusi vastu ja jagas laiali. 
Pärast messi lõpetamisel istusime žüriiga ühes lauas ja nad ütlesid, et kurat, mingid postsotsialistlikud vennad tulevad meile kapitalismi 
õpetama! 

Võistlus kestis 24 tundi: algas ühel päeval kell kolm ja lõppes teisel päeval 
kell kolm, seejärel hakkasid järjest esitlused tulema. Meie saime esitlema 
alles kell kümme õhtul, kui olime 24 tundi üleval olnud. Mina kirjutasin kogu koodi ja 
projekteerisin peas asjad. Veiko, hea suhtleja, süstematiseeris mu küsimused ja tõi 
žürii käest vastused. Tiiu joonistas vorme ja tegeles kogu kasutajaliidesega. 

Alustasin 15:20 ja kell viis öösel läksid näpud krampi -- umbes pool tundi 
ei liikunud, siis läks uuesti lahti. Korraks tekkis psühholoogiline tõrge, aga siis tegime edasi, kaks tundi enne tähtaega saime 
valmis. Hommikul kell kaheksa tekkis uuesti jama tunne, kui üks
Maci meeskond juba lõpetas. Mõtlesin, et olen ikka jube sant mees. Lõpuks 
selgus, et nad katkestasid.

\question{Kes sellist üritust korraldas?}

Täpselt ei mäleta, aga üks rootslaste softiliit. 

Ma teadsin sellest tänu ühele sõbrale, Tartu EPT juhile Kalle Kullmanile\index[ppl]{Kullman, Kalle}, kes oli seal kunagi ülesande tegijana osalenud. Temaga saime 
kokku marksistliku-leninliku kommunismi kandidaadimiinimumi täiendusloengus, kus me istusime kõrvuti. Loengut 
luges Otto Stein\index[ppl]{Stein, Otto}, keda kutsuti Otto 
von Steiniks. Ta oli saadetud Tartust Tallinnasse kommunistlike filosoofide 
kaadri tugevdamiseks, mispeale nii kaader kui seltsimees Stein tugevnesid. Hull vanamees oli.

Mäletan siiamaani, et loeng oli \enquote{Kommunismi on kolm allikat, kolm 
komponenti}. Need on inglise poliitökonoomia, saksa utopism ja \ldots\sidenote[][-.7cm]{Kommunismi kolm allikat 
toonase õppe järgi olid saksa klassikaline filosoofia (peamiselt Georg Wilhelm 
Friedrich Hegeli ja Ludwig Feuerbachi järgi), inglise poliitiline ökonoomia 
(Adam Smith, David Ricardo) ja prantsuse utopistlik sotsialism (Claude Henri de 
Saint-Simon ja Charles Fourier). Neid \enquote{arendasid edasi} 
marksismi-leninismi kolm komponenti: dialektiline ja ajalooline materialism, 
poliitiline ökonoomia ja teaduslik kommunism.}. Stein läks esimese tudengi juurde: 
\enquote{Öelge esimene, nii, õige.} Siis teise juurde: \enquote{Öelge teine.} 
Kolmanda juurde: \enquote{Öelge kolmas.} Ja nii edasi ühe inimese juurest teise juurde: 
\enquote{Teine, esimene, kolmas, teine.} Kui tiir minuni jõudis, tõusin püsti 
ja ütlesin: \enquote{Teate, mina selles tsirkuses ei osale!} ja jalutasin välja. 
Kui Stein püüdis asja leevendada ja ütles Kallele, et no öelge siis 
teie, teatas Kalle: \enquote{Mina ka mitte!} ja tõusis samuti püsti. Läksime 
välja, istusime Tuljaku baari maha, ajasime juttu ja oleme siiamaani suured 
sõbrad.

\question{Kuidas sa ikkagi panka sattusid?}

Olin nii-öelda vabakutseline häkker. 1991. aastal oli 
elutempo selline, et päevarütm oli täiesti sassis võrreldes teistega ja 
tööpäevad olid 72tunnised, pärast mida sõitsin autoga Valka ja Põlvasse asju 
üle andma. Mul oli siis juba kaks last, Anna oli just sündinud, ja selgus, et 
selline töörütm ei klapi enam. Seesama Tiiu\index[ppl]{Lumberg, Tiiu}, 
kellega käisime maailmameistrivõistlustel, rääkis, et 
Innovatsioonipank\index{Innovatsioonipank}\sidenote{18. 
septembril 1989. aastal ENSV Ministrite Nõukogu presiidiumi otsusega number 21 
Eesti NSV riigieelarve \enquote{üle plaani laekunud tulude} arvel asutatud 
pank.} otsib IT-juhti. Seda panka juhtis Peep 
Sillandi\index[ppl]{Sillandi, Peep}, pärastine mikro- ja makroökonoomika 
õppejõud EBSis, kes õpetas tudengeid softi peal mudeleid koostama. 
Peep oli lahe kuju, meil jutt klappis kohe ja lõpuks ta küsis: \enquote{Homme siis tuled 
või? Näe, tool on siin.} Mõtlesin, et olgu peale. 

Ja nii saigi minust IT-juht. Hommikul tuli minna poole üheksaks tööle, harjuda ära
sellega, et ei saa kella kolmeni öösel üleval olla. Ma ei ole sellega siiamaani 
harjunud, lähen üsna tihti praegugi öösel kell kaks magama ja ärkan kell 
seitse. Viis tundi on minu jaoks \emph{enough}. 

\question{Mida kujutas endast 1991. aastal väikepanga IT-juhi töö?}

Igasuguseid asju. Kui esimene päev uksest sisse tulin, istusin maha ja mõtlesin, kuidas mind on võimalik 
vangi panna (ma olen muuseas seda 
pärast igas ettevõttes teinud). 

Hakkasin seda kohta otsima ja leidsingi. Tollal käis keskpangaga 
infovahetus programmiga, mille kirjutas väidetavalt üks
armeenlane, kes oli kõik juuksed peast ajanud, et kammimise peale aega ei 
läheks, ja kuna suhkur on ajutoit, sõi ainult suhkrut -- geniaalne vend! Ta oli 
teinud nii käsuliidese kui ka dialoogiga käiva suhteliselt pisikese programmi, mis natuke 
krüptis ka -- küll väga vähe, aga tolle aja kohta ilmselt kõvasti -- ja saatis maksed 
panka ära. Iga päev tehti pangas kaks faili, üks hommikul ja teine õhtul. 
Õhtuses failis olid hommikuni käibed, mis tuli keskpanka saata, ja teises oli 
teistpidi. Igal pangal oli oma aeg, millal ta pidi failid ära saatma, ja samal 
ajal sai teistest pankadest tulnud asjad vastu. Kogu see asi seisis vabalt. 

Sain kohe aru, et panka saab röövida niimoodi, et ma ei võta kellegi kontolt 
raha ära, vaid tekitan sellesse kanalisse raha juurde. Siis ei hakka keegi 
selle järele igatsema, natuke aega tuleb ainult nostro- ja vostro-kontode sisu 
varjata, et ei oleks näha, et seal on mingi jama tekkinud, aga sellega 
saab hakkama. 

Esimene asjana tegime nii, et sellesse kohta sai faile panna 
ainult üks konkreetne programm, teine sai faile võtta ja kui keegi sellesse 
piirkonda sisse logis, katkestati kõik ära. 

Teine jama oli 
pangakontorite vaheliste ühendustega, näiteks kuidas saada 
Mustamäele kontor püsti nii, et see meie süsteemiga kokku saaks. Ehitasime ja 
testisime mingeid seadmeid. Telefonikanal ju ei 
püsinud.

Juhtus ka triviaalsemaid asju. Ühel päeval tuli teller minu juurde ja ütles: 
\enquote{Kuule, Priit, klient küsib oma käivet sellest ajast, aga seda ei ole.} 
Uurima hakates selgus, et süsteem oli üles ehitatud niimoodi, et kaks aastat vanad 
käibed hävitati ilma küsimata ja pikemat aega panna ei saanudki. 

\question{Mis too panga tuum oli, mis niimoodi tegi?}

See oli Midas Kapiti süsteem Kapiti, mis istus AS/400\index{AS/400} otsas. 
Aga kuna meil AS/400 ei olnud, siis meil käis OS/2 Warp\index{OS/2!OS/2 
Warp}\sidenote{OS/2 oli IBMi arendatud personaalarvutite 
operatsioonisüsteem. OS/2 Warp oli selle kolmas versioon, mis tuli turule 1994. 
aastal.}, mille peal istus AS/400 emulaator ja mille sees käis panga tuum. 

\question{Kas see oli kuskilt ostetud?}

Jaa, Kapiti käest, Midaseks muutus see hiljem. Ja mida 
Priit tegi? Istus maha, poisid hakkasid sortima \emph{backup}'e (neid me 
tegime hoolega) ja kirjutasime sellise softi, mis lappas \emph{backup}'idest SQLBase'i 
peale kokku andmelao. Me siis ei teadnud, et see asi on andmeladu. Kui ma panka läksin, lasin SQLBase'i osta, sest see oli hea 
kliendi-serveri lahendus, lihtsasti kättesaadav ja ei olnud väga kallis võrreldes 
teistega. 

Lisaks olin panga nõukogu sekretär. Peep arvas, et ma oskan 
piisavalt loetavalt kirjutada, pärast puhtaks lüüa ja saan asjast aru 
ka. Ütles veel, et ega sa liige ole, aga arvamuse saad ikka sekka 
öelda. Nii et olin nõukogus hääleõiguseta arvamusliider.

\question{See oli ju IT-juhile väga praktiline koht, said info kätte!}

Just nimelt, see oligi Peebu mõte ja jube hea mõte. Nõukogus 
olid väga vahvad liikmed, kellega ma tuttavaks sain. Näiteks Arvo 
Kallion\index[ppl]{Kallion, Arvo}, omaaegne
parteiboss ja valitsuses keegi, aga väga tark mees. 

Innovatsioonipank oli taskupank. Genin\index[ppl]{Genin, Alex}\sidenote{Alex 
Genin, Innovatsioonipanga nõukogu esimees.} oli niisugune juut Ameerikast, kes 
elas sellest, et tegi erinevates riikides panku, ajas nad riigi süül 
pankrotti ja siis hakkas kahjutasu nõudma. 
Sotsiaalpank\index{Sotsiaalpank} läks pankrotti, sealt pankrotipesast 
ostis ta kõige vingema kontori ning lasi panga põhja. Ma nägin ette, et see läheb nii. Selleks ajaks Peepu enam ei olnud 
sest Genin oli oma Miša (ma ei mäleta Mihhaili perekonnanime) panga etteotsa 
pannud. Tore poiss muidu, aga tegi täpselt, mis Genin ütles. Läksin Miša juurde, 
panin avalduse lauale ja ütlesin, et Miša, ma lähen nüüd ära. 
\enquote{Aga miks sa lähed?}  \enquote{See pank läheb varsti pankrotti.} 
\enquote{Sa eksid!}  \enquote{Ei eksi, Miša, ma olen majandusharidusega ja ma 
näen igal õhtul bilanssi, see pank läheb varsti pankrotti.} 

Mul oli on uus koht olemas, Tööstuspank\index{Tööstuspank}. Üks 
laenujuht läks sinna ja kutsus mind arendustiimi juhiks, kuna neil oli vaja uut 
infosüsteemi. Ma ei saanud aru, mis toimub: sain seal kõik, mida küsisin. Oma kontor ehitati Koplisse koos magamisruumi, köögi ja kõige muuga. Tirisin Innovatsioonipangast Eriku\index[ppl]{Matt, 
Erik}\sidenote{Erik Matt.} ja Raivo\index[ppl]{Tali, Raivo}\sidenote{Raivo 
Tali.} kaasa ja kuskilt tõin ära Ville Remmeri\index[ppl]{Remmer, 
Ville}.

Kuus kuud uurisime ja puurisime, ja kui meil oli kõik valmis, siis öeldi mulle: 
\enquote{Tead, me valetasime sulle. Me ei kutsunudki sind siia uut süsteemi 
tegema. Meil on siin IT-osakond, aga me ei usalda neid.} Seal IT-osakonnas 
oli igasugu tegelasi ja juht oli Poldzadze, kes nägi välja nagu Kirgiisi bai. Aga 
panga juhtkonnal oli vaja inimest, kes teaks, mis panga ITs toimub, sest nad 
hakkasid just Hoiupangaga kokku minema. 

Mulle öeldi: \enquote{Sa võtad nüüd IT juhtimise üle.} Sain endale kõige uhkema ametinimetuse, 
mis mul kunagi olnud on: panga esimehe 
volitatud eriesindaja IT küsimustes. Mul oli õigus käskida, puua ja lasta. 
IT-osakond oli sotsialistlik kamp, kellel oli vaja nimega ülemust. Mul oli õigus 
teha ükskõik mida, peaasi, et pank püsti püsiks ja midagi ära ei läheks. Ja 
ühel päeval läkski kuus ja pool miljonit krooni minema. Täpselt sedasama teed pidi, nagu 
ma arvasin. 

Sellel päeval, kui ma võimu võitsin, ei olnud mul veel võimalik midagi teha. 
Tõnu Liik\index[ppl]{Liik, Tõnu}, kes oli Hoiupanga IT-juht, tuli sinna, Ants 
Leitmäe\index[ppl]{Leitmäe, Ants} kaasas. Ants istus aknalauale ja kuulas 
tuima näoga pealt, tema pidi mind oma tiimiga tehniliselt toetama hakkama, sest ma 
ei teadnud kedagi usaldada. Siis kutsuti kõik kokku ja Tõnu teatas, et 
uus juht on nüüd siin. Ja kõik läks ilusti. 

Esimese asjana sai kõigil 
kasutajaõigused ära võetud, panin oma poisid masinate taha ja hakkasime uuesti 
õigusi õigetesse kohtadesse tagasi andma. Õhtul istusin pearaamatupidaja Irina juures (perekonnanime ei mäleta), kes toodi ka enne liitumist majja, ja ajasime 
juttu. Irina võttis lahti tagasi tulnud hommikuse kontrollsummade faili 
ja tegi istmelt meetrise hüppe üles. Kuus ja pool miljonit oli jagunenud pankade 
vahel teisiti, kui oli hommikul välja läinud -- raha oli läinud Rakvere Maapanka\index{Rakvere Maapank}. Egas midagi, 
joostes minema, kogu IT-osakond puhtaks ja arvutid tuli lahti jätta. Panime oma poisid 
peale ja hakkasime kontonumbri järgi \emph{search}'i tegema. Ja leidsimegi ühest 
masinast, kusjuures avastasime tänu sellele, et võtsin kõigil hommikul 
õigused ära. Tüübil oli kontrollsumma muutus ka tehtud, aga ta ei saanud 
õhtul enam vajalikule kohale ligi. Irina helistas kohe Rakverre, blokkis summa 
ära ja järgmisel päeval saime tagasi, sest õhtul olid pangad juba kella neljast kinni ja pangaautomaati 
ei olnud. Nad olid plaaninud tegutseda järgmisel hommikul. 

Vend võeti kinni ja viidi raudus minema. Masina panime raha transportimise kotti, pitseerisime 
kinni ja lukustasime seifi. Järgmisel päeval tegime manukate juuresolekul lahti, võtsime 
ketta välja ja tegime sellest kolm koopiat. Üks läks TTÜsse analüüsi, teine Hoiupanga 
tiimile ja kolmas minu tiimile. Seal masinas oli peale kontonumbri veel üks 
klimp, mis oli parooliga zipitud. Küsime venna käest: \enquote{Mis \emph{password} on?} -- \enquote{Ma ei 
tea, kui te mulle ütleksite, oleks mul endalgi huvitavaid asju seal sees.} 
Läksime kontorisse, et installida murdmisklaster. Siis läksime 
Raivoga\index[ppl]{Tali, Raivo} suitsu tegema ja tagasi tulles ütles 
Erik\index[ppl]{Matt, Erik}: \enquote{Poisid, vabandust, te oleksite ka 
kindlasti seda näha tahtnud. Ma lasin prooviks klastri käima, leidis, sõnastiku alguses, 
\emph{konjak} väikse tähega\ldots}. Raivo ütles selle peale: \enquote{Jube madal 
profiil, ma oleksin vähemalt Mercedes-Benz pannud.} 

Seal failis oligi kogu värk sees. Süsteem oli Foxis, andmebaas oli DBF, siis 
oli tehtud üks \emph{browse}, mida klikkisid ja mis jäeti meelde, ning lõpuks F10 
vajutades kanti kõik ühe konto peale kokku ja tehti fail valmis. Samuti
kontrollsumma fail, mis oleks õhtul lihtsalt õigesse kohta 
tõstetud. 

\question{Nii et sulle köögi ehitamine tasus kohe esimesel päeval ära?}

Ma ütlen sulle, et siin ei ole midagi oodata. Kõik juhtub kohe.

Hoiupank ostis Eesti Kindlustuse ära ja ma läksin sinna IT-juhiks. Seal oli ITs 
kaks ja pool meest, kellest ühte ma ei näinudki. See oli osakonna juhataja, kellel 
olid mingid probleemid, ja kui ta kuulis, et on uus IT-juht, siis ta ei 
tulnudki enam. Iseenesest geniaalne mees, mitmeid matemaatikaõpikuid 
kirjutanud. Võtsin oma poisid Ville 
Remmeri\index[ppl]{Remmer, Ville}, Erik Matti\index[ppl]{Matt, Erik} ja Raivo 
Tali\index[ppl]{Tali, Raivo} kaasa ja kolistasime sinna. 

Leidsime eest mingi õuduste maa. Katastroofi kuubis. Või okse kolme x-iga -- ma ei tea, kuidas seda kõige paremini
iseloomustada. Näiteks elukindlustus käis niimoodi, et paberi peal korjati dokumendid 
kokku ja need läksid kümne fakiirsisestaja kätte. Neil oli ekraani peal triip, 
kus olid postid vahel, ja triibu vahele sisestasid nad andmed, mille põhjal 
tehti reserviarvutusi ja kõike muud. Neil oli käigus postivõrk -- venelased kutsuvad seda
pastlavõrguks. Kõik raportid pandi posti, 
saadeti Tallinna, siin sisestati ära, trükiti raportid välja, pandi postikotti 
ja saadeti tagasi. Ma panin Ville kiiresti kirjutama lokaalset kasutajaliidest. 
Raivo ja Eerik hakkasid otsima võimalusi, kuidas teha ära side kõikide meie 
kontoritega. Ise kappasin esimestel nädalatel mööda kõiki esindusi, et uurida, mis 
probleemid seal on. Ühe tehnikutest võtsin kaasa, et ta vaataks tehnilist 
poolt. 

See kõik juhtus augustis. Mulle lubati kahe töökoha vahel viis päeva puhkust, 
nii kiire oli asjaga. Uue süsteemi tõmbasime käima jaanuaris. Selleks 
ajaks olid meil kõik ühendused tehtud, uus server olemas, 
soft töötas ja inimesed koolitatud. Iga agent hakkas ise oma asju sisestama.

\question{Mis aastal see oli?}

Umbes 1996. 

Nüüd tuli hakata ka ülejäänud süsteemi uuendama, aga võrk oli kohutav. 
Tellisime võrguehitustööd, ehitasime korraliku serveriruumi, panime seina
tulekindlad materjalid ja väljapoole jahutuse. Serveriruum asus maja keskel kõige paksemate 
müüride vahel, lasin sealt WC ja duširuumid koos torudega välja 
lõhkuda. Ja siis ühel päeval, vist märtsis, juhtus niisugune lugu, et kui 
hakkasime üle viima oma viimast serverit, siis
selgus, et \emph{backup} ei loe ja ketas läks nässu. \emph{Backup} oli meil 
korralikult tehtud, aga majas käis remont ja tolm oli selle ära rikkunud. 
Lindiseadmed olid tol ajal nii lollid, et ei näidanud seda. Ja kes see tol ajal ikka
\emph{backup}'e kontrollis -- kui kahes eksemplaris teha, siis oli ju piisav. Aga nüüd olid mõlemad tuksis. Seal peal olid raamatupidamisandmed ja me olime ju 
börsiettevõte. 

Egas midagi, Priit võttis ketta kaenlasse ja sõitis järgmisel päeval Inglismaale OnTracki, kes taastab kettaid. Väga kihvt firma, kõik maailma 
suured on nende kliendid, kaasa arvatud CIA ja KGB, aga vaevalt, et nad oma 
kettaid taastavad. Jõudsin kohale reedel ja tagasi tulin teisipäeva hommikul 
kahe kassetiga, kus olid kõik andmed peal. Kutid olid selle ajaga kõik 
valmis pannud ja nii kui ma lennukist maha astusin, võeti lindid, taastati 
ära ja asi läks käima.

Siis hakkas Hansapank Hoiupanka ära sööma. Tegelikult alguses oli ühinemine, 
pärast ülevõtmine. Mul ei ole selle kohta paberit, aga seest 
vaadates käis minu arvates asi niimoodi, et kõigepealt kuulutati välja ühinemine ja kui kõik 
oli teada, siis võeti lihtsalt üle. 

Tõnu Liik\index[ppl]{Liik, Tõnu} viis suurema osa ITst 
Hanschmidti\sidenote{Toonane legendaarne Ühispanga juhatuse esimees Ain Hanschmidt.} juurde ehk tollasesse Ühispanka\index{Ühispank}. Me tegime uut vinget 
infosüsteemi ka, aga see ei saanud valmis, sest kindlustus lõpetati ära. 

Elasin Mähel suvilas ja Tõnu helistas mulle ühel laupäeva hommikul. Tundsin Tõnu 
juba Tööstuspanga aegadest, kui tegime koos esimesi kaardiprojekte. 
Sulo Muldia\index[ppl]{Muldia, Sulo} Raepangast\index{Raepank} ja kes meil seal kõik olid, omaaegsed 
karismaatilised kujud erinevatest pankadest. Niisiis, Tõnu helistas mulle: \enquote{Kus sa oled? Ma tulen kohe sinu juurde.} Jõin parajasti aias kohvi ja ei jõudnud hommikumantlitki ära võtta, kui
Tõnu astus juba uksest: \enquote{Nüüd on sihuke värk, et 
ütle jah ja ma lähen kohe ära. Ega ma enne ei lähe ka. Tule, ma annan 
sulle uue kindlustuse, tee see valmis, mis tegemata jäi.} Pakkusin talle kohvi, 
jõime selle ära ja oligi kokku lepitud. Läksin SEBsse\index{SEB} kindlustuse 
arendusjuhiks. Tegime hea mudeliga elukindlustuse, mis oli 
selles mõttes märgiline süsteem, et see on \emph{proof of concept}. 
Mul on nimelt oma andmete modelleerimise teooria ja too süsteem on mudel, mis 
näitab, et see teooria töötab, sest need süsteemi osad, mis me siis tegime, on 
aastast 1999 samad. 

\question{See on kõige parem kvaliteedinäitaja, et asi peab kõigile muutustele 
vastu!}

Tegelikult on üks veel vahvam näide, aga see on varasemast ja ma ei teinud seda 
teadlikult. Aastal 1986 kirjutasin ma ühe palgasüsteemi ja müüsin seda ka 
mõnele, aga siis lõpetasin ära, sest see oli FoxPro ja musta ekraaniga. Ja aastal 1994 
helistati mulle ja öeldi, et \enquote{kuulge, see teie palgasüsteem \ldots} 
\enquote{Mul ei ole ühtegi palgasüsteemi.} -- \enquote{Mäletate, te müüsite selle meile aastal 1986, aga meil nüüd firma nimi muutus ja me ei oska seda ära vahetada. Proovisime
teisi süsteemi ka, aga seal on vead sees, oleme kõik muu suutnud selle järgi 
häälestada.} Sotsialismist tuli süsteem kapitalismi ja elas selle asja üle! 

\question{Mida sa praegu teed?}

SEBs\index{SEB} olin ma 17 aastat. Käisin küll 
vahepeal ära, kui mul oli üks kümnekuune huvitav periood. Tänapäeval nimetatakse 
neid \mbox{startup}'ideks, aga tol ajal me lihtsalt arvasime, et oleks tarvis teha üks 
produkt, mis õnnetuseks sattus IT-mulli lõhkemisega samale ajale ja me ei 
saanud enam riskirahasid peale. Me lõime sellist süsteemi, mis teeb kirjelduste 
pealt suvalisi dialooge, ühesõnaga salvestab andmed andmebaasi. 
Kirjeldused on olemas ja samast kirjeldusest võib teha \emph{voice}'i või 
mobiilirakenduse või ükskõik mida. Oma aja kohta 
oli see kõvasti ajast ees.

Sealt ma tulin tagasi SEBsse ja siis läksin RIAsse.\index{Riigi Infosüsteemi 
Amet}\sidenote{Riigi Infosüsteemi Amet.} RIAs juhtus niisugune lugu, et Katrin 
Reinhold\index[ppl]{Reinhold, Katrin}\sidenote{Priidu kolleeg Riigi 
Infosüsteemi Ametis.} tuli TEHIKusse\sidenote{Tervise ja Heaolu Infosüsteemide Keskus, sotsiaalministeeriumi 
IT-maja.} direktoriks. Katrin hakkas mind aeg-ajalt kutsuma arvamust avaldama 
ja kord, kui me siia alla kööki läksime, küsisin ta käest, et 
Katrin, sa vist lubasid Taimarile\index[ppl]{Peterkop, 
Taimar}\sidenote{Riigi Infosüsteemi Ameti peadirektor ajal, kui Katrin ja Priit 
seal töötasid.}, et sa ei võta kedagi kaasa. \enquote{Jah, ma lubasin.} -- 
\enquote{Aga kui ma ise küsin?} Ja ta vastas nagu Teele: \enquote{Ma 
mõtlesin, et sa ei küsigi!} See oli elu kõige lühem tööle värbamise vestlus. 

Tööle asudes oli meil analüüsi osakond, aga nüüd on selle 
nimi andmekorralduse ja andmeanalüüsi osakond. Selle all on kaks talitust. Üks 
on andmekorralduse talitus, kes teeb HL7\sidenote{Meditsiinis laialt 
kasutatav andmestandard.} standardi peale andmevorminguid, millega kogu 
tervishoiu infovahetus Eestis käib. Meil on väga hea ja detailse mõtlemisega arhitekt Andrus 
Tamboom\index[ppl]{Tamboom, Andrus}, kellele mina 
käin algul mõne asja välja ja tema mõtleb edasi ning joonistab lõpuni. Kahe analüütikuga ehitan üles 
analüüsikeskkonda ja palkan ka parasjagu andmelao talituse juhti, et kogu 
andmelaondus korralikult üles ehitada. 

\question{Järeldan kõige selle põhjal, et sul 
läheb hästi.}

Ei mul pole häda, väga huvitav on! Tegelen 
selliste asjadega, millega ma varem pole otseselt kokku puutunud, aga see valdkond muutub ka 
sellise kiirusega, et teinekord on raske kannul püsida. Kogu aeg 
lappan kohutavat kogust materjale läbi, kas on midagi uut. 

\question{Sul on vist kogu aeg nii olnud?}

Just nimelt, ma ei ole kunagi töötanud sellise ameti peal, kus mulle ei ole 
meeldinud. See on kõige olulisem. 