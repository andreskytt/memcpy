\index[ppl]{Martens, Tarvi}

\question{Kuidas sina said arvutite juurde ja arvutid sinu juurde?}

Ma olen pärit tegelikult Pärnust ja seal nagu arvuteid ei olnud minu arust. Aga ma olin sihuke olümpiaadidel käia, et matemaatika ei olnud minu jaoks mingi mingi teema, eks ole. Hakkas kunagi sealt peale umbes, kus ma viiendas klassis võitsin kuuenda klassi linnuolümpiaadi ära matemaatikas. Kõik olid suhteliselt jahtunud selle peale. Ja mingi riikliku olümpiaadi käigus siis veeti meid ekskursioonile Nõo Keskkooli\index{Koolid!Nõo Keskkool}. Seal oli suur arvuti olemas. See oli ju kuidagi nagu teistsugune maailm ja ühesõnaga tassiti mind päeva lõpuks Nõo Keskkooli. Ma ei tahtnud sel hetkel väga minna enam, sest mul juba oma bänd ja nii.

\question{Sul oli oma bänd?}

Ja noh, punki nagu ikka tehti sel ajal. Ma käisin muusikakallakuga koolis, see oli elementaarne, et sul on bänd. Siis kooliteater tegi oma esimesi samme, mis algaski Pärnust. Kadunud Aare Laanemets\index[ppl]{Laanemets, Aare} ja Elmar Trink\index[ppl]{Trink, Elmar} tegid seal esimest kooliteatrit, kus ma olin osaline. Kõik see oli nii tore ja ma mõtlesin, et mina ei viitsi küll kuhugi kaugele kooli minna. Aga matemaatikaõpetaja käis mu vanemate juures ja rääkis nad tursaks ja nii läkski. 

\question{Kas sel ajal Nõo legend alles kujunes või oli see juba tuntud paik?}

Jah, oli kindlasti. No olid teised tugevad koolid ka siin-seal, eks ole, Tartus-Tallinnas,  aga põhimõtteliselt Nõo kool oli jah üle kõige. Põhiliselt sellepärast, et neil oli oma arvutuskeskus ehitatud ja sinna  tuldi ikka üle vabariigi kokku. Kuigi peab ütlema, et enamus olid niisugused ümberkaudsed inimesed, kes ei olnud võib-olla väga suured geeniused, et pigem pigem maa-lapsed. 

Nõo Keskkoolis oli meil küll Nairi 3-1\index{Arvutid!Nairi-3-1}, niisugune \emph{mainframe}, kus said perfolinti sisse sööta ja laiprinterist oma tulemuse kättesaada. Aga see ei olnud nagu väga tore. Ma ei suuda meenutada, mis ajenditel ma leidsin Tartust ülikooli Vanemuise õppehoonest\index{Tartu Ülikool!Vanemuise tänava õppehoone}  keldrikorruselt üles kabineti, kus oli nii-öelda kaks koma viis Apple II-te\index{Arvutid!Apple II}. Kaks koma viis sellepärast, et üks oli katki kogu aeg, siis Andres Peiker\index[ppl]{Peiker, Andres} (kes oli nagu selle selle keldri kunn) remontis teda.

Mina olin üheksanda klassi poiss ja konkureerisin  arvutiaja pärast tõeliste üliõpilastega nagu Tanel Tammet\index[ppl]{Tammet, Tanel} ja Margus Liiv\index[ppl]{Liiv, Margus} ja niisugused mehed. Aga eks ma sain ikka vahele ka niimoodi ja, ja, ja siis oligi, et ma ikkagi enamus ajast ei käinud väga palju koolis, vaid ronin rohkem seal seda Tartus ja ometi nii Nõo koolis oli ju Nõo kool oli disainitud selle jaoks, et sinusuguseid inimesi hoida nagu arvuti juures ja harida ja süvendatult ja surelik ka oli vaja veel sügavamale minna. No aga mis sa seal lairi juures hoia, tonni perfolinti? Säästsin, asi on ju see? Ei, see ei ole väga efektiivne. Aga no saatuse vingerpuss oli see, et aasta hiljem ehk siis kümnendas klassis saabus too koolihunnik Catherine uue ja, ja, ja, ja mis olid siis happel kahe kloonid põhimõtteliselt ainult või värvilise, on see, et sina oled samamoodi selle kohta, mitte hat nõude võiks olla vene keeles, vaid ütles samamoodi nagu mina ütlesin kunagi vanasti, aga ma võin igatpidi öelda, ma lihtsalt see on vene keele sõnakott eesti keeles, aga kuna see on Venemaa on tehtud, siis ta võiks olla kaid. No vot. Ja siis, ja kõige naljakam oli see, et siis kohalikud arvutiõpetajaid ja niimoodi, ega nemad ei teadnud nendest Jumine. Ja siis tuli välja, oleks võinud jah, ja, ja siis tuli välja, et on, üks tarvi on ja kes tunneb seda protsessorit nagu läbi ja lõhki, eks ole seal küll oma operatsioonisüsteemi ja omab vene keeles programmeerimiskeeled, aga vahet polnud, on ju. Nii, et et noh, ühel hetkel mul oma arvutuskeskuses oma oma kabinetti oma arvuti, nii et olen ja seda siis puhtalt selle pealt, et sa oled käinud Tartus Apple'i kahte uurimistäpselt, nii said nii palju nagu nukkasin. Et ametite ja muude sarhiivide vahele, et said piisavalt palju teadmist, et naas olla kunn. Täpselt nii läks, õpetasin õpetajaid pärast, nii et see oli väga lahe. See oli siis Apple kahe kloon nagu, kuni nagu emaplaadi disainid on üks taluarhitektuurile välja, nad on vähemalt. Protsessori mõttes oli kindlasti ka mina väga suur riistvara inimene ei ole, et kuigi ma Assengeris programmeerisin vabatsel ajal ei, ei mingit kontrolleritega ja mäe, ma arvan, et oli ikka üsna täpne kloon, aga veel kord oli värviline võrreldes kahega, nii see tähendab, pilt virvendas kogu aeg ees. Väga halb kvaliteet, värviline, see oli üldse üks üsna õudne aparaat ja kui just mehaanilise disaini mõttes looja
Ja siis oli kabinet seal ja siis oli vist päris uhke tunne või? Ei no mis seal ikka, noh tasapisi sai oma oma asja ajada, ei pidanud Tartu vahet enam käima, siis see oli nagu hea ja see kuidagi õppimist jaganud, segama ei hakanud. Mul ei ole sellega kunagi mingeid probleeme. Tuleb käia lihtsalt oma kontrolltööd eksamid ära teha ja siis keegi müriseb.
See terve suhtumine haridusse.
Lood ja Aassoon sealmail peal olid, mida last lastele õpetati, ei olnud, nagu seal oli tõsiste inimeste keeled, nagu algul onju, võis kirjutada. Aga lastele õpetati programmeerimiskeeli nimega on raps ja kõps. Need olid, millisel teadlaste väljamõeldud, mingisugused asjad nagu hea, kellegi väljamõeldud, niisugused eestikeelsed ja eestikeelsed ja ja, ja, ja siis krõps oli minu arust selline, mis sai, sai programmeerida. Noh, ütleme siis joonistamist, et, et kuidas, kuidas.
Blocter näiteks on ju liigub, et meile üles või alla meil paremale jätta joone rajad näiteks raps oli päris programmeerimiskeel. Ja siis ma tegin need keeled ka halli peale ringis said lapsed seal teha, ei pidanud Mairi peal tegema, samad keeled olid mul praegu, jäi kuskile auk vahele, et see, et see on nagu matemaatika tuli lihtsasti. Sest seda on mõista noh, matemaatikahuvi ja matemaatika. Kui aga kustkohast see matemaatikahuvi läks niisuguseks arvuti huviks üle, et see näost käisid Tartus Apple'i kahe juures, mis on selle asja juures nagu niimoodi tõmbes, see on päris äge tõmme olema.
See on hea küsimus, aga mul ei ole head vastust selles mõttes oli, oli, oli ikkagi noh, selgelt tundi ta midagi teistmoodi täiesti on seal nii-öelda Praktiline matemaatika kui tavaline praktiline niisugune rehkendus massi, mis on ikka natukene kalkulaatorist natukene natukene intelligentsem. Ja, ja noh, selles mõttes oli noh, ma ei tea, ma vist juba sel hetkel mõtlesin, et see on paratamatu tulevik. Et teistmoodi, see lihtsalt lihtsalt kainelt mõeldes ei saagi olla. Huvitav, et minu jaoks näiteks see side, et arvuti on üldse kuidagi matemaatikaga seotud, tekkis nagu hoopis-hoopis kuidagi nagu hiljem, aga sinul on see kohene valgusest, sa vaatasid ka seda kui kalkulaator, matemaatiline loogika on üks niisugune minu lemmikdistsipliine kogu aeg olnud ja arvutid on ja muusika muuhulgas on väga loogilised asjad ja nii on. Nii, on, ja siis see sind tõmbas sinna arvuti ligi ja siis sa lõpetasid keskkooli ära ja õps keskkooli ära.
Läksin tippi ja veel mästipleksin kartuli ju sealt tuttav inimene seal juba.
Me teame, mulle tundus nagu tipp oli natukene praktilisema hoiakuga ja, ja ütleme sel ajal ülikooli informaatika oli sihukene distsipliin, millest räägiti, et noh, joonistatakse ikka tahvli peale rohkem. Ja, ja päris matemaatikuks ma kindlasti tahtnud saalist, mis aeg see oli, mis asi see ligi kaheksakümmend seitse. Aga, ja siis võis olla küll jah-jah-jah. Ja, ja tegelikult ma olin Tallinna vahet ennem käinud, oli selline üritus nagu Õpilaste Teaduslik Ühing ja, ja seal oli Peeter Loorents tegi nii-öelda matemaatikas sektsioonis, ma käisin Lorentzi juures aeg-ajalt ja ta andis mulle niisuguseid kaelamurdvaid ülesandeid ja seisma. Siis ma neid lahendusi seal murdsin end ülesannete kaelapaela võtsin ära ja see oli, see oli ka nagu huvitav, aga niisugune kahekordsete intervallidega seesugune elu. Nii et ta oli kuidagi loogiline tundesse tippi minna ja mida sa õppima läksin? Ega ma täpselt ei oska öelda, näeksid automaatikateaduskond oli ja, ja eriala kokku kakskümmend õlil, lii Elli, nii arvutid ja arvutitehnika, midagi niukest, midagi niisugust ja kas jalutus kohe nagu mitu asja. Kõigepealt mõtlesin esimeses programmeerimist tunnised, siia tundi rohkem ei tule. Mille peale õppejõud solvusin? Ei, ei solvunud, on ta hea inimene, oli see absoluutselt? Sest ma sissejuhatavast tunnist kirjutasin mingi selline proge valmis salaja, et näitasin õpetatud nii ja, ja teine asi oli see, et oli hiljuti leiutatud kooli arvuti Juku Oojaa. Seda tegi siis Teaduste Akadeemia Küberneetika Instituudi erikonstrueerimisbüroo juhtimissüsteemide osakond.
Kus ma põhimõtteliselt septembri esimesel nädalal aga samas Küberneetika majas, kus ma olin nagu käinud juba ja ja sadasin sinna sisse, ütlesin, et niuke lugu mul mul jäi nagu mure noh, nende õpilaste keskkondade pärast, et kui tuleb kooli arvuti, siis võiks olla ka näiteks õpilastele mõeldud programmeerimiskeeled ja nii-öelda ropsiportimine, Jukul oli teil tegemata. Läheksin sinna, hakkasin niukest juttu rääkima, et oleks vaja vaja. Leidsin mingid inimesed ja hakkasin rääkima, et, et, et on vajalikest Jukule arendusi teha vastavasuunaliselt ja siis ma ei tea, siis ma lubasid mind seal hängida ja kuskil nelja kuu pärast olin tööl, nii et kuidagi nii ta läks. Siis ülikool jäi? Ei no eks vist ikka ei, ma käisin ikka eksamit tegemas, korralikult. Ikka jõudsid eksamile. No ei, see vahepeal vahepeal käisin peale esimest Kursk või siin vene kroonus ka ikkagi mulin viimane lend, kes sai minna, ma olen väga õnnelik, kuusid viidi Leningradi lähistele, aga olin selle puhkpilliorkester ja tegelikult tegin bändi jälle ja polnud häda midagi. Jälle üks kogemus juures. Ja, ja sealt tulles noh, toonust tulles paljud langevad välja ülikoolist, sest nad leiavad, et võiks midagi praktilist teha ja see ennast targaks ajamine ei tasu ära.
Ja eks mul ka oli mingi kriis, aga kui ma teise kursuse poole peal mõtlesin, et nüüd on kõik kohal käimata ja värk, et kui ma neid eksameid ära ei tee, siis ma siis on kõik. Aga tegin ära ja siis ma võtsingi selle elustiili, et, et ma pühendasin ülikoolile umbes kolm nädalat poole aasta kohta selleks, et imendada materjal sisse, teha eksamid ära ja ja kõik nagu see sees ja see töötas, see töötas täiesti. Sest laval minu puhul küll oli nii, et keskkool möödus mängides ja lauldes, sellepärast et kõik oli väga lihtne, aga siis kui läksin ülikooli, siis selgus, et see kõik lihtsus lõppes ära, aga siin pole lõppenud. Egiptuses tõesti lõppes. Õigemini keerukus oli esimene poolteist aastat või kaks aastat, siis kui nad sulle fundamentaalsete kõrgemat füüsikat ja matemaatikat toovad, mis lihtsalt lõpp kaane pealt ära lihtsalt. Aga edasi läheb natuke erialaseks ja inimlikumaks ja, ja noh, ei ole see asi nii teoreetiliselt, et tappev enam, et et siis on nagu natukene lihtsam. Älenda tegelesid Juku tegelikult ei, mis Juku ja ka siis, kui ma kroonust tulin, oli juba esimene kaks kaheksa, kuus kontorisse toodud ja, ja siis läks juba sedapidi ja sel ajal oli niisugust huvitavad ajad, et töös oli sel hetkel vähe, on ju, käisid seal tööl küll, aga tööd oli vähe ja see oli väga soositud suhtumine, kui sa leidsid endale haltuuraotsi teha. Ja, ja see oli täiesti okei asjad kõige jämedam haltuura ots, mida mäletan, oli see, et tuldi koos arvutiga, öeldi, et näed, sul on siin nagu personaalarvuti selle haltuura tegemiseks mulle tööandja eraldas kabineti selle jaoks väga okei, kes, kes need haltuurapakkujad olid siis? No ei, igasugused toomine nõuab. Antud juhtumil me räägime Soome laevaehitaja eest, kes arvutiga tuli, ta tahtis ja meil on siiamaani olnud kõige vingemaks programmiks mis ma teinud olen, et noh, ülesanne oli selline, et kujuta ette, et sul on sõjalaev, kämme, tekki nii oli, niisiis meil on ja me räägime nagu elektrivarustusest, meil on kuskil veel jõuallikad. Generaatorid on ja kuskil on tarbijad. Ja nüüd me hakkame nende asjade vahele kaableid vedama erineva jämedusega kaableid, ega abi, rennison olemuseni. Ja ühel hetkel saab kaablisein täis. Nii, mis me siis teeme, voki? Veame teistpidi. Aga kes ütleb, et kaablikulu on selle juures kõige optimaalsem? Oh Jumal, pane pott, aga see oli nõukogude aegne veel ei olnud, äi on siis oli juba sulamas mingi, ma ütlen, see oli peale kroonu, et oli siis olema mingi kaheksakümne võib-olla üheksakümmend, üheksakümmend üks niisugune hello shell, sel ajal ju inimesed on rääkinud, kuidas isegi mitte arvuteid ei tohtinud nõukogu Nõukogude Liitu tuua, sinna arvutasid sõjalaevade mingeid kaableid ja mis see on, mis siis on sellest, kes seda teadis, sellest teisi see soomlane võis küll kinni minna ja minu arust oli juba siuke sulaaeg täiesti, kui ka päris iseseisvusmanifest toona näitama, kuidas saavad Tura pakkuja, oskasime minu juurde tulla, välja tuli selle küber juurde ja siis sealt anti sulle otsa. Ütleme siis niimoodi, et see õppejõud, kellele ma esimesel tunnis ütlesin, et et, et ma rohkem ei käi sinu juures, tema leidis, pole neid otsigi muuhulgas kes õpivad õli, mis ei ole oluline. Ja see, ma arvan, et see ei olegi tähtis. Ja, ja, ja noh, ikka inimesed teadsid, mingid siuksed oskaksid soovitada. Et need olid väga, erib palgelisi asju, mis, mis sai tehtud ja ülikooli ajal just panin siis hirmus progeja. Ma kirjutasin oma andmebaasi süsteemi, mis oli fox proost kordades kiirem. Muuhulgas vaat vanasti oli niimoodi, et kõvaketta poole pöördumine oli sihukene, rant tegevusel, see võttis aega see siiamaani, mida saht ja kõik need mahud ja kiirused, et need on kriitilised. Praegu SSD ka nagu ei räägi nagu sellest aga sel hetkel oli see kõige aeglasem tegevus. Ja siis ma kirjutasin andmebaasi süsteemi, millel ei olnud fikseeritud pikkusega väljad, vaid olid noh, sujuva pikkusega väljade, koos isa ja mis tähendab seda, et andmeid oli ketta peal täpselt nii palju kui andmeid oli, mitte et sul oli eraldatud, mingisugune megabaiti sunnib no siis tähendab seda, et keskmise andmebaasina tõmbasin umbes kaheksa korda kokku. Ja, ja vastavate täidaks selle töötlemiskiirus nagu üles selle väljamõtlemine, et kuidas sa seal neid kirjeid, mis siis saab, kui välja tikkus, muud muutuda, kuna pakid seda, see ei olnud ju lihtne. No miks see poleks lihtne mis geniaalse programmeerijale matemaatikud ära ja läks siis välja rehkendatud nagu, nagu nagu sõjalaevade kaalutud graafi, eks ole, kuidas toimub siis siis siis kõige optimaalsem Fabia kulu on. Aga see toredad asjad selles mõttes juba, läheb ju nagu teadusesse ka tol ajal maailmaski ei olnud neid andmebaasi nagu väga palju. Noh, teadlaseks ei tahtnud saada. Ei, mul mul meeldis praktiline pool kogu aeg ja et, et ma pika hambaga läksin magistrantuuri ka ja siis virelesin seal kuus aastat umbes ja siis hakkasid ainepunktide ära kustunud, olin jõuga tegin lõputöö. Et mul pole see nagu re, noh kui teooria ei, ei paku eriti, et mulle meeldib nagu maailmamurdmaailm on muidugi. Eks me kõik soovime, aga väga harva on inimesed julgust seda nagu tunnistada, et nii on. Ega ma su juttu siin kuulan, siis see kuulus siis või ei olnud, kus siis olid osati niimoodi tulla. No veel kord, et, et eks mingi ühe või teise tehtu renomee käis kuidagi ringi ja ja veel kord mul oli Peeter Loorents, mul oli see õppejõud on ju?
Siit-sealt linnapead ikka tutvuste kaudu või kuidagigi sidemete kaudu keissi värk, aga noh, see jõud, massiline ei noh, ma räägin nüüd me räägime ikka mingiseks. Projektidest ma ei tea, kümmekond või niimoodi kogebina. Suured on jah, päris suured asjad siis, eks ole. No ikka sai tehtud jah, no tööasju oli ka loomulikult aeg-ajalt, aga tööd oli sel hetkel veel kord vähe ja, ja poliitik oli niimoodi, et parem inimene olgu olemas ja, ja valmis, kui töö tule panen, siis, siis saab ometi teha see see kontor, mis on tänase nimega Ekdaku, onju selle konstrueerimisbüroo järeltulija. See oli omal hetkel oli fantastiline koht, sest seal oligi, see oli umbes viiskümmend inimest, tehti riist, taht, Eesti tarkvara, umbes fifty-fifty niimoodi. Ja noh, üks asi oli muidugi Juku on ja mis ma mainisin, mis oli nende tehtud. Muuhulgas Elleri poppi tegi esimese hiire valmis maailmas, mis oli tehtud sellest ehtekarbist Elleri pappi siis täisnimega ta praeguseks kadunud? Jah, jälle keldri. Ja, ja seal olid pooled inimesed olid cum laude tipi lõpetanud, nii et see oli kohutav, see täiesti rumalaid inimesi oli palju, ütleme nii, ilmselt. Ei, ei, vastupidi, seal visuaalne ajupotentsiaal oli, oli nauditav, on ju seal seda pidasingi silmas ja, ja et, et see, mis seal tehti, noh ülemuse või sünnipäev, me enamus, mis teeme, oht nagu ratta rääkima vaba või tegid kõik. Peatse trendis oli lihtsalt absoluutselt lahedaid vendi, nii et selle kõik tore. No mis see, nagu siis ikkagi töö sisu oli. Kas, kas, kas see oli nagu sedalaadi osutus, millele tuli nõu töö siis, kui nad ise leiutasid? Ega neil ei tulnud mingit tellimuste, debaga hakk jukud tegelema, nendega ise pakkusid välja sele. See oli nii ja naa, eks ole, üks põhiline ajalooline rida, mis neil oli, oli tööstuskontrollerid mis nad olid, ise ise tegid, noh, või see väljamõeldis ja terviseprobleemideni Nõukogude Liiduga siis ja ajas ja absoluutsed. Ja noh, need olid siuksed korralikult Räki suurused, täna täna on samasugune asi või Hiinast osta enam kiibi peal või noh, põhimõtteliselt miniatuurse, aga, aga analoog, sisendid, analoogväljundid, Digital sisendid Digital väljundeid, eks ole, ja siis seal vahepeal mingi loogika, noh, see ongi, kontroller on kontroll oma olemuselt on ta aga noh, sel ajal jah, oli see vaene aeg ja, ja tehtigi, rektaku tuli ühisettevõtte mingi Soome partneriga. Ja tänase päevani teeb ehtakujuni, kas sa siis teed mingi kompukes nimi, kuid aeg-ajalt näha baarides siiamaani. Ja soligi soomlane tuli ütles, et okei, tehke mulle proovitööd, tehti klaviatuur, spets klaviatuur, et kus ma vahetan maatriks klaviatuur, nupud, baarmen vajutab võlu, kes on kohe olemas, mingi niisugune ja siis kui see tuli välja, siis mindi, mindi edasi, et et ega, ega see lihtne ei olnud sel ajal seda tellimust või tööd või raha leida on ja ja seetõttu oligi, noh, oled inimestesse pool aega jõude. Ja see oli umbes riistvara tarkvara umbes pooleks või nad on lihtsalt nii
Põnev ja sina muudkui programmeerisid seal reaniku Rogrammerts. Aga siis mingil hetkel ma arvan, ma olin näiteks kokku viis aastat oligi ülikooli ajakirjandus plaanis. Siiski üheksakümmend kaks ja üheksakümmend kaks võiks olla kus ma läksin ikkagi peamajja nii-öelda tagasi või Küberneetika Instituuti sel ajal siis kus tekkis võimalus siukse uue rakukese tekkimiseks, mis esialgu jumalast krüptograafia alusuuringutest. Ja seal olid mõned, mõned ülikooli siis tibi poisid ka nagu dialase moodi inimesed nagu Ahto Pulles ja nii edasi jahedaks. Noh, nii nad on päris teadlased tol hetkel ja, ja, ja, ja siis enam-vähem oli Ülo jaoks oli toonud välismaalt paksu raamatu krüptograafiaalused ala, eks ole, et umbes umbes lugesin ja seda kas koos või, ja siis keegi oli peatükki lugenud ja proovinud aru saada ja siis seletas teistele kuidagi niimoodi see vaikselt alguse sai. Siis tuli noh, et aretamisel see seadus nagu puudus teadmine Eestis oli noh, täielik null krüptograafias, kuna, kuna noh, arusaadavatel põhjustel jah, selline osa ju keelatud selle keelatud muidugi see oli keelatud ja ja siis, kui iseseisvus nii-öelda tuli on ja siis oli Päts lage ja pidi kuskilt alustame no vot siis niimoodi alustasimegi. Kuidas te tol hetkel, kuidas ta maailmast või krüptoprotokolli tol hetkel Sist maid, keda rahva ajaloost on globaalselt, ma ei tea. Niikuiniimoodi, mis asi tolleks hetkeks juba olemas oli, kui sa mõtled tagasi, RSA oli olemas järjest saali olema, siis kaheksakümnendal eks ole seitsekümmend kaheksakümmend üks eksmaailma ei tea, ma ei ole noh, niimoodi tegelikult iseennast kunagi käptoloogiks pidanud, et.
See nii-öelda Rakendus krüptograafia on rohkem, minul ei ole mitte mitte sügav sügav rektoraat aga ometi te selle alust alustele nagu oluliselt panna loomulikult jah, et kuskilt pidi nagu õppima hakkama Brüsseli segane, et kaevandust miks sa läksid siin ja seal oli ju mõnna olla, seal oma tegid oma projekte, võilill. Ma ei näe nagu halvasti öelda, aga põhimõtteliselt oli nii, nagu ma ütlesin, seal oli pooled inimesed on nii-öelda suurepäraselt inseneridega cum laude lõpetanud ja nii edasi. Aga firma ei tegelenud või saanud aru, et ta peaks tegelema nende nende inimese inimeste noh, nagu arenguga ka. Et see on väga selge seisukoht, et igale areng on täna enda asi. Et niisugust asja nagu Internet firmasse kui, kuigi see oli olnud olemas, et seda küll ei saa kulutada ja nii edasi või mingit ajakirja osta või, või mingite inimeste konverentsile saata. Ei, see ei tulnud kõne alla ookeani uuesti ja siis noh, see kogunes ja, ja mingil hetkel
Jah, oma sünnipäeval, siis saatsin kohalikku võrku essee, mida siia ikka aastaid ja aastaid tsiteeriti pärast mis nagu firmas valesti on, siis kümme aastat hiljem võeti välispraktikal samal tol ajal oli juba võrdil. No oli kohtvõrk, oli. Kes selle siis püsti pani, see oli ka kulugi. Ei no see tuli kuidagi naturaalselt, see, mis see on nagu nutikatele inimestele asi, see siis ma ei, ei, see ei olnud ka see kuidas see kamp selle ülejäänud selle eesti kogukonnaga kokku käis, sest tol hetkel juba Eestis nagu pruulis küll ju igasuguseid. Esimesed püssid hakkasid juba ju tekkima, nii, kuskil mingeid inimeste koju toimetasid. Nojah, mu hea sõber, kolleeg Heiki, kas, kui nad emaga pidas ühte Peebeeessi otsa ja ja mingisugune suur tegija ei olnud, noh, ja siis ma kuidagi koos olime seal, tema oli küll nagu opereeris, aga, aga ma kuidagi liitusin sellega ja siis ma kuidagi sattusin sinna nende teiste Fidofido nautide sekka lõpuks enne ja, ja kuidagi. Kuidagi sai hakata läbi käima. Aga põifo, kus oli ikkagi noh, see ei ole nagu sinu jaoks nagu oluline või kriitiline või kõne tähtis asi, Fido ei olnud minu jaoks kriitiline, on tähtis, see oli lahe ja andis nagu esialgse maigu suhu, aga nii kui tuli Internet siis, siis ma armusin sellesse ja, ja hakkasin seda, seda asja nagu rohkem uurima trendi juures niisugust armastusväärselt oli tall.
Meelis enne saate ja ma teadsin, et ja juhataja ei, lugu on selles, et ma teadsin, et kuskil noh, meil oli nagu uudsebee ja, ja modemiga helistamine mitu aastat esialgu üheksakümne esimesest eksi kuni üheksakümne kolmandani. Ja noh, sai meili saata on ju väga tore, aga aga mulle jõudis kohale, et kuskil on olemas nii-öelda püsiühendusega Internet. Et saab nagu reaalajas kuidagi soov see oli, see oli nii nii võlu minu jaoks. Et, et loomulikult seda tahtsid ühel või teisel moel uurida ja, ja see tundus, tundus väga lahe. Nii et, et veel nendel uudsebee ajastutel, ma mäletan ennast pühapäeviti kuskil modem külges rippuvat ja, ja RF Tseesid taundoodimas, et need kõik läbi lugeda algusest peale.
Lihtsalt oli võimalik, oli, oli see järjest see ülemaktusele kuskil tuhande cal andis alles, nii et, et see on mingisugune probleem tuhat terrasse ja läbi lugenud, enamus osad on lühikesed ja osad on mõttetud ja nii edasi, eks ole, aga noh, selleks, et see noh, see oli niisugune mõte, veel kord me oleme mamps on maniakaalses ilmselgelt, et kogus endale hästi palju nagu materjali, et küll ma ükspäev loed
Mõni kogub potililli või Uuberrenomissigaliseerep sees. Ja Internet, see oli sinu jaoks seal selles uues üksuses rohkem niisugune nagu infoallikas. No eks ta ikka oli jah. Naftainfoallikas jah, sain meili kirjutada, värki oli lihtsalt lahe ja võib-olla on meil on rääkinud seda lugu, kus tegelikult enne beebi oli noh, seal oli põhiliselt FTP saidil, siis sai pidanud mõtlema, et mis nõudis seda, kus enamusel FTP saidil, kus sa said siis mõnikord ka mõne mängu kätte ja ma ei tea mis iganes, noh, et seal ka liikluskraamisse Saidi aploodid laulnud ja samamoodi nagu filonitisena. Et, et FTP seal igasuguseid asju igast asju oli, aga ta arvutimängimist ja küljele mainin. Ega ma ei olnudki suur mängumees, et aga nukrat noorest peast ikka sõin midagi põlistatud, tähistatud õhtuti või midagi. Aga miks Ojuland? Ma ei tea.
Veel kord, see on seal umbes niimoodi, kui sul on mingisugune huvi ja siis sul on lõõgastumise aeg, et kes on lõõgastumise aeg, siis see võib olla mängida, aga, aga, aga arvutimängud ei olnud mu huvi, et see organ sees. Niipidi võib vist öelda, et mulle küll ülikoolis kursa Ena mõned kadusidki mudasse ju mudaauku ja just et ei tulnudki enam väljapaistva väga palju ei ole kohtunud, et sinuga niukest tunnet ei olnud. See ei olenenud Euroopast matemaatika peale. Progemise peale rohelise proovimise peale, mulle meeldiks arvata, tegema, panna oma pilli järgi tantsima panna, mitte see, et mina arvuti pilli järgi tantsida. Ja kui te lõpuks ütlen juba ette, et siis kui nagu Windows tuli, siis ma kaotasin usu arvutites, sellepärast et ma ei suutnud enam iga igalt biti kontrollida. Et kuni sinnamaani ma teadsin opsüsteemi tasemel Epp Romney tasemel, mis, mis sünnib, ma saan kontrollida, nagu kindlasti oli, kontroll kadus, aga siis kui mul läks tuju ära, siis kui Linux tuli, siis tuli tuju tagasi. Siis oli, ma ei tea, kuna on Linuxi olnud meistri, noh, Linux, siis aitas mind üle elada selle Vilniuse. Ütleme nii, aga päris hulluks teenuse kasutajaks ikkagi hakanud, nii et, et kui ma läksin sealt tehtavast kogemust ära Küberneetika, siis, siis ma põhimõttelistes progemise maha. Siis ma viimane asi, mis ma tegin, oli Veske kuuendal aastal mail punkt ee. Ja, ja peale seda hea tegid Mont. Nojah, üks juhendaja, kes väikese brändi selle tegemiseks. No ma ei tea, ei söö, käis kahes etapis kuidagi, et et kõigepealt see hea mõte, et igaühel võiks olla meiliaadress, kes tahab.
Hoian ma pean alustama muidugi sellest, et kuskil üheksakümne neljandal aastal sai tehtud selline firma nagu Teleport.
Mitte ajada segi selle sajandit läpordige ja see oli siis Tsäkline kaheksa tudengit, kes me olime, kuus, kuus õppisid välismaal näiteks, sest miks, miks niisiis neil oli vaata raha eest, tudengitel joon raha ja, ja kaheksakesi panime rahad kokku, ostsime portsu modemeid Soomest ja tegemiseks ja sissehelistamiskeskuse kus sai ilma lepinguta nii-öelda üheksa numbri kaudu helistada, lugenud, nii, et saad kohe niimoodi oma raha kätte selle läbi üheksasaja teenuse rongide ja, ja, ja siis Internetipakkumine sellel ajal oli kommertsiaalselt oli vaata et olematu. Ja noh, laiadele massidele mõeldes muidugi täiesti puudulik uni, et oli üheksakümmend, mis ajast on, Uninet oli olemas, aga, aga Uninet ütleme veel kord, sa pead lepingu sõlmima kõik eelmised asjad. Noh, Estatu oli olemas mehhanism Esto võrguga tegelikult. Ja aga siis tuli, kuu hiljem tuli Paikro link Online ja sõimidele massiga. Aga telefonist sai edasi meediamaa eksisteeri v punkt-EE, okei kaasates kui need partnerid ja sealt läks, siis oli siis Eesti Eesti esimene Nikoni veebiäri kus me proovisime inimestele rääkida, et kui sa, kui sind pole, internetis pole sind olemas. Ja, ja, ja tulevikus pole sul muud vaja. /url oma kaua auto peal vaatas mind nagu idiooti, eks ole. Noh, nüüd sa Urliga ja kaubaautosid ainult näebki. Võrguvad QR kood on visuaalselt raske. Veel on mul raske eelistada, kuna sealt jällegi, las ma nüüd urgitseb natuke, et miks te tolle firma tegite. Programmeerijad.
Pigem tudengiideaale tudengeid. No veel kord, sest Tarvi selgitas, et, et, et niisugust teenust turul ei ole vaja ja see tundus väga lahe, kui inimesed saavad juurdepääsu Internetile. Sellekohast siis niisugune nagu missiooni küsimus, eks seda võib, võib-olla me lootsime rahaga, teed on mõnes mõttes, aga, aga see tundus nagu olematu bisnes, kus on võimalik kanda kinnitada nii-öelda. Miks see peaks. Ja veebiga sama lugu, eks ole, nii-öelda palju paljuski missiooni eestvedamise värk oli noh, lihtsalt mingi raamatukogu üheksakümmend kuus Internetist mis oli eestikeelne, esimene originaalraamat, et see on Interneti propageerimine, ma samal ajal EE töö mõttes ehitasin juba riigile andmesidevõrkusid ja, ja kogu selle tissid, pia, epitehnoloogia niisugune laialdane levik tundus mulle sellel kümnendil väga-väga tähtis.
Miks?
Miks tundus tähtis ja sellepärast, et selle saavutamaks seda olukorda, kus me täna oleme. Ja see sul oli pea sees olemas, et selline olukord peab olema ja saab olema, see on hea. Ma, ma teadsin, et see on hea, ma tean, ma ei teadnud, kui kiiresti ja, ja kui massiliselt, aga, aga, aga see, need, need hüved olid ilmselged. Mulle küll ei olnud mul lalinat.
Vot ja siis sellepärast propageerisid Interneti Nurmsalu siiski kuulas ka. No ma arvan küll, jah. Ma arvan küll, et, et me oleme näinud, et igavest uuem tehnoloogia hüvitamine võtab ikka ju palju aega. Ja siis et see on täitsa loomulik, et, et me räägime ajast, mis oli nagu kakskümmend kakskümmend viis aastat tagasi, et siis neid järel meil võib siis näha, mis täna nii nagu muude asjade nagu ID-kaardikasutuse või e-hääletamise puhul, et, et need, need tulemused tulevad, mitte mitte järgmine päev, mitte järgmine kuu, nüüd järgmine aasta kaheksakümmend viis korda psühholoogidel üheksakümmend, mis ma teen. Et ütles Tarvi, et sa oled hull, need asjad, mis sa teed, see on inimeste käitumise muutmine, inimese käitud, ühiskondliku käitumise muutumine võtab miinimum seitsmekümne kaheksa aastat aega. Et sa ei saa enne oma tibusid lugeda, kui ma ei tea, kui sellise endisesse vanaks, kas sellele on ka ju see nii-öelda takkajärgi on see algimpulsi peaaegu sisulised, tuvastavad tuul, nii et vähe sellest, et et see võtab tükk aega aega pärast, nagu keegi aitäh kajutle jäin. Ma ei igatse selle üle osta, sest see on väga okei, ma lihtsalt vaatan ümberringi ja ma naeratan. Sa tead, et sa oled pannud sinna oma aluse.
Tuleme korra, selle. Sa mainisid, et sa tegid riigile mingisuguseid andmesideühendusi. Oja, see on üks tore lugu.
Usun, et oli aasta üheksakümmend kolm, kui tuli toll ja piirivalve. Ta oli Küberneetika Instituut ja noh, meil oli palju suhteid riigiga nõksudega, tehke mingi standard või mõtleme andmekogude peale. Kuidagi andmekaitse inspektsioon sai seal disainitud poolsema segadusega pettuks, et niisugused asjad. Ja, ja nad tulid ja ütlesid, et kahekesi käsikäes suitsu, et vaadake, meil on siin uus vabariik, on ju meil vaja nagu piirivalvet ühel toll teha. Ja meil on niisugused piiripunktid, kus me oleme koos asjal pole nagu mingit sidet.
Mõnikord isegi mitte telefonisidet, nii et et kas sa saaksid aidata, et julgi uuesti, et joonistage äkki mingi projekt, okei, no siis taru joonistas projekti eanis projekti see noh, see projekti kätte käisid linna peal või paari kuu pärast tagasi ütlesid, et mitte keegi ei suuda seda ellu viia. Et tehke nüüd ise see projekt ära ka. Ja siis kuidagi juhtuski niimoodi, et me pidime paberimäärimisest hakkama minema tegudele üle, et koostöös Eesti tolle telefoniga, kus mul mingid sidemed, sassis esimesed ühendused nagu ära tehtud ja siis hakkas mullina paisunud kogu see tegevus, et, et et järgmisena tuli politsei ja, ja, ja, ja siis riburada teised ka järgi ja me olime nagu teostasime Küberneetika Instituudi katuse all instituudi varajased amorfne asutus, tahtsin teid teadust, tahtsid, tegid äri, tahtsid, tegid, olid nagu riigi niisugune.
Asi rei asi jah. No vot ja siis sai tehtud kui asja, kes juba niimoodi raha läksid liikuma ja ruuterit pidi ostma ja siis siis oli vaja moodustada mingi formatsioon, on tegelikult midagi niisugust, mida ei tohtinud tegelikult seal soomlase järgi teha, me tegime, põhimõtteliselt tegime MTÜ riigiasutustest. Ühesõnaga, meil oli nii olnud, nii, nimed on esid, osakond suure tähega ja andmesideosakond ta teda juhtis nõukogu kus oli iga riigiasutuse esindaja kuulus asemel sõnakuulmises vähem. Ja, ja iga aastaraamatupidamistoimkond ei teaks midagi lõkst, siis on siiani on olemas olevat järisenud, et, et sellist asja ei tohi teha ja siis kõik need suured ülemused ja ministrid ütlesid, et ärge ärme ärme lõhu toimivat asja. Jah, kõla kõlab nagu õigesti jääb, see tähendab, et me skripti sinna kulusid, öeldes sama politsei, piirivalve näide ju sul on vaja piiripunkt üks ruuter osta temp, kulud pooleks, nihkunud loogiliselt, väga loolin seikadest ja see, et riigis ka keegi nagu mõtles kaasa sellele, et see ei saa olla nii, et arv ütles, et kuulge, poisid, teeme nüüd, teeme nüüd nii. Sul pidi olema keegi, kes sind nagu kuulas riigi poole.
Aga needsamad kuulasidki ja see on see tippjuhte välisIT-juht või, või, või, või mis ametit ja kes kesist kuulus? Ei no ma arvan, et kõigepealt kuulasid nii-öelda IT-juhid, siis nad rääkisid sellest oma tippjuhid. Kas need tippjuhid tegelikult ma mäletan väga selgelt, kuidas kolmekümne kolmekümnendal kolme esimesel detsembril istusid toonase piirivalve
Älema kõutsi kabinetis kõik asjaosalised.
Ümber laua, neid moodustajaid oli siis nii-öelda kolm, ehk siis neli, et politsei, piirivalve ja toll, noh ja siis Küberneetika Instituuti kirjutati alla või selle läbi lööd. Ma saan aru, et meil on siin juhi kandidaat ka laua taga ja siis noh, ma olin kahekümne viie aastane. Aga selleks väga huvitavaks edasi, sellepärast et noh, Hallisegune asutused nagu valitsusside iseenesest olemas. Kes tegi, tegeles erivõrkudega ja puha ja kurjaks saanud noot, eksinud teatav konflikt, konflikt, nagu tekkis erinevatel põhjustel ka ütlemiseks koolkondade, vastasseisul jaoks saab sealses Lippmaad, eks ole. Mis vanemad vanemad inimesed teavad seda väga hästi, et me sinna läheme. Äärne sinnale. Jaa, jaa, aga asi juhtus jah, niimoodi, et piirivalvel ühel hetkel tal oli kagupiir, oli täiesti lage selles mõttes, et seal polnud mingit sidet, siin oli vaja uus side üldse tekitada. Ja selle asemel, et linnavalitsust sidesse, mis, kes peaksid neid asju tegema tulid nad minu juurde, lihtsalt näed, Tarvi, siin on kümme miljonit, on ju umbes kahekümne viie aastane poiss, mis siin kümme miljonit, et meil on seal lage plats minu kaheksa piiri muutva nimetada tee midagi. Maja väga huvitav. Või lihtsalt enda jaoks reflekteerida. Aastal üheksakümmend kuus, üheksakümmend kolm, Ast oli üheksakümmend neli, aastal üheksakümmend neli ja suutis keegi, milliseid sellest neli-viis kuskil sealkandis suutis keegi välja peksta kümme milli, mis oli meeletu raha tol ajal selleks, et välja ehitada mingit sidevõrku, kohe näha. Et kõlab nagu täitsa uskunud, nii vaene aeg oli ja keegi arvas, et sidet oleks vaja arendada. See mõte tekkis või? No kui sul ei ole telefoni, seda ka mitte, eks ole, me ei räägi siin mingist Transist veebibrausimisest või me räägime ikkagi telefonisidest. Sest noh, elementaarsest sõnumivahetusest ehk siis kas seda võrku kujul olek keset platsi ja sul ei ole mingit mobiilevise või mitte midagi. Kuidas sa seda piiri peale nii sõjaväepunkt oma sagedustele ning raadioside nõuab, võis, käi, ei olnud. Okei, nõuet pole majani põhise teha või vähemalt mulle öeldi alal vaja andmesidet ja, ja telefoni siiani keegi riigis siis mõtles-mõtles ja eraldas raha ja ressurssi juurde. Jah, esmapäeva lõpuks oli see projektijuht, kes ehitas tühjale kohale sada kaheksa vastu ja panin sinna talvikud otsa. Mis see oli, taldrikutega ju mingisugune üle õhu, mingi side, raadioside nagu mobiilis käib. Mobiilimastide vaheline, noh, kes selle sageduse andis? Kas seal pidi eraldi andma, ma ei teagi seal mingi kaheksa koma nelja viga peal ja seda sai nagu pruukida, ma ei mäleta, kas seal oli mingi kooskõlastamist, võib-olla oli see ju selle noh, raadioside disaini mõttes ei ole ka nagu triviaalne üles on hiies lihtsalt kapist selgeks, kuidas seda minagi kuulatama jäin, eks ole, kõige hullem oli see, et no mis ma mõtlesin, ma mõtlesin, et kõrge sageduse peab olema, otsenähtavus oli lihtne. Aga kuidas seda otsenähtavus kindlaks teha? Okei, Lõuna-Eesti maastik, mäed, mäed ja orud. Siis ma leidsin.
Maa-ametist ühe tuttava, kes, kes oli sihukene fänn, et oli hakanud vene ohvitseri kaart kõige täpsemaid, mis oli sel hetkel oli digiteerima hakanud. Ja ta oli selle kõige huvitavam osa ehk siis Võrumaal nagu sisse saanud. Ja siis ta suutis nagu väljastada mulle profiiliga, ma ütlesin, punkt A punkti B on profiilist, andis mulle mingi arvude ja jada on ju, noh, siis ma küsiks nii, et proge on ja panin, keerasin maa kumeraks, panin Mastmastik kasvuga, vaatasin, kas on otsenähtavus. Ja siis selle järgi sai rehkendatud mastide kõrgus, sellepärast et mida kõrgem, seda kallim. Noh, Emp oli, sel ajal ei vaatame mingit profiili. Pani kaheksakümmend igale poole, mul oli seal viiskümmend kaks ja viiskümmend, nelisada viiskümmend, peab juba lennutuled olema jälle kallim ja noh, selles ja, ja siis ma sattusin mingi mingisuguse teadjamehe peale, kes vaatas Eesti telefonist siis vaates melonid.
Tehtud ütles, et mis asja sa tegid, nagu nädala ajaga on üks projektile sul ühed kuule mees, ühe ühe trassi projekteerimiseks läheb poolteist aastat jala kõik läbi käima, puud kõik ära kaardistama, siis näitas mulle neid vanu värk selle nädalaga veidi noh ja siis, aga mullimastid tellitud juba, nii et siis ta rääkis mulle, et ta on olemas frenelli tsoone ka veel ja, ja et seda, et niisugune vorst on ju tegelikult otse kiir, päris ei ole. Sutist natuke junni jahedaks küll, aga noh, varsti tuli tellitud ei ole, käib mõõda kõike ainult selleks palun järgmine aasta tegin Peipsi äärde sama sama viguri, nii et siis oli juba tehnika. Sa ei teadnud, et nii ei saa teha? Ei, ma ei teadnud ja ma mõtlesin inseneri mõistusega, kuidas käib. Ma tahtsin selle optimiseerimise kohta küsida tsementi ikkagi nagu päris oluline raha ja see on nüüd nagu kaks niisugust lähenemist, et selline lähenemine on see, et et ma hakkan arvutama ja optimiseerima ähemastidel viiskümmend neli, teisi viiskümmend kolm meetrit muidugi, aga, aga väga paljud inimesed, uimul on palju raha, olen korralikult, võtame välismaalt, teeme ja kõrged mastid, paneme vägede tehnika. Miks see nii põgule me räägime ikkagi vabariigi algusaegadest, kus raha ei ole korralikult ja palju, eks ole, et sul sa pead olema igas asjas optimaalne loomulikult, ja ega noh, või kooner, kui tahad, eks ole, ja, ja tegema parimat, mis, mis mis teha annab, raha ei olnud ülemäära palju ja siis ei olnud. See ei olnud nagu teab mis üle üleliia raha, seega kulus ära kõik need muidugi raha kulub alati ära. Et tal on see omadus, ei, mulle ei antud seda, piirivalve finantseeris seda lihtsalt mina olen endiselt nagu nende projektijuht Sistikumis, nende seas ehitus, kuidas kulutada intsidendi. Just nii, et see oli väga tore tore aeg, kus sai tõesti käegakatsutavat riigi arengut toetatud, mis oli mis oli väga lahe. Ja pealegi minu lemmiktehnoloogiale ehk siis internetitehnoloogia internetitehnoloogiaid. Et see on, ma küsin, kuulan siis ühel hetkel sul nii-öelda sai programmeerija, programmeerija, programmeerija, siis saabus Internetti, siis sa leidsid, et tegelikult tuleb panustada hoopis uus hinna, sellepärast et maailm läheb paremaks. Sellest ja programmeerida oskasid sel ajal juba rohkem ja rohkem ja rohkem inimesi. Ma ei olnud väga unikaalne enam, eks ole, ja võib-olla ka niipidi võib mõelda. Aga kaua sa ikka programmeerib, noh, et mõni programmeerib eluaeg ja ma saan aru, jah, noh, ühesõnaga ma saan aru, aga ma ei tea, niisugused kõrgemad hullemad mõtted tundusid tundis, et järjest-järjest paremat kujulise isiksuse arenguga seotud ja kui ma olin programmeerija, siis ma ausalt öeldes, kui telefon helises, ma ei, ma kartsin seda väga, ma ei saa palju inimestega suhelda ja maalima. No vot, vot see on vot see on, mingil hetkel läks see kuidagi teist teistpidi üle. Linnapeal, tehes kõik Martin lihtsalt, kui ta jaurama tuleb, seda kindlasti proovimist küberneetika, kas see tööle meelitada?
Et keda näiteks võlts mingisugune meelitaja mees, mingisugune juhtkonnas juba, lausa asupealik solideks asugajaid, kasu, päälik, kasu sai üle antud. Informaatikakeskusele oli Riia eelkäija või informaatika, mis iganes, mingi asi, mingi asi oli, triaid oli küll niuke asi. Ja, ja sest tekkis reformatsioon aastal üheksakümmend seitse kus instituudid, kui sellised
Austati eraldiseisvana pidid nagu liituma ülikoolidega. Üldjuhul siis mis Küberneetika instituudiga juhtus, oli see, et põhimõttelises plitus kolmeks kõige väiksem osa, ehk siis andmesideosakond, see, siis läksin informaatika keskmesse. Ja ja ülejäänud kaks läks kaheksa, kuna noh, miks läks Tallinna Tehnikaülikooli alla. Aga kuna küberregistris praktilist tegevust hästi palju, siis kõik kõigest praktilisest moodustati Küberneetika Aktsiaselts, mis siiamaani alles sel hetkel olid, moodustati riigiettevõttena. Nüüd on vist erastatud. Aga jah, aga see oli väga huvitav selline kombinatsioon, et seal oli osakond, kes tegeles tolli, tolliametisüsteemidega, infotehnoloogilised probleemid seal siis oli see, mille, kus mina olin, oli, oli siis keskendunud infoturbe peale, nii teoorias, praktikas konsultatsioonides, analüüsides, millesse iganes. Ja seal kõrval oli siis kõik see mere märgindus, mis siiamaani on ja, ja merenavigatsiooni ja valgusfooride tegemata minu meelest, kui see niimoodi räägid ja kinnisvarahaldus ka välimisel alusel on kõige olulisem ja see on nüüd läinud. Ärge noh, tänu kinnisvaraarendusse mitu inimest rikkaks, nii ma siin kuulen, siis sa oled olnud nagu tegite mingi teleport ja meediamaa tiigist äkki. Et kus on niisugune ja projektijuhte. Kui juht on nagu hakkab tekkima nagu nimesse. Sul naguniisugune inimeste juhtimine? Ei kuidagi ei tulnud nagu teemaks, et mõned inimesed leiavad, saab maigu suhu ja siis pärast tegelevad kandi inimeste juhtimisega.
Kas ei olnud? Eks ma jõuga pidin maigu suhu saamisest, tegevust oli vaja ju ju kuidagi laiendada ja, ja töö tahtis tegemist sel hetkel juba ja inimesi oli vaja ja inimesed oli meelitada inimestele juurde, noh seal küber need ka aktsiaseltsi moodustamisel sai minust arendusdirektor, nii et Jumala ette ja kohe direktori
Ikka ka oli, olin isegi mis asi. Ja, ja, ja noh, mõeldi küll, et ma vaatan nagu laiemates ja võib-olla ka tegelenud meremärkidega ühe poidega, aga see sinna õnge ma ei läinud ja, ja, ja ja tegelesin siis ma hakkasin tegelema juba infoturbetoodete arendamisega, et me tegime seal neid jões kuus, tegime esimese tulemüüri valmis ja ja sealt läksin lihtsalt VPN toota ja mingitesse selli või siis Soxi, siis oli seepärast meediamaad ja seda asja pärast ja ja, ja see läks esialgu hirmsasti noh, mõtlen, et me tegime Linuxi peale, tegime põhimõtteliselt veebipõhise interfeisi nendele jubinatele, millesse saab niikuinii ise tulemüüri teha, kui sul on väga osav insener. Noh, tegime selle lihtsamaks veebiliidese kaudu ja oligi toode valmis jämedas plaanis. Et siis meil oli eesmärk teha viis korda keskmisest odavam toe, et keskmine tulemus Electronics kolm tuhat dollarit tähtsa tundeline tuli veel veel veel veel vähem oleks. Ja, ja loomulikult tuli välja ja sai väga noh, vot siin peab nüüd mõtlema, et kas siin oli side, ilmselt oli aga ka just riigiasutused väga meelelt ostsin neid tooteid, kas me kõige sai kuskil öeldud, no Tarvi tegi võrgud nüüd nüüd neile turvaka peale, eks ole, seda praktikas niimoodi palju küll paljusid. Kust kohast ta on ikka imestan seda, et enamasti mu riikides käin, siis enamasti on seal tekkinud mingisugune hullus, et näiteks tema omale privaat Internetti nagu riigiasutuste vaheline turvaline ja äge. Kuidas see sündisid Eestis niisugust hullusti tehtud? No üldsegi üritati või ikka üritati jah, loomulikult üritati ja aga just läbi selle, et, et me tegime siukse VPN toote. See oli niisugune, nagu ta oli, unikaalne, selles mõttes. Võrreldes praeguste VPN toodetega. Kui sul see Gaston eesvõrgul sai, saab Interneti sa ei saa keelata ja lasen, ta ei lase, ta laseb ainult Eesti teiste teise omasuguse juurde. Nii et, et sul oli, nägi asi välja niimoodi, et, et sul on kontorid, ütleme igas maakonnas on ja siis sa paned iga maakonnas selle rohelise kasti eteni ja siis kama peale. No ütleme, Tallinnas on ju, on üks tulemärga ja ainult selle läbi ühe tulemüüri sealt välja, kui sa millegi pärast vaja, aga muidu on täiesti, vajutas sisesidevõrk. Nii et, et, et mul pus üks klient, kellel nime isa nagu noh, võib nimetada ka pea tuleb, tuleb minu juurde, ütleb, et kuule, et mul oleks seda VPN nime lahendusteni. Mis nii Aurora tore, et paned sinna siis sul on ju ühte tulemüüriga vaja, et välja saada lihtsalt ei ole. Vaatas sügavalt silma, et see ei ole. Ei ole vaja, siis ei ole need droonid Eestist, see aga kuule ütleb, et kus see, see, mis sa kirjeldad, sven xD? Ei, see ei ole, see on ju selles mõttes, aga noh, arhitektuuri mõttes on ju väga ainukene sarnane, et kui sa tahad teistega rääkida, siis üks roheline kast nurgas, siis sa räägid sellega, siis ise vaatab, kuidas kuidas teiste juurde jõuab, kas sa võid ju nii mõelda, aga siin pole nagu andmete semantika mingit pistmist. Ühesõnaga, seal ei olnud mingisugust risttolmlemist hetki, ei vala peale, ütles see oli ikka puhal torude ehitamine, eks ole, privaattorude ehitamine, et, et, et siin X-tee on nagu isaisa osi tasemetest natuke võrdsel seal vahepeal mõnevõrra. Nii neli-viis ja ol neli. Kas tol ajal sa tegelesid selle internetiprouaga veel paralleeli edasi või oli, see on niisugune üks baas, kus sa käisid läbi? Ei, mulle ei tuleks, nagu et ma siis oli juba turul oli piisavalt tegijaid ja, ja ma ei tundnud, et ma peaks tegelema, pigem on nagu see, et, et ehitada Interneti minu jaoks oli saabunud faas number kaks. Et nüüd tuleb see Internet turvaliseks teha. Aga kus on niisugune hullus, mõtlen küll, et see asi purun? Ei no ma ju rääkisin sulle, et me hakkaksime teoreetiliselt turvaga tegelema, üheski kaks juba on ja kõik need kontserdid, et poliitolid mulle väga, vägagi tuttavad, enne et kuidas, kuidas nagu teha ja miks seda vajalikud ja ja selles mõttes enamik inimesi mõtlesin, et see on ju mingisugune niisugune kuramuse hipikommuun, et noh, reele info tahab vaba olla ja puha. Ma just rääkisin rohelistest kastide estonian, kus on puhas ning turvaline. Andmeside ongi kogu eesmärk, rohkem ei olegi kui lugu. Minu minu sõnum oli nagu see, et ärme teeme eraldi x kakskümmend viis võrku on ju umbes vaid vaid me saame üle avaliku. Interneti toimetades on palju kuluefektiivsem.
Muidugi.
Ma tahaks uurida, et kus on see arusaam, et Internet et see Internetti mitte turvalisuse probleemid, millele hakata lahendama?
Tol ajal, kui sellega joosti, häkkerid jooksid pikali või mingi keegi luures või, või kus see nagu probleem üldse tekkis.
Vaheaegade alguses pai, mis sa mõtled, et kuskilt tuli mulle ilmutus või, või lugesin mingit artiklit ei saa. See on elementaarne, et, et see on kogu aeg olnud niimoodi ja ja olles nagu infoturbe ka algusest peale tegelenud selgelt võrkudes on infoturbe nagu teema. Aga jah, et see on elementaarne, mainib mida, see, mis jällegi nii ava oma ajaloo peale mõtlen siis on tükk aega nagu võrgutasin, tegin asju niimoodi, et mul ei olnud aimugi, et see võiks kuidagi turvaline olla, see, mis ma teen. Noh, ja, ja oli kui keegi veebiserveriga ma siiamaani räägin, et ma siiamaani rääkinud, ma tulen, tulen nii-öelda infoturbe erialalt, eks ole, ükskõik mis, mis, mis ametis, et see on minu põhimõtteliselt Päcrond. No kuidas, kuidas sa said sinna, sest sest me rääkisime, et kas see oli seesama koht, kus te lugesite koos seda ühte raamatu täpselt nii. Ja seal sai Andres, sinu infoturbe, päka ongi ehitamine. Mis tähendab, et sul ei ole mitte ainult nagu infoturbe Black Chrome, vaid sul on ka nagu matemaatika pluss programmeerimine ja pluss nagu antenni ehitamine, et sul see nagu nagu läheb nagu sügavale väljuma jeep promineks, nagu sa ütlesid meie grupile hea jah, ma olen kirjutanud Jukul püsiv elu põhimõtteliselt.
Olles nagu oma olulised hetked elanud üle Juku taha, siis ma vaatan, et ma toolilt maha kukkuma, suurest austusest katsume ennast siiski pidada peaaegu lae, sa saad teha Räkse. Noh, mul oli see osa, mis puudutas täpne tähtede joonistamist ekraanile ja koos oli ilusti sai siis see noh, näed, seal olid, said teha aknaid, mis kõik käskudega. Ja oli ja valides rollis kas aeglaselt või jõnksatas, on ju kaks varianti oli sees ja aeglases rolli vaid aknaid, eks roll nagu trummid asemel. Joonis, sest sest selle neetud selle nende fontide realistiks, sest kasvav Ponte teha. Ja teiseks sa said nende peale või mis iganes graafikat ja mis käis nii nagu välk, sest saab rohelist omal need tallid valmis ja siis muudkui lausuda puhvrisse ja muudkui joonistas. Nii on ja kõik oli kaunis. See oli, see oli ilus aeg, on ilus aeg, oh jumal kus arvutid olid, lihtsad ja aknaid ei olnud. No ma ütlen, mul oli EP, ainult apturoopalisele.
Me oleme natukene ütelnud, aga see on kõik. See on kõik puhas, puhas haridus.
Sa oled siis niisugune asjade
Käiboläkk Visureerija. No kui sa nii ütled, ma küsin. Kes sa niisugune oled, ja siis pärast programmeerijana või ole, selles, heitsid maha ja.
Ja enam, mis, mis siin ikka, ei oska ennast nagu, väga palju sildistada ega tea, eks ma, Eks ma kuidagi mõtlen kuidagi laiemalt kogu aeg ja see häda külges, et see häda noh, võiks ju midagi näpu vahel midagi teha, mingisugust kaltsuvaiba või.
Tükk taga, et müts suurtest sõnadest või aga aga jah, ma võib-olla jäime siis sinna kuskile, et ma tuletan sulle meelde kuumeen. Et, et teeme sinna kõvera üks oli siis see infoturbeperiood väga tugev, eks ole, me rääkisime sellest, et kõigepealt tuleb Internet valmis ehitada ja siis läks turvaliseks turvalisust ja ja siis, kui on Internet urvalind tehtud, siis on elementaarne, kolmas aste, ehk siis tuleb kõik osapooled internetis identifitseerida. Et siis saab nagu Business ka teha. Muidu muidu on see puhas niimoodi seal alla, et seal selle koha peal vaatan nii ajaaja markereid, siis ma ütlen selle põneva koha peal, miks teil praegu pausi see aga hirmus Kibur edasi rääkida, aga ma tean, et kui me sinna teemasse lähme, siis see võtab veel teise tunni vähemalt. Sest enamasti riigid ei ole jõudnud siiamaani selle arusaamiseni, et selleks, et saaks internetist bisnist teha, peaks inimeste ära tuvastama. See on siiamaani nagu enamasti võõras kontseptsioon ja kust see sulle niisugune mõte pähe tuli, et sellest ma tahaks küll kangesti küsida. Aga küsin hoopis seda, et millega sa praegu tegeled? Mis teid? No ma endiselt tulen, elektroonilise hääletuse juht juba siis aastast kaks tuhat kolm.
Hiljuti oli meil siis kümnendat valimised, tulevad, tuleb kohe on algamas üheteistkümnendad, ehk siis Europarlamendi valimised. Loom jah, aga see võtab võib-olla kaks-kolm kuud pere valimine seda tähelepanu valimistevahelisel ajal. Ega ma palju suteerima jõudumööda nii nagu kutsutakse, käin maailmas ringi ja, ja proovin siis natukene inimesi inimesi aidata nende, nende arengutesse erinevates riikides, nii elektroonilisi identiteedi alal kui ka. Kui kedagi huvitab noh, IKT nii-öelda valimismajandusest, siis sellel ajal ka selge. Siinkohal ütlen suure-suure ja ümmarguse ana südamest tulev aitäh, et viitsisid minuga juttu rääkida ja ma arvan, et me peame ühe korra veel rääkima. Sellepärast et see ID-kaardi, maailma, see asi, see, see on ikka nagu põnev, kus, nagu see tuli. Ja, ja mitu kohta ja sinu jutus jäid mulle ka kõlama, et sealt tahaks nagu rohkem murde. Aitäh sulle. Palun.

%------------------------

Äkki äkki äkki?
Tere See siin on Memm kopi. Tänane episood on järg eelmisel korral alguse saanud jutule Tarvi Martensi ka head kuulamist. Memm labi tööst. Tere, ma teen väikse sissejuhatusega, oleme jälle kogunenud Nõmme mändide alla. Sest elu juurde jutt pooleli. Kuhu meil jutt pooleli jäi?
No tegelikult ei pooleli sinnapoole, et sa tahtsid teada hakata saama juba, kuidas selle elektroonse identiteediga ja kuidas see arenes, aga mul tuli vahepeal meelde igasuguseid toredaid asju, mida tegelikult seal juhtus. Üheksakümnendad No räägi, hakkame, jumal hoidku. Ja nad on väga killustatud, polnud niisugused hullud. Arula EXPO kuulasin sünteesi neid üles, mõtteid ka ja siis väga paljud inimesed mainisid. Jaak Loondet. No mina olen Tallinna poiss, ei olnud minul noh, kuulsin teiste käest, et niisugune mees on olemas. Ahaa, et ta oli siis nagu kuulus nagu kogukonnas juba tol ajal, isegi arvestades tol ajal suhteliselt nagu väikest kommunikatiivset võimet. No täpselt jah, et need Fido inimesed rääkisid Loondest ja, ja, ja nii, eks ole. Minul taga eriti kokkupuudet ei olnud, aga siis juhtus niisugune lugu, et mingeid varajast selle üheksakümnendatel, kus Eestis on veel isegi leiba, ainult varitalongid ja juhtus niisugune lugu, et et Soome Rotary klubi otsustas Eesti koolidele kinkida arvutit natukene uhke ja. Ühesõnaga ühel tehase inimesel jäi kuus tükki üle. Ilmselt oli mingi piisi aeg peale tulnud ja tal olid mingid väga kummalised siuksed. Kompuutri, ma isegi ei mäleta, aga nad olid võrguks, mis oli lahe, olid võrgus ja talle ema, arvuti oli ka, eks ole, ja, ja, ja ja nad tegid siis selle ettepaneku haridusministeeriumile, et meil intiim Eesti riigile nagu kuidagi koolidele kuidagi arvuteid. No, ja siis oli Peeter Loorents monument rolli sel ajal Haridusministeeriumis mingit tegelinski ja tema sattus selle teema peale ja siis oligi, et läksime kolmekesi. Autojuht Peeter ja mina oleksin, mina olin siis nagu ekspert, kes mis koha peal vaatama, mis kuradi loomad seal on ja aru saama, et kuidas nad käivad ja läksime kohale, tõime need arvutid ära ja siis tekkis muidugi mõtet korrata, siis me nendega peale hakkame ka palju ning paljud olid nii kuus-seitse tükki umbes. Noh, ja niisugune klasside eest Eesti peal ei olnud näha ka Rotary klubi sai endale kurat linnukese kirja, Eestit aidatud kõikjal värki laadne riik on tore muidugi, aga eks nad isegi noh, kui heategevustehtud, onju tegelikult õhtu ilus, tark. Ja siis mulle meenus Jaak loonud ja siis nad Pulet, et ma räägin a ploomidega ja et äkki äkki seal on, tal on vaja kuskile klassi panna, et saavad lapsed mängida. Ja nii oli, Jaak sai nõukoguga kokku ja Jaak oli kohe nõus, läksid silmad põlema nagu ikka ja, ja, ja, ja siis ma sain need arvutid kätte, siis mõne aasta pärast nägin, tõdes, küsinud, kas nad nagu pruukimist koli ja, ja tuli välja, et oli ikka väga hästi vastu võetud ja saab igasuguseid vigureid tehtud. No kuidas siis see ju tähendab seda, et vot see on oluline kild informatsiooni sellepärast et see tähendab seda, et Jaak toimetas ka pärast seda, kui see igasuguste Ahti Heinla ja muude inimeste nagu jutt ära lõppes, et ikkagi üheksakümnendate jooksul ajast kaasas asja ikka toimetes minu arust nii-öelda lõpuni, eks ole, et ta oli ka legendaarne niisugune, ta tahtis, et lapsed saaksid näppude arvuti külge ja see oli tema põhiline, nagu siis mõtte ja mitte ainult ei tahtnud, vaid ka tegi aktiivselt sõna ja teoga. Täpselt nii on mõttega mees vaat siuke me jutujutu jutulõng, oluline hea siis rääkides sugusest suhtlemisest ja noh, kus filos võib-olla hakkas esialgu pihta, et noh, said nagu meili saata ja värk ja siis tuli Internetiaeg tuli peale. Ja ma pean tunnistama seda, et Tarvi tegi üheksakümne kolmandal aastal esimese jututoa. Ahaa, ja jututuba ja siis selline asi, nagu meil parasjagu on, mingi suur depressiinser või värk, aga noh siis nagu, et hulk inimesi logivad sisse teksti terminal, need olid ju igavesed kuulsatele. No muidugi, et see ei anna jututuba, mis, kas see käis mingisuguses ja oma softi peal või sov sait kuskil ma sain kuskilt selle softi, ma tõlkisin selle käsud eesti keelde. Nii ja naa, et siis oli seal olid need punktid, hakkas käsk või siis sai teha mingisugust värki. Ja siis jah, ma olin sel hetkel
Ja ütleme kooris asumisel neljaks kuuks ja mulle on seal suurt-suurt midagi teha ja siis ma mõtlesin seda jututuba ja otsin seda üleval, et seal peaks loe. Et selles jututoas ana jututoas isegi kasti üks üks üks diplomitöö ära. Isegi nii hea inimene oli, inimene oli Austraalias, Uus-Meremaal olid ja, ja õppejõud olid siis kogunesid jututuppa ja ja siis toimus vestlus ja naudid Tallinna Tehnikaülikool või noh, tipp selle tõsise. No mis nende jutukate nagu fenomen oli, sellepärast et nagu, kui ma kuulen nagu teksti terminal ja niisugune noh, mingit käsupunktiga, eks siis see on nagu tehnikute värk. Aga ma nagu mäletan, siis seal nagu jutukates käisime igasugust rahvast ja see oli tegelikult ikka kommi Unity pildid ja et ta ei ole ikka kes, kes oli seal sees, oli noh, umbes samasugune grupp nagu ütleme Fido grupp, eks ole, või mis iganes text, jutukate grupp, et sealt edasi läks igasugused OK jutukad ja kahva jutukad ja meil oli isegi anna anna nende kasutajate kokkutulek oli mingil hetkel tugevamini veskis Viljandi lähistel ja see siiamaani Jüri Ruut peab seda küll nüüd kevade nime all, nii et, et see pole Samine kuhugi kadunud, see oli täitsa omaette subkultuur, söönud oma paroolid ja omad mingisugune keelendid ja omanikud lähevad asjad, mis selle, kes eilse unustatud praeguseks ja seltskond oli, roos. Ikkagi tekkis, tekstid on ikkagi jah, ütleme niisugune seltskondlik üritus, et aga, aga mis selle avastuse kogunes, kas alternatiivide puudumisel või täiesti täiesti suvaline, eks ole, seal ei olnud õnneks sõna, on üks, et tehno friigid, ainult eks mul oli tütarlaps iga, eks ole reaalseks. Nii et, et seal oli kõik väga lahe, see pidi siis olema hirmus vajalik, sest nagu mitte teenov riigile, kõik see nagu tehnika, et sa oled del netiga kuskile porti ja ja see käsk asju noh, see ei ole ju nagu lihtne. See on täiesti lihtne on tegelikult, et kusagil seal terminal ligi saitaks, Oläpatel net ja anna punki otse puhta vee ja läksin. Ühesõnaga Sa hoidsid seda käberis Küberneetika Instituudis Üleoja ja seda küll seda peab, tunnistan Siberis oli nagu üldse see sai. Aga alustasime juuniksid pruukimisega sel aastal mingi üheksakümmend kaks siis sai nagu see Liinuks toodud floppidega Soomest teistmoodi kätte, otse otse Linusega ja, ja on vähem. Jah. Ja see oli väga lahe igal juhul jah, seal seda juuniksi kultuur proovisin aretada ja sai siis sai hirmsa raha eest, üks on ostetud ja, ja, ja siis oli esiti mis nimeks panna, mõtlesin keeks suvaliselt ja siis pärast pidin välja, noh, siis oligi keeks punkti otsetee oli kaua see kuulus FTP serveris, mina mäletan seda purki. No vot, vot vot vot siis ma pärast pidin selle keeksi lahti mõtestama ja siis ma arvasin, et see on käberi eesti klaveri esimene eestimeelset ja kasutajatest. Ehk siis
Keeks niivõrd tulema, kuna nende putukate juurde tagasi, et selleks, et kui mingi asi läheks nagu vaeraliks nagu käima, siis on tarvis nagu lisaks sellele, et seal on mingisugune nii-öelda nakkuvus, et inimesed, kes sinna satuvad, jäävad sinna käima. See on tarvis mingisugust nii-öelda. Pesin siru või noh, see esimene seltskond, kes seal nagu käima hakkab kes need inimesed olid ja kuidas sa selle tekitasid, et see, et sa panid lihtsalt pardi peale mingi asja kuulama, et see ei tähenda seda, et selle kiige külge tuli. Seda ma nüüd täpselt ei mäleta, küllap ma rääkisin sõpradele seda kaheksateist sõpradele ja kuidagi niimoodi ta kuidagi niimoodi ta vaikselt Lewis, et ma küll mingit erilist reaktsiooni ei mäleta, et see selle jaoks lihtsalt sõpradega läkski, läks käima, läks käima Jansa jooga, seal neid sõprade, sõprade sõprade ring läks väga-väga laiaks ja ja, ja noh, selles mõttes, et ikkagi üle pool olid täiesti tundmatud inimesed seal või rohkem isegi noh ega tollal alternatiiv ei olnud väga niikuiniisuguseks. Suhtleb, aga on, aga juhtus tõesti niimoodi, et ühel hetkel vaatasin, et teised jutukad hakkasid ka tekkima ja ma võin ühel hetkel ta pidulikult kinni tolmu mõjul ja anna matused olid tükke eraldis. Selline ettevõtmine. See oli ka nagu asi, mida siis asi on meile õpetanud, seda, et seda tuleb, peab mingi asja ära lõpetama, mitte et hääbub lihtsalt, et. No kaua võib. Juuniks jutu juurde tagasi tulles oli see, et meil oli kunagi üks järjekordne moodustis oli niisugune asi nagu Eesti juuniksid, pruukjate selts. Ups oli nii, kui tahtsin pahnapercaxow Kasutajate Seltsi ees, uks on ju, see kõlab nagu halvasti, siis mina, mina olin nagu siis kogu aeg sai pisikesest projekt, kes ütles, et pruukima käbid.
Ja meil oli isegi Tõraveres mingi kokkutulek ja, ja et, et on eksinud rühmitusi, tänapäeval pannakse Windowsi sisse ka Linux need, sest see läheb see Päpa kasutajate ringlev nagu laiaks või? No kes ikka ja kui me vaatame neid Linuxi kasutajad, sest nad on ikkagi niisugune null koma viis protsenti ja, ja, ja noh, ei jäädaks ongi supp, subkultuur on lihtne selle Linuxi niisugused Soomest ära toita, mis teil ju sageli tahtrahanda ja sain leid ahtranud ühelt poolt ja teiselt poolt noh.
Nonii noh, see oli niisugune, kõlas, kõlas kena uus uus, värske tuul oli mida, mida oli vaja ära proovida ja loomulikult Linuxi eelis oli see, et ta käis nagu PC peal, eks ole nüüd selle viisi rauda kasutada ja ja selles mõttes oli ta nagu väga-väga hea. No praegu on Linuxile on kuulus ning kelle jaoks on aastaid hädas olnud sellega, et see, see, kui sisenemisbarjäär on maru kõrged, kuidagi ei saa nagu liikuma ja käima, et üsna tol ajal oli kõikvõimalikud draiverid ja muu selline tants. Tegelikult me ikkagi rääkisime limaksist serveri mis etapis, et, et see töö tööjaama Linux ei ole nagu teema isegi ehk serveri peale. Point oli selles, et ma ei lahnud. Mis sellel hetkel ma ei mäletagi, mis oli mingi raha eest pidi ostma mingisuguse softi, et failiserveri ringiajad on siis need, ärme ärme tee, nii et panen Linuxi risti on ju, panen kasutame seda failiserverit, eks. No vot, see sealt hakkas pihta ja vot siis tuletan meelde, et see oli veel see aeg, kus mingiks naljakat asjadest taheti raha saajad. Nagu veebiserver näiteks. Ja fail, ma ei tea muidugi, kas tuli ikka väga kähku, ausalt öeldes, et elu ikkagi see mõte failiserver oli küll, jah, seal oli mingi, ma ei mäleta, mis asi oli sedapidi ostma ja nüüd see kommertstarkvara oli tol ajal ikkagi juhu. Tollis tuli ja tuli ropult kallis ja sealt tulin ütlenud, kui me, kui me hakkasime käbelist tegema neid esimese tulemusega nimega barrikaade ka üles, kui kuus on ja siis sel hetkel maksis, keskne tuleb, nüüd on maailmas kolm tuhat dollarit kolm tuhat dollarit, see on ju absurdne. Naasesime Linuxi tegime näo pähe, eks ole, ja ja müüsime viis korda odavamalt. Noh, selles oligi asja mõte. Nii et omal ajal oli, jah, see tarkvara oli väga-väga kallis, kuna kirjutajaid oli vähe ja see oli nagu eksklusiivne. Siis tuli Linux ja siis ühtäkki igaüks kirjutas. No absoluutselt tõene, et niimoodi ta kuidagi läks. No vot, igasuguste Komiorititega ja kest, anna kommi United ja, ja juukseid ja värgid ja mingil hetkel siis ei saa mainimata jätta ka olulist kummimatid, mille nimi oli WC Fauna ja legendaarne. Siis, mis see oli üldse ja ma olen käinud seal ja näinud, aga see, mis ta oli, see on raske öelda, mis oli, see oli jällegi hunnik inimesi, kes tegid igasugu asju kes pahatihti käisid kõnemoodi kõrtsides lihtsalt õnne või, või tegid niisama nalja või, või ehitasid lumelinna või aga, aga kuhu ja kuidas ta vot vanasti oli vanasti ja vanasti oli olid veel kompuutri, messid olid, olid tähtsad ja, ja siis oli kõikide IT-inimeste kokkutulek nagu Coca Fest ju oli tähtis kus siis sai ta kolm minutit kokku, et mingist hetkest muutus ta täiesti mõttetuks sellepärast et kõik pidasid ennast bitimeheks, see siis oli tuhat osavõtjat ja siis oli noh mõõdet. Aga sel ajal oli veel, oli, ta on nii-öelda konteinud, et Lindsay Essema, kes seal vetset vanuselitsus, nous, veets, fauna ja kuulutas, on ka mõtteviis ja eluviis, tähendab siin niimoodi niimoodi ei saa öelda, et kes see loll ja, ja sõjaväge ma arvan, terve meil oli ühel kompuutri, messil oli WC Fauna leviala kaart. Kui sa seletad Eesti kaarte vorstinahkade ausalt läbipaistvalt punaste vorstinahkadega tehtud, levis leviala kaart on järgulisusega, kunstirühmitus lausa ei, Einasse on, meil pakuti pudrumassil messil oma messiboksi ja siis me pidime selle kuidagi sisustama. Ja siis seal oli yks kompuuter, mis luges sekundeid tuleviku alguseni ja, ja siis oli WC Fauna leviala kaart ja siis oli kõik politseilindiga ümber tõmmatud. Ja see on ju puhas kunst, tavaliselt kui asi läks kunstiprojektide Irja ilmselt on, jääb kalliks häppeningi värk ja üks häppening asju sai tehtud, aga kes fauna nimi sai muidugi selle sellest alguse, et, et sealsamas hukkafestival aastal oli jalgpallimeeskond, kumb on siis okei, mingi kontorijookidest vaielda, siis me nimeks paneme siis et noh, et FC Flora Aivo on, eks ole, aga noh, spets folk, loogiliselt kõlab täiesti täiesti loogiliselt. Aga see oli ka viimane kord, kui me just jalgpalli mängisin. Ei noh, see vist ongi BC, mitte ehtsa ja kui ma siin teile niimoodi kuulan, siis sul on hästi palju niukest ühistegevust kumab siit läbi. Aga nagu arvutiinimene, kui tüüpiliselt inimene ei tegele arvutitega, sellepärast et talle õudselt meeldib inimestega tegelaseks, ta tegelebki arvutitega. Et kuidas sul see nõukoguga ja noa vahel, et ühest küljest nagu nagu inimesed justkui just nagu arvutid. No ja eks me olemegi imelik loom, et minu meelest pole kunagi aru saanud, mul on üks tuttav psühholoog, kes ütles, et, et on olemas insenerid ja on olemas kunstid ja head teadlased. Kumb sina oled arusa, kas rongi lihtsalt veel kord? See on täitsa okei, lihtsalt sulle, kui loomulikult käivad need asjad nagu käsikäes, et seesuguse programmeerimise värk ei tähenda seda, et mul on nagu introvert ja sa ei taha inimesi, eks ole, ma, eks see võib olla iga inimene areneb ka vaikselt, on nagu ma rääkisin, siis kui ma olin tsüklon kindralist istusin nurgas, et lugesin algusaegadel ja, ja ei karda, kartsime, kui telefon helises. Aga siis kuidagi kuidagi aru, kuidas asjad muutusivad ja aktsiate inimesed suhtlema ja, ja siis pärast juba mingisugust nii-öelda ühiskonda nägema ja, ja sealt tulid need ka niisugused riigi laiused. Vaata ka, et maailm saab laiused asjad, et tegelikult üks lugu, milles mida kindlasti maksavad rääkida võib-olla seda on räägitud ka. Et, et, et kust, kust otsast nagu Skype alguse tegelikult sai üks. Just see kamp kogunes, et see oli üks üks naljakas lugu. No räägib. Ja see oli üks naljakas lugu, oli, ma arvan, et nii mõnigi inimene mäletab, et kuskil
On mingi üheksakümne viiendal aastal, neil on näide, oli, oli kuulutus lehes, et otsime programmeerijat maksma viis tuhat krooni päev. Ja ma võtsin osa ja täiesti hääd, mäletan sõprade ja õppisime siit programmeerima, selle jaoks, et saaks osaleda, ei tee Apeerunive ehk B4 vajalikke kulutusi, mitut? Ei tea jah, sest näiteks ka Vilve vere rääkisid, tema sattus oma mingisugusesse Rootsi ametisse ka kuulutuse kaudu, lehes ilmus kuulutus, et otsitakse naisprogrammi. Tol ajal tundus nagu lehekuulutuse kaudu programmeerija otsimine täiesti mõistlik asi. Ei noh, see selleks, selles mõttes, et ilmus igal pool, aga viis tuhat krooni oli, oli kaks kuupalk keskmisest jõhker rohuga päevas lubati siis niimoodi. Ja see oli muidugi, mis selle taga oli siis see, et et Tele2, kes oli siis Eestis olemas juba ja Bonnier meedia kunagi eksi, siis seal rootsis sepitsid nii-öelda uue põlvkonna portaali.
Mille nimi on Seewride komm ja nii kui nad selle uudise välja lasid, et, et niisugune portaal tuleb palju siis nende turuväärtust doosist mingi poolteist miljardit. Mingi absurdne failis ajalooliselt palju ja see on sel ajal toimus. Aga mis jäi siis Eesti tuldi jutuga, et hea, et meil on igas tiimid olemas ja kõik kurat progenud siin juba tükk aega ja, ja, ja Itaalias on, meil on ja siis on Rootsis on siis on ma tean Leedus või kuskil oli veel. Ei taanistanis, et et kahte programmeerijat on veel vaja. Siis saab korda ainult ainult kahte, ainult kahte selle pärast mõtlesime paari head programmeerijat Eestist, vot teeme niisuguse tapmise, mis te Eestist otsisid. No vot millegipärast. Ja mina sattusin seda siin on kuidagi noh, nagu nõustuma ja siis lõpuks nagu projektijuhiks on jah. Et kes need inimesed siis nagu valida ja, ja, ja ja mis siis tegelikult ja siis oligi. No vot ja siis tulid need kuupoisid, enne, sai välja valitud ja siis ma sõitsin kõik need Itaaliat ja Taanit ja Rootsit läbi ja siis sain aru, et,
Peale Rootsi, kus oli teinud väiksemaks
Andmebaasi mootori, eks ole, portaal läks kõik, ülejäänud oli täit kräpp, mis tuli selle projekti jaoks tehtud ja et seda projekti päästejäänud muud üle, kui, kui kogu värk ise vea tooge. No ja nendel kolmel poisil ei olnud mingi probleem see nagu pihku võtta ja siis nagu nädala-paariga ära puhub, portaal kokku veeretada, kuigi nad seda PHP tundnud Sprammeerid oskad siis vahet ei ole, eksamist, kuidas see Pascaleeruvuse värk tol ajal käis, see isegi tollase juured ei, ikkagi sai nagu koormust võrreldes tavalise veebisaidiga kest, seega mingisugune probleem oli siis punuti pehappess kokku ja vaata, mis juhtub. Ei no Marenes andmebaasi pool või see oli, oli, oli rootsi tarkade inseneride poolt tehtud, et see oli täiesti kasutatav, nii et ilmselt skaleerus, ma ei mäleta, et sellega oleks mingit teemat olnud. Okei noot, aga siis oli.
Tulevane miljardär oli, siis oli seal telekoos, projektijuhtidest on, hakkaksin poisid meeldima, eks ole. Noh siis järgmine asi, mida, mida oli näha, oli Kazaania ja millega, kes ka usinasti äri mingil hetkel maa lits siin-seal uuesti proovisime. Ehk kui kaalukas aasta koguni neljandik interneti liikluses globaalselt oli ka, saab neljani neljandik. Siis me mõtlesime, et enne meid enne Kosovo cache'i. Ja et siis ma mõtlesin isegi, tegimegi, aga noh
Nagu ütlesin seda koguni välja, et meie risustasime teie Internetti nüüd veel puhastumiseks.
Et see oli natukene, proovisin ees imelikke asju teha, aga siis sealt hakkas skalbi asi tulema ja lihtsalt jällegi protokolli panekuks ka saali muusika jagamine, eks ole. Ei, mitte ainult see suvaline faili, aga see kõik proto BitTorrent või ah jaa, aga noh, tal oli nagu siuke hierarhiline mudel, eks ole, või siis vastupidi, tähendab.
Shea jaotud süsteemidel ta ise organiseeruma võrk oli, et sinna tekkisid mingisugused nõudid ja supernõudid on ja kes siis koordineerisid neid nõudeid. Ja siis dünaamiliselt ise moodustus, vaadates seda, et kuidas parasjagu kellelgil noh, nii-öelda võrgu ressursse käes on ja muud ressursse, nii et et sa laadsed ühe programmi alla ja panid käima, mõtlesid, et hakkaksin vaikselt toimetama, aga ühel hetkel olid võib-olla supernõudega siis oli siis, kui sa lits masina välja, nii saaks keegi teine, nii nagu Skype Kreesiks me oleme selle ämbri teie juurde korra tagasi, et sina oled seal projektivateks ja siis tegid selle valmis ja siis ei tekkinud niisugust mõtet, et peaks kuskil suure Rootsi korporatsioonist tegema, on ju kosmilist karjääri liikudes ametiredelil üles absoluutselt mitte absoluutsed. Mitte et see oli ikkagi minu jaoks niisugune aitamisprojekt ja, ja raha maksti ka ja no las ta olla see minna, kui sel hetkel ei, ei paelunud, mul olid ikka seal ikka kõrvaltegevus mul ja sul tolle põhitegevusena, arvan, et sel hetkel ma ikka eits seda riigivõrku ja juhatasin seda neid vägesid, et, et et sinna sinna sinna kampa käib üks supi kõrvale tegemist, sai tehtud oli, oli maipunkti ka muidugi. Ja seda eelmine kord sai põgusalt murdeealist, on pulss väga põgusalt ja et see läks sinnasamasse kampa lõpuks, et telefoon sai, sai selle omanik, sest see meil punkt ee jooksis. Kuidas sa seda tegid, see oli mingisugune standartne SMTP server all, mille peale leiame otsused täpselt nii, jah, ta algas ka pihta sellest, et et oli nii-öelda Redri ehk õigemini vastupidi ei olnud rendi räpp nagu Eesti punkte ja vaid oli ikkagi meilboks, aga ilma NATO meilboks. See tähendab, et igaüks sai endale selle luua. Aga sa pidid ikka enda meile neid kasutama. Ja, ja, ja siis Saltsaidoman mail punkti meilid kätte ja siis teine arengufaas oli siis tõesti beebi nägu pähe teha sellele kogu asjale, et, et seal oli veebi meiliaga ja see sai täitsa ise kirjutatud, see veebi veel ei oska, aga ma all olin loomulikult standartne kompati, aga see, olles ka ühe korra siukest beebi meeli nagu üritanud punuda sundiv lõputule ussipurk kõikvõimalikud variatsioonid sellest nimedest ja meeliformaatidest ja küll pani auto, pani niipidi kokku ja Fander vööd või naapidi neid kirju kokku ja kuidas sa seda kõike Parsid ja näidata üheksakümne viiendal-kuuendal aastal ei olnud, see maailm on kirju, võib-olla veab ja versioon maailma natukene teistsugune ja, ja seetõttu olid asjad lihtsalt, et olid mingid standart Laivoidmis Tiit, toimetame või toimetanud, siis oli ülakeha kõiki ehitisi, Volumusi isegi erandiks isegi ära. Nii et, et siin ei ole nagu mingit teadet. Sa ei läinud nagu ise seda, seda maailma nagu leiutama võtsid teki torusid kokku, hea, see on kogu aeg nagu veres olnud. Mingil hetkel sain aru, et, et, et see progemine, mütsid on kurjast, sest selles maailmas on kõik juba äraproovitud peaasjalikult, et tegelikult see kunst on need tükid üles leida ja, ja oskuslikult kokku panna. Siis lugupeetud härra Spolski on selle jaoks teinud stack õues lugu tänapäeval tikkidest Kokkola klubisid. Aga noh, tänapäeval on see midagi hoomamatut, tükkide arv on muutunud hoomamatuks ja, ja, ja väga raske on siis on ilmselt mingi kildkonna. Voolud on just see, kus see kunstidele muutunud kunst on jälle muutunud ja mis üldse on sinu jaoks nagu programmeerimine. Tänapäeval.
Eks ta kipub sihukene järjest igavam asi olema, sellepärast et vanasti oli ikkagi niisugune loometöö päris selgelt.
Nii kui hakkasid igasugused mudelid ja rukkid asjad tulema, siis, siis järjest rohkem hakkas mulle tunduma, et see on nagu rohkem kraavi kaevamine. Et, et seal tegelikult mingit arhitektide ja selle asja kätte ja, ja sina lihtsalt noh, täidab funktsiooni on ja sa ei ole eriti keeruline ometigi vist kõige niuks.
Kõige spektakulaarselt viis, kuidas on intellektuaalset ressurssi kulutatud, on vist üks asi, mis mängis läbi terminali videosid maha, Renderdada sind jooksu pealt nagu olekski ju arstimaal täiesti nagu töötusakent suuremaks-väiksemaks teha ja, ja töötab nagu kulda ja et seal küll mingeid arhitekti õppinu mängu seal Tšanräneela loomulikult nalja pärast mitmed ja ma räägin ikkagi niisugust raha eest või noh, tööstuslikust kogemusest või, või et, et sellel mingit konkreetset asja tegema, aga vanasti olid nagu mees nagu orkester oli ees ja mõtlesid ise välja, kuidas kuidas arhitektuur võiks nii-öelda välja näha ja, ja tegi nagu oma äranägemist järgi ja ja keegi sulle kobisenud. Arhitektid ja nüüd on neetult arhitektide ja ma ei tea, ühesõnaga kas loovust on vähemalt progemise mõttes päris programmeerijatele jäänud ja noh, kindlasti vähemaks kui, kui, kui see vanasti oli. See läheb kokku sellega, mida, mille Vene rääkisid, tema ütles ka, et et noh, see programmeerimis on nagu hügieenifaktor, et sul need huvitavad asjad tulevad mujalt. Nii see on nagu, kui ma juba jälle selle juba juurde jõudsime, siis ma küsin selle ka Vilve käest küsisin, et milline on ilus gootsivanud.
Nojah, sa küsid minu käest asju, millega ma tegelesin tõesti üle kahekümne aasta tagasi, ei, aga seda enam on mul põhjust ja aega reflekteerida sel teemal ja siis see, mis on nagu jäänud pinnale, see on oluline, eks Nojah. Ilus koodon, loetappuvad loomulikult, siin ei ole siin kahtepidi mõtlemist, see tähendab seda, et seda tähendab ma rohkem ei oska siin midagi öelda, versiooniks on loetletud loetleda ammendav vastus.
Endast uuringud tegelikult ongi, et mõned keeled on paremini loetavad kui teised keeled ja nii edasi, nii nagu päris keelte ugri türgi keel on väga halvasti loetav näiteks. Või siis arvab Hint, sundi on hindi on veel hullem.
Ja nojah, ega tegelikult ei olegi siin mulle eriti olulist midagi, ei aga no siis siis me jäänud ja jõuame ringiga tagasi selle selle va ID-kaardi juurde ju ID-kaardi juurde ja kuidas sa teed, kuidas, nagu see tekkis, kuidas see sündis üldse. Ja no ma rääkisin, käberist tegelesime siis mina tegutsesin kahel rindel, et ühelt poolt Eesti võrk, aga kogu aeg ma võin selle nagu infoturbe ja krüptograafia keskel tegelikult. Ja lisaks turgu turgudele, mis oli ka väga oluline, sel hetkel oli avaliku võtme krüptograafia nagu huvitav, selgelt huvitav ala, eks sellised pakkus pinget.
Oma rakenduste poole pealt ja ja küüris nagu jälgitud. Soojad kuidas, kuidas nalja? Üheksakümmend viis võib olla Rootsi Post alustas.
Oma on üks riidekaardi väljalaskmisega ja tekites piire kaardiprofiili, mille avaldas ja see oli, see oli niisugune asi, mida jälgida. Ja, ja tore asi, et noh, mul endal päris päris varas, ma tean üheksakümnendatel toodiuma Lektakos mingit kiib kars Lamber, see kiipkaardid öeldi, et vaata, mis elukas on siis, ma kirjutasin smart pardile programmi nimega Cliverk haarde. Savo Tartut ei, ta ei olnud joova head, polnud veel välja mõeldud selle sügavpuhas niisugune nii-öelda. Noh. No kuidas sulle öelda, noh, mälukaarte ei ole noh, talve protsessoril ikka seest oli see käsk anda teine fail ja tee kombineerisin lihtsalt sai logifailid ja asjad topib veel kokkukirjutuse Tzipi talule. Ei, ei, see käisid ka niimoodi, et sul on kaardioperatsioonisüsteem, mida sa saad kavandada. Ja spetsiifiline vastavalt kaardile, sel hetkel hea kaarte nagu ei olnud krüptokaarte ka ja pealegi said, said konsooli ette ja pea, ei noh, siis on mingid paid ajasid siis paljud tulid vastu, noh siis see ongi lihtsalt mingi piisk hiljem mingeid probleeme, millega seda siis mõnusalt tehasega. Ja ja noh, selles mõttes oli kiiportilitult tuntud loom ja, ja noh, vahel läks vaikselt edasi ja usun, et see oli mingi üheksakümne kuuendal aastal, kui teine Äripäeva lahti, sel hetkel ma veel lugesin Äripäeva millegipärast. Ja seal oli
Esimesel esimesel leheküljel oli kaane peal, oli siis ajagu viil, kuigi suur biit, kes oli sel hetkel KMA kodakondsus-migratsiooniameti asedirektor keda ma millegipärast tundsin. Mis ütles, et riik planeerib uut dokumenti.
Ja siis jutt rääkis sellest, et jaa-jaa, et näed, et esimesed passid võeti üheksakümmend kaks, kaks tuhat kaks, saavad need otsa ja praegu on viis aastat aega ja ja et kaev naas on moodustatud töörühm, kes siis suurim, mis variant oleks, et midagi uut uuega välja tulla? No siis.
Läksin kohale ja, ja rääkiski vajutus ja meil on töörühm ja parki ja siis meil on, näeb töörühma juht, siine värk ja siis ta näitas mulle salaja natukene on materjale ka, mis töörühmas oli arutatud ja, ja, ja kuna materjalide läbivaatamine, siis ma sain aru, et, et nende tehniline teadmus on, on noh, võrdub sinna allapoole nulli, eks ole, olles seal räägiti nagu kiibiga varustatud vöötkoodidest muuhulgas ja toredat seda mida nad siia siis arutasid, nende eesmärk oli ikkagi uus pass välja mõelda või? Jah, eks nad sinna kaardi poole mõtlesid ikkagi jah, päris õigel kohal Kaarli baasi poole mõtlesid ja see, see oli õige, aga milline kaart on kas või kiibiga varustatud tööd kood või mis siis on põnev, sellepärast et mina olen kogu aeg elanud teadmises, et mingit tehnikud panid kellelegi Kilgi pähe, et on vaja kaardi peale minna, aga sinu jutust tuleb välja, et see oli ikkagi nagu bürokraatia poolt tuli niisugune soov. Aga sel hetkel oli see väga-väga hägune soov, eks ole, et seal on, loomulikult olid igast variandid olid laua peal, aga, aga oli selge, et et kuna tekib niisugune suurem passi vahetus, siis on võimalik inimesi üllatada millegi uuega ja siis vaadata, mis maailmas tehnoloogia vallas toimub. Nõuate Nojah, ja siis oli päris selge, et, et sellel töörühmal KMA sisesel töörühmal nagu jätkata ei ole nagu mõtet, et siis oli ettepanek kohe kähku, et vaiet teeme, nüüd teeme nüüd ühe natukese laiema töörühma ja võtame siit nõus eksperte ka nagu laua taha, eks ole, et pangad ja, ja kelgud ja riigi inimesed ja abiveri inimesed ja nii edasi arutame seda asja, see tundub tänapäeval kuidagi nagu veider, et riik võtab pangad ja Telekom nagu laua taha mingit niukest dokument. Absoluutselt, ma arvan, siiamaani tunda, sel ajal ei tundnud kindlasti, et sihuke laiapõhjaline koostöö riigi erasektori vahel on alati edu toonud meile nii ühes kui teises. Vaata ka. Rõhutangi siis selle koha peal seda üle, et see on tegelikult nagu vana suhtumine, et, et see läheb nõu väga kaugesse minevikku ja et see on olnudki nagu just nimelt kasulik. Ja see on haruldane asi, seda mujal ei ole väga palju. Väga sageli ei ole seda, jah, sul on õigus selles mõttes, aga noh, Eesti on nii väike inimese põhimõtteliselt tead kõik, kes midagi teavad, on ja, ja, ja ei ole nagu mõtet kedagi kõrvale jätta, sellepärast et on erasektoris paras avalikesse, justkui me räägime ikkagi, miks te ekspert teadmisest ja ekspertkogumist mitte mingisugust Institut Šionaalsest asjast. Ja siis tuli tüdruk kokku ja siis tuli teha kokku töörühm arutas asju regulaarselt, tellis kaks teed, tähendab kaema maksistel telliti kaks dujad. Üks oli umbes selline töö, mida ansambel A aktsiaid saab raudtee läbi viis, kes uuris siis, et et, et milleks pikki võiks seda kaarti kasutada. Mis oli, läksid näiteks tanklaketi juurde küsida, et noh mis te tahaks, suits, töö, see oli seal juba selle kuskilt kätte saaks, praegu see on, kuid see on väga huvitav lugemine. Saab ikka kätte vast. Ja. No see oli väga ulmeline muidugi ja see oli vajalik tegevus või on lihtsalt sinine. Sellepärast on huvitav lugeda. See on vaata et kohustuslik samm on, ja seda nagu uurima turu nagu ootusi, et et mida siis tahetakse. Teine töö, mida siis me käberist tegime, oli, oli see, et me tegime niisuguse tehnoloogilise ülevaatajat, mida tänapäeva kiipkaardid on suutelised on ja kaasa arvatud mida on Rootsis tehtud, mida on Soomes tehtud on ja ja nii edasi on ja millised need profiilid on, millised tehnoloogiad Microsofti tehnoloogiaid, pyysiiesse, tehnoloogiaid, värgid, särgid, noh, niisugune tehnoloogiline dokument. Ja kurat, mingi üheksakümmend kuus, ma mäletan, ma joonistasin projektiplaani, et davai, neljateist kuuga toome kaardi välja kaasaarvatud piloot enne ja, ja kõik märgid kuidas viis aastat noh, nagu ikka tavaliselt nagu ikka tavaliselt, sest see oli väga oluline ikkagi samm ühiskonnas ja, ja tahtis, tahtis pikemat kaalumist peale selle tuli teha nagu seadusi juurde ja ringi ja seda kõike vedas nagu KMA. Ei, kindlasti mitte, seal oli ikkagi tiia Alcry seadust vedas selgelt. MKM, eks ole, või majandusministeerium Oo, mis, kust, kust, kust tekivad selle koha peal lünk, et kus kohas tekkis see arusaam, et niisugune asi nagu digiallkiri üldse on võimalik? See asi algas ju sellest, et no kaks dokumenti väljastama. August süst, nüüd äkki sekused, selgus, et kus dokumendi oleme vist iialgi.
No kindlasti oli mingi suunanäitaja Saksamaa, kes, kes võttis esimesena vastu digiallkirja seaduse ja kus, kus Eesti oma on ka väga palju maha viksitud. Mõnes mõttes meie meie digiallkirja seadus sai ikkagi nägi enne info hiljuti ilmavalgust vähemalt trakt sellest, kui, kui oli olemas Euroopa direktiiv üheksakümmend üheksa. Seetõttu meie seadus on, oli, tähendab siis mõnevõrra erinev Euroopa direktiivist, mis lubas igasuguseid lahjasid allkirju ja ega sellist koledust näpuga kirjutamist, näpuga kirjutamist kaasa arvatud Ene ja, ja, ja, ja mis siis lõpuks ei läinudki tööle ja, ja selleks sellest tuli ka lõpuks Seidasena, sellepärast et direktiiv oli väga lahja ja ja seda kehitati õlgu ja kasutati kasutatud ja tehti lahjasid Walkeri öeldi, et nüüd ongi kõik naist. Aga meil oli algusest peale, selles mõttes tänab tänases mõttes oli seadus ütles, et on ainult kvalifitseeritud allkirjad. On aktsepteeritud, et mingid lahendused allkirju nagu ei No kus seansid käsitleb keeles ja kas seal olid ka juba eksperdi sees, mis oli MKM endale nagu seal Gerdi, see seal olid Jõhvi, kes oli mina sari olnud. Aga kus olid nii-öelda kräptoloogist, kuni noh, juura teadlasteni olid, olid inimesed ja nad tegid seda tööd kaks aastat pakub, ehk nii et see on niisama põhjalikult ikkagi väga-väga rohelised. Jajaa see, sest veelkord, et väga palju peale Saksamaa ei olnud kuskilt ennast mitte võtta. Ja, ja need kontseptid, igasugused asjad
Mis nii mõnigi seaduspunkt oli, oli ikkagi inspireeritud seal nii-öelda krüpto krüptograafilise mõtlemisest, et nii võiks teha. Mitte et mitte, et, et see oleks mingi üldine määret. Aga su jutust ei kõla läbi niisugune kõikehõlmav sihukene, õilis visioon sellest, kuidas ühel päeval sünnib Eesti digiühiskond ja, ja kõik saab uueks loot loodud e-teenuste abil internetis. Ning seda võib-olla kuskil aju taganurgas on, aga no mis sellest ikka rääkida, asju tuleb teha, mitte. No nüüd ma pean silmas, on see trügiks õiges suunas ja sealt see toimetamine nagu käis suhteliselt niimoodi nagu samm-sammult niukest, nagu tegeleda konkreetsete asjadega. Suhteliselt et seesama ID-kaardi väljatoomine võib digiallkirja seaduse vastu võtta, mis meiega kaks tuhat võeti. Aga, aga noh, kui inimestel oli vahevahendit mis, millega seda digiallkirja anda, siis pole sellel seadusel suuremat mõtet ka, et see nägemus, mis Euroopas valitses ka selle direktiivi tegemise ajal, et oht nüüd tulevad kommertsfirmad hakkavad ilgelt sertifikaati müüma. Siis meil on vaja neid reguleerida. Noh, kuidas siin Eestis mütsist valk võtsime Valdo Praust minenukrakest turule teel, putkavõtjad, suured sertifikaadid. Looni, rohelised, kollased, mida soovite. Et see on see üks niisugune asi, mida jah, etteulatuvalt võib öelda, et, et niisugune nägemus, et me müüme sertifikaati inimestele ja siis siis nad kuidagi ostavad neid ja siis ja siis hakkavad kasutama, et see on üdini, siis pole asi, mis ei lenda kuhugi, et selles mõttes oli, mil Soome väga heaks näiteks, kes ID-kaardi tegi mitte kohustuslikuks ja pani mingi nelikümmend eurot kohe nagu hinnaks. Ja siis inimesed tulevad. No vaata, ID-kaart noh, miks fotod noh, lahe ja ma ei tea, miks ta lahe noh, saad kasutada äkki mõni päev kuskil ja mis juhtub, on tegelikult see, et teenuse pakkujad ökosu ID-kaardi toetame, sellepärast et nad teavad, et inimestel ei ole, teil on kliendibaasi ei ole noh, viiel protsendil on vaja, sellepärast nagu tõmblev inimestel ei ole ja inimestel ei ole sellepärast et kuskil teenustajale ja siis ongi nokk kinni, saba kinni on ja, ja see on see mudel, mis, mis, mis ei toimi, on ju. Et selgelt on ikkagi selline elektroonilisi identiteedi infrastruktuur, suur infrastruktuur või taristu ütlebki nagu seda, et sa pead kõigepealt selle tõttu imp taristu looma ja siis võib-olla hakkavad asjad juhtuma samamoodi nüüd nagu sa ei lähe ja ei proovi kuskil metsa sees müüja kilomeetrit maanteed on ja kohalikule metsa Janekile, kes ütleb, mis ma teen seal mile, -d, mis õudust küll, kõlalävi on üsna suur usaldus nende ekspertide vastu. Sa ütlesid, et krüptoloorid kirjutasin mingit punkti seadusse sisse ja nii edasi. See on nüüd ikkagi mingisugune usalduslik vahekord olema selle nii-öelda poliitika eest vastutava inimese ja selle protoloogi ja tehniku ja nende inimeste vahel.
See ei ole nagu põhjust. Ma ei, ma ei saa, see on nagu väga raske seda skepsist tekkida, sellepärast et kui sul on ikkagi Eesti paremat peal laua taga on misasja ka kahte või kõhkles.
Kahtlusi ja kõhklusi ja igasuguseid poliitika ja poliitikategijate tihtipeale sa ole, eks ta ole, mida seal ongi nagu see, et mida targemaks inimesed saavad ja mida rohkem eksperte on, seda rohkem võib-olla tekib diskussioon ja värki, aga kui sul on ikka väga käputäis neid ja vajasid ära teha, siis tehakse ära ja kui suur see ekspertide ring oli, kes seal käis nendes töörühmades ja seda asja nagu kujundasid siis on väga suur. No see ei saanud väga suur olla ja no see mingit numbrit nimetada laenu suurusjärku, see nagu nelikümmend inimest, mis kümme inimest, mis on viis inimest, pigem pigem ütleme viis kuni kümme inimest ja et minnakse, et võtmeinimesed olid, olid selgelt, kes olid. Ja, ja kogu see tarkus tugines tollel salapärasel Raadumis kiverissvee Oy. Ma rääkisin, et see oli lihtsalt algus ei räägi ainult kritoloogiast või noh, infoturbe eest laiemalt võib-olla ja ütleme. Noh, Viire ID-kaart näiteks ei ole mingisugune puhas infoturve, et seal on väga palju rakenduslikke ja, ja öeldakse, et sotsiaalselt isegi moment, et siin ei saa nagu rääkida. Et, et siin on krüptograafia, kas maailma hea ja on oluline arusaam, et teiste inimeste aruanded, kataloog, krüpto, et kui Saksa need asjad krüpt selleks maailma ära päästa ja sotsiaalne aspekt, kus unustatakse, et juurest ära. Siin. Jah, ma jään selle juurde, et sa ei saa nagu kilomeetrit maanteed müüja küla elanikuna, sest noh, sa oled seda müümast, selliseid on siis, ma teen selle maanteede Kuletajad, hakkad sellega noh, autoga sõitma, tõsi, mis asi on auto?
Nii et sellised ufo müümine, mis on täiesti mõttetu tegevus, et samamoodi on see, et et sa teed teed, valmis on ja siis laseda oot-oot, laseb müüki, eks ole, ja paned juhtimisviise juhi juhtimisõpetused Pästeemine juhtima õpetamise sõidu, koolitanud pysti. Ja siis ühel päeval võib-olla inimesed avastavad, et transpordist on midagi kasu oli. Aga kui sa lähed kigi, hakkad sellest pihta, et proovib igaühele jupi kilomeeter kilomeetrit nagu maanteed müüa siis see noh, see toimetamis ei saa hästi lõppeda, eks ole. Neid on.
Jällegi.
Üllatuslikult hakkab jälle aeg otsa saama. Tasapisi. Kuidas see küll niimoodi juhtub, neil ei saa, ma räägin, räägin küll kiiresti, aeg möödub lennates.
Vot ma tahaks käberi kohta küsida. See Küberneetika Instituut ja selle järelmid on olnud Eestis kuidagi nagu hästi niisugune. Vähe sellest, et ta nagu on mänginud kriitilist rolli, ta on hästi pikalt mänginud nagu niisugust olulist rolli. Ja sina oled olnud seal sees, oled sellest tulust ja veelgi olulisem sa oled sest ka väljas olnud. Oskad sa nagu öelda, et mis selle nagu asja niisugune, miks, miks see on niimoodi läinud, et miks, mis seal nagu, mis seal see maagiline Ingridi Hint on, mis teeb seda, et Küberneetika Instituut ja siis aktsia siis Küberneetika tema teised sellised, nii öelda järeltulijad? Miks nad jätkuvalt on olemas ja jätkuvalt toimetama? Nojah, raskem võib-olla rääkida ikkagi sellest osast, noh, üheksakümmend seitse, toimus siis suur jagunemine, Küberneetika Instituut läks kolmeks. Oleks ülikooli osast moodustades aktsiaselts on side, osaleks siis riigile.
Ja, ja muidugi see võib-olla kõige nähtavam osa inimeste jaoks on see käberi kas instituudi või aktsiaseltsi või just see noh, infoturbe progemise osa. Noh, seda tehakse meremärke ka, aga noh, sellest palju et meil on imenemiseda ja merele ei paista veel.
Jah, mul on endal au olnud, umbes ma ei tea, ikka üks, üks, kolmkümmend kindlasti inimesed sinna tööle võtta, omale Tapa-Tartu laululava ära kasutada ja nii edasi mis on nüüd suurem kui, kui Tallinn isegi isegi niimoodi. Nii et see oli väga toredad ajad. Aga mis seal fenomen, Nikon fenomen on see, et, et.
Põhimõtteliselt peetakse ka jõed nagu ainukeseks firmaks Eestis, kes oskab turvaliselt kogeda seda teab midagi info turbest. Ja, ja seetõttu on neile ka sattunud niisugused tegevused ja projektid, noh alates x-teest ja, ja, ja Lubits olete millegagi kus, kus see lõpetades siin aastaid hiljem muuhulgas näiteks kus, kus see turvalisus, krüptograafia komponent on omal kohal ja, ja seal tõesti inimesed kes tegelevad puhtalt teadusele, ehk siis kes ei kirjuta koodi, kes, kes ongi, seal on oma teadusosakond nii-öelda ja, ja oma teadusdirektor ja, ja ongi noh, inimesed peavad seal küber ja, ja ülikoolide vahet niimoodi kes seal teadusega tegelevad. Ja, ja noh, selline sümbioos on mida otsitakse nagu Spunkil mööda mööda Eestit ta Teaduste Akadeemia president ei väsi rääkimast, et küber on fenomen, et algsuur, suur erand selles mõttes, sest mis on just miks siis miks, miks sul Helmes tööle endale teadusdirektorit ja asunud ka tegema? No vaat see ongi, et kas sa teed kõigepealt nii-öelda et teadust ja siis hakkad seda rakendama ühiskonda või sa proovid seda vastupidi teha, et see kõigepealt võivad ilgelt proge ja, ja, ja siis mõtled, et poleks ikka teadust ka kuidagi. Et see päris niimoodi käia, see need juured on natuke sügavamad. See põhimõte, et sa, sa pead.
Niisuguseid infoturbe ja krüptograafia alusteadmisi ikkagi omama, et, et, et seal isegi toimetada. Naasin just täiesti kogemata kiverid töökuulutust paar päeva tagasi. Eks see oli ostjaks projektijuhti. Projekti projektijuht, juhi peab olema sisa sertifikaat. Või siis laed projektijuhile olevat siseseks. Halleluuja.
Okei, no see juba vastab küsimusele, sellepärast et nojah, et kui sul on nagu nutikad inimesed on koos ja teadust tehakse, et siis see hoiab nagu organisatsiooni koos. Ma ütleks, on ja ma, ma, ma, ma mäletan, see oma omal ajal oli see kõik teadsid, Martens tuleb jälle, meelitab pudelisse tööl. Nii nagu nagu vastik jaure.
Mul mul oligi, äriidee oli väga lihtne. Äriidee oli see, et ajad kõige targemad inimesed ühte suurde ruumi. Ja annab neile mingisuguse teema kätte, sest olevat plaks tööle, isepäev, jalad seina peal. Töötas väga hästi, töötas siin, see on võitlev valem, nagu mitte ühes kohas absoluutselt, see on väga-väga niisugune tuttav.
Mingipärast pärast lugesin seda kuskilt raamatust, et niimoodi tuleb käitud, käituda siis ma olen täpselt niimoodi käitunud.
Täpselt nii. Täpselt niimoodi on tundeid. Edasi.



