\index[ppl]{Tamkivi, Sten}

\question{Kuidas sa arvutite juurde jõudsid?}
                 
Kõneleja 3:
Tere, see siin on memm, kopi. Täna on meil külas mees, keda ehk kõige paremini iseloomustab tema võime teha nii. Tehnoloogia abil hakkab inimeste elu paremaks minema. Selles on osa ettevõtlikkust, inimeste ja organisatsioonide juhtimist. Ja lisaks kõigele muule on tegu inimesega, keda ma tean juba üle 30 aasta. Külas on Sten Tamkivi head kuulamist.
                 
Kõneleja 1:
Tervist minu nimi on Sten Tamkivi.
                 
Kõneleja 2:
Oleme kuulnud siia suurepärasesse kontorisse hall sügisesel päeval rääkimaks sellest, kuidas asjad alguses said ehk siis olulistest asjadest. Kuidas teha, et autopoisiks olla, mina olendatus Tartus sündinud ja kasvanud kuni tegelikult selle saate mõttes kogu selle aja, kuhu sa oled natuke nagu noorema põlvkonna inimene kui ülejäänud rahvas, kellele ma rääkinud olen. Sündinud 78 ja loojub jah, sellepärast kuvas saateid kuule toe need siis minu jaoks enamik neid inimesi.
Et kui üheksakümnendatel, kui mina sattusin või kaheksakümnete lõpul sattusin niimoodi arvutite interneti juurde, siis need olid nagu sihukesed legendaarsed juba establisnimed, et kellega ma võib-olla veel mõnede elu jooksul hiljem tuttavaks saanud ja avastad, et oh, et ma ei tea, Madis kaalu nagu päriselt ka olemas ja ja ei olegi nii palju vanem kui maailmas just nimelt takkajärgi need vahed lähevad nagu kokku paar-kolm aastat, vahet ei ole enam nii suur, aga tol ajal tundusid tõesti kuludena. Kuidas räägi oma sellest kujunemisloost, et siin mõned on rääkinud, et nad mingit hirmsat olümpiaadi hundi toonud ja teised on rääkinud, et neil huvitasin kuu raamat, kuidas sina selle värgi juurde jõudsid? Mina olin seal paati nagu erinevat suunda, et üks asi on see, et ma olen pärit teadlast suguvõsast või mu isa ja vanaisa füüsikut mõlemad ja ma kasvasin üles lapsepõlves Tartus FIE rajoon, mis on füüsika instituudi jõgi mis tähendab seda, et sul kõik lastekaaslased, kõik naabripoisid, kellega õues mängida, et kõik on kuidagi seotud tõenäoliselt füüsikainstituudiga. Ja, ja noh, ma ei tea, kui seal 80 lõpus 90 alguses Füüsika Instituudi elektroonik paneb majadesse Piaat kaabeltelevisiooni ja füüsika instituudist, saad esimeste arvutite ligi ja kõik seda, kui asi asi oli sedapidi seotud. Ja teine ma õppisin mina ema gümnaasiumis või tol hetkel, kui ma läksin, 85, oli Tartu teine keskkool mis on mõnes mõttes nagu humanitaarsihukesed Tartu, nagu tolle aja siukestest esikoolidest, et Navaid esimene keskkool ja minu ema teine keskkool Treffner oli palju selgemalt reaalainete kallakuga, aga siiski mina ilmas ka oli selline reaalainete suund täiesti olemas ja.
Et mina olen, see oli ka suund olemas. Sedapidi ma käisin olümpiaadil ka varasemaid pilte, mis mul on kuskil on Maidan mingis vanemate või, või vanavanemate albumis olemas on selline mustvalge foto, kuidas keegi tõi sinna Miina härma või teise keskkooli algklassilaste näha kooli arvuti Juku. Kuidas selline? Ka mustvalge pildi pealt on näha, et silmad juba läigivad, kui lähedalt näed.
Fiisoris esimene arvutikogemus. Ma arvan küll selline, kus nagu sa oled nagu päris oma aega siia sina ja arvuti, ehk siis et nagu isa pani, pani kuhugi mingil õhtul kellegi kabinetis kuskil asi üldse välja kui arvutiaega ja et seal istuda neid kohti oli veel, et, et selles suhtes mul näiteks pinginaabri isa töötas Tartu Ülikooli raamatukogu sotsiaalMe, käisime arvuti taga ja seal üheksakümnete alguses ta tema hakkas isegi tegelikult pidama vuti kaupu müüma või kooperatiiv poodi pidama, et siis siis oli Amisjoni kodus kaevuti ja siis oli üks koht, ma mäletan, oli esimene koht, kus ma Amingat nägin, oli aasta või kaks noorem koolivend Lemmik Kaplinski, kellel oli isa ilmselt siis kirjanikule kuskilt maailma pealt seda oomika abiga kätte saanud ja siis olid kuulsad tähetornis oli veel mingisugused eksis Noorte tehnikute maja, kus olid jamad. Ehk siis ma ütleksin, kui niimoodi vaidlema hakata, siis just võib-olla iseloomustabki see, et kellelgi ei olnud nagu püsivat kohta, vaid otsiti seda aega, kus sa saad ja, ja mida tehti Tartu-Tal oli ikkagi toimima. Siis noh, see on see koht, kus ta ikkagi nagu on Eesti nagu Cambridge või, või, või projekti, et kui sul nagu väike suhteliselt väikeses asulas on nii domineeriv ülikool, siis see tähendab seda, et noh, et ja kogu see interneti algus ja kõik see, et et sa saad akadeemilistes võrkudes, hakkas peale. Et mäletan Tatu püsiühendused, internetiühendused, mis tekkisid, olid ju ka palju satelliidiga lootsi tähetonnist satelliidiühendus ehk siis enne kui tekkis Tallinn-Tartu liin, tekkis nagu Tatu sõimlikku otsi ülikooli, mis iganes Stocktonis isegi selle üle kosmose just. Aga sinna pole veel sinuga veel jõuame. Kui sa ütled, et sa oled nagu sinu arvutiaega, et.
Kui palju ja kuidas sa sõitsid õpetust sealt nagu mis teed siis lihtsalt vajutasin nuppe sees. Ma arvan, et see muster ikkagi enamikel inimestel täpselt samas alguses tahab mängida ja siis sa tahad aru saada, kuidas neid tehakse, siis hakkab natuke programmeerime. Mul oli nii, et ma üheksandas klassis läksin, pärast kooli hakkasin programmeerijana tööl käima üheksandas klassis ja ma olin 15, ma arvan. Siis ma tolleks hetkeks olin, ma arvan, kuskil kaks-kolm aastat niimoodi omal käel pagenud. Ja see oli jällegi, see oli mu isa oli Tartu teaduspargi asutaja ja siis siis seal teaduspargis tegutseb nagu mitmete firmasid ja siis ta küsis, et kas ilmselt ma eeldan, et ta küsis, et kas keegi seal poisile mingit kasulikku tegevust leiaks. Ja siis oli sihuke hulljulge mees nagu valettindabamov, et üheksakümnete alguses teatavasti toimus Eestis ohjeldamatu metalli jäi, siis oli Tartus sihuke metallikonglomeraat nagu põimeks ja peksid, oli, tütaksime imekslaata kus tehti igasuguseid asju, põhiliselt pandi mingisuguseid pissi kloone kokku. Siis olid seal mõned inimesed, kes pagesid, mingit projektijuhtimistark, mõned inimesed, kes pagesid, mingit raamatupidamistarkvara, näiteks Tarmo Tali ja, ja siis ta palk, Tinn või vàlja palkas mind nii-öelda Promee jaoks, aga tegelikult oli see sellise koolipoisi nagu päästa midagi, ma just käisin ka. Ma ei usu, et sealt midagi üldse. Ma tahaks juhtida. Aga, aga külmal seal haldasin kohaliku ajuti kui aitasime arvutit kokku panna või, või hakkasin BBC pidama ja seal oli selliseid tee, mida tahad või maalimas kõlab nagu mõnusa maailm. Aga selleks ajaks on ikkagi olema piisavalt testikulaarset portituudi, väitas programmeerija raamatute järgi õppisite internetti VPS-i olnud, eks ole, ja, ja seal oli noh, ega neid raamatuid ei olnud ka ju seal alguses kätte saada ja selles suhtes ma kuskil nagu mingites see vist see, mis oli seal noorte tehnikute majas, see oli vist ajuti rinny nime all, aga kuna ma käisin seal nii hooti või see ei olnud nagu niisugune suhteliselt korraliku nagu progrimise alg, alghariduse ma mäletan, et ma olen kirjutanud ka paberi peal koodi, et kui sul parajasti see periood, kus sul ei ole ligipääsu ühelegi arvutile, aga vaata, mis te A ja arvutustehnika ja andmetöötlus, andmetöötlus või näiteks mingi ajakirja, mingid mingid sellised paberimaterjali, siis sealt nagu üritad midagi nagu tuletada või, või teha, teha midagi mõttes valmis, enne kui arvuti taha saadeta. See oli vot see on nüüd läbiv joon kõikidelt, et kui me praegu räägime sellest, kuidas õpetada lapsi programmeerima siis nendest juttudest tuleb järjest välja, nagu keegi ei oska, kuidas nad õppisid programmi, kuidagi imbus läbi naha või läbi õhust või kuidagi tuli. Kuidas see nii on.
Üks asi, mida ma olen näiteks mõeldud, on see, et et eriti, mis puudutab neid nii-öelda kooli arvuteid, nõukogude aegseid aga ette ja, ja Yamahhalsid ja jukusid, et seal oli ikkagi arenduskeskkond, oli esimene asi, kuhu see sisse ennast Puttide alguses need suhteliselt raske oli seda arvutit kasutada niimoodi, et sa komistaks nagu arendusvahendite otsa. Et kui seda ära võtta taifuuni, siis pead kurja vaeva nägema, et saada üleskeskkond, millega sa saaksid siin midagi teha. Et, et see on kindel niisugune muutus. Ja ma mäletan, et tegelikult just, et see kooliarvutite ajastu oli nagu, nii palju põnev
Et kui mul oli onu, ostis endale kunagi mingisuguse läptopi, mis oli selline Tossiga ilmselt mingisugune, ma ei tea kompaq, et kui ma tal külas käia, nagu seda kasutasin nagu ikka tahad ikka arvutiaega ja kuna seal ei olnud ühtegi arendusvahendid, siis no mida sa teed seal kaua tossi, tossi, direktori uus nagu ongi nagu et see väga huvitav ei ole tekstiga traktoreid, niuke, äri, arvuti on ju, et siis ma hakkasin nagu just ükspäev mõtlema, et see, tegelikult see hetk, kus ma esimest korda sattusin, kasutame arvuteid, mis kus ei olnud, mis ei olnud nagu eelkõige arendamiseks mõeldud. Mäletan, sinuga seoses oli ka üks, kui me kunagi tuttavaks saime, oma võrus, sattusin sulle külla, kus said laenanud koolist suveks kooli auti klassist üha kati koju, nii. Ja sellise ekstreemne juhtum, kus, kus selleks, et üldse midagi teha, sa pidid kõigepealt eksis sisestama, ma ei mäleta, kes sisestas teisiku koodi selleks, et saaks punkti, kuhu saaks hakata kirjutama nagu ilm loetud koodi müüb. Hullus aga et noh, kui koolipoiss oma suveajal istub ja kuueteistkümnendsüsteemis koodi sisse toksib arvutisse, selleks et siin on midagi mõistlikku saaks teha, siis selgelt su suhe selle arvutiga on teistsugune kui lihtsalt meediatarbimine. Seda küll jah, see lõksu suhe ilmselt teistsugune, et see on hästi oluline asi, et tulla suhe arvutisse või teine. Ja siis kõik muu tulenes sellest, eks ole. Ja üks asi on veel, et, et sellel arvutite lihtsusel või ütleme piiratusel see, et kui sul on 25 rida korda 80 tähemäki, on su nagu visuaalne mängumaa või hiljem mingisugune AEGA või VGA graafika ehk siis see teeb selle kõik kättesaadavaks, tegelikult nagu ka laps suudab kogeda midagi, et see on nagu nii palju rohkem jäetakse nagu fantaasia jaoks, et kui keegi nagu tekstirežiimis mängu siis see ongi nagu nii-öelda selle arvuti tipptase. Kui täna keegi võtab mingisuguse koduse mängu PC, teeb seal midagi tekstirežiimis, siis noh ühesõnaga kõik, mis ei ole nagu tohutult videokaardi võimalusi kasutav kolmdee, Vendedus reaalajas on, tundub nagu naeruväärne, aga tol hetkel see kõik, mida sa ise suutsid oma kätega teha, ei olnud naerda. Vot täpselt. Ja selle juurde käis mingisugune raamatuhuvi mingisugune, sul pidi olema siis selles seltskonnas, kus liikusid seal liikumi ingliskeelset kirjandust.
Kus ikka, aga ma no jällegi Miina härma oli koolil ja selles suhtes nagu äge äge, et et enamik asju, mis seal toimusid, olidki nagu see nagu sihukeste nagu asjad, mis, mis nagu jäljendavad ma pigem nagu teiste õpilastega. Kooli bände oli kõvasti, ma ise üheski bändis ei olnud küll. Aga ma ei tea jällegi 90.-te lõpus piss näiteks kellega ma väga palju hängisime, kes usatega siiamaani nagu läbi käidud, siis see oli nagu mina ilma kooli bändist välja kasvanud asi, aga mis oli ka nagu otsapidi väga elektrooniline ja väga niisugune ma ei tea, avas ma ei tea, arvuti ja muusikaseose maailma minu jaoks näiteks on ju raamatut mõttesse, mida loeti. Jah, aga see ei olnud nagu ma ütleks, et ma ütlen emps üks küberpungi Šveits Fiction juurde jõudsin pigem juba 90 teises pooles ja siis, kui ma Ameerikasse sattusin, et enne seda võib-olla ma lugesin pigem kääbikute tult, Kiili kui kuiv Priveks, siis programmeerijad. Erinevus on selles ja see oli 93 kus müüakse ülikooli, siin oleks seal pimexis. Eks sellest korra veel, et miks seda softi selle prohveti, kas see oli nagu enda tarbeks sellest äri teha? Minu arust Eesti IT-tööstuse ajalugu on ju sellise lainena, et selline 80 90 alguses oli see, et kui sa alustad täiesti tühjalt lehelt kõigile arvuteid vaja, siis kõik teid, juppe, võidalveteid kokku, et oli püheks saata samal ajal seal mingisugune kõrval majanduse või kuskil seal Hoioodi. Vastan Macro link oli Tallinnas Asto data ja kõik need kõik tegid sama asja. Siis liiguti tasapisi tarka kihti, aga see pigem oli selline. Ma ei tea, riigil raha ei olnud pankadel raha. Kolijaga võib olla huvi olnud hästi palju süsteeme, ainult ise tekkisid need firmad, kes nagu ajendasid
Nagu teenusena ja kogu see, ma ei tea helmeste ja meediat, see laine on nagu selle kõige tugevamad näiteid võib-olla. Ja siis lõpuks tekkisid nagu esimesed nagu või meistri, miks on täna muutunud nagu toodete ehitamine? Et võimeks oli naljakas hübriid, sellest võib üheksakümnendates oli, nii et ühest küljest oli seal see arvuti, mis oli see, mida kõik tegid ja kus nagu tuli, põhiline käive. Aga teisest küljest hakati tegema nagu ikkagi tootena, need olid nagu mingid asjad, mida Nad lootsid ilmselt klapi peal nagu müüa ja et sa ostad. Ma ei tea, kes need Eesti Melita teevosoftid ja kõik need raamatud, mida just hakkab, mingisugune torgib, seal tekkis ja osad neist ju siiamaani. Et, et seal ja see projektijuhtimise tark, kuidas ma mäletan kaks kaklejat nimega uimas ja Jürgen, kes siis minu arust tegid sellest samal ajal Tartu Ülikoolis oma magistritööd vist, noh, et projektijuhtimine, kant kanti graafikud ja selle kohta eesti keeletark, et noh, ilmselt see oli nagu sihuke akadeemiline asi, mida nad lootsid müüjaga. Jällegi ma ei mäleta, et sellest nagu mingit suurt äri oleks nagu tekkinud ja see hübriidid ja pidamine suhteliselt jabur, et see oli nagu mäletan. Ma ei mäleta, kas oli minu arvutiga Tarmo Tali ajutiga või mõlemaga istusime kõrvuti, et siis oli see, et seda hommikul tööle või pärastlõunal ja võtad, nüüd siis hakkaks pagema ja selgub, et keegi soovitust mälu ära müünud näiteks. Et mälu, lihtne võrk, korter on sellistes üles täpselt ja siis sisse nagu tegelikult nagu alustatakse sellega, et okei, et enne oli kaheksa megabaiti mälu, aga äkki laost nelja megabaidi see kivi kuskilt leiab. Ja siis noh, jällegi siukest tarkvara maailma ja. Sest et läbipõimumist oli mättad, istud seal ja nokitseb midagi teha ja siis tuleb siis Jaan, Tallinn, kellest ma olin kuulnud kosmonaut ja, ja noh, legendaarne mängu tiimi siis jaan tuleb monitori ostma, sest et niisuguse põhikooli või keskkoolipoisi käed värisevad, et ma nagu vist vara müügiga tegelenud, aga siis jube põnev, kuid võitsid ligidal. Tundus olevat seda porist koostatud võrrandi läbi käinud, tol ajal oli arvuti arvutimüük, oli nii ägedama originaaliga asi et sealt selle marginaali sisse mahtus igasuguseid naljakaid asju sai teha, mingi tudengi, pidada endal projekti softi kirjutamise ajakirju välja anda. Et selles mõttes oli, oli ilmselt siis jõuavad siis ka siuke vaata, mis juhtub ja no miks mitte, eks täpselt see noh ja ma ei tea, minule on õpetanud see, et näiteks ma arvan, et ma kindlasti täna võtan oluliselt parema meelega endale praktikante intern, töövarjusid, et see, kui palju mind see mõjutas, et see võimalus, mis nagu valja andis sellega, et sa saad lihtsalt.
Ma hakkasin mul isegi palka ju selgelt nagu selgelt nagu olukord, kus noh, eks poleks mingit vahetut, kui nad ei oleks mulle palka maksnud just tuppa sisse lasknud, siis täiesti puudu. Ja seal tegelikult sealt edasi, mis oli, huvitav, oli see seal ma sattusin esimest korda võrkude juurde, et enne seda oli ikkagi arvuti, oli nagu iseseisev eraldi olev asi. Siis kui see oli 93, et siis oligi täpselt umbes sel ajal, kui Tartusse ilmus internet minust, ma, ma pakun, üheksandate võis olla esimene eraettevõte vähemasti teadus pakkimistele. Esimene niisugune Elo ettevõttete koht, kus, nagu mitte ülikooliasjad sattusid võrku. Ja see nägi välja nii, et laseb nagu kuskilt tuleb.
See jadaühenduses nagu võrgud olid, onju, et kus oli otsast terminaator, muidu selges eaka vanemate arvuti. Kuskilt läbi seina tuleb see nagu, nagu ots oli ju, sa ei tea, mis masinad veel selles jadas on, kõik on nagu ühes võrgus maja peal laiali ja siis siis meil oli 93. aastal oli meil nagu püsiühendus internetti lünksi põhine nagu veeb, enne kui need Skype välja või mosaiik nagu välja ilmus. Ühest otsast ma pidasin piibi essi P9 ta nime all ja teistpidi oli meil olemas püsiühendus, kus oli võimalik Maida tuumi demo saada, Wulfelistani, mingid asjad juba FTP ka kätte ja siis panevad BBC üles, mis, kujuta ette, et paljude teiste piibitekstide jaoks oli see kõik nõuske modemiga sihuke loksutamine, mingi faili leviks, aga meil oli nagu niisugune, me olime nii-öelda pumba juures. Ja, ja teine asi, mis ma mäletan, sest sest nagu kuskilt see arvutifirma ja asjad, mida nagu hobina tehes oleks hoopis teistsugused, välja tõid. Meid oli sisevõrk, sisevõrgus oli novelliserver, millel oli 300 megabaidi kõvaketas, millest noh, tööasjadeks oli kulutatud umbes paarkümmend megavatti kood, mida seal kirjutati ja paned nagu hinnagi ja Vöödi fail onju. Ja siis siis ülejäänud oli mingisugused hollandi tüübid, laadisid selle alati öösiti mingit akva täis, sellepärast et kui sa teed ukse lahti kuskil Ida-Euroopas, kus ei ole ilmselt kaid, elektuaalavate viskus Euroopas ei olnud neid reegleid nagunii rangeid, ise mitte nagu ligilähedaselt, rääkimata siis veel üht Eesti Vabariigist ja siis siis tuuled selle läbi ja vaata, et mis on need asjad, mida piibeeessis nagu ülejäänud Eestiga jagada. Aga siis see tähendas seda, et sa pidid nii Interneti otsas kui BBS-i osas ikkagi päris nii ei ole, ovat avad FTP pordi, siis sina hakatakse kohe mingeid faile toppima, et see pidi ikka mingi võrk tekkima. Kuidas see tekkis? Mul on võrgustik selles mõttes.
Ma arvan, et seal oli Youznet. Ta oli ja just need mingeid uudisgrupid oli võib-olla esimene selline kogukondade teema kooma, nagu rahvusvaheliselt sattusin. Eesti Fidanetiga õpid ka, eks ma arvan, et seal oli see õuelapp ilmselt oli täitsa olemas, et kus nagu ka eesti filoneti gruppides arutati juba, kus nagu internetis käia ja kus keegi istub, et et kus, nagu ta akva ligi pääseb, kõik see, et see
Minu arust on minu jaoks igasugused sellised reaalajas, Chatrumid tulid nagu hiljem, et mingisuguseid vändum ja need jututoad, et see Aias siin mingitest kanalitest ma ka istusin, aga ma selle kohta isegi see on, ma ei mäleta, et oleks näiteks mingisugused tohutu nagu side või mingi sõpruskond tekkinud, et see oli nagu rohkem sihuke ajada. Ma ei usu, et oli see peamine võrgustiku ehitamise ja kust sul tuli sihuke mõte hakata öösiti tema valitsuse tööle.
Vot, see on hea küsimus. Ma mäletan, ma, see idee müüsin küll maha sellega, et siin jube kasulik võimekuse ta nagu tarkus või kus nähtaval. Aga, aga pagan teab, mis võib-olla see oli äkki see, kuna laos oli modem, et siis mida teha saab või no mingi tõesti mul kodus ei olnud arvutit 90. lõpuni, et siis ei olnud nagu asi, millega ma mainin piibesside kasutaja, nüüd tekkis võimalus siis ise püsti panna, et see oli kuidagi. Ju ta ikka sedapidi oli, et laos oli modem. Sellega sai helistada sisse teistesse piibeeessidesse, mis juba tallis olid üks, kusjuures enne kui sa külla tulid, siis ma hakkasin mõtlema, et ma mäletan Fidoleti nõudi numbrit peast jätkuvalt, mis oli kaks, neli, üheksa, ükskord üks, kaks punkt kaks, üks oli vist Jaan Kuulmann, kaks sellist Tatu tsoon oli kaks suhteliselt õe veel ja kõik need Tallinnas just ühe ühe all. Ja, ja et siis noh, et oligi, et kui me BBC püssi panin, siis selgus, et on võimalik olla teine pingestatus. Üsna mängu alguses siis pimesi TPS n utsitada.
Tuleb siis. Mis tema number oli, ei tea, aga ma mõtlen, veiko tundes võis olla 66 meetrit vett. Nii palju, et võib-olla see puhtlida. Võib-olla siis see politseile püsti ja, ja sul üks liine, üks vorme.
Hiljem olin uue selle kohta, kui ma olen küsinud, et mida inimesed seal oma selles Peebeeessis võitsid, siis reeglina inimesed ütlesid, et seal üleval nuku asjad, mis nagu endale nagu kasulikud nende huvitavat tundus. Inimesed tõmbasid kuskilt alles, tegid teistele kättesaadavaks, mis sorti seal oli?
Ma arvan, et seal oli, oli suhteliselt kaootiline sörk, et puhtaks selle failist oleks kuskil Audes ses suhtes seal oli ilmselt mingisugune noh, mingi mängude teema siis mingisugune jutt, lülitite teema oli kõigil on ju, et sul alati oli noh, jällegi ressursi kitsikuses on mingid asjad, mida, mis alati oli nagu huvitav või vajalik tehnoloogia, näiteks pakkimisalgoritmid, onju, et alguses on, sul on, ma ei tea. Chip ja ette siis tuli midagi uuemat sihukesi Jutilitis nagu asjad, pluss nendega nendega oli see rõõm, et meil olid väiksed, et see kiiresti liigutada ja siis kui midagi uut, noh, isegi kui see nagu osa sellest masinasse kuidagi sõid, siis olete juba sinna mingisugune pakkija. Mäletan mingi asi, mis tossiss, mälu, võimalus kasutada, mis nõu piirangutest mööda niuksed kohe käima panna, tööstuse Surfraktist loosungiks.
Ja jaa, jaa, Fidoneti gruppide mõttes oli ka ju ikkagi noh, tegelikult huum oli ju ikkagi kallis ja see modemi aeg, et siis, kui sa vaatad, et sul on mingisugune öine meili sünkimine on läinud jälle mingiks tunni ajas, eks, et siis siis jälle vaatad, et võib-olla kõiki neid gruppe pole vaja, mida ise ei loe. See pool oli ka, et see nagu reaalselt käib üle. Seda vaatad vahepeal, palju sa palju sellest aimu, kes küljes käivad, kes lugemuski lihtsalt mingid numbrid eristuvad? Jah, minimaalselt mäletan sellest või, või ma ei tea isegi, kui palju sellest teadsid, et see pigem ma mäletan seda tunnet, et oli nagu, kui sa ise juhtumisi seal olid, noh jällegi, kuna ma öösiti käisin kodus magamas on ilmselt inimesed, kes kuskil kodust välja helistasid, siis siis tegid seda hiljem aga et kui, nagu sel ajal, kui ma arvuti taga olin ja siis see modem nagu kõne vastu võtab, et siis nagu põnevusega vaatavad, mida see inimene teeb seal ja seda sai vaadata? Ma ei tea, kas see oli juba tol hetkel, oli äkki Windows või oli see OS kaks, millega ma katsetes, aga mingil hetkel ma sain nagu sihukese vist OS 200 jooksutada jõudossile tolmu esinenud või mingisuguseid lossiprogramme kuskil aknas, et mul oli multitaskis asi taustad, et isegi kui ma tööd tegin. Ja teine teine asi, mis meil oli, oli välja helistada, seal alguses oli. Minu arust istus Tartu Teaduspark kõige Tatu kõige vanema telefonijaamad, aga sellist legendi õigemini 50. Ma ei tea, kui palju sellest, tõsi on, aga mingisugune viiekümnendatel püsti pandud kanal ka. Ja siis me üritasime ikkagi seda nagu lahti häkkida ka nagu ühte ja teistpidi üks asi, mis minu arust veel kaevad, töötas, võib-olla niisugune linnalegend, et et meil, kui sa võtad kettaga telefoni ja teed nagu üks ühest kuni üheksani on nagu klõpsuda ööbin ja 10 klõpsu null. Ja siis me lugesime kuskilt, et kui, kui sa teda 11 klõpsu, siis sa saad kaugemale trankigetu. Ja siis me tegime 11 klõpsu, saime teise tooniga, Eli, helistasime mingisse Hongkongi, BBC ilusti tulnud kunagisele Pedalvet. Eks selliseid võib-olla mälestus, aga see lõbu kestis jube vähe, et see oli kuule nii tõesti nii muldvana jaam ja ilmselt palju kasutajaid taga, et siis see mingil hetkel vahetati esimeste seas digikeske vastu välja. Aga kuivaks oli siuke. No ma ei tea, jällegi 90 lõpus, kui ma mingisuguste äkki reagi esimest korda nägin, mingi 2600 ja et siis neid lugema hakkad ja siukest nagu USA 90.-te 80 lõppu 90.-te found Froikingu laine, et me saime nagu koiva seda korraks oli Eestis teine, teine asi, mis töötas kindlalt, oli see pooleaastane periood, kus Eesti Telefoni ootus Chipiga kaadiga telefoniautomaadid oli, kus see kinni teipida ühe klemmi ja tasuta helistada. Et see oli ka korralik? Tooge vist ei saagi huviliste toetaja. Ustasin teoseid, kus kuradi kuuesadesse sattusid? Ma arvan, et osa neist hakkasid jällegi, et kui sa pidanetis nagu saad esimese ringi peale Eesti gruppide lisaks tellida kui mõne mõne rahvusvaheliselt USA grupi või siis kuskil Youznetist siukese häkkerigrupid, et seal hakkas osa neid asju nagu skii tekstina, lihtsalt nagu paberkoopiaga paberkoopiat ma nägin siis kui Moussoleksin. Ja siis ma olin tõesti šokeeriv, ütleme, see oli nagu jälle niisugune, nagu ma ei tea. Ilmselt mingitel usklikel on see, et kui sa satud mingisuguse piibellik teksti originaali juurde, et siis, kui sa raamatupoes näed, et see asi on paberid, füüsisite värvilisena täiest või siis sa sattusid Ameerikasse, vahetusüliõpilased keskkooli ja, ja et mitte üliõpilased, vaid lihtsalt ma tegin avalduseoto klubivahetusprogrammi, millel on selline nagu noh, ankeet oli hästi selle keskne, et et mis sa teha või noh, lugu, et aru saada, et kes nagu kooliõpilane on ja foto või vahetus on selline nagu kahesuunaline, nagu üks klubi saadab kuhugi kellelegi kuskile välja ja siis samal ajal võtabki mujalt vahetusõpilasi vastu. Aga see kujuneb välja, saatsid juhtumisi ma ei tea, kas sellepärast, et muuseas ankeet nii nagu ajuti asjade kestmine, aga ma sattusin oma nagu 11. klassi, siis Eesti mõttes sattusin Silicon Valley keskele. Ehk siis sihukeses õrnas eas elasin aasta aega kuppetinos. Mis on see linn, kus on Apple'i päeva kohta ja käisin Monta vista faili, mille siis keskkoolis, mis oli jällegi nagu paljude juhuste kokkulangevuse USAs, oli, just oli seal mingi Ifromission, suprahai või hullus puhkenud ja hallkool oli eelmisel aastal enne minu sinna kohale jõudmist oli USA-s 35 kooli nagu internetipilootkoolideks ja see oli üks nendest. Et see oli põhimõtteliselt Apple'i Apple'i peakorterist mingisuguse paari kilomeetri kaugusel ja sulle, Meil oli, kui ma õieti mäletan, koolis oli 1400 õpilast ja kaheksa arvutit tekkida ja selgus, et arvutilaboris assistent olemise eest saab nagu ainepunkte või et sa võid nagu ühe tunni asemel iga päev istuda arvutilaboris oli, kus šveid nagu Mäkid ja sarni silikon, krahviksi, mingid asjad ja ja, ja jahud, kasutasime aadressil jahutad, siesta stan, fortatiidi ju oli, et sest sest see olgu söömaks muutunud sattusid tundmist paradiisi. Taimesid küll jah. Et see, see kindlasti on ka nagu tohutult mõjutanud seda. Mis, mis edasi. Aga kuidas toime sellega sa ütlesid, et jumala armastatud jalust nõrgaks, kui sa tuled nagu nõukogude liiduvabariigist sisuliselt satud üksuse kohta.
Hea küsimus, see oli mu esimene lend Eestist välja, oli nagu uss, onju, et üksi. Ja, ja ma arvan, et vanemad nagu raha laenanud lennukipiletit saaks lubada ja see kõik nagu ilmselt nende poolt ka nagu üsna hullumeelne. Aga, aga see oli, ütleb ütleme nii, et ega see ette see on nagu vanemate asi on muretseda, ega sa ise ise sellises vanuses nagu lihtsalt teed ja lähed ja oled ja, ja oled nagu käsl ja võtad kõike seda sisse, mis tuleb. Koolikogemuse mõttes oli see, et kui sa eestist vööri, kui sa oled nagu siukest matemaatika füüsika huviga ja nagu vähegi nagu olümpiaadil käinud nagu natuke mingit sihukest niisugust elu elanud, et siis ei ma võtsin, olles seal 11. klassis, ma otsin kõik reaalained olid mul 12. klassi panerist tasemeained ja ma olin kõik selle Eestis juba läbinud, eksis, koolisüsteemid, on nii palju erinevad. See, mis oli teistmoodi seitse esimest korda elus, peab minema mitte lahendama ära üksiküritaja oma testi tulemused vaid vaid sa pead moodustama grupi kolme inimesega, kellega sa oled koos töötanud, midagi koos välja mõtlema, minema, klassi ees ette kandma, mis sa tegid, eks siis nagu õppeviis või on need kõik need asjad, mis tegelikult jällegi noh, et, et Eestis nagu siukest reaalainete tugevus, vöö, mis need asjad olid, kõik, mida me seal piinaga me esimest korda pigem seal oli selline eba ebamäärasem see värk. Ja ühiskondade kontekstis oli küll, et nagu, et ise tuled mõtlen, kui ise mainin enne seda juba nagu tööl käinud ja siis sinna jõudes ka väga heasoovlikult klassivennad vahest küsivad, et kas tal Eestis telekaid on meil see nende arusaam sellest, mis, mis seal raudse eesriide taga nagu toimus, üsna hägune, mis inimesed mõlemat pidi. Aga ma saan aru, käisite tööl? Või vähemalt ju. Ma ei tohtinud seal tööl käia, see oli nagu see vahetusõpilase tegevus kuskil just, et siis ma ütlesin, ma ei tohtinud seal tööl käia, siis selgus alguses see kurvastas mind väga, aga siis ma sain aru, et seda defineeritakse läbi palga. Eks ma käisin mingisuguses arvutipoes nagu pärast kooli seal abiks, ma ei tohtinud palka saada ja siis pärast, kui ma Eestisse kolisin, output kinkis mulle arvuti. Mõlemapoolselt kasulik ööst, et see, see, see oli see lahendus, mis ma leidsin selle. Aga see oli siuke ütleme.
Noh, kuidagi seal see asi oli nagu noh, tõesti pigem käsitleti nagu koolipoissi, kellel lubati arvutit kokku panna, et see seal nagu see idee oli ju palju Klementeeritum kui see, mis nagu Eestis samal ajal toimus juba töö siin, nii kvalifitseerumise lihtsalt. Me ja teine asi, mis mulle väga meeldib, mida puudutanud, mis me enne alustasime, oli see, et meil oli, tekkis sihuke rühmitus nimega intellekti vastu eluks. Mis juhtus ka, ma arvan, seal üheksandas kümnendas klassis olin mina Mark Def Kristjan Jansen ja alavi Aho natuke hiljem. Kes Mac ja kikka olid Treffneris, mina alavi olime minemas ja Mark on ilmselt Ta ütleb, et mis, mis mind nagu, miks ma ei ole, tänapäeva meedia oli see, et mul õnnestus väga noores eas õrnas eas näha lähedalt inimesi, kes tegelikult oskavad programmeerida. Ja Mark oli nagu juba tol hetkel, kui koolipoisina oli, oli inimene, kes hommikul aks tekstifailides Hendrik õhtu, palju seal käima siis töötas, oli, kirjutas näiteks mingisuguse graafikamootor või midagi. Ja, ja see sümbioos, mis meil tekkis, seal oli selles Mac, pages kikka disainis. Mina korraldasin asju, mis võis tähendada mida iganes, alates sellest, et Femeksist laenata sound solklasteri kaarti, et maksaks ka sellele audiodraiveri kirjutada. Ja Alari Aho tegi muusikat ja see toode, mida me ehitasime ajuti mängime trancard.
Ja ja siis minu see.
Isegi ei suutnud meiega välja mõeldud, mis ma, mis ma seal seltskonnas valmistu orga, naise ametlikes paberites, aga et siis mina, mina hakkasin seda maha müüma. Et siis ma olin tihedas kirjavahetuses epic megageim siia äpolsi mingite selliste ettevõtetega, kes nagu kõik nagu olid valmis meiega rääkima ja siis, kui ma usse läksin, siis sel hetkel nagu tekkis nagu veider olukord, et ma sain nagu USA postiaadressid saata flopiga demosid ja välja näha, nagu meil oleks nagu mingi pärisfirmad. Et see kõik, nagu me panime paari aastaga, esiteks kooliga pealt me ei teinud seda asja lõpuni valmis, et nemad olid olemas. Aga, aga me panime paari aastaga selles suhtes nagu mööda, et, et me kirjutasime kaks t platvormikat mis ilmselt nagu 91 oleks nagu lennanud, nagu see taseme juures, mis matjagi ikka nagu, nagu kokku töötasid oleks olnud ilmselt täiesti vabalt nagu müüdav, täpselt nii nagu bluumuni kutid, venda, mängija, maailmaviisid. Aga aga Me komistasime täpselt sel hetkel, kui meie saatsime demod, siis vist oligi vist kas ID soft või Wulf Einstein oli juba väljas tuum tulemas või midagi sellist, ehk siis nagu ütleme see, mis me enne rääkisime ka, et seal graafikatase läks sinna, kus kolm D. Just Tatu koolipoiste platvormikas ei paistnud nagu, nagu säravalt silma, aga põhimõtteliselt asi isegi nagu töötas.
Ise mäletan, tol ajal tundus ettevõtmine sellepärast, et sul on mingisugune muusika, mälutandigid asjades, mitu Seidi käid, muusika, mingid taustad, seda oi, seal oli noh, teine asi oli see, et vaata kõiki ka mängugi, et isegi tänapäeval, aga tol hetkel iga mängukirjutaja, sa alustad sellest, et sa pageda endale töövahendid. Ega sa ei saa, sul ei ole tööstuseks levelit disainida. Mingisugused sul täkkeid täkkeid selliseid nelja või kaheksa kanaliga taustamuusika tegemiseks olid olemas, mida jälgi bluubel tootis veel. Aga, aga mai tea leveli disainimiseks või, või, või isegi mingisuguste. Kui kasutada mingi pildi egi diktoritehas bait valmis, aga kus aspationi animeeritud siis selle jaoks nagu sa pead jälle mingi omava tööandjad tegema, et meil oli kogu sees täku olemas.
Mis teid siis tundus, vägev, oli mingisugune motivatsioon rikkaks saada? See tundus ikka äge, lihtsalt ma mak make giga tegustasid, enne jällegi peaks küsima, kuidas me kokku saime.
Kikka mingi paar aastat tagasi, ma ei tea, ta vist võttis maha, aga paar aastat tagasi oli kokku kogunud kõik meie tolle hetke kirjavahetus ja palju seal avaliku interneti kõik nagu pildifailid ja mis me siis siis mul oli seal juba hea mitu aastat tagasi, et siis oli koju sihuke nostalgiarännak. Aga, aga neil oli Pokk sees, seal mängu taustaloos oli ka mingeid. Ma mäletan c trankad, kes pidi korjama, et oli ta veres, alkoholitase ei langeks. Siis selleks pidi koju pudeleid, siis ta sai tühjade pudelitega loopida ja olid mingid olukorrad, kus ta pidi natuke ajutiselt sama Kainimaks, siis neil oli seal Mackiegi ikka üks klassivend Treffneris olid, kui ta koolipeol liiga palju õlut ja siis ta hakkas kükke tegema Kainimaks, siis trancadiga see ka noolt hoiad all, siis ta tegi kükke alkoholitase veres, langes.
Aga seal on mingisugune uus element, oli sees, sest jooksmine kindlasti ka pold. Meil mingi idee, täpselt meil mingi idee oli, et, et me tahaks teha mängu, kus ei käi nagu tulistamine ja tapmine ja mis oleks nagu selgelt nagu teistsugune.
Näiteks kui teadlikum seda mõtlesime, et see nagu täitmatu uudis, et, et nii-öelda täiskasvanute mäng, mida muidugi ala alaealiste Ida
Arusaam sellest, mis asi on mõnevõrra teistsugune kui tohtidele seega selgelt vastas 90.-te tatuda
Ja olid ka.
Seal Californias midagi huvitavat.
Seal ma ka ikkagi pagesin, ma seal hakkasin. Seal juhtus selline asi, ma ei mäleta, kas ma seal natuke pidasin, viib s, aga noh, kui see eraisikuna pead nagu koduliini BBC ei ole nagu see ja pluss, et nagu sa satud nagu USA telefoninumbrit ruumi ju, tegelikult noh, pigem mai siis nagu BBC kasutaja ikkagi seal. Ja, ja siis hakkas ka internet noh, ütleme graafiline veebibrauseri nagu ilmus orbiidile ja see kõik pilt hakkas muutuma. Ma kirjutasin seal ka ühte BBC softi. Sihukese hobiprojektina tundus, et see võiks olla noh, jällegi nagu sageli teed, et nagu, et kui sa mängu ei pagenud, mis on see asi, mida sul endal vaja on, siis ma hakkasin selle käigus vist juhtus niimoodi, et ma ma uurides, mis seal veel ja kui tuju peal on, leidsin ühe ägeda BBC softi, mis mulle meeldis mis oli šev ja kinni keeratud. Ja siis siis ma mõtlesin selle lahti, et nagu sellise noore häkkeri nagu, et noh, midagi nagu siukest komplitseeritud, aga sa jooksutad seda asjade, pagari, pagari, siis vaatad, et kui kui imelike ootamatute kohtade peal hüpatakse mingisuguseid mälus mingi aadressi peale, tehakse seal müks mingi väga lihtne tehe ja siis lihtsalt nagu nagu masi koodi tasemel. Muudad ära, et sinna ei ame, hüpatakse siis ootamatus, õigused oligi koopiakaitse maas. Siis ma ropoteesin seal autojuht, et muide, selle saab niimoodi maha võtta dist andis mulle eluaegse tasuta litsentsi ja see võttis mul oluliselt poti maha, et oma BBC softi kirjutada, sest jumal oli selle koledad. Aga, aga, ja teine asi, ma mäletan, see oli õudselt hea õppetund jällegi siukest enda tasemest pagejana, et et ma kohutavat abstraheeris siin selle asja üle. Et kui sa nagu Siiblus plussis piibi essi, kus ma katsusin hoida väga puhtalt eri kihtidena näiteks seda, et kuidas käib modemi händlimine, kuidas käib terminali jändamine, mingid asjad nagu olles valmis igasuguseks tulevikuks, et sul neid asju, millega liidestada mitmeid ühesõnaga, ja siis ma ei saanud nagu jube jube kaugele oli kogu aeg sellest, et see asi nagu minimaalses skoobis töötaks sihukeses hilismaaeluprojektideks.
Täpselt kunagi valmisid, aga võib-olla see, et see, et ma täna rohkem start-upidega tegeleda MVP nagu kuidagi Alpso, mõtlesin, pigem näeks nagu vähemaga ja kujundus ette ja siis ta ütles, et Ameerikast tagasi kodus arvuti ja raamatud. Jah, oligi umbes sihukeselt pae kastioidki, mis siis, ma läksin tööle, siis ma läksin keskkooli edasi 12. klassi, aga, aga siis mul oli juba natuke hoog sees ja siis ma läksin tööle tööle sellisesse firmasse nagu triip. Ja mis oli tatus, selline algselt trükikoda disainipeo, aga kuna disaini osa ja mul on eluaeg kuidagi nagu isegi siis, kui ma nagu pagesin, mulle on alati meeldinud nagu see, kus on nagu tehniline osa ja visuaalne osa või kuidagi nagu kokku saavad, et ma olen eluaeg kõik asjad, mida ma olen teinud, ma olen alati töötanud gospagejate ja disaineritega, hiljem on ju, et et olete ka ma ise USAs üks asi, mis me tegime, oli see, et et mättad olid siukesed, kunstirühmitused kes tegid Askyjaatia hiljem VGA mingi taati ja siis ma mingite varjunimede all koos ühe koolivennaga isegi komistasime paari sinna sisse, et minu, minu aski ansi aeti, on olnud mingisugustes distitutsiooni pakkides isegi. Aga et siis selle tausta pealt ja USA-s üks kooliaine, mis seal koolis oli, oli ka laat, oli selline tootedisaini ja pakendi ja mingi selline asi, et siis ma nende näidistega ilmusin sinna triipu välja, ütles, et ma tahaks nagu pääst kooli natuke arvuti taga istuda mis antud juhul tähendas disainimist ja siis sattusingi seal tööle ja pluss noh, see kõik oli nagu üsna kaootiline, teine hästi-hästi mõju, kas inimene või hästi palju mõjuttoni tol hetkel, Marek Tiits kes pidas seda IPS-i või balti õpingute instituuti, mille alt ta testis edukalt mingisuguseid eurorahasid ja tegi sellega ägedaid projekte. Eesti.
Seaduste otsingumootori Kiwomehide oli veel selle ja Marek ka kunagi kuidagi nagu andis mulle, kui lihtsalt siukese ringi hängivad koolipoisile võimalused, tule, tee midagi aeg-ajalt ja, ja see tähendas jällegi ligipääsu arvutitele. Tähe toimis. See oli ka väga naljakas koht, seal oli näiteks üks silikon krahviks arvuti veebikaamera selline 90 keskel ja seal oli ka oluline funktsioon, et sinna oli võimalik sisse logida ja vaadata veebikaamerast, kus kohvimasina täis jooksnud, et siis ei pidanud alumiselt korruselt teisele korrusele, tulevad kas tühja tassiga. Ja seal oli üks mingi erakordselt oluline projekt, miski ei mäleta, miks seda vaja oli. Aga siis ma kirjutasin Berlis, seal oli üks tsükseli modem, millel oli ka faksi funktsionaalsus ja kuna seal majas oli ka internet, siis ma kirjutasin Bellis veebipõhise faktide saatmise vastuvõtmise aplikatsiooni Pirlis veebipõhise foksid, et kui keegi keegi saadab faksi sellele numbrile, mis võttis maden vastu, võttis need failid, kirjutas maha mingisse sunni sööbess, siis oli võimalik üle veebiteid faktsia teha. Et.
Ma ei ole kindel, kas see oli asi, mis, mis ma ise mõtlesin, et võiks teha ja muidugi, või oli see mingi asi, mis, miks projekti jaoks oli vaja, aga, aga sihuke asi seal tekstis ja veel aastaid hiljem vöös teatavas suures ettevõttes, mis kindlasti ei olnud telekommunikatsiooniettevõte, räägiti sellest, et oleks äge Foxy saatori, internet. Minu arust oli kaks aastat tagasi Viljo, tegi aprillinaljana mingeid pagejategid Twilio faksi appi ja nüüd on mingi oluliselt kasvõi välisuutel. Sellepärast et sa pead miljoneid inimesi, kes tahavad kogu aeg faksi saata. Äge jätk, mõtle, mis kõik oleks võinud olla triibupealik, kes su sinna, kuidas seal oli juss, peedimaa Juhan peedimaa oli jälle mu otsapidi kontakt seal ja Eva, kes nüüd on ka peedimaa või sellest ajast ja siis Priit Jagomägi, kes oli Jago mägede, pere kuulus Regio ja kartograafia taga ja Priit oli siis selline räbal vend, kes tegi oma firmat, mitte mitte ei tööta tegijast. Ja, ja see oli siuke k kuidagi 90.-te alguse. Põhimõtteliselt päris mitmed asjad, ma olen Eesti suhteliselt sellepärast, et sul on nagu täiesti tühi maa, on ju, et ma mäletan seda, seda hetke, kui ma keskkooli lõpetasin, Eestis oli umbes 40 panka ja keskmine panga mingis jõude, vanus oli mingisugune 28 umbes et siis tekib nagu tunne mingist rongist maha jäänud, et kui sa oled 18. Ta on joad just ja, ja, ja see tehnoloogiaettevõtlusinternetiga seotud, et see ettevõtlussee oli tegelikult see laine, mida me tol hetkel ei teadnud, kuhu me maandume. Aga mis, mis selgelt oli see, et ei olnud rong veel läinud, et me saime omale ongi ehitada ja see triip oli täpselt selline, kus ta alustas sellest, et oli eesti tühid täiesti tühi maa, iga päev tehakse kümneid firmasid igale firmale vaja logo ja visiitkaarti, siis nad hakkaksid neid tegema, siis ostsid sellest või nii palju rahapealise disaini pealt ostsin oma trükikoja siis ostsid maine veel mingisuguseid asju kokku ja siis selle sisse komistasime selle internetiasjaga jälle. Et kui ma seal noh, et, et seal võib olla, kujundati mingisuguseid trükiseid, reklaame, aga siis mina hakkasin seal tegema, aga veebilehti nendesamade klientidele. Ja siis sealt kasvaski välja, et kui ma keskkooli lõpetasin, siis ma läksin rõivust ära ja tegin oma esimese ettevõte, mis siis tegelikult nagu Voltaire halo, onju? Jah, oli küll. Ja siis saime meie tuttavaks jälle siis või mitte tuttavaks, vaid nagu töiselt. Teadsin varem.
See, kuidas horo sündis see on veel.
Eriti keeruline, ma mäletan vist praegu Gazpromi endised tooteproduktsiooniettevõtjad projektivine töötamas. Saabus Clint siis õlid projekti ühistult näha, laual tanksaapad, mille sees olid projektijuht, kes magas. Oi, seal oli palju selle sõnaga seal kõige ilusamaid mälestusi veel, et kui Lähed kontorisse hommikul ja siis vaatad üks disainer poeb valgete linade vahelt välja ja siis selgub, et ta tõsteti juba kolm kuud tagasi ühikast välja. Aga mina, ta kogu aeg esimesel ööl ja hapupiimast keldrist elasse.
Aeg-ajalt tuli ette, 90.-te Tartu oli mõnevõrra teistsugune, kui näiteks. Aga sa rääkisid, et tekkis võimalus oma rongid. Kas sul oli siis sul oligi sihuke selge visioon, tuleb laine ja ma lähen sinna laine peale ja nüüd on seal oluline asi, mida teha. Või sa lihtsalt kuidagi tegid seda, mis nagu mõnus tundmus.
No ette mõtlemist selgelt liiga vähe, et seesama all oleks ikkagi nelja aastaga pankrotti kaaslastega mingi piirini, nagu on võimalik ehitada, ehitada asju intuitsiooni pealt, aga mingil hetkel peaks nagu jalutama läbi ka või mõtlema, et mida see et, et Seli see oli, noh, mingil määral oli selline.
Avastus, et, et see asi, mida inimestel vaja või nagu suurtel firmadel näiteks noh, et kes nagu halo klienditöid ja seal tõibus jumakas kleidiks amet nagu täiesti kes on kes kõik mingisugused suured pangad, mingisugused rahvusvahelised bändid, kes olid Eestisse jõudnud, ma mingi audi või, või ESS, mis tol hetkel tohutult kasvas, aga mingeid siukseid nagu kauba mänginud ja kõik teadsid ja siis nad on valmis nagu see internetivärk oli nende jaoks nii arusaamatu, et sinu jaoks on see intuitiivne ja lihtne ja nagu milles siis probleem on, teeme ära ja siis on nagu suured kauba kõigega nõus, nagu alla neelame selle, et nad ostavad mingite kaheksateistaastast üheksateistaastaste tartu kuttide käest teenust. Ja noh, ei osanud seda hinnastada, on ju, et noh, et iga kord mõtled, et kurat, see number kõlab nagu liiga suur, et ei tea, kas nad selle nagu alguses toimis nagu liiga hästi lihtsalt ehk siis. Ja siis ostis, üheksin kuus alustasime 97 kevadel ostis üks suureklaamingeid teedeebee halo juba ära või kontrolli seal ettevõttes. Ja, ja siis me sattusime ootamatult nagu päris ärikeskkonda, kus on päris inimesed, Soome juhid, kes olid nõukogus, tahtsid mingisugust eelarveid, näha mingisuguseid asju ja et ühest küljest oli see kõik nagu selline nagu vaeses nooruses nagu kokkupuutumine päis asjadega. Teistpidi ka need nagu reklaami ja, ja nagu investorid, siis tol hetkel ka nemad Pavesti läksid siis internetibuumi sisse, et ka nende kliendid loopisid raha vasakule või ma tea, rahvuslased, kes seal olid veisel siis kes võttis meid 100000 dollarit esimese kohtumise eest mingid sihuksed agentuurid. Ja, ja, ja et see tähendas seda, et nad ei, Nad ei suutnud. P näiteks palkasime selgelt kohe liiga palju inimesi, sellepärast kohe-kohe pidid nagu need lääne kliendid tulema Eestisse asju arendama ja meil nagu tõsised, mingisugused jutud seal teedeebeegeti sees, et siis kui see mull nagu 2000 nagu lõhkes, siis lõhkas kõigi jaoks kojaga, et lõpp, kes meil Eestis lõhkes seal ja, ja vot see oli asi, mida, nagu sellist nagu täiesti vaheliste tüüpi teame ja sul on see, et majanduses on tõusud ja langused. See, see oli nagu täielik müstika, kui see juhtus, oli ta ei olnud ühtegi orienteerunud. Ja see noh, mitte ühelgi hetkel ei olnud seal sellist tunnet, et oh, et me, Me oleme ettevõtjad, võimlemisstartup või, või, või ikkagi sa teed asju, mida sa oskad ja mida nagu mingi tõmme tujust või tahetakse, et seda teeks ja järgmine homme küsitakse multimeedia CD, homme teeme neid ka multiCD-ROM. Müstiline asi, et kas tõepoolest ühispank küsimuseks, kas ma võin seda teha küll ja, ja võis olla küll, jah. Ma arvan, et ühiskaks tükki tegime suut, üks oli ühispank, millele tegime ja selle lainelt müüsime ära Eesti Telekomi. Ja see idee oli selles, et kuulge, et varsti on aasta 2000, mis ta oma aastaraamatut paberitükkidele
Mäletan üks neist maksid mäleta kumb max 200000 Eesti krooni selle eest. Aga teisest küljest ära mõtled, 15000 euro eest ei saa ühtegi proget liigutama. Kostitada.
Mind, mis mind tagasi vaadates seda rohkem tõenäoliselt siit ka see võib olla siis sind ka, et.
Teadmata midagi virsiooneerimisest, testimisest ning süsteemselt. Kuidagi suutsime mingisuguse softi, mis enam-vähem töötas, meil häbematus, klientide klient maksja varbad ära. Seal oli üks asi, mis ma olen, nagu mõelnud paar korda, on see, et et nii nagu vaata see kahe tuhanded või see 90.-te internetilaine peal kogu aeg uuest majandusest ja tõesti palju neid kohti, kus, nagu hajuti, et uus majandus ei alluvale majandusreeglitele. Mõnes asjas ei allu ka, aga kui turu suuruse mõõtmine või füüsiline kaugus ja mingid asjad aga, aga mingites põhiasjades ikkagi vood tuludega töötada kulusid nagu alluvaid. Ja, ja siis samamoodi nagu selline vastandamine e-kommerts ja päris kaubandus ja kõik nagu kuidagi käsitleda, öeldi, et minu arust me tegime mitu aastat seda äri nagu selle koha pealt valesti, et me mõtlesime, et see veebiehitamine ei ole nagu tarkvaraarendus. Tegelikult meil ei olnud seal tiimis keegi, kes oleks nagu mõelnud ühegi veebisaidi ehitamisest nagu tarkvaraarendusprojektist või lugenud mõne raamatu selle koha pealt, nagu, et kuidas suust pätid kuule tunnust veebi oleks ju nagunii lihtne. Et siis, kui see kogemata ehitas sinna taha ka mingi sisuhaldussüsteemi, mida me üritasime, mingid asjad siis sellel hetkel oleks pidanud nagu üle minema. Ma arvan, et tagantjärgi see, mis Taavi, Kotka ja veel meedias tegi, on ju, et nad hakkaksid mõtlema oma tarka õenduse protsessi peale oli see, mis nagu hoidis neil nagu elus nad alustasid, vestlesin natuke hiljem aga, aga, aga ta nagu kiiremini hammustasid läbi, et kui nad lähevad suurt maksuameti infosüsteemi tegema, siis ei peaks, nagu seda veebilehena käsitleb. Ehitus on ikkagi veel ühtesid, mis midagi kogemata mingeid skripte taga. Ja siis kuna andmebaas tundus keeruline, siis võeti kõiki maailma asju, hoiti koos projekti failide näiteks mida reageeriti tellija puhul, et kuidas see oli tol ajal tiiv hüppega kuidagi seotud sinna otsa koperdas ja see oli ka üks kooliaegne asi, kus jällegi seesama Tatu ja füüsika instituut ja, ja füüsikute seas oli ka Jaak Aaviksoo, kellega mu isa käis koos ülikoolis kunagi samal ajal. Ja kes oli ka Miina härma vilistlane või teise keskkooli vilistlane, ehk siis mingisugused seosed jälle seal ja siis siis kui tiigrihüpet tegema hakati, siis oligi Toomas Hendrik Ilves ja Jaak Aaviksoo mõtlesid välja, nagu nad ise räägivad kolmekesi, et Ilves, Aaviksoo, Johnny Walker ja, ja mõtlesid selle oma rollides tol hetkel välisministrina ja haridusministrina välja ja moodustasid selle ümber tiigrite peakomiteed. Ja siis 96 koma jäin keskkooli viimases klassis, siis, siis kutsusid mind sinna õpilaste esindajaks. Sest nagu komiteesse, mis on iseenesest nagu žestile nagu suhteliselt ilusaid tehes nagu haridusprojektiks või õpilane kaitsega seotud. Aga kuna see oli selline kujutanud ette jällegi oli, et ma lähen Tartust bussiga Tallinnasse, haisus ministeeriumisse, kus toimub koosolek, kus on umbes nagu Peeter Marvet ja Marju Lauristini, Ants Sild ja mingid sellised korüfeed nagu laua ümber, siis ma sisuliselt istusin seal komitees lihtsalt vait ja kuulasid nagu et nüüd tundus lugudest müstiline. Et, et ma arvan, et minu kõige suurem nagu tiigri palus, oli see, kui me käisime ühe kaevama sel maailmal, mu esimene tele, ETV ka otsesaates seda täpselt ei mäleta, kas ma kunagi üldse kaameratele lisatud, et aga aga otsesaade, kus oli Tiigrihüppe teemani väitlus, kus oli Marju Lauristin versus Lauri Leesi ja mina. Ja siis kuidas sa oled otse-eetrisse kunagi televisioonis on see keskkoolis ja siis siis noh, jällegi kaks suhteliselt nagu lõbu tunnustatud inimestel raske, ja siis vaatad, et kui Lauri Leesi jahub sulle midagi sellest, et et arvutit kooli vaja. Kristel on aastasadu vastu pidanud ja siis sa saad aru, kui läinud on, nagu need selle tüübi jaoks oli, et sa ise juba tunnetada seda, et kuhu maailm liigub ja mis seal nagu juhtub. See oli ka tegelikult see koht, kus ma tegelikult sattusin sinna, et ma ei, ei läinud.
Ülikooli kui ma läksin, siis ma läinud õppima arvutiteadus siin, sest umbes sellest hetkest. Ma ei suutnud seda võib-olla nii hästi sõnastada, aga et ma juba juba nagu pagesin juba piirasin BBC hängisin internetis ja mul tekkis tunne, et see asi, kus on mul nagu tegelik lünk, on see, et miks need asjad toimivad nagu võrgustikke ja see sotsiaalne pool. Ja kuidas täpselt sel hetkel sealtsamast saatest välja tulles Lauristin küsis, et et kuule, et me teeme thats uute eriala, kus me hakkame kommunikatsiooniteooria, ütleb, et tema asju ei tule sinna. Et siis siis seal oli ka see, kuidas sattus üldse nagu sotsiaalsee kompromiss, mis ütleme paljude jaoks ka ma enda jaoks oli suhteliselt üllatav, et polügooni sihukestesse tegevused
Jah, võrreldes kõik need paketid, kes olid dialoogid, läks mul isegi hästi füüsikat. Millega sa praegu tegeled? Teekond sind on toonud tänaseks.
Lühidalt põhist ehitan ettevõtteid, et ehitan ettevõtteid. Pigem nende esimeses otsas aeglases faasis. Et mõnes mõttes võibki öelda täpselt sama, mis 90.-te tegime, aga lihtsalt iga kord nagu tead seda võib-olla natuke uue kogemusega või natuke juba tead ka, mis et ma mitte ettevõtteid siis sedapidi, et ma panen osade alla oma aega, et ma hakkasin tegema siukseid võivad nagu telefon viis aastat tagasi ja kaks aastat tagasi meid osteti ära, et ma tegutsen edasitud trupi ja kes meid ostis. Ja juhin nagu põhjust oli see niisugune asutajale see hetk, et kas nagu tahan suurt tüki väikest pisukest või väikest tüki suurest pirukast ja ja, ja trupi ja on. Või see telefonimüük ja õppija edasi tegutsemine jättis nagu võimalused on paar aastat nagu vahele jätta või hüpata trepist kõrgema öösse 12 inimesega, start upis. Nüüd meil on 100 170 inimese tiim ja, ja mul on kuskil 70 inimest tootevaldkonnas tootearenduse valdkonnas, kellega saab sama visiooni Giulia ehitatud. Ja siis, kui mul aega ülejäänud, siis ma investeering start-upidesse ja annan mõne nõu mulle tassi pudeli. Põhiliselt. Või noh, see sarnasus 90.-tega on see, et selleks, et et ma tean täpselt, mida ma suudan üksi ehitada. On see siis BBC oht või mitte, aga ma tean, mis väärtus, mis võlu on sellel, kui, kui proge ja disain ja äriinimene koos mingit asja teevad ja mida siis saab valmis teha alates siis trankad või, või, või mingitest veebiprojektidest halos. Ja ma tean, et ma ei taha kunagi elus müüa oma aega tonni põhiselt, sest neil on tunde lõplik hulk, järelikult tuleb ehitada tooteid, ehk siis see on noh, see läheb juba selle ajalisi aknaskoobist välja või hiljem Skype'is nähtud asi, et kui sul, kui Supaaviimiste ehitavad mõne kuu mingit asja, mida kuuega hiljem kasutab miljon inimest, et siis siis on asi, mida me kui see 90.-te selline rahmeldamine võib-olla õpetas, oli see, et et asi, mida ma alati otsin, nüüd on, on see, et kus sisse pandud töötundide hulk konverteeriks võimalikult suureks väärtuseks. Et tol hetkel kippus ikka väga palju olema, et kui sa nagu puhta õhuõhinapõhiselt üheksakümnendatel tegid noh tõesti, et oleks kõik neid samu asju teinud ka siis, kui oleks paka saanud. Et, et siis kui sa müüd oma töötunde, siis sulle lihtsalt nagu väga pikad päevad, aga lühikesed ööd ja, ja.
Eks neid tüüpe, kes seal üheksakümnendatel nagu väga läbiga põlesid, on juba juba kohe ostetavast hiljem, aga ja siis olid näiteks ma ei tea, kes olid seal legendaarsed mehed, nagu näiteks väga austan sellest ajast või Taavi Talvik näiteks on ju, et kui sa nagu samade tausta huvide pealt ehitanud nagu unineti, kus ongi, et sa võid ka magada nagu bitid, müüvad ennast ise. Et see seal on jälle nagu see oli küll selline teenuse infrastruktuuri, aga seal oli see nagu alge olemas, kuidas ehitad midagi, mis nagu saab hakkama ka siis, kui sina ei ole näppupidi juures kogu aeg?
                 
Kõneleja 1:
Ja see on väga õpetlik ja väga tore lõpp jutuajamisele, aitäh sulle, aitäh.
