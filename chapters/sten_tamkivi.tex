\index[ppl]{Tamkivi, Sten}

\question{Sa old tartu poiss?}
                 
Olen Tartus sündinud ja kasvanud tegelikult selle saate mõttes kogu selle aja. 

\question{Sa oled natuke noorme põlvkonna inimene, kui teised, kellega rääkinud olen. Mis aastal sa sündinud oled?}

1978. Noorem jah, sest kui ma memcpy  saateid kuulanud olen, siis minu jaoks enamik neid inimesi olid sellise legendaarsed ja \emph{established} nimed, kui ma kaheksakümnendate lõpus arvutite ja interneti juurde jõudsin. Kellega ma võib-olla veel mõnedega elu jooksul hiljem tuttavaks olen saanud ja avastanud, et \enquote{oh, et Madis Kaal\index[ppl]{Kaal, Madis} on  päriselt ka olemas} ja ei olegi nii palju vanem, kui ma arvasin.

\question{Takkajärgi need vahed lähevad kokku, paar-kolm aastat vahet ei ole nii suur. Aga räägi palun oma kujunemisloost. Mõned on olnud hirmsad olümpiaadi-hundid, teised on rääkinud, et neid huvitasid raamatud, kuidas sina selle värgi juurde jõudsid?}

Oli paar nagu erinevat suunda. Üks asi on see, et ma olen pärit teadlaste suguvõsast,  mu isa ja vanaisa olid mõlemad füüsikud  ja ma kasvasin üles lapsepõlves Tartus füüsika instituudi järgi nime saanud FI rajoonis. Mis tähendab seda, et sul kõik lasteaiakaaslased, kõik naabripoisid, kellega õues mängida, kõik on kuidagi tõenäoliselt füüsika instituudiga seotud. Ja noh, ma ei tea, kui seal kaheksakümnendate lõpus ja üheksakümnendate alguses füüsika instituudi elektroonik paneb majadesse piraat-kaabeltelevisiooni ja füüsika instituudis saad esimestele arvutite ligi ja kõik asi asi oli sedapidi seotud. Ja, teiseks, ma õppisin Miina Härma Gümnaasiumis\index{Koolid!Miina Härma Gümnaasium}, tol hetkel, kui ma sinna läksin 1985. aastal, oli ta Tartu 2. Keskkool\index{Koolid!Tartu 2. Keskkool|see{Miina Härma Gümnaasium}}. See kool oli mõnes mõttes nagu Tartu tolle aja esi-koolidest homanitaarsem. Treffner\index{Koolid!Hugo Treffneri Gümnaasium}, mis oli 2. keskkool, oli palju selgemalt reaalainete kallakuga, aga siiski Miina Härmas ka oli selline reaalainete suund täiesti olemas. Sedapidi ma käisin olümpiaadidel ka. Ja üks varasemaid pilte, mis mul on kuskil vanemate või vanavanemate albumis olemas on, on selline mustvalge foto, kuidas keegi tõi sinna 2. keskkooli algklassilastele näha kooliarvuti Juku\index{Arvutid!Juku}. Ka mustvalge pildi pealt on näha, et silmad juba läigivad, kui lähedalt näed.

\question{Ühesõnaga su esimene arvutikogemus oli FI-s?\index{Tartu Ülikool!Füüsika instituut}}

Ma arvan küll. Selline, kus nagu saad nagu päris oma aega, sina ja arvuti, isa lasi  mingil õhtul kuskil kellegi kabinetti kuskil. Neid kohti oli veel, mul näiteks pinginaabri isa töötas Tartu Ülikooli Raamatukogus\index{Tartu Ülikool!Raamatukogu}, seal me käisime arvuti taga ja seal üheksakümnete alguses ta hakkas isegi tegelikult arvutikaupu müüma või kooperatiivi poodi pidama. Amish Mody\index[ppl]{Mody, Amish} oli mu pinginaaber, tal oli kodus ka arvuti. Ja esimene koht, ma mäletan, kus ma Amigat nägin, oli aasta või kaks noorem koolivend Lemmit Kaplinski\index[ppl]{Kaplinski, Lemmit}, kellel oli isa ilmselt  kirjanikuna kuskilt maailma pealt Amiga kätte saanud. Ja Tähetornis\index{Tartu Tähetorn} olid veel mingisugused  kuulsad arvutid. Noorte Tehnikute Maja\index{Arvutiklubi!Tartu Noorte Tehnikute Maja}, kus olid Yamahad. Ehk siis, ma ütleks, et kui niimoodi loendama hakata, siis võib-olla iseloomustabki toda perioodi see, et kellelgi ei olnud nagu püsivat kohta, vaid otsiti seda aega, kus sa saad.

\question{Neid kohti oli ju Tartu peale palju!}

See on see koht, kus Tartu ikkagi on nagu on Eesti Cambridge või Berkeley, kus suhteliselt väikeses asulass on nii domineeriv ülikool. Kogu see Internet ju hakkas akadeemilistest võrkudest peale. Mäletan, et  Tatu püsiühendused, internetiühendused, mis tekkisid, olid ju ka  satelliidiga Tähetornist Rootsi, ehk enne, kui tekkis Tallinn-Tartu link, tekkis Tartu link mingisse Rootsi ülikooli. 

\question{Kui sa ütlesid, et sa said \enquote{sina ja arvuti} aega, siis kui palju ja mis õpetust sa said, mis tolle arvutiga teha?}

Ma arvan, et see muster on ikkagi enamikel inimestel täpselt sama. Alguses sa tahad mängida, siis sa tahad aru saada, kuidas neid tehakse, siis hakkad natuke programmeerima. Mul oli nii, et ma üheksandas klassis hakkasin pärast kooli programmeerijana tööl käima,  ma olin 15, ma arvan. Tolleks hetkeks ma olin, ma arvan, kuskil kaks-kolm aastat niimoodi omal käel progenud. Mu isa oli Tartu Teaduspargi asutaja. Seal teaduspargis tegutses  mitmeid tehnoloogiafirmasid, ja ilmselt ma eeldan, et ta küsis, et kas keegi seal poisile mingit kasulikku tegevust leiaks. Ja leidus sihuke hulljulge mees nagu Valentin Abramov\index[ppl]{Abramov, Valentin}. Üheksakümnete alguses teatavasti toimus Eestis ohjeldamatu metalli-äri, ja Tartus oli sihuke metallikonglomeraat nagu Primex ja Primexil oli tütarfirma nimega Primex Data\index{Primex Data}, kus tehti igasuguseid asju, põhiliselt pandi mingisuguseid PC-kloone kokku. Siis olid seal mõned inimesed, kes progesid, mingit projektijuhtimistarkvara, mõned inimesed, kes progesid mingit raamatupidamistarkvara, näiteks Tarmo Tali\index[ppl]{Tali, Tarmo}. Valja\index[ppl]{Valja|see{Abramov, Valentin}}\sidenote{Valentin Abramov} palkas mind nii-öelda programmeerijaks, aga tegelikult oli see sellise koolipoisi nagu pärastlõunane ajaviide. Midagi ma vist progesin ka, ma ei usu, et sealt midagi üldse \emph{production}-isse jõudis. Aga kül ma seal haldasin kohaliku arvutivõrku, aitasin arvutit kokku panna või hakkasin BBS-i pidama, oli selline tee-mida-tahad  maalim.

\question{Kõlab, nagu mõnus maailm! Aga sul pidi olema piisavalt alust väita end programmeerija olevat, kas sa õppisid siis raamatute järgi? Internetti ju veel ei olnud?}

Ega neid raamatuid ka ei olnud ju päris alguses kätte saada. See, mis oli seal Noorte Tehnikute Majas\index{Arvutiklubi!Tartu Noorte Tehnikute Maja}, oli vist arvuti ringi nime all, aga ma käisin seal nii hooti või see ei olnud nagu niisugune, et s saad sealt  korraliku progremis alghariduse. Ma mäletan, et ma olen kirjutanud ka paberi peal koodi, et kui on parajasti see periood, kus sul ei ole ligipääsu ühelegi arvutile, aga sa mingite pabermaterjalide pealt nagu üritad midagi tuletada või teha enne, kui sa arvuti taha saad. Mis need olid, Arvutustehnika \& Andmetöötlus näiteks. 

\question{See tundub olevat läbiv joon, et kui meil praegu on küsimuseks, kuidas lapsed programmeerima õpetada, siis nendest lugudest tuleb järjest välja, et keegi ei oska öelda, kuidas programmeerima õpiti. See kuidagi imbus õhust või läbi naha või kuidagi tuli külge. Kuidas see nii on?}

Üks asi, mida ma olen näiteks mõeldud, on see, et eriti, mis puudutab neid nii-öelda kooli-arvuteid, nõukogude-aegseid Agat-e\index{Arvutid!Agat}, Yamahasid\index{Arvutid!Yamaha} ja Jukusid\index{Arvutid!Juku}, et seal oli ikkagi arenduskeskkond esimene asi, kuhu sa ennast alguses sisse \emph{boot}-id. Suhteliselt  raske oli seda arvutit kasutada niimoodi, et sa ei komistaks  arendusvahendite otsa. Kui seda ära võtta iPhone-i, siis pead kurja vaeva nägema, et saada üles keskkond, millega sa saaksid midagi teha, onju. See on kindel niisugune muutus. Ja ma mäletan, et see kooliarvutite ajastu oli nii palju põnev, et kui mul vist onu ostis endale läptopi, mis oli mingi selline DOS-ga, ilmselt Compaq, et kui ma tal külas käies seda kasutasin (ikka nagu tahad arvutiaega) ja kuna seal ei olnud ühtegi arendusvahendit, siis no mida sa teed seal? Kaua sa seal DOS-i \emph{directory}-puus nagu ringi surfad, see nagu  väga huvitav ei ole. Tekstiredaktorid, asjad, selline äri-arvuti. Ma hakkasin nagu just ükspäev mõtlema, et tegelikult see oli hetk, kus ma esimest korda sattusin kasutama arvutit, mis ei olnud nagu ennekõike arendamiseks mõeldud. 

Mäletan, sinuga seoses, kui me kunagi tuttavaks saime, sattusin sulle Võrus külla, kui sa olid laenanud kooli arvuti klassist suveks ühe Agati\index{Arvutid!Agat} koju. Ja seal oli selline ekstreemne juhtum, kus selleks, et üldse midagi teha, sa pidid kõigepealt heksis või BASIC-us sisestama koodi selleks, et saaks \emph{prompt}-i, kuhu saaks hakata midagi kirjutama. Et kui koolipoiss oma suveajal istub ja arvutisse kuueteistkümnendsüsteemis koodi sisse toksib arvutisse,  et sellega  midagi mõistlikku saaks teha, siis selgelt ta suhe selle arvutiga on teistsugune, kui lihtsalt meediatarbimine. 

Ja üks asi on veel, et see arvutite lihtsus või, ütleme, piiratus (kui su visuaalne mängumaa on 25 rida korda 80 tähemäki või hiljem mingisugune EGA või VGA graafika) teeb selle kõik kättesaadavamaks. Nii palju rohkem jäetakse fantaasia jaoks, et tegelikult ka laps suudab  midagi progeda. Et kui keegi teeb tekstirežiimis mängu, siis see ongi nagu nii-öelda selle arvuti tipptase. Kui täna keegi võtab mingisuguse koduse mängu-PC ja teeb seal midagi tekstirežiimis, siis \dots Ühesõnaga, kõik, mis ei ole tohutult videokaardi võimalusi kasutav 3D-renderdus  reaalajas tundub nagu naeruväärne, aga tol hetkel see kõik, mida sa ise suutsid oma kätega teha, ei olnud naeruväärne. 

\question{Kas sinna juurde käis mingisugune raamatute või muusika huvi ka? Selles seltskonnas, kus sa liikusid, pidi ju liikuma ingliskeelset ägedat kirjandust?}

Liikus ikka, aga\ldots Jällegi, Miina Härma oli koolina  selles suhtes nagu äge, et enamik asju, mis seal toimusid ja mis nagu jätsid olid pigem seotud teiste õpilastega. Kooli bände oli kõvasti, ma ise üheski bändis ei olnud küll. Aga 90.-te lõpus Bizarre näiteks, kellega ma väga palju hängisin ja osadega siiamaani läbi käin, näiteks, oli nagu Miina Härma kooli bändist välja kasvanud asi, aga mis oli ka nagu otsapidi väga elektrooniline ja avas minu jaoks arvuti ja muusika seoste maailma. 

Raamatuid loeti, aga  ma ütleks, et ma küberpungi ja \emph{science finction}-i juurde jõudsin pigem üheksakümnendate teises pooles ja siis, kui ma Ameerikasse sattusin. Enne seda ma lugesin võib-olla  pigem Kääbikut ja Tolkieni kui \emph{science finction}-it.

\question{Primexis programmeerijana sa töötasid enne, kui sa Ameerikasse läksid?}

Jah, see oli aastal 1993.

\question{Nojah, mina läksin ülikooli, sina Ameerikasse. Aga räägime korra Primexist. Miks seda softi progeti? Enda tarbeks või taheti sellest mingit äri teha või?}

Minu arust Eesti IT-tööstuse ajalugu on käinud sellise paari lainena. Kaheksakümnendate lõpp üheksakümnendate algus oli see, et kui sa alustad täiesti tühjalt lehelt ja kõigi on arvuteid vaja, siis kõik tõid juppe ja panid arvuteid kokku. Oli Primex Data\index{Primex Data}, kõrval majas või kuskil seal oli Ordi\index{Ordi}, Microlink\index{Microlink} oli Tallinnas, Astrodata\index{Astrodata} tegutses ja kõik need tegid sama asja. Siis liiguti tasapisi tarkvara kihti, aga riigil pigem raha ei olnud,  pankadel raha oli aga võib olla huvi olnud ja tekkisid firmad, kes arendasid tarkvara teenusena. Kogu see Helmeste ja Webmediate laine on selle kõige tugevam näide võib-olla. Ja  tänaseks on \emph{mainstream}-iks  muutunud  toodete ehitamine. 

Primex Data oli naljakas hübriid,  et ühest küljest oli seal see arvuti-äri, mis oli see, mida kõik tegid ja kust nagu tuli põhiline käive. Aga teisest küljest tarkvara hakati tegema nagu ikkagi tootena, need olid nagu mingid asjad, mida nad lootsid ilmselt flopi peal nagu müüa. Mingisugune turg tekkis, Merit Tarkvarad\index{Merit Tarkvara}\sidenote{Ehitab raamatupidamis- ja personalitarkvara, tegutseb siiani.} ja Eetasoftid\index{Eetasoft}\sidenote{Ehitab Eeva-nimelist raamatupidamistarkvara, tegutseb siiani.} ja kõik need raamatupidamistarkvarad ja osa neist tegutseb ju siiamaani. Projektijuhtimise tarkvara tegid kaks progejat nimega Urmas ja Jürgen, kes  minu arust tegid sellest samal ajal Tartu Ülikoolis oma magistritööd. Projektijuhtimine, gantti graafikud ja selle kohta eesti keelne tarkvara, ilmselt see oli nagu sihuke akadeemiline asi, mida nad lootsid müüa ka. Jällegi ma ei mäleta, et sellest mingit suurt äri oleks nagu tekkinud. 

Ja sellise hübriidi  pidamine oli suhteliselt jabur. Ma ei mäleta, kas oli minu arvutiga või Tarmo Tali\index[ppl]{Tali, Tarmo} arvutiga või mõlemaga (me istusime kõrvuti), et tuled pärastlõunal tööle ja võtad, et nüüd siis hakkaks progema, ja selgub, et keegi on su arvutist mälu ära müünud, näiteks. Ja siis sa  alustuseks tegeled sellega, et enne oli kaheksa megabaiti mälu, aga äkki laost neljamegabaidise kivi kuskilt leiab. 

Üks näide tarkvara maailma läbi põimumisest muu maailmaga oli, et istud ja nokitsed midagi teha sisse tuleb  Jaan Tallinn\index[ppl]{Tallinn, Jaan}, kellest ma olin kuulnud. Kosmonaut ja nii, legendaarne mängutiim. Ja Jaan tuleb monitori ostma! Niisugusel põhikooli või keskkooli poisil käed värisesid isegi, kui ma riistvara müügiga tegelenud, aga jube põnev oli!

\question{Seda on mitmest loost kosta, et tol ajal oli arvuti-äri nii änksa marginaaliga, et selle kõrval sai igasugu asju pidada. Tudengid projektijuhtimise tarkvara kirjutamas või siis sinu amet ilmselt ka. Vaatame, mis juhtub, miks mitte!}

Täpselt. Minule on see andnud näiteks, et ma kindlasti täna võtan oluliselt parema meelega endale praktikante interne, töövarjusid. Kui palju mind see võimalus mõjutas, mis Valja \index[ppl]{Valja|see{Abramov, Valentin}} mulle andis! Nad maksid mulle isegi palka aga see oli selgelt, olukord, kus poleks mingit vahet olnud, kui nad ei oleks mulle palka maksnud.

Seal tegelikult teine huvitav asi oli see, et ma sattusin esimest korda võrkude juurde. Enne seda oli ikkagi arvuti nagu iseseisev eraldi olev asi. 1993 oligi täpselt umbes sel ajal, kui Tartusse ilmus Internet ja ma pakun, et Primex Data või vähemasti Teaduspark\index{Tartu Teaduspark} võis olla esimene koht, kus mitte-ülikooli asjad sattusid võrku. Jadaühendusega võrgud, eks, otsas on terminaator ja saad ilge siraka, kui arvuti on maandamata. See võrk nägi välja nii, et reaalselt nagu kuskilt läbi seina tuleb mingi ots ja sa ei tea, mis masinad veel selles jadas on. Kõik on ühes võrgus maja peal laiali. Meil oli 1993. aastal  püsiühendus internetti ja lynxi-põhine veeb enne kui Mosaic nagu välja ilmus! Ühest otsast ma pidasin Primex Data nime all BBS-i ja teistpidi oli meil olemas püsiühendus, kust oli võimalik, ma ei tea, Doom-i\index{Mängud!Doom} demo või Wolfensteini\index{Mängud!Wolfenstein} mingid asjad juba FTP-ga kätte saada ja BBS-i üles panna. Paljude teiste BBS-ide jaoks oli failide levitamine see kõik selline modemiga \emph{peer-to-peer} loksutamine, aga me olime nii-öelda pumba juures. 

Teine asi, mis ma mäletan sellest, kuidas arvutifirma hobina tehes hoopis teistsugune välja nägi. Meid oli sisevõrk, sisevõrgus oli Novelli server, millel oli 300-megabaidine kõvaketas\sidenote{300 mega oli ulmeliselt suur andmemaht. Tuletagem meelde, tol ajal oli normiks 3.5" flopiketas mahutavusega 1.44 megabaiti. Ja ka see oli minu jaoks uus asi - Võru inimene oli suutnud hankida Tallinna komisjonipoest mõned 5.25" kettad mahtuvusega mõnisada kilobaiti ning neile kääridega (!) lisa-sälgu lõiganud, et oleks võimalik salvestuseks tarvitada ketta mõlemat poolt. Neid kettaid sai spetsiaalses karbis kaasas kantud, sellesse karpi mahtus kogu mu digitaalne elu.}, millest tööasjadeks oli kulutatud umbes paarkümmend megabaiti. Kood, mida seal kirjutati ja paned hinnakirja wordi fail, eks. Ja ülejäänu laadisid mingisugused hollandi tüübid öösiti mingit tarkvara täis, sellepärast et kui sa teed ukse lahti kuskil Ida-Euroopas, kus ei ole ilmselt ka intellektuaalse omandi kaitset\ldots

\question{Tollal isegi mitte Euroopas polnud isegi ligilähedaselt nii rangeid intellektuaalse omandi reegleid, rääkimata veel Eestist!}

Jah. Ja hommikul tuulad tolle ketta läbi ja vaatad, mis asju BBS-is ülejäänud Eestiga jagada. 

\question{Päris ikka nii ei olnud, et avad aga FTP pordi või telefoninumbri ja muudkui hakatakse väärt kraami laadima. Sul pidi mingi võrgustik olema? Kuidas see tekkis?}

Ma arvan, et usenet ja selle mingid uudisgrupid oli võib-olla esimene selline kogukond, kuhu ma rahvusvaheliselt sattusin. Eesti Fidanetiga grupid ka. Ma arvan, et \emph{overlap} ilmselt oli täitsa olemas, kus Eesti Fidoneti gruppides arutati juba, kus Internetis käia ja kus keegi istub, kus tarkvarale ligi pääseb. Ma arvan, et minu jaoks igasugused sellised reaalaja \emph{chat-room}-id tulid hiljem. Random\index{Jutukad!Random} ja teised jututoad. Ma istusin ka mingites IRC kanalites, aga ma  ei mäleta, et sealt oleks mingisugused tohutu side või sõpruskond tekkinud. See oli nagu rohkem \emph{ad-hoc}. 

\question{Kust sul üldse tekkis mõte hakata BBS-i pidama\index{BBS!Primex Data}? Anti tööülesanne?}

See on hea küsimus! Ma mäletan, et idee ma müüsin küll Primexile maha sellega, et jube kasulik on tärkavas võrgus nähtaval olla. Pagan teab, võib-olla  oli äkki see, laos oli modem, et siis tekkis küsimus, mida sellega teha saab. Tõesti, mul kodus ei olnud arvutit üheksakümnendate lõpuni, et ei olnud nii, et ma oleksin olnud BBS-ide kasutaja ja nüüd tekkis võimalus üks ise püsti panna. Juu ta ikka sedapidi oli, et laos oli modem, sellega sai helistada sisse teistesse BBS-idesse, mis juba Tallinnas olid.

Kusjuures, enne kui sa külla tulid, ma hakkasin mõtlema, et ma mäletan Fidoleti \emph{node} numbrit peast jätkuvalt, mis oli 2491/2.2. Tartu tsoon oli vist kaks, kõik Tallina asjad olid ühe all. Ja siis selguski, et Tartu tsoon on veel suhteliselt hõre ja on võimalik olla teine BBS Tartus, esimene oli vist Jaan Pruulmann\index[ppl]{Pruulmann, Jaan}.

\question{See tähendab siis seda, et Primexi BBS oli olemas enne, kui Luciferi\index{BBS!Lucifer BBS} oma?}

Mis tema number oli? Ma ei tea, aga Veikot\index[ppl]{Tamm, Veiko}\sidenote{Veiko Tamm, Lucifer BBSi pidaja. Vt. lk. \pageref{chptr:lucifer}.} tundes võis see olla 666 näiteks, kui ta omale sellise nime panni. Et võib olla need numbrid ei olnud puht lineaarsed. 

\question{Panid BBS-i püsti, üks liin, üks modem\ldots}

Jaa. Ma arvan, et alguses oli 2400 boodine modem, hiljem oli kiirem.

\question{Kui ma olen küsinud, mida inimesed BBS-is hoidsid, siis reeglina on öeldud, et seal hoiti endale huvitavana tundunud asju. Mis sorti kola sul seal oli?}

Ma arvan, et see värk oli suhteliselt kaootiline. Huvitav, kas see \emph{file list} oleks kuskil alles? Ilmselt oli mingisugune mängude teema, siis mingisugune \emph{utility}-te teema oli kõigil. Ressursi kitsikuses on mingid asjad, mis alati olid  huvitav või vajalik ja kiiresti tehnoloogia, näiteks pakkimisalgoritmid. Zip ja arj\sidenote{ARJ (\emph{Archived by Robert Jung}) oli üheksakümnendatel väga levinud efektiivne pakkimisformaat, mis praeguseks on enamasti unustatud. Tema eeliseks oli võimekus suuri faile sujuvalt mitme flopi peale laiali jagada.} ja siis tuli midagi uuemat. Pluss nendega nendega oli see rõõm, et nad olid väiksed, neid sai kiiresti liigutada, kui midagi uut tuli.

\question{See oli ju osa rutiinist, et esimesena asjana pidi arvuti juures käepärast olema mingi pakkija, siis mingid vahendid mälu laiendamiseks, see kraam oli igapäevaselt kasulik!}

Tegelikult ruum ja modemi aeg oli ju ikkagi kallis. Ja kui sul on mingisugune öine meili sünkimine on läinud jälle tunniajaliseks, siis jälle vaatad, et võib-olla kõiki neid gruppe pole vaja, mida ise ei loe. See pool oli ka. 

\question{Kui palju sul sellest aimu oli, kes sul BBS-is küljes käisid? Või lihtsalt mingid numbrid helistavad?}

Jah, minimaalselt mäletan sellest, ma ei tea isegi, kui palju sellest teadsin. Pigem ma mäletan seda tunnet, et kui sa ise juhtumisi seal olid (ma öösiti käisin ikkagi kodus magamas)  ja  modem kõne vastu võttis, et siis nagu põnevusega vaatad, mida see inimene teeb seal. Ma ei mäleta, kas see oli juba tol hetkel äkki Windows või oli OS/2\index{OS!OS/2}, millega ma katsetasin. Aga vist OS/2 peal said jooksutada DOS-i virtuaalmasinaid või  DOS-i programme kuskil aknas. Mul multitaskis see asi taustal isegi, kui ma tööd tegin.

Teine teine asi, mis meil oli, oli välja helistamine. Minu arust istus Tartu Teaduspark\index{Tartu Teaduspark} Tatu kõige vanema telefonijaama taga, mis oli legendi järgi (ma ei tea, kui palju sellest tõsi oli) viiekümnendatel püsti pandud analoogjaam. Ja me üritasime teda ikkagi nagu ühte ja teistpidi lahti häkkida ka. Üks asi, mis minu arust meil korra töötas (võib-olla niisugune linnalegend) oli see, et kettaga telefonil on ühest kuni üheksani klõpsude arv, ja kümme klõpsu on null. Ja me lugesime kuskilt, et kui sa teed 11 klõpsu, siis sa saad kätte kaugekõne \emph{trunk}-i. Tegime 11 klõpsu, saime teise tooniga heli ja helistasime mingisse Hongkongi BBS-i ja minu arust ei tulnud selle peale kunagi arvet. Aga see lõbu kestis jube vähe, sest see oli tõesti nii muldvana jaam ja ilmselt palju kasutajaid taga ja vahetati esimeste seas digikeskjaama vastu välja. Üheksakümnendate lõpus, kui ma esimest korda mingeid häkkeri-ajakirju nägin, 2600\sidenote{Pika nimega \enquote{2600: The Hacker Quarterly}. Enamasti lugejate endi poolt sisuga täidetud kultuslik perioodiline ajakiri, mis käsitles kõikvõimalikke arvuti-, interneti- ja telefonisüsteemidega seotud tehnilisi teemasid ning üldisi arvuti-\emph{underground}-i uudiseid. Nagu varasemalt (vt. lk. \pageref{sisu:2600}) juttu on olnud, oli 2600 Hz. toonaste telefonijaamade jaoks oluline sagedus, sealt ka publikatsiooni nimi.} ja teised, ja neist USA kaheksakümendate lõpu ja üheksakümnendate alguse \emph{phone phreaking}-u laine kohta lugema hakkasin,  siis see oli korraks Eestis ka relevantne. 

Teine asi, mis kindlalt töötas oli see, et kui telefoni kaardil (selle pooleaastase perioodi jooksul, kui Eesti Telefoni juurutas \emph{chip} kaadiga telefoniautomaadid\sidenote{See aeg kestis siiski kauem, tõenäoliselt andis Eesti Telefon kiibiga kaarte välja aastatel 1995 kuni 2010. On iseküsimus, kui palju sajandivahetuseks kaarti aktsepteerivaid taksofone järel oli.}) üks klemm kinni teipida, sai tasuta helistada. See oli ka selline tehnoloogiahugiliste noorte rõõm.

\question{Kust te sellest kõigest teada saite ja kuidas need 2600-d Eestisse sattusid?}

Ma arvan, et kui sa Fidonetis saad nagu esimese ringi Eesti gruppidele peale ja lisaks tellid  mõne USA grupi või siis kuskil Usenetist mõne häkkerigrupi, siis seal osa neid asju oli lihtsalt ASCII tekstina olemas. Paberkoopiat ma nägin siis kui USA-sse läksin, see oli  niisugune šokeeriv kogemus. Ilmselt usklikel on samasugune tunne, et kui satud mingisuguse piibelliku teksti originaali juurde. Et kui sa raamatupoes näed, et see asi on paberil, füüsisielt, värvilisena täiest olemas. 

\question{Ameerikasse sa sattusid vahetusõpilasena keskkooli ajal?}

Jah. Ma tegin avalduse Rotary klubi\index{Rotary klubi} vahetusprogrammi, mille ankeet oli hästi selle keskne, et aru saada, et kes see kooliõpilane on. Rotary vahetus oli nagu kahesuunaline, üks klubi saadab kellelegi kuskile välja ja siis samal ajal võtabki mujalt vahetusõpilasi vastu. Aga see, kuhu nad välja saatsid, juhtumisi, ma ei tea, kas sellepärast, et mu ankeet nii arvuti-asjade keskne, aga ma oma Eesti mõttes 11. klassi sattusin Silicon Valley keskele. Ehk siis sihukeses õrnas eas elasin aasta aega Cupertinos, kus on muu hulgas Apple'i peakontor, ja käisin Monta Vista High nimelises keskkoolis. Jällegi,  paljude juhuste kokkulangevus: USAs oli just mingi \emph{information superhighway} hullus puhkenud ja Al Gore\index[ppl]{Gore, Al} oli aasta enne minu sinna kohale jõudmist USA-s viis kooli kuulutanud  interneti pilootkoolideks.  Ja a Monta Vista High üks nendest. Põhimõtteliselt sa asud Apple'i peakorterist mingisuguse paari kilomeetri kaugusel ja meil oli, kui ma õieti mäletan, koolis 1400 õpilast ja 800 arvutit. Ja selgus, et arvutilaboris assistent olemise eest saab ainepunkte või et sa võid nagu ühe tunni asemel iga päev istuda arvutilaboris, kus olid Mac-id, Sun-id, Silicon Graphics-i mingid asjad. Yahoo-d kasutasin aadressil yahoo.cs.stanford.edu, sest ei olnud veel firmaks muutunud.

\question{Ühesõnaga, sattusid paradiisi!}

Põhimõtteliselt küll, jah. See  on kindlasti tohutult mõjutanud seda, mis edasi sai. 

\question{Kuidas sa toime tulid sellega? Ma saan täiesti aru, et see võtab jalust nõrgaks, kui sellisesse kohta sattuda sisuliselt nõukogude liiduvabariigist?}

Hea küsimus. See oli mu esimene lend Eestist välja. Üksi. Ma arvan, et vanemad pidid raha laenanma, et seda lennukipiletit saaks lubada ja see kõik oli ka ilmselt nende pool tüsna hullumeelne. Aga, ütleme nii, et see on nagu vanemate asi muretseda.  Ega sa ise sellises vanuses lihtsalt teed ja lähed ja oled ja oled nagu käsn ja võtad kõike seda sisse, mis tuleb. Koolikogemuse mõttes oli see, et kui sa Eestist tuled ja kui sa oled  matemaatika-füüsika huviga ja vähegi nagu olümpiaadil käinud ja mingit sihukest elu elanud, siis olles seal 11. klassis, olid kõik mu reaalained 12. klassi \emph{honours}-taseme ained. Ja ma olin kõik selle Eestis juba läbinud, koolisüsteemid on nii palju erinevad. Teistmoodi oli see, et  esimest korda elus pead minema ja mitte lahendama üksiküritaja ära oma testi, vaid  sa pead moodustama grupi kolme inimesega, kellega sa ei ole enne koos töötanud, midagi koos välja mõtlema ja klassi ees ette kandma, mis sa tegid. Eestis reaalainete tugevus oli olemas aga sellised asjad olid need, mida  me seal pidin esimest korda tegema. Õppeviis oli seal pigem selline ebamäärasem. 

Ühiskondade kontrast oli küll. Kui ise ma olin enne seda juba nagu tööl käinud, siis sinna jõudes ka väga heasoovlikud klassivennad vahest küsisid, et kas teil Eestis telekaid on. Nende arusaam sellest, mis seal raudse eesriide taga  toimus, oli üsna hägune. 

\question{Ma saan aru, et sa käisid seal ka tööl?}

Ma ei tohtinud seal tööl käia, vahetusõpilase värk. Alguses see kurvastas mind väga, aga siis ma sain aru, et seda defineeritakse läbi palga. Ehk ma käisin mingisuguses arvutipoes pärast kooli abiks, ma ei tohtinud palka saada ja kui ma Eestisse kolisin, siis see arvutipood kinkis mulle arvuti. See, oli see lahendus, mis ma leidsin. Aga seal mind käsitleti tõesti pigem nagu koolipoissi, kellel lubati arvutit kokku panna, et IT-äri oli ju palju reglementeeritum kui see, mis  Eestis samal ajal toimus. 

Teine asi, mis mulle meelde tuli ja mida me ei puudutanud ja mida me enne Ameerikasse minekut alustasime, et tekkis sihuke rühmitus nimega Interactive Aspelungs\index{Interactive Aspelungs}. See juhtus, ma arvan, seal üheksandas-kümnendas klassis.  Mina, Mark Tehver\index[ppl]{Tehver, Mark}, Kristjan Jansen\index[ppl]{Jansen, Kristjan} ja natuke hiljem Alari Aho\index[ppl]{Aho, Alari}. Mark ja Kika\sidenote{Kristjan Jansen}\index[ppl]{Kika|see{Jansen, Kristjan}} olid Treffneris\index{Koolid!Hugo Treffneri Gümnaasium}, mina ja Alari olime Miina Härmas\index{Koolid!Miina Härma Gümnaasium}. Ja Mark\ldots Ütleme, et miks ma ei ole tänapäeval programmeerija oli see, et mul õnnestus väga õrnas eas näha lähedalt inimesi, kes tegelikult oskavad programmeerida. Ja Mark oli juba tol hetkel koolipoisina inimene, kes hommikul hakkas tekstifaili assemblerit kirjutama ja õhtul pani selle käima ja see töötas. Kirjutas näiteks mingisuguse graafikamootori või midagi. See sümbioos, mis meil tekkis, seal oli selline, et Mark proges, Kika disainis. Mina korraldasin asju, mis võis tähendada mida iganes, alates sellest, et Primexist laenata SoundBlasteri kaarti, et Mark saaks ka sellele audiodraiveri kirjutada. Alari Aho tegi muusikat ja see toode, mida me ehitasime, oli arvutimäng nimega Drunkard\index{Mängud!Drunkard}.

Ma isegi ei suutnud praegu välja mõeldud, mis ma seal seltskonnas olin, \emph{organizer} vist neis ametlikes paberites, aga  mina hakkasin seda maha müüma. Olin tihedas kirjavahetuses Epic Megagames-i\sidenote{Praegu tuntud kui lihtsalt Epic.} ja Apogee\sidenote{Praegu tuntud kui 3D Realms.} ja mingite selliste ettevõtetega\sidenote{Mõlemad mainitud ettevõtted olid omal ajal mängutööstuse absoluutsed gigandid}, kes nagu  olid valmis meiega rääkima.  Ja kui ma USA-sse läksin, siis tekkis  veider olukord, et ma sain USA postiaadressilt saata flopiga demosid ja välja näha, nagu meil oleks nagu mingi päris firma. 

Aga me ei teinud seda asja kooli kõrvalt lõpuni valmis, ainult demod olid olemas. Ja me panime ka paari aastaga selles suhtes mööda, et me kirjutasime 2D platvormikat, mis ilmselt 1991.aastal oleks selle taseme juures, mis Mark\index[ppl]{Tehver, Mark} ja Kika\index[ppl]{Jansen, Kristjan}  kokku töötasid, olnud ilmselt täiesti vabalt müüdav, täpselt nii nagu Bluemooni\index{Bluemoon} kutid oma mänge maailma viisid. Aga aga me komistasime täpselt sinna hetke, et kui meie saatsime demosid, siis  ID Software Wolfenstein\index{Mängud!Wolfenstein 3D} oli juba väljas\sidenote{Avaldati 5. mail 1992} ja Doom\index{Mängud!Doom} tulemas\sidenote{Avaldati 1993. aastal.} või midagi sellist. Ehk siis, see, mis me enne rääkisime ka, et graafika tase läks sinna, kus Tatu koolipoiste platvormikas ei paistnud enam säravalt silma. Aga põhimõtteliselt see asi isegi nagu töötas.

\question{Ise mängu kirjutamine tundus toona täiesti hoomamatu ettevõtmine: muusika mängib taustal, kuidas sa seda teed?!}

Isegi tänapäeval, aga tol hetkel iga mängukirjutaja alustas sellest, et sa proged endale töövahendid. Ega sul ei ole tööriistu selleks, et levelit disainida või midagi. Mingisugused \emph{tracker}-id, selliseid nelja või kaheksa kanaliga, taustamuusika tegemiseks olid olemas, mida jällegi Bluemoon\index{Bluemoon} tootis. Aga \emph{level}-i disainimiseks või isegi animeerimiseks. Mingi pildi \emph{editor}-iga tegid spraidi\sidenote{Ingl. \emph{sprite}. Kahemõõtmeline ühik rastergraafikat, mis integreeritakse suuremasse stseeni.} valmis, aga kui seda animeerida, siis selle jaoks sa pead jälle mingi omava töövahendi tegema, meil oli kogu see \emph{stack} olemas.

\question{Miks te seda tegite? Tundus äge? Tahtsite rikkaks saada?}

See tundus ikka lihtsalt äge, ma arvan. Mark\index[ppl]{Tehver, Mark} ja Kika\index[ppl]{Jansen, Kristjan} tegutsesid juba enne, neil oli juba hoog sees. Peaks küsima, et kuidas me kokku saime. Ma ei tea, ta vist võttis selle maha, aga paar aastat tagasi oli Kika kokku kogunud kõik meie tolle hetke kirjavahetuse ja pani selle avalikult internetti, kõik  pildifailid ja mis meil tollest hetkest olid. See oli juba hea mitu aastat tagasi, oli  sihuke nostalgiarännak. 

Mängu taustaloos\ldots Ma mäletan et peategelasel, Drunkardil, oli oluline, et ta veres alkoholitase ei langeks. Selleks pidi ta korjama pudeleid, siis ta sai tühjade pudelitega loopida. Ja olid mingid olukorrad, kus ta pidi ajutiselt sama natuke kainemaks. Markil\index[ppl]{Tehver, Mark} ja Kikal\index[ppl]{Jansen, Kristjan} oli üks klassivend Treffneris, kes kui ta koolipeol liiga palju õlut jõi, ta hakkas kükke tegema, et kainemaks saada. Ja Drunkardiga oli ka nii, alla noolt hoiad all, siis ta tegi kükke ja alkoholi tase veres langes. Sellised olulised asjad.

\question{Mingi loovuse element on siin sees, sest ei tulnud lihtsalt joosta ja kirvestega loopida?}

Täpselt, meil mingi idee oli, et me tahaks teha mängu, kus ei käi nagu tulistamine ja tapmine ja mis oleks nagu selgelt teistsugune. Ma ei mäleta, kui teadlikult me seda mõtlesime, et sihtida täitmata turu-nišši, et teha nii-öelda täiskasvanute mäng. Muidugi alaealiste Ida-Eurooplaste arusaam sellest, mis asi on \emph{adult entertainment} oli mõnevõrra teistsugune kui päris \emph{adult}-idel, aga selgelt vastas üheksakümnendate Tartule.

\question{Kas sa seal Californias veel midagi huvitavat tegid?}

Seal ma ka ikkagi progesin. Ma ei mäleta, kas ma seal natuke pidasin BBS-i. Aga kui sa eraisikuna pead nagu koduliini peal BBS-i, siis see ei ole nagu see ja pluss sa satud  USA telefoninumbrite ruumi. Tegelikult olin ma ikkagi pigem BBS-ide kasutaja. Ka graafiline veebibrauser  ilmus orbiidile ja  pilt hakkas muutuma. Ma kirjutasin seal hobiprojektina ka ühte BBS-i softi. Tundus, et see võiks olla vajalik. Nagu sageli teed, et et kui sa mängu ei proge, siis mis on see asi, mida sul endal vaja on. Ja selle käigus vist juhtus niimoodi, et  uurides, mis veel turu peal on, leidsin ühe ägeda BBS-i softi, mis mulle meeldis, mis oli \emph{shareware} ja kinni keeratud. Ma murdsin selle lahti. Noore häkkerina, ei midagi  komplitseeritut. Jooksutad seda asja \emph{debugger}-is, vaatad, et kui kui ootamatute kohtade peal hüpatakse  mälus mingi aadressi peale, tehakse seal müks mingi väga lihtne tehe. Masinkoodi tasemel muudad ära, et sinna enam ei hüpataks ja ootamatult selgub, et ongi koopiakaitse maas. Raporteerisin selle autorile, et selle kaitse niimoodi maha võtta, ja ta andis mulle eluaegse tasuta litsentsi. See võttis mul oluliselt motti maha, et oma BBS-i softi kirjutada, sest mul oli selle olemas. 

Teine asi, ma mäletan,  oli ka õudselt hea õppetund  enda tasemest progejana. Ma kohutavat abstraheerisin selle asja üle. Kirjutasin C++-s\index{Keeled!C++} BBS-i, kus ma katsusin hoida väga puhtalt eri kihtidena näiteks seda,  kuidas käib modemi händlimine, kuidas käib terminali händlimine olles valmis igasuguseks tulevikuks, et sul neid asju, millega liidestuda, on mitmeid. Ühesõnaga ma olin kogu aeg nagu jube  kaugel kogu aeg sellest, et see asi  minimaalses skoobis töötaks. Hilisema elu tarkvaraarendusprojektideks väga hea õppetund. See, et ma täna rohkem start-upidega teglen,  MVP on nagu kuidagi armsam. Pigem vähem aga varem.

\question{Ühel päeval tulid sa Ameerikast tagasi kaasas arvuti ja raamatud?}

Jah,  umbes sihukesed paar kasti olidki. 

\question{Mis sa siis tegid?}

Siis ma läksin keskkooli edasi, 12. klassi. Aga  mul oli juba natuke hoog sees ja ma läksin tööle sellisesse firmasse nagu Triip\index{Triip}. Mis oli Tartus algselt trükikoda ja disainibüroo. Mulle on eluaeg, isegi siis, kui ma progesin, meeldinud see, kus tehniline ja visuaalne osa kuidagi kokku saavad. Eluaeg, kõik asjad, mida ma olen teinud, ma olen alati töötanud koos progejate ja disaineritega, ka hiljem.

Ahaa, USA-s üks asi, mis me tegime. Mäletad, olid sihukesed kunstirühmitused kes tegid ASCII \emph{art}-i ja hiljem VGA \emph{art}-i ja siis mingite varjunimede all koos ühe koolivennaga me isegi komistasime paari sinna sisse. Minu ASCII ja ANSI \emph{art} on isegi olnud mingisugustes distributsiooni pakkides. Ja USA-s üks kooliaine oli \emph{commercial art},  selline tootedisaini ja pakendi ja mingi selline asi. 

Ja selle tausta pealt ja nende näidistega ilmusin  Triipu välja. Ütlesin, et ma tahaks pärast kooli natuke arvuti taga istuda, mis antud juhul tähendas disainimist ja siis sattusingi sinna tööle.  See kõik oli nagu üsna kaootiline. Teine mind hästi-hästi palju mõjutanud inimene tol hetkel oli Marek Tiits\index[ppl]{Tiits, Marek},  kes pidas Balti Uuringute Instituuti\index{Institute of Baltic Studies}, mille alt ta tõstis edukalt mingisuguseid eurorahasid ja tegi ägedaid projekte. Näiteks Eesti Seaduste otsingumootori, Ivo Mehide\index[ppl]{Mehide, Ivo} oli veel seal ja. 

Ja Marek ka kuidagi nagu andis mulle, kui lihtsalt  ringi hängivale koolipoisile võimaluse, et tule, tee midagi aeg-ajalt. See tähendas jällegi ligipääsu arvutitele Tähetornis\index{Tartu Tähetorn}. Mis oli ka väga naljakas koht. Seal oli näiteks üks Slicon Graphicsi\index{Arvutid!Silicon Graphics} arvuti, millel oli veebikaamera. Üheksakümnendate keskel! Sellel oli ka oluline funktsioon, sinna oli võimalik sisse logida ja vaadata veebikaamerast, kas kohvimasin on täis jooksnud. Ei pidanud tühja tassiga alumiselt korruselt teisele tulema. Seal oli üks mingi erakordselt oluline projekt, ma isegi  ei mäleta, miks seda vaja oli. Igatahes majas oli üks Zyxeli\index{Zyxel} modem, millel oli ka faksi funktsionaalsus ja  majas oli ka internet.   Ja ma kirjutasin Perlis\index{Keeled!Perl} veebipõhise faktide saatmise ja vastu võtmise aplikatsiooni. Et kui keegi saatis sellele numbrile faksi, siis võttis modem vastu, kirjutas failid mingisse Suni serverisse maha ja üle veebi oli võimalik neid fakse näha. Ma ei ole kindel, kas see oli asi, mis ma ise mõtlesin, et võiks teha ja Marek ütles, et \enquote{tee muidugi} või oli see mingi asi, mis mingi projekti jaoks oli vajalik, aga sihuke asi seal eksisteeris.

\question{No mis, veel aastaid hiljem ühes teatavas suures ettevõttes, mis kindlasti ei olnud telekommunikatsiooniettevõte, räägiti sellest, et äge oleks üle interneti faksi saata!}

Minu arust kaks aastat tagasi\sidenote{Ajasime Steniga juttu novembris 2019.} tegid mingid Twilio\index{Twilio} progejad aprillinaljana Twilio faksi API ja nüüd on see mingi oluliselt kasvav ärisuund. Sellepärast et võrgus on veel miljoneid inimesi, kes tahavad kogu aeg faksi saata. Mõtle, mis kõik oleks võinud olla! 

\question{Räägi Triibust palun!}

Juss, Juhan Peedimaa\index[ppl]{Peedimaa, Juhan}, oli mu kontakt seal ja Eva, kes nüüd on ka Peedimaa\index[ppl]{Peedimaa, Eva}. Ja Priit Jagomägi\index[ppl]{Jagomägi, Priit}. Jagomägede perekond on kuulus Regio ja kartograafia taga, Priit oli selline \emph{rebel} vend, kes tegi oma firmat ja mitte mitte ei töötanud Regios. 

Põhimõtteliselt päris mitmed asjad, ma arvan Eestis juhtusid  sellepärast, et sul oli nagu täiesti tühi maa. Ma mäletan seda hetke, kui ma keskkooli lõpetasin, Eestis oli umbes 40 panka ja keskmine panga CEO vanus oli umbes 28. Et tekib nagu tunne, et sa oled mingist rongist maha jäänud, kui sa oled 18. Ja see tehnoloogiaettevõtlus ning internetiga seotud ettevõtlus oli tegelikult  laine, mida me tol hetkel ei teadnud, et me sinna me maandume. Aga  oli selge, et rong ei olnud veel läinud, et me saame omale rongi ehitada. Triip\index{Triip} oli täpselt selline. Ta alustas sellest, et Eestis oli täiesti tühi maa, iga päev tehakse kümneid firmasid, igale firmale on vaja logo ja visiitkaarti,  hakkaksidki neid tegema. Disainist tuli nii palju raha peale, et ostsid oma trükikoja, ostsid veel mingisuguseid asju kokku ja siis selle sisse komistasime selle internetiasjaga. Kujundati mingisuguseid trükiseid ja reklaame, aga ma hakkasin nendesamade klientidele ka veebilehti tegema. Ja sealt kasvaski välja, et kui ma keskkooli lõpetasin, siis ma läksin Triibust ära ja tegin oma esimese ettevõte, mis tegeles veebiga \emph{full time}, ehks siis Halo\index{Halo}. 

Halo aegadest oli ilusamaid mälestusi, näiteks, et kui hommikul lähed kontorisse ja  vaatad, et üks disainer poeb valgete linade vahelt välja ja selgub, et ta tõsteti juba kolm kuud tagasi ühikast välja. Lihtsalt mina arvasin, et ta on kogu aeg esimesena ööl ja lahkub viimasena. Tegelikult ta lihtsalt elas seal\sidenote{Lugu vastab minu mälu järgi tõele ja, minu mälu järgi, ei olnud see ainus juhtum, kus Halo kontor kellelegi ajutist peavarju pakkus. Üheksakümnendate Tartu oli mõnevõrra teistsugune startupi-keskkond, kui selle sajandi kahekümnendate Tallinn.}.

\question{Sa rääkisid, et oli võimalus nii-öelda \enquote{oma rong ehitada}. Kas sul oligi sihuke selge visioon lainest ja endast selle peal või lihtsalt tegelesid asjadega, mis mõnusad tundusid?}

No ette mõtlemist oli selgelt liiga vähe. Seesama Halo läks ikkagi nelja aastaga pankrotti ka, sest mingi piirini  on võimalik asju ehitada intuitsiooni pealt, aga mingil hetkel peaks nagu läbi ka arvutama või mõtlema, et mida sa teed. 

See oli mingil määral selline avastus, et see on asi, mida inimestel vaja on või  suurtel firmadel. Et kes juba Triibus hakkasid kliendiks saama ja Halo  kliendid olid, oli täiesti kes-on-kes: kõik suured pangad,  rahvusvahelised brändid, kes olid Eestisse jõudnud nagu Audi, või ESS, mis tol hetkel tohutult kasvas. Mingid sellised kaubamärgid, mida kõik teadsid. Ja see internetivärk oli nende jaoks nii arusaamatu. Sinu jaoks on see intuitiivne ja lihtne, \enquote{milles siis probleem on, teeme ära}, ja need suured kaubamärgid on  nõus nagu alla neelama selle, et nad ostavad mingite kaheksa- ja üheksateistaastaste Tartu kuttide käest teenust. Ei osanud me seda hinnastada ega midagi, et iga kord mõtled, et \enquote{kurat, see number kõlab nagu liiga suur},  alguses kõik toimis nagu liiga hästi või liiga lihtsalt. 

1996 alustasime Haloga\index{Halo}, 1997 kevadel ostis üks suur reklaamikett, DDB, kontrolli Halos juba ära ja  me sattusime ootamatult nagu päris ärikeskkonda. Kus olid päris inimesed, mingid Soome juhid, kes olid nõukogus, tahtsid mingisuguseid eelarveid näha ja mingisuguseid asju. Ühest küljest oli see kõik nagu varases nooruses  kokku puutumine päris asjadega.  Teistpidi ka need nagu reklaami-inimesee,  investorid siis tol hetkel, ka nemad läksid parajasti internetibuumi sisse. Ka nende kliendid loopisid raha vasakule ja paremale.  Kes seal olid, Razorfish, kes võttis  100 000 dollarit esimese kohtumise eest, mingid sihuksed agentuurid. See tähendas, et ka nemad ei suutnud kõike ette näha. Me näiteks palkasime selgelt kohe liiga palju inimesi, sest kohe-kohe pidid  lääne kliendid tulema Eestisse asju arendama, meil olid tõsised jutud DDB keti sees. Ja kui see mull  2000. aastal lõhkes\sidenote{NASDAQ Composite indeks tõusis 1995. ja 2000. aasta märtsi (mida loetakse mulli lõhkemise hetkeks) vahel 400\% ning kukkus 2002. aasta oktoobriks 78\% kaotades kogu mulli jooksul kogutud kasvu.}, siis ta lõhkes kõigi jaoks korraga, lõhkes meil Eestis ja lõhkes seal. Ja vot see oli asi, mida me  täiesti roheliste tüüpidena ei suutnud ette näha. Et majanduses  on tõusud ja langused, oli nagu täielik müstika. 

\question{Sest sa ei olnud ühtegi varem näinud!}

Ei olnud. Ja mitte ühelgi hetkel ei olnud seal sellist tunnet, et me oleme ettevõtjad, või et me oleme startup. Sa ikkagi teed asju, mida sa oskad ja mille jaoks on nagu mingi tõmme turult või tahetakse, et sa seda teeks. Homme küsitakse multimeedia CD-ROMi: \enquote{oh, teeme neid ka!}. 

\question{See multimeedia CD-ROM\sidenote{Tänapäeval küllalt raskesti seletatav asi. Kujutage ette, et teil on interaktiivne veebileht, sisaldab videosid, muusikat, teksti. Aga see ei ole kättesaadav mitte läbi brauseri ja interneti, vaid levib CD-plaadi kujul ning on realiseeritud täiesti teistsuguse tehnoloogia abil.} oli täiesti müstiline asi. Kas tõesti Ühispank\sidenote{Üks Halo suuremaid projekte oli Ühispanga\index{Ühispank} aastaraamatu välja andmine CD-l.} seda meie käest küsima? Andres Aarma\index[ppl]{Aarma, Andres}\sidenote{Andres tegeles tol ajal Ühispanga avalikkussuhetega.} samas võis tulla küll\ldots}

Jaj, võis olla küll. Kaks tükki me neid suuri tegime. Üks oli Ühispank, millele me tegime ja selle lainelt müüsime ära Eesti Telekomi\index{Eesti Telekom}. Idee oli selles, et \enquote{kuulge, varsti on aasta 2000, mis ta oma aastaraamatut paberile trükite}. Mäletan, et üks neist, ma ei mäleta kumb, maksis selle eest 200 000 Eesti krooni. Aga teisest küljest, täna mõtled, et 15 000 euro eest ei saa ühtegi progejat liigutama.

\question{Mis mind tagasi vaadates kõige rohkem hämmastab, ja ehk sind ka, oli see, et me ei teadnud midagi ei versioneerimisest ega testimisest ega üldse mingist süsteemsest tarkvara arendusest ja siiski suutsime kuidagi punuda softi, mis enam-vähem töötas. Ja meil oli häbematust see kliendile tarnida ja klient maksis arved ära!}

Seal oli üks asi, millele ma paar korda mõelnud. Üheksakümnendate interneti lainel kogu aeg räägiti uuest majandusest, eks, ja hästi palju oli neid kohti, kus arvati, et uus majandus ei allu vana majanduse reeglitele. Mõnes asjas ei allu ka, nagu turu suuruse mõõtmine või füüsiline kaugus. Aga mingites põhi asjades, nagu selles, et tulud võiksid ületada kulusid, allub. Samamoodi vastandati e-kommertsi ja päris kaubandust ja kõike nagu kuidagi käsitleti eraldi. Minu arust me tegime mitu aastat seda äri nagu selle koha pealt valesti, et me mõtlesime, et see veebiehitamine ei ole nagu tarkvaraarendus. Tegelikult meil ei olnud seal tiimis kedagi, kes oleks mõelnud veebisaidi ehitamisest nagu tarkvaraarendusprojektist või lugenud mõne raamatu selle koha pealt, et kuidas suuri projekte teha. Tundus, et see veebi või HTML-i värk on nii lihtne. Et kui sa kogemata ehitad sinna taha ka mingi sisuhaldussüsteemi, mida me üritasime tootestada, siis sellel hetkel oleks pidanud  üle minema. 

Ma tagantjärgi arvan, et see, mida Taavi Kotka\index[ppl]{Kotka, Taavi}  Webmedias\index{Webmedia}\sidenote{Praegu Nortal.}   tegi, et nad hakkaksid mõtlema oma tarkvara arenduse protsessi peale oli see, mis hoidis neid  elus. Nad alustasid meist  natuke hiljem, aga kiiremini hammustasid läbi, et kui nad lähevad suurt Maksuameti infosüsteemi tegema, siis ei peaks seda veebilehena käsitlema. 

\question{Jah, me ehitasime ikkagi veebilehti\ldots}

\ldots millel kogemata oli mingeid skripte taga\sidenote{Kuna andmebaas ja selle haldus tundus keeruline, siis hoiti kõiki maailma asju tekstifailides. Mida muudeti Perli abil.}. 

\question{Sa olid tol ajal Tiigrihüppega\index{Tiigrihüpe} ka kuidagi seotud, kuidas see oli?}

See oli ka üks kooliaegne asi. Jällegi seesama Tatu ja Füüsika Instituut\index{Tartu Ülikool!Füüsika Instituut}. Füüsikute seas oli ka Jaak Aaviksoo\index[ppl]{Aaviksoo, Jaak}, kellega mu isa käis kunagi samal ajal koos ülikoolis ja  kes oli ka Miina Härma\index{Koolid!Miina Härma Gümnaasium} või 2. Keskkooli vilistlane. Ja kui Tiigrihüpet tegema hakati, siis Toomas Hendrik Ilves\index[ppl]{Ilves, Toomas Hendrik} ja Jaak Aaviksoo\index[ppl]{Aaviksoo, Jaak} mõtlesid selle välja. Nagu nad ise räägivad, kolmekesi: Ilves, Aaviksoo ja Johnny Walker. Mõtlesid selle oma rollides tol hetkel välisministrina ja haridusministrina välja ja moodustasid selle ümber Tiigrihüppe peakomitee. 1996, kui ma olinkeskkooli viimases klassis, nad kutsusid mind sellesse komiteesse  õpilaste esindajaks. Mis  iseenesest oli žestina nagu  ilus, tehes  haridusprojekti  võiks mõni õpilane ka sellega seotud olla. Aga kujuta nüüd ette, et ma lähen Tartust bussiga Tallinnasse, Haridusministeeriumisse, kus toimub koosolek, kus on umbes Peeter Marvet\index[ppl]{Marvet, Peeter} ja Marju Lauristin\index[ppl]{Lauristin, Marju}, Ants Sild\index[ppl]{Sild, Ants} ja mingid sellised korüfeed laua ümber. Ma sisuliselt istusin seal komitees lihtsalt vait ja kuulasin, see tundus nagu täiesti müstiline. 

Ma arvan, et minu kõige Tiigrihüpesse antud panus, oli see, kui me käisime (ma arvan, et see oli mu esimene tele-eeter) otsesaates, kus toimus Tiigrihüppe teemaline väitlus, kus olid Marju Lauristin\index[ppl]{Lauristin, Marju} versus Lauri Leesi\index[ppl]{Leesi, Lauri} ja mina. Sa oled otse eetris, sa ei ole  kunagi televisioonis olnud, sa oled keskkoolis ja sinu vastas on kaks oma valdkonnas suhteliselt nagu tunnustatud inimest. Ja vaatad, kuidas Lauri Leesi jahub sulle midagi sellest, et arvutit pole kooli vaja ja et krihvel ja tahvel on aastasadu vastu pidanud ja siis sa saad aru, kui läinud on rong selle tüübi jaoks, sest sa ise juba tunnetada seda,  kuhu maailm liigub ja mis juhtub. See oli ka tegelikult see koht, kus ma tegelikult sattusin sinna, et ma ei läinud Ülikooli  arvutiteadusi õppima. Sest umbes sellest hetkest, ma ei suutnud seda võib-olla nii hästi sõnastada, aga mul tekkis tunne, et ma juba progen, juba pean BBS-i, juba hängin internetis ja et see tegelik lünk, on see, et miks need asjad toimivad. Võrgustike ja see sotsiaalne pool. Täpselt sel hetkel, sealtsamast saatest välja tulles, Lauristin küsis, et \enquote{kuule me teeme Tartusse uut eriala, kus me hakkame kommunikatsiooniteooriat ja asju õpetama, et tule sinna}. See oli ka see, kuidas ma sattus üldse  sotsiaalteadusi õppima. Mis, ütleme, paljude, ka minu enda, jaoks oli suhteliselt üllatav, et ma sihukest asja tegema sattusin.

\question{Üheksakümnendatel juhtus igasugu üllatavaid asju!}

Jah, võrreldes kõigi nende progejatega, kes olid teoloogid, läks mul isegi hästi.

\question{Millega sa praegu tegeled? Kuhu see tee sind tänaseks on toonud?}

Põhiliselt ehitan ettevõtteid. Pigem nende esimeses otsas või algses faasis. Et mõnes mõttes võibki öelda, et teen täpselt sama, mis me üheksakümnendatel tegime, aga lihtsalt iga kord teed seda võib-olla natuke uue kogemusega või natuke juba tead ka, mis sa räägid. 

Ma ettevõtteid siis sedapidi, et ma panen osade alla oma aega. Hakkasin viis aastat tagasi tegema sihukest ettevõtet nagu Teleport ja kaks aastat tagasi meid osteti ära, tegutsen edasi ettevõttes nimega Topia, kes  meid ostis. Põhimõtteliselt oli see asutajana see hetk, et kas ma tahan suurt tükki väikesest pirukast või väikest tüki suurest pirukast ja Teleport-i müük Topia-le andis võimaluse paar aastat nagu vahele jätta ja hüpata trepil kõrgemale. 12 inimesega startupist sai 170 inimesega tiim ja mul on kuskil 70 inimest tootearenduse valdkonnas, kellega saab sama visiooni kiiremini ehitada. Ja kui mul aega üle on, siis ma ka investeerin start-upidesse ja annan mõne nõu. 

Sarnasus üheksakümnendatega on see, et ma tean täpselt, mida ma suudan üksi ehitada, on see siis BBS-i soft või mitte, aga ma tean, mis väärtus ja mis võlu on sellel, kui progeja, disainer ja äriinimene koos mingit asja teevad ja mida siis saab valmis teha. Alates siis mängust Drunkard või  mingitest veebiprojektidest Halos. Ja ma tean, et ma ei taha kunagi elus müüa oma aega tunni põhiselt, sest tunde on lõplik hulk. Järelikult tuleb ehitada tooteid. See läheb juba meie ajalisest skoobist välja, aga see on see hiljem Skype'is\index{Skype} nähtud asi, paar inimest ehitavad mõne kuu mingit asja, mida kuu aega hiljem kasutab miljon inimest. Et see asi, mida  see 90.-te  rahmeldamine võib-olla õpetas, ja asi, mida nüüd ma alati otsin, on koht, kus sisse pandud töötundide hulk konverteeruks võimalikult suureks väärtuseks. Tol hetkel kippus ikka väga palju olema, et tegutsesid  puhtalt õhinapõhiselt, oleks kõiki neid samu asju teinud ka siis, kui ei oleks paka saanud. Kui sa müüd oma oma töötunde, siis sul on lihtsalt väga pikad päevad, väga lühikesed ööd ja\ldots Eks ole neid tüüpe, kes seal üheksakümnendatel nagu väga läbi ka põlesid, juba kohe. Osad põlseid hiljem. 

Kes olid seal legendaarsed mehed, nagu näiteks (väga austan sellest ajast)  Taavi Talvik\index[ppl]{Talvik, Taavi} näiteks. Kui sa  sama tausta ja huvide pealt ehitad Unineti\index{Uninet}, kus sa võid ka magada, aga bitid müüvad ennast ise. Eee oli küll  teenuse infrastruktuuri äri, aga seal oli see alge olemas, kuidas ehitada midagi, mis saab hakkama ka siis, kui sina ei olekogu aeg näppupidi juures.