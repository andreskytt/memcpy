\index[ppl]{Ruukel, Henn}


\question{Kuidas arvutid 
sinu juurde jõudsid? }

Kooli arvutiklassi kaudu.

\question{Mis kool see oli?}

Tallinna 10. Keskkool\index{Tallinna 10. keskkool}, tänane Nõmme 
Gümnaasium\index{Nõmme Gümnaasium|see{Tallinna 10. keskkool}}. Kui ma 
õigesti mäletan, siis pärast üht suvevaheaega tulime kooli ja 
mataklassi oli ehitatud arvutiklass. Laudadel sai plaadi üles tõsta ja seest tulid 
välja pööratava metallkonstruktsiooniga 
Elektronika\index{Elektronika} arvutid. Need olid omavahel võrgus 
niimoodi, et õpetaja laua peal oli flopidraiviga 
Iskra\index{Iskra}. 

Elektronikates jooksis ainult BASIC. 
Lükkasid käima, BASIC\index{BASIC} hakkas kohe jooksma, said koodi 
kirjutada ja olid BASICu käsud, millega sai oma programmi Iskra flopikettale salvestada. Viisime oma flopiketta õpetaja kätte, ta pani selle masinasse, tegi ilmselt midagi Iskras ka ja 
siis saigi salvestada oma proge maha või pärast järgmine päev laadida. Ja nii 
kui see juhtus, siis me hakkasime muidugi kohe mataõpetajat manguma. Õpetaja 
Ramil Izamentinov\index[ppl]{Izamentinov, Ramil} oli äge 
mees -- ma \emph{never} oma elus pole näinud teist sellist inimest (ta oli astmahaige), 
kes nii ägedalt naeraks oma naljade üle. Kui 
ta tegi nalja, siis spetsiifilise kähinaga, et kõik saaksid aru, et see oli nali. 

Ta lasi meil pärast tunde seal klassis olla ja me 
hakkasime kirjutama suvalisi progesid. 

\question{Mitmendas klassis see oli?}

See pidi olema enne keskkooli, võibolla seitsmendas 
klassis ehk 1986. aastal. Käisin kooli arvutiklassis 
umbes ühe õppeaasta. Põhiliselt sai kirjutatud progesid, 
mänge ei olnud seal üldse, nii et kõik mängud, mida 
tahtsime mängida, tuli ise BASICus teha, kirjutada flopile ja sealt pärast laadida. Tegime
lihtsaid mänge, näiteks trips-traps-trulli. Elektronikal oli graafiline liides, millega
sai ekraanile joonistada. Näiteks üks programm oli 
selline, et andsid ruutvõrrandi parameetrid, programm arvutas lahendid välja ja joonistas 
graafiku. 

\question{Miks te pidite manguma, et arvutitele ligi saada?}

See tundus kohe äge asi. Küsisime, kas 
võib käia ja mis õhtutel. Algul oli vist õpetaja ise kohal, 
aga lõpuks oli meil mataklassi võti. Paar huvilist 
oli veel minu klassist ja õppealajuhatajaga tuli teha eraldi \emph{arrangement}, et saada algklasside pikapäevarühma söögi 
peale. Siis me ei pidanud koolist ära minema, kui tunnid lõppesid, vaid saime koos
väikeste lastega sööklas süüa ja olla õhtuni arvutiklassis. 

Arvuti oli ikka nii põnev asi, siis oli ju hoopis teine aeg -- kodus polnud 
videomakkigi, arvutist rääkimata. Või et kellegi töö juures oleks arvuti -- 
sellist asja polnud. 

Seejärel kuulsin mingil peresünnipäeval
ühelt kaugelt sugulaselt, et tema käib TPIs 
arvutiringis\index{TPI arvutiring}. See oli see kuulus 
arvutiring, mida pidasid Julius\index[ppl]{Raimla, Tõnu}\index[ppl]{Julius|see{Railma, Tõnu}}\sidenote{Henn peab ilmselt silmas Tõnu Raimlat toonase hüüdnimega Julius.} ja Aare Tali\index[ppl]{Tali, Aare}. 

See oli selline \emph{need to know 
basis} arvutiring -- pidid teadma, et lähed teisipäeval raadiotehnika 
kateedri\index{Tallinna Tehnikaülikool!Raadiotehnika kateeder} taha otsa ja 
ootad ühe ukse taga. Seal oli jõnglasi nagu murdu ja 
kindlal kellaajal tegi Julius ukse lahti. Klassis olid kolmes reas
Yamahad\index{Yamaha MSX}, MicroBeed\index{MicroBee} ja 
Robotron 1715d\index{Robotron!Robotron 1715}. Yamahade peale käis 
tegelikult tormijooks, kõik tahtsid sinna saada, sest seal olid 
kihvtid mängud. Robotronid olid ka põnevad, samas kui MicroBeed ei huvitanud eriti kedagi. Mina olin põhiliselt Robotronide peal, kus sai 
progeda ja koolireferaate teha. Yamahadesse ei saanud ma
kunagi löögile. 

Yamahad olid 3.5tolliste diskettidega ehk juba järgmine tase. See pidi olema 1988. aasta.

\question{Kas oligi nii, et uks tehti lahti ja kes istuma sai, see sai?} 

Jaa. Kui midagi ära lõhkusid, siis oli selleks päevaks \emph{ban}, aga järgmisel 
päeval võisid tagasi tulla. Mäletan seda sellepärast, et ükskord istusin 
ühe Robotroni taha, kus oli probleem klaviatuuriga, ja ma võtsin teise 
masina klaveri. Sellel oli imelik pistik ja panin selle kuidagi nii kehvasti sisse, et rikkusin pistiku kontaktid kõveraks. Siis tuli minna tagaruumi 
ja öelda, et näed, niisugune jama. Ma ei mäleta, kumb neist tuli, kas Julius 
või Tali, vaatas ja ütles: \enquote{Okei, tänaseks kõik, tule järgmine 
teisipäev tagasi.} 

Tegelikult olid nad kateedris tööl ja ilmselt progesid tagaruumis. Meie 
olime kõik alles 13aastased jõnglased. 
See pidi olema 1988. aasta, sest 1989. aastal läks mu ema tööle 
Diagnostikakeskusse\index{Diagnostikakeskus} ja 
ma liikusin sinna edasi. 

Ürituse lõpp sõltus sellest, millal nemad ei viitsinud enam olla ja tahtsid 
koju minna. Ametlik lõpp oli ka, aga sellest ei pidanud keegi 
kinni. Jõnglased said teisipäeva õhtul umbes kell kuus uksest sisse ja kõigil oli jube 
põnev: kes proges, kes mängis mänge ja kes häkkis mänge, et teha 
sinna oma tegelased sisse. Arvutiring lihtsalt oli, meid ei pandud 
kunagi kuskil kirja ja keegi ei õpetanud midagi, ise tegime. Ühel hetkel 
viskas kuttidel üle, nad tegid tagaruumi ukse lahti ja ütlesid: 
\enquote{Viie mintsa pärast toitekas.} See tähendas, et meil oli viis minutit aega mis 
iganes pooleli oli ära salvestada, sest viie minuti pärast tuli emb-kumb neist, ei hakanud jõnglastega midagi vaidlema, jalutas kilbi juurde ja 
tõmbas laksti peakilbi välja. Klass läks pimedaks ja see oli signaal, et nüüd tuleb koju 
minna.

\question{Võru 1. Keskkoolis oli sama skeem ja kedagi ei huvitanud, mis 
arvutiga juhutub. Minu meelest ei juhtunudki suurt midagi.}

Täiesti \emph{ruthless}. Väga ei juhtunud jah, aga eks ilmselt oli oht, et kui 
kirjutamine jäi pooleli, siis võis kettaga midagi juhtuda. 

See oli tegelikult äge periood. Mäletan uduselt paari tüüpi, sest
koostööd või suhtlust oli vähe. Yamahade juurest oli justkui mingi 
generatsioon, kes tundsid üksteist varasemast ja hoidsid 
kokku, aga ülejäänud nokitsesid igaüks omasoodu. 

See kõik oli ikkagi alles algus. Niisugune \emph{breakthrough}, kus 
mina jõudsin paradiisi ja õndsusesse, toimus tänu Nõukogude Liidu kompartei esimehele Mihhail Gorbatšovile, kes otsustas 
perestroika käigus aastal 1986, et meditsiiniga on Nõukogude 
Liidus suur probleem. Ja kõige suurem probleem 
on see, et pannakse valesid diagnoose või diagnoosi jaoks vajalik info on 
ebatäpne, vale või liiga aeglane. Seega tuleb investeerida infrastruktuuri, 
millega meditsiinipersonal saaks kiiremini teha ära olulised mõõtmised ja 
analüüsid, näiteks südame EKG, vereanalüüsi, 
magnetresonantstomograafia. Asjad, mis on täna põhimõtteliselt 
külahaiglaski olemas. Siis selliseid asju polnud ja vereanalüüsi tegemine võttis mega 
kaua aega ning tulemused tulid tagasi ebatäpsed. 

Gorbatšov otsustas, et kuna tal ei ole piisavalt valuutafondi\sidenote{Nõukogude 
elu valitsesid laias laastus kaks terminit: fond ja limiit. Fond ütles, kui 
palju midagi kellegi jaoks olemas oli, ja limiit seda, kui palju midagi tohtis
toota või tarbida. Näiteks \enquote{pole piisavalt valuutafondi} tähendas, et 
ei olnud eraldatud piisavalt valuutat.}, et teha neid 
võimalusi igasse jumala haiglasse, siis igasse Nõukogude Liidu osariiki, 
näiteks Eestisse, pidi pealinna moodustatama selline asi nagu 
diagnostikakeskus, kuhu investeeritakse valuutafondi, ostetakse välismaalt 
\emph{top notch} tehnika sisse ja terve osariigi analüüsid või uuringud 
tehakse ühes kohas. Näiteks kui sul on kopsuhaigus, siis sind ravib küll
Mustamäe haigla arst, aga Mustamäe haiglasse ei ole ressurssi 
osta kopsuanalüsaatoreid, vaid need ostetakse ühte kohta 
eraldi asutusse.

Eestisse pidi ka tulema selline keskus. Tol ajal kehtis Nõukogude Liidule välismajandusembargo ehk 
liitu ei saanud sisse tuua lääneriikide moodsat tehnoloogiat, välja arvatud
meditsiinitehnoloogiat, see oli okei. PCsid ei olnud meil ka selle pärast, et embargo oli peal. Isegi kui välkarid oleksid tahtnud müüa ja
kellelgi oleks siin valuutat olnud, ei saanud selliseid asju osta. 

Hakatigi siis Tallinnasse moodustama 
Diagnostikakeskust\index{Diagnostikakeskus}. Seda vedas Ago Kivilo\index[ppl]{Kivilo, Ago}, kes on tänaseks siit ilmast kadunud. Keskus pidi tulema 
sinna, kus praegu on vana Tallinna Panga maja Tallinna linnaosavalitsuse 
kõrval. Muidugi läks eht eestlaslikult kemplemiseks selle ümber, kes 
saab keskuse juhiks, millise haigla kõrvale see üldse teha, asukoht ei ole 
ikka hea, haiglatest kaugel ja äkki teha Mustamäe haigla kõrvale. 
Kivilo ütles, et teeb selle ise, eraldi asutusena. Mustamäe haigla 
ilmselt tahtis seda enda allasutuseks. Ma detaile ei tea, 
olin umbes 14, aga natuke tean sellepärast, et 
aitasin Agu Kivilol teha kõiki PowerPointe, mida tal oli vaja. Tol ajal oli
see \emph{skill}. 

\question{Kas PowerPoint oli tollal olemas?} 

Eesti riigis polnud selliseid 
asju nagu laserprinterid, PCd, värvilised kuvarid ja arvutivõrgud siis veel olemas. Okei, arvutivõrke oli, aga need olid pigem 
CP/Mide arvutivõrgud tehnikaülikoolis. Aga kuna 
Diagnostikakeskus oli meditsiiniasutus ja nad ostsid Soomest 
meditsiinitehnikat, siis oli Soomes üks vahendusfirma
Pekka OY (see oli reaalselt omaniku eesnimi), kes vahendas 
Soomest kopsu- ja vereanalüsaatoreid ning muud kama. Küllap juriidiline skeem pidi ka suhteliselt keeruline olema, kuna see kõik liikus 
Moskva valuutafondide kaudu. Kuna see kaup liikus, 
siis selle raames \emph{top notch} PCsid Eestisse tuua polnud mingi probleem, 
sest paberil paistis see kõik nagu meditsiinitehnika. Tänu sellele oligi meil PowerPoint olemas.

\question{See oli vist võrreldes muu tehnikaga ka odav?}

Esimene magnetresonantstomograaf tuli sealtkaudu. Omaette lugu oli see, et Gorbatšov eraldas valuuta, aga seda ei suudetud ära kulutada, sest lõputu aeg 
kulus vaidlusele, kuhu keskus ehitada. Tomograafi jaoks pidi maja vundament olema spetsiifiline, et ei tekiks
värinaid ja vibratsioone. Aeg muudkui kulus ja Nõukogude Liit hakkas juba lagunema, 
aga valuuta oli ikka veel kontol. Tekkis reaalne probleem, kuidas see 
kuhugi ära kulutada, enne kui kaduma läheb. 
Kivilo\index[ppl]{Kivilo, Ago} tegeles sellega. Kokkuvõttes ehitati keskus Magdaleena 
haigla kõrvale. Kivilo sai tomograafi niimoodi kätte, et 
tarnijafirma või mõni vahefirma (detaile ma ei tea) sai 
ettemaksu ja aastaid hiljem tarnis seadme kohale. 

Kõige selle käigus tuli ka arvuteid. Mina 
sattusin sinna niimoodi, et ema läks vereanalüsaatori peale tööle (ta on apteeker) ja küsis Kivilolt, kas tema poeg võib seal pärast kooli
arvutis käia. Kivilo lubaski. Seal sain 
kokku Mart Palmasega\index[ppl]{Palmas, Mart}, kes 
põhimõtteliselt tõigi mind ja Madis Kaalu\index[ppl]{Kaal, 
Madis} Eesti ITsse. Madis Kaalu püüdis ta kuskilt TPI pealt kinni (ta oli sealt 
välja kukkumas), ja ütles, et kuule, tule nüüd, ma panen su arvutite 
peale. Nii et Palmas õpetas mind progema, enne olin omal käel 
harjutanud. 

\question{Mida Mart Diagnostikakeskuses tegi?}

Diagnostikakeskuse arvutite hooldamiseks oli loodud väikeettevõte 
Skriining\index{Skriining}, mille asutas Kalle 
Lotamõis\index[ppl]{Lotamõis, Kalle} ja mis on siiamaani olemas. Mart töötas seal vist programmeerijana.
Skriining hoidis Diagnostika{\-}keskuse arvuteid korras ja mina sain seal ettevõttes
hängida, ma ei olnud nendega kuidagi juriidiliselt seotud. 
Välja arvatud minu üldse esimene töö, kui TPLsid\sidenote{Töö- ja 
puhkelaager -- Nõukogude koolilastele pakutud võimalus suviti organiseeritult 
tööd teha ja elu nautida.} ja rohimist mitte arvestada. Vedasin neile maja peal laiali koaksiaalvõrgu, mille otsa panime käima 
Novell \mbox{Networki}. Arvutitesse tuli paigaldada võrgukaardid, neid häälestada, panna 
õiged IRQd ja ka soft peale. See oli mu 
esimene suvetöö.

\question{Kes sulle sellise ülesande andis ja miks tal oli alust arvata, et sa oskad
seda teha?}

Esiteks olin seal juba kevad läbi hänginud. See oli 
Kalle\index[ppl]{Lotamõis, Kalle}, Skriiningu juhi otsus. 
Küsisin, kas neil oleks mingit suvetööd. Kalle ütles: \enquote{No tule ja pane 
arvuteid kokku -- võta kastist välja, pane üles ja kui kellelgi on 
mõni mure, siis aita.} Me olime Suur-Ameerika 18 majas. 

Mainisin, et meil olid PowerPointid ja laserprinterid. Samal ajal oli terves TPis heal 
juhul kümme 8086 ehk XTd ja ilmselt mustvalge 
\emph{display}'ga. Meil oli terve ruum triiki täis 
avamata arvuteid, mis olid kõik 80286d, kõigil 40 mega IDE vinti ja VGA 
graafika. 

Kui näiteks Mamers\index[ppl]{Mamers, Tarmo} meile tööle tuli, 
siis ta pakkis endale kohe kaks tükki lahti! Ühe peal jooksis BBS, teise 
peal tegi tööd. Kui panna 
tagantjärele konteksti, mida tänapäeval tööks nimetatakse, siis seal tööd 
tegelikult ei olnud. Meditsiinipersonali arvuteid oli 
umbes kümme ja neid hooldas kümme inimest, kes põhiliselt 
lihtsalt kaifisid seda, mis tehnika keskele nad sattusid. Tegime 
muidugi kõik asjad ära, mis teha oli vaja. 

Mina sattusin sinna minnes niivõrd viljastavasse keskkonda. Seal 
olid Mamers\index[ppl]{Mamers, Tarmo} ja Palmas\index[ppl]{Palmas, Mart}, 
Hannu Krosing\index[ppl]{Krosing, Hannu} astus kord nädalas läbi. See oli ka 
koht, kus sain aru, et minust ei saa kunagi progejat. Mäletan nii 
hästi, kuidas ma nädal aega pusisin millegi kallal, ja siis tuli Hannu, näitasin 
talle selle nädalaga kirjutatud koodi ning ta võttis paberilehe ja 
kirjutas kahe reaga sellesama koodi Turbo Cs\index{Turbo C}.

\question{Leidsid ka, kellega ennast võrrelda!}

Ma sain aru, milline on delta. Võibolla see ei olnud ainus põhjus, aga 
ma mõistsin, et talendi ja \emph{skill}'i vahe on ikka 
hüpersuur.

\question{Hannuga võrreldes on kelle iganes talendivahe väga suur!} 

Tõsi-tõsi.

Põhiline projekt, mille kallal ma tol ajal töötasin ja millel oli ka üks 
\emph{user}, oli clabel. Ostsin sellele
ametlikult 25 krooniga Madis Kaalult\index[ppl]{Kaal, Madis} graafilise 
liidese \emph{library}, millega sai menüüd, \emph{pop-up}'e ja muud sellist
teha. Nii et juba toona 
maksin selle eest, et mul oleks ametlik \emph{lifetime}-litsents. 

Niisiis, kirjutasin programmi clabel, mis tegi sellist 
asja, et sul oli muusika (meil olid tol ajal kõigil kopeeritud muusika 
kassetid\sidenote{Vt ka märkus lk \pageref{sisu!kassetid}.}), kassette oli palju ja oli vaja normaalset andmebaasi,
mis lood ja bändid millise kasseti peal on. Lisaks oli vaja need 
välja trükkida, et ilusti ümber kasseti voltida 
ja panna karbikaane alla. Programm trükis sellise paberi välja, et üleval serval 
oli näha, mis on A- ja mis B-poolel, ning suurel küljel kõik
A- ja B-poole laulud. Programmil oli graafiline liides, kus 
sai brausida, edida ja printida. 

\question{Ma teenisin oma esimese arvutiga teenitud raha täpselt samasuguse softi 
abil.}

Mul oli üks väga kasulik \emph{user}, Toivo Annus\index[ppl]{Annus, 
Toivo}, kes reaalselt kasutas seda ja tegi mulle 
kogu aeg bugireporte.

\question{Kuidas sa Toivoga kokku said?}

Fido kaudu. Ilmselt \emph{upload}'isin selle softi kuskile BBSi ja promosin Fidos, et mul on niisugune asi. Toivo hakkas kasutama ja mul külas 
käima ning rääkima, mis ei tööta ja mis töötab. Tema oli siis vist 16, mina 
umbes 14. 

\question{Järelikult sukeldusid sa juba Diagnostikakeskuses Fido maailma?}

Ma ei mäleta, millal Fido tekkis, või kumb oli enne, kas minu Diagnostikakeskusse minek või Fido tulek. Ilmselt 
läksin kõigepealt keskusesse ja siis millalgi tekkis Fido. Aga ühel hetkel olid Fido ja BBSid meil seal 
nagu \emph{bread and butter}. Mardil\index[ppl]{Palmas, Mart} oli oma 
\emph{node}, Mamersil\index[ppl]{Mamers, Tarmo} muidugi oma Mambox\index{MamBox}. 
Siis tuli BBSummer\index{BBSummer}, esimene toimus vist esimesel või 
järgmisel suvel. 

Lisaks kõigele muule oli Diagnostikakeskusel humanitaarabina Rootsist saadud täiesti töökorras Volvo 
põhjale ehitatud kiirabiauto. Kuna keskusel polnud sellega 
midagi teha, sõitis sellega ringi Mart. Autol olid kõik 
operatiivauto load olemas, vilkurid peal, täismäng. Taga oli kanderaam, 
pane või inimene sisse. Käisime sellega koos
Mamersi\index[ppl]{Mamers, Tarmo} ja vist Kaido Kärneriga\index[ppl]{Kärner, 
Kaido} Saku Õlletehasest BBSummeri jaoks õlut toomas. Sellega oli hea vedada. Esimene summer toimus Väänas Tugamanni veskis, mina läksin 
mopeediga kohale. See oli juba ülemineku ajal, 
kui poes polnud midagi saada. Kellegi tutvuste kaudu saadi kuidagi
Sakuga kokkuleppele sealt otse õlut osta. 

\question{Kas see oli seesama kord, kui, nagu Mast\index[ppl]{Kaal, Madis} rääkis, 
õlut sai villitud Fanta tünnidesse ja õllel oli apelsinimekk man?}

Täitsa võimalik. Mina olin selles vanuses, et õlut ei joonud, ja ei oska kommenteerida. Aga mäletan, et üritus oli kihvt.
Esimesel BBSummeril kuulsin esimest korda ka
asjast nimega internet ja asjast nimega e-maili aadress. Tartu füüsika 
instituudis\index{Tartu Ülikool!Füüsika Instituut} oli mingi kamp, lausa 
perekond itikaid, kelle nime ma ei mäleta.\sidenote{Tarmo Mamers\index[ppl]{Mamers, Tarmo} arvab, et tõenäoliselt oli tegu perekond Pruulmannidega ehk muudestki juttudest läbi käiva Jaan Pruulmanni\index[ppl]{Pruulmann, Jaan} ja tema abikaasaga.} Igatahes keegi 
nendest pidas ettekande ja oli tolleks ajaks juba käima pannud interneti ja 
FidoNeti vahelise \emph{gateway} meili jaoks. Nii et põhimõtteliselt kõigil 
Fido inimestel, ja kohe ka mul, sel ajal 
meiliaadress. Ma ei mäleta, mis tagumine ots oli, st domeen, aga 
ülejäänud ots moodustus sellest, mis \emph{node}'i küljes sa olid ja kes sa olid. Nii et teoreetiliselt sai mulle saata meili ja mina sain saata välja ka, 
aga ma ei mäleta, et oleksin seda praktikas kasutanud. Mul oli 
too meiliaadress isegi kuskile märkmikusse üles kirjutatud, aga sellega 
polnud kellelegi saata. 

\question{Esimese faksiomaniku probleem\ldots{ }Järelikult oli BBSummeril ka
hariduslik ja sisuline sisu?} 

Jah, loenguid ja \emph{knowledge-sharing}'ut oli väga palju. Minu, 
14aastase arust oli see väga äge. Mäletan Martiini\index[ppl]{Martiini|see{Rinne, 
Martin}} ehk Martin Rinne\index[ppl]{Rinne, Martin} demost, mida sai teha ja 
mida nad tegid Amigaga. Ta tegi Eesti Tele{\-}visioonis\index{Eesti Rahvusringhääling!Eesti Televisioon} saatetiitreid ja muid asju ning demos seda poolt. Minu jaoks oli see täiesti uus maailm, ma polnud 
seda osa üldse näinud. Mäletan internetiteemalist loengut, aga 
eks seal oli ka väga palju vaba suhtlemist ja jutuajamist. Nägin seal
esimest korda paljusid inimesi, kellega olin juba umbes aasta suhelnud. 

\question{Kas sõnumitega Fidos?}

Jah, olid jututoad teemade kaupa, põhimõtteliselt nagu tänapäeval foorumid. 
See ei olnud kaugeltki \emph{real-time} -- värsked sõnumid tuli alla tõmmata, läbi lugeda, vastused valmis kirjutada ja siis sisse helistada ja valmiskirjutatud asjad \emph{upload}'ida. 

\question{Järelikult pidi sul olema mingi kliendisoft?}

Jah, sellega oli lihtne. Kõige keerulisem oli see, et pidid olema modem ja 
telefoniliin. Neid oli tol ajal raske hankida. 
Niipea kui saime Novelli võrgu püsti, siis meil maja sees liikusid 
sõnumid sealtkaudu. Selleks et mina lugeda ja kirjutada saaksin, ei pidanud ma 
oma masinas modemit omama, sõnumid läksid otse MamBoxi. Täpseid vaheetappe ma ei mäleta, pigem seda, et kui liikusin 
sealt Salva Kindlustusse, kus me Toivoga moodustasime Salva Kindlustuse IT-osakonna, 
siis oli meil lihtsalt ühes masinas \emph{point}. 

\question{Miks te seda kõike tegite? Diagnostikakeskuse jaoks ei olnud seda ju 
vaja. Kas teil oli aega ja tahtmist mängida või andis mõni visiooniga 
inimene teile ülesande võrk ehitada?}

Novelli võrk oli väga praktiline asi, tänapäeval on sama praktiline 
teha igasse kontorisse internetiühendus. See andis ettevõttele väärtuse 
mõttes kaks asja: võrguketta (mina salvestan maha ja sina saad teisest 
arvutist kohe kätte) ja võrguprinteri. See oli ju \emph{pre-Windows} aeg: 
said omale võrguketta külge, said faile jagada ja programmidele ligi ning 
printida. Laserprinterid olid kallid ja see andis hästi suure efekti, kui
said maja peale ühe laseri osta ja ükskõik mis arvutist sinna trükkida -- see oli 
tol ajal \emph{magic}. Kõiki asju tuleb ju konteksti panna ja see oli kindlasti tolle aja kohta eriline. 

\question{Kas Diagnostikakeskuses olid kuni keskkooli lõpuni?}

Diagnostikakeskus\index{Diagnostikakeskus} oli 
ikkagi meditsiiniettevõte, Skriining\index{Skriining} oli selle küljes 
tütarfirma. Algul pidigi väikeettevõtted moodustama mõne
riikliku ettevõtte juurde. Ma ei tea, mis aastal see täpselt oli, aga 
millalgi sai Skriining ennast Diagnostikakeskuse küljest lahti aktsiaseltsiks. 

Kõigepealt hakkasin Skriiningus suvetööl käima, tegin võrguhaldust. Üsna pea 
õnnestus mul ennast sebida kooli kõrvalt palgale, nii et kogu keskkooliaja 
käisin sellisel režiimil, et pärast kooli läksin kohe linna 
Skriiningusse. Siis oli meil juba rohkem kliente: mitte ainult 
Diagnostikakeskus, vaid erinevaid meditsiiniettevõtteid. Põhiäri oli arvutivõrk, kas mõnes haiglas või mujal. Näiteks vedasin
Haigekassasse\index{Haigekassa} arvutivõrgu. See asus praeguse Prantsuse Lütseumi 
algkooli hoones.\sidenote{Hariduse 8, Tallinn.} 

Mäletan seda sellepärast hästi, et 
esiteks ei laseks tänapäeval keegi kaablit lihtsalt pinna 
peale vedada. Tõmbasin kaablit ja lõin klambreid seina peale, mis praegu
tunduks robustne. Teiseks, töövahenditeks olid haamer, klambrid, 
kaabel ja midagi, mis meenutab kaugelt vaadates trelli. Tänapäeval ei kavatseks ilmselt keegi sellega ühtegi auku puurida. Mina pidin sellega kõik augud 
tegema. Trelli otsa käis üks asi, mis meenutas puuri, ja sellega tuli 
suvalisest materjalist nühkimismeetodil läbi minna.

\question{Kas puuri diameeter oli suurem kui kaabli diameeter?}

Jah, vähemalt see oli hea. Põhimõtteliselt ainsad lootustandvad kohad, kust 
õnnestus läbi minna, kuna puuri pikkus ei olnud väga pikk, olid 
uksepiidad. Koaksiaalkaabel peab ju läbi kogu maja 
moodustama pideva \emph{loop}'i. See saab kuskil alata ja 
kuskil lõppeda, aga ei tohi katkeda ja seda ei saa olla mitu. Nii et
pidin mõtlema, et arvutivõrku on vaja kõigisse tubadesse ja
kuidas see ahel teha. Pidin terve toa läbi jalutama ja uuesti kaabliga välja 
minema. Ja kui see kuskil katki läks, oli kogu võrk maas, 
sest see ei olnud selline nagu \emph{twisted pair}'i võrk, kus ruuterist 
või \emph{switch}'ist läheb kaabel seadmeni ja kogu moos. 

Ühesõnaga, Skriiningus käisin ma palgatööl. Ühel hetkel kolisime Suur-Ameerikast 
ära, Skriining sai oma ruumid ja olime pikalt, vist 
kogu mu keskkooliaja, Estonia puiesteel, kus 
praegu on Mati Mobiiliäri.\sidenote{Estonia puiestee 5, Tallinn.} See oli äge aeg. 

Aaslaid\sidenote{Andrus Aaslaid\index[ppl]{Aaslaid, Andrus}.} elas kontoris. Tal oli kuskil korter ka, aga 
ta ei viitsinud seal käia mis iganes põhjusel. Mina tiksusin ka pärast kooli viimase 
bussini seal kas oma programme kirjutades või tööd 
tehes: vedasin võrku laiali, 
hooldasin võrke, paigaldasin servereid. Need muidugi käisid vahepeal maha ja 
tuli joosta kuskile teise linna otsa, et asjad uuesti käima ajada. 

Kõva \emph{bread and butter} Skriiningus oli see, et arvuteid pandi 
komponentidest kokku. Klient ütles, et andke mulle üks arvuti. Lepiti 
mingis enam-vähem konfis kokku, aga ma ei ole kindel, kas
tellija teadis, mis need parameetrid on. Linna pealt otsiti komponendid 
partnerarvutifirmadelt kokku: ühelt mäluplaat, teiselt kõvaketas, 
kolmandalt korpus. Mina keerasin selle kõik kokku, panin ööseks 
testid peale jooksma ja hommikul, kui kõik toimis, läks arvuti karpi, Skriiningu 
kleepekas peale ja kliendi juurde. Siis panin arvuti üles, näitasin inimesele (kelle 
jaoks oli see enamasti elu esimene arvuti), kust sisse 
lülitatakse ja et turbonuppu\sidenote{Vanematel PC tüüpi arvutitel oli küljes nupp 
sildiga \enquote{turbo}. Vastupidiselt ootusele sellele vajutamine vähendas, 
mitte ei suurendanud arvuti töökiirust. Nimelt sõltusid 8088 
protsessoriga arvuti jaoks loodud tarkvara (eriti mängud) mõnikord 
masina 4,77 Mhz taktsagedusest ja seetõttu uuematel, kiirematel 
masinatel korralikult ei käinud. Tagurpidi ühilduvuse jaoks lisati 
riistvaraline võimalus arvutit aeglasemaks teha.} ära vajuta, las see olla kogu aeg sees. Näitasin ka, mida arvutiga teha saab ja kuidas 
võrku logida (põhiliselt oli Novelli võrk). Enamasti oli 
kasutajal vaja saada ligi mõnele raamatupidamisprogrammile, mida ka 
Skriining ise kirjutas, või siis oli tegu Skriiningu enda \emph{custom} 
haiglate infosüsteemidega, näiteks kaardiregistritega.

Sedasi läks mu keskkooliaeg. 

\question{Kas see õppimist ei hakanud segama?} 

Ei hakanud. Pigem ei seganud õppimine tööd. Ega ma viieline ei 
olnud, tol ajal oli mu elu ikkagi väga IT poole kaldu. 
Kool ei huvitanud mind absoluutselt. Mitte et ma poleks
aru saanud, aga ma tegin \emph{bare minimum}'i, et saaks kähku arvuti taha. Tundus, et 
kõik põnev asi toimub seal. 

Ja mitte ainult arvuti taha, vaid, kuna me olime nii tsentraalses asukohas, 
Skriiningu kontor oli nagu läbikäiguhoov. Kogu aeg astus keegi uksest 
sisse, oli see siis Tanel Raja\index[ppl]{Raja, Tanel}, Hannu 
Krosing\index[ppl]{Krosing, Hannu} või keegi teine. Ajas juttu ja sai jälle 
midagi teada, mida tema oli kuskil näinud või kuulnud. Mast\index[ppl]{Kaal, 
Madis} oli Forekspangas kohe üle hoovi, tema käis külas. Nii et elu oli väga 
sotsiaalne. Tänapäeval on kõik \emph{online}'is, tol 
ajal oli suhtlus ka IT-meeste vahel üsna \emph{offline}. 

\question{Aga BBSid?}

BBSid olid olemas, aga minu arust sel ajal hakkas Fido vunk juba hääbuma. 
Sellel oli ilmselt mitu põhjust. Üks oli see, et kapitalism jõudis kohale, 
tööd oli vaja teha. Ei saanud päevad läbi lihtsalt istuda ja häkkida 
arvuteid, et kui palju ma suudan mälu siin efektiivsemaks panna 
või kui ilusti ma oskan oma faile pakkida kettale niimoodi, et neile
võimalikult kiiresti ligi pääseda. Selliste asjadega ei olnud ühel 
hetkel enam aega tegeleda, vaid tuli teha reaalset tööd, kas siis tellimustöid
progeda või arvuteid kokku panna. 
Elu läks tõsisemaks ja suhtlemist vähemaks. 

\question{Kas füüsiliselt ikka üksteisel külas käisite?}

Käisime, aga ka see hakkas ühel hetkel hääbuma. Skriiningu\index{Skriining} ajal see veel oli nii, aga kui me 
Toivoga\index[ppl]{Annus, Toivo} Salva Kindlustuse ITd tegime, siis ma ei 
mäleta, et seal oleks hängimist või kohvitamist olnud või et keegi 
oleks külla tulnud ja oleksime pikalt pläkutanud. 

\question{Kui minna korraks BBSide ja Fido juurde tagasi, siis kas sa oskad tuua mõnda näidet, mis kohtades ja mis teemal tubades sa juttu rääkisid?}

Minu mäletamist mööda olid olemas näiteks sellised ruumid 
nagu EW.NALJAD, kus keegi jagas anekdoote, ja EW.JUTUTUBA, kus olid üldised teemad. Väga kõva 
diskussioon käis igasugustes tehnilistes 
\emph{channel}'ites. Ma kahjuks enam ei mäleta, mis teemade kaupa need olid. 
Kusjuures seal ei olnud ainult Eesti kanalid, vaid sai 
\emph{subscribe}'ida ka globaalsetesse kanalitesse. Näiteks mingil hetkel lugesin globaalseid Novelli halduse ja adminnimise kanaleid, mis olid tänapäeva foorumi laadsed asjad. Need olid 
teemapõhised. 

Päris algul oli Eesti Fido ringkond 40--50 inimest, täitsa 
\emph{manageable size}. Ühel hetkel läks see läbuks kätte ära, kui 
maht kasvas ja lisandus väga erineva taustaga inimesi. Näiteks Salvas, kui meil 
oli \emph{point} niikuinii püsti ja Novellist ligipääsetav, istusid seal sees sekretärid ja kes iganes, kellel oli aega. Varieeruvus läks väga 
suureks, inimeste taust ja huvid väga erinevaks. 

\question{Enam ei olnud nii elitaarne klubi?}

Fido ei olnud minu arust kuidagi suletud või elitaarne klubi, ma ei ole 
kunagi tajunud, et see oli salajane või 
erilise müstilise ligipääsuga, vaid pigem oli algul lihtsalt inimesi vähe. 

Ühesõnaga, Fido jäi minu elus kõrvale ja ka üritused hakkasid hääbuma.

\question{Kas tolleks hetkeks olid sa juba 
leidnud, et kuna maailmas on olemas Hannu Krosing, siis sina enam 
programmeerida ei taha?}

Jah, läksin tööalaselt 
pigem mujale. Kõigepealt oli võrkude adminnimine ja haldus mulle väga jõukohane ja ma sain aru, kus ma väärtust lisan. Sama ajal 
progemises oli nõudlust vähe, vähemalt ses osas, 
kuhu mina oleksin saanud pakkumist esitada. Buum oli ikkagi seal, kus 
oli vaja vett kõrbe viia: arvutid kokku panna, võrku ühendada ja
tööle saada. Paljud inimesed, kes 
oskasid progeda, teenisid oma leiva ikkagi sellist tüüpi asjadega. 
Arvuteid kokku panime ja konfisime me Skriiningus küll kõik. 

Ega see ei olnud nii lihtne nagu tänapäeval. Aitasin just oma poistel mänguarvutid kokku panna. 
Neil olid läpparid, aga nad ütlesid, et need on lahjad ja et nad tahavad \emph{desktop}'e. 
Siis ma huvi pärast tellisin komponendid ja tegin nendega koos. See on 
tänapäeval super lihtne: pistikut ei ole võimalik valesti 
panna, see läheb ainult ühte kohta kogu süsteemis. Tol ajal pidi väga 
täpselt teadma, mis \emph{jumper}'id panna, kuidas ja kuhu panna pistik, ja kui 
tegid seda valesti, siis masin kärssas läbi. Ja ei olnud olemas 
Google'it. Oli dokument või dokumendilaadne asi, mille põhjal leiutada, 
mis \emph{jumper}'itega see kõvaketas töötada võiks, mida sa ilmselt nägid 
esimest ja viimast korda, sest ka juppide tarned Eestisse käisid suhteliselt 
naljakal viisil. 

\question{Kes tõi kohvris kuskilt Singapurist või teab kust.}

Sellega seoses on mul üks hea lugu. Mul olid siis juba autojuhiload, 
pidin üle kaheksateistkümne olema. Ja oli üks habemega 
mees, kes pidas Mustamäel arvutifirmat, mille nimi mul pole enam meeles, 
ja kes läks hiljem Estoniaga põhja. Suur mees nagu karu. 
Ja siis oli Kalle Lotamõis oma Skriininguga\index{Skriining}. Emb-kumb neist 
sai Hiinast faksi (info liikus tollal faksidega), et on soodsalt pakkuda mälu 
SIMe, mälumooduleid. Hind oli röögatult hea. Nad ajasid kahe peale ja ilmselt kuskilt juurde laenates raha kokku ja 
tegid ülekande ära. Kui pappkast jõudis kohale, käisime meie Palmasega sellel tollilaos järel. Kasti lahti tehes avanes kurb 
pilt: sees oli ainult 
vahtplast. Seda raha ei nähtud muidugi enam kunagi ja Skriining lakkus neid haavu päris kaua. Samas, kui seal oleks 
olnud \emph{legit} asi, siis oleks olnud kohe vinge marginaal. 
\emph{Cowboy times}. 

\question{Sa mainisid, et tahtsid rohkem väärtust lisada. Kas sa tõesti mõtlesid juba tol ajal, 
kuidas kasulik olla?} 

Võibolla mõtlesin lihtsalt sellele, mis mul hästi välja kukub. Mulle tundus juba tol ajal, et mul kukub hästi välja tehnoloogia ja inimeste vahel liimiks 
olemine. Lähen inimese juurde, kes pole kunagi ühtegi arvutit 
näinud, pakin selle talle lahti, panen tööle ja näitan, kuidas käib. Ma 
olin kõrvust tõstetud sellest, et sain talle kasulik olla. Tema sai hakata oma tööd 
nüüd hoopis teistmoodi tegema kui varem. 

\question{Sul ei olnud abstraktne klient, vaid konkreetne 
inimene, kellel läks nägu särama!}

Jah. Sealt hakkas tulema ka esimene 24/7 kogemus. Servereid
öösiti enamasti õnneks küll ei kasutatud, aga need olid tegelikult ju 
ärikriitilised ja kui need läksid maha, siis tuli väga kiiresti kohale jõuda ja 
need ruttu tööle saada. Pluss väga vihase kliendiga tegeleda, 
seletada talle, mis juhtus. Kliendi jaoks olid need mingid maagilised kastid 
nurgas. Kuidas siis inimarusaadavas keeles seletada, mis juhtus ja miks sa 
arvad, et seda uuesti ei juhtu?

\question{Ja miks see sinu süü ei ole!} 

Või miks ma arvan, et tõenäosus, et see kohe uuesti juhtuks, on 
väiksem. Eks tihti oli raua probleeme, voolukõikumisi, miljon muud 
asja. 

\question{Tol ajal käis kõik tati ja teibiga kokku. Serveriruume ju polnud.}

Ei-ei, sellist asja polnud olemas. See oli liiga moodne sõna selle aja kohta. 
Millalgi need muidugi tekkisid, aga siis neid veel ei olnud. Valiti lihtsalt mõni 
puhas ruum, kus otseselt vett ei tilguks, et veeavarii tõenäosus 
oleks väiksem. Tihtipeale oli selleks raamatupidaja kabinet või mõni muu ruum, kuhu seda oli kõige 
loogilisem panna. 

\question{Ja ühel hetkel läksid sa Salvasse?}

Jah, Skriiningust sattusin ma Salvasse\index{Salva Kindlustus}. Mast\index[ppl]{Kaal, Madis} oli siis juba 
Foreksis\index{Forekspank}. Toivo\index[ppl]{Annus, Toivo} kutsus mind Salvasse ja Madisega oli ka juttu 
Forekspanka minekust. Ma ei mäletagi, mis põhjusel sai 
Toivo kasuks otsustatud. 

Salvas oli lihtne. Toivo ise käis ülikoolis, tal 
polnud aega sellega \emph{full time} tegeleda ja ta otsis kedagi, kes oleks 
päeval kontoris kohal. Midagi arvutivõrgust oli juba olemas, aga 
ettevõte kasvas kiiresti, nii et esiteks oli vaja tööjaamasid ja võrku 
hooldada ning teiseks inimesi \emph{support}'ida. Ajad läksid kogu aeg kiiresti moodsamaks. Meil ei olnud enam 
Novell 3.11, vaid 3.12. Laserprinterid olid mitmel korrusel. Minu projekt oli vedada maja peal laiali kaablikanalid nii, et kaablid 
polnud enam lihtsalt naelaga seina peale löödud, vaid käisid ilusasti 
plastkanali sees. 

Suur ja äge asi oli internetipanga eellane telefonipank. 
See käis niimoodi, et minu arvutis olid modem ja telefoniliin. Ja kuna me 
olime Novelli võrgus, siis raamatupidaja sai oma arvutis maksed ette 
valmistada, sisetelefoniga mulle helistada, et nüüd on kõik valmis, ja mina 
tegin sessiooni oma modemiga Hansapanka\index{Hansapank}. Ülekanded 
läksid üle ja samal ajal tõmbasime ära panga väljavõtte. 

\question{Hansapangal oli Telehansa siis vist tõesti juba olemas.}

Ma ei mäleta, mis toote täpne nimi oli, aga see käis modemi teel ja mingi 
\emph{fat client} tuli installida, millega sai tõmmata panga väljavõtteid 
ja teha ülekandeid. Arvutiekraanil oli ülekandevorm, mille täitsid ära, 
ja ka \emph{roles and rights} oli juba olemas. Näiteks mina sain teha 
pangasessioone, aga ei saanud teha ülekandeid. 

\question{Tõenäoliselt ei olnud see fail krüptitud, nii et 
mõningase vaevaga oleks saanud raha kanda ka mujale?}

Ma ei mäleta, kui kõva \emph{security} seal oli, aga 
esimesed jäljed \emph{roles and rights}'ist olid juba olemas, selle peale mõeldi. 

Töö või protsessi mõttes oli see suur samm edasi. Vanasti 
ju raamatupidajad tiksusid kogu aeg panga vahet, aga nüüd ei pidanudki 
enam pangas eriti käima. Ainult sularaha oli vaja ära viia. Kontole laekumised, kontojääk ja muud sellised asjad olid
\emph{magically} raamatupidaja arvutis olemas põhimõtteliselt iga kell, kui ta 
tahtis. 

Salva tegi siis ka otsuse, et enam ei maksta palka sularahas, mis oli tol ajal tavaline, vaid Salva Kindlustus avas kõigile 
töötajatele Hansapangas kontod ja deebetkaardid. Need
olid nii kallid asjad, et kui see oleks jäetud töötajate teha, siis ilmselt enamus 
poleks selle projektiga kaasa tulnud. Palju lihtsam oli palk sulas iga 
kuu välja võtta, kõik olid sellega harjunud. Aga siis hakkas palk reaalselt panka tulema 
ja pidi leidma ATMi, kust palk korraga või jupikaupa välja võtta. 

\question{Miks see Salvale kasulik oli?}

Et saaks sularahaga mässamisest lahti ja kõik oleks \emph{clean}. Ega 
kellelegi, eriti mitte raamatupidajale, ei meeldinud sellega tegeleda. 

\question{Mille peal Salva peamine äriprotsess jooksis?}

Hea küsimus. Minu arust oli meil mingi naljakas 
Inglismaalt ostetud programm, nime ei mäleta. Küll aga mäletan, et 
käisime Tõnu Laagiga\index[ppl]{Laak, Tõnu} ühes Lõuna-Soome 
kindlustusseltsis nende infosüsteemi vaatamas plaaniga see osta. Ost 
jäi kokkuvõttes küll katki. Ajastu näitena veel selline infokild, 
et Toivo\index[ppl]{Annus, Toivo} ei saanud sinna kaasa tulla sellepärast, et 
tal ei olnud välispassi, aga minul oli.

\question{Miks sul välispass oli?}

Ma ei mäleta, miks. Olin ehk kuskil spordi pärast 
võistlemas käinud. See võis olla veel 
Vene pass või siis ikkagi Eesti pass ja mul oli viisa, aga Toivol polnud? 
Igatahes Soomes oli neil 
korralik infosüsteemi moodi asi. 

\question{Kas sa siis tegid sporti ka?}

Pigem sel ajal just enam ei teinud. Kooliajal tegin
orienteerumist ja kui tulin ITsse, siis jäi see katki. Nii et sisuliselt ma 
vahetasin spordi IT vastu osalt Nõukogude Liidu lagunemise tõttu ja osalt
arvutite tuleku tõttu. Orienteerumine oli meil väga tugevalt finantseeritud Saue 
sovhoosi\index{Saue sovhoos}\sidenote{V. I. Lenini nimeline 
köögiviljakasvatuse näidissovhoos.} poolt ja kui sovhoos ära kadus, vajus treeningugrupp ka laiali. Samas minu 
jaoks tuli IT varem.

\question{Aga sa teed ju praegu ka sporti?}

Jah, pärast sõjaväge hakkasin uuesti tegema. Sõjaväkke sattusin aastaid 
hiljem, kui olin juba väga vana. Algul olin ülikoolis ja tol ajal 
ülikool vabastas sõjaväest või lükkas seda edasi. Aga mul oli vaja nii palju tööl käia, 
et ülikooli enam ei jõudnud ja kukkusin välja. Siis 
jõudsin veel olla mõnda aega nii, et ei saadetud sõjaväkke, aga lõpuks ikkagi 
läks asi tõsiseks ja tuli ära käia. 

\question{Mida sa ülikoolis õppisid?}

TPIs\index{Tallinna Tehnikaülikool!Informaatika} infosüsteeme. 
Käisin koolis Salva kõrvalt. Keskkooli lõppedes oli mul koolist ja 
töö kõrvalt õppimisest nii suur tüdimus peal, et lubasin endale
kõigepealt aasta aega ainult rahus tööd teha. TPIsse läksingi aasta pärast, aga see oli tegelikult viga, sest siis ma 
olin juba niivõrd töö-\emph{mode}'is, et kooliskäimisest ei tulnud eriti midagi välja. See oli kõige madalama 
prioriteediga asi ja pigem käis kummivenitamine, kuni lõpuks tuli kahe-kolme aastaga eksmatt. 

Pärast sõjaväge tegin EBSis baka ära. EBS tegi 2000ndatel IT-juhtimise 
eriala. Ma olin esimeses lennus 
koos Alek Kozlovi\index[ppl]{Kozlov, Alek} ja muu kambaga. Tegime 
viieaastase baka, nüüd on ju kolmeaastane. See aeg oli 
ka väga kihvt, aga see on juba teine lugu. 

\question{Mida sa pärast Salvat tegid?}

Salvast sattusin kohta, kus nüüd saan uhkustada, et töötasin seal 
koos hilisema Eesti Vabariigi presidendiga\index[ppl]{Kaljulaid, Kersti}. Üks sugulane 
rääkis mulle, et üks firma, kes tegeleb sidesüsteemidega, otsib 
inimest ja kas tahaksin nendega rääkida. Miks mitte. Ma olin Salvas juba mitu aastat olnud. 

Sain kokku Rene 
Maksimovskiga\index[ppl]{Maksimovski, Georgi-Rene} ehk president Kaljulaidi 
abikaasaga, kes oli sellise ettevõtte omanik, mis pani Eestis üles Siemensi 
telefoni keskjaamasid. Sihtgrupp oli suured ettevõtted: pangad, 
riigiasutused. See oli selles mõttes \emph{next phase}, et kui 
personaalarvutid olid raamatupidamisosakondades olemas, siis umbes aastal 1995 
oli sidet vaja: võimalust helistada nii 
majas sees kui ka välismaale nii, et ei krõbiseks ja saaks kaugekõnesid teha. 

See, kuidas me tol ajal Siemensi telefoni keskjaamasid paigaldasime ja hooldasime, 
oli samamoodi vee viimine kõrbesse. Täna pole lauatelefone 
õieti kellelegi tarvis, mobiiltelefonid ajavad asja ära. Aga see oli enne 
mobiiltelefonide aega ja iga inimese lauale oli vaja 
telefoni, millega saaks helistada maja piires ja majast välja. 

\question{Jagada suure hulga inimeste vahel  väikest hulka telefoniliine!}

See oli vinge etapp, aga juba ITst kaugemal, telekomi maailmas. Haberstist\index{Haberst} 
edasi sattusingi Uninetti\index{Uninet}. Seal tegin kaasa selle aja, 
kui Uninet tuli telefonivõrgu turule ja paigaldasime telefonikeskjaama 
Eestisse. Sealt omakorda liikusin Elisasse\index{Elisa} ja Elisast Skype'i\index{Skype}. 

\question{Sinu jutust jääb mulje, et sa oled alati
kaablit tõmmanud, aga nii kaua, kui mina sind tean, on su tegevuse 
subjektiks eestvedaja või juhina pigem inimene kui kaabel. Mis hetkel sul see 
vahetus toimus ja kas sa üldse näed siin vahet?}

Kaablivedamisega sai mu karjäär alguse, 
tegelikult ma ei oska seda üldse hästi. Kuni Haberstini oli mu töö
ikkagi sügavalt arvutite ja arvutivõrkudega seotud ning kogu Habersti ja 
Unineti periood oli pigem keskjaamade progemine ja laiendamine või 
veasituatsioonid -- 24/7 töö. Näiteks kui Jõhvi piirkonna politsei side läheb maha ja seal piirkonnas 112 ei tööta, siis on päris suur 
probleem. 

\question{Aga inimesed?}

See algas Haberstis\index{Haberst}. Organisatsiooni kasvades oli ühel hetkel vaja 
mu roll formaliseerida ja keegi ilmselt tegi mulle ettepaneku hakata teisi insenere juhtima. Ma 
kasvasin tiimist välja tiimi ehk tehnikaosakonna juhiks. See on kindlasti murdepunkt, kui sa ei
võta vastutust mitte ainult enda, vaid ka teiste 
inimeste töö eest. Edasine karjäär on paraku pea kogu aeg 
olnud kas tootejuhtimine, projektijuhtimine, inimeste juhtimine või 
nende segu. 

\question{Miks \enquote{paraku}?} 

Ilus oleks minna tagasi spetsialisti liistude juurde, kus saaksin 
vastutada ainult oma töö eest. 

\question{Miks sa pole läinud?}

Pole vist julgenud. Üks põhjus on kindlasti see, et olen oma kompetentsi 
kõvasti kaotanud. Juhtides 
tehnilisi tiime Skype'is, Elisas või nüüd Fleepis\index{Fleep}, näen, kui andekaid tehnilisi talente on tegelikult olemas. Ja et ma ei ole
tehnilistes oskustes üldse konkurentsivõimeline.

\question{See on väga tuttav tunne.} 

Teine põhjus on illusioon, et olen omandanud inimestevahelise suhtluse ja 
koordineerimise \emph{skill}'id ja 
parem sõidan nende peal.

\question{Kas sul tuli see pärast esimest otsust loomulikult?}

Jah, kuigi Uninetti läksin ma ka ikkagi telefonikeskjaama paigaldama ja haldama. Haberstis olid väiksed ettevõtete keskjaamad. Uninetiga panime üles nii-öelda telekomi keskjaama, 
mille külge käivad väiksed klientide keskjaamad ja mis 
ühendub SS7ga ülejäänud telefonivõrku. Meil olid ühendused nii Soome 
Finneti kui ka kõigi Eesti operaatoritega. See oli ka väga põnev aeg, aga sealgi kasvasin ma 
paraku jälle tehnikatiimi või võrguoperaatori üksuse 
juhiks. Pärast Elisas juhtisin mobiilivõrku, kus oli ka raadio-pool. 
Sisuliselt oli tegu juba inimeste ja tehnoloogia koordineerimisega, aga enamasti on selles 
rollis ikkagi ka tehnoloogiastrateegia pool sees. 

\question{See kõik seletab väga hästi, miks ma tean sind 
praktilise inimesena.}

Ma ei ole oma karjääri kunagi teadlikult planeerinud. Pigem olen sattunud väga huvitavatesse tiimidesse 
või kollektiividesse, mis on mind kuhugi suunas arendanud, 
kogemusi andnud ja järgmisi uksi avanud. Ma ei mõtelnud samme ette, 
kuhu ma tahaksin jõuda. Pigem olen otsinud meeskondi, kus ma 
sisse minnes tajun, et olen kõige rumalam inimene ruumis. 

\question{Tahtsingi just küsida, milline on äge tiim.}

Võimalikult mitmekesine ja üksteist toetav -- 
niisugused asjad on mulle ka tähtsad. Aga ennekõike see, et valdkond arendaks mind. Võibolla kõige parem 
näide oli Salvast Haberstisse minek. Novelli adminnimises ja 
arvutivõrkudes tundsin ennast juba suhteliselt kindlalt, aga mis on telefonikeskjaam, ISDN ja 2megabitised ühendused? Õrna aimugi 
polnud! Veelgi enam, mind saadeti rahvusraamatukokku, kus oli just keskjaam 
üles pandud, et mine tee koolitus. Ma polnud seda telefoni, mille koolitust ma tegema pidin, kunagi elus näinud. Hüppasin vette, ujusin ja õppisin selle käigus, kuidas ujumine käib. 

Väga kihvtid inimesed on olnud ja see on kõige tänuväärsem asi. Lisaks on kihvt, kuidas mingid inimesed käivad ringiga, näiteks kuidas 
jõuan lõpuks Masti\index[ppl]{Kaal, Madis} või Toivoga\index[ppl]{Annus, Toivo} 
Skype'is\index{Skype} kokku tagasi. Mõnes mõttes on see maailm väga suur, aga teistpidi 
jälle tulevad teatud inimesed su juurde ringiga tagasi. Samamoodi hooned, näiteks see, kus me praegu oleme\sidenote{Ajasime juttu Tehnopoli hoones 
Akadeemia teel.}, või Suur-Ameerika 18, kuhu ma sattusin 
Haberstisse minnes uuesti ja töötasin selle ruumi kõrvaltoas, kus ma esimest korda 
286 taga istusin. 

\question{Siinsamas majas oli ju Skype!}

Täpselt selles tiivas -- mina tulin tööle siiasamasse esimesele korrusele. 
Vastas üle koridori oli Paananen\index[ppl]{Paananen, Tiit} oma 
\emph{certification}'i tiimiga. Mul oli mõnes mõttes nagu \emph{coming 
back home}, kui see osa valmis sai ja meile siia ruumi 
pakuti. 