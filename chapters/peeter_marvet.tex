\index[ppl]{Marvet, Peeter}
\index[ppl]{Tehnokratt|see{Marvet, Peeter}}

\question{Kuidas ja umbes millal sa jõudsid arvutite juurde?}

Ma arvan, et see oli umbes täpselt aastal 1985, ma pidin siis olema 15 aastat vana. Ilmselt ma olin arvuteid eelnevalt näinud Soome telekast, ma usun, et seal reaklaamiti ilmselt Commodore 64 ja Spectrum masinaid. 

\question{Sa oled siis Tallinna poiss, ma järeldan?}

Jah. Ma olen sündinud Tartus aga pikalt olnud Tallinnas. 

Aga kui ma veel tagasi krutin, siis ma arvan, et kokkupuude arvutitega oli veidi varem, mu papsi laboris. Laboriks oli toonase TPI Veekvaliteedi Labor\index{Tallinna Tehnikaülikool!Veekvaliteedi labor}, asus selle koha peal, kus keset Järvevana teed on praegu Maru ehituse maja. Seal oli olemas üks terminal, mis käis Datasaabi\index{Arvutid!Datasaab} nimelise arvuti\sidenote{Datasaab oli Rootsi lennukitootja Saab arvutustehnoloogia eraldi ettevõtteks kasvanud division. Toodeti nii tsiviil- kui militaarkasutuseks mõeldud arvuteid.} külge, mis asus kusagil Mustamäe teel. See masin oli ostetud ühelt nii-öelda Rahvamajandussaavutuste Näituselt, kus meil vahetevahel käisid väljakad ka kohal. Oli saadud raha ja ostetud valuuta eest välismaa arvuti, mille külge käisid modemitega sellised oranžid või \emph{amber} värvi terminalid. 

Datasaabi lugu oli selline, et see osteti väidetavasti ilma operatsioonisüsteemi ja igasuguse rakendustarkvarata. Selle jaoks nagu rohkem rutsi sellel ajal ei olnud. Aga Nõukogude insenerid olid vinged, nõukogude insenerid kirjutasid sinna ise mingi operatsioonisüsteemi peale. Või siis pandi midagi kusagilt tuuri. Ei, ma arvan, et tuuri ei pandud, sest mulle meenub, et vist õnnestus sellest kirjutatud tarkvarast veel mingisugune jupp Datasaabile tagasi müüa ja saada selle eest ilmselt siis vastu mingisugust mälu või mingisugused lisakomponente. 

See oli esimene termininal, millega ma kokku sattusin. Seal õnnestus terminali peal lihtsalt ilma ühenduseta niimoodi \emph{backspace}-i ja  tühikuga \enquote{ronge kokku haakida}. See on minu esimene mälestus sellest, kuidas ma olen arvutiga suhtestunud. 

Aga järgmine mälestus on  aastast 1985, kui ma ise olin viieteist aastane. See oli ilmselt seitsmendas klassis. Toimus üks Tallinna koolide füüsikavõistlus, päris viktoriin ta ei olnud,  toimus toonases Pedas. Ja  meid viidi seal lisaks sellele füüsika asjale ka arvutisaali. Ma arvan, äkki oli see  Minsk\index{Arvutid!Minsk}. Seal ma sain tuttavaks ühe koolivennaga, kes oli minust aasta vanem, ehk siis juba kaheksandas klassis, ja kellel oli kaasas isiklik perfolint programmiga. Selleks oli ei keegi muu kui Sulo Kallas\index[ppl]{Kallas, Sulo}. Ma oletan, et seal lindi peal oli mingisugune mäng, mingisugune niisugune evolutsiooniline, mingisuguste asjade arengu mingisugune kood. Ma isegi ei suuda nüüd meenutada, mis asi see täpselt oli. Selle perfolindi peal oli igal juhul mingisugune selline mängulaadne asi, mis minu arust ise  arendas mingisuguseid organisme või midagi sellist\sidenote{Tõenäoliselt oli seal lindil Briti matemaatiku John Horton Conway poolt välja mõeldud rakuautomaat, mida tuntakse nime all \emph{Game of Life}. Tegu on mängijateta mänguga, mis ainsa sisendina vajab algseisu määratlemist. Automaat on ühest küljest levinud programmeerimisülesannne ja teisalt  põnev uurimisobjekt, seetõttu võis selle realiseerimine olla noorele arvutihuvilisele  nii huvitav kui ka jõukohane.}. Sulo lugu  oli selles, et tema vend oli Raadiomaja Arvutuskeskuses\index{Raadiomaja Arvutuskeskus}, nii et selleks ajaks oli Sulo juba mõnda aega arvutitega tegelenud. 

Ja see oligi nagu esimene kord, kus sattusin arvutiga kokku ja mõtlesin \enquote{oo, vinge!}.

\question{Mis seal vinget oli, mis konks see oli, mille külge sa jäid?}

Tol hetkel  oli see rohkem selline nagu \enquote{ahaa, vau, et teebki mingisugust asja}. 

\question{Aga Sulo oli kõva mees küll oma perfolindiga?}

Nojaa, sul on nagu kaheksandik ees,  nagu vanem koolivend, kellel on,  kujutatad sa ette, isiklik perfolint! Vau! Sellised kutid on ümberringi! No siis peab ikka ka vaatama, et mis seal tehakse. Ma arvan, et  midagi Soome telekast arvutitega seoes nähtutoli tegelikult see, mis tekitas soovi, et olgu või nõukogude oma ja perfolindiga, aga ikkagi arvuti. 

Ja järgmine asi tegelikult ei olnud sellest kaugel. See oli, vast mõni kuu või natukene hiljem, kui kooli tulid paar djuudi, kes tekid mingit arvutiklubi ja kutsusid sina osalema. 

\question{Kas see oli legendaarne arvutiklubi Ahhaa?}

Ei. See on legendaarne arvutiklubi Juta\index{Arvutiklubi!Juta}, mida vedas juudi papi nimega Lev Moišeejevitš Šoroht\index[ppl]{Ruht, Lev}. Klubi tegutses Raua ja  Kreutzwaldi tänava nurga pealt, kui mööda Raua tänavat  kesklinna poolt tulla, siis vasakut kätt on niisugune maja, kus on  kaares aknad keldrikorrusel. Vaat sealsamas kaares akende taga vedas ennast Arvutiklubi Juta. Kui Ahhaa puhul võiks kujutada ette, kust see tuleb, siis kust tuleb nimi \enquote{Juta}. See tuleb vene keelest: \begin{russian}Юный Техник Автомат\end{russian}. Ma eeldan, et Lev Moišeejevitš Šoroht ei satu seda  kuulama, aga kui  kellelgi meenub, et aastal 1985  on ta kas TPI või ka Peda üliõpilasena vedanud mingisuguseid noori arvutiklubisse, siis ma suurima hea meelega saaks kokku ja teeks väikesed õlled välja. Sealt see alguse sai. 

Klubis õpetati meile programmeerimist keeles PL/I\index{Keeled!PL/I}.

\question{Ja klubisse käidi kutsumas, mitte ei joostud ust maha, et arvuti ligi saada?}

Ma arvan küll. Ega ma täpselt ei mäleta, aga ma arvan, et see toimus kuidagi kooliks või õpetajate kaudu jõudis see sõnum meieni. Igal juhul mul on mälestus, et ma sattusin sinna kuidagi nagu koolist. Ja ma arvan, et Sulo oli ka seal. Sest et kui oli võimalik kusagil veel arvutisse saada, siis loomulikult seda kasutati.

\question{Seda ma mõtlengi, et tol ajal ju otsiti tikutulega neid kohti, kus \enquote{arvutisse saada}?}

Minu arust aastal 85  see teadlikkus arvutitest oli umbes selline, et kusagil seitsmenda klassi klassi lõpupoole tuli kusagilt arutuskeskusest üks mingisuguse minu programmi väljatrükk, laia aukudega paberi peal, \emph{line} printeri peal välja lastud. Keegi keegi mu klassivendadest, kes siis ilmselt oli ka meiega seal koos käinud, tõi selle mulle klassi ja siis kõik vaatasid, et \enquote{oo, Soome telekavad}. Sest et tollel ajal kõige selline üldisem seos arvutitega, mis inimestel oli, oligi see, et arvutuskeskustes trükiti välja Soome telekavasid. Seal olid tabuleeritud kujul mingid asjad, kus olid see, et kust seriaalid jooksid ja mingisugused selline üldisem leht. Parimatel vendadel olid olemas nädalakavad. 

\question{Kust need kavad saadi?}

Nad liikusid arvutuskeskuste vahel ja ma tean, et (jällegi nüüd täpselt ajaloolisi hetki kokku panemata), aga vähemasti mingil hetkel oli üheks selliseks kohaks Postimaja Arvutuskeskus\index{Postimaja Arvutuskeskus}. Mis on iseenesest üks väheseid kohti, kus ma ise pole käinud. Ja seal oli mäletamist mööda SM-4\index{Arvutid!SM-4}. Selle külge oli ehitatud mingisugune teksti-TV vastuvõtja  ja sealt see kava tuli.  Ma arvan, et see oli mõni aasta hiljem: kui ühes kohas oli Minsk ja teises SM-4, äkki ma ajan mingisugused ajajooned natukene sassi. Selle SM-4  küljes oli kolmesajaboodine modem, millega siis pumbati neid asju mööda linna laiali. Ma mäletan mingisuguste järgmist etappi sellest, kus  minu päralt oli üks PC välja arvatud, ma arvan, et see oli kolmapäeviti, kui lõuna paiku saabus postimajast üle modemi Soome telekava ja siis maatriks printeri peal enam-vähem nähtamatuks kulunud lindiga trükiti seda välja. Siis ma pidin omale  muud tegevust leidma, muudel pärastlõunatel ma sain seda arvutit kasutada.

\question{Ma katkestasin sind seal, kus sa PL/I keeles programmeerisid\ldots}

Meile õpetati natukese programmeerimist ja seal oli hästi segane selles mõttes, mingisugune asi nimega jobiohjur ja mingeid täiesti arusaamatuid asju. Oli programmeerimiskeel mingisugust  muutujate ja mis nagu mõnevõrra koitis. Ja siis ilmselt esimene programm, mida meile seletati, kuidas teha oli ruutvõrrandi lahendamine. Annad paar muutujat sisse ja siis trükitakse sulle tulemus mingisugune käraka  paberi peal välja. Alguses vist oligi see,  meid nagu päriselt arvuti juurde ei lastudki. Keegi vist \emph{punch}-is meie programmid sisse ja pärast saime väljundi kätte. 

Pärast meile leiti võimalus arvutitega tegeleda kahes kohas.  Üks oli Tihnikus, kus oli ETKVL-i Arvutuskeskus\index{ETKVLi Arvutuskeskus}. Järgmised põlvkonnad teavad seda kohta kui esimest Maksimarketit, seal ühes majas oli üks vinge ES\index{Arvutid!ES EVM}. Teine koht oli see, kus Endla tänaval asub (või vähemalt mõni aeg tagasi asus) Maksuameti üks ots\index{2013. aastani asus Maksu- ja Tolliameti teenindussaal aadressil Endla 8.}. Seal  üleval kolmandal korrusel olid Ehituskomitee\index{Ehituskomitee} ES-id\index{Arvutid!ES EVM}. 

Minuga läks sealt edasi läks umbes niimoodi, nagu õpetatakse tööõpetuses, et on oluline  anda lastele mingisugune selline asi, mille nad saavad valmis voolida ja tulla koju, perele näitama. Siis ta saab kiita ja edasi läheb väga hästi ja ta teeb paremaid puulusikaid edaspidi. Minuga läks niimoodi, et olles teinud oma esimese  kolmeteist-realise programmi ruutvõrrandi lahendamiseks, laekusin ma  selle väljatrükiga koju ja köögis näitan, siis  lapsevanematele, et näe sihukse raha eest tegin sihukse asja. Paps, kes oma teadustegevuse poole pealt tegeles elektrokeemia ja hapnikuanduritega ja teise poole pealt olles džässpianist vaatas mu tööd ja ütles, et \enquote{mul oli siin just mingisugune tudeng, aga kadus ära. Temast jäid ainult mingid listingud järgi. Kas sa saad sotti,  mul oleks vaja mingisuguseid  teadusandmeid töödelda? } Seal toimus mingisuguste anduri toime kõverate kokkuajamine mingite matemaatiliste valemitega et õnnestuks  mingisuguseid digitaalseid mõõteriistu teha või midagi sellist. Ja niimoodi juhtuski, et olles  programmeerinud oma esimesed kolmteist rida esimeses mulle täiesti tundmatus keeles, \emph{switch}-isin ma koheselt järgmisele keelele.

Nii ongi mu pea selles mõttes nagu puder ja kapsad, et ma suudan kirjutada ainult dokumentatsiooni abil, kaasa arvatud keeli, mida ma igapäevaselt kipun  kirjutama nagu  PHP\index{Keeleled!PHP} ja JavaScript{Keeled!JavaSctipt}. Need süntaksid on peas nii segi selles mõttes, et ma kunagi ei mäleta, täpselt mis oli PHP-s see mingisuguse \verb|for| tsüklis  parameetrite järjekord. Õnneks on tänapäeval olemas kõikvõimalikud IDE-d, mis teevad mõningase töö ära ja aitavad \emph{auto complete}-da ja mida iganes. 

\question{Aga siis su puulusikas mitte ainult ei saanud kiita vaid pandi kohe ka tööle!}

Puulusikas võeti kohe tööle, sealt edasi ma olen olnud lapstööjõud mõttes olnud, et hakkasin papsile tegema. Mingil hetkel ju tal hakkas nagu kahju ka, et laps võiks  lisaks ekspluateerimisele natukese ka raha saada. Ehk ma olin siis ametlikult kirjas mingi veerandkohase laborandina või midagi sellist.  Parem pool oli muidugi see, et tänu sellele oli mul ligipääs kõikide papsi sõprade arvutuskeskustele. Ja kuna paps tegeles selle oma hapnikuga TPIs, siis loomulikult oli TPI ja port seltskondi, kes olid seotud mingisuguste anduritega. Näiteks see pool, mis tegutses kunagi Pirital Masti tänaval,  kus  aretati sportlaste mõõtmise lahendusi. Selle mingisugune teine ots asus Kiirabihaigla arvutuskeskuses\index{Kiirabihaigla Arvutuskeskus}. See oli see koht, kus mul oli üks arvuti pidevalt kasutusel, välja arvatud kolmapäeviti. Sanyo PC, väga vinge. Seal oli muuseas Apple II\index{Arvutid!Apple II} olemas, selle peal sai mängitud Karatekat\index{Mängud!Karateka}, minu arust mingeid muid muid toredaid rakendusi tolel masina peal ei olnud. Ja kui ma hästi mäletan, oli seal ka mingisugune Labtam-i\sidenote{Labtam oli suhteliselt obskuurne Austraalia arvutitootja, kes tegutses aastatel 1972 kuni 1990. Miskipärast olid neil oma lõpu-aastatel Nõukogude Liiduga head suhted: aastal 1984 disainis Novosibirski Riikliku Ülikooli tudengite Kronos Research Group neile emaplaadi, URAL-LABTAM OOO tegutseb Venemaal siiani ning nende arvuteid leidub lisaks Austraalia arvutimuuseumile ka Tartu Ülikooli omas. Labtami arvuteid osteti naftadollarite eest ka Küberneetika Instituuti\index{Küberneetika Instituut}} nimeline \emph{kone}. Lisaks, oli seal mingisugune selliste suurte trumlitega andmetöötlus-\emph{kone}, mille nime ma ei mäleta. Seda äkki oskab Kalle Lotamõis\index[ppl]{Lotamõis, Kalle} või keegi selline  meenutada, et mis see oli. Mina selle suure masinaga sellel hetkel ei suhtestunud. 

\question{Kas sulle see andurite maailm ja elektroonika ei  pakkunud huvi?}

Mitte liigselt. Progemise pool oli  huvipakkuvam või siis katsetada, mida annab teha. 

Siis oli üks keskus, mis oli TPI Santehnika Kateedri\index{Tallinna Tehnikaülikool!Santehnika Kateeder} SM-4\index{Arvutid!SM-4}. Ehituse all oli selline kateeder, vee kvaliteet ja kõik selline kuulus  sinna alla. Mingil hetkel tekkis sinnasamasse Järvevana teele, kus asus ka Läänemere Instituut\index{Läänemere Instituut}\sidenote{Ei ole selge, mis asutust Peeter silmas peab. Eestis on Läänemere Instituut tegutsenud eelmise sajandi kolmekümnendatel ja praegu tegutseb sellenimeline asutus Soomes.}, ka SM-4\index{Arvutid!SM-4}, mis oli  enam-vähem niisugune masin, mille ma läksin kohanle ja lülitasid ise sisse ning pärast töö tegemist jälle viisakalt välja. 

\question{Neid arvutid siis ikkagi ju oli?}

Neid oli selles mõttes, et kui sa nagu sattusid õigesse kohta ja ilmselt oskasid õigel ajal  vait olla ja mitte liiga palju täiskasvanuid  segada nende tähtsas töös,  siis üldiselt neid nagu jagus. 

Tulles korraks veel tagasi  alguse, ehk Juta\index{Arvutiklubi!Juta} juurde siis selle asutajal oli endal ka paar niisugust  huvitavat projekti, millega ta üritas Nõukogude Liidu tasemel kuulsaks saada. Üks neist võiks olla  võrreldav Facebookiga. Tal oli üleliiduline projekt, kuidas inimesed, kirjasõbrad, saadaksid oma andmed kõik kokku, need sisestatakse perfokaardid peal \emph{mainframe}-i ja see kuidagi teeb mingisugust \emph{match-making}-ut ja siis saadetakse kirjad leitud \emph{match}-idele laiali. 

Mina tegelen igasuguste muude projektidega ja siis mingisugusel hetkel ma arvan, et me olime selleks ajaks juba jõudnud keskkooli, tekkisid arvutit ka meile Reaalkooli, ma olen  algusest lõpuni käinud Reaalkoolis.  Nendeks masinateks oli klassitäis Yamaha MSX-e\index{Arvutid!Yamaha MSX}. Nendega seoses on meeles, et kuna erinevate koolide vahel oli tolle klassi saamiseks konkurents,  olla kool saanud ka mingisuguse ähvarduskõne. Mille peale loomulikult vaprad raadioruffi ja füüsikaklassi tagaruumi noored organiseerisid öö läbi valve koolimajja. Arvutikastid olid kusagil kas direktori kabinetis ja kus iganes ja siis me ööbisime seal ja valvasime neid. Pärast see arvutiklassis sai suhteliselt meie selliseks nagu kodukuks.

\question{Mis seltskond see raadioruffi ja füüsikaklassi tagaruumi oma oli?}

Seal olime mina ja Sulo Kallas\index[ppl]{Kallas, Sulo} ja Heiki Savitš\index[ppl]{Savitš, Heiki} ja Vallo Veinthal\index[ppl]{Veinthal, Vallo}  ja Reimo Mesipuu\index[ppl]{Mesipuu, Reimo} ja no ma kindlasti jätan kedagi ebaviisakalt mainimata.  Avo Nappo\index[ppl]{Nappo, Avo} tiirles meie ümber rohkem nagu arvutiseltskonna poolest, raadioruumis olime rohkem vist mina, Sulo ja Reimo.

\question{Ma küsin pigem seda, et mille alusel too seltskond moodustus. Klassivennad? Tehnikahuvi?}

Otseseseid klassivendi oli  vähe, me olime kõik sama mingi paariaastane vahemik. Sulo oli kõige vanem, Vallo ja Heiki  olid meist aasta nooremad. Ehk et see oli selline parasjagu  klass üles, klass allaseltskond, kes siis oli  kooli aktiiv niisuguse tehnilise poole pealt. Kuidas me täpselt sattusime raadioruumi? Ju me siis hängisime füüsika kandis ringi ja  tuli välja, et seal on raadioruum, kus on ka mingeid nuppe, mida saab keerata. Kuidagi nagu sealt ümbert tekkis kogu seesama punt, kellest nagu väiksem osa oli  alguses raadioruumi ümber ja kellest siis suurem seltskond arvutiklassi ümber tekkis. 

\question{Kas programmerimine ja raadioruumitamine koolitööd ei hakanud segama?}

Ei tea. Ma usun, et ega ma mingisugune medaliga lõpetaja oleks niikuinii viitsinud olla. See nagu ei olnud kuidagi minu maailmavaates sees. Noh, neljade-viitega lõpetasin selles mõttes probleemi ei olnud.

\question{Eks see ongi tunnetuse küsimus, kumb primaarne oli tollel hetkel?}

Ma arvan, et eks arvuti pooli oli põhiline. Keskkool möödus üleüldse enam-vähem niimoodi  kuidagi, et vahepeal sai käidud kohvikus ja siis vahepeal sai käidud olümpiaadidel. Kui olid olümpiaadid, olid jälle hinded, sest õpetajad ei saanud ometi olümpiaadil esinejatele halbu hindeid panna. Aga kui olümpiaadil ei käinud, siis kippusid hinded kehvemaks minema sest et oli ununenud ära koolis käimine ja kõik muu selline.  Ma Siiamaani näen sihukesi  unenägusid sellest, et on tulekul eksam ja ma olen unustanud terve veerandi käigus tunnis käia. Aga sellest on  tekkinud selline  mõtteviis, et ma ei pea olema midagi õppinud. Ehk et kui TPI-sse läksin, oli ka tore selles mõttes, et kui mata eksam tehti koos raamatutega, siis ma jõudsin eksami käigus alati ära õppida selle, mida oli eksamiks vaja. Ma nagu ei pidanud  eelnevalt liiga palju mingisugusele loengus käimisele pühenduma, ma võisin lihtsalt tulla ja eksamid ju raamatuga  ära teha. Ülejäänud semestri võis lihtsalt arvutitega tegeleda. Ma kindlasti ei soovitaks seda kellegile noortest, aga noh, näed, niimoodi on see juhtunud. 

See on tekitanud mingisugune väga sellise teistsuguse arusaamise kuidagi kõigest ümbritsevast tehnikast. Ma nagu  ilmselt ei karda midagi selles mõttes, et ma kui on midagi vaja, siis lihtsalt tuleb võtta \emph{manual}  või kood ette. Läheb aega, selles mõttes, et ma loomulikult läksin eksamile esimesena sisse ja tulin viimasena välja. Aga  põhimõtteliselt kolme tunniga sain nagu õige asjaga hakkama. Tundus nagu efektiivne lähenemine. Võib-olla ma  oleksin saanud targemaks, kui ma oleksin seda asja süsteemsemalt õppinud. 

\question{Mida sa sinna TPI-sse õppima läksid?}

See oli TI ehk majandusinfo töötlus\index{Tallinna Tehnikaülikool!TI}. Linnar Viik\index[ppl]{Viik, Linnar} on lõpetanud sama asja mõned aastad enne mind. Aasta sattus kaheksakümmend üheksa olema, kus päris nagu Pol.Ök.-i ja Kompartei Ajalugu\sidenote{Nõukogude ajal kuulus ülikoolihariduse juurde Poliitökonoomia, Kommunistliku Partei Ajaloo ja teiste niiöelda \enquote{punaste ainete} läbimine.} ei tahaks õppida sinna kõrvale, aga nad ei olnud veel päriselt välja mõelnud, et mis see teine asi on, mida õpetada. Oli ka muid asju, millest ma päris täpselt toona aru ei saanud, miks seda peaks tegema. Kaasa arvatud see, et miks ma peaksin tegema tansistoritest valmis 8080 protsessori paar käsku. Eriti, kui normaalsed inimesed kasutavad vähemasti Z80-t ja mitte 8080-t. Teismelise värk, ei ole \emph{cool} piisavalt. \enquote{Intel 8008? Zilog\index{Zilog}\sidenote{Zilogi toodetud 8-bitine Z80 protsessor oli Inteli 8080 protsessoriga ühilduv aga märkimisväärselt odavam} on normaalne!} No täpselt selline asi, nagu on täna hipsteri habe või mingisugused muud välised tundemärgid. Pean takkajärgi tunnistama, et kuigi ma olin selle suhtes kriitiline, siis ma hetkel käin koolitusel, kus räägitakse sellest, kuidas \emph{fuzz}-imisega\sidenote{\emph{Fuzzing} on tarkvara (turva) testimise meetod, kus programmile söödetakse süsteemselt juhuslikku sisendit.} leida üles mälukorruptsiooni juhtumeid. Ja lektor ütles, et see on maru  keeruline, räägime hästi aeglaselt ja mitu korda nagu miilitsatele. Siis minu arust mida seal nii rasket oli, \emph{stack} on \emph{stack}. Protsessoril registrid, eks ju, ma põhimõtteliselt olen neid transidest teinud. Et kui sa pead nagu protsessori arhitektuuri tasemel läbi mõtlema, kuidas käskude töötlemine toimib, kuidas seal inkremenditakse mingisuguseid pointereid ja kuidas need asjandused selle mäluga seotud on, sa tegelikult saad aru, kuidas arvuti masinkoodi tasemel töötab. Mul on nagu väga tore kuulata, et kui mu vanem poeg on Tartu Ülikoolis, siis neid sunnitakse ka aru saama protsessori sise-ehitusest ja kuidas ta töötab. Tõsi küll  raamatu tasemel, aga progevad ka assemblerit, see on hästi oluline. 

\question{Kas sind akadeemiline maailm ei tõmmanud, kuigi sa seal servapidi juba sees olid?}

Ei, sest et ma olin keska ajal sattunud sellisesse seltskonda, nagu seda oli Vabariiklik Õpilasstaap\index{Vabariiklik Õpilasstaap}, mis oli selline Komsomoli Keskkomitee juures tegutsev mittekommunistlik vastupanuliikumine. Tiina Tšatšua\index[ppl]{Tšatšua, Tiina} oli näiteks üks selle eestvedajaid. Sellest sai üks selliseid toonaseid orgunn tiime vabariigis, kes korraldas suurüritusi, milleks alustuseks olid komsomoli ja EKP kongressid. Aga orgunni mõttes on ju suht savi, kas on EKP kongress või Eesti Kongress või Rahvarinde oma. Inimesed tulevad kohale, sul on mingisugused tegevused nagu  registreerimine ja kusagil tuleb neid toita. Kui on mingisugune selline dokumentidega üritus (mida tänapäeval eriti toimu, aga kõikide Eesti Kongressi ja Rahvarinde kongressid olid sellised), siis on sul näiteks mingisugune redaktsioonitoimkond. Me olime seal kui arvutitiim, kes orgunnis seda, et mingisugune registratuur toimiks mingite listidega ja samuti toetasime redaktsioonitoimkondi  kõikvõimalike tekstitöötluste pooltega ja  väljatrükkimiste ja vormistamistega ja millega iganes. Kui mul  keskkool sai läbi 1989, siis meenub, et suveks oli tööots, toimus ÜRO inva-ekspertide mingisugune hüper tipptaseme kokkusaamine Tallinnas. Sellega seoses mäletan, et see oli see, kus Tallinnas lõigati esimesed äärekivid faasi ja minu arust oli Jack Lipmaa\index[ppl]{Lippmaa, Jaak}\sidenote{Peeter peab ilmselt silmas Jaak Lipmaad} see, kes isiklikult ehitas ümber paar Ikarus-bussi\sidenote{Ungari tootja Ikarus bussid olid Eestis laialt kasutusel liinibussidena}, selleks et neisse  kuidagi  ratastooliga sisse saaks. Kuidas see võimalik oli, ma ei kujuta ette. Meie see Reaalkooli tiim toetas  redaktsioonitoimkonda,  kes seal mingisuguseid dokumente vormistas. Aga ÜRO-le kohaselt kõigis põhikeeltes. Mis tähendas, et oli ilge posu tõlke. Aga noh, aastal kaheksakümmend üheksa ükski tõlki ei olnud ilmselt arvutit näinud rohkem kui võib-olla Soome telekast reklaamist. Meid oli piisavalt palju selles mõttes, kümmekond tükki,  ja hoidsime siis kätt ja jalga. Kogu aeg oli olemas mehitus  praktiliselt igaühe jaoks, kui kellelgil tekkis niisugune kivistunud pilk, siis keegi tuli ja \emph{rebootis} selle tõlgi arvuti taga või arvuti, kumb igatahes parasjagu oli rohkem kinni jooksnud. 

Minu enda niisuguses hilisemas eluloos on see episood huvitav, sellepärast et tolle ürituse jaoks, olles parasjagu keska lõpetanud aastal 89,  õnnestus mul lihtsalt omaenda sõna peale linna pealt toatäis arvuteid, PC-sid, kokku laenata. Ma lihtsalt läksin, ütlesin et mul oleks nagu vaja, ja siit sai jälle paar tükki, sealt sai paar tükki ja niimoodi sai need umbes kaheksa arvutit kokku. Kõige kihvtim neist tuli surnukuurist. See oli üks PC, aga tema peal oli selline asi nagu Xerox Ventura Publisher koos Xeroxi graafilise kasutajaliidesega, milleks oli GEM, mis nägi põhimõtteliselt välja nagu MacOS\sidenote{GEM (\emph{Graphics Environment Manager} oli tõepoolest üks varastest graafilistest kasutajaliidestest. Liigne sarnasus Apple tarkvaraga viis ka kohtuasjani.}. GEM sai DOS-ist üles \emph{boot}-itud,  läks ilusti siukseks mustvalgeks ja halliks kasutajaliideseks ja seal peal jooksis minu esimene küljendusprogramm. Ja me saime neilt ka ühe laserprinteri kasutada, mis ei olnud küll PostScript tollel hetkel aga mis oli täiesti  laserprinter. 

\question{See oli ju nõukogude aeg veel?}

Ilmselt siis meditsiin oli ikkagi saanud mingeid asju kusagilt valuuta eest osta. Tegelikult Kivilo\index[ppl]{Kivilo, Agu} plaanis oma  diagnostikakeskust\sidenote{1988. aastal asutatud Diagnostikakeskus oli omal ajal märgilise tähendusega. Ühest küljest oli tegemist kõrgtehnolootiliste teenustega, keskuse algusaegadel asus seal Eesti ainus kompuutertomograaf. Teisalt oli aga tegemist väga innovatiivse organisatsioonilise konstruktsiooniga, milline asjaolu viis hiljem mitmete keskust ümbritsenud kõrge profiiliga afäärideni.} kesklinna ja meditsiini poolel olid  tegelikult  väga kõvad tegijad. Eesti arvutinduse arendusest teatakse rohkem seda seltskonda, kes oli nagu Tartu poole pealt ja seotud geeniga ja võib-olla Küber\index{Küber}. Mina olen tulnud  nagu meditsiiniliini pidi sisse, seal valdkonnas tegeldi päris kõvasti teadus- ja arendustegevusega. 

\question{Ja sealt said sa oma küljendamise-konksu?}

Jah. Otse loomulikult sai hunnik flopikettaid ära kopeeritud, eks ju. Mis toona oli nagu igati tavapärane \emph{standard operating procedure}:  kõigest, mis kätte satub, tehti koopia. Ja niimoodi siis sattuski et kusagil sealsamas 1989-1990  oli minu jaoks ülikoolis käimisest palju huvitavam asi see, et arvuti peal on  võimalik mingisugust sellist küljendust või kujundust või disaini teha. 

\question{Kas sul muidu ka niisugune joonistamise soon oli?}

Ei ole. Ma kahtlustanud, et inimesed, kes on pidanud minu küljendatud raamatuid tarbima kindlasti on selle all kannatanud, nii et ma väga vabandan. Kunagi  Avita\index{Avita kirjastus} kirjastuse  algalgusaegade raamatutest suur hulk oli minu tehtud  näiteks.

\question{Aga mis sind selle küljendamise juures niimoodi köitis, kui sul muidu sellist visuaalkunstide huvi ei olnud?}

See oli pisut teistmoodi, mingisugune selline hoopis teistmoodi  arvutiga tegelemine, kui seda  oli  programmeerimine ja andmetööstus mis olid ka tore. Aga see, et sul õnnestub mingeid asju ekraanil teha, see oli see, mis mind kuidagi väga sellel hetkel tõmbas. 

Ehk siis 89. aasta suvel too üritus sai läbi, mina läksin ülikooli. Ja siis Mart Siilmann\index[ppl]{Siilman, Mart}, kes oli tolle äsjalõppenud ürituse orgunni pealik, ütles, et \enquote{Kuule, mul järgmine suvi on ka mingi asi, et arvutiabi vaja, et tule}. Ja see oli aastal, siis 1990 toimunud European Nuclear Disarmament Convention, ehk suur rahuvõitlejate ja roheliste üritus. Sellega seoses tekkis meil ühte kontorisse, mis asus enam-vähem nüüdseks lõpetanud No-teatri ruumides, kusagil teisel korrusel üks PC, ma arvan, et äkki oli Sanyo. Ja selle küljes oli ma arvan, et 1200 boodine või bps-ine modem. Mingi koha peal lähevad boodid ja bps-id vist lahku, kui ma hästi mäletan.  Meie ametlik tegevus oli suhtestumine Orgkomiteega. Ja see toimus niimoodi, et sai siis helistatud Tallinnast Helsingisse kaugekõnega. Eestis oli otsevalimine meil oli selles mõttes väga vinged positsioon, Baltikumis mujal ei olnud otsevalimist. Ka Tallinnas igal ei olnud, aga meil oli olemas sest see oli ürituse jaoks oluline. Mart Siilman, kes on endine Fila\sidenote{Peeter peab silmas Eesti NSV Riiklikku Filharmooniat\index{Eesti Riiklik Filharmoonia}, mille järeltulijaks on alates 1989. aastast Sihtasutus Eesti Kontsert. Tegu oli tohutult mõjuka asutusega, mille korraldada oli kogu kontserdi-elu Eestis, sealhulgas ka levi- ja jazzmuusika ning estraad. Kuna ENSVs olid suhted kõvemast kõvem valuuta, siis oli \enquote{Fila endine direktor} äärmiselt mõjukas inimene, kelle jaoks Soome otsevalimise korraldamine kindlasti võimalik oli.} direktor ja muud sihukesed asjad, siis üldiselt juba oletan, et ta orgunnis, mida vaja. Kuidas, ei tea. Igatahes saime helistada Datapakki X.25 võrku,  X.25 kaudu oli siis võimalik suhelda ühe Rootsi serveriga, mida me kasutasime, ja üks server oli ka Kanadas. Sealtkaudu me siis nagu suhtlesime tolle ürituse orgkomiteega aga  hakkasime ka vaatama, et kuhu veel õnnestub helistada.

\question{Kuidas te \emph{bootstrap}-isite seda asja? Mida kliendi poole vaja oli, et sinna võrku saada?}

Tavaline modem ja tavaline modemiga suhtlev mingisugunegi terminaliproge. Sellega modemiga helistas, siis Datapaki liidestuspunkti, kust edasi läks, asi pakettvõrguks või X.25-ks. Ja siis sealsamas terminali peal nagu terminali peal ikka: lehekülg \emph{scrollib} ja siis seal on mingi menüü, valid mingi üks, kaks, üksteist, eks. Mingi meilboks oli seal, kus sai kirju vahetada, oli  jututubade või listide alajaotus. Aga siis  parafraseerides Heinleini, et \enquote{\emph{Have modem, will find BBS-es}}\sidenote{Peeter viitab Robert A. Heinlein-i 1958. aasta jutustusele \emph{Have Space Suit - Will Travel}.}. Loomulikult leidsime kusagilt üles ka selle, et on  BBS-id umbes samas 1989. aasta lõpus tekkis Lembit Pirnil\index[ppl]{Pirn, Lembit} esimene PirnBoxi\index{BBS!PirnBox} nimeline vist BBS, mis asus seal kusagil, kus trammid väga kõva kriginaga keerasid toona ehk  praeguse SEB vastas oli Autotranspordi Arvutuskeskus\sidenote{Asutuse täpne nimi oli Eesti NSV Autotranspordi Arvutuskeskus (ATAK)} ilmselt. Ehk tal oli seal arvuti ja me kõik alguses helistasime  sinna Pirnile sisse. Natukese aja pärast tekkis selline asi nagu HNS ehk \emph{Hacker's Night System}\index{BBS!HNS}. Ja siis kolmas oli meie Goodwin BBS\index{BBS!Goodwin} meil Suloga\index[ppl]{Kallas, Sulo} mis  ilmselt jooksis sellesama väljahelistamise liini otsas. Öösel jätsime  arvuti sisse ja kõik said sinna sisse helistada. Et kui sa tahad kuhugi sisse helistada aga liinid on kinni, siis ainuke võimalus olukorda parandada on see, et panna ise ka mingi \emph{box} püsti, eks ju. 

Sealt kusagilt tekkis siis ka Fido pool. Jällegi see sissehelistamise küsimus. Et kui meil on  võimalik teha see, et e-post ja jututoad oleksid omavahel nagu kuidagi  sünkroniseeritud eri masinates, siis pole ju vahet, kuhu me sisse helistame. Masinad vahepeal öösel või päeval käivad ja vahetavad omavahel need sõnumid ära. Fido oli selles mõttes nagu tõsiselt distributeeritud nett. See, mida nüüd räägitakse, kuidas  veeb kolm tuleb äkki nagu sarnane. 

\question{Igasugu võrgustike \emph{bootstrap}-imine on keeruline just inimeste mõttes. Selleks, et kuhugi külge minna, peaks seal olema huvitav. Selleks, et seal oleks huvitav, peaksid seal olema inimesed. Mis te näiteks seal PirnBoxis tegite, et huvitav oli?}

Eks ilmselt sai lihtsalt nagu lämisetud. Ma pean tunnistama, et ma ei mäleta, mida me  tegime, aga igal juhul väga huvitav oli. Ma oletan, et kusagilt  pääses ligi mingisugustele faili kujul sci-fi  raamatutele ja mingitele muudele laiematele uudisgruppidele, mingi hetk enam kusagilt igaljuhul liidestusid need Fidonetiga ära. Ehk et seal informatsiooni nagu liikus. Ja lihtsalt selles mõttes oli ka huvitav kirjutadagi, et vau, et traadi kaudu see kõik liigubki! See oli nii \emph{amazing} selle aja kohta. Sellest ma sain aru, et programmeerida saab ja midagi kujundada aga et saab nagu  reaalselt suhelda ka.

\question{Mis tol ajal tegi ühe BBS-i populaarsemaks kui teise? Goodwin ja HNS olid ikkagi pikalt populaarsed, kuigi PirnBox oli esimene?}

Ta oli esimene, aga ta vist jooksis mingisugust asja, mis toona oli vist vähem levinud. Meil oli, kui ma hästi mäletan, äkki  Maximus. Ja ma ei mäleta, kuidas Pirni ja Fidonetiga läks, võib-olla oli tal palju tööd teha? Kuidagi ta nagu nende noorte poiste käe alla läks igatahes. Ma ei oska öelda, miks.

Sulo oli muidugi omamoodi nagu arvamusliider, ehk selles mõttes, et tal olid  kõikvõimalike asjade suhtes oma sellised väga toredad ja  tugevad seisukohad. Mina olin ka niisuguse tutu-lutu taustaga, olles olnud muu hulgas Reaalkooli\index{Koolid!Tallinna 2. Keskkool} viimane komsomolisekretär.
Ja arvestades seda, et enne mind oli komsomolisekretär. Eelviimane oli Karl Martin Sinijärv\index[ppl]{Sinijärv, Karl Martin} selles mõttes, me  ei võtnud seda asja väga tõsiselt.

Kuidagi me sattusime seda asja vedama  kuna meil oli tänu sellele tuuma-üritusele ressurssi käes. Meil mingil hetkel tekkis igal juhul  kaks telefoniliini. Võib-olla see oli mingi aastake hiljem, kui  üritus läbi sai ja me olime juba Eesti Instituudi\index{Eesti Instituut} ruumides, veidike enne seda kui Eesti Instituut osutus tegelikult Eesti välisesinduste ja iseseisvuse ettevalmistuslavaks. Näiteks sellel hetkel, kui kuulutati välja iseseisvus, tuli järsku välja, et Jüri Luiged\index[ppl]{Luik, Jüri} ja kõik muud, kes seal mööda maailma laiali olid, et neil olid juhuslikult kaasas ka pruunid ümbrikud esitamiseks kohalikule võimupealikule küsimusega, et kas teie ekstsellents lubaks meil asutada suursaatkonda. 

Eesti Instituudis olid meil ka mingid omad arvutid, jällegi ei mäleta täpselt kelle arvutid need olid, kust nad pärit olid. Äkki olid instituudi omad, äkki olid meie omad. Me olime Suloga\index[ppl]{Kallas, Sulo} need, kes öösiti  faktse saatis. Seal mitmed toredad kolleegid, vähemasti niimoodi huumoriga pooleks, olid sügavalt veendunud, et faks ongi selline  seade, et kui sinna  peale panna paber koos kollase postitiga, kus on telefoninumber, siis on see hommikuks ennast ära saatnud, eks ju. Sest et tollased liinid olid sellised, et nad öösiti toimisid oluliselt paremini, kui päeval. 

Tingituna sellest, et seda välisühendust oli meil läbi modemi helistamise  suhteliselt piiramatult käes ja liine oli seal ka mitmeid, siis oli meil seal kaks modemiühendust. Mingil hetkel hakkas meie kaudu Fidaneti kaudu väljapoole ühenduma Läti, Läti Fidonet käis üle meie.

\question{Ma teadsin, et Vene Fidonet käis läbi meie aga et Läti ka?}

Venemaa, tekkis ka millaski jah. Oli Läti, mis oli Eesti all  mis iganes see Fidoneti järgmine selline tase oli, mingi alamühik oli Eesti all, mis oli nagu Läti tegelikult. Leedukad loomulikult ei oleks millegagi selliseni laskunud, et nad on mingisugune Eesti regioon kusagil võrgustruktuuris. Nad selle asemel kord nädalas enam-vähem helistasid ja tõid nagu ämbriga e-posti. Välja arvatud, ma arvan, et see oli Kaunase Ülikoool\index{Kaunase Ülikool} ja Leedu parlament\index{Leedu Vabariigi Parlament}, kes olid Goodwin BBS-i pointid. Seal oli hädasti vaja ja siis uhkus jäeti kõrvale. 

Lätlased käisid meil külas. Panid rahad kokku ja tõid meile,  selle eest, et on ühendus. Me saime mingisugune, ma ei tea, kakssada dollarit, mille meie investeerisime sellesse, et  ostsime endale kaks modemit. US robots\index{US Robotics}-i  HST-d, mis tegid vist kas neliteist kilo või midagi sihukest kiirust. Väga väärt aparaadid, niimoodi on lätlased panustanud  Eesti neti  arengusse.

Samal ajal Internetiga nagu ametlikku postivahetust pidas seltskond Küberist\index{Küberneetika Instituut}. Aga neil seal Mustamäel oli ikka selline suhteliselt sant jaam,  mis  krabises ilmselt rohkem, kui sidet läbi lasi. Pluss veel see, et need akadeemilised tüübid olid millegipärast koledad UNIX-i sõbrad ja kasutasid PEP-i TrailBlazer-eid\index{Telebit TrailBlazer}, mis esiteks olid 9.6 kilo, ehk mõttetult aeglased ja teiseks kuidagi nende post-sovieti liinidega HST suutis nagu paremini oma sidet vilistada. Me olime sügavalt veendunud, et nad olid ka oma reaalselt võimekuseltpikki seansse ja kiirust üleval pidada  oluliselt paremad. 

\question{Ja kui sa ütled \enquote{liin}, siis sa mõtled ikka telefoniliin?}

Jah, liinid olid kõik tavaline analoog, kus otsa käis kettaga telefon. Ja keskjaamas, ma arvan, et kui sa ikka numbri valisid, siis kusagil  mingisugused  releed jooksid kontakte mööda ringi. Kui modem valis oli kuulda  klõbinat, tal oli seal mingisugune relee või herkon või mis iganes seal sees oli, millega ta katkestusi tekitas. Kõik oli reaalselt selline elektriline, sellepärast ma ka kujutan ette, kuidas andmeside  tegelikult toimub. Aga  kuidas on võimalik panna läbi ADSLi selliseid andmemahte võrreldes sellega, et me panime enam-vähem samasugust asjast läbi neliteist kilobaiti, nüüd panevad mingid vennad kümme mega, hästi arusaamatu. Või wifi täpselt samuti. Ma ei kujuta ette kuidas see põhimõtteliselt saab üldse toimida. 

\question{Kas kujundamise rida käis sul kõige selle kõrvale?}

Seis kuidagi sinna jah, selles mõttes kõrvale, et ma seal mingis umbes samas ajajärgus sattusin seltskonda, kellel oli arvuti ja printer ja vaja  midagi trükkida. See oli poistekoor\index{RAM-i poistekoor} ja Venno Laul\index[ppl]{Laul, Venno}\sidenote{Venno Laul asutas 1971. aastal Riikliku Akadeemilise Meeskoori juurde poistekoori ning oli kuni 1990. aastani selle kunstiline juht ja peadirigent.}. Neil oli esimene printer PostScript printer, mille ostus ma olen osalenud. See oli kas Tectronicsi või millegi sellise A4 formaadis kolmesaja DPI-ga laserprinter. Aga see oli PostScript printer. Kui sina sai Venture külge ühendatud, siis \ldots Põhimõtteliselt kõik toonased kujundusprogrammid olidki niimoodi, et sul ikkagi enam-vähem bitmap fondid olid arvutis, eks ju. Kuidagi need bitmapid saadeti juhet pidi printerisse ja kõik see mõtles hästi pikka aega. Aga PostScript tegi nii, et said selle lehekülje sinna nagu lasta joonistada, nagu programmi saata preinterisse ja printer oma tarkusega joonistas. Mis oli kunagi  Adobe ja Apple vendade poolt väga mõistlik valik, olles Silicon Valleys kokku saanud ja  otsustanud, kes mis osa maailmast vallutama hakkab. See oli neil väga targasti jagatud selles mõttes, et tõesti, kui sul on kontor, ilmselt igal vennal ei ole printerit ja selleks, et inimesed saaksid printida, võiks olla printer ka tark. Väga spetsiifilise tark, et ta suudaks joonistadanii-öelda lehekülje endal mälus valmis ja siis välja trükkida. Ja kusagil samal ajal ma sattusin kokku ka, ma ei tea, ka see toona oli Sirp või Sirp ja Vasar veel, Sirbiga\index{Sirp}, mis oli üks esimesi ajakirjandusväljaanded, mis läks digitaalse töövoo peale. On raske öelda, mis see täpselt esimene oli, aga igatahes Sirp läks ka. Alguses protsess oli umbes selline, et toimus tinaladu. Tinalaoga tehti kas siis üks tõmme paberi peale ja see siis vist pildistati üles. Põhimõtteliselt sellel hetkel ofset-trükk toimus veel põhimõtteliselt läbi tinalao. Ja nüüd, kui oli võimalik minna üle selle peale, et arvutist saaks välja trükkida, siis see oli ikka mega raju.

\question{Põhimõtteliselt PostScripti printerist lasti kile peale trükitavad asjad, eksole?}

Põhimõtteliselt jah, ja peegelpildis. Üks asju, mis ma mäletan, mis me Suloga\index[ppl]{Kallas, Sulo} koos tegime, või Sulo tegi, kui me Eesti Instituudis\index{Eesti Instituut} olime, oli PostScripti \emph{pre-header}. PostScripti puhul sageli oligi, et programmiga tuli kaasa mingisugune koodijupp, mis siis kirjeldas sellise nagu programmeerimiskeskkonna, defineeris täiendavad funktsioonid ja mingisugused muud oma käsud. \emph{Pre-header}? Preambul oli vist. Ja siis tuli  kood ise ja lõpus mingisugune koristusfunktsioon või midagi, mis välja trükis. PostScript oli tore selles mõttes, et ta oli nagu \emph{open source} selles mõttes, et mitte nii-öelda vaba tarkvara, aga ta oli nagu nähtava lähtekoodiga. Ehk oligi võimalik võtta sama Venture ette. Ta kusagilt laadis selle preambli, see oli tekstifail, seda oli võimalik muuta. Ja oli võimalik kirjutada selline transformatsioon sinna ette, mis keeras pildi peeglisse. Ja siis ma mäletan, et sellega seoses Sulo PostScripti preambul õnnestus meil maha müüa Avita\index{Avita kirjastus} kirjastusele Tiit Aunastale\index[ppl]{Aunaste, Tiit}, kes hakkasid tegema kooliraamatute kirjastamist. Ja ma ise sattusin sealt siis mingi aeg hiljem  Avitasse tööle, mis oli ka mingi a'la üheksakümmend üks, asjad liikusid väga kiiresti tol ajal kõik.

\question{Jah, sest umbes viis aastat hiljem, mina mäletan sind Eesti vaieldamatu autoriteedina teemal, kuidas arvutist värviline asi trükki saada}

Eks ma olin praktiseerinud seda piisavalt. Me olime teinud ilmselt mingisuguseid nii-öelda haltuura otsasid sellesama Ventura peal. Igast muud softi oli ka, näiteks Arts \& Letters, millega oli võimalik panna tähti ümber ringi käima. Toona, kui kõik asutasid endale aktsiaseltse ja börse, siis kõigil neil oli vaja endale pitsatit. Selline väga tore asi, et oli võimalik arvutist teha pitsatit ja ei pidanudki kujundajatädi lõikama välja mingisuguseid fotolao tähti ja neid  ise  kleepima, et me saime arvutist ühe matsuga põhimõtteliselt pitsati valmis teha. See oli nagu meeletu innovatsioon, eksju. 

Sirbi  toimetus andis välja mingit jutulehe nimelist asja, mille  \emph{layout}-i vis mina tegingi tegelikult läbi pika aja. Ja nii edasi, et jah, ma sattusin kuidagi selle ala peale. Sai käidud vaatamas, kuidas Helsingis Helsingin Sanomat\index{Helsingin Sanomat} tehakse, kus olid mingisugused miniarvutid ja   rohelised terminalid. Seal oli ka Linotronic\sidenote{Mergenthaler Linotype Company poolt toodetud kõrgekvaliteediline printer. Tegu oli kalli seadmega, kuid võimaldas trükkida resolutsioonis kuni 2540 dpi.}, millega trükiti põhimõtteliselt veergu välja fotopaberi peale. Fotopaber oli kolmkümmend senti lai ja sinna lasti välja  üks ajaleheveerg. Ajaleheveerud lõigati kääriga sealt välja ja pandi sellise suure maketi peale, mis oli mingisuguse vaha või millegagi koos. Rastreeritud fotod pandi veergude vahele, niimoodi pandi Helsingin Sanomat kokku ja tulemus saadeti faksiga trükikotta. Faks ei olnud loomulikult see faks, mis tavaline faks, aga mingisugune selline \emph{industrial grade}ajalehe formaadis kõrgeresolutsiooniline,  mis siis  skännis ühelt poolt sisse ja teisel pool, siis  ilmselt trükis filmi välja ja siis filmiga valgustati trükiplaadid. 

Sealt elu näiteks, et siin valdkonnas on nagu ikka väga pull tegevus, nii ma nagu sattusin selle peale.

\question{Räägi, mis see \enquote{pull} oli? Kas tehnoloogiline keerukus või see, et protsessil oli palju samme või veel midagi?}

Kõige huvitavam on tegeleda asjadega, millega teised ei tegele parasjagu. Või siis, teistpidi, mingi asi, mis toimib nagu kuidagi teistmoodi, kui sa oled siiamaani  arvanud, et asjad toimivad. \enquote{Aa, ongi niimoodi, et ma saan seda teha, okei?}. Ja niisama Pascalis programmeerida, ma ei tea, seda õpetati koolis, et see ei olnud nagu midagi nagu väga huvitavat. Aga kuna mul oli ilmselt  parasjagu olemas selles mõttes selline taust, kus mul oli arusaamine sellest, kuidas need asjad töötavad, et mis seal masinates on,  siis ma suutsin neid asju panna ka efektiivsemalt tööle. Ehk siis see, kuidas siduda kirjastuse kontekstis see, et sul on  küljendusprogramm ja sul on tekstitöötlusprogramm. Tekstitöötluseks oli sul WordPress või WordPerfect (Perfect? Prefect? Ford Prefect ja Word Perfect!\sidenote{Peeter viitab tõenäoliselt Douglas Adamsi loodud tegelaskujule, mitte omaaegsele populaarsele automargile.}),  me kül kasutasime rohkem Volkswriter-nimelist asja, see oli kuidagi sattunud sinna seltskonda. Just see, et sa said lasta tekstitoimetajal võtta ette selle WordPerfecti faili, tema toimetab seal mingid asjad ära, seal kuidagi on juba sees see märgendus, mis ütleb ära, mis on  stiilid. Siis sa loed selle oma küljendusprogrammi tagasi ja sul on põhimõtteliselt võimalik teha teksti korrektuuri ilma et,  keegi oleks nagu kallima arvuti või keerulisem programmi taga. See oli see, mis meil õnnestus väga efektiivselt kuidagi juurutada. 

Kusagil sealsamas enne rubla-aja lõppu, kas siis üheksakümmend kaks alguses, tekkis Prisma Printi\index{Prisma Print} tekkis esimene Linotroni ehk filmiprinter. Siiamaani nagu kuuesaja dpi-ne laser oli juba ikka nagu väga hea ja nüüd tekkis järsku tuhande kahesaja dpi-ne filmiprinter. Prismas  alumisel korrusel olid Crosfieldi suured trummelskannerid\sidenote{Crosfield Electronics oli Briti firma, mille toodetud skännereid peetakse siiani ühtedeks parimateks, mis iial tehtud.} millega sai teha värvi lahutusi, need tehti filmi peale juba rastrisse. Ja endiselt kogu selline montaaž toimus niimoodi, et sul olid siis teksti kile ja pildi kile või film ja need siis valgustati või füüsiliselt lõigati kuidagi kokku. 

Kuna mul oli nagu keskmisest parem ettekujutus sellest, kuidas need süsteemid töötavad, siis enamasti,  kui mina jõudsin oma failidega kohale, siis kõik seal nagu nägid vaeva, kuidas QuarkExpressist midagi välja printida muud sihukest. Minul olid kaasas oma flopid ja võib-olla hiljem mingi magnet-optilised või mingid \emph{syquest}-i kettad (vist mitte syquest, see on üks väheseid asju, mida mul pole kunagi olnud) ja ma tulin sinna vahele, et \enquote{laske minu omad vahepeal välja, ei viitsi oodata teiste järel}. Tehti ära. Põhjus oli selles,  et minu asjad käisid tõesti kiiresti läbi. Sest ma kujutasin kujundust tehes ette seda, kuidas see PostScriptiks läheb. Minu jaoks nii küljendusprogramm, kui, ütleme, seesama Ventura või mis seal hiljem olid seal mingid PageMakerid ja Quarkid ja samuti graafikaprogrammid nagu Illustrator või Freehand, oli tegelikult programm, mis aitas mul visuaalselt valmistada ette PostScripti. Ma põhimõtteliselt teadsin, kuidas see asi koodis välja näeb, ma võin võtta faili ette ja näha, kuskohas miski asi on. Ja tänu sellele ma teadsin, kas seda, mis on printeri jaoks keerulised asjad ja ma oskasin neid asju lihtsustada ja mitte kasutada asju, mis on liiga keerulised. Sest et see prose, mis seal taga oli,  oli ikka suhteliselt vaene protsessor. Kui sa suudad nagu tekitada olukorra, kus sa mingile programmile annad ette olukorra, kus on tsükkel tsüklis tsüklis (tänapäeval tuleb sinna otsa veel SQL-i päring), siis üldiselt see asi nagu on suhteliselt ebapädev.  

Arvuti taust ja siis kuidagi kokku sattumine selle kujundusega on tekkinud selle, et kuidagi ka sõprade hulgas on suur hulk igasuguseid disainereid ja, ütleme, kõike kes tollest ajast on tegevad olnud, neid kõiki ma kuidagi nagu tunnen sellest samast Prisma Prindi väljatrükijärjekorrast. 

Sealt ma sattusin edasi Uniprinti. Alustuseks töödades andetu disainerina aga siis leides tasapisi võimalusi selliseid, noh, nagu vähem disainimaid asju teha, kus mängis rolli just see, et ma suutsin võtta, et mingi Eesti Näituste näituse kataloogi andmebaasi. Akki oli Microsoft Accessis? Ja sealt genereerida väljundiks tekstifaili, mis oli juba märgendatud stiilidega mis oli võimalik lihtsalt lugeda  küljendusse sisse. Jällegi asi, mida väga palju sellises \emph{desktop publishing}-us ei olnud tavaks kasutada on  see, et sa  valmistad ette stiilid ja siis tekst nagu lihtsalt kasutab neid stiile. Et sa tegelikult seal kohapeal midagi tegema ei pea.

\question{Põhimõtteliselt CSS?}

\emph{Right}, täpselt. Põhipõhimõtteliselt nagu CSS, ainult et tekst ja paber ja vanasti. Need kõik töötavad täna siiamaani niimoodi, aga see oli selline meetod, kuidas nagu mina tegin. See  tekitas võimaluse teha selliseid huvitavaid töövoogusid, ehk et minul oli põhimõtteliselt  oli see andmebaas käes, tüdrukud näitustel müüsid seal järgmisi bokse maha ja tegid ürituste korraldust ja tõstsid asju ümber ja täiendasid firmade ja parandasid telefoninumbried ja mida iganes. Mina mingisugusel hetkel trükkisin \emph{layout}-i välja, viisin neile ja nemad tegid andmebaasi korrektuuri, mina nii kaua joonistasin logosid puhtaks, ja siis niimoodi ma õppisin. Ma pean tunnistama, et ma olen andetu disainer. Aga nende tehniliste protsesside ja töökorralduste poole pealt ilmselt teadsin oluliselt rohkem, kui keegi teine jah, tollel hetkel. 

Pärast ma sattusin Uniprinti ja oli vaja hakata tegema oma reprot ja \ldots

\question{Aga see tuleb ju kenasti selle juure, mida sa praegu tundud tegevat? Millega sa praegu üldse tegeled?}

Jah, on igasuguseid seoseid. Kui ma veel trüki alal tegutsesin, oli mul mingil hetkel ilmselt liiga palju vaba aega tänu tänu sellele, et ma olin suutnud oma tööd ära optimeerida. Ja tänu sellele, et Uniprindi pealikud Sirje ja Andrus Reinsoo\index[ppl]{Reinsoo, Sirje}\index[ppl]{Reinsoo, Andrus}, kes on just mõlemad lahkunud\sidenote{Intervjuu Peetriga leidis aset märtsis 2019.}, et nemad jätsid mulle ka piisavalt vabadust tegeleda ja nii ma käisin ja kolasin Ameerikas konverentsidel. See oli ajal, kui enamasti oli suhtumine, et \enquote{Mis mõttes väljamaa ja konverentsid? Me oleme Eestist ja teame väga hästi}. Mina käisin, olid mingisugused \emph{cyber publishing} seminarid, mis just olid seotud selle trükipoole alaga, mis mulle huvi pakkus: plaaditrükkida ja kõik muu. Ja vist aastast 1994. olen ma sattunud  kirjutama.

See sai alguse niimoodi, et koolivend ja paralleelklassivend Peeter-Eerik Ots\index[ppl]{Ots, Peeter-Eerik} oli Äripäevas ajakirjanik ja kirjutas mingisugused tehnoloogiateemalisi lugusid. Aga minul reaalikana hakkas nagu mõnevõrra piinlik, kirjutatu ei tundunud olema piisavalt pädev. No see oli ka kõik sellest post-BBS-i ajast, ma olin kindlasti ka võrdlemisi \emph{opinionated} noor inimene. Eelarvamuskindel,  omade kindlate eelarvamustega. Ma kirjutasin teisele Peetrile paar lugu ette, et avalda parem neid, vähemasti  kirjutatud kellegi poolt, kes enam-vähem saab aru, et millega tegemist on. Peeter ütles, et kuidagi nagu väga imelik, et võiks ikka minu nime alt ka hakata avaldama ja saaks honorari ka maksta. Nii ma sattusin Äripäeva kirjutama, eks ma sattusin sealt igale poole mujale kirjutama. Arvutimaailmad\index{Ajakiri!Arvutimaailm} ja kuhu iganes. 

Ja oli lugu selles, et kuna ma olin põhimõtteliselt nagu ajakirjanik, siis mul oli võimalik möllida ennast igale poole konverentsidele, mis muidu maksid mingi paar tuhat taala (röögatult kallis tolle aja mõttes) ma sain põhimõtteliselt  ajakirjaniku passiga sisse. 

\question{Tehnoloogia tehnoloogiaks, aga mis sind ikkagi kirjutamise juures paelus?}

Kirjandite kirjutamisega sain suhteliselt nagu hakkama juba kooliajal. Minu esimene avalikustatud töö oli Pikri\index{Ajakiri!Pikker} mingisugune noorte huumorivõistluse võidutöö. Ilmselt ma olin midagi lugenud ka,  selline sõprus sõnaga oli nagu olemas, ma olin juba teinud  kooliga omavalitsust ja muud muud sellist, siis ma olin ilmselt niisugune parasjagu jutukas ka muidu eks ju. See kirjutamine ei olnud keeruline. Võib-olla meeldis mulle ka õpetada,  läbi kirjutamises on võimalik teisi inimesi õpetada ja panna midagi teisiti tegema, eks ju. Pluss, klassikaline küsimus, miks ma olen väga tänulik Peeter-Eerik Otsale on see, et ta tegi midagi valesti. See tüüpiline küsimus interneti puhul, et \enquote{\emph{wait, somebody is just wrong on the Internet!}}. Kirjutamine ilmselt sai alguse sellest, et \emph{somebody was wrong} ja mul oli vaja kaitsta oma  seisukohta, ja noh, Reaalkooli au loomulikult ka, eks. 

Sealt sai asi alguse ja edasi inimesed ütlesid, et ma võiksin neile kirjutada. Noh, ja  olles õppinud, palju asju, ma siiamaani ei oska \enquote{ei} öelda. Ilmselt mingi edevus, ka niisugune, et \enquote{oh, keegi tahab, et ma midagi teeksin!} Ja eks ma siis kirjutasin ja hetkel, kus tol ajal nagu suuresti midagi selle valdkonna oma ei olnud kirjutatud, siis see kõik nagu kuidagi hakkas toimima. Mingil hetkel seal kusagil  üheksakümmend kas mingisugune viis-kuus, kui Avo Raup\index[ppl]{Raup, Avo}, kes tegi Raadio 2-s saadet \enquote{Võrgutaja} kutsus mind, kuna ma olen juba kirjutanud sellest asjast ja tuntud inimene, saatesse külaliseks. Ja kuidagi hakkas meil see asi niimoodi klappima, et minust sai resident-saatekülaline. Esimene inimene, keda ma sattusin üldse esimesena üksinda tegema  (Avo oli haigeks jäänud või midagi) oli Abobase Systems-ist\index{Abobase Systems} Kaido Saarma\index[ppl]{Saarma, Kaido}. 

Kusagil sealsamas üheksakümmend üheksa tuli mingil hetkel minu juurde Sarvik\index[ppl]{Sarvik|see{Sarv, Henn}}\index[pp]{Sarv, Henn}\sidenote{Legendaarne it-mees Henn Sarv.} ja ütles, et Kukust kas siis Lang või Tiido või keegi oli öelnud et on vaja teha arvuti saadet. Istusime sealsamas Uniprindi lähedal Pärnu maanteel Westmani poe vastas keldris ühes mingisugune Hollandi õlletoas ja mõtlesime välja, et võiks olla selline asi nagu Tehnokratt. Ja hakkasime tootma raadiosaadet. Juba esimesel hooajal me sattusime Kukus kokku tegelastega, kellel oli mõte ETV-sse ka midagi teha\index{Eesti Rahvusringhääling!Eesti Televisioon}. \emph{Whatever}, toodame! Nii ma sattusin telesaatesse olema korraga enam-vähem toimetaja, saatejuht ja (mis muidugi mõnevõrra üllatuseks tuli) pidin panema kokku ka montaažiriba.

\question{Ja nüüd sa oled tagasi ringiga\ldots} 

Kas nüüd tagasi või edasi. Praegu ma olen Zone-s\index{Zone}, mis on täiesti juhuslikult ajaloos esimene kord, kus ma olen töötanud mingit otsa pidi IT-firmas. Ma olen vahepeal olnud reklaamiagentuuris, küll digi-tiimi juht, võiks öelda, et ka natuke IT poole, aga ta oli ikka nagu reklaamiagentuur, eksju. Ja siis trüki poole peal ja kus iganes, ma olen koolitanud ja kõike muud teinud aga selles see on esimene kord, kus mingid IT-tüübid mõtlesid, et palkaks ka Marveti tuututama siia. Ametlikult mu  müts on seotud turunduse ja kommunikatsiooniga. Aga ma tegelen ka selle poolega, et kui on  keegi ütleb, et midagi ei tööta ja kõik ütlevad, et ei noh, aga töötab, siis kuidas saada aru, et mida inimene tegelikult tahab. Äkki tal on õigus, et tal ei tööta. Äkki tema kontekstis see asi, mida meie arvame, et me oleme nunnutanud ja silunud ja teinud maailma kõige-kõige paremaks. Aga äkki on olemas võimalus, et see tema kontekstis ei tööta. Ja täiesti üllatavalt tuleb välja, et kui sul on piisavalt keerulised süsteemid, siis  neid olukordi, kus on mingi asi, mis vaatamata kõige suurematele ja parematele püüdlustele vajaks ikkagi teisiti toimima panemist või siis võib-olla seletamist,  et seda on uskumatult palju. 

\question{Küll sa selle turunduse asja ka ära optimiseerid, nagu sa kõik asjad ära oled optimiseerinud!}

Jah, ma üritan. Mul see lootus on natuke teistpidine. 

Kunagi Andres Kulli\index[ppl]{Kull, Andres} ja Kroonpressi\index{Kroonpress} seltskond tuli küsima, et kuidas panna reklaami ajalehte. Mina rääkisin, et on olemas PDF. Teeme parem nii, et kõik nad teeksid korraliku PDF-i, leheküljendaja tõstab selle küljendus-softi sisse ja kõik töötab. Kull selle peale, ikkagi suure trükikoja juht, ütles \enquote{Väga hea, siis  me otsustame niimoodi. Kõik peavad saatma PDF-ina asju Postimehesse}. Ja üllatus-üllatus, nii läkski. Mu enda pool selle kõige juures oli see, et ma olin olnud pikka aega Prismas ja muudes reprodes selline majasõber, kes sageli tolkneski seal ja üritas endale tegevust leida ja saada aru, kuidas need asjad käivad. Kuni, kaasa arvatud see, et  Eesti esimese Linotronicu me oleme pulkadeks lahti võtnud ja midagi seal jootnud, sest ta  otsustas lõpetades parasjagu töö ja oli vaja midagi. Ja kui see PSDF-i asi hakkas tulema meile endale majja ja ma olin näinud, et millist roppu vaeva näevad kõik minu  sõbrad, kes on sellised repropealiku või  sellise repro tehniku rollis,  kehvasti ette valmistatud originaalidega. Ja siis ma mõtlesin, et no mina selle ussipurgi avamist küll enda peale võtma ei hakka (tänapäeval räägitakse rohkem surströmmingust, kui Pandora laekast). Et ainuke asi, mis ma saan teha, on see, et ma õpetan kliendid paremini originaale saatma. Mis loomulikult tundus äärmiselt lihtne. See on ju nii lihtne teha, kui ma lihtsalt ütlen teile, et seal on vaja nagu mõned linnukesed panna, et siis see kõik lähebki niimoodi. Aga tuleb välja, et ei. Ma olen õppinud, et on päris kõva pingutus aru saada,  mida teised inimesed teavad, mis on nende taust. Ja siis viia nad selleni, et nad saaksid aru millestki, millest sina aru saad seejuures võimalust mööda ise mitte liigselt kas siis masendumata või siis nende peale kurjaks saamata. Nii ma sattusingi õpetama  kõiki neid Pagemakereid ja InDesigne ja Photoshopoe ja kõiki muid asju just sellise töökorralduse poole pealt. Ja hetkel ma lihtsalt Zones näen, et kui vaadata kogu seda veebiga seonduvat, siis ilmselt tuleb selle kõigega proovida nagu rohkem edasi minna. 