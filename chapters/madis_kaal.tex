\label{cptr:mast}
\index[ppl]{Kaal, Madis}
\index[ppl]{Mast|see{Kaal, Madis}}

\question{Kuidas arvuti Saaremaale sai?}

Arvuti ei saanudki Saaremaale. Minu esimene kokkupuude päris arvutitega oli 
Rahvamajanduse Saavutuste Näitusel\sidenote{Tänapäeva mõistes oli tegu 
messikeskusega, kus ajutistel või püsinäitustel demonstreeriti 
kas liiduvabariigi (nagu Tallinnas asunud näituse puhul) või kogu Nõukogude 
Liidu majanduslikku võimekust. NSVLi Rahvamajanduse Saavutuste Näitusest arenes 
välja Eesti Näituste Messikeskus.}, praeguses Pirita näitusehallis. Käisin 
seal koos oma emaga. Ühes nurgas olid üles pandud 
terminalid, mida manageeris kaks imeilusat tüdrukut. Seda siis ajaloolisest perspektiivist, tõenäoliselt oli tegemist üsna keskmiste 
operaatoritega, aga siis tundusid nad imeilusad ja targad. Terminalide peal oli 
nõukogudeaegne venekeelne raamatukogude otsingu andmebaas. Terminalid ise olid ka 
venekeelsed. See oli esimene kord, kui ma reaalselt nägin, et ekraanil olid 
tähed ja klaviatuuril sai kirjutada. 

\question{Mis aastal see oli?}

Arvatavasti 1983. Ja need terminalid jätsid kustumatu mulje. 

\question{Kas pärast seda tekkis sul selge soov terminalide 
juurde pääseda?}

Pärast seda tekkis väga selge mõte, et see asi huvitab mind. 
Seejärel sattusin Tartusse ja ostsin sealt venekeelse 
raamatu \enquote{Programmeerimine keeles PL/I\index{PL/I}} ning lugesin 
seda. Ma ei teadnud arvutitest veel midagi, aga tasapisi hakkas selgeks saama, 
misasi on programmeerimine ja näiteks \verb|for|. See oli mingi imeline struktuurkeel, mitte päris vene, 
vaid kõlas nagu piraatversioon.

Järgmine kord nägin arvuteid Tehnikaülikooli\index{Tallinna 
Tehnikaülikool}, tolleaegse TPI\index{TPI} lahtiste uste päeval, kus me käisime 
pinginaabriga, kellega koos pärast ka kooli sisse astusime. 
Meile tehti ekskursioon automaatikateaduskonna kõigis 
kateedrites\index{Tallinna Tehnikaülikool!Automaatikateaduskond} neljal 
korrusel ja mõnes kohas olid arvutid. Mäletan selgelt, et Indrek 
Saul\index[ppl]{Saul, Indrek}, kes oli minu meelest sel ajal tudeng ja hiljem 
kinnisvaraärimees, näitas meile analoogarvutit. Sellega
sai analoogpingete ja skeemiga diferentsiaalvõrrandeid lahendada.

\question{Vanasti sihiti ju õhutõrjekahureid analoogarvutitega.}

See masin võis täiesti olla sedalaadi projekti osa. Igatahes mul tekkis kindel soov seda valdkonda
õppima minna, aga pinginaaber veenis mind ümber, et lähme parem 
raadiotehnikasse, ikkagi sama maja.

\question{Kas esimest korda arvuti nägemise ja ülikooli sisseastumise vahele jäi veel 
midagi arvutitega tegelemise mõttes?}

Ainult see üks raamat. Otsus arvuteid õppima minna sündis esimesel korral ja raamat tuli 
pärast seda. Ainuke imelik asi oli otsus raadiotehnikasse 
minna, aga selle vea parandasin ruttu ära. Ülikooli teise korruse otsas oli arvutussaal, kus oli kaks või 
kolm SM-4\index{SM EVM!SM-4}. Need olid PDP-11\index{PDP-11} vene 
versioonid. Pärast seda, kui sain aru, kuidas sinna sisse saab, ma enam 
tundidesse ei jõudnud. Ja kuna olin maalt tulnud poiss ja raha ka üldse ei olnud, 
käisin lihtsalt kõik kateedrid läbi ja küsisin iga ukse vahelt, kas neil on tööd anda. Raadiotehnika kateedris\index{Tallinna 
Tehnikaülikool!Automaatikateaduskond!Raadiotehnika kateeder}\label{sisu!mast_raadiotehnikas} oli, 
mind võeti sinna laborandiks tööle ja nii see läks. Kool jäi pooleli, kateedrisse
jäin seitsmeks aastaks paika.

\question{Mitmendal kursusel kool pooleli jäi?}

Esimesel kursusel. Algul olin raadiotehnika kateedris laborant ja pärast 
tehnik. Sattusin tuppa, kus olid väga toredad inimesed: Mart 
Palmas\index[ppl]{Palmas, Mart}, kes õpetas mulle peaaegu kõike, mida ma 
programmeerimisest tean, ja Villem 
Vannas\index[ppl]{Vannas, Villem}, kes praegu töötab Datelis\index{Datel}. Tema 
õpetas mulle enam-vähem kõike, mida ma rauast tean.

\question{Siis ei jäänud ju haridus pooleli.}

Formaalselt siiski jäi. Tol ajal oli
laborant rohkem nagu abitööline. 
Parandasin seda, mida vaja, aitasin seal, kus vaja. Mu esimene töö oli 
kolikamber tühjaks tõsta.
Algusaegadel oli üsna suvalisi projekte, hiljem tekkis
suund kommunikatsiooni poole, mis tundus mulle sel ajal huvitav. 

1990. aasta paiku tekkis Eestis 
mitu huvitavat suunda. Kõigepealt hakkas tulema 
personaalarvuteid. Sinnasamasse, kus oli kunagi SM-4 arvutiklass, tekkisid 
personaalarvutite klassid. Neid oli mitu tükki ja erinevate portsudena 
toodi Austraaliast MicroBeesid\index{MicroBee}\sidenote{1982. aastal 
Austraalias algselt komponentide komplektina müügile tulnud koduarvuti. Tuntud 
huvitava videolahenduse ja patareitoitel mälu poolest, mis 
võimaldas arvutit teisaldada mälu seisu kaotamata.}. Kuskilt tuli terve klassi jagu MSXi 
arvuteid\index{Yamaha MSX} ja siis mõned 
Robotronid\index{Robotron}\sidenote{Robotron (originaalis VEB Kombinat 
Robotron) oli Ida-Saksamaa suurim arvutitootja.}. 
Raadiotehnika kateedris oli juba siis, kui mina sinna sattusin, olemas 
Apple II\index{Apple II} ja mõned aastad hiljem tekkis sinna IBM 
PC\index{IBM PC}. See oli omapärane kogemus. Apple II peal olid 
harjunud, et lülitad sisse ja pilt on ees. IBMi sisse lülitades ei juhtunud midagi. Ühel hommikul tööle tulles vaatasin, et uus 
arvuti, ja lülitasin sisse. Midagi ei juhtunud. Ootasin natuke aega ja lülitasin välja, ise 
tegin näo, et midagi pole toimunud. Hiljem selgus, et masin tegi \emph{self 
test}'i. Seal oli tublisti mälu sees ja testimine võttis palju aega -- ma ei 
suutnud nii kaua oodata. 

\question{Midagi pidi see ju ekraanil senikaua näitama?}

IBMil oli roheline long-fosfor\sidenote{Katoodkiirtel põhinevates monitorides suunati laetud osakeste kiir fosforühendiga kaetud ekraanile. Kasutatud ühendi tüübist sõltus nii elektronkiire mõjul tekkinud värv kui ka see, kui kauaks ekraan peale kiire edasi liikumist helendama jäi. Selle viimase järgi liigitataksegi ekraanides kasutatavaid fosforühendeid  \enquote{pikkadeks} ja \enquote{lühikesteks} (ingl. \emph{long} ja \emph{short}).} monitor, mis läks tükk 
aega käima, ja ma ei jõudnud esimese \emph{boot}'imise ajal ära oodata, millal 
midagi toimuma hakkab.

Üheksakümnendate paiku tekkis meile tuhande kahesajane modem, mis läks 
PC sisse. Sel ajal olid just tulnud esimesed BBSid ja umbes samal ajal otsustas TPI 
automaatikateaduskond\index{Tallinna Tehnikaülikool!Automaatikateaduskond} 
ehitada arvutivõrgu. Toodi kohale viiesajameetrine kaablirull 
kollast sõrmejämedust Etherneti kaablit ja umbes kümmekond komplekti 
kobakaid kaabli peale, mille külge käis teine jäme kaabel, 
mis läks võrgukaardi taha. See oli nagu esimene Etherneti tehnoloogia. 
Mäletan selgelt, et meile toodi ainult kaabel ja kobakad, ei mingeid tööriistu, pistikuid ega terminaatoreid.

Kateedris oli sel ajal eterniiditahvlitest lagi, mille peale me selle Etherneti kaabli tõmbasime. Et kobakad külge saada, tegime
naaskliga kaabli kesta sisse augud, ajasime nõela läbi ja ühendasime
arvutite külge ning tinutasime otsa terminaatorid ja takistid. 

\question{Tarmo Mamers\index[ppl]{Mamers, Tarmo} rääkis, kuidas te PC ja Maci 
vahele traati vedasite. Kas too kaabeldamine oli enne või pärast seda?}

See oli meil kahe PC -- sellesama raadiotehnika kateedri PC ja Tarmo oma -- vahel, Tarmol oli 
veidi vägevam AT arvuti. Ühendasime need 
kaabliga ja tegime väikese 
\emph{chat}'i programmi, et teineteisega suhelda. 

Arvutivõrk tekkis sellest hiljem. Tehnikaülikooli toodi Novell 2.15\index{Novell} 
server, mille ma installisin ja mis oli üks esimesi väheseid asju, millel oli manuaal 
olemas, nii et kõik oli justnagu ametlik. Novelli serveri peal panin käima Pegasuse 
Maili\index{Pegasus Mail}-nimelise asja, kuhu külge kirjutasin \emph{gateway}, 
millega sai UUCP meili, mida toimetati Küberi 
majja Soomest (ma ei mäleta, kas Soomest siiapoole lükates või siit 
üle telefoniliini tõmmates). Tõmbasime selle oma pisikese modemiga 
Tehnikaülikooli majja ja jagasime kasutajate vahel laiali.

\question{Siin tundub jälle suuremat sorti lünk olema selle vahel, kuidas sulle 
hakati programmeerimist õpetama ja kuidas sa naaskliga kaablit torkisid ja 
\emph{gateway}'sid programmeerisid.}

Mõned aastad tuli õppida asjade kirjutamist lihtsalt erinevaid asju tehes ja ehitades, aega katsetamiseks oli palju. 
Olin noor inimene, peret polnud ja praktiliselt elasin raadiotehnika 
kateedris\index{Tallinna Tehnikaülikool!Automaatikateaduskond!Raadiotehnika 
kateeder}. Meil oli seal omamoodi seltskond: arvutussaali 
kamp, Tarmo\index[ppl]{Mamers, Tarmo} kohe kõrval sama 
koridori peal ja mina üleval raadiotehnikas. Vana kooli mees 
Lõvi\index[ppl]{Lõvi} oli kõrvalkorpuses ja käis aeg-ajalt Apple II peal 
oma projekte arendamas.

\question{Kas meetodiks oli siis peamiselt katsetamine, mitte 
manuaalide tudeerimine?}

Manuaale ega dokumentatsiooni ei olnud üldse. Riiklikul 
tasemel tarkvara varastamise programm pakkus küll ägedat tarkvara, aga 
enamasti ilma dokumentatsioonita. See oli nagu infovaakumis 
tegutsemine ja disassembler\sidenote{Programm, mis teeb masinkoodist 
oluliselt loetavamat Assemblerit.} oli justkui sõber.

\question{Keegi pidi sulle ju ometi ütlema, et selline asi nagu disassembler 
on olemas.}

Jaa, seda tegid head vanemad kolleegid, kes hoidsid kätt ja 
juhendasid. Lõviga\index[ppl]{Lõvi} tegutsesime pikalt koos, temal oli kindlasti 
väga suur mõju minu arengule. Aga see lünk, kuidas ma BBSideni 
jõudsin, sai täidetud nii, et mul oli raadiotehnika kateedris\index{Tallinna 
Tehnikaülikool!Automaatikateaduskond!Raadiotehnika kateeder} 
arvuti, mille sees oli modem ja millega sai helistada. Lähim BBS
asus Küberneetika Instituudi otsas, kus tollal asus 
Proekspert\index{Proekspert} ja kus nüüd on Tehnopoli kontor. Andrus Suitsu\index[ppl]{Suitsu, Andrus} 
oli BBSi mees, käisin tema juures oskusteavet ja tarkvara 
hankimas. Panin algul BBSi ja peatselt pärast seda ka Fido, algul vist
\emph{point}'i, ja käitasin seda üsna mitu aastat. 

\question{Miks sa seda tegid?}

Huvist kommunikatsiooni vastu.

\question{Kas sa mõtled kommunikatsiooni masinate või inimeste vahel?}

Mõlemat. See moment, kui täielikust infopuudusest saab järsku täielik 
infovabadus, on väga ergastav. Tänapäeva inimestel, kellel on internet olemas, ei 
kujuta ette, kuidas saab olla ilma, aga ilma oli väga pime.

Üks asi oli tehniline info, aga Fidoneti ja Useneti grupid
(UUCP meiliga koos toodi ka Useneti gruppe) olid ka muidu väga 
huvitavad. Sealsed diskussioonid olid väga 
informatiivsed. Suurem osa 
juttudest olid muidugi tehnilised, sest seal käisid tehnikud ehk need, kes 
said kanalile ligi.

\question{Kas too kollase kaabliga võrk hakkas tööle ka?}

Ikka, see töötas uhkelt. Novelli server käis veel 1992. aastal, kui ma 
sealt ära läksin. Inimesed said omavahel meilida ja ka välismaailmaga 
suhelda. Ainukene probleem oli see, et arvuteid, millel oli see 
Etherneti äge \emph{interface}, oli suhteliselt vähe, paar tükki kateedri 
peale vist suudeti tekitada.

\question{Kas Etherneti kaart oli defitsiit?}

Tol ajal oli kõik defitsiit, siis oli veel rublaaeg. Millise projekti 
raames see toodi, ei tea. Avo Ots\index[ppl]{Ots, Avo} tegi minu meelest 
doktoritöö selle kohta, kuidas ehitada arvutivõrku. See oli 
oluline kogemus, et toimuks järgmine samm. Pärast tehnikaülikooli 
töötasin lühikest aega Microlinkis\index{Microlink}, kus ma olingi 
arvutivõrkude installeerija ja ühtlasi .EXE\index{.EXE} kirjutaja.

\question{Miks sa sinna läksid?}

Ühel päeval astus uksest sisse Margus 
Kliimask\index[ppl]{Kliimask, Margus}, keda ma teadsin Rainer 
Nõlvaku\index[ppl]{Nõlvak, Rainer} kaudu, ja tegi ettepaneku hakata 
tegema ajakirja. Sellest sai .EXE.

\question{Miks ikkagi? Jälle kõlab suure muutusena, et ühel päeval tõmbasid kollast 
kaablit ja järgmisel päeval tegid ajakirja.}

Täpselt nii oligi. Ma arvan, et Rainer tahtis Microlinki promo teha. 
See võis olla suur motivaator, aga seda peab Rainerilt endalt küsima.

\question{Kust sul üldse tuli mõte, et ajakirja tegemine võiks huvi pakkuda?}

Tundus huvitav. Mul ei olnud siis rohkem kõrgeid eesmärke kui see, et elu oleks 
huvitav.

\question{See on tegelikult kõige kõrgem eesmärk, mis üldse saab olla.}

Algul oli jutt, et teeme ajakirja, ja siis selgus, 
et mul oleks ka uut töökohta vaja. Nii sattusingi korraks Microlinki\index{Microlink}. Olin seal aga loetud kuud, sest 
siis hakati tegema Eesti Forekspanka\index{Eesti Forekspank}\sidenote{Eesti 
Forekspank sündis 1992. aastal ja ühines 1995. aastal Raepangaga\index{Raepank} 
1995.}. Pangal olid oma sidevajadused ja mind kutsuti sinna tööle.

\question{Üheksakümnendate algus oli Eesti panganduses ju hull aeg!}

Jah, ja Forekspank oli sel ajal pisikene valuutavahetuskontor, mis opereeris 
rubla-dollari börsi.
See tegutses tolleaegses hulgifirmas Abestok\index{Abestok}. Selle ühes toas olid 
inimesed, kes otsustasid panga teha. Margus 
Kliimask\index[ppl]{Kliimask, Margus} oli nendega seotud, vist IT-poisi 
staatuses. Temaga läksimegi Rävala puiesteele, istusime koos 
ehitusjuhiga ühte tuppa, mille ühes nurgas hoidsid
ehitajad oma tööriistu, ja ehitasime panka.

\question{Kust tekkis mõte, et panga tegemiseks ei piisa kilekottidega 
sularaha edasi-tagasi lohistamisest?}

Need mehed, kes panga tegid, olid piisavalt targad, mõistmaks, et pank käib 
teistmoodi. Kui palju teistmoodi, sai alles siis selgeks, kui 
Inglismaalt osteti pangatarkvara ja konsultandid rääkisid, kuidas panka 
tehakse. Aga see ei olnud kohe esimesel aastal. Esimestel aastatel ehitasime, 
tõmbasime kaablit ja panime laua alla püsti serveri. Ühel ilusal päeval lükkas
Margus Kliimask\index[ppl]{Kliimask, Margus} kogemata varbaga 
toite välja ja pank jäi seisma. Aga mitte kauaks. 

Nii Rein Usin\index[ppl]{Usin, Rein}, Ivar Lukk\index[ppl]{Lukk, Ivar} kui ka Margus Kliimask\index[ppl]{Kliimask, Margus} olid 
visiooniga inimesed. See pidi olema suhteliselt algusaastatel -- BBSid ja 
Fidonet olid siis veel kuum teema --, kui Margus Kliimask ütles, et teeme 
modemipanga. Tal oli kindel mõte, et see peab olema Norton 
Commanderi\index{Norton Commander} F2 menüüs\sidenote{1986. aastal turule 
tulnud ja 1998. aastal viimase versiooni saanud Norton Commander oli 
ülipopulaarne failihaldur MS-DOSi platvormile. Ekraanil oli korraga kaks 
nimekirja faile ja käsurida, allservas nimekiri saadaolevatest 
klahvivajutusega käivitatavatest käskudest. Nii oli kasutajal ilma suurema 
koolituseta kohe selge, mida ja kuidas teha. Ohtralt kasutati F-klahve 
ja neist olulisemate funktsioonid on inimestel siiani peas (F3 -- faili sisu 
vaatamine, F5 -- faili kopeerimine).}. Kõik kasutasid Norton Commanderit 
ja kõigil oli see olemas, aga keegi ei ostnud, sest tol ajal tarkvara ei ostetud. 

\question{Jah, ma mäletan poes karpe, aga ei mäleta, et keegi oleks neid kunagi 
ostnud.}

Hämmastav oli see, kuidas mõtte väljakäimisest 
modemipanga \emph{launch}'ini läks umbes kaks kuud.

\question{Tegite kahe kuuga nullist modemipanga?}

See oli programm, mis oli mingil määral Norton Commanderiga integreeritud: 
läks sealt menüüst käima, nägi välja nagu Norton Commanderi 
osa, võimaldas makseid ette valmistada, kontoväljavõtteid ja panga teateid 
saada ning enda makseid panka saata.

\question{Ja teisel pool võttis mingi asi kõned vastu, suhtles panga 
tuumaga ja tegi arveldused ära?}

Just. Panga tuumaga suhtlemine oli üsna traagelniitidega asi, kuna selleks 
ajaks oli juba toodud Inglismaalt panga tarkvara, millel ei olnud ühtegi head 
liidest peale terminali.

\question{Ja siis tegite terminali emulaatori?}

Mina jah kirjutasin terminali emulaatori ja üks kolleeg kirjutas programmi, mis 
lükkas emulaatorist maksed pangasüsteemi, ning see toimis 
aastaid niimoodi, enne kui tekkisid tehnilised vahendid, et seda 
natukene viisakamalt teha. \emph{Launch} toimus 
tolle aja kohta suure pressikäraga: tehti korralik meediaüritus, imekenad 
Hansapanga\index{Hansapank} tüdrukud istusid ka seal ja tegid märkmeid. Ja läks mööda vaid
mõni kuu, kui Hansapangal tuli välja
Telehansa\index{Telehansa}.

\question{See kamp, kes tollal
BBSides suhtles, võis olla kokku paarsada inimest. Kust tulid
kliendid modemipangale?}

Kliendid jagunesid umbes pooleks. Forekspanga klientuurist arvestatav 
protsent oli Venemaal, sest suur 
raha oligi tol ajal Venemaal, aga ka Eesti klientuur ei olnud sugugi kehv. Pank 
müüs seda suhteliselt suure summa eest ja Eesti firmad 
ostsid. Käisin seda ise Tallinnas installeerimas. Küsimus ei olnud 
selles, et inimesed ei saanud tulla maalt linna pangaasju ajama, vaid nad 
lihtsalt ei tahtnud kontorist välja tulla. Pangas sai mugavalt ära käia 
laua tagant püsti tõusmata.

\question{Ja see kõik tasus ära, et hakata isegi arvutiga 
makseid ette valmistama?}

Sel ajal oli igas firmas raamatupidamiseks arvuti olemas ja raamatupidajate 
arvutites maksed olidki. Ilmselt mugavus ja aja kokkuhoid tõukasid
Eesti firmad sinnapoole.

\question{Kui palju seal telefoniliine küljes oli?}

Alustasime kahega ja lõpus oli vist kuus. Kuna 
sideseanss oli nii lühike, mahtus enamik sideseanssidest paari minuti 
sisse. Kõik pakiti kohapeal kokku ja saadeti ühe portsuna ära -- Fidonetist õpitud tehnoloogia. Alguses tegin mina kliendipoole ja 
Margus Kliimask\index[ppl]{Kliimask, Margus} kirjutas serveripoole. Hiljem kirjutasin serveripoole veidi paremaks, et see oleks paremini eskaleeritav.

\question{Mida see tähendab?}

Ühe masina taha sai panna mitu modemit.

\question{Kas sa oma BBSi hoidsid siis veel püsti?}

Minu meelest oli meil pangas ka BBS veel mõnda aega, 
Microlinkis\index{Microlink} oli kindlasti. Kuna Forekspank asus Rävala puiesteel, siis kohe, kui 
üheksakümnendate alguses tekkis internet, oli selge, et meil on ka 
seda vaja. Tõmbasime koos Andrus Aaslaiuga\index[ppl]{Aaslaid, 
Andrus} oma valgete käekestega mööda majakatuseid Forekspanga kõrvale KBFI\index{KBFI} majja, 
kus sündis Uninet\index{Uninet}, Etherneti kaabli.

\question{Te olite siis otse Unineti küljes?}

Otse Unineti küljes, olime ühed esimesed kliendid, kodukootud 
ruuteri softiga, mis läks flopi pealt käima. Mõlemas otsas oli üks 
arvutikast ja nii me ennast internetti panime. Muide, ükskord 
lõi meil sinna välk sisse.

\question{Mida te internetis tegite?}

Algul õppisime, mis see on. Ja pangas oli hädavajalik meilivahetus, et suhelda. Üks esimesi asju, mis pangas sai 
ehitatud, oli teleksi \emph{gateway} Pegasus Maili\index{Pegasus Mail}. 

\question{Misasi on teleks?}

Teleks oli viiekümneboodine\sidenote{\emph{Baud rate}, eesti keeles lihtsalt \emph{boodid}, 
näitab, mitu korda sekundis signaal liinil muutub andes indikatsiooni side kiirusest.}  telegraafisüsteem. Kahtlustan, et paljud pangad maailmas kasutavad seda endiselt. Suhtlus ei käi telefoniliini 
pidi, vaid selleks on eraldi teleksivõrk, mis toimib mööda telefonitraate 
hoopis teistsuguste signaalidega kui tavaline telefon.

\question{Kas see oli \emph{circuit switched}\sidenote{Ahelkommuteeritud. 
On ju ilus eestikeelne sõna?}, eks? Siis see vajas eraldi keskjaama?}

Jah. Põhimõtteliselt tuli ikkagi kõne teha ja ühendus püsti seada. 
See ehitati veel sel ajal, kui olid teletaibid -- klaviatuur ja 
paberirull.

\question{See \emph{gateway} ei saanud siis ju olla ainult tarkvaraline, vaid
oli ka riistvara vaja?}

Jah. Seal oli üks kast vahel, mis tegi sellest jadapordi. Esimese kasti tegi minu meelest
Küberneetika Instituudi\index{Küberneetika Instituut} majas üks Sass, Aleksander\index[ppl]{Reitsakas, 
Aleksander}.

See oli väga keeruline kast, tegin hiljem sellest peopesasuuruse 
versiooni flopikarpi.

\question{Mind hämmastab see, et sa ehitasid järjest keerulisemaid asju, aga kust sul tulid selleks teadmised, seda ei selgu.}

See on nagu Youtube'i videot vaadates -- tundub, et kõik asjad juhtuvad ise. 
Vahepeale mahtus siiski kuude kaupa õppimist, häkkimist ja katsetamist.

\question{Sul pidi hirmus kihu seda teha olema.}

Kindlasti, peaasi, et oli huvitav. 
Pangas töötades hakkas esimest korda ka kohusetunne vaevama, sest kui pank hommikul ei toiminud, olin ju mina paha.
Töötunde kulus kõvasti, aga üksiku inimesema ei olnud mul eriti muid kohustusi.

\question{Lisaks rääkisid muudkui teistega juttu BBSides.}

Panga ajal enam mitte, siis võttis töö kogu aja ära. Varem toimus jah BBSides suhtlus, aga kui tuli internet, võttis meilindus asja üle. Meiliga tuli kohe ka  \emph{gateway} 
kohe panga serverisse. Pank oli selles mõttes väga hästi kommunikeeruv.

\question{Legend räägib, et sina kirjutasid esimese eestikeelse klaviatuuri draiveri, 
on see tõsi?}

Nii ja naa. Rainer Nõlvak\index[ppl]{Nõlvak, 
Rainer} leidis esimesena, et klaviatuuril võiks eestikeelne \emph{layout} 
olla. Veel enne, kui infotehnoloogid jaole said, tellis Rainer eestikeelse 
klaviatuuri ära.  Nii et pärast, kui kehtestati  uus standard (EVS 8:1993),  
olid olemas klaviatuur ja oli kirja pandud standard. Lisaks klaviatuurile oli aga vaja ka standardile vastavat 
lokalisatsiooni. Eriti hull lugu oli Windowsi fontidega -- sel ajal oli olemas
Windows 3\index{Windows}. Ja siis korraldati konkurss, kus kõik lähenemised 
olid lubatud.

\question{Kes konkursi korraldas?}

Ma ei mäleta organisatsiooni nimesidenote{Tegemist oli Eesti Informaatikafondiga\index{Eesti Informaatikafond}, sellest sai hiljem Eesti Informaatikakeskus\index{Eesti Informaatikakeskus}, Riigi Infosüsteemi Ameti\index{Riigi Infosüsteemi Amet} eelkäia.}, aga see oli riiklik 
konkurss, mille auhind oli tolle aja kohta täitsa korralik, vist kakskümmend 
tuhat krooni. Olime selleks ajaks Raineriga juba natuke sel alal 
koostööd teinud -- Microlink pani enda klaviatuure müües kaasa draiveri, mis seda 
\emph{layout}'i toetas ka, nii et osa tööd oli juba tehtud. Kui konkurss 
välja kuulutati, ütles Margus Kliimask\index[ppl]{Kliimask, Margus}, visiooniga mees,
et teeme nii, nagu Microsoft teeb. Me \emph{reverse 
engineer}'isime kogu selle DOSi lokalisatsiooni ja klaviatuuri draiverid ning 
tegime installeerimisprogrammi, mis paigaldas 
standadkomponendid: \verb|KEYBOARD.SYS|i, \verb|COUNTRY.SYS|i ja muud
sellised asjad. Kuskilt õnnestus hankida soft, mis tegi Windowsi 
fonte, ja ma joonistasin fondid ka. See ei olnud küll kuigi hea soft, 
ei teinud TrueType'i \emph{hint}'ingut; \emph{kerning} vist 
on see teine, mis teeb fondid ilusaks, kui need väikseks muudad. Eesti 
fondid paistsid ekraanil karvased, aga me ei saanud sinna kahjuks midagi parata. Igal juhul
oli meie lähenemine teistega võrreldes nii palju parem, et võitsime konkursi.

\question{Kas pank läks konkursile osalema?}

Ei, ainult meie Margus Kliimaskiga\index[ppl]{Kliimask, Margus}. 
Meil oli pisike OÜ, koos pangaga tehtud ühisfirma Forex Communications modemipanga müümiseks. 
Selle firma alt osalesimegi. 

\question{Ja osalesite seepärast, et tundus huvitav?}

Sinna läksime ilmselt raha pärast ja võibolla ka 
Näitusepaviljonis toimunud joomingu pärast, mille seesama riiklik asutus piduliku sündmuse puhul 
korraldas.

\question{Kas sul sellepärast saigi panga aeg otsa, et pank sai valmis?}

Pigem pean olema tänulik pangajuhtidele, kes andsid meile 
hämmastavalt vabad käed igasugust tehnoloogiat katsetada ja uurida ning mõelda 
uusi asju. Tänu sellele oli Forekspank ka üks esimesi internetipanga tegijaid -- meil oli olemas internetiühendus ja me juba mõistsime, mis toimub. 

\question{Millega tollast internetipanka tehti?}

Forekspanga esimene internetipank oli minu meelest 
IISi\sidenote{1995. aastal turule toodud \emph{Internet Information 
Server (IIS)} oli Microsofti veebiserver, mis üritas (mõnevõrra tulutult) 
konkurentsi pakkuda tol ajal domineerinud Apache'i veebiserverile.} peal ja töötas
Windowsis\index{Windows}. 

\question{Eksootiline valik tolle aja kohta ...}

Oli küll imelik valik. Aga sel ajal olid meil juba arendus- ja 
hooldusmeeskonnad eraldi. Margus Kliimask\index[ppl]{Kliimask, Margus} oli 
arendusmeeskonnas. 

\question{Ehk te olite \emph{DevOpsist}\sidenote{Arendusmetoodika, kus tarkvara 
ehitamine ja selle edasine käitamine korraga nime kaotavad ehk omavahel 
lahutamatult kokku saavad.} astunud sammu tagasi?}

Panga käigushoidmine ongi natuke omapärane tegevus. Margus 
\index[ppl]{Kliimask, Margus} juhtis internetipanga arendust, tema meeskonnas
oli ka Pronto\index[ppl]{Pronto|see{Raja, Tanel}}\index[ppl]{Pronto}\sidenote{Vt
 lk \pageref{sisu:pronto}.} ja veel paar 
hakkajat selli. 

\question{Kas sina olid ka sellega seotud?}

Mina ei olnud internetipangaga peaaegu üldse seotud. Sel ajal oli
modemipank veel põhikanal, kuna internet oli siis vähestel. Forekspank oli juba üsna suureks kasvanud, 
hooldusmeeskonnas oli kümmekond inimest.

\question{See kõlab juba nagu terve organisatsioon, kahe telefoniliiniga ei saanud enam 
hakkama?}

Sel ajal tekkisid teised probleemid. 
Pangale ostetud tarkvara käis kummalise IBMi platvormi peal, mida aeg-ajalt 
tuli \emph{upgrade}'ida. Selle tarkvara jaoks oli COBOL uus keel. 
Tarkvara oli kirjutatud imelikus keeles nimega \emph{Report Generator Language}, mis 
oli pärit System/36\index{System/36}\sidenote{System/36 oli IBMi poolt 
1983. aastal turule toodud väike mitme kasutaja jaoks mõeldud mitmetegumiline 
server, mida programmeeriti peamiselt platvormipõhises RPG II\index{Report Program Generator} (\emph{Report Program Generator - RPG}) keeles.} ajast. Sellest keelest 
kumasid perfokaardid ikka veel kõvasti läbi.

\question{Vähe sellest, et teil oli visioon, aga raha pidi ju ka olema, et brittide juurde 
minna.}

Server maksis sel ajal meeletu raha. Algul ei olnud pangal jaksu õiget masinat osta, hangiti üks karm 
PC ja selle peal käis System/36 emulaator, millel jooksis 
panga tarkvara. Õnneks kasvasime sellest üsna ruttu välja. Pärast oli meil selline unikaalne platvorm nagu
AS/400\index{AS/400}\sidenote{AS/400, hiljem tuntud kui 
\enquote{System i}, oli IBMi keskmise suurusega serveriplatvorm, mis 
toodi turule 1988. aastal.}, mida ka korduvalt uuendati.

Ilmselt sai pank tarkvara ostes 
ka teadmise sellest, kuidas panka teha. See oli võibolla 
rohkem väärt.

\question{Teil oli Margusega juba siis kahe peale pisike OÜ, aga mõni 
veedab terve elu oma huvi üksnes akadeemilistes sfäärides rahuldades. Kust sul tekkis
arusaam ärist?}

Nagu ma mainisin, siis OÜ sündis modemipanka tehes ja pean jällegi kiitma 
tolleaegseid pangajuhte, kellega koos me ühisfirma lõime. Otseselt äritegemist kui sellist ei olnud: meie 
kirjutasime tarkvara ja inimesed maksid selle eest OÜ-le, pärast 
jagasime pangaga raha ära. Klassikalise äri mõistes ei pidanud meie midagi 
müüma, pank müüs. Muidugi tekkis ettekujutus näiteks
raamatupidamisest, aga erilist ärisoont see minus ei arendanud.
OÜ käigushoidmine mingit tähelepanu ei nõudnud, kogu fookus oli tehnoloogial.

\question{Tõnu Samuel\index[ppl]{Samuel, Tõnu}\sidenote{Vt lk \pageref{sisu:tonu}.}  rääkis mulle, et Mastsidenote{Ehk siis käesoleva loo kangelane.} oli see mees, kelle juurde sai minna riskantsete 
asjadega. Kui oli vaja emaplaadi peal vaibanoaga radu lahti kratsida 
ja sinna relee vahele panna, siis Tõnu teadis, mida teha, aga ei 
julgenud. Seevastu Mast julges.}

Ilmselt oli abiks raadiotehnika kateedri kool. Kui saad aru, 
mida teed, siis sa ei karda lõigata.

\question{Nii et sul sellist aukartust masina ees ei olnud?}

See kadus suhteliselt vara ära, kuna raadiotehnika kateedri Apple 
II\index{Apple II}s oli mitu laienduskaarti sees. Kui sellel oli
kaas peal, siis kuumenes üle, aga kaas ei olnud kunagi peal. Seal võis 
vabalt näppupidi sees sobrada ja mitte keegi ei öelnud, et sa ei tohi seda kivi 
välja võtta. Kõik oli pesades, kõike võis välja võtta. Kui katki läks, siis 
tuligi võtta. 

\question{Kas läks katki ka?}

Ikka läks, aga Apple II\index{Apple II} oli 
lihtsa loogika järgi ehitatud, Vene kivid läksid sinna asemele ja taktsagedus oli üks 
megaherts. Seda sai parandada ja see oli väga õpetlik. Ka 
esimese IBM PC\index{IBM PC}ga tulid kaasa (meil olid 
kõik juhendid olemas) BIOSi \emph{listing}'ud ja skeemid. Kõik olid 
standardtükid, kõike sai parandada ja parandatigi. 

\question{Mida sa pärast panka tegid?}

ITd ühele väikesele investeerimiskontorile. Kirjutasin 
Exceli Visual Basicus\index{Visual Basic} väärtpaberite 
kauplemise programmi. Tol ajal tehti paljusid asju Excelis, näiteks arvutati intressi. Tegin suured Exceli makrod, millega sai 
väärtpaberiportfelle hallata ja tehinguid jagada. 

\question{Kas jälle selle pärast, et oli huvitav?}

See oli rohkem vajaduspõhine. Meie enda investeerimiskontoril oli seda 
vaja ja ühe koopia müüsin maha ka. 

\question{Nii et tegelesid siiski ka müügiga?}

Ma ei tegelenud müügiga. Enamasti oli nii, et keegi tuli ja ütles, et tal 
oleks ka vaja. 

\question{Kui on väärt asi, siis lõpuks ikka tullakse.}

Jah, kui hind sobis, siis miks mitte.

\question{Sa oled BBSummeri\index{BBSummer} kuulsa grupipildi peal. Kas käisid tolle seltskonnaga läbi, kuigi töö võttis enamiku ajast ära?}

BBSummerid algasid siis, kui olin alles tehnikaülikoolis, ja neid ei olnud üldse palju. See grupipilt, mida sina vist 
mõtled\sidenote{Memcpy podcast'i kaanepildiks olev foto, kus on peal 
hämmastavalt paljude suurte asjade toonased või hilisemad algatajad.}, ei ole esimesest BBSummerist, vaid teisest või kolmandast, kus käisid ka FidoNeti tublid mehed Soomest. Seal 
pildil on üks habemega mees nimega Ron Dwight\index[ppl]{Dwight, Ron}, kes 
oli FidoNeti kunn Euroopas, regiooni pealik. Ron 
oli väga tore mees, ma olen tal isegi paar korda külas käinud ja tema juures Soomes 
ööbinud, kui piirid lahti läksid. Ja ma ei ole Eesti kambast ainukene, kes tal
külas käis. 
Soomlased, kes FidoNeti Soomes vedasid, olid tol ajal üldiselt väga toetavad. 
Sa oled teistega rääkinud, kuidas te Soome helistasite, ja keegi ei ole 
maininud, et tegelikult algusaegadel helistasid soomlased siia. Ei olnud nii, 
et ainult sealt oleks tõmmatud. Hiljem, kui BBSid ja firmad said siin jalad 
alla, saime "rinnapiima" otsast lahti, aga algusaegadel 
soomlased toetasid meid tublisti. 

\question{Kas puhtalt missioonitundest? Hõimuvelled ja nii?}

Ma ei tea, kui palju hõimuvendlus rolli mängis, pigem arusaam, et tehnoloogiat tuleb huvitatud inimestega jagada. 

Mul on nendest aegadest väga head mälestused ja sellepärast kutsusimegi neid ka BBSummeritele\index{BBSummer}. Ron käis minu meelest kahel. Igatahes oli
soomlasi esimestel BBSummeritel palju ja ma mäletan, kuidas nad olid selle grupipildi aegsel BBSummeril äärmiselt
hämmastunud sellest, et kõik võivad õlut juua ja et teisel päeval ei toimunud mingeid 
kaklusi!

BBSummeri korraldamise juures oli veel tore see, et korraldustasu
tagas söögi ja joogi kõigiks päevadeks. Ja õlut pidi kõigile jätkuma. Ühele BBSummerile toodi küll õlut Fanta tünnides, nii et 
õllel oli kerge Fanta mekk juures.

\question{Tundub, et sul on inimestega vedanud.}

Mul on jah sõpradega vedanud. Kui ma üksi elasin ja 
tehnikaülikoolis\index{Tallinna Tehnikaülikool} 
vabakutseline olin, siis suhtlesin väga paljudega. 
Hiljem võttis perekond nii palju aega ära, et kahjuks ei jõudnud enam kõigiga 
kontakti hoida.

\question{Aga kriitilisel hetkel olid nad olemas?}

Nad on siiamaani olemas. Näiteks 
Lõvi\index[ppl]{Lõvi} kohtasin ma umbes viis aastat tagasi Selveri 
parklas, nüüd käisin tal hiljuti tehnikaülikoolis külas.

\question{Ahti\index[ppl]{Heinla, Ahti}\sidenote{Vt lk \pageref{sisu:ahti}.}  ütles 
väga targasti, et seltskond noori inimesi sai
omavahel suheldes inimeseks koos Eesti riigiga. Kas sul on ka selline 
tunne?}

Jah, me olime kõik suhteliselt üheealised. Täpselt selles 
vanuses, kui oli huvi teha midagi uut ja selleks tekkis võimalus ning ka omavaheline klapp. Oli ka erandeid, näiteks Henn Ruukel\index[ppl]{Ruukel, Henn} 
oli esimesel BBSummeril selgelt alaealine, aga õlletünni juures passi ei 
küsitud.

\question{Mida sa praegu teed?}

Pean pausi. Aitan ülikoolil satelliiti\sidenote{Masti panusega satelliit lendas 
2020. aastal ka edukalt kosmosesse.} ehitada. 

\question{Sellepärast, et on huvitav?}

Sellepärast, et on huvitav. Kosmos on huvitav.

\question{Kosmos on suur ka, seal ei ole karta, et huvitavad asjad 
saavad otsa.}

Praegu käib sebimine enamjaolt Maale väga lähedal. Orbiidid, kuhu 
väikseid satelliite lastakse, on viie- kuni seitsmesaja kilomeetri kaugusel.

\question{Kas sul üldse on kunagi juhtunud, et järgmist huvitavat asja ei ole 
silmapiiril?}

Ei.

\question{Kuidas see sul on õnnestunud?}

Isegi kui päevatööl ei ole huvitav, siis mul 
kodus käib kogu aeg mõni projekt. Kui üks saab valmis või läheb 
sahtlisse (sinna läheb enamik, sest huvi kaob ära), on 
järgmine kohe laual. Sellist asja ei ole, et mul ei ole midagi teha.

\question{Kas sul sahtel juba täis ei saa?}

Saab. Jube täis on. 

\question{Mida sa siis teed?}

Viskan ära. Suur osa neist on ju eksperimendid. Võtan ära tükid, mis lähevad  
järgmise eksperimendi peale, ja ülejäänu on prügi. Teadmised jäävad alles.
