\index[ppl]{Kalju, Kain}
\question{Kuidas sina arvutite juurde jõudsid?}
See oli umbes aastal 1990--1991, kui sõpradel tekkisid esimesed arvutid, 
olime kaksteist kuni neliteist aastat vanad. Ühel sõbral oli selline 
imelik asi nagu Texas Instruments TI-99\sidenote{Täpsemalt Texas Instruments 
TI-99/4\index{Texas Instruments TI-99/4}. Ärilistel ja arhitektuursetel 
põhjustel lühikese elueaga koduarvutite perekond. Oli koos samal 1979. aastal 
turule tulnud Atari 8bitiste arvutitega esimesi omataolisi, millel oli 
audio- ja videoülesanneteks eraldi protsessorid.}, see oli Commodore'i ja Apple 
II sarnane riistapuu selles mõttes, et see oli 16bitise protsessoriga ja 
\emph{boot}'is otse BASICusse\index{BASIC}. 

See arvuti oli telekaga ühendatud ja seal olid mõned primitiivsed mängud, nagu näiteks 
Space Invaders\index{Space Invaders}, ning 
loomulikult ka BASIC. Kogu programmi kood tuli kassetilindilt nagu tol 
ajal kombeks, flopisid polnud olemas. See oli minu esimene kokkupuude 
arvutiga, millel oli klaviatuur, kuhu sai sisestada programmi koodi ja 
kus katsetasime ka esimest korda BASICus ise programme teha, toksides 
neid ajakirjadest ja mõeldes ka ise välja. 

\question{Mis linnas see oli?}

Ma olen pärit Keilast\index{Keila} ja mul ei ole kunagi olnud 
spetsiaalset ligipääsu arvutitele mõnes teadusasutuses või koolis. Võrreldes mõne teisega oli minu ligipääs arvutitele suhteliselt piiratud.

\question{Kas sul reaalainete vastu oli huvi?}

Koolis käisin reaalkallakuga klassis. Meil oli 
gümnaasiumis\index{Keila Gümnaasium} väga vahva lend, peaaegu kõik poisid olid 
arvutihuvilised ja nii palju, kui olen nende elukäiku jälginud, on 
pea kõik mingitpidi arvutimaailmas tegevad.

\question{Kuidas see juhtus? Kas teil oli koolis nii korralik tase?}

Gümnaasiumi esimestes klassides (olime just 
läinud üle kaheteistkümne klassi süsteemile) olid kooli arvutiklassis 
Jukud\index{Juku}, mis meid loomulikult absoluutselt ei huvitanud, sest
seal oli Pascal\index{Pascal}, aga meil oli siis juba ligipääs PCdele.

\question{Juku oli omal ajal igavesti äge aparaat!}

Jah, aga Juku tuli hilisemas faasis, kui meil oli 
juba PC ligipääs olemas ja mul endalgi kodus PC. Minu 
suur arvutihuvi läkski lahti sellest hetkest, kui vanemad otsustasid mulle 
PC osta. Seda lugu peab natuke tagasi kerima: seesama sõber, 
kellel oli Texas Instrumentsi imepill, sai aasta hiljem monokroomekraaniga
286, mille ta isa tõi Ameerikast. Nad elasid
kortermaja esimesel korrusel ja PC oli raudkapis luku taga. Oli suur hirm, et 
keegi murrab sisse ja varastab arvuti ära. Aeg oli selline.

\question{Arvuti maksis tollal ju rohkem kui korter. Mõni ime, et PC 
kappi pandi!}

Mu vanematele käis see kohutavalt pinda, et ma ei viibinud üldse kodus, vaid olin
kogu aeg hilisööni sõbra juures külas. Millalgi üheksakümnendatel, 
vahetult enne Eesti krooni tulekut oli aeg, kui rubla devalveerus väga
kiiresti. Olen isa käest küsinud, kuidas see täpselt oli, ja ta on meenutanud, et 
tema sai millegipärast palka Ameerika dollarites ja ostis
kuskilt kooperatiivist dollarite eest
ühe 286. Hinnaklass oli umbes tuhat dollarit. See oli 
VGA ekraaniga, täiesti uus ja väga äge, kuigi 286 oli
tolleks ajaks ilmselt natuke \emph{outdated}, kuna siis oli juba 386 
ajastu.

\question{Võrreldes XTdega, mille abil Tartu Ülikoolis 
programmeerimist õpetati, oli see ikkagi väga kõva sõna. Mida sa selle arvutiga tegid?}

Nagu noor poiss ikka, tõenäoliselt mängisin, aga mind huvitas ka kõikvõimalik 
tarkvara.

Meenub tore lugu, kuidas umbes aasta pärast arvuti saamist käisime sama sõbraga 1993. aastal Ameerikas. 
Meil oli kolmene punt, kes me 
elasime üksteise lähedal, ja meil kõigil oli kodus kas isiklik või vanemate 
tööarvuti. Kord rongiga Tallinnast Keilasse sõites
hakkasime rääkima, et jube lahe oleks minna 
Ameerikasse. Ühel sõbral elas tädi seal ja ta oleks meid hea meelega vastu võtnud, ainult et
kuidas sinna saada. Minu isa töötas Muuga 
sadamas ja tuli jutuks, et põhimõtteliselt saaks ka 
laevaga minna. Olime siis
viie-kuueteistaastased. Rääkisin sellest kodus, kuidagi hakkas 
pall veerema ja ühel hetkel taotlesime juba USA saatkonnas viisat. 
Järgmisel hetkel oli isal kokku lepitud, et saame minna kaasreisijateks 
suurele Ameerika kaubalaevale, ning me sõitsimegi Muuga sadamast laevaga üle Atlandi ookeani 
New Orleansi. Seal pani laevakompanii meid lennukile ja 
edasi lendasime JFK lennuväljale New Yorki, kus sõbra tädi meid vastu 
võttis. Kusjuures me saime laeva peal palka, sest laevafirmale oli palju odavam 
vormistada meid töötajateks. Muidu oleks olnud vaja tasuda suuri 
kindlustusmakseid. Selles mõttes täiesti kreisi käik.

\question{Kaua te sõitsite sinna ja kas te midagi kasulikku ka laeva peal tegite või sõitsite lihtsalt kaasa?}

Laevasõit üle ookeani võttis umbes kaks nädalat. Midagi kasulikku me ei teinud, hängisime nii-öelda ohvitseride
alal. Meile küll näidati, kuidas laev töötab, aga me näiteks ei koristanud tekki ega teinud muud kasulikku. 
Võibolla heal juhul saime ülevaatlikku õpet mootoriruumist justnagu muuseumis. Loomulikult ei lastud meil midagi teha, võibolla avaookeanil saime korraks rooli keerata ja nii-öelda
laeva juhtida.

\question{Millega te tagasi tulite?}

Tagasi tulime lennukiga. Aga miks ma sellest üldse räägin, on see, et 
Ameerika pinnale astudes oli meil päris palju raha, kuna saime laevast 
palka, umbes tuhat viissada dollarit, mis oli tolle aja kohta üüratu 
summa. Mina kulutasin raha ära loomulikult arvutipoes -- tõin endale Ameerikast elu ühe kõige tähtsama riistapuu, 
milleks oli modem. Pärast seda läks elu lahti. 

See oli 2400boodine modem, tüüpi ei mäleta. Lisaks 
tõin Sound Blaster 16\index{Sound Blaster} helikaardi, mis oli tollal täiesti 
tipp\sidenote{Sound Blaster oli Singapuri firma Creative 
Technology (tuntud USAs kui Creative Labs) helikaartide perekond. Need kaardid 
olid PC-maailmas \emph{de facto} standardiks, kuni Windows 95 vastavad liidesed 
standardiseeris ja PC audio muutus tarbeesemeks.}. See oli just paar kuud varem välja tulnud. 

Üks asi, mille ma hiljem avastasin ja mis levisid BBSides, olid helimoodulid. 
Mul oli neid päris palju, kogusin neid mõnda aega. Ilmselt toona 
tegid seda paljud. Need on helifailid, mida tollal pandi kokku
Amiga arvutites ja mis koosnesid sämplitest. Oli umbes kaheksa \emph{track}'i, kuhu sai miksida sämpleid niimoodi 
kokku, et tekkis muusika. 

\question{Ja need liikusid BBSides?}

Jah. Loomulikult sai üritatud neid ka ise teha, aga mul erilist muusikatausta ei olnud, nii et sellest ei tulnud midagi välja.

\question{Kas Ameerikast tagasi tulles panid kohe BBSi püsti?}

Ei, hakkasin siis alles avastama BBSi 
maailma. Vanu asju üle vaadates selgus, et üks mu lemmik BBS oli Dark 
Corner\index{Dark Corner}, mida vedas Priit Kasesalu\index[ppl]{Kasesalu, 
Priit}. Esmalt loomulikult üritasin alla laadida kõike, mida sain. 
Kõik oli ju puhas kuld, kogu tarkvara. 
Tol ajal veel
Kadaka turg\index{Kadaka turg}\sidenote{Aastal 1991 avatud ja 2002. aastal 
kaubanduskeskusega asendatud, Mustamäel asunud turg oli küllaltki metsik 
müügikeskkond, kust oli võimalik hankida kõike alates karvamütsidest ja 
Nõukogude aurahadest kuni kõikvõimaliku piraatkaubani. Sisuliselt oli tegu 
endise Nõukogude Liidu territooriumil toiminud varimajanduse väljundiga 
Eestisse. Turg oli turistide seas hinnas, parematel aegadel käisid sinna 
Tallinna sadamast eribussid.}, kus müüdi piraattarkvara. Nii et
väga palju sain ka sealt. Minu mäletamist mööda BBSides 
otseselt piraattarkvara väljas ei olnud, pigem 
häkkimise stiilis tarkvara.

\question{Windowsi sealt vist keegi endale ei laadinud?}

Jah, just, selliseid asju otse faililistides ei olnud, need olid taha 
nurkadesse ära peidetud. Aga seda mäletan küll, et meil oli kodus 
telefoniliin ja minutitasu ei olnud või siis 
oli see väike. Igal juhul oli meie koduliin enam-vähem
ööpäev ringi kinni, sisse ei olnud võimalik helistada, sest minu 
arvuti helistas ja laadis kogu aeg midagi alla.

\question{Kuidas te alguses rea peale saite? Kuidas sa teada said, mis numbri 
peale helistada?}

Võimalik, et see teadmine tuli .EXE ajakirjast\index{.EXE}. Aga kui oled ühte BBSi juba sisse pääsenud, siis avaneb kogu 
maailm. Üks teema, mida BBS levitas, oli teiste BBSide 
aadressidega failid. Ühel hetkel pani Priit Kasesalu\index[ppl]{Kasesalu, Priit} 
kogu oma BBSi viimase versiooni veebi üles. Laadisin selle alla ja 
avastasin selle kettalt igasugu huvitavaid asju. 

\question{Mida seal leidus?}

Kõikvõimalikke häkkimisvahendeid, C-programmide näiteid, raamatuid 
nagu \enquote{Terrorist Handbook}\sidenote[][-4cm]{Ilmselt peab Kain silmas William 
Powelli raamatut \emph{The Anarchist Cookbook}. Vietnami sõja vastaste 
protestide laineharjal 1971. aastal USAs ilmunud (ja mitmel pool keelatud 
olnud) raamat sisaldas kõikvõimalikku vastandkultuuriga seotud sisu 
termiidi ja LSD valmistamise õpetustest kuni telefonisüsteemide 
murdmise juhisteni. Raamat levis tekstifailina laialt ülikoolide serverite ja FidoNeti 
kaudu ning seda täiendati pidevalt; eriti kuulsad on anonüümse 
autori \enquote{\emph{The Jolly Rogeri}} täiendused.}. Igasugune 
selline kraam, mis pakkus noortele inimestele põnevust.

\question{Tuleme veel kord sinu arvutihuvi alguse juurde. Kas sa olid pigem 
seda tüüpi mees, kes mängis arvutiga, võrgutas arvutit või programmeeris 
arvutiga?}

Tagantjärele mõeldes on olnud mitu ajajärku. Koduse 286 ja BBSide ajal üritasin pigem 
sisse krahmata kõike, mida nägin. Leidus ka 
arvutimänge, aga ma ei mäleta, et oleksin väga palju mänginud. 
Kui mul endal veel arvutit ei olnud, siis sõbra juures mängisime loomulikult, 
mitte ei programmeerinud. Hiljem jäi mängimine tagaplaanile ja püüdsin
aru saada, kuidas arvuti töötab. Näiteks üks teema, mis mind 
kohutavalt paelus, olid viirused. Mul oli alati kõige viimane viirusetõrje 
tarkvara. Mul oli selleks hetkeks juba mitu kõvaketast ehk 
võimalus katsetada, mida viirused teevad. BBSides levitati ka nii-öelda 
viirusekollektsioone ja ma uurisin, kuidas viirus 
põhimõtteliselt töötab. 

Järgmine ajastu tuli siis, kui avastasin enda jaoks 
Linuxi\index{Linux}, samal ajal tekkis ka internet. Gümnaasiumi 
kaheteistkümnendas klassis sattusin tööle Riigi Elektriside 
Inspektsiooni\index{Riigi Elektriside Inspektsioon|see{Tehnilise Järelevalve 
Amet}}, mis on täna Tehnilise Järelevalve Amet\index{Tehnilise Järelevalve 
Amet}. Sattusin patsiga poisiks, kuigi patsi pole mul
kunagi olnud. Olin tavaline arvutipoiss, seadistasin arvuteid.

\question{Kuidas sa kooli kõrvalt sinna sattusid?}

Seesama sõber töötas Pennus\index{Pennu} ja temalt kuulsin,
et otsitakse arvutitüüpi, kes oskaks arvutitega midagi teha. Läksin 
kohale ja mind võeti poole kohaga tööle.

\question{Kas teil klassist töötasid mitmed keskkooli ajal?}

Jah, üks klassivend töötas näiteks Keila linnavalitsuses. Ta oli juba siis kõva programmeerija, kinkis mulle 
mu esimese programmeerimisraamatu \enquote{C Programming Language}\index{The C 
Programming Language}, autoriteks Brian Kernighan ja Dennis Ritchie.

\question{See on seesama salapärane väljaanne\sidenote{\phantomsection\label{sisu:richie_vene}Kainil oli raamat 
jutuajamisel kaasas, selles puudus igasugune märge väljaandja ning trükkimisaja ja -koha kohta. Raamat oli 
korralikult köidetud ja kopeeris isegi värvilist kaanekujundust täpselt. 
Paigas olid ka sellised detailid nagu indeksis mõiste \enquote{recursion}, mis viitas (nagu ka mõiste sisu nõuab) 
tagasi mõistele endale. Mart Palmas\index[ppl]{Palmas, Mart} mäletab, et raamatut 
olla trükitud Novosibirskis. }, mis minulgi oli.}

Just, Amazonis on täpselt seesama raamat müügil. 
See oli mu esimene programmeerimisraamat, aga leidis kasutamist ka
aastaid hiljem, kui tegin netit\index{neti.ee} ja mul tekkis praktiline 
vajadus programmeerida suurema jõudlusega otsingusüsteemi.

\question{Kuidas teil ikkagi juhtus olema selline klass, 
kus mitmed töötasid-programmeerisid juba keskkooliajal?}

Võibolla just sel ajal arvutid ilmusidki rohkem koju ja 
kontorisse ning oli tohutu puudus oskusteabest. Vanemad 
inimesed ehk ei julgenud arvuteid veel kasutada, samal ajal kui noored julgesid nendega 
igasuguseid asju teha.

\question{Igas keskkooliklassis ei olnud nii, et neli-viis 
poissi töötasid arvutispetsialistidena. Miks teil oli?}

Ma ei oska seda tagantjärele öelda. Küll aga mäletan sellist huvitavat seika, et 1995. aastal oli meil kaheteistkümnendas klassis
arvutieksam, mis aga ei seisnenud 
programmeerimises, vaid me seadistasime koolile
arvutiklassi. Kool sai Tiigrihüppe kaudu peaaegu 
klassitäie arvuteid ja siis R-klassi\sidenote{Reaalklass.} poiste ülesanne 
oli võrgutada klass 
füüsiliselt Etherneti kaabliga ning installeerida arvutid ja 
võrguserver, milleks oli Linuxi server. Server jäi minu peale, 
kuna ma olin tollel hetkel kõige suurem Linuxi\index{Linux} käpp võrreldes 
teiste poistega.

\question{Linux ei olnud selleks ajaks ju kuigi vana, kuidas sa selle otsa 
komistasid?}

Linuxi otsa komistasin siis, kui töötasin Riigi Elektriside 
Inspektsioonis\index{Riigi Elektriside Inspektsioon}. Kui ma sinna 
läksin, siis seal veel internetti ei olnud, aga tekkis paar kuud hiljem, 1994. aasta lõpus. 
Inspektsioon asus aadressil Ädala 4d, mis on ka 
legendaarne internetihoone. Meie allkorrusel oli 
Valitsusside\index{Valitsusside}, kus toimetas Taavi Talvik\index[ppl]{Talvik, 
Taavi}. Taavi andis Riigi Elektriside Inspektsioonile juhtmeotsa kätte, 
milleks oli kümnemegabitine koaksiaalkaabel, ja ütles, et palun, siin on 
internet. See koaksiaalkaabel sai veetud kõikidesse ruumidesse, ei mingeid 
\emph{hub}'e ega tähttopoloogiat.

Siis avastasingi enda jaoks interneti. Koolis loomulikult poistele rääkisin, et 
FidoNet on vana ja aeglane jama, toimib üle modemi, aga meil on 
üks palju uuem ja huvitavam asi. Kusjuures Valitsussidest edasi olid kanalid 
üsna kiired. Mäletan, et Tartu Ülikooli FTP-serverist sai kahemegabitise 
kiirusega faile alla laadida, see oli meeletu kiirus. Välislink oli 
loomulikult kuskil 64 või 128 kilobitti. 

\question{Mida Tartu Ülikoolist tõmmata oli?}

Seda ma täpselt ei mäleta, aga ju midagi oli, sest mul on väga selgelt 
meeles kadri.ut.ee\index{kadri.ut.ee}\sidenote{Tartu Ülikooli masinad kadri.ut.ee ja madli.ut.ee said Toomas Soome\index[ppl]{Soome, Toomas} andmetel nimed Otto Telleri\index[ppl]{Teller, Otto} tütarde järgi.} FTP-server. 

FidoNet oli selles mõttes tohutu kullaauk, et avas kõik oma 
\emph{echo}-kanalid. Internet aga avas meililistid ja ühes listis ma 
lugesin, et Anto Veldre\index[ppl]{Veldre, Anto} teeb 43. 
Keskkoolis\index{Tallinna 43. Keskkool} 
sissejuhatavaid kursuseid. Toona ilmus ka ajakiri .EXE\index{.EXE}, 
kuhu Anto artikleid kirjutas. Ma ei mäleta, kumb kummale täpselt eelnes, aga 
igatahes ühel hetkel olin 43. keskkoolis, et 
\enquote{siin ma olen ja ma tahan teadmisi saada}. Seal tegutsesid  
sellised legendaarsed koolipoisid nagu Indrek Mandre\index[ppl]{Mandre, 
Indrek} ja Heno Ivanov\index[ppl]{Ivanov, Heno}. Tagasi tulin  
juba kuue Slackware\index{Slackware}'i distributsiooni installeerimisflopiga. Installeerimisprotsess käis flopi kaupa. 

\question{Kas lisaks kõigele muule jäi Anto peale ka Linuxi-pisiku 
levitamine Eestis?}

Tal oli väga suur roll selles, et Linux 
Eestis käima läks. Igal juhul mina sain küll selle pisiku. Kuna olin tolleks 
hetkeks juba mõnda aega Elektriside Inspektsioonis\index{Riigi Elektriside 
Inspektsioon} töötanud ja ka palka saanud, oli mul päris korralik 
\enquote{taskuraha}. Müüsin oma 286 FidoNetis maha 
(FidoNetis käis ka suur riistvaraga hangeldamine) ja ehitasin endale uue arvuti,
486, kusjuures see ei olnud mitte lihtsalt 486, vaid 486DX4 
100 MHz\sidenote{Inteli nomenklatuuris oli DX-tähistusega protsessorite 
kiibil eraldi matemaatika kaasprotsessor, mis andis märgatava
jõudlusvõidu.} -- absoluutne tipp. 

See oli kõige kõvem 486, mis üldse kunagi tehti. Sel ajal oli mul juba \emph{node} registreeritud. Olin varem saanud Dark Corneri 
BBSist\index{Dark Corner} esimese FidoNeti \emph{point}'i, kust 
pääsesin ligi FidoNeti uudisekanalitele, aga ühel hetkel tundus, et 
\emph{node} oleks ägedam. Kirjutasin Tarmo Mamersile\index[ppl]{Mamers, 
Tarmo} (tema oli Eesti regiooni \emph{manager}, kes jagas aadresse), 
kas oleks võimalik registreerida \emph{node} number kuuskümmend kuus, ja 
Tarmo vastas, et \enquote{tehtud}.

Millalgi seadistasin Elektriside Inspektsioonis 
Linuxi\index{Linux} serveri, sest meil on praktiline vajadus kasutada
printerit, faksi ja faile. Linuxi server jagas 
faile üle Samba teenuse ja võttis vastu fakse. Mul õnnestus ka enda 
FidoNeti \emph{node} samasse serverisse sokutada. Kui muidu töötas FidoNeti 
tarkvara MS-DOSi peal, siis oli ka alternatiiv Unixitele 
Ifmaili\index{Ifmail}-nimelise programmi näol.

\question{Miks riigiametis üldse internetti vaja 
oli? Kas see oli puhtalt sinu huvi või tehti seal sellega midagi kasulikku ka?}

Jah, praktiline vajadus interneti järele oli olemas, sest 
inspektsioon\index{Riigi Elektriside Inspektsioon} tegi koostööd 
ITUga\sidenote{\emph{Rahvusvaheline Telekommunikatsiooniliit.}}, kes 
juhib sageduste jaotust, protokolle ja kõike muud sellist. 
Inspektsioonil oli ITUga tihe kirjavahetus, ilmselt meili teel. Ma 
ei suuda meenutada, kuidas meilivahetus enne kaabliga internetti käis, aga 
pärast oli seesama Linuxi masin ka loomulikult meiliserveriks. Meil
tekkis oma domeen rei.ee ja Linuxi server hakkas rei.ee kirju vastu 
võtma ning ka mina sain endale esimese isikliku ülilühikese meiliaadressi, mis 
oli tollal ülikõva -- kain@rei.ee\sidenote{Lühikesed meili- ja muud aadressid 
olid staatusesümbolid, mis näitasid kuulumist kas serveriadministraatorite 
kõrgesse kasti või neile väga lähedasse ringkonda.}.

\question{Nii et sa avastasid ennast suhteliselt õrnas eas Linuxi ruuduna 
riigiasutuses?}

Just. Ja kui külastasin Anto 
Veldre\index[ppl]{Veldre, Anto} arvutiklassi 43. 
keskkoolis\index{Tallinna 43. Keskkool}, jäi mulle sealt üks asi elu 
lõpuni meelde: kuidas kõik need noored tüübid, kes seal 
siil.edu.ee\index{siil.edu.ee}-nimelise SCO\index{SCO UNIX} masina 
taga istusid, olid tohutult kõvad häkkerid. Nad demonstreerisid,
kuidas suudavad \emph{exploit}'ida Tartu 
Ülikoolis olevaid masinaid\phantomsection\label{sisu!ylikooli_root} ja sealseid professoreid jälgida. See 
avaldas mulle nii suurt muljet, et mind hakkas lisaks 
viiruseteemale huvitama arvutiturvalisus.

Me olime kõik ühise Etherneti kaabli peal. Räägin sellest esimest korda avalikult, et ma 
\emph{sniff}'isin loomulikult ka meie võrgus, mida 
Valitsusside\index{Valitsusside} insenerid seal tegid. Sama kaabli otsas oli kaks ametit: Riigi Elektriside 
Inspektsioon\index{Riigi Elektriside Inspektsioon} ja 
\emph{Valitsusside}. Ja kui Valitsusside insenerid käisid oma ruutereid või 
keskjaamu üle Telneti konfigureerimas, siis levis liiklus lahtise 
tekstina võrgus. Nende tegevust oli päris huvitav jälgida. 
Loomulikult ei kasutanud ma seda kunagi pahatahtlikult ära.

\question{Eks see seik iseloomustab suurepäraselt toonast aega. Kui 
praegu kasutaks keegi lahtise traadi peal lahtist kanalit, korraldataks 
poole tunniga mingi jama.}

Jah, ma arvan ka. Siis oli kogu võrguvärk 
niivõrd ebaturvaline, et niipea kui Soomest keegi härrasmees\sidenote{Tatu 
Ylönen, Helsingi Tehnoloogiaülikooli teadlane} tegi 
\emph{secure shell}'i esimese versiooni, hakkasin seda kasutama kohe, kui teada sain. 

Kui nüüd Keila Gümnaasiumi\index{Keila Gümnaasium} juurde tagasi tulla, siis 
pärast kooli lõpetamist jäin ma seal edasi 
administreerima kooli serverit. Nagu tol ajal ikka, pidid 
kõikidel Unixi masinatel olema ilusad nimed. Kodus sellest rääkides pakkus 
isa välja, et Kratt oleks hea nimi. 
Vaatasin hiljuti nimeserverist järele, et Keila Gümnaasiumi serveri 
nimi on siiamaani kratt.keila.edu.ee\index{kratt.keila.edu.ee}.

\question{Hakka siis nime tagantjärele muutma \ldots Loodetavasti riistvara ei ole 
päris seesama?}

Riistvara ei ole kindlasti sama, sest seda koolimaja füüsiliselt enam alles 
ei ole. Keilas on nüüd uus koolimaja, kus mu enda lapsed käivad, sest ma elan 
siiamaani Keilas. Aga serveri aadress on sama.

\question{Seepärast ongi asjade nimetamine oluline, et nimed võivad pikalt 
kesta.}

Just. FidoNeti ajast on veel üks huvitav asjaolu osutunud 
hiljem väga kasulikuks. Nimelt töötasid modemid
AT-käsustikuga\index{AT-käsustik}\sidenote{Hayesi käsustik, tuntud ka kui
AT-käsustik, on käsukeel, mille Dennis Hayes lõi 1981. 
aastal omanimelise ettevõtte 300-boodise modemi Smartmodem juhtimiseks.}, mis 
oli selles mõttes universaalne, et seda kasutati hiljem 
erinevates muudes rakendustes. BBSidesse sissehelistamine toimus loomulikult
lihtsa terminaliga ehk pidid nagu häkker käsustikku teadma. Enne 
helistamist pidi sisestama ATDT, telefoninumbri ja nii edasi, võibolla ka
seadistama protokolli. Tolleaegsed inimesed teavad täpselt, 
missuguse protokolli heli on kuulda memcpy podcast'i avakõllis. 

\question{Kas BBSil oli kliendisoft ka?}

Ei olnud. Helistada tuli terminaliga, ainult FidoNetil oli kliendisoft 
nimega FrontDoor, mis helistas, ja teine soft, mis pakkis kokku FidoNeti 
\emph{echo}'d ja saatis selle paki edasi. BBSil kliendisofit ei olnud, tuli minna Telnetiga külge ja seal 
edasi tegutseda.

\question{Oleks ju olnud loogiline, et 
keegi oleks teinud BBSide ette, näiteks \emph{cache}'i jaoks 
mingi tarkvara.}

Jah, kui vaadata, mis toimus Ameerikas, kus olid 
\emph{online service provider}'id, nagu AOL ja 
CompuServe\index{CompuServe}\sidenote{Internetieelsel ajal domineerisid USA 
turul agressiivsete turunduskampaaniatega (ühel hetkel oli pool \emph{kõigist} 
toodetud CDdest AOLi logoga) teenusepakkujad, kes pakkusid kummalist segu 
BBSi-laadsetest ja internetiteenustest. Neist suurimad olid CompuServe, Prodigy 
ja America Online.}, siis neil oli tarkvara olemas. Mäletan, et 
kui olin USAst modemi ostnud, siis noorte poistena tahtsime seda loomulikult 
proovida. Kujutad ette, me keerasime kruvikeerajaga lahti 
ühe suure soliidse arvuti, vist Computer 2000\sidenote{Computer 2000 
oli küll ka siinmail tegutsenud arvutiäri, kuid ilmselt peab Kain silmas Gateway 
2000 nimelist ettevõtmist, mis tootis sama nime all personaalarvuteid.}, mis oli 
tol ajal väga kõva valge PC bränd. Sõbra tädimehel oli 
väike arhitektuuribüroo ning meil oli julgust 
omavoliliselt kruvikeerajaga lahti keerata üks nende suur \emph{tower} ja 
selle sees proovida seda sisemist modemit. Modemiga oli kaasas kas 
CompuServe'i või mõne muu sarnase teenuse CD-plaat või flopi, ja siis sai helistatud Ameerika BBSi. 

\question{Kui sa BBSides ringi kolasid, kas sulle jäi midagi muud peale tarkvara 
ka silma? Sa mainisid raamatuid ja MODe.}

Raamatud mind siis eriti ei köitnud, BBSidest laadisin ikkagi 
peaasjalikult tarkvara ja muusika MODe. Aga kogu infovoog 
tuli FidoNetist, see oli minu jaoks kullaauk. Nagu varem 
mainisin, ei olnud mul ligipääsu teadusasutustesse ja
ülikoolidesse ega ka mentorit. Meil oli kamp 
poisse, kes omavahel infot vahetasid, ja kõik käis katse-eksituse meetodil.

\question{Hea, et te selle kambaga paha peale ei läinud. Noored 
poisid, tont teab, mida oleks võinud teha.}

Ju siis olime piisavalt mõistlikud. Sellest ajast saadik on mul 
ise õppimise oskus. Võibolla see sai ka saatuslikuks, miks ma ei suutnud
tehnikaülikoolis kaua õppida, ainult ühe aasta nagu
paljud teisedki tollal.

Peale gümnaasiumi läksin tehnikaülikooli informaatikasse\index{Tallinna 
Tehnikaülikool!Informaatika}, aga kuna ma juba ka töötasin, siis tekkis
igasuguseid huvipakkuvaid projekte. Mina eeldasin, et 
saan hakata ülikoolis programmeerimist ja muid huvitavaid asju 
õppima, aga tuli välja, et kõigepealt tuli läbida füüsika ja 
matemaatika. Mul oli matemaatikast natuke kopp ees, kuna 
meil oli gümnaasiumis väga püüdlik matemaatikaõpetaja ja tegelesime 
matemaatikaga põhjalikult, nii et ülikooli sissesaamise probleemi 
ei olnud -- matemaatikaeksamist lihtsalt 
lendasime läbi.

Ja nii see ülikool järgmisel aastal pooleli jäi.

\question{Kuidas sul kaitseväega lood on?}

Seejärel tuligi kaitsevägi\index{Kaitsevägi}. Kui ülikoolis ei õpi, siis varem või 
hiljem leitakse sind üles. Kaitseväkke läksin 1997. aasta suvel ehk 
olin siis juba aasta otsa Netit teinud. 

Ahjaa, et kuidas ma sinna sattusin. Töö Elektriside Inspektsioonis\index{Riigi 
Elektriside Inspektsioon} hakkas pisut ära tüütama, tahtsin 
edasi areneda ja kuhugi huvitavasse kohta tööle 
minna. Mul tekkis soov kindla peale töötada arvutifirmas, et saada arvutitele väga 
lähedale.

Vanu \emph{backup}'e läbi kammides jäi silma Helmes\index{Helmes} ja ma isegi kandideerisin sinna, aga ei saanud. Õnneks, mõtlen ma nüüd tagantjärele. Keskkooli ja ülikooli vahelisel suvel töötasin poolteist kuud 
Tõnu Samueli\index[ppl]{Samuel, Tõnu} IT-firmas nimega Eramees\index{Eramees} 
ja maandusin samale kohale, kust oli just lahkunud Pronto\index[ppl]{Pronto}. 
Tõnu ütles mulle, et Pronto müüs Gravis 
Ultrasoundi\sidenote{Üheksakümnendatel väga populaarsed helikaardid, mis 
esimesena omataoliste hulgas suutsid toimetada pärisinstrumentide 
sämplingutega.} kaarte ja et hakkaksin ise sellega tegelema. Aga ma olin 
noor koolipoiss ega teadnud kaubandusest mitte midagi. Vaevalt
minust seal ettevõttes muud erilist kasu oli, kui et olin nii-öelda patsiga poiss. 

\question{Päris mitmed inimesed on ühel hetkel tegelenud müügitööga ja üldse mitte 
halvasti.}

Eramehest on mul veel üks asi eredalt meeles. Tõnu BBS oli tal 
kontoris, mis asus Eesti Talleksi majas, Mustamäe tee 1, kui ma ei 
eksi\sidenote[][-8mm]{Siiski Mustamäe tee 4.}. Ja see BBS kujutas endast aknalaual laiali laotatud arvutijuppe: seal 
oli USR Courieri\index{US Robotics!Courier} modem\sidenote[][-8mm]{US Roboticsi 
 Courier tooteliin oli oma töökindluse ja suurte kiiruste tõttu 
BBSide ja varaste internetipakkujate lemmik, ka Eestis.}, emaplaat, toiteplokk 
ning hunnik juppe ja juhtmeid. See siis oligi Tõnu BBS või \emph{node}.

Pärast Erameest kandideerisin Estpak Datasse\index{Estpak Data}, sest mulle 
tundus, et ISP on tegelikult veel huvitavam asi, ja nad 
tegelesid internetiga.

\question{Kas Estpak oli tol ajal juba Eesti Telefoni oma või veel eraldi?}

See oli eraldi. Kui ma õigesti mäletan, siis Estpak Data omanik oli 
Eesti Telekom\sidenote[][-2.8cm]{Eesti Telekom ehk pika nimega Riigiettevõte Eesti 
Telekommunikatsioonid oli Teede- ja Sideministeeriumi haldusalas töötav 
\emph{holding}-ettevõte, mis valdas Eesti Telefoni, Eesti Mobiiltelefoni, Eesti 
Kaugotsingu, EsData, Estpak Data ja TeleMedia aktsiaid. Hiljem viidi ettevõte 
börsile ja ainuomanikuks sai Telia.}, mitte Eesti Telefon. See oli Eesti Telefonist täiesti eraldiseisev ettevõte. Huvitaval kombel oli kellelgi 
tulnud idee edendada veebi 
virtuaalhostimist. Keegi oli välja mõelnud neti.ee\index{neti.ee}-nimelise 
domeeni, mille alt üritati müüa traditsioonilist 
veebihostingut. Tollal see veel traditsiooniline ei olnud, aga 
tänapäeva mõistes küll. Estpak Data palkas mu veebihalduriks,
kes pidi hoolitsema veebi hostinguserveri ja -teenuse eest. Muu seas tekkis 
neil mõte, et kuidas veebihostingu äri ikka muudmoodi 
edendada, kui et on vaja kataloogi. Inimesed peavad ju 
need veebilehed, mida kliendid sinna panevad, üles leidma.

\question{Kas tol ajal oli Meediamaa juba olemas?}

Meediamaa\index{Meediamaa} startis umbes samal ajal. Enne seda oli olemas 
Eesti veebisaitide nimekiri, mis oli nlibi ehk 
Rahvusraamatukogu\index{Rahvusraamatukogu} domeenis, kus tegutses Toomas 
Mölder\index[ppl]{Mölder, Toomas}. Ilmselt kolis tema
selle nimekirja Meediamaasse ja sealt www.ee\index{www.ee}-sse. Kuna 
Meediamaa üks tegelane oli Tarvi Martens\index[ppl]{Martens, Tarvi}, siis neil 
õnnestus EENetilt\index{EENet} välja meelitada domeen nimega 
www.ee\sidenote{Alates oma asutamisest 1993. aastal kuni 2013. aastani oli 
EENet .ee domeeni registripidaja ja rakendas mitmeid suhteliselt rangeid 
reegleid. Näiteks oli domeeni registreerimine küll tasuta, kuid ühel 
organisatsioonil tohtis olla vaid üks domeen.}. Ma arvan, et mitte kellelegi 
teisele kui Tarvile ei oleks sellist domeeni elu sees välja antud.

\question{Kas sa kataloogi tegid algul käsitsi?}

Jah, alguses käsitsi, see oligi väga 
algeline ja puine. Asi hakkas lendama siis, kui kutsusin appi Jaanus Vainu\index[ppl]{Vainu, Jaanus}, kellega tutvusime 
Riigi Elektriside Inspektsioonis\index{Riigi Elektriside Inspektsioon}. 
Jaanus on ka omamoodi huvitav tegelane. Inspektsioonis mõtles tema 
välja kogu meie FM 108 sageduse plaani ehk kõik Eesti raadiojaamade 
sagedusnumbrid on tema tehtud. Nõukogude ajal oli meil teistsugune FM 
sagedusala, et takistada raadiost 
välismaiste raadiojaamade kuulamist. Eesti Vabariigi alguses 
koliti lääne sagedustele üle. Jaanus oli üks nendest, kes käis mööda Eestit 
mõõtmas ja tegi sagedusplaani. Tal joonistas väga detailselt Corel Draw's\index{Corel 
Draw} kõik sagedusringid Eesti kaardile. Eesmärk 
oli planeerida sagedused nii, et saatjatel oleksid kogu Eestis sagedused, 
millel on võimalikult vähe häireid naaberriikidega ja omavahel. 

\question{Kas kogu seda teadust tehti Corel Draw abil?}

Jah. Jaanus on tohutu pedant ja suure töövõimega katalogiseerija. 
Tema enda isiklik huvi on \emph{bluegrass}-muusika. Mäletan, et tema oli esimene 
inimene minu tutvusringkonnas, kes välismaalt e-poest asju tellis, näiteks plaate 
CDNow'st\sidenote{CDNow oli 1994. aastal asutatud internetipõhine muusikamüüja, 
kes paraku ei elanud esimest dot.com-mulli üle ja sulges sajandivahetusel uksed.}, ja imestasin, kuidas selline asi üldse 
võimalik on. Ta tellib jumal teab kust CD ja see tulebki pakiga kohale.

\question{Jaa, isegi üheksakümnendate lõpus oli Amazonist raamatute tellimine 
suhteliselt eksootiline tegevus. Aga mis hetkel ja kuidas te 
neti.ee\index{neti.ee} automatiseerisite?}

Meie tandem Jaanusega töötas selles mõttes ülihästi, et mina olin 
programmeerija ja arendasin tarkvara ning Jaanus oli katalogiseerija. Kui ta 
projektiga liitus, siis hakkas see täielikult 
lendama. Meil läks paar kuud aega, kui olime 
Meediamaast\index{Meediamaa} igatpidi kõikide näitajate poolest mööda läinud. 
Olime tollal ajal võibolla isegi natuke liiga ebaviisakad noored mehed. 
Näiteks reklaamisime netit spämmides: tegime 
masspostituse, saates kõikvõimalikele meiliaadressidele teate, et nüüd 
on selline huvitav teenus olemas nagu neti.ee, tulge ja külastage. Kui vaatasin hiljuti enda \emph{backup}'e, siis 
avastasin, et nimetasin oma \emph{crawler}'it ehk otsingurobotit, kes mööda lehti 
ringi kolab, Nuhiks. 

Huvitaval kombel olin Nuhi programmeerimist alustanud juba mitu kuud 
varem ehk miski oleks mind nagu suunanud sellele teele, et seda võib vaja 
minna. Otsingumootoreid olin ka varem pisut teinud. Kui ma pärast 
Erameest ülikooli läksin, siis üks sealt saadud tuttavatest kutsus mind 
tegema üht ärikataloogisarnast teenust Bartanet. See 
asus EsData\index{EsData} Suni serveris Akadeemia tee 21 
teisel korrusel, samas majas, kus me hetkel viibime. Ja selles Suni serveris 
sain teha FTP-serverite otsingut. Panin püsti otsinguteenuse Filerix, mis töötas umbes 
kolm-neli kuud ja võimaldas väga hõlpsasti 
faile üles leida igasugustest kohalikest FTP \emph{mirror}'itest. 
Marek Tiits\index[ppl]{Tiits, Marek} hostis tollal IBSist\index{Institute of Baltic Studies} 
sellist asja nagu TuCows\sidenote{TuCows (The Ultimate Collection 
Of Winsock Software) keskendus oma algusaegadel tasuta tarkvarale. Kuna 
interneti kiirus sõltus veel väga suurel määral geograafiast, hoidis
ettevõte käigus skeemi, kus huvilised võisid jooksutada lehekülje TuCows.com 
lokaalseid peegleid. Ühte sellist Marek pidaski.}. Minu otsingumootor 
võimaldas kergesti failinimede järgi üles leida tarkvara tolleaegsele Windows 
95-le, vanadele Windowsidele ja nii edasi. Nii et tolle pooleaastase projekti kõrvalprojektina tegin failiotsingut.

\question{Suure hulga failide indekseerimine ei ole naljaasi, vaid 
eeldab programmeerimisoskust. Kust sa selle üles korjasid?}

Tol hetkel oskasin ma programmeerida Perli\index{Perl} ja kõike 
seda, mis Unixi \emph{shell}'is oli saada. See oskus tuligi sellest perioodist, 
kui uurisin, mis on nii-öelda Unixil kõhus.

\question{Kas sa korjasid algoritmika ja muu sellise ise üles?}

Jah, aga mis puudutab veebi \emph{crawl}'imist, siis selle peale tuli juba 
mõelda.

\question{Kaua su \emph{crawler}'il aega läks, et 
kogu Eesti veeb üle käia?}

Umbes ööpäev, veeb oli 
tollal väga väike. Kataloogi suurus võis olla paar tuhat linki, mitte rohkem. 
Keskmine koduleht oli ka kolm kuni viis lehekülge. Huvitavamaks läks pärast, kui linkide hulk ulatus juba 
miljoniteni. Ühel hetkel oli käigus selline \emph{crawler}, mis 
töötas paralleelselt paljudes \emph{thread}'ides, aga see oli 
loomulik evolutsioon. 

Estpak Datasse\index{Estpak Data} võeti mind ilmselt tööle seetõttu, et olin ühe sellise kõrvalprojektina 
teinud HTMLi tutvustuse. Pidin seda siis, kui Keila gümnaasiumis
serverit administreerisin, kellelegi õpetama, kuna 
eestikeelset materjali polnud ja tegin ise ühe esimese eestikeelse 
HTMLi tutvustuse, mis võttis läbi kõik üksikud elemendid.

\question{See tuleb tuttav ette, olen sealt ilmselt isegi infot otsinud.}

See HTMLi tutvustus on samal aadressil praegu ka üleval ja ma 
olen üsna kindel, et see on üks vanemaid veebilehti 
Eesti veebiruumis, mis on originaalkujul originaalaadressil. 

\question{Mis aastast see on?}

Aastast 1996. Olen muide ühe projektina teinud veel ka veebipokkeri. Nii et ei saa öelda, nagu mul poleks kunagi huvi olnud 
mänge teha, aga rohkem olen programmeerinud nii-öelda 
veebiasju kui \emph{desktop}'is või masinas töötavaid rakendusi. 

Nende teadmiste baasil mind Estpaki tööle võeti. Tõenäoliselt 
näitasingi neile veebipokkerit ja 
HTMLi tutvustust ning võibolla rääkisin ka seda, et olen 
\emph{crawler}'i teinud. Igal juhul mind võeti tööle.

\question{Kes teil tootepoolt tegi või polnud siis veel niisugust mõistet nagu 
tootejuht?}

Ei olnudki. Piltlikult öeldes pandi mind laua taha istuma, et palun tee. 
Tegelikult oli see mõnes mõttes ikkagi läbi mõeldud. Estpak Data\index{Estpak 
Data} tegi koostööd reklaamiagentuuriga PRC Nord Decor\index{PRC Nord Decor}, mis rentis ruume Kullo majas 
Mustamäe teel. Nii et minu füüsiline töökoht asuski seal. Mul oli arvuti, millel oli püsiühendus 19.2 
kilobitti sekundis. Noore mehena ei huvitanud mind, kuidas raha liigub, vaid ainult 
tehniline pool. Idee seisnes selles, et reklaamiagentuur aitas 
potentsiaalsetel Estpak Data klientidel teha kodulehti ja neile 
reklaami. Üks kolleeg, Tiit Sermann\index[ppl]{Sermann, 
Tiit}, kes Nord Decoris töötas, oli kunagise 
OK jutuka\index{OK jutukas}\sidenote{OK jutukas oli üks esimesi massidesse läinud 
sotsiaalvõrgustikulaadseid rakendusi Eestis. Jututube ehk kohti, kus sai üle 
telneti kaaskodanikega suhelda, oli teisigi, aga 1996. aastal käivitatud OK oli 
esimesi veebipõhiseid jutukaid ja tõenäoliselt omataolistest siin kandis 
suurim. Üheaegselt lobises omavahel kuni 300 inimest ja jutuka esimese 
aastapäeva pidu kajastas isegi toonane Päevaleht.} üks asutajatest. Teine oli
Kaupo Kalda\index[ppl]{Kalda, Kaupo}. Naljakas oli see, et Tiidu alias oli Ott \sidenote{OK tulenes asutajate nimedest Ott ja Kaupo.}, kuigi
pärisnimi oli Tiit. Praegu tundub, et kogu see 
maailm oli tollal nii pisikene, et kui natukenegi seal ringi 
käisid, siis puutusid paratamatult kõikide nende inimestega kokku, 
kes siis toimetasid.

\question{Räägi palun sellest, kuidas te Hoti tegite.}

Kaitseväest tagasi tulles oli Eesti 
Telefon\index{Eesti Telefon} Estpak Data\index{Estpak Data} ära söönud, see 
lakkas olemast. Ma töötasin Eesti Telefoni teleteenuste arenduse
allüksuses, kelle eesmärk oli välja töötada uusi teenuseid, ja neti.ee tegemisega sattusimegi sinna. 

Kontoriruumi jagasin ühe noormehega, kes arendas 
sissehelistamisteenust. Meil vedeles kapi peal üks pisike 
Ascendi sissehelistamiskeskus ja ma küsisin, 
kas võin seda uurida.

\question{Kas selle külge käisid tavalised modemid 
või oli see juba valmislahendus?}

Ei, see oli spetsiaalne sissehelistamiskeskus: see tuli 
installeerida \emph{rack}'i ja panna juhtmed külge, et see hakkaks numbreid 
kuulama ja teenust osutama. Keskust 
uurides avastasin, et see autendib ennast 
vastu sellist autentimisserverit nagu Radius. Edasi uurides sain teada, et Radius on lihtne sõnastikupõhine protokoll, 
ja nii ma programmeerisingi Radiuse serveri, mis suutis 
sissehelistamiskeskust juhtida. Avastasin ka, et sissehelistamiskeskuse 
\emph{firmware} võimaldas teha igasuguseid huvitavaid asju, näiteks sai kohe Radiuse serverist öelda 
sissehelistamiskeskusele, kui kaua konkreetne kasutaja võib ühenduses olla. 

Sellest teadmisest sündis näiteks selline toode nagu Atlas Surf\index{Atlas Surf}, mida 
Eesti Telefon ettemaksulise internetina\sidenote{Sarnane kontseptsioon nagu mobiiltelefoni kõnekaardid.} müüs. Ühesõnaga, see toode sündis 
sellest, et häkkisin väikest sissehelistamiskeskust, mis oli 
mõeldud mobiiliga sissehelistamiseks. Too keskus toetas V.35 protokolli, millest paljud pole ilmselt kunagi 
kuulnud, aga see oli \emph{wideband}-protokoll, mis töötas üle GSMi. 
Kui sul oli GSM-telefon, mida sai arvutiga ühendada, siis see võimaldas V.35 protokolliga
sisse helistada ja kiirus oli veidi suurem kui 
tavalise modemiga üle mobiili vilistades. 

Hüppan korraks veel minevikku. Oli aasta 2000, ilmselt kõik mäletavad 
Y2K\sidenote{Nagu kogenud programmeerijad ütlevad: \enquote{Sinu lapselapsed neavad päeva, mil sa otsustasid oma 
koodi optimeerida}. Kuna pikka aega optimeeriti koodi hoides aastaarvu  kahekohalise 
numbrina, kulutati aastatuhande vahetuse paiku üüratus koguses tööaega ja raha, 
tagamaks, et aasta 2000 ei oleks arvutite arvates võrdne aastaga 1900. Aastal 2038 ootab meid sarnane probleem, kui Unixi aeg oma andmetüübi jaoks liiga suureks kasvab.} probleemi: kardeti, et arvutid 
lähevad katki, sest nende kell lakkab aastatuhande vahetusel töötamast. Ka Eesti Telefonis 
kardeti seda, sest \emph{legacy}-süsteeme oli tohutult palju. Kõik süsteemidega seotud insenerid pidid jääma valvesse. Ma ei mäleta, kuidas 
mul õnnestus sellest kõrvale nihverdada, aga tol hetkel olin sõpradega Soomes 
suusatamas ja lumelauaga mäest alla laskmas. Paar päeva enne 
aastavahetust tuli mulle klienditeenindusest kõne, et enam ei saa sisse 
helistada. Läksin autosse, kus mul oli sülearvuti, panin telefoni arvuti külge, helistasin 
V.35 protokolliga meie privaatkeskusse sisse ja 
hakkasin uurima, miks Hoti kliendid ei saa sisse helistada. 
Tuli välja, et keegi oli viimasel hetkel Y2K hirmus peale laadinud ühe turvapaiga, mis muutis Radiuse serverile 
minevat teadet, mispeale Radius läks katki, kuna sellele tuli tundmatu 
sisuga \emph{dictionary}.

Surfist edasi juhtus nii, et Eesti Telefoni kontsessioonileping 
oli juba lõppenud või lõppemas ja turule tuli Tele2\index{Tele2} Rootsist. 
Tele2 idee oli korrata Eestis täpselt sama, mida Rootsis: nad
soovisid suurelt \emph{telco}'lt palju raha välja imeda. Kuna Eesti 
Telefon üüris ruume ja liine, oli meile teada, et Tele2 paneb oma 
sissehelistamiskeskuseid püsti. Eesti Telefoni juhtkond oli paanikas, 
ma ise ka külastasin laiendatud juhatuse koosolekut, kus 
seda arutati. Tulin sealt üsna mornilt tagasi -- mulle 
tundus, et vanemad kolleegid ei suuda midagi otsustada ega ära teha. Mina 
noore mehena oleksin tahtnud kohe tegutseda. 

Pidasin telefonikõne Priit Pirsoga\index[ppl]{Pirso, Priit}, kes oli selle valdkonna juht 
Eesti Telefonis, ja me otsustasime teha Eesti 
Telefoni osutatavale Atlas Starteri teenusele alternatiivse teenuse, sellepärast 
et Atlas Starter ei sobinud Tele2ga konkureerimiseks. Meil oli vaja 
teenust, kus kasutajate registreerimise protseduur oleks 
automaatne, st kasutaja registreeriks end ise. Kuna kuutasu poleks pärast Tele2 jampsi niikuinii enam olnud, siis 
ainukesed, mis maksid, olid kõneminuti hinnad. Tele2 lootis raha teenida sellest, et
termineerib kõnet ja Eesti Telefon on sunnitud talle 
vahendama kliendi käest küsitud kõneminuti hinda. 

Selle telefonikõne käigus me leppisime kokku, kes, mida ja kuidas teeb, ja et toode saab 
nimeks Hot\index{hot.ee}. Ma olin siis juba arvutist järele vaadanud,
millised huvitavad domeenid olid vabad. Tollal oli veel see 
aeg, kui EENet\index{EENet} ei nõustunud andma ühele ettevõttele mitut 
domeeni, aga ühel mu praegusel kolleegil, Guido 
Kõivul\index[ppl]{Kõiv, Guido}, õnnestus saada EENetist meile
hot.ee domeen, sama skeemiga, nagu 
Tarvi\index[ppl]{Martens, Tarvi} ilmselt kasutas www.ee jaoks. Igatahes kõik käis ruttu ja kaks nädalat hiljem olime \emph{live}'is: 
meil toimus teenuse \emph{launch} ja kasutajaid hakkas registreeruma 
tempoga tuhat tükki päevas.

Sealt saigi hot.ee alguse ja minu teha jäi Radiuse pool. 
Hoti\index{hot.ee} puhul oli meie huvi see, et 
inimesed helistaksid meile sisse. Tollal hakati juba
kasutajatele ka meiliaadresse andma. Kuna varem küsiti meili eest raha, siis 
meile tundus, et lihtsalt niisama meiliaadresse jagada ei tahaks. Siis sai 
tehtud sedasi, et kasutaja sai küll veebipõhiselt konto luua, aga 
meili- ja ka kodulehekonto ei hakanud tööle enne, kui 
registreeritud kontoga oli tehtud vähemalt üks telefonikõne
sissehelistamiskeskusesse. Seda loogikat võimaldas minu \emph{custom} 
Radius, kes kasutajatel järge pidas. 

\question{Ühel hetkel oli hot.ee-s ka veebimeil, eks?}

Veebimeiler oli suhteliselt algusest peale esimese 
kujunduse osa, aga see ei olnud minu programmeeritud, vaid internetist 
leitud vabavara. Me isegi ei \emph{rebrand}'inud enda värvidesse, vaid see oli lihtsalt meie lehelt 
lingitud ja me ise majutaasime teda.

\question{See seletab, miks meil mõned aastad hiljem veebimeileri tegemine 
Hansapangas\index{Hansapank} nurja läks -- meil miskipärast ei tulnud pähe mõtet 
seda lahendust internetist alla laadida.}

Mulle ei tulnud pähe seda ise teha. Küll aga mäletan 
sellist huvitavat protokolli nagu WAP\sidenote{Wireless Application 
Protocol (WAP) oli sajandivahetuse paiku tehtud 
katse luua toona kasinate sidevõimalustega mobiiltelefonide jaoks lihtsamaid 
internetiprotokolle 4. kuni 7. OSI kihini. Muu hulgas sisaldas WAP
erilist \emph{markup}-keelt toonase mobiiltelefoni mõnerealisele ekraanile 
sobivate kasutajaliideste loomiseks.}, mis kujutas endast interneti mobiilivarianti. Selle WAP-meili tegin Hotile küll täiesti 
nullist.

\question{Õnneks see ei olnud väga pika elueaga, sest ka WAP ei kestnud kaua.}

EMT tollane arendusjuht Ando Meentalolt\index[ppl]{Meentalo, 
Ando} kommenteeris minu WAP-meili nii: \enquote{Sa võid ju sinna suahiili keele ka panna, aga 
ilmselt pole sellest väga palju kasu.} Mul sai WAPiga tegelemine 
alguse sellest, et olin saanud endale WAPi-võimelise telefoni (kusjuures see oli vist ainuke telefon, mida ma olen iialgi tööandjalt saanud). See oli suure ekraani ja klapiga
Nokia 7110\sidenote{Tegu oli 1999. aastal uskumatult innovatiivse 
telefoniga: mitut tekstirida näitav ekraan, rullikuga kasutajaliides, T9 
ennustav tekstisisestus sõnumite puhul, vedruga uhkelt lahti hüppav klapp, WAP, 
ebamaiselt küütlev korpus jne. Oma iseäraliku kuju tõttu sai aparaat rahva seas 
hüüdnimeks \enquote{banaan}.}. 

\question{See telefon oli suurepärane põhjus Hansapangale WAPi-põhine 
internetipank teha, sest selle testimiseks pidi ju pank ometigi väljastama ka 
sobiliku seadme.}

Mul juhtus \emph{vice versa}: kõigepealt sain telefoni ja siis 
tuli idee, et äge oleks enda postkasti vaadata sellisel mugaval moel. Ja siis tegingi 
WAP-meili.

\question{Sellega algab juba uus sajand ja sellest räägime võibolla 
mõni teine kord. Lõpetuseks küsin, mida sa praegu teed?}

Praegu teen Bolti\index{Bolt}\sidenote{Endise nimega Taxify ja asutatud kui mTakso.} serveri infrastruktuuri. Minu üks kauaaegseid 
kolleege Eesti Telefonist Tarmo Kople\index[ppl]{Kople, Tarmo} on 
üks nendest inseneridest, kellega alustasime Bolti 
serverimajandust algusest peale. Ja kui algul oli kliente ja sõite tuhandeid kuus, siis nüüd juba miljoneid.
