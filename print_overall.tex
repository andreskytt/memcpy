%!TEX TS-program = arara
% arara: latexmk: { clean: partial }
% arara: xelatex: { shell: true, synctex: true} 
% arara: makeindex
% arara: xelatex: { shell: true, synctex: true} 
% arara: xelatex: { shell: true, synctex: true} 
% arara: latexmk: { clean: partial }

\documentclass{tufte-book}
\usepackage[
    type={CC},
    modifier={by-nc-nd},
    version={4.0},
]{doclicense}


\ifxetex
  \newcommand{\textls}[2][5]{%
    \begingroup\addfontfeatures{LetterSpace=#1}#2\endgroup
  }
  \renewcommand{\allcapsspacing}[1]{\textls[15]{#1}}
  \renewcommand{\smallcapsspacing}[1]{\textls[10]{#1}}
  \renewcommand{\allcaps}[1]{\textls[15]{\MakeTextUppercase{#1}}}
  \renewcommand{\smallcaps}[1]{\smallcapsspacing{\scshape\MakeTextLowercase{#1}}}
  \renewcommand{\textsc}[1]{\smallcapsspacing{\textsmallcaps{#1}}}
\fi


\usepackage[T1]{fontenc}
%\usepackage[utf8]{inputenc}
\usepackage{polyglossia}
\setmainlanguage{estonian} 
\setotherlanguage{russian}
\newfontfamily\russianfont[Script=Cyrillic]{Linux Libertine}

\hypersetup{colorlinks}% uncomment this line if you prefer colored hyperlinks (e.g., for onscreen viewing)


%%
% Book metadata
%\title{print(memcpy[])\thanks{Thanks to Edward R.~Tufte for his inspiration.}}
\title{memcpy.print()}
\author[Andres Kütt]{Andres Kütt}
\publisher{TeamConsulting}

%%
% If they're installed, use Bergamo and Chantilly from www.fontsite.com.
% They're clones of Bembo and Gill Sans, respectively.
%\IfFileExists{bergamo.sty}{\usepackage[osf]{bergamo}}{}% Bembo
%\IfFileExists{chantill.sty}{\usepackage{chantill}}{}% Gill Sans

%\usepackage{microtype}

%%
% Just some sample text
\usepackage{lipsum}

%%
% For nicely typeset tabular material
\usepackage{booktabs}

%%
% For graphics / images
\usepackage{graphicx}
\setkeys{Gin}{width=\linewidth,totalheight=\textheight,keepaspectratio}
\graphicspath{{graphics/}}

% The fancyvrb package lets us customize the formatting of verbatim
% environments.  We use a slightly smaller font.
\usepackage{fancyvrb}
\fvset{fontsize=\normalsize}

%%
% Prints argument within hanging parentheses (i.e., parentheses that take
% up no horizontal space).  Useful in tabular environments.
\newcommand{\hangp}[1]{\makebox[0pt][r]{(}#1\makebox[0pt][l]{)}}

%%
% Prints an asterisk that takes up no horizontal space.
% Useful in tabular environments.
\newcommand{\hangstar}{\makebox[0pt][l]{*}}

%%
% Prints a trailing space in a smart way.
\usepackage{xspace}

%%
% Some shortcuts for Tufte's book titles.  The lowercase commands will
% produce the initials of the book title in italics.  The all-caps commands
% will print out the full title of the book in italics.
\newcommand{\vdqi}{\textit{VDQI}\xspace}
\newcommand{\ei}{\textit{EI}\xspace}
\newcommand{\ve}{\textit{VE}\xspace}
\newcommand{\be}{\textit{BE}\xspace}
\newcommand{\VDQI}{\textit{The Visual Display of Quantitative Information}\xspace}
\newcommand{\EI}{\textit{Envisioning Information}\xspace}
\newcommand{\VE}{\textit{Visual Explanations}\xspace}
\newcommand{\BE}{\textit{Beautiful Evidence}\xspace}

\newcommand{\TL}{Tufte-\LaTeX\xspace}

% Prints the month name (e.g., January) and the year (e.g., 2008)
\newcommand{\monthyear}{%
  \ifcase\month\or January\or February\or March\or April\or May\or June\or
  July\or August\or September\or October\or November\or
  December\fi\space\number\year
}


% Prints an epigraph and speaker in sans serif, all-caps type.
\newcommand{\openepigraph}[2]{%
  %\sffamily\fontsize{14}{16}\selectfont
  \begin{fullwidth}
  \sffamily\large
  \begin{doublespace}
  \noindent\allcaps{#1}\\% epigraph
  \noindent\allcaps{#2}% author
  \end{doublespace}
  \end{fullwidth}
}

% Inserts a blank page
\newcommand{\blankpage}{\newpage\hbox{}\thispagestyle{empty}\newpage}

\usepackage{units}

% Typesets the font size, leading, and measure in the form of 10/12x26 pc.
\newcommand{\measure}[3]{#1/#2$\times$\unit[#3]{pc}}

% Macros for typesetting the documentation
\newcommand{\hlred}[1]{\textcolor{Maroon}{#1}}% prints in red
\newcommand{\hangleft}[1]{\makebox[0pt][r]{#1}}
\newcommand{\hairsp}{\hspace{1pt}}% hair space
\newcommand{\hquad}{\hskip0.5em\relax}% half quad space
\newcommand{\TODO}{\textcolor{red}{\bf TODO!}\xspace}
\newcommand{\ie}{\textit{i.\hairsp{}e.}\xspace}
\newcommand{\eg}{\textit{e.\hairsp{}g.}\xspace}
\newcommand{\na}{\quad--}% used in tables for N/A cells
\providecommand{\XeLaTeX}{X\lower.5ex\hbox{\kern-0.15em\reflectbox{E}}\kern-0.1em\LaTeX}
\newcommand{\tXeLaTeX}{\XeLaTeX\index{XeLaTeX@\protect\XeLaTeX}}
% \index{\texttt{\textbackslash xyz}@\hangleft{\texttt{\textbackslash}}\texttt{xyz}}
\newcommand{\tuftebs}{\symbol{'134}}% a backslash in tt type in OT1/T1
\newcommand{\doccmdnoindex}[2][]{\texttt{\tuftebs#2}}% command name -- adds backslash automatically (and doesn't add cmd to the index)
\newcommand{\doccmddef}[2][]{%
  \hlred{\texttt{\tuftebs#2}}\label{cmd:#2}%
  \ifthenelse{\isempty{#1}}%
    {% add the command to the index
      \index{#2 command@\protect\hangleft{\texttt{\tuftebs}}\texttt{#2}}% command name
    }%
    {% add the command and package to the index
      \index{#2 command@\protect\hangleft{\texttt{\tuftebs}}\texttt{#2} (\texttt{#1} package)}% command name
      \index{#1 package@\texttt{#1} package}\index{packages!#1@\texttt{#1}}% package name
    }%
}% command name -- adds backslash automatically
\newcommand{\doccmd}[2][]{%
  \texttt{\tuftebs#2}%
  \ifthenelse{\isempty{#1}}%
    {% add the command to the index
      \index{#2 command@\protect\hangleft{\texttt{\tuftebs}}\texttt{#2}}% command name
    }%
    {% add the command and package to the index
      \index{#2 command@\protect\hangleft{\texttt{\tuftebs}}\texttt{#2} (\texttt{#1} package)}% command name
      \index{#1 package@\texttt{#1} package}\index{packages!#1@\texttt{#1}}% package name
    }%
}% command name -- adds backslash automatically
\newcommand{\docopt}[1]{\ensuremath{\langle}\textrm{\textit{#1}}\ensuremath{\rangle}}% optional command argument
\newcommand{\docarg}[1]{\textrm{\textit{#1}}}% (required) command argument
\newenvironment{docspec}{\begin{quotation}\ttfamily\parskip0pt\parindent0pt\ignorespaces}{\end{quotation}}% command specification environment
\newcommand{\docenv}[1]{\texttt{#1}\index{#1 environment@\texttt{#1} environment}\index{environments!#1@\texttt{#1}}}% environment name
\newcommand{\docenvdef}[1]{\hlred{\texttt{#1}}\label{env:#1}\index{#1 environment@\texttt{#1} environment}\index{environments!#1@\texttt{#1}}}% environment name
\newcommand{\docpkg}[1]{\texttt{#1}\index{#1 package@\texttt{#1} package}\index{packages!#1@\texttt{#1}}}% package name
\newcommand{\doccls}[1]{\texttt{#1}}% document class name
\newcommand{\docclsopt}[1]{\texttt{#1}\index{#1 class option@\texttt{#1} class option}\index{class options!#1@\texttt{#1}}}% document class option name
\newcommand{\docclsoptdef}[1]{\hlred{\texttt{#1}}\label{clsopt:#1}\index{#1 class option@\texttt{#1} class option}\index{class options!#1@\texttt{#1}}}% document class option name defined
\newcommand{\docmsg}[2]{\bigskip\begin{fullwidth}\noindent\ttfamily#1\end{fullwidth}\medskip\par\noindent#2}
\newcommand{\docfilehook}[2]{\texttt{#1}\index{file hooks!#2}\index{#1@\texttt{#1}}}
\newcommand{\doccounter}[1]{\texttt{#1}\index{#1 counter@\texttt{#1} counter}}

% Generates the index
\usepackage{imakeidx}
\makeindex[name=ppl, title={Nimede register}]
\makeindex[title={Indeks}]

% See also
\makeatletter
\newcommand{\gobblecomma}[1]{\@gobble{#1}\ignorespaces}
\makeatother

\usepackage{csquotes}


%% Versioneerimine

\newcounter{run}
\InputIfFileExists{\jobname.runs}{}{}
\stepcounter{run}

\usepackage{atveryend}
\usepackage{newfile}
\AtVeryEndDocument{%
  \newoutputstream{runs}%
  \openoutputfile{\jobname.runs}{runs}%
  \addtostream{runs}{\string\setcounter{run}{\number\value{run}}}%
  \closeoutputstream{runs}%
}

%% Küsimuse vormistus
\newcommand{\question}[1]{\textbf{\enquote{#1}}}
%\newcommand{\question}[1]{\begin{minipage}{\textwidth}\textbf{\enquote{#1}}\end{minipage}}

% Reset the sidenote number each chapter
\let\oldchapter\chapter
\def\chapter{%
  \setcounter{footnote}{0}%
  \oldchapter
}

\begin{document}

% Front matter
\frontmatter

% r.1 blank page
\blankpage

% v.2 epigraphs
\newpage\thispagestyle{empty}
\openepigraph{%
Design and programming are human activities; forget that and all is lost.
}{Bjarne Stroustrup%, {\itshape Design, Form, and Chaos}
}
\vfill
\begin{fullwidth}
\sffamily\large
\begin{doublespace}
%\noindent\allcaps{Ärge valetage isad }\\ % The quote
%\noindent\allcaps{ära hoia kinni ema mind}\\ % The quote
%\noindent\allcaps{Need ei ole halvad sõbrad}\\ % The quote
\noindent\allcaps{\ldots}\\ % The quote
\noindent\allcaps{see on minu Vennaskond ja ring}\\ % The quote
\noindent\allcaps{Vennaskond. \enquote{Jumal kaitse vennaskonda}} % The author
\end{doublespace}
\end{fullwidth}
%\vfill
%\openepigraph{% 
%Ärge valetage isad ära hoia kinni ema mind Need ei ole halvad sõbrad see on minu Vennaskond ja ring}{Vennaskond. \enquote{Jumal kaitse vennaskonda}}
%\vfill
%\openepigraph{%
%\ldots the designer of a new system must not only be the implementor and the first 
%large-scale user; the designer should also write the first user manual\ldots 
%If I had not participated fully in all these activities, 
%literally hundreds of improvements would never have been made, 
%because I would never have thought of them or perceived 
%why they were important.
%}{Donald E. Knuth}


% r.3 full title page
\maketitle


% v.4 copyright page
\newpage
\begin{fullwidth}
~\vfill
\thispagestyle{empty}
\setlength{\parindent}{0pt}
\setlength{\parskip}{\baselineskip}
Copyright \copyright\ \the\year\ \thanklessauthor

\par\smallcaps{Published by \thanklesspublisher}

%\par\smallcaps{tufte-latex.googlecode.com}

\par \doclicenseThis 

\par\textit{\monthyear. Version V0.\therun}
\end{fullwidth}

% r.5 contents
\tableofcontents

%\listoffigures

%\listoftables

% r.7 dedication
\cleardoublepage
~\vfill
\begin{doublespace}
\noindent\fontsize{18}{22}\selectfont\itshape
\nohyphenation
Toivole
\end{doublespace}
\vfill
\vfill


% r.9 introduction
\cleardoublepage
\chapter*{Sissejuhatus}
Juhatame sisse. 
\begin{itemize}
	\item Miks ma seda teen
	\begin{itemize}
		\item \enquote{Tahan kord saada selliseks, nagu on Villu või Freddy või Rott või Striit.}\sidenote{Villu Tamme, "Paneme punki"}
	\end{itemize}
	\item Eesmärk: kujutada inimesi ja nende suhteid (mitte näiteks kurioosseid hetki või ettevõtteid)
	\item Lühike ajalugu: idee, otsing, podcast, siis analüüs ja raamat
	\item Sisu kohta
	\begin{itemize}
		\item Kõik ei mahtunud raamatusse, kõik ei soovinud rääkida ja kõik ei tulnud pähe. Andestust!
		\begin{itemize}
			\item Välja on enamasti jäänud näiteks Mainor ja natuke vanema põlvkonna (näiteks kadunud Ahto Kalja ja Monika Oit'i) tegemised
			\item Tartu seltskonda on kahetsusväärselt vähe hulgas
		\end{itemize}
		\item Lood lähevad omavahel vastuollu, see on OK. Samas on otsesed kõrvalekalded teadaolevast reaalsusest osundatud ning, kui intervjueeritav kahtles, on üritatud õige välja tuua
		\item Kõik on isiksused. Mõned kergemad, mõned raskemad. Olen üritanud suhte-taagast üle olla
		\item Nii \enquote{läbipaistev} vaade kui võimalik. Sealhulgas näiteks ka ehk liigselt anglitsismirohke keelekasutus, sugude tasakaalu puudus jne.
		\item Mõeldud olema ka mitte-arvutiinimesele üldjoontes arusaadav: konteksti mõistmiseks olulised terminid on lahti seletatud, kuid detailid otsib huviline ise välja. Samas ei ole eesmärk anda struktureeritud ülevaadet arvutustehnika ajaloost või vanade tehnoloogiate toimimisest. Olen üritanud küsida võhiku positsioonilt. Mis on seda lihtsam, et paljus ma olen võhik.
		\item Inimesed tähestikulises järjekorras	
		\item Oma jutt on ka, sest muidu jääks juttudesse kummaline auk, lisaks tuleks ju anda aimu, mis prisma läbi ülejäänud asjad on kirjutatud. Intervjueerisin ennast ise
		\item \enquote{Patsiga poisid} kui üldnimetus. Enamasti siiski poisid. Kahju küll, aga nii oli. Raamat on läbilõige toonasest seltskonnast ja oleks vale toda seltskonda kuidagi teistsugusena kujutada
		\item Jutt on enamasti täies mahus nii, nagu räägitud sai, kirjakeelde pandud. Mõnes üksikus kohas lühendasin\sidenote{Tarvi jutt on kahe intervjuu kombinatsioon, seal tuli selguse huvides asju natuke ümber tõsta ja tihendada} ja mõnda üksikut asja ei pidanud paslikuks sisse jätta - Mõnest asjast ei taha inimesed väga rääkida ja mõnda asja ma ei taha avaldada. Üheksakümnendad oli päris hull ja teistsugune aeg. Need on siiski detailid ja suurt pilti ei muuda.. 
	\end{itemize}
	\item Kuidas lugeda
	\begin{itemize}
		\item On indeks, eraldi inimeste oma
		\item On lühikesed selgitused mainitud riistvara ja arvutite osas
		\item Detailsema jutu leiab igaüks ise Internetist
	\end{itemize}
	\item Tänuavaldused
	\begin{itemize}
		\item Rein Rüüsak, A\&A ajaloo välja uurimine
		\item Ott Köstner, memcpy kaanepilt
		\item Vootele Voit, info ZX Spectrumi kiibistiku kohta
		\item Kõik intervjueeritud
		\item Veebipõhine transkriptsioon (Alumäe, Tanel; Tilk, Ottokar; Asadullah. "Advanced Rich Transcription System for Estonian Speech" Baltic HLT 2018)
		\item Mroos toimetamine, kaasamõtlemine ja tehniline tugi
		\item Wikipedia
	\end{itemize}
	\item Küsimused
	\begin{itemize}
		\item The C Programming language raamat
	\end{itemize}
 \end{itemize}
 
\chapter*{Notes}
\begin{itemize}
	\item kadri.ut.ee ja madli.ut.ee
	\begin{itemize}
		\item Toomas Soome: Need olid Otto Telleri tütarde järgi
	\end{itemize}
	\item jutukad
	\begin{itemize}
		\item Tartus oli Helgeri Muumi, aga teised Anna konkurendid olid Lõvi VC ehk Rändom ja siis tulid Puhh ja Cafe.
		\item Puhh jooksis chem.ut.ee 6666
	\end{itemize}
	\item Kaur Virunurm: Ilmselt sellest ERAK-ist eraldus Tallinna 21. Koolile (siis veel N. V. Gogoli nimeline, Raadiomaja kõrvalmajas) igivana Ungari Videoton 1010B. Mingi Prantsusmaa miniarvuti kloon. Mille peal me siis rõõmsalt puhtas masinkoodis programmeerisime.
\end{itemize}

%%
% Start the main matter (normal chapters)
\mainmatter


\chapter{Andrus Aaslaid}
%!TEX TS-program = arara
% arara: myindex

\index[ppl]{Aaslaid, Andrus}
\textbf{\enquote{Kuidas sa arvutite juurde jõudsid}}

Tihti on nii, et meil on elu muutvad otsused aga me ei mäleta, kuidas me neid 
tegime. Aga neid seda juhust ma mäletan täpselt. Mul oli juba toona 
raadio-hobi. Olin selline põhikooli juntsu ja mulle meeldis hirmsasti mööda 
lühilainet ringi kammida. Meil oli kodus selline Melodija 101 stereo, Riia 
raadiotehase\sidenote{A. S. Popovi nimeline Riia Raadiotehas, alates 1951 Rigas 
Radio Rupnica} toodang. Sellega ma siis seiklesin suviti, kui midagi targemat 
teha ei olnud, mööda eetrit. Mul oli tegelikult kaks raadiot: esimene juba 
mainitud Melodija, sellele lisaks veel detektorvastuvõtja, mille mu poolvend 
oli mulle ehitanud. Selle viimasega ma istusin pööningul, kus oli muu seas mitu 
aastakäiku ajakirja \begin{russian}Техника - 
молодёжи\end{russian}\sidenote{Aastast 1993 ilmuv algselt Nõukogude ja nüüd 
Vene populaarteaduslik ajakiri.}. Kuna perekond tegeles põllumajandusega ja oli 
üks konkreetne põllumajandusnipp. Nimelt raamatukogudest toodi vanu ajakirju, 
need rebiti lehtedeks, need lehed keerati sisuliselt ümber sellise õõnsa 
põhjaga pudeli sellisteks väikesteks pottideks ja  sinna sisse istutati taimed. 
Paber lagunes mulla sees, ära taim pääses põllul vabaks. Ka neid ajakirju oli 
seal tohutu hunnik ja nende ajakirjade juures oli mitu aastakäiku 
\begin{russian}Техника - молодёжи\end{russian}'t. Lappasin siis neid ajakirju, 
detektoriklapid peas. 

Igatahes ükskord astusin ma tuppa, lülitasin Melodija sisse ja sealt öeldi, et 
Tallinna 43. Keskkool\index{Koolid!Tallinna 43. Keskkool} on otsustanud hakata 
eksperimentaalseks Tehnikaülikooli\index{Tallinna Tehnikaülikool}\sidenote{Tol 
ajal oli ta veel Tallinna Polütehniline Instituut, TPI} ettevalmistuskooliks ja 
nad võtavad kümnendasse klassi vastu õpilasi, kes tahaksid edasi õppima minna 
TPIsse. Kuulasin uudise ära, lülitasin raadio välja, läksin vanemate juurde ja 
teatasin, et ma lähen Tallinnasse kooli. Ma olin siis 14.


\textbf{\enquote{Aga kust sa siis pärit oled, et niimoodi Tallinna kooli pidi 
minema?}}


Pärit olen ma tegelikult kahesaja meetri kauguselt sealt, kust ma täna elan, 
ehk siis Tallinnast. Aga mul perekond otsustas evakueeruda 
Muhusse\index{Muhumaa}, kui ma olin kahe- või kolmeaastane, siis mind 
deporteeriti sinna. Nii et oma põrsapõlve olen kõik Muhus  veetnud ja siis ühel 
hetkel sealt siis tagasi putku tulnud tehnoloogia juurde. 

\textbf{\enquote{No mõni ime, et te Mastiga\index[ppl]{Kaal, 
Madis}\index[ppl]{Mast|see{Kaal, Madis}} hästi läbi saate!}}

Me oleme Mastiga ühe kooli poisid tegelikult selles mõttes, et Mast oli samas 
koolis keskkoolis, kui mina olin seal põhikoolis. Me oleme isegi sama 
raadiosõlme väisanud mõnda aega. Aga tollel ajal noh, nagu ikka, eriti veel 
maakohtades, et ega nooremad ja vanemad väga läbi ei käinud. Aga Mast oli hea 
poiss, ei peksnud nooremaid ega midagi. 

\textbf{\enquote{Mis seal lühilaine pealt kostis, mida sa kuulasid? Muusikat?}}

Ei muusikat kuulati raadio Luksemburgist. Sealt lühilaine pealt tulid erinevad 
hääled. Tuli morset, tuli mingeid huvitavaid kahinaid ja sahinad ja siis keegi 
luges numbreid ja. Ega tegelikult lühilaine on siiamaani päris hea tervise 
juures, et seal on  eeter siiamaani on  maast laeni sodi täis, et ega ta olemus 
ei ole, väga palju muutunud. Võib-olla mingeid propagandasaateid on vähemaks 
jäänud ja Hiina raadiojaamu on vaikselt kinni pandud seoses sellega, kuidas 
Internet peale tuleb. Aga, üldiselt on see lühilaine samasugune, nagu ta oli 
nelikümmend aastat tagasi, ma arvan.

\textbf{\enquote{Kas seal sinu ajakirjade hulgas juba arvutiajakirju ka oli?}}

Esimest arvutit ma nägin tegelikult just tänu sellelesamale poolvennale, kes 
mulle selle detektori ehitas. Ühel hetkel ta ütles, et Guido 
Tammissaar\index[ppl]{Tammissaar, Guido}, tema oli Eesti Energia 
Arvutuskeskuse\index{Eesti Energia Arvutuskeskus} üks tegelastest. Ja ühel 
hetkel tuli sinna maale ja ütles, et tule kaasa paariks päevaks, näed, mis asi 
see arvuti on. Et sind see tehnika, asi huvitab. Ja lubati mind siis maalt 
linna paariks päevaks, Estonia puiestee arvutuskeskuses olid veel põhiliselt 
ESM-id tollel ajal\sidenote{Kas \url{https://en.wikipedia.org/wiki/BESM}? }. Ja 
üks mingisugune CP\textbackslash Mi\sidenote{CP\textbackslash M oli 1974. 
aastal Inteli 8080/85 protsessorisarja tarvis turule toodud 
operatsioonisüsteem, mille 1980ndatel asendas mitmes mõttes sarnane MS-DOS} 
masin, mis tagantjärgi tundub jube kosmiline selles mõttes, et ta tundus 
mingisugune sotsmaa disain, eks ta, mingi Bulgaarlane oli. Olen mõelnud, et 
peaks üles otsima, et  masin see selline võis olla, aga ma siiamaani täis sada 
protsenti kindel ei ole. 

Selle peale ma niisama natuke klõbistasin. 
SM-4\index{Arvutid!SM-4}\sidenote{SM-4 oli PDP-11/40\index{PDP-11} ühilduv 
Nõukogude päritolu ja terves Idablokis toodetud arvutisüsteem} peal ma 
kirjutasin oma esimese BASICu\index{Keeled!BASIC} programmi sellel samal 
päeval. See oli derivaat mingist asjast, mida mulle näidati, et näed, umbes nii 
käib. Ja edasi ma olin \emph{hooked}. Sellest ühest päevast piisas, et sõltlane 
tekitada. 

\textbf{\enquote{See oli siis enne seda, kui sa otsustasid, et nüüd sa oled 
neliteist ja lähed Tallinnasse kooli?}}
Ma ei oskagi öelda, ma ei ole sada protsenti kindel, kumb oli enne, kumb oli 
pärast. Et et kas, kas see huvi, et tulla Tallinnasse, mängis rolli. Ega nad ju 
arvutikallakut tegelikult ei propageerinud.  Suurema rõhuga oli  elektroonika, 
tarkvara osa, seda nad väga ei reklaaminud. Minust pidi ikka elektroonik saama 
tegelikult, mis minust nüüd ka vahepeal sai, aga tollel hetkel ikkagi arvutid 
tundusid nagu see päris asi. 

\textbf{\enquote{Kas selles 43. keskkoolis valmistati päriselt ette ka 
ülikooliks? Oli sellest kasu?}}

See oli selline kahe teraga mõõk selles suhtes, et valmistati ette ja 
valmistati väga hästi. Sellepärast et see keskkooliprogramm oli pandud kokku, 
tolle aja inseneri õpetajate poolt, kes teadsid suhteliselt hästi, mida oleks 
vaja õpetada selleks, et põhi alla tuleks. Mis tähendas seda, et me saime 
läbisegi  tavalisi keskkooliaineid ja siis ühel hetkel tuli härra 
Tiidemann\index[ppl]{Tiidemann, Tiit} ja hakkas meile rääkima võllide 
epüüridest\sidenote{Epüür (prantsuse sõnast épure) on teatava suuruse asukohast 
olenevate väärtuste graafiline esitus}. Sisuliselt me tegime käsitsi võllidele 
rakendavate jõudude arvutusi, et kust kohast läheb võll katki, kui ta on 
sellise jämedusega siit tollase jämedusega sealt. Siis vahelduseks loeti meile 
mingisugust teise kursuse elektrotehnikat. Sinna vahele me saime mingisugust 
inseneripsühholoogiat, mida Toomsalu\index[ppl]{Toomsalu, Arvo} luges, mis ei 
olnud vist üldse TPI õppekavas sees. Ta oli nagu kõikidele eksperiment. 
ETEK\index{ETEK}  oli tema lühikene nimi, Ants Reili\index[ppl]{Reili, Ants} ja 
Peeter Krosberg\index[ppl]{Krosberg, Peeter} teda tegid. Ta oli selles mõttes 
väga äge üritus, et ta oli ikkagi täiesti \emph{green-field}. Eriti kuna me 
olime esimene lend. 

Meil olid muidugi seal veel omaette sellised lahedad asjad nagu see näiteks, et 
enne meid oli keskkool tühjaks löödud ja me olime kolm aastat keskkooli kõige 
vanem klass. Mis tähendas seda, et me olime sisuliselt nagu jumalad koolis. 
Tänu sellele mitmed probleemid jäid olemata, mida muidugi tavalistes 
keskkoolides tol ajal veel eksisteeris. Keegi kedagi väga ei toginud ega 
nüginud ja samal ajal kuidagi see areng toimus, nii et, et mingisugune väärikus 
tekkis kõigile. 

Kahe teraga mõõk oli ta sellepärast, et ta andis nii kõva põhja, et väga paljud 
läksid otse tööle. Me saime ju kõik, kes keskkooli lõpetasid, see on 
automaatselt TPIsse sisse. Meil ei olnud vaja sisseastumiseksamit teha. Mis 
tähendas, et kogu see vist kaheksateist õpilast marssis otse TPIsse. Nendest 
kooli lõpetas nominaalajaga vist  kas üks inimene või kaks või kolm, ma päris 
ei mäletagi. Ikkagi käputäis. Hästi paljud läksid otse tööle. Kuna aeg oli ka 
selline, et et kuidagi see, mida TPIs tollel ajal arvutiteadusena õpetati 
ikkagi päris elule veel järgi ei jõudnud. See pidi olema aasta 91-92. Siis, kui 
see kambriumi plahvatus siin Eestis toimus.

Ühest küljest  mina istusin arvuti taga ööd ja päevad ja kirjutasin 
mingisugusele suurele autopargile ihuüksi  mingisugust tarkvara, mis pidi 
üleval hoidma tervet autoparki. Ja samal ajal siis üritasin kuidagi nügida 
ennast läbi SuperCalci\index{SuperCalc}\sidenote{Varajane tabelarvutussüsteem, 
algselt loodud CP\textbackslash M operatsioonisüsteemile} arvestusest TPIs kus 
aegajalt tuli nagu õppejõule näidata, et ära nii tee, nii päris ei käi see asi. 
Mitte, et nad oleks rumalad olnud, aga nemad õpetasid seda, mida nad olid kogu 
aeg õpetanud. Aga nüüd tekkis ühel hetkel selline seis, kus reaalne elu liikus 
palju kiiremini kui õppekava.

\textbf{\enquote{Aga kuidas sa selle programmeerimise juurde ikka jõudsid? Sa 
pidid ikka kuskil harjutada saama seda?}}

See oligi  tänu sellele 43. keskkoolile, noh, tänane siis 
Tehnikagümnaasium\index{Koolid!Tehnikagümnaasium|see{Tallinna 43. Keskkool}}, 
et, see eksperiment kestis ja mõnes mõttes kestab tänaseni. Seal oli 
põhimõtteliselt  esimest korda selline päris arvuti-inimese elu. Kuna 
IT-spetsialiste  liiga palju ei olnud, siis juhtus selline hämar lugu, et meile 
Eero Tohvriga\index[ppl]{Tohver, Eero} ulatati arvutiklassi võtmed ja hakati 
koolist palka maksma kümnendas klassis. See natukene vist oli tegelikult seotud 
sellise koolipoolse kerge kaastundega. Sellist otseselt tööstuskooli peale 
kaheksandat klassi tulemise traditsiooni juba mõni kümmend aastat ei olnud 
vahepeal olnud ja kõigile tundus see, et laps tuleb üksi Tallinnasse kangesti 
hirmus. Ja kuidagi ma arvan, et see oli pigem selline koolipoolne stipendium. 
Aga jah, põhimõtteliselt meile maksti kahe peale täis õppejõu palk välja, mis 
oli põhimõtteliselt ma kahtlustan, et mitte palju väiksem, kui need õpetajad 
ise seal tollel hetkel said. Nii hästi ei ole ma kunagi oma elus ei varem ega 
hiljem elanud nagu keskkooli ajal. 

\textbf{\enquote{Mis te siis tegite selle raha eest?}}

Käisime restoranis söömas ja mida ikka lapsed rahaga teevad. Aga kool sai 
selle, et nad rohkem ei pidanud selle arvutiklassiga tegelema.  Kõik, mis seal 
oli, neid 
Iskraid\index{Arvutid!Iskra}\sidenote{\begin{russian}Искра\end{russian} oli 
mitmel pool Nõukogude Liidus eri modifikatsioonides toodetud IBM/XT kloon} oli 
siis vist kolm või neli (alguses oli kolm, siis tuli üks uuemat tõugu juurde), 
seda me siis seal niimoodi püsti hoidsime, et tunnid seal toimusid. Meie asi 
oli hoolitseda, et masinad töötaksid ja seal midagi saaks õpetada. Mingil 
hetkel, kui me juba ise natuke vanemad olime, hakkas sinna juurde tekkima 
mingisugune kamp nooremaid huvilisi, kes seal ka siis hängisid. Ta oli selline 
täitsa tüüpiline arvuti ökosüsteem. Kunagi suvel käisime, remontisime sellesama 
arvutiklassi ära: värvisime ja panime uued põrandakatted. Ühesõnaga käitusime, 
ma loodan,  heaperemehelikult temaga. 

\textbf{\enquote{Ei ole kuulnud, et kellelgi oleks heaperemeheliku käitumisega 
probleeme olnud sarnastes situatsioonides}}

Tead, aga ajad olid sellised, inimeste usaldus oli suur. See arvuti oli nagu 
selline ühtepidi eriti müstiline, teistmoodi teda nagu kardeti vanema 
generatsiooni poolt. Kujutad ette, et oli (ma siiamaani ei tea nende inimeste 
nimesid naljakal kombel, ega ma vist pole ka kunagi teadnud), aga kunagi oli 
Eestis selline turismibüroo nagu Sarved ja Sõrad\index{Sarved ja Sõrad}. Minu 
meelest just sedapidi, mitte Sõrad ja Sarved. Asus Rävala puiesteel seal, kus 
täna on NO-teater. Ja mina läksin selle aknalt seal Sakala 
tänavat\sidenote{Külgneb Rävala puiesteega} mööda ja nägin, et inimestel on 
arvuti, see oli aastal 1991 või midagi. Igal juhul ma veel ei töötanud 
Skriiningus\index{Skriining}. Aga arvutit tahtsin hirmsasti kasutada. Keskkool 
oli läbi, sinna enam sisse ei lastud ka no sõltlane käis mööda linna, eks ole. 
Järsku näed, akna taga arvuti. Ja marsid sama hooga sinna sisse täiesti 
tundmatusse firmasse, täiesti tundmatu värske keskkoolilõpetaja, et 
\enquote{teil on siin arvuti, ma tahaksin seda kasutada}. Ja ilma mingisuguse 
tänapäeval heaks kiidetud taustauuringu või millegita ütleb firma omanik sulle 
oma kirjutuslaua tagant \enquote{Jah, ta on meil siin küll, me tahaksime teda ka 
kasutada, loomulikult}. Ja ilma mingi töövestluse ja ilma mingisuguse sellise 
pikema jutuajamiseta antakse sulle kontorivõtmed, öeldakse, et \enquote{tee 
nii, et meie saaks seda arvutit kasutada, tee ta korda}. Ja sa avastad ennast  
arvuti tagant. Ilma et keegi oleks isegi su dokumenti vaadanud, et kas sa oled 
varas, või sa ei ole varastega või kas sa tahad terve firma ära varastada või 
ainult arvuti. See usaldus, mis tollel ajal valitses inimeste vastu, kes 
oskasid arvuti sisse lülitada ja sellega midagi teha, see oli \emph{enormous}. 
See oli selline, mida tänapäeval ei ole võimalik ette kujutada. See oli 
selline, et need värsked keskkoolilõpetajad, kes seal tulid, nendel 
põhimõtteliselt oli võimalik küsida ükskõik millise firma ükskõik millise  
arvuti võtmed, sest see kõik töötas. 

Noh, see lõppes muidugi sellega, et lõpuks tuli Imre Perli\index[ppl]{Perli, 
Imre}\sidenote{Imre Perli oli pehmelt öeldes raju elulooga Eesti 
arvutispetsialist, kes sai kuulsaks \enquote{Perli andmebaasi} koostajana. 
Kasutades ära ligipääsu mitmetele andmebaasidele, lõi ta üheksakümnendate keskel 
\enquote{superandmebaasi}, mis sisaldas isikustatud andmeid autode, (toona üsna 
haruldaste) mobiiltelefonide, aadresside jms. kohta. Andmebaas levis althõlma 
laialt. Imre Perli hukkus segastel asjaoludel 15. aprillil 2000 
politseioperatsiooni käigus} ja kopeeris ära kellelegi andmebaasid, eks iga 
aeg saab lõpuks otsa. 

\textbf{\enquote{Kuidas sul ikkagi see programmeerima õppimise protsess käis?}}

See on sihuke  kuidagi viimasel sajandil tekkinud paradigma, et 
programmeerimine on midagi, mida peab õppima ja see on midagi, millega tuleb 
nagu spetsiaalselt vaeva näha. Programmeerimine juhtub. Vajadusest. 
Programmeerimine juhtub tahtmisest. Keegi ei ole mulle mitte kunagi õpetanud 
ridagi Cd, keegi ei ole mulle kunagi õpetanud ridagi Assemblerit. 

\textbf{\enquote{Kuskilt saju ometi said teada, kuidas \texttt{malloc} käib?}}

Aga see on see tahtmine teha. No mina hakkasin õppima 
Pascalit\index{Keeled!Pascal}, sellepärast et see oli ainukene raamat, mis 
mulle kätte sattus. Seesama õudne Jürgensoni Pascali pruunide kaantega 
raamat\index{\enquote{Programmeerimine Pascal-keeles}}\sidenote{R. Jürgenson 
\enquote{Programmeerimine Pascal-keeles}, 1985. Legendaarne raamat, mis 
miskipärast huviliste hulgas laialt levis}, mis on ikka tagantjärgi vaadates 
päris õudne algus programmeerimisele. Aga sellega sai alustatud. Ja siis, kui 
Turbo Pascal hakkas ära tüütama, sellepärast et see tegelikult oli ka niisugune 
keel, milles midagi normaalset teha oli väga keeruline. Siis ühel hetkel ma 
leidsin, et ikka Assembler\index{Keeled!Assembler} on see päris asi. Kuna tol 
ajal oli popp kirjutada igasuguseid demosid ja häkkida kõiki tarkvarasid, mis 
kätte sattus, siis siis \ldots No küsi, kuidas õpetad inimesele nagu x86 
Assemblerit? No võtad raamatu ühte kätte ja AT86e teise kätte ja hakkad tegema.

\textbf{\enquote{Aga kust sa said selle raamatu? Neid ju ei liikunud?}}

Liiklus küll selle koha pealt tuleb anda tõenäoliselt varem või hiljem 
presidendi medal Tarmo Mamersile\index[ppl]{Mamers, 
Tarmo}\index[ppl]{MomraT|see{Mamers, Tarmo}}, kes tollel ajal 
TTÜ-s\index{Tallinna Tehnikaülikool}? Ma ei teagi, kes ta seal oli. No ta oli 
seal üks paljudest nendest, kes seda arvutiasja püsti hoidis esimesena. Aga 
Tarmo kaudu põhimõtteliselt kõik see asi liikus. Minu minu varane mentor oli 
raudselt Tarmo ja no oli seda tegelikult veel pikka aega ka peale seda, kui ma 
juba tegelikult tööl käisin. Kõik, see materjal käis käest kätte. Hiljem tuli 
juba FidoNet\index{FidoNet}. Kui ma oma esimese esimese FidoNeti \emph{point}i 
püsti panin, siis oli juba kõik palju lihtsam, sest siis oli nagu aken maailma 
olemas. \emph{Point}i püstipanemine käis ka loomulikult läbi TPI. Tarmo 
istus natuke eraldi, et Tarmo oli teises teises ruumis samal korrusel. Aga siis 
Aare Tali\index[ppl]{Tali, Aare} ja Tõnu Raimla\index[ppl]{Raimla, Tõnu} olid 
seal, kus käis elu nii-öelda.  Tarmo juures oli selline natuke rahulikum 
õhkkond. Seal käis igatahes kogu aeg \emph{action}. Ja siis mul oli nagu ühel 
hetkel kinnisidee, et ma tahan endale teha nüüd FidoNeti \emph{point}i, et 
ikkagi lõpuks olla maailma osa. Siis ma töötasin juba 
Skriiningus\index{Skriining}. Läksin siis Aare juurde, et \enquote{noh, Aare, 
sa oled siin \emph{sysop} ja värk} ja Aare talle omase abivalmidusega ütles, 
jah, masin on seal. Mille peale leidsin ma ennast BBS-i masina tagant ja pidin 
endale sinna valmistama FidoNeti \emph{node}. Ma kardan, et Tõnu või keegi 
lõpuks halastas mu peale ja näitas, kuidas seda päriselt teha. 

Aga siis edasi oli materjal juba palju kättesaadavam, siis said juba kõiki 
igasuguseid dokumente risti-rästi laadida alla. 

\textbf{\enquote{Mis sa sinna TPIsse õppima läksid?}}

Ma läksin LIsse. Ma arvan, et selle tolle aja nimi oli informaatika, äkki?. 
Kuna ma suhteliselt ruttu sain aru, et ma ei ole võimeline hommikul loengutes 
käima, siis mina ja üks väikene punt teisi, kes olid otsustanud, et nemad 
peavad õhtuõppes käima, läksime dekanaati ja nõudsime, et tarvis on õhtust 
vahetust sellele üritusele. Sest  õhtust vahetust tol hetkel konkreetsel alal 
ei olnud. Läksime kateedrisse, kateeder ütles, et jaaa, väga tore mõte. Aga 
meie kogemus ütleb, et kui te juba sihukese jutuga tulete, siis mitte keegi 
teist ei kavatse seal õhtuses ka käia. Mis tähendab, et me ei hakka teie jaoks  
eraldi rühma püsti panema. Käite ehitajatega esimese aasta koos koolis. Ja kui 
teisel aastal veel siin olete, siis vaatame seda asja. No kas nüüd osalt selle 
pärast või sellepärast, et dekanaadil oli õigus, nii või teisiti kukkusime 
sealt kõik robinal kolmanda kuu lõpuks välja ja läksime tööle igale poole. Nii 
et TPI on mul siiamaani lõpetamata. 

\textbf{\enquote{Sa mainisid, et sa kirjutasid mingit autobaasi softi. Kuidas 
sa seda tegema sattusid?}}

Siis ma juba töötasin. Me kõik läksime ju ka suhteliselt kiiresti ikkagi päris 
tööle. Tol ajal mingid startupi kultuuri ja ettevõtluse ehitamist veel ei 
eksisteerinud. Me ka lõpetasime sellisel ajal, kui need esimesi 
arvutikooperatiive oli siin väike käputäis. Minu esimene ametlik töökoht pidi 
olema tegelikult Noorsooteatri\index{Noorsooteater} valgustaja. Kuna mulle juba 
tollel ajal meeldis audioga tegeleda, siis ma tahtsin sinna helimeheks minna, 
aga helimees oli juba värskelt tööle võetud ja valgustaja koht oli vaba. Aga 
siis minu meelest päev või kaks enne seda, kui ma pidin lepingu alla kirjutama, 
tuli Tarmo Mamers\index[ppl]{Mamers, Tarmo} küsis, et kas ma ikka päris tööd ei 
taha teha, et Skriining\index{Skriining} otsib programmeerijat. 

Siis ma sattusin Skriiningusse Kalle Lotamõisa\index[ppl]{Lotamõis, Kalle} 
tööle. Minu esimene siis ülesanne oligi see, et autopark on sellel aadressil. 
Neil oli mingi eriti eksootilise asja peal jooksev andmebaasisüsteem, see ei 
olnud isegi \emph{mainframe}, see oli mingi mini. Ja  see oli vaja siis 
moodsale vahendile ümber kirjutada. Moodne vahend tähendas tol ajal siis Novell 
Netware'i\index{Novell} ja värskelt oli Paul Leis\index[ppl]{Leis, Paul} toonud 
Eestisse asja nimega Dataflex\index{Keeled!Dataflex}. Mis oli selline päris 
päris korralik objektorienteeritud kõrgkeel tol ajal. Ma hakkasin selle 
Dataflexiga siis ühest otsast õppima, kuidas Dataflexis programmeeritakse ja 
teisest otsast õppima, kuidas autopark töötab. 

\textbf{\enquote{Ahhaa, läksid kohe äriprotsessi ka sisse!}}

Äriprotsessid olid seal paljuski ees olemas selles mõttes, et töötav tarkvara 
oli olemas. Pigem oli seal äriprotsesside seisukohast hea lastetuba, et ära 
kunagi eelda midagi. Näiteks mina oma IT-inimese mõistusega tegin seal oma 
arust mõned asjad paremaks ja siis selgus, et päris nii ei saanud hea, nagu 
mina olin mõelnud. Sest raamatupidaja vaatasid mind nagu idiooti ja küsisid, et 
\enquote{sa ikka saad aru, palju ma neid numbreid pean siia päevast sisestama 
ja mitu korda ma seda enterit, mille sa siia vahele toppisid, peal vajutame 
lihtsalt niisama. Need arvud on neljakohalised. Ma sisestan neli numbrit ära, 
ta läheb ise järgmisele väljale, mitte ma ei pea vajutama. Ja eriti ma ei pea 
vajutama tabi, mis on teises klaviatuuri otsas. Saad aru, ma ühe käega kasutan 
pabereid teise käega vajutan klaviatuuri. Kuidas ma sinna tabi juurde sinu 
meelest saan, kui mul on teises käes paber?} 

Nad olid väga innovatiivsed seal tegelikult selles mõttes, et nad olid 
kasutanud sedasama andmetöötlust selleks ajaks juba aastat kuus-seitse. See oli 
 meditsiinitehnika autobaas, Termak\index{Termak}, siiamaani elu ja tervise 
juures. 


\textbf{\enquote{Nad siis juba Nõukogude ajal alustasid arvuti-asjandusega?}}

Nad olid juba sügaval nõukogude ajal end täiesti ära automatiseerinud. Selleks 
ajaks, kui mina aastal 92 sinna jõudsin, oli nendel juba esimene IT-süsteemi 
jõudnud kätte moraalselt nii ära vananeda, et see tuli PCde peale ümber 
kirjutanud. Neil oli aastal 1992 juba \emph{legacy}. Nad olid nii palju ajast 
ees.


\textbf{\enquote{Kuidas Skriining jõudis selleni, et neil on programmeerijat 
vaja? Lihtsalt kasti võis ka ju edukalt müüa?}}

Kalle\index[ppl]{Lotamõis, Kalle} hammustaski läbi selle, et kuna nad olid kogu 
aeg seal meditsiinitehnika ümber sebinud ja meditsiinisüsteemi neid arvuteid 
proovinud müüa, siis nad avastasid ühel hetkel, et seal on arendusvõimalused 
ka. Siis tegelikult Skriiningust\index{Skriining} saigi sealsamas 
üheksakümnendate alguses  arendusfirma. See arvutimüük käis ka, aga mina tema 
tollal noore inimesena väga ei süüvinud sellesse, kust raha tuleb. Aga mulle 
tundub, et see suht palju sellest tuli arendusest puhtalt.


\textbf{\enquote{Kas sa Tehnikaülikoolis ka veel ringi hängisid?}}

Ma hängisin seal pikalt aga ma kunagi õppinud seal. Ta oli ikkagi niisugune elu 
elu epitsenter, kuna seal töötasid kõik olulised inimesed. 
Mast\index[ppl]{Kaal, Madis} ülemisel korrusel Tõnu\index[ppl]{Raimla, Tõnu} ja 
Aare\index[ppl]{Tali, Aare} ja Tarmo\index[ppl]{Mamers, Tarmo} alumisel 
korrusel. Hiljem oli seal siis epitsenter siis, kui sinna läksid veel tööle 
Martin Rinne\index[ppl]{Rinne, Martin}, ja Merle Alliksoo\index[ppl]{Alliksoo, 
Merle} ja kõik teised, kes hiljem Microlinkis\index{Microlink} lõpetasid. Ta 
oli selline  sotsiaalse elu keskus. 

\textbf{\enquote{Mulle see variant, et sa ei õpi aga hängid, tundub palju 
mõnusam, kui see, et sa õpid aga ei hängi}}

Nojah, eks ma ise ikka soovitan inimestel reeglina, et  proovige oma kool kohe 
ära lõpetada, et pärast osutub see palju raskemaks. Nüüd mina ja mu ja sõbrad, 
kõik on sisuliselt neljakümnendates hakanud oma haridusega lõpuks tegelema. On 
tekkinud natuke rohkem vaba aega uuesti ja ka mingisugune moraalne vajadus, et 
kuidas sa oled kõige väiksemate pagunitega mees ruumis.

\textbf{\enquote{Tol ajal, kui tagasi mõelda, ülikool palju praktiliselt 
kasulikku ei andnud. Tänapäeval on teistmoodi}}

No nii nagu kõik ütlevad, et  ta oli selline ta ei olnud mitte tempel selle 
kohta, et sa tuled sealt välja targemana, vaid on tõestus selle kohta, et sa 
oled võimeline, järjepidevalt mitu aastat asjaga tegelema. Pigem ikka 
vastupidavuse ja hoolsuse proov kui koolitus.

\textbf{\enquote{Räägi palun BBSidest. Kuidas sa selle node ikkagi püsti said, 
selle jaoks oli vaja ju ennast kuskil registreerida?}}

BBS, kes ei tea, oli varane  arvutivõrk, mille mõte oli selles, et sa helistad 
kuhugi oma modemiga ja seal teises otsas on modem, kes sulle vastab. Nad saavad 
omavahel andmeühenduse püsti ja siis sa saad sellest teises arvutis, mille 
küljes modem oli, saad ringi sobrada. Kusjuures tõepoolest selles mõttes 
sobrad, et ega tollel ajal arvuti turvalisus oli selline noh kokkulepete 
küsimus. Ma arvan, et suvaline suvaline üks BBSi omanik oleks võinud teise 
BBSi omaniku BBSi lasta neljaks tükiks kaks korda tunnis ilma mingite 
probleemideta, aga seda lihtsalt ei tehtud. Sellepärast, et see oli nagu 
saarlase ukselukk. Et kui sa oled ta paika pannud väljapoole ukse ette, siis kõik 
teavad, et sind ei ole kodus ja nii on. Et ei ole vaja katsetada, et kas uks 
on lahti või kinni, kodus kedagi ei ole. Ja BBSidega oli  turva umbes sama. 
Nüüd BBSi teine ja tegelikult palju kasulikum omadusi oli see, et kui sul juba 
oli modem ja juba oli arvuti, siis sa said ennast FidoNeti\index{FidoNet} 
\emph{node}ks registreerida. Et BBS iseenesest ei eeldanud midagi sellest, lihtsalt 
modemi ja vastava tarkvara olemasolu. Kuskil mingeid hämaraid teid pidi levisid 
need telefoninumbrid, et kus see telefon on, kuhu helistada, seal kohapeal sa 
said ennast ära registreerida.

Nüüd FidoNet oli ikkagi juba esimene selline ülemaailmne arvutivõrk selles 
mõttes, et modemeid helistasid üksteisele automaatselt. Ja ta oli kaunikesti 
hästi toimiv,  tolle aja kohta elektronpostiteenus, mille üks  eriline omadus 
oli veel see, et ta kuidagi liikus väljaspool KGB huviala. Eks küll 
kahtlustati, et teda kuulatakse pealt ja aeg-ajalt mingid imelikud modellid 
üritasid sinu modemiga poole jutu pealt rääkida ka, aga üldiselt teda väga ei 
monitooritud vist. Mingisuguseid probleeme ma ei tea, et kellelgi oleks 
kaheksakümnendatel olnud nende modem-modemiga sidepidamisega ei Eesti ega 
välismaaga. Mis on selles mõttes eriti huvitav, et et kui läksid 
kaugekõneliinid nii palju lahti, et ta oli juba võimalik kuidagi automaatvalida 
kuhugile, siis ju me helistasime igale poole välja. Ja see 
FidoNeti\index{FidoNet} \emph{mail} oli tegelikult  esimene võiks öelda vaba 
demokraatlik sidekanal tegelikult väljapoole juba kaheksakümnendatel. See oli 
kaheksakümmend üheksa, kaheksakümmend kaheksa, umbes niimoodi ta Eestisse tuli.

Mina olin siis keskkoolis, esimese \emph{node} panin püsti hiljem, jah, aga 
noh, see süsteem ise töötas varem. Ma olingi vist Aare \emph{point}. 
\emph{Point}i numbrit ma enam ei mäleta, mis mul oli. \emph{Node} number mul 
lõpuks endal oli kolmkümmend viis, aga, aga kas \emph{pointi} number, mis oli 
kaksteist-kaksteist. Mis \emph{node} all, ka ei mäleta. Mina panin selle jah 
püsti minu meelest vist üheksakümmend üks. Aga siis oli ka veel see aeg, kus 
ikkagi veitsa oli see asi nagu hämar selles mõttes, et me ju veel päris 
vabariik ei olnud. Me olime selline üleminekuvabariik. Ja selline 
registreeritud postiaadress andis sulle  võimaluse mingite siis foorumites juba 
kaasa rääkida. Eestis endas oli kümmekond gruppi, kus käis jutt erinevatel 
teemadel. Ja selles mõttes ta oli ikkagi päris selline elu, nagu me täna oleme 
harjunud, kuigi natuke teistsuguste tehniliste vahenditega. Ta oli aeglasem ja 
ta ei olnud reaalajas selles suhtes, et see post saabus sulle paar korda 
päevas. Ta ikkagi ei olnud selline, et kirjutan oma kirja ja see läheb kohe 
kõigile laiali. Aga ta täitis kõik need ülesanded, millega me täna tegeleme 
ära, nii et tegelikult  kaheksakümmendate lõpus, üheksakümnendate alguses see 
ökosüsteem, mida me täna oleme harjunud nägema, oli tegelikult täiesti olemas. 
Ja väike käputäis inimesi Eestis omasid seda privileegi, et seda kasutada. 

\textbf{\enquote{Kas see väike käputäis olid pigem entusiastid, akadeemilise 
seltskonna inimesed või kes?}}

Need olid ikka sada protsenti entusiastid. Ma arvan, et akadeemilised inimesed 
sellel hetkel, kes läks ärisse, üritas sellest raha teha ja panid püsti 
esimesed arvutifirmad kes oli lihtsalt tegevuses  ellujäämisega, kes õpetas 
seda, mida ta kogu aeg õpetanud oli. See ökosüsteem minu meelest koosnes sada 
protsenti entusiastidest.

\textbf{\enquote{Kas eksisteeris mingi spetsialiseerumine ka, et siit ma saan 
tarkvara ja seal on huvitavaid jutte, seal raamatuid?}}

BBSidel väike spetsialiseerumine oli. Aga ma arvan, et mitte eriti suur. Eks 
kõik enam-vähem proovisid korjata, mida nad vähegi endale seal kõhu alla said. 
See oli see aeg, kus juba esimesed need sellised suuremad kõvakettad tekkisid. 
Mis tähendas seda, et tegelikult lühikest aega valitses olukord, kus tarkvara 
oli vähem kui ruumi. Ruumi mõiste oli ka muidugi tollel ajal huvitav. Kõige 
rohkem ruumi võtsid Sierra\index{Sierra Entertainment}\sidenote{1979. aastal 
asutatud Sierra Entertainment (varasemalt On-Line Systems ja Sierra On-Line) 
oli vastutav paljude toonaste hitt-mängude eest. Eriti populaarsed olid nende 
seiklusmängude sarjad \emph{King's Quest}, \emph{Space Quest} ja \emph{Leisure 
Suit Larry}} mängud, mis olid flopiketaste peal. Neist suuremad, Space 
Questid\index{Mängud!Space Quest} ja muud, tulid mingisuguse viie-kuue flopi 
kaupa. Mäletan, et me istusime Eeroga\index[ppl]{Tohver, Eero} ja arutasime, et 
kui oleks võimalik panna kokku oma unelmate masinat siis kui suur kõvaketas 
peaks tal olema. Jõudsime sinna, et kui oleks umbes kaheksakümmend megabaiti, 
siis ilmselt jätkuks eluajaks, sinna saaks kõik mängud peale panna, kõik 
tööasjad ka ja jääks veel umbes pool jääks üle.

\textbf{\enquote{Sierra oli omaette fenomen, tema asju mängiti ikka palju. Kas 
neid müüs ka keegi?}}

No küsime siis laiemalt, kas Eestis üldse keegi tarkvara müüs tollel ajal. 
Äritarkvara, nagu Novell, oli võimalik osta. Eks teoreetiliselt oli kusagilt 
Windowsi või DESQview'd\index{DESQview}\sidenote{DESQview oli kaheksakümnendate 
lõpus ja üheksakümnendate algul populaarne tekstipõhine mitmetegumiline 
keskkond. Ta käis DOSi peal ja võimaldas korraga mitut programmi eri akendes 
käimas hoida}  kindlasti võimalik osta. Aga peale Novelli serveri ja DataFlexi 
litsentside, ma ei mäleta, et me oleks üheksakümnendatel näinud mingisugust 
legaalset tarkvara kellelgi. 


\textbf{\enquote{Tuleme tagasi selle BBSinduse juurde. Kas selle sisu hulk, 
mida enda kõhu alla õnnestus kokku kuhjata, oli ka mingit pidi staatuse 
sümboliks?}}

Ma ei oska öelda. Ma oskan ainult enda BBSide kohta nagu rääkida. Mina 
sisuliselt korjasin kokku kõik, mida ma kätte sain ja pakendasin ringi. See 
oli selline kultuuriküsimus, et sa skaneerisid selle tarkvara viiruste vastu 
kõige värskema viiruste skanneriga, mis sul parasjagu käeulatuses oli. See käis 
automaatselt muidugi. Siis sa lisasid sinna mingi väikese faili sinna arhiivi, 
mis sisaldas mingit sinu \emph{header}it. See oli siis niisugune väike 
failijupp, kus oli graafiliselt või siis tollel ajal pseudograafilised, sinu 
logo sisse punnitatud. Ja siis sa panid ta välja ja panid oma faili listi siis 
mingisuguse sellise nupukese, mis asi ta on. 

See oli nagu \emph{basic housekeeping}. Et kui see sinu fail läks mingisse 
järgmisse BBSi siis järgmine BBS viskas sinul logo välja ja pani enda oma 
endale  asemele. Et nagu \emph{tag}iti ära nagu \emph{graffiti}ga, et see on 
minu käest tulnud asi. Ja minul vähemalt on küll tunne selline, et välja läks 
kõik, mida sa ise olid endale mingil põhjusel hankinud. Et see ei olnud nüüd 
nii, et sa läksid ja tõmbasid öösel mingisuguse HNSi\index{BBS!HNS} tühjaks ja 
panid enda lehekülje peale välja. Aga mingid asjad, mida sina olid kätte 
saanud, sa panid üles. Neid duplikaate ei olnud väga palju üllataval kombel.

\textbf{\enquote{Ma tahtsingi küsida, et nii oleks pidanud üks hetk ju kõigil 
kõik asjad olemas olema, seda siis ei tekkinud?}}

Seda ei tekkinud, sest kuna need BBSid olid väga stabiilset üleval, siis 
tõmbasid ära mingeid asju, mida sina pidasid enda jaoks vajalikuks ja panid nad 
siis ka omakorda enda juurde üles. Aga sellist mõttetut \emph{leach}imist  väga 
palju ei olnud. Püüet iga hinna eest oma faili andmebaas kõige suuremaks saada, 
ma ei mäleta, et seda oleks nagu eraldi eesmärgina keegi järginud. 

\textbf{\enquote{Too mõni näide, mis laadi asjad sulle toona huvi pakkusid?}}

No mina olin juba tollel ajal vihane \emph{nerd}, minu jaoks igasugu 
programmeerimismaterjalid ja igasugused käsiraamatud ja igasugused 
tööriistakesed ja  programmeerimisvahendid ja need olid minu spetsialiteet. 
Kahjuks mul ei ole seda vana faililisti alles, sest kui ma Skriiningust ära 
läksin, siis suhteliselt lühikese aja peale lendas see vana SCSI ketas õhku, 
mille peal see BBS jooksis ja sellest ei olnud \emph{backup}i ja sinna ta jäi. 
Läks kogu FidoNeti \emph{node} koos failibaasiga hingusele.

Ma ise teda järgmisse kohta kaasa ei võtnud, sest ma läksin Skriiningust panka 
ja seal olid kõvad mehed nagu Mast\index[ppl]{Mast} ja  Marx\index[ppl]{Marx|see{Kliimask, Margus}}\index[ppl]{Kliimask, Margus}  ees, kes olid oma ökosüsteemi püsti pannud, siis ühele BBSile seal rohkem ruumi ei olnud. 

\textbf{\enquote{Mis panka sa läksid?}}

Mina läksin sellesse panka, mille lõpupidu nüüd siin kohe nädala-paari pärast 
kätte jõuab\sidenote{Intervjuu Andrusega toimus 2019. aasta novembri algul} 
novembris, mis lõpetas siis Danske\index{Danske Pank}\index{Danske 
Pank|see{Forekspank}} nime all aga alustas Forekspangana\index{Forekspank|see{Eesti Forekspank}}. See 
oli jälle omaette selline innovatiivne pangandustoode. 

\textbf{\enquote{See oli väga äge pank omal ajal. Miks sa sinna läksid? 
Skriiningus said programmi kirjutada ja BBSi pidada ju?}}

Nagu ma paljudesse kohtadesse oleme läinud, läksin sellepärast, et kutsuti. Ja 
teiseks, kuna parasjagu Eestis jooksis teleseriaal \emph{Capital City}, mis 
näitas panganduse elu väga glamuurse \emph{highroller}ina, siis mulle tundus, 
et mina tahan ka nii elada. Tuleb tunnistada, et üheksakümnendate panganduses  
väga ei pidanud pettuma, elu oli täitsa täitsa lill. Ütleme nii, et nii nagu 
selles eesti teleseriaalis Pank päris elu ikkagi meie majas vähemalt ei käinud. 
Päris hulle pidusid sai peetud, aga seda, et keegi oleks kuskile kokaiinise  
ninaga ringi käinud, seda mina ei tea. Meie meie kandis  kokaiin oli täiesti ma 
ei tea, kas tundmatu või seda tehti salaja või midagi, aga igal juhul ma neid 
narkootikumidega pidusid ei tea. Aga pidutsetud sai hästi.

\textbf{\enquote{Aga kas ma õigesti mäletan, et tollal te panka tõmbasite ikka 
püsiühenduse\sidenote{Enamik varasest internetiühendusest Eestis käis kuhugi 
sisse helistades. See tähendas, et pidev side puudus ja muul ajal sõltus side 
kvaliteet suuresti analoogtehnoloogial põhinevatest telefonikeskjaamadest. 
Püsiühenduseks kutsuti seda, kui asutusest füüsiline kaabel Interneti külge 
jooksis ja selle olemasolu oli IT-inimeste unelmates kesksel kohal} sisse?}}

Püsiühenduse tõmbasime me sisse väga konkreetsel päeval. Ühesõnaga, meil 
modemitega see nii-öelda poolpüsiühendus oli juba pikemat aega olemas. Kuna 
Forekspank asus Rävala puiesteel ja Rävala puiesteel asus seal suhteliselt 
lähedal ka juhtumisi KBFI\index{KBFI}\sidenote{Keemilise ja Bioloogilise 
Füüsika Instituut (KBFI). 1979. aastal Endel Lippmaa\index[ppl]{Lippmaa, Endel} 
poolt loodud teadusasutus. Tuntud ka kui \enquote{Lippmaa Instituut}. Just 
Lippmaade perekonna aktiivse ning laiahaardelise tegutsemise tõttu mängis 
instituut paljudes toonastes olulistes protsessides (sealhulgas kohaliku 
Interneti arengus) olulist rolli.}. Baumaniga\index[ppl]{Bauman, Andres} ning 
oli läbi räägitud, et kuidas seda Internetti saab ja meil olid  suhteliselt 
rivitu ligipääs. Aga mingil hetkel tundus, et see asi võiks ikka päris 
permanentne olla ja siis me võtsime Mastiga\index[ppl]{Mast} kaablirulli ja 
hakkasime siis üle Rävala puiestee katuste KBFI poole liikuma. Mis oli selle 
juures tähelepanuväärne oli see, et see juhtus päeval, mil esimest korda paavst 
Eestit väisas. Kõik katused olid snaipreid täis, oli mingi tohutu 
\emph{lockdown} et keegi paavsti siin käigu pealt ära ei tapaks. Meie sellel 
samal päeval, kui paavst sõitis ringi oma mobiiliga, käisime kaablikeraga. 
Seletasime kõigile, et meil on vaja kaablit vedada, meie paneme Interneti 
püsiühendust. Ja see oli selline maagiline valem, mis  võimaldas ligipääsu 
kõikidele kesklinna katustele, ilma et keegi oleks küsinud sult midagi 
rohkemat. Me otseselt snaiperitega samale katusele ei sattunud, aga üldiselt 
jah, midagi ei küsitud ka. Natukene oli seal vist vaja mingit häkkimist ka, et 
mingist koodlukust pidime vist ikkagi läbi minema kogemata. Aga see oli tollel 
ajal mehaaniline ja seda tehti nuppude kulumise järgi, et see ei olnud kõige 
suurem takistus. 

\textbf{Millest ma siis järeldan, et toona oli maailm teistsugune. Internet ei 
olnud veel kommertsiaalne vaid pigem kogukondlik nähtus?}

Selle eest vististi ikka keegi maksis ka kellelegi lõpuks midagi, aga palju, 
seda ma jällegi ei mäleta. Eks ta ju paljuski käis inimsuhete baasil ikkagi. Et 
kuna me Andres Baumanni\index[ppl]{Bauman, Andres} tundsime siis kuidas see 
sealt tegelikult  liikus ega mina ka ei tea. Mast\index[ppl]{Mast} seda asja 
ajas. Millegi pärast ma arvan, et  me maksime KBFIle mingit mingit raha ka 
selle eest. Jällegi. Aasta oli siis minu teada üheksakümmend viis ja 
üheksakümne viiendast aastast alates tegelikult oli meil 
Forekspangas\index{Forekspank} elu, nagu me seda täna tegelikult näeme. 
Suhteliselt samal ajal tuli Mosaic'i\index{Mosaic}\sidenote{NCSA Mosaic oli üks 
esimesi internetibrausereid ja mängis WWW populariseerimisel olulist rolli. 
Sama meeskond lõi hiljem Netscape\index{Netscape} Navigatori, mis omakorda on 
Firefoxi eelkäijaks} brauser, suhteliselt samal ajal hakkas veeb arenema, 
suhteliselt samal ajal tekkisid meile kõigile e-maili aadressid (mis olid 
natuke küll varem juba KBFI kaudu olnud korraks, aga siis tekkisid nad päris 
meie oma foreks.ee domeeni külge). Kõik see ökosüsteem, miinus Facebook, olid 
üheksakümne viiendal aastal tegelikult meil käes. Sealt edasi,  tegelikult me 
elasime täpselt sellist elu nagu inimesed infotehnoloogiliselt täna ette 
kujutavad. 

See elu oli natuke tillukesem selles suhtes, et me täitsa tõsimeeli arutasime, 
et see ühendus, mis meil KBFIsse on, on ikkagi nii aeglane, et äkki peaks kogu 
veebi kohalikku serverisse ära kopeerima. Siis me isegi vahepeal arutasime, et 
kuna see mahub ühele DVD-le tõenäoliselt ka ära, ehk siis äkki peaks tegemiseks 
äri, et hakkaks müüma Internetiga DVD-d, et siis oleks saanud kohalikest 
\emph{cache}dest tõmmata. Kogu see WWW oli tollel hetkel selline, tõsimeeli sai 
arutatud, et paneks ta ühele DVD-le ära. 

\textbf{Ka teistest lugudest käib läbi, et toonane maailm käis väga suuresti 
inimsuhete peal. Aga ometi inimesed ei hakka arvutitega tegelema, kuna neile 
meeldib inimestega tegelda. Siiski tunduvad Eesti arvuti-inimesed olema küllalt 
suhte-altid ja neis osavad. Kuidas see nii on?}

Ma arvan, et see on sellepärast, et kuna sul on sellised huvid, siis sa oled 
terve keskkooli ja pool ülikooliaega olnud sotsiopaat ja kellegagi väga ei ole 
sul rääkida millestki olnud. Ja nüüd sa ühel hetkel leiad omasugused, 
omasuguste huvidega. Täitsa puhas \emph{nerd}i ja nohiku käitumine, eks ole, et 
kui sa paned  nohikud  kõik ühte tuppa kinni, siis nüüd ühel hetkel leiavad 
üksteist ja siis on kõigil järsku lõbus sest kõik naeravad samade naljade üle 
lõpuks ometi. Ja peod täpselt samamoodi, eks. Ega kõige karmimat peod, kus ma 
olen  osalenud on ikkagi olnud inimestel, kelle igapäevatöö on kaunikesti 
\emph{boring}. Tahtmata anda hinnangut mingisugustele inimgruppidele, aga   
raamatupidajad ja  andmesisestajad, kui nad ikka käima lähevad, siis see on 
ikka ikkagi täiesti teine tase. Et siis lõbusad inimesed on keskmiselt lõbusad 
kogu aeg. Aga kui sellised nohkarid lõpuks lõbusaks lähevad, siis juhtub asju.

See ökosüsteem toimis tänu sellele, et inimestel oli hea meel üksteist leida. 
Ta oli tollal tõesti väike ka, ta oli ikkagi alla saja inimese kindlasti,  
võib-olla isegi alla viiekümne inimese alguses. Need olid siis sellise uue 
laine  arvutitegelased, kust hästi palju meie tänast  start-up ettevõtlust 
tegelikult ju välja on kasvanud. See kamp juba tollel ajal oli väga tihedalt 
koos ja väga õnnelik üksteise leidmise üle. Ja tänu sellele ju hakkasid siis 
toimuma legendaarsed BBSummeri\index{BBSummer} nimelised üritused. 

\textbf{Räägime lõpetuseks sellest ka, et mis sa praegu teed?}

Ma ei tea, see on võib-olla masendav tõdemusega, ega ta pole mind väga palju 
sellest nagu kaugemale ega kuhugi mujale viinud. Ikka väga laias laastus 
tegelen täna täpselt sama asjaga, millega ma tegelesin kakskümmend viis aastat 
tagasi. Olen pendeldanud elektroonika ja tarkvara vahel siin edasi-tagasi ja 
olnud mitme firma CTO ja  asutanud firmasid ja neid kihva keeranud ja töötanud 
teiste juures ja töötanud endale. Ja kui keegi küsib, et millega sa tegeled, 
siis ma tavaliselt ütlen, et ma annan masinatele hinge. 


\textbf{See on ilus ütlemine ja läheb kokku küsimusega, mis mul enne jäi 
küsimata. Tavaliselt inimesed tegelevad kas riist- või tarkvaraga aga sinul 
tundub olevat üks jalg ühes ja teine teises?}

Vaadates oma elu, siis ma muidugi tahaks, et tarkvara oleks mu tõmmanud endasse 
sellepärast, et see on nii palju  lihtsam ala, mõnes mõttes. Vigu on palju 
lihtsam parandada ja asju ära visata peaaegu üldse ei tule, mis katki lähevad. 
Kettaruum ei maksa täna eriti palju erinevalt elektroonika valmistamisest ja
utiliseerimisest.

Mul on kuidagi juhtunud niimoodi, et mul on see reaalne maailm, et kui ma panen 
tule vilkuma ja näen, kuidas mu tehtu  manifesteerub päris päris asjades, et 
siis mul kuidagi läheb tuju paremaks. Kuna mul see elektroonika disaini puhul 
tundub, et tuleb ka välja, samal ajal ma ikkagi taustalt oled nagu 
programmeerija, siis ma olen nagu sattunud sinna sidemeheks. Ma suudan tõlkida 
riistvara tarkvara jaoks ja vastupidi. Selle kõige konkreetne töönimetus on 
\emph{embedded engineering}. Mis on, tundub täna, vaadates, mis meil koolidest 
saabub, siis täiesti väljasurev kunst. Neid tegelasi, kes suudavad nii 
riistvara valmistada kui sellele tarkvara peale kirjutada, neid üritatakse 
nimetada mehhatroonikuteks või kelleks iganes, aga aga fakt on see, et nende 
juurdekasv on järsult pidurdunud ja see varem või hiljem hakkab meil 
probleemiks muutuma. Tõsi küll, töömeetodid ka muutuvad. Me kasutame täna
võibolla töövahendid, mis iseenesest annavad näiteks tarkvaratiimile parema 
ettekujutuse riistvarast kui see vanasti oli. Kirjeldused ja mingisugused 
\emph{markup language}d, millega  seda tehakse, on paremad. See, mis ma enne ka 
ütlesin oma tööd kirjeldades, et ma annan masinatele hinge, et see on 
tegelikult see osa, et kui sa lülitad oma pesumasina sisse, siis mida ta sinu 
heaks teha oskab või ei oska. Kui hästi see  raua ja tarkvara vaheline kooslus 
on välja mõeldud, sellest tuleb ka see kasutajakogemus. 

\textbf{Sa ütlesid enne, et sa oled ka CTOna toimetanud. See tähendab ju, et 
kolmas element juurde, sa pead selle kõik suutma ka äriks tõlkida?}

No vot seda CTO ametit on kaht sorti. Tavaliselt väikestes firmades tähendab 
CTO olek seda, et koosolekule on vaja kedagi kaasa võtta ja kuidas sa võtad 
kaasa ja ütled, et on mul programmeerija, eks. Sa pead talle lihtsalt andma 
visiitkaardi, millega ta näeb välja presentaabel. Tihti väikese firma CTO 
tähendabki lihtsalt seda, et sa teedki kõike, millel on tehnikane maitse 
küljes. Suurema firma CTO tähendab seda, et sa oledki see\ldots. Täna on 
startupi maailmas \emph{customer fit} ja \emph{market fit} hästi kõva teema. 
Kui vanasti väga ei tegeletud sellega, siis nüüd, kus on tohutu kuhi 
investorite raha põlema pandud, ilma et sellest isegi sooja oleks saadud, et 
siis kõik on hakanud rääkima sellest, et su toodetut kellelegi nagu päriselt
tarvis peaks ka olema. See tundub olevat mingi uuem asi, viimase paari aasta 
paradigma. Juba mingi kaks-kolm aastat tagasi hakkas Silicon Valley poole pealt 
pihta see kultuur, et laste kätte ei taheta raha enam hästi anda. Ehk siis 
nende kaheksateistaastaste ime-ettevõtjate, kes suudavad  väga suure kuhja raha 
korraga põlema panna, nii et sooja ei saa, aeg sai seal mõned aastad 
tagasi läbi. Nüüd on siis  selgunud uus innovatiivne lähenemine, et toodet peab 
kellelegi tarvis ka olema, teine selline paradigma muutus. Mis tähendab muidugi 
seda, et erakordselt raske on olnud hakata raha saama projektidele, sest kõik 
on hirmus järsku pirtsakas muutunud ja nõudnud, et kust raha tagasi tuleb. 

\textbf{See jutt läheb ju kokku sinu kunagise ettevõtte uksest sisse minekuga: 
seal sa pidid ju ka kohe hakkama kasulik olema ja ei tohtinud asju tuksi 
keerata}

Selle kasumlikkus, see on tegelikult õudselt valus teema. Riistvaraga on see 
asi  selgem selles mõttes, et riistvara ei eskaleeru kui keegi teda ei osta. Sa ei 
saa valmistada sedasama \emph{recorderit}, millega me siin praegu salvestame, 
miljon tükki, kui keegi seda ei osta, sellepärast et sa lähed lihtsalt 
pankrotti. Tarkvara tiražeerimine ei maksa midagi. Ja täpselt samamoodi võib ju 
juhtuda, et tarkvara, millest mitte kellelegi mitte pennigi raha ei teki, on 
tegelikult väga kasulik. Ehk siis kasulikkus ja ärimudel ei tähenda veel mitte 
midagi omavahel. Ja mis  dotkommi mullidega tavaliselt kipub juhtuma ja 
igasuguste tarkusemullidega, on see, et piir selle vahel, kus asi ei teeni 
raha, sellepärast et ta on väga hea mõte, mida veel ei ole õpitud raha panna 
teenima ja nende asjade vahel, mis ongi täiesti mõttetud, on väga raske tõmmata. 
Seetõttu on väga palju tegelasi, kes suudavad maha müüa  täiesti kasutu idee 
öeldes, et see ongi enne monetariseerimist faas ja see peagi midagi tootma. 
Unustades ära selle, et see on ka ühtlasi täielik kräpp, eks ole. Siin  
viimasel ajal on tekkinud paar niisugust suuremat skandaali, üks neist on 
muidugi see õnnetu Theranose \emph{case}, kus sa suudad nii veenvalt endale 
valetada. Et sul ongi ehitatud üles terve ökosüsteem väga kasulikkudest 
asjadest, mille ainus viga on see, et see fundamentaalne eeldus, millele ta 
rajatud oli, oli täiesti vale. 

\textbf{Tundub, et selle kahekümne viie aastaga maailm väga teistsuguseks 
saanud ei ole aga siiski natuke toimib teisti?}

Üks asi on oluliselt erinev. Tollel ajal tarkvara valmistati kahel põhjusel. 
Üks oli see, et teda oli tarvis, mis tähendas, et  oli tugev kliendipoolne 
tõmme. Ja teine oli see, et ma tahtsin, et midagi sellist eksisteeriks 
maailmas, mis tähendab, et ma lihtsalt võtsin kätte ja kirjutasin ta kas enda 
või teiste rõõmuks. Ja lasin ta lihtsalt maailma. Hästi palju mingeid väikesi 
utiliite, mis midagi kasulikku tegid, olid ju tegelikult kirjutatud kellelgi 
enda jaoks ära, pakendatud ja saadetud laiali. Eestis seda, et tarkvaraga 
õnnestuks mingit raha teha, et mina kirjutan mingisuguse vidinaga ja keegi 
maksab selle eest, seda kontseptsiooni polnud olemas. \emph{Corporate} maailmas 
küll seal igasuguseid  raamatupidamissüsteeme osteti-müüdi juba tol ajal väga 
edukalt ja see kõik töötas. Mujal maailmas tegeleti mingisuguste utiliitide 
pealt raha teenimisega ka väikesel viisil. Aga Eestis üldse mitte. Nüüd 
tänapäeval eks ole, see tarkvara tootmine on läinud niimoodi, et mul tuleb mingi 
ilgelt hea idee. Ja ma tahan sellest nüüd teha raha tootmise masina, mis 
tähendab, et sa teed nagu teistpidi. Et see ei ole mitte nii-öelda 
vajaduspõhine vaid selline unistus-põhine. Et mina tahaksin, nagu me siin 
aeg-ajalt  Ivar Sarantsiga(Kas on õige nimi?) naerame, et tänapäeva 
maailmas  inimesed  otsivad probleeme neid vajavatele lahendustele. Et kui 
vanasti otsiti probleemidele lahendust, siis nüüd otsitakse vastupidi ja see on 
 kõige suurem paradigma muutus selle kahekümne viie aasta jooksul.


\chapter{Sergei Anikin}
%!TEX TS-program = arara
% arara: myindex

\index[ppl]{Anikin, Sergei}
\question{Kuidas sina arvutite juurde sattusid?}

See oli päris huvitav lugu. Ma olin nii-öelda \emph{entitled}, mu isa oli
elektroonikainsener ja töötas Kalinini tehases\index{Kalinini
tehas}\sidenote{Algselt Balti Raudtee Peatehased, mis ehitati 1870. aastal ja
kandis aastatel 1902–1903 seal töötanud Nõukogude riigitegelase järgi 1940.
aastast alates M. I. Kalinini nime. 2007. aastast asub sellel
territooriumil ja osalt samades hoonetes Telliskivi Loomelinnak restoranide,
kohvikute, kontorite ja loomeruumidega}. Nüüd on seal kõige popim koht noorte seas, seesama Kalamaja ja Lendav Taldrik.
Lapsena käisin koos isaga tehases. Isa projekteeris rongidele
elektrimootoreid ja jõuelektroonikat. 
Hobi korras on ta teinud igasugust raadiotehnikat ja ma ise olen proovinud
väikest raadiot kokku panna, kuigi olin täielik võhik. Käisin küll raadiotehnikaringis. Minu esimese arvuti aga pani kokku isa.

\question{Kust ta vajalikud jupid sai?}

Isal oli selline Vene ajakiri nagu 
\begin{russian}Радио\end{russian}\index{Radio}\sidenote{Igakuine populaarteaduslik 
raadiotehnika ajakiri, mida andsid välja Nõukogude Liidu Siseministeerium ja 
DOSAAF (\begin{russian}Добровольное общество содействия армии, авиации и флоту 
России\end{russian} – vabatahtlik Vene armee, lennunduse ja mereväe 
abistamise selts). Ilmus eri nimede all alates 1925. aastast, 1975. aastal oli 
ajakirja tiraaž 850 000 eksemplari.}. Aastal 1986 avaldati seal 
kõigepealt arvutiskeemid ja siis kokkupanemise juhend. See oli 
Nõukogudemaal välja töötatud arvuti, aga skeemid võtsid nad ZX Spectrumi 
pealt\sidenote{\begin{russian}Радио-86РК\end{russian}\index{Radio-86RK}, populaarne
Nõukogude liidus loodud koduarvuti. Kuigi Nõukogudemaal 
kopeeriti ZX Spectrumit usinasti, oli see arvuti siiski väidetavasti 
originaalse disainiga, autoriteks Dmitri Gorškov, Juri Ozerov, Gennadi 
Zelenko ja Sergei Popov.}. Isa korjas komponendid kokku, 
joonistas ise plaadi, tegi tehases plaadi valmis ja 
pani arvuti kokku. Mäletan, et tal läks paar kuud, enne kui kõik 
vigased kohad ostsilloskoobiga välja juuris. Siis pani ta selle 
teleka külge, mis asendas monitori. See oli mustvalge telekas, 
värvitelekat meil ei olnud. Ega ma selle arvutiga midagi väga teha ei 
saanud, sel ei olnud isegi opsüsteemi. Oli küll \emph{interface} 
kassettmakiga, aga meil ei olnud kassette, mille pealt 
opsüsteemi laadida. Sellessamas ajakirjas oli trükitud baitkoodis opsüsteemi kood – 
kakskümmend lehekülge bait baidi haaval. Istusin kaks nädalat pimedatel talveõhtutel arvuti taga 
ja trükkisin kõik need koodid sisse.

\question{Miks sa seda tegid? Normaalne laps ju ei toksi niimoodi 
pimedatel õhtutel baitkoodi?}

Ka sellel on eellugu. Isa sõber tõi mulle umbes aasta varem lasteraamatu, kus tegelased õppisid programmeerima 
BASICus\index{BASIC}. Lugesin raamatu läbi, sain aru, kuidas 
programmi kirjutada, ja kirjutasin BASICus umbes kümnerealise programmi, mis 
midagi arvutas. Kuna aga paberi peal ei saanud ju
kompileerida, siis näitasin seda isa sõbrale, kes kontrollis ja  
ja ütles, et töötab küll.

Koodide sissetoksimine käis plokkide kaupa. Seal oli 
umbes poole leheküljeline plokk, millel oli kontrollkood. Sain seda 
valideerida ja kui see klappis, siis salvestasin makile. 
Kui ei klappinud, siis pidin viga otsima, mis oli väga 
keeruline. Ilmselt sellest ajast tekkis mul esiteks 
kannatus ja teiseks tähelepanu detailidele. Koode sisestades sain
lõpuks aru, et hästi oluline on need õigesti ja õiges 
järjekorras sisse toksida, sest ümbertegemine oli nii piinlik.


\question{Sulle tehti väiksest peale selgeks, et võid küll üle jala 
lasta, aga siis toksid ise neidsamu asju kolm korda.}

Jah, aga enamiku ajast veetsin loomulikult arvutiga mängides. Tollal 
oli olemas tavapärane maomäng ja ka tennis. Isale meeldis arvuteid kokku panna, sealhulgas sedasama 
ZX Spectrumi\index{ZX Spectrum}. Tegelesime ka 
selle väliskorpusega. Eestis on ju talviti kuiv õhk ja 
meil olid siis plastist õhuniisutajad, mis käisid radiaatori peale. Sellest sai väga 
hea korpuse arvutile: see oli õige kujuga ja selle sisse sai lõigata 
klaveri, toiteploki, plaadi ja kõik muu vajaliku. Makk oli eraldi.

\question{Miks sulle elektroonikaosa huvi ei pakkunud?}

Mul ei olnudki tegelikult arvuti vastu suurt kirge, siiamaani ei ole. Minu arust on see ikkagi vaid vahend. Tänapäeval on ju
teada, et arvutitega tegelejad teenivad päris korralikult raha. Tol ajal oli see ka mõnes mõttes staatuseküsimus, kui peres oli
arvuti. Kui paljudes peredes
üldse oli? Alles aastaid hiljem tekkisid arvutiklubid või
arvutimängukohad. Aga mul oli kodus olemas, kuigi me ei olnud
jõukas pere, kel oli raha niisugust asja osta. 

Arvuti on jah pigem vahend, ka meeldiv hobi, aga mitte ainuke. Mõnda aega ei tegelenud ma üldse arvutitega, 
mängimine enam kirge ei tekitanud ja programmeerida lihtsalt enda jaoks ei
tundunud väga huvitav. Aga mul oli üks sõber, kellega koos me mängisime. Tema
mainis: \enquote{Hoo, ma käin nüüd arvutiklubis. Me õpime seal programmeerima,
kuigi mina käin muidugi enamasti mängimas}. Siis ma mõtlesin, et tema ju
tegelikult ei oskagi midagi, aga mina küll, ja et peaksin koos temaga minema. Sa
ilmselt oled rääkinud paljude inimestega Eesti kogukonnast, aga mina sattusin
siis Vene kogukonda. Selle arvutiklubi nimi oli Interface\index{Interface}.

\question{Kes seda klubi pidas ja kus?}

Seda vedas Nina Botina\index[ppl]{Botina, Nina}, kes töötas vist bioloogiainstituudis Mustamäe teel. Me käisime 
Reaalkoolis\index{Tallinna 2. Keskkool}\index{Reaalkool|see{Tallinna 2. Keskkool}} tundides,
seal olid arvutiklassid.

\question{Mis koolis sa ise käisid?}

Koolis number kakskümmend kuus\index{Tallinna 26. Keskkool}. 
Viimasesse klassi läksin Tõnismäe Reaalkoolis\index{Tõnismäe Reaalkool} 
kus oli väga tugev matemaatika. Tegelikult seesama Nina Botina õhutas mind ja 
veel ühte klassiõde teise kooli minema ja 
matemaatikaklassi lõpetama. Tema pärast läksimegi sinna, seal oli hästi palju  
tuttavaid arvutiklubist.

Hiljem kasvas sellest arvutiklubist venekeelne tehnikakool või
arvutitehnikakool, mis asus Erika tänaval. 

\question{Ma teadsin, et Tartu ja Tallinna vahel on erinevus. Aga 
selgub, et ka Tallinna sees on kaks täiesti isesugust Tallinna.}

See on huvitav jah. Kusjuures minu huvi arvutite vastu 
vaheldus. Ühe aasta olin klubis, aga uude kooli minnes ei olnud mul 
selleks aega. Siis kutsus Nina mind appi, arvutiklassi instruktoriks, ja see 
tekitas uuesti huvi. Kui ma lõpetasin kooli ja läksin
ülikooli majandust õppima\index{Tallinna 
Tehnikaülikool!Majandusteaduskond}, tekkis seal esimese aasta lõpus 
võimalus spetsialiseeruda majanduslikule andmetöötlusele. Meil oli pisike 
grupp, seitse inimest. Kui kõik, kes olid majanduses, õppisid majandusaineid, siis enamik meie tunde olid arvutitehnika gruppidega.

Ma läksin küll venekeelsesse majandusteaduskonda, 
aga grupp oli eestikeelne. Huvitaval kombel ei pidanud me õppima arvutitehnika baasaineid. 
Esimese aasta arvutitehnikas õpiti nimelt füüsikat-keemiat, kõiki üsna 
keerulisi aineid. Ma olen kuulnud õudseid lugusid, kuidas inimesed ei saanud ülikooli 
lõpuni neid tehtud. Aga meie õppisime mikro- ja makroökonoomikat ning 
inglise keelt. Alates teises aastast hakkasime niisiis koos arvutitehnika omadega õppima ja
erilist jõudluse vahet ei olnud.

See, kus ma praegu olen, on ilmselt  
põhjustatud ka sellest, et ma ei läinud väga süvitsi arvutitehnikasse, vaid pigem 
oli arvuti alati vahend mõne probleemi lahendamiseks.

\question{Sa mainisid, et matemaatika tuli sul hästi välja. Kas käisid 
olümpiaadidel ka?}

Käisin, aga ma olin keskmiste seas. See sõltub palju õpetajast. 
Mäletan, et olin kas viiendas või seitsmendas klassis\sidenote{Selle põlvkonna inimestel jäi 
nii vene kui ka eesti koolides üks klass vahele, sest koolid läksid 
kaheksakümnendate teisel poolel üle aasta võrra pikemale õppele}, kui hakkasid geomeetria ja muud sellised ained. Ja siis mul 
klikkis, et iga teoreemi kohta, mida meile räägiti, tekkis mul teine 
viis, kuidas seda tõestada. Ma sain aru, et asjad ei ole alati ainult 
ühtemoodi, saab ka teisiti. See omakorda klikib õpetajaga: kui 
õpetaja näeb, et õpilane mõtleb, siis pöörab talle rohkem tähelepanu. Paraku läks see õpetaja ära ja järgmised ei olnud nii head.

Meil oli üks väga hea füüsikaõpetaja, kes tegi palju kontrolltöid. 
Tema juures õppisin seda, et valemeid ei pea üldse meelde jätma. Piisab, kui oskad neid rakendada. Loomulikult ei olnud spikerdamine 
lubatud, aga mul olid valemid ikkagi spikrina vihiku tagakaanel. Sa pead 
aru saama probleemist ja vahenditest, mida selle 
lahendamiseks kasutada. See õpetaja vaatas valemite teadmisele läbi sõrmede, sest 
kui probleemist aru ei saa, siis lihtsalt valemid füüsikas ei aita. 

Tõnismäe Reaalkoolis oli legendaarne 
matemaatikaõpetaja Mihhail Vassiljevitš\index[ppl]{Vassiljevitš, Mihhail}, kes õpetab seal
siiani. See inimene on tõeline autoriteet, kohtleb 
õpilasi ühtemoodi! Meie matemaatikaklassis oli kolm-neli tippõpilast, kes 
võitsid kõik riiklikud olümpiaadid ja käisid ka maailmaolümpiaadidel. Loomulikult ta 
tegeles nendega, aga ka kogu ülejäänud rahvaga. Oli neidki, 
kes ei saanud matemaatikast väga aru, aga tema juures nende tase tõusis. Ta oskas 
selgitada ka keerulisi asju nii lihtsalt, et kogu klass 
oli paar taset teistest koolidest üle. Ainuüksi selles  
keskkonnas olemine tõstis taset nii kõvasti.


\question{Jällegi tuleb välja, et matemaatikatunnis õpiti lisaks 
matemaatikale suhtumist, ja just see on sul aastate järel meeles.}

Mina ei saanud olümpiaadidel küll mingeid kohti, aga 
meist aasta vanemas klassis oli selline lugu, et umbes kümme inimest läksid keemia-, kümme 
matemaatika- ja kümme füüsikaolümpiaadile. Põhimõtteliselt terve klass osales Tartus riiklikel
olümpiaadidel, aga erinevatel aladel. Ja kuna nad olid juba seal kohal, siis 
neil oli lubatud ka teiste ainete olümpiaadidest osa võtta. Selle tulemusel said enam-vähem kõik, isegi need, kes algul ei kvalifitseerunud,
kõikidel aladel esikümnesse. Hämmastavalt võimas klass!

\question{Miks sa läksid majandust õppima?}

Sest mu vanemad ütlesid, et meil on peres juba kaks inseneri olemas, ema oli 
soojustehnik. Eks ma mõtlesin ka muid variante, aga kodu juures 
oli palju lihtsam. 

\question{Kas sul oli mingi ettekujutus ka sellest, mida sa tahad pärast oma 
haridusega ette võtta?}

Erilist ettekujutust ega plaani mul ei olnud. Tahtsin lihtsalt näha, mis see majandus 
õigupoolest on. Ühel suvel proovisin töötamist müügiinimesena ja selgus, et 
see ei sobi mulle absoluutselt. Müügitöös ütleb 
üheksakümmend kaheksa protsenti inimestest \enquote{ei}, aga mulle ei 
meeldi feilida ja minu jaoks oli \enquote{ei} tol ajal feil. 
Tegelikult nüüd, kui olen Pipedrive'is\index{Pipedrive} juba seitse 
aastat töötanud, saan aru, et see on osa protsessist, statistika. Feil on see, kui sa ei tee seda üheksakümne üheksandat 
müüki, mis võib õnnestuda. Müük on see, et tead neid statistilisi 
numbreid ja plaanid vastavalt nendele. See ei ole feilimine, kui esimene juhuslik 
inimene ütleb, et tal ei ole seda teenust vaja.

\question{Mida sa müüsid?}

See oli tänavamüük, müüsime erinevaid tooteid, näiteks 
tööriistakaste, mis läksid päris hästi, 
elektroonilisi hambaharju ja nii edasi.

\question{See on ju igavesti raske töö!}

See oli väga raske töö. Tulime igal hommikul lattu ja saime päevakvoodi, näiteks tuli müüa viisteist 
tööriistakasti. Kui täitsid kvoodi kahe nädala jooksul, siis 
said järgmise tiitli ja koos sellega endale õpilasi. Ja kui viis 
õpilast said omakorda kvoodi täidetud, siis said 
nii-öelda enda äri. Mina sain õppetunni, et see töö ei ole kindlasti minu 
jaoks. Teadsin, et kui lähen programmeerijaks, saan oluliselt 
rahulikuma töö eest oluliselt suuremat tasu. See sundiski mind umbes 
pool aastat hiljem ütlema: \enquote{Okei, ma lähen.} Ja nii ma läksingi ülikooli 
teise aasta keskel informaatikagruppi.

\question{Kas sa siis programmeerisid juba tõsisemaid asju ka 
või puutusid nendega ainult loengus kokku?}

Tegin kahte projekti, mis tõid natuke raha sisse.

Tol ajal olid hästi populaarsed 
SAT-TV\sidenote{Kaheksakümnendate lõpus ja üheksakümnendatel oli isiklik satelliidivastuvõtja ületamatult 
kallis, piraatlusele vaadati läbi sõrmede, suuri teenusepakkujaid veel polnud, aga väikestel oli juba 
võimalus tegutseda. Siis pandigi mõne kortermaja katusele satelliiditaldrik, 
hangiti piraatkaart tasuliste kanalite jaoks, veeti üle katuste 
ümberkaudsetesse majadesse kaablid ja asuti teenust müüma} firmad. Mõnes 
väikeses rajoonis oli oma kunn, kes pakkus SAT-TVd kuutasu eest. 
Mul oli üks tuttav, kes palus teha infosüsteemi, kus oleks kirjas, kes on 
liitunud, kes ei ole, kui palju nad maksavad ja mis teenust kasutavad. 
Emal oli tööl arvuti, millega sain teha Accessi\index{Microsoft 
Access} andmebaasi ja selle peale väikese liidese.

Teine projekt oli veel huvitavam. Kui sain teada, kui palju raha ma selle töö eest 
saan, olin väga imestunud. Isa sõbrad tegelesid valvesüsteemidega ja neil oli 
üks vanglaprojekt, valvesüsteemi panemine vanglasse. Neil oli 
tarvis joonistada vangla projekti järgi skeem, 
kus oleks näha, kus on alarmid tööle läinud. See ei olnud otseselt 
programmeerimine, rohkem disain. Mina pidingi selle skeemi
joonistama, sain selle kolme nädalaga tehtud ja tasuks
umbes isa poole aasta palga. Siis sain aru, et 
arvutitega tasub toimetada.

\question{Kust sa infot said? Accessis programmeerimine ei ole 
niisama lihtne, et hakkad muudkui otsast tegema.}

Accessi kohta ma ei mäletagi, eks vist lugesin dokumentatsiooni. 
Programmeerimist õppisin 
raamatutest. Mul oli üks venekeelne Pascali raamat, mis õpetas objektorienteeritud 
programmeerimist. Ka ülikoolis olid mõned ained väga-väga 
kasulikud, näiteks andmebaaside projekteerimine. Tänapäeval paljud 
inimesed ei oska relatsioonilist andmebaasi projekteerida, aga see on üks 
vajalikumaid oskusi, kui tahad kasvõi lihtsat süsteemi kokku 
panna. Tänapäeval lahendatakse selliseid asju tihti jõuga.

\question{Kas sinu reaalainete ja arvutihuvi juurde käis ka 
mõne spetsiifiline, näiteks ulme- või raamatuhuvi? Vene keeles oli ju palju 
rohkem asju kättesaadavad, mina ei olnud suuteline tol ajal Strugatskeid 
originaalis lugema.}

Ei mäleta, et oleks väga olnud. Raamatuid lugeda mulle meeldis, samuti
ulme või fantastika. Aga arvutite suhtes ei tekkinud mul tugevat
tunnet, minu jaoks oli arvuti nii praktiline asi, kui olla saab. Lugesin
Bulõtšovit\sidenote{Kir Bulõtšov (1934--2003), Nõukogude 
ulmekirjanik} ja Strugatskeid\sidenote{Arkadi Strugatski (1925--1991)  ja Boris Strugatski (1933–2012). Nõukogude ulmekirjanikud, kes kirjutasid enamasti koos, seega tuntud kui \begin{russian}братья Стругацкие\end{russian} või lihtsalt Strugatskid}, aga ka 
välismaa asju. Olen ka kõik Barbar Conani\sidenote{Robert E. 
Howardi (1906--1936) 1932. aastal loodud tegelane, kes on sellest ajast tembutanud 
kõikvõimalikes meediumides ajakirjadest ja raamatutest filmide ja 
videomängudeni} ja Tarzani\sidenote{Edgar Rice Burroughsi (1875–1950) 1912. aastal 
loodud tegelane, kes sai Nõukogude Liidus tuntuks kinodes näidatud 
trofeefilmidest (Johnny Weissmulleri kehastatud tegelane erines küll oluliselt 
raamatukangelasest)} raamatud läbi lugenud.

\question{Mis su esimene päris programmeerijatöö oli ja millal?}

Veebruaris 1996 läksin tööle Aeteci
Finantsvara ASi\index{Aeteci Finantsvara AS|see{Profit Software}}, mis nüüdseks 
on Profit Software\index{Profit Software}. Mul olid seal sõbrad ees. Nad tegid soomlastele 
finantskindlustussüsteeme. Oma esimese tööülesandega ma 
feilisin, sest mulle anti mingisuguse
valemi programmeerimine. See pidi Cs\index{C} olema ja sellest pidi 
\emph{library} saama. Ma ei teadnud, kuidas Csi 
kirjutada, ma ei saanud sellest valemist aru (see oli kõrgem matemaatika). 
Ühesõnaga, sellega ma feilisin. 

See-eest olin väga hea Lotus 
Notesi\index{Lotus Notes Domino} tarkvaras, mida kasutati suhtlemiseks omavahel 
ja soomlastega. See oli dokumendiandmebaas, millel oli oma 
skriptimiskeel. Sellega ma kirjutasin reisikindlustuse süsteemi 
kindlustusagentidele, et nad saaksid välja arvutada, palju reisimine maksab, 
ja poliisi teha. Ja see oli internetipõhine aastal 1997. Dominoga 
oli võimalik samu dokumente, mida muidu nägi Lotus Notesis kliendi 
kohta, ka veebiserveri kaudu ehk HTML-dokumentidena näidata.

See kogemus aitas mul saada Hansapanga\index{Hansapank} internetipanga 
tiimi.

\question{Kuidas sa sinna sattusid?}

Hansapanga ITs või üldse pankades ilmselgelt 
oli rohkem raha kui mõnes IT-firmas. Kui olin kaks aastat Aeteci
Finantsvaras töötanud, tundsin, et võiks nii-öelda karjääri teha. Proovisin tegelikult 
kõikidesse pankadesse tööle saada, igal pool oli vabu kohti. 
SEBs\index{SEB|see{Ühispank}} ehk toonases Ühispangas\index{Ühispank} ma ei 
saanud isegi vist jutule, aga Hoiupangas\index{Hoiupank} rääkisin Aleksei 
Bljahhiniga\index[ppl]{Bljahhin, Aleksei}. Hansas oli ka tööintervjuu, läksime sinna koos 
Vilve Vene\index[ppl]{Vene, Vilve} ja Heiki Kübbariga\index[ppl]{Kübbar, 
Heiki}. Ja sain mõlemast pangast tööpakkumise umbes sama summa peale. 
Otsustasin Hansapanga kasuks, sest arvasin, et seal võib olla natuke rohkem 
karjäärivõimalusi. 

Minu esimene tööpäev Hansapangas oli 
19. jaanuaril 1998. Fuajeesse astudes märkasin värsket 
Äripäeva, kus oli kirjas, et Hoiupank ja Hansapank ühinevad. Nii et minu 
esimesel tööpäeval teatati ühinemisest ja see määras kogu mu järgneva karjääri.


\question{See tähendab, et pidid suhteliselt ruttu hakkama 
internetipanga asemel tegelema hoopis Light Telleri\label{sisu:teller} nimelise telleri 
töökohasüsteemiga?}

Sinna läks veel natuke aega. Otsus hakata seda tegema sündis 
umbes viis-kuus kuud peale seda, kui ühinemine pihta hakkas. Alguses  
ei olnud ju veel selge, kumba süsteemi üldse hakatakse kasutama ja kuidas see 
otsus tehakse. Sellel ajal õppisin mina, kuidas internetipanka teha.

\question{See kõik on mulle üllatus. Mina läksin sinna panka 
1999. aasta lõpus. Light Teller oli selleks ajaks olemas ja laua taga oli 
vana kala nimega Sergei, kes oli selle oma käega valmis teinud. Kui nüüd 
näppudel arvutada, siis järelikult tegid sa nullist 
täisfunktsionaalse veebipõhise telleri töökoha umbes kolme kuuga?}

Ega ma seda üksi teinud. Aga astume sammu tagasi. Hansapanga esimene 
internetipank oli üles ehitatud tehnoloogiale, mis oli ajast ees. See 
oli Oracle'i\index{Oracle} veebikomponent või -server, kus 
sai PL/SQLiga\index{PL/SQL} tekitada HTMLi, mida kliendid 
vaatasid. See oli omal ajal hästi lihtne, ilma igasuguse disainita, sest 
disaineritest ei teadnud tol ajal vist keegi, et on olemas selline amet nagu disainer. 
Trükidisainerid kindlasti olid, aga kasutajaliidese disaineritest polnud keegi kuulnud. 

Mina mõtlesin, et oo, milline 
ebavõrdsus, et internetipank on ainult eesti keeles. Ütlesin, et ma võin teha selle 
mitmekeelseks. Seepeale öeldi, et tee. Ja tegingi. Kaks nädalat tegelesin 
sellega, et võtsin kõik tekstid välja ja asendasin \verb|lang|-funktsiooniga, mis 
arvestas ka kasutajaprofiiliga. Samal ajal õppisin veel ülikoolis, olin sel
päeval, kui Madis Ollisaar\index[ppl]{Ollisaar, Madis} asja tootmisse pani, koolis. 
Logisin sisse, et vaadata, kas töötab. Eesti keel töötas, inglise keel 
töötas, vene keel aga näitas küsimärke. Ilmselt inimesed mässavad siiamaani nende 
\emph{encoding}'utega, aga see oli minu esimene kokkupuude sellega, et minu 
arvutis töötab, aga serveris mitte.

Samal ajal hakkas ühinemise tõttu juhtuma mitu asja korraga.  
Aleksei Bljahhin\index[ppl]{Bljahhin, Aleksei} tegeles \emph{data} migraga. 
Tekkis probleem, kuna telleriprogramm oli kirjutatud Oracle Formsis ja igas 
kontoris oli Formsi server. Kõik tellerid kasutasid Formsi klienti, mida 
serveeriti serverist, ja nad võtsid peaserveriga Oracle'i 
andmebaasiühenduse. Oracle'i litsentside eest maksti teatavasti ühenduste arvu 
pealt. Hansapangal oli tol hetkel, no ma ei tea, mingi nelikümmend 
kontorit. Nüüdseks see on juba suur number, aga Hoiupangal oli nelisada 
kontorit!. Paljudes maakohtades ei olnud isegi nii head sidet, et 
hoida pidevat ühendust andmebaasiga. Kui nad arvutasid, kui palju 
Oracle'i litsentsid oleksid kokku maksnud, siis nad ütlesid, et võib-olla anname 
Hoiupanga tagasi. 

Tegelikult tehti väga julge otsus teha interneti telleriprogrammi. Otsustajateks olid ilmselt needsamad Vilve\index[ppl]{Vene, Vilve} ja 
Gibbs\index[ppl]{Gibbs|see{Kübbar, Heiki}}\sidenote{Sergei peab silmas Heiki Kübbarat, kes oli toona paljude Hansapanga innovatiivsete ideede taga.}. Samal ajal müüdi meile internetipanga tegemiseks uus tehnoloogia, BroadVisioni\index{BroadVision} platvorm. 
BroadVisioni müügiargumendiks oli, et saame põhimõtteliselt e-kommertsi 
platvormi, millel sai igale kasutajale näidata personaalselt välja nägevat 
rakendust.
Samas iga kasutaja maksis, mis tähendas, et me ei kasutanudki kunagi seda
võimalust, süsteemi mõttes oli kõik anonüümne. Ühtlasi pakkus BroadVision
\emph{template}'imise võimalust, mis oli väga suur samm edasi võrreldes Oracle'i 
PL/SQLiga, kus tuli oma HTML ise kokku panna. Nii et selle peale me internetipanga ehitasimegi. 
Tagantjärele mõeldes oli see telleri arhitektuur lihtne, aga võimas. See võimaldas 
kiiresti ja suures koguses funktsionaalsust toota.

\question{See arhitektuur oli siis toonaseid vahendeid kasutades täpselt selline, nagu tänased \emph{de facto} 
veebirakendused on. JavaScript\index{JavaScript} jooksis brauseris ja 
tegi \emph{backend}'i poole päringuid. See lahendus oli 20 aastat ajast ees, 
kuidas see sündis?}

Meil tuli arvestada piirangut, et maakontorite ühendus oli väga aeglane. 
Pidime optimeerima, kui palju \emph{data}'t kliendi ja serveri 
vahel liigutada. See sundiski palju tööd juba 
kliendipoolel ära tegema. Kliendiks oli brauser ja JavaScripti versioon oli 
selline, et parimal juhul sai teha valideerimist. Midagi joonistada või dünaamiliseks teha eriti
ei saanud. Samal ajal tuli 
Internet Explorer 4.0\index{Internet Explorer}, kus olid \emph{custom} 
JavaScripti võimalused, mis lasid palju dünaamilisemat 
lehte ehitada. Tol ajal ei olnud ju mingisuguseid JavaScripti \emph{library}'sid, nagu 
Reactid\index{React} ja muud, mis võimaldavad kõike teha. Sa kirjutasid puhast 
JavaScripti, isegi Githubi ega Stack Overflow'd ei olnud. Oli Internet Exploreri 
dokumentatsioon.

Ja kuna kõik Hoiupanga töötajad olid harjunud ilma hiireta
terminaliga (hiire kasutamine aeglustab tööd), siis oli ka 
nõue, et kasutaja pidi saama navigeerida brauserirakenduses ilma hiireta. 

\question{Põhimõtteliselt ju tehtav, aga kasutajaliidese disaini mõttes 
päris keeruline ülesanne.}

Arvestades kõiki neid piiranguid pidin välja tulema mingisuguse kliendipoolse 
raamistikuga ja tulin ka. Seal tekkis päris palju koodi ja tol 
ajal tuli tüüpilises brauserirakenduses vajutada \emph{submit}-nuppu, mispeale 
terve leht laeti uuesti. Meil aga ei olnud kontorite vahel \emph{bandwidth}'i. Näiteks kui viis tellerit, kes istusid 28 K 
modemi\sidenote{Sidet üle telefoniliinide 
reguleerisid Rahvusvahelise Telekommunikatsiooni Liidu V seeria 
soovitused. V.34 kirjeldas sidet kuni 33,6 kbit/s, kuigi levinuim oli 
mainitud 28,8 kbit/s kiirus.} peal, vajutas nuppe, hakkas iga nupuvajutusega 
tulema sadades kilobaitides lehte. Tollal tekkisid 
\emph{frame}'id ja \emph{frameset}'id\sidenote{HTML 4.0, mis avaldati 1997. aastal 
W3C soovitusena, sisaldas eraldi variatsiooni \enquote{raamide} (ingl 
\emph{frame}) toega. Raamid võimaldasid jagada HTML-lehe eri aadressidelt 
laetavateks alamosadeks. HTML 5.0 enam raame ei toeta.}, mille vahel sai andmeid 
vahetada brauseri sees. Nii et oligi üks \enquote{menu} \emph{frame}, kus oli 
enamik JavaScripti loogikat, mida kunagi uuesti ei laetud, ja 
\enquote{main} \emph{frame}, mille sees laeti iga konkreetne tegevus.

\question{Seal tehti veel mõningaid huvitavaid asju, näiteks olid peidetud raamid, 
mis käitusid nagu praegune brauserist algatatud REST päring.}

Eks see arenes. Rakenduses oli \enquote{main} \emph{frame} ja 
kliendiandmete \emph{frame}, sest tavaline \emph{workflow} oli selline, et kui 
klient tuli, siis leidsid tema konto ja said seal teha makseid, 
hoiuseid ja mida iganes. Klienti otsides tuli laadida 
tema andmed eraldi kliendiraami, kus olid nähtavad kliendi nimi, konto nimi ja 
kontonumber, aga seal all olid veel ka brauseripoole peal kliendiandmed. Ja siis 
meil oli \enquote{foori} \emph{frame}, mille kaudu \emph{submit}'isime vormi 
andmeid, sest valideerimine pidi jällegi toimuma kohapeal. Nupp käivitas 
valideerimismeetodi, mille tulemusel saadeti andmed teise vormi 
kaudu serverisse. Ma ei mäleta, miks me nii tegime, ju oli vaja. Aga see 
oli nagu raam, mille sees said kõik pangafunktsioonid tehtud. Selle püstipanekuks ja esimese 
Eesti-sisese maksevormi tegemiseks kulus kuu aega. Kui see sai valmis, siis kõik 
ülejäänud funktsioonid tulid kahe kuuga. Põhimõtteliselt 
\emph{copy-paste}, midagi keerulist ei olnud, ainult pärast pisut vigade 
parandamist ja optimeerimist.

\question{Kui sa nüüd tagasi mõtled, siis mis sulle andis põhja, et selline asi teha? Oli see 
ülikool, lihtsalt häkkerimentaliteet või veel midagi?}

Ei olnud mitte midagi peale probleemi, mida oli vaja 
lahendada. Muidugi oli sealjuures ka muid nõudmisi, millest ei 
saanud üle ega ümber. Näiteks tellerirakenduse puhul oli spetsiifiline nõue, et see ei tohi inimest väsitada, st me
ei tohtinud kasutada erksaid värve, sest selle programmiga tehti päevas kaheksa tundi 
tööd. Sellepärast see saigi hall. Tol ajal me tegime ka hanza.net'i\index{Hansapank!hanza.net} 
ja see oli värviline, disaini mõiste oli juba olemas.

\question{Sellise asja peale tänapäeval sageli isegi ei mõelda, kust 
see nõue tuli?}

Meil oli tubli pangatehnoloogia osakond, kes mõtles, kuidas tellerid saaksid oma tööd teha
hästi efektiivselt. Kordan, et mina olin ainult 
teostaja, asja taga oli terve tiim. Meil oli Toomas Rand\index[ppl]{Rand, 
Toomas}, kes kirjutas kogu pangaloogika; mina tegelesin 
ainult kasutajaliidesega ja andsin talle andmed. Pangasüsteemis 
toimuv oli tema teha ja ta istus täpselt samamoodi kaksteist tundi päevas töölaua taga ja 
tegi. Tänu sellele projektile tekkis pangasüsteemi arhitektuuris 
korrastatus. Oracle Formsiga sai kutsuda suvalisi funktsioone otse vormist, seevastu kui 
meie arhitektuuri ütles, et üks nupuvajutus ja ongi kogu tehing tehtud. Ennekõike tuli kokku leppida liideses 
ja siis said osapooled oma osaga edasi tegeleda. See 
võimaldas testimist, testimise automatiseerimist ja töö paralleliseerimist. 

Kitsendused sunnivad tegelikult tegema õigeid otsuseid. Paljudel inimestel ei ole piiratud  
ressurssidega toimetamise kogemust, eriti välismaalastel. Näiteks tuleb Silicon Valleyst inimene, kes ei saa aru, miks
me ei palka inimesi juurde. Mis mõttes ei saa kõiki oma ideid realiseerida, vaid
peab prioritiseerima? See on tema jaoks probleem, kuna ta ei saa 
aru, mis tähendab, et raha ei ole. Näen, et Eestis 
on palju ära tehtud väga vähese ressursiga täpselt selle pärast, 
et inimesed oskavad teha õigeid valikuid. Prioritiseerima peab, sest ressurssi ei 
ole.

\question{Kui ilusast arhitektuurist edasi minna, siis milline on ilus kood?}

Ilus on kood siis, kus inimene ei pea küsima, mida see teeb. Väga 
paljud, kes oskavad programmeerida, arvavad millegipärast, et mida 
optimeeritum või lakoonilisem kood on, seda parem, kuid see teeb halba. On piir, kust
edasi teine inimene ei saa enam aru, mida kood teeb. Selline kood ei ole hea, isegi kui teeb õiget asja. See on üks asi. Teiseks pean ma 
ütlema sulle suur aitäh selle eest, et tõid omal ajal Eestisse Joshua 
Kerievsky\index[ppl]{Kerievsky, Joshua}\sidenote{Joshua Kerievsky on USA firma 
Industrial Logic asutaja ja üks pikema kogemusega agiilse tarkvaraarenduse 
praktikuid ja koolitajaid maailmas. Tema Eestisse toomise Hansapanga arendajate 
koolitamiseks kas 2000. aasta lõpus või 2001. aasta algul algatas siiski Erik 
Jõgi\index[ppl]{Jõgi, Erik}}. Elus tekivad hetked, kui saad aru, 
et see on nüüd \emph{step function}. Tema koolitus viis
väga paljud asjad oma kohale. Joshua on tegelenud koodi \emph{refactor}'iga ehk kuidas teha kehvast koodist ilusat. Samuti rääkisime temaga \emph{unit}'i 
testimisest \ldots

See aitabki ilusat koodi kirjutada: sa 
pead seda mitu korda ümber kirjutama, enne kui see näeb loogiline välja.

\question{Tagantjärele mõeldes oli kogu see 
agiilse arenduse liikumine ja mõtteviis tol ajal veel väga noor.}

Kui ma tulin Skype'ist\index{Skype} Pipedrive'i, siis siin on meil 
igasugu \emph{agile coach}'e. Ma korraldasin sellise eksperimendi, et rivistasime oma 
\emph{agile coach}'id, arendajad selle järgi, kes on 
\emph{agile} liikumisega kõige kauem tegelenud või sellest vähemalt teadlik olnud. Enamiku puhul oli see aeg 7-8 aastat. Mina olen sellega 20 
aastat tegelenud! \emph{Agile Manifesto}\sidenote{Vt. \url{https://agilemanifesto.org/}} tekkis vist 2001. või 
2002. aastal. Tegelikult me kõik saime seda maitsta enne, kui see popiks muutus.

\question{Mida sa praegu teed?}

Ma isegi ei saa öelda, et juhin \emph{engineering}'u 
organisatsiooni, sest ma juhin ka muid organisatsioone. Olen 
Pipedrive'is\index{Pipedrive} juba seitse aastat olnud. Aastal 
2013 meeskonnaga liitudes oli see väike ja ambitsioonikas firma. Tööintervjuul küsiti minult, kas ma usun, et suudame Salesforce'iga võistelda. Ütlesin, et päris 
Salesforce'iks me ei kasva, aga võib-olla veerand sellest on võimalik. Siis 
oli meil kakskümmend inimest, kümme inseneri. Nüüdseks, kuus aastat hiljem ja natuke peale, on meid kuussada.

Kõik need aastad olen tegelenud skaleerimisega: nii infosüsteemi kui ka 
organisatsiooni skaleerimisega. Selle aja jooksul ei ole kordagi tekkinud mõtet, et 
äkki meil ei õnnestu, äkki me ei kasva. Niipea kui hakkad niimoodi mõtlema, siis 
ei kasvagi. Ma ei ole tegelikult siiamaani kindel, kumb on põhjus ja 
kumb tagajärg: kas see, et oleme skaleerinud \emph{engineering}'ut, 
aitas Pipedrive'il kasvada või see, et ta kasvas, aitas meil skaleerida 
\emph{engineering}'ut.

Kui vaadata teisi osakondi, siis näiteks turundus ei skaleerunud. 
\emph{Product} pidi skaleeruma koos \emph{engineering}uga, muidu inseneridel 
poleks midagi teha. Müük ei skaleerunud, \emph{support} skaleerus 
nii-öelda tagantjärele. Tegelikult \emph{engineering}'u kasvatamine 
kasvatas firmat. Samas, kui ettevõte ei kasvaks, siis ei saaks ju ka 
inimesi juurde palgata. Küsimus on, 
mis tõukas kasvu tagant. Me eriti ei mõelnud sellele, vaid olime 
kindlad, et peame skaleeruma. Minu kõige suurem hirm on olnud jääda pudelikaelaks. See, et \emph{engineering}'u peale hakatakse 
näpuga näitama, et tahaks küll seda või toda teha, aga 
\emph{engineering}'ul ei ole ressurssi või süsteemid hakkavad 
kokku kukkuma, kui kliente on liiga palju. Või et palkame inimesi juurde ja 
nad ei saa oma tööd teha, sest kuskil protsessis on pudelikael. 
Või me ei saagi inimesi palgata, sest nad ei taha meile tööle tulla. Neid 
pudelikaelu, millega korraga tegeleda, on olnud palju. Aga kui kuidagi ei saa, siis kuidagi ikka saab!


\chapter{Arne Ansper}
\index[ppl]{Ansper, Arne}
\question{Nagu ikka alustame sellest, kuidas asjad alguse said. Kuidas nad siis 
said sinu jaoks alguse?}

No minu jaoks need asjad alguse sellest, et kui ma põhikooli lõpetasin siis 
minu matemaatikaõpetaja arvas, et ma peaksin minema Nõkku\index{Koolid!Nõo 
Keskkool} edasi õppima. Ja suutis mu vanemaid ära veenda, et see on suurepärane 
mõte, siis ma sinna läksingi.

\question{Aga kus sa põhikooli lõpetasid?}

Jõõpres\index{Koolid!Jõõpre kool}\index{Jõõpre}, selline pisike koht Pärnu 
lähedal. Sada õpilast oli see põhikool meil vanas  mõisas, mitte mõisamajas 
endas aga koolimaja oli mõisa keskel. Niisugune väga mõnus koht oli. Ja siis 
mul matemaatika nagu sobis ja õpetaja oli väga usin, andis mulle lisaülesandeid 
ja lõpuks saatis olümpiaadile ja seal läks ka suht hästi.


\question{Sa siis tulid puhtalt matemaatika ja mitte arvutite nurga alt sinna 
Nõkku?}

Ei, mul oli null kokkupuudet arvutiga enne. Vanemad seejuures pigem nagu 
tahtsid, et ma läheks. Ma ise olin väga  kahtleval seisukohal, et kas kodust 
nii kaugele minek, et see on äkki kuidagi raske ja paha ja nii edasi. 

\question{Mis aastal see oli?}

1985. 

\question{Sel ajal oli juba logistiliselt ju keeruline Pärnu lähedalt Nõkku 
saada?}

See oli lihtne ja tüütu, selles mõttes, et olid bussid, mis sõitsid neli tundi 
ja olid tavaliselt maast laeni rahvast täis ja siis veel Pärnust koju kus buss 
käis kahe tunni tagant. Seal ikkagi võttis aega, ütleme nii.  

\question{Ja Nõos pandi kohe arvuti ette?}

Ei, Nõos see oli tavaline keskkoolielu selle väikse vahega, et tuli ühikas 
elada. Mina olin viimane aasta, kes elas poiste ühikas, mis on selline 
suhteliselt raju ja legendaarne koht. Ehitatud kuskil tsaariaja lõpus, Eesti 
aja alguses. Talvel oli niimoodi, et tulid  kodust, tõid sihukesed suured 
märjad puunotid, läksid oma tuppa, mis oli  null kraadi lähedal kütsid ta siis 
üles selleks, et magada saaks. Hommikul lõid ikkagi pesukausi pealt jää katki, 
kui hakkasid hambaid pesema, niisugune koht oli. Esimene aasta oli hästi lahe. 

Alguses oli tavaline keskkond ja siis tuli programmeerimise õpetamise lihtsalt 
ühe regulaarse ainena sisse ja hakati õpetama. See oli ikkagi matemaatika ja 
füüsika kallakuga kool aga programmeerimise õpetamine seal oli lihtsalt nagu 
aine nagu mida iganes muugi. Mahud, loomulikult, olid suuremad nii 
matemaatikal, füüsikal kui ka sellel, programmeerimisel, millel mujal oli null, 
et seal oli siis nagu mingi muu number.

\question{Räägi palun Nõo kooli taustast, kuidas sinna üldse sai?}

Tead, ma ei tea. Mina olin tollal niisugune inimene, et emaga koos me sinna 
läksime. Ma arvan, et me käisime direktori juures rääkimas. Et kuna mul oli 
tegelikult olümpiaadilt mingisugune koht ette näidata siis kuidagi ma sinna 
igatahes sisse sain. Kuidas täpselt, kas seal oli mingi konkurss või mingi muu 
süsteem, ei tea. 

\question{Kes Nõo kooli direktor tol ajal oli? See kool tundus kellegi 
entusiasmi peal käivat?}

Enn Liba\index[ppl]{Liba, Enn} oli minu meelest tol ajal direktor\sidenote{Nõo 
kooli arendas selliseks reaalteaduste ja programmeerimise õppe keskuseks, nagu 
me teda praegu tunneme, Kalju Aigro\index[ppl]{Aigro, Kalju}. Ta oli kooli 
direktoriks aastatel 1951---1982, talle järgneski selles ametis Enn Liba.}. Aga 
seda entusiasmi aspekti ja ajalugu, ma pean tunnistama,  ma ei oska 
kommenteerida tollal huvitusin  oluliselt muudest asjadest.

\question{Aga mis asjad need olid, millest sa huvitusid?}

Tegelikult mulle meeldis põhikoolis elektroonika. Aga see oli selline 
platooniline huvi, kuna juppe oli hullult raske kätte saada. Ja mulle meeldisid 
mudellennukid, mis oli ka suhteliselt platooniline. Aga Nõos tuli 
programmeerimine hästi kiiresti peale, kui hakkasime seal õppima. Seal oli suur 
Vene \emph{mainframe} Nairi-3-1\index{Arvutid!Nairi-3-1}\sidenote{1964. aastal 
Jerevanis välja töötatud Nõukogude arvutiperekonna Nairi kõige võimekam liige. 
Kool sai selle arvuti 1977. aastal.}. 
KÕPS\index{Keeled!KÕPS} ja ROPS\index{Keeled!ROPS}\sidenote{\label{sidenote:ROPS}KÕPS ja ROPS on 1980. 
aastate teisel poolel Nõo Keskkooli arvutuskeskuses välja töötatud eestikeelsed 
programmeerimiskeeled, millede loomisel osales ka Arne esimene arutiõpetaja Nõos 
Uuno Puus\index[ppl]{Puus, Uuno}. KÕPS oli sarnane MIT-is välja töötatud keelega 
LOGO, võimaldas vaid graafikat ning tugines LOGO looja Seymour Papert-i ideoloogiale. 
ROPS oli KÕPS-i edasiarendus, mis olla sarnanenud Algolile ja võimaldas lisaks 
graafikale ka arvutusi.}, eesti keeles sai 
programmeerida, need olid  vahvad. Siis olid seal Agatid\index{Arvutid!Agat}, 
mille ligi suht ruttu sai, mis olid teistmoodi vahvad, kus sai mingit 
valmistarkvaraga ka kasutada. Ikkagi mingite mängude mängimine oli oluline ja  
siis ise mingite asjade proovimine. See nagu hakkas väga kiiresti meeldima.

a \question{Oskad sa takkajärgi kuidagi reflekteerida, mis sulle seal meeldima 
hakkas?}

Väga ei oska, ausalt öeldes. Ma üritasin mõelda, et mis ma siis tegin nende 
arvutitega toona. Mul on umbes kaks asja meeles mida ma Agatiga tegin. Esimene 
programm oli umbes see, et oli \verb|for| tsükkel: muutis värvi, trükis mingi 
teksti nagu, ütleme, \enquote{tere}. Kõigis keeltes ja siis veel vilkuva 
taustaga ka. Sellega sai vähemalt üks õhtu kui mitte kauem möllatud ja timmitud 
neid efekte, tekste ja asju. Ja siis teine asi, mis mul on meeles, ma püüdsin 
ühte Nintendo mängu (need pisikesed puldi mängud, mis olid\sidenote{\label{sidenote!gameandwatch}Arne peab 
ilmselt silmas Nintendo Game \& Watch\index{Nintendo Game \& Watch} seeria käes hoitavaid mänge. 
Originaalidest oluliselt rohkem oli liikvel nende Nõukogude kloone, mida müüdi 
Elektronika kaubamärgi all. Tegu polnud siiski alati täpsete koopiatega: 
Nintendo EG-26 kloonis IM-02 püüdis mune Miki Hiire asemel hunt tuntud 
Nõukogude multifilmist \begin{russian}Ну, погоди!\end{russian}}) taasluua, ma 
lõingi. Seal oli, nagu ta on, mingi fikseeritud arv positsioone, mingi tegelane 
liikus, mingid teised tegelased liikusid ja siis olid mingid surmasaamised ja 
mingid boonuste saamised. Probleem oli selles, et ma ei teadnud tollal, mis asi 
on massiiv. Põhimõtteliselt oli niimoodi, et iga objekti jaoks oli mul muutuja, 
mis ütles, et kas objekt on või ei ole. Ja kui seal mingid asjad liikusid, siis 
mul oli lehekülgede kaupa \verb|if| lauseid, et kui see muutuja omab seda 
väärtust, siis järgmisel sammul ta omab teist väärtust. Ja muidugi 
refaktoreerimis-tööriistu ei olnud. Kui ma kuskil vea tegin, siis ma nägin 
päevade kaupa vaeva, et ma nimetasin neid oma muutujaid ja \verb|if| lauseid 
ümber.


\question{Väga huvitav. Tol ajal tundus asjadest mitte rääkimine olevat 
õpetamise metoodika osa. Meile näiteks ei räägitud \texttt{for} tsüklist tükk 
aega}

Ütleme nii, et seda Agati\index{Arvutid!Agat} ei õpetanud meile keegi. Õpetati 
Kõpsi ja Ropsi. Kõik, mis Agati peal sai tehtud, see oli puhas enda välja 
võidetud ja võideldud  arvutiaeg, enda entusiasm. Ma isegi ei mäleta, \emph{by 
example} käis see asi vist, et vaatasid, mida keegi teine oli teinud. Mina küll 
ei mäleta, et oleksin ühtegi, Agati või BASICu\index{Keeled!BASIC}  kohta 
käivat raamatud lugenud kunagi. Kõik see oli lihtsalt nagu folkloor, 
katsetamise ja kõlakate tasemel. Et oleks keegi lekitanud selle info, et 
massiivid on olemas, oleks selle Nintendo mänguga palju rutem valmis saanud. 

\question{See oli suur töö ju, pidi ikka kihu olema?}

No aega oli palju, segavaid faktoreid oli vähe, eks ole. Ja ilmselt siis see 
arvuti alistamine meeldis, nagu välja tuleb. Agatiga\index{Arvutid!Agat} ma 
mäletan seda kindlasti, et ma hankisin endale selle 
assembleri\index{Keeled!Assembler} nii-öelda manuaali. Mis oli põhimõtteliselt 
paar-kolm ruudulist lehte, kuhu ma siis kirjutasin tähtsamad käsud ja registrid 
ja värgid üles ja siis studeerisin seda. Ja ma tean, et ma ikkagi nagu 
tuuseldasin seal Agati assembleri poole peal ringi. Aga mida ma tegin, seda ma 
kindlasti ei mäleta. Mäletan olulisimaid registreid, mida näppides käis piiks 
ja kust sai lugeda mingit vist klaviatuuri sümboleid või midagi sellist, aga 
\emph{that's it}.

\question{Kuidas Nõos tase oli, seal olid kõik sinusugused koos?}

Seal oli  selliseid inimesi, kes olid üle vabariigi kokku tulnud, kellel olid  
mingid huvid ja eeldused  reaalainetega tegelemiseks. Aga seal oli ka noh 
lähikonna inimesi. Et see on nagu päris, selline geto kuskil, see oli ikkagi 
nagu natukene spetsialiseeritud kohalik kool, et seal oli igasuguseid inimesi

\question{Kas sealkandis mingit äri tegemist ka juba käis, keegi raha eest 
programmi ei kirjutanud? Kaheksakümnendate lõpp ikkagi?}

Võib-olla keegi tegi, aga  ma julgeks öelda, et ma isegi ei huvitunud sellest 
ja ma ei tea sellest midagi. 

\question{Tartu vahet ka käisite?}

Jaa. Mingil hetkel, ma ei mäleta enam mis klassis, aga siis ma sain teada, et 
Tartu Ülikooli Raamatukogus\index{Tartu Ülikool!Raamatukogu} on mingisugune 
XTde\index{Arvutid!XT} klass. Kaheksa kuni kümme arvutit oli seal. Kuidagi ma 
sain sinna juurde, ma ei mäleta, mis alustel sinna seda aega sai reserveerida. 
Igatahes ma tean, et ma seal ikkagi jõlkusin päris mitu õhtut nädalas. Sa 
said seal mingisuguse tunni või kahese \emph{slot}i, mul oli umbes kaks flopit, 
millest ühe peal oli Turbo C\index{Keeled!Turbo C} ja teise peal oli siis tüüpi 
opsüsteemi oma asjad ja siis  midagi ma seal programmeerisin. 

\question{Aga kust sa said tolle Turbo C?}

Ma ei kujuta ette, kus ma selle saada võisin. Seal ma käisin päris tükk 
aega aga seal ma põhiliselt tegelesin ka sellega, et mängisin selle Turbo Cga. 
Aga kas mul ka mingi eesmärk oli, seda ma ei mäleta. Aga Turbo C see oli 
igatahes.


\question{On ikka paras hüpe Kõpsust ja Ropsust C ja mälu ja pointeriteni? 
Mille pealt too hüpe tuli sul?}

Jällegi nii kauge aeg, et ma kardan, et meile koolis ma isegi mäletan seda, mis 
meile üheksandas klassis  programmeerimist õpetati, aga ma ei mäleta, mis edasi 
sai, ausalt öeldes. Mida meil seal üldse räägiti. Ilmselt ise liikusin 
kiiremini edasi. Pärast TPIs\index{TPI|see{Tallinna Tehnikaülikool}} 
\index{TPI} ka see asi esimestel kursustel, et need  programmeerimise loengud 
olid  sellised, et sealt ei olnud midagi uut saada. Seal  mingid teised asjad 
olid pigem  need, mis olid uued, aga mitte see programmeerimise pool. 


\question{Kuidas sa sealt Nõost TPIsse\index{TPI} sattusid? Oleks ju loogiline, 
et sa lähed sealt Tartusse matemaatikasse?}

See oli ka suht \emph{random}iga selles mõttes, et ma mõtlesin, et võib sinna 
minna või tänna minna. Need argumendid, miks  Tallinnasse proovida, need olid 
niisugused väga otsitud ja õrnad, et miks ma just sinna Tallinnasse läksin 
proovima,  seda ma tegin. 

\question{Mida õppima?}

LI\index{Tallinna Tehnikaülikool!Automaatikateaduskond!LI}. Ma täpselt ei mäleta, kas oli arvutid ja 
arvutisüsteemid, tõenäoliselt võis olla.

\question{See LI lühend jookseb mitmelt poolt läbi aga keegi ei tundu teadvat, 
mida see tähendas}

Kas ta üldse midagi tähendas? Et \enquote{L} on tõenäoliselt mingi 
automaatikateaduskonna kood, eks ole, ja \enquote{I} on mingi muu asja kood. 
Seal oli LA, mis oli äkki rohkem automaatika teisi tähti ei mäleta, äkki on LS 
ka olemas olnud. LI  oli jah see, kus mina oma aega veetsin.

\question{Sa ütlesid, et programmeerimise õpe sind väga edasi ei aidanud, kas 
seal üldse midagi õpetati, mis sulle midagi juurde andis?}

Tagantjärgi  vaadates tundub, et  seal LI-s räägiti nagu laiuti alates sellest, 
kuidas transistori teha, kuidas transistoridest saaks teha mingeid 
mikrolülitusi, kuidas saaks kõik see, mis sorti registrid meil on, kuidas 
registritest mingit automaatikat ehitada. Kuidas protsessorit teha, kui sul on 
neid registreid hulgi käes. Ja  teisele  poole minnes ka, kõik sellised asjad 
nagu siduteooria. Need asjad andsid, tagasivaates, need teadmised, et kui sa 
vaatad tänapäeval enda ümber, siis maagilisi asju, mille kohta ma ei tea, et 
kuidas seda saaks teha või ma pean uskuma midagi või ma vajaduse korral ei 
saaks sinna lõpuni välja kaevuda, neid on väga vähe. Ja see, ma arvan, on üks 
asi, mis mina olen leidnud, hästi kasulik. Tänapäeval on neid kihte sinna nii 
palju juurde tulnud, et vanasti oli ikkagi väga lihtne. See oli umbes nagu 
renessansiajastul, kui üks tüüp suutis  kõike, mida oli mõtet teada, teada. 
Natukene, kui  mina seal TPIs käisin, see aeg hakkas läbi saama. Ütleme 
niimoodi, et tänapäeval ilmselt ei ole võimalik, et sa tead kõike, mida oleks 
kasulik teada arvutiasjandusest. Ma mõtlen just tänapäeval seda, mis riistvara 
poole peal on juhtunud. Sinna on laotud neid kihte ja neid virtualiseerimise 
tasemeid ja mida iganes veel juurde. Ja siis \emph{soft}i poolel on ka vastu 
tuldud, sinna neli kihti virtualiseerimist vahele laotud ja nii edasi. See on 
nagu see, kus kipub nagu raskeks minema see järje pidamine.

\question{Kas TPIsse minek oli asjade loomulik käik või oli sul mingi plaan ka, 
mida tegema hakata?}

Mul niisugused pikaajalisi plaane ausalt öeldes ei olnud. Mulle meeldis teha, 
mulle meeldis nende arvutitega mässata, kas ma mässan Tallinnas või mässan 
Tartus, vahet pole. Ja siis ma mässasin nendega Tallinnas. Üks huvitav nüanss 
on veel see, et et umbes seal keskkooli lõpus ma sain isikliku arvuti ka. See 
oli midagi teistsugust, see oli Atari 520 STf\index{Arvutid!Atari 520 STf}. Mis 
oli siis Atari Motorola 68000 prosega tükk. 512kB oli tal mälu, selle ma 
\emph{upgradesin}  ühe megani mingil hetkel. Selle peal ma siis elasin ja 
siis selle peal ma püüdsin nagu süvitsi minna kogu sellega, mis seal nagu teada 
oli. 


\question{Kust sa sihukese aparaadi said kaheksakümnendate lõpus?}

Mul olid vanaonud, kes elasid Rootsis. Ema ja isa ükskord käisid seal ja siis 
sealtkaudu ma selle siis sain. 

\question{See pidi Agati kõrval ikka ulmeline aparaat olema}

Tegelikult oli niimoodi, et teised olid PCde peal. Kui ma nüüd vaatan, siis 
need inimesed, kellega me siis igal pool nagu koos ringi käisin, siis noh 
üheksakümnenda aasta paiku umbes, normaalsed inimesed said PCdele ligi ja siis 
toimetasid nendega.  Ja siis minul oli kodus Atari  ja tegelesin sellega 
põhiliselt. 

\question{Ataril on kihte vähem, sai lihtsamini sügavale välja minna}

Jaa, see oli nagu hoomatav täiesti,  mis seal toimus, midagi väga ulmelist 
polnud. Natuke mängisin ka, aga mitte liiga palju. Mul ikka see 
programmeerimine meeldis kõige rohkem selle asja juures. Selle Atari peal ma 
tegin igasuguseid imelikke asju.

Ma üritasin CAD programmi teha, joonistamisprogrammi. See isegi lõpuks selles 
mõttes töötas, et seal sai teha ringe ja jooni, igast värke, salvestada ja 
laadida. Ja siis mul oli, tagasi vaadates jälle hullumeelsus, et mulle nagu 
kohutav tegi muret see, et mälu saab otsa. Et kui sa teed dünaamilist 
mäluhaldust, eks ole, et siis saab mälu otsa. Üritasin seda siis minimeerida. 
Näiteks mulle tundus, et nagu lokaalsed muutujad, mis on \emph{stack}is, on 
kuradi ebaefektiivsed. Ja sisuliselt see CAD programm oli kirjutatud 
sajaprotsendiliselt globaalsete muutujate otsa. See oli täiesti hullumeelsus 
nagu tagasi mõeldes, seal tuli ikka kõvasti refaktoreerida, sest ma ikkagi panin 
täitsa puusse alguses. Seal seda loll ümberkirjutamist oli nii palju, sealt ma 
sain selgeks, et okei, nii ma mitte kunagi rohkem ja mitte ühtegi asja ei tee. 
Väga-väga palju vigu sai igatahes tehtud.

\question{Eks see on ju õppeprotsess, mõnda asja teoreetiliselt selgeks ei saa}

Jah, absoluutselt nõus. Ütleme, et nii võimekaid inimesi, kes kogu aeg teiste 
vigadest õpivad, et neid väga palju ei ole. Ikka enamus kipub oma vigadest 
õppima. 

\question{Kui sa TPIsse\index{TPI} jõudsid, kas sa seal teisi omasuguseid ka 
kohtasid?}

Meil oli hästi lahe kursus. Aga tegelikult oli niimoodi, et seal TPI ja alguses 
ma ikkagi õppisin, eks ole. Mis sest, et seal programmeerimise vallas mul ei 
olnud väga huvitav, aga neid muid ained ma ikka õppisin korralikult. Ma olen 
ikka väga usin õppur olnud. Ja mul juhtus niisugune asi, et mind 
Tarvi\index[ppl]{Martens, Tarvi} kutsus ühel hetkel Ektaco-sse\index{Ektaco}, 
ma arvan, et see oli üheksakümmend üks aasta. Ja see oli siis see 
\emph{community}, kus ma siis hakkasin nagu inimestega koos olema ja oli siis 
ka töise karjääri algus. Ma arvan, et see võis olla, see võis olla 1991, aga  
sada protsenti kindel ei ole. Mingi kolmas kursus äkki umbes.

\question{Kolmas kursus on üsna hilja ju?}

Tegelikult ongi see, et programmeerimise õppimine, üldse arvutiasjanduse 
õppimine võtab ikkagi aega. Ma tagasi vaadates mõtlen, et mis ma siis tookord 
oskasin või kuidas ma mõtlesin või  kuivõrd hästi ma siis programmeerisin.  
Ütleksin, et palju varem ei ole mõistlik seda tööd üritada teha. See võib  
frustratsiooni tekitada. Mis mul oli, ma olin ikka viis aastat nüüd innustunult 
selle asjaga tegelenud. Ma arvan, et kui ma  tööle sain, siis ma olin ka noh, 
enam-vähem miinimumtasemel, kus oleks  mõistlik, et keegi annab sulle 
ülesandeid, millele on ka mingi tähtsus ja tähendus ja sa teed nad  ära.

\question{Kas sul midagi sellist ei olnud, nagu inimesed on rääkinud, et 
lihtsalt arvutiaja saamiseks tekkis mingi arvutiklassi admini koht?}

Ei, mul ei ole ju midagi taolist. Ütleme tõesti mälu võib olla natuke petab, et 
mis aastal mul see Atari sinna täpselt tekkis, aga mul kuidagi oli alati 
mingisugune võimalus olemas, nii palju, kui mul seda tarvis oli ja sellest 
piisas. 

\question{Oskad sa mõnda näidet tuua, mida sa seal Ektacos alguses 
programmeerisid?}

Ektaco oli niisugune  firma, kus tehti riistvara ja tarkvara. Ta tegi 
tööstuskontrollereid, automatiseeris tehaseid, eks ole. Ja olid need 
sardsüsteemid, seal on väiksed mikroprotsessorid neid oli vaja programmeerida 
ja need programmaatorid olid kallid. Ja siis Ektaco hakkaks tegema oma 
programmaatorit. Põhimõtteliselt mingisugune lisaseade PCle, millega sa saad 
neid kivisid kõrvetada. Üks teine tüüp, kes oli nagu riistvara poole peal (ma 
ei tea, aga ma arvan, et ta oli umbes nagu mina, värskelt laekunud staatuses) 
ja mina tegin siis softi. See oli selles mõttes nagu päris huvitav, et meil oli 
PC/AT platvorm, seal oli ISA siin ja selle arvuti me süstemaatiliselt kogu aeg 
ajasime ikka täiesti lukku. Ja selleks, et saaks mingit sotti, siis meil oli 
seal siuke äge asi nagu loogikaanalüsaator. See on niisugune aparaat, et kui 
Ostsilloskoobiga saab visualiseerida mingit analoogsignaali, siis 
loogikaanalüsaatoril on palju-palju pisikesi klemme, mis sa paned kuskile prose 
või mingite digitaalsignaalide külge. Siis sul on teine arvuti mis  
visualiseerib, et kuidas need signaali mustrid on ja siis sa saad panna 
\emph{triggereid}, et kui mul tekib selline muster,  siis salvesta ja 
taasesita. Ehk et kui me ajasime selle selle PC täiesti hulluks, siis me saime 
sealt loogikaanalüsaatori pealt pärast vaadata, et mis siis juhtus, et mis me 
valesti tegime. Ühesõnaga tema siis tegi riista ja kirjutas siis sinna 
kontrolleri peale programmi ja mina kirjutasin PC peale siis põhimõtteliselt 
draiverite programmi vastu, mis omavahel suhtlesid. Ja siis tegin sellele ka 
kasutajaliidest.

Meil olid igasugused Inteli ja IBM-i \emph{manual}id laua peal, neid me siis 
seal sobrasime ja dekodeerisime, et mis me peame nüüd tegema, et siit 
midagigi läbi läheks. 


\question{See kõlab kuidagi hästi süsteemse ja korraldatud ettevõtmisena?}

Ei, see oli hull häkkimine. Nojah, Ektacos seda kraami, mille abil nagu häkkida, 
seda oli ja meil meil oli võimalus seda kasutada. Ja tegelikult ma tõesti selle 
teise tüübi  tausta ei tea, et võib-olla tema oli  kuidagi kogenum, tema tuli 
ju loogikaanalüsaatoriga sinna laua taha. Aga see oli suhteliselt niisugune 
kasulik ja kergesti omandatav seade, et noh kuidas sa seda pruugid. 

\question{Jah, aga võrreldes sellega, kui (nagu on räägitud) inimesed 
vaibanoaga emaplaadi pealt radu maha kratsisid, et modem tööle saada on tegu 
ikka \emph{high-tech} häkkimisega}

No me tegime ikka sinna radu juurde selleks et see kuidagi tööle saada, me ei 
kratsinud midagi maha! Mina ise seda riista-poolt tol ajal ei puutunud. Ehkki 
meil Ektacos programmeerija töövahendite hulgas oli kindlasti tinutus kolb, et 
nii raua lähedal oli seal see enamus sellest elust. 


\question{Kas te saite tööle ka selle kupatuse?}

Ja, loomulikult. Ja siis sellega seoses muidugi, kuna  see oli veel see aeg, et 
 see Borlandi\index{Borland}\sidenote{Borland Software Corporation oli 1983. 
aastal asutatud ja eri nimede all siiani toimetav tarkvaraettevõte, tuntud 
eelkõige arendajate töövahendite poolest. Neist kuulsaimad olid 
\enquote{Turbo-} eesliitega keeled Assembler, BASIC, C, C++, Pascal ning hiljem ka Delphi}
toodang, igasugused Turbo-blaahid, mis neil olid, need olid nagu  standard, eks 
ole. Siis loomulikult sai kirjutatud oma akendussüsteem, mis nägi välja nagu 
see Borlandi Turbo Vision\index{Turbo Vision}\sidenote{Borlandi poolt 
üheksakümnendate alul arendatud tekstipõhine kasutajaliidese raamistik Pascali 
ja C++ jaoks}, aga oli hoopis parem ja teistmoodi tehtud ja seega töötas väga 
kenasti. 

\question{Milles see väljendus, et ta parem oli?}

Ta oli nagu ägedamini struktureeritud. Siis mul hakkas juba 
C++\index{Keeled!C++}  meeldima, ta oli hullult objektorienteeritud. Tal olid 
mingid oma kontseptsioonid, et kuidas sa neid aknaid ja asju  esitad, kuidas sa 
sündmusi käsitled  selles mõttes, et sul on klaviatuur ja hiir. Mingi asi on 
fookuses, kuidas need sündmused jõuavad õige objektini, ja see on  klaviatuuri 
ja hiire puhul väga erinev loogika. Ja kõik see oli selliseks loogiliseks 
kompotiks keeratud, et sinna oli lihtne rakendusi teha. Sellel tükil oligi 
umbes üks programm, mis  seda ägedat raamistiku kasutas, see oli see sama 
programmaatori kasutajaliidese. Aga noh, selles mõttes oli Ektaco väga tore, et 
need tööülesanded ei olnud väga piiravad. Sa võisid ikkagi, ma ei tea, 
kuude või isegi aastate kaupa rahulikult häkkida ja sealt lõpuks tuli mingi asi 
välja. 

\question{Ja teistpidi, ega sul ei olnud neid akende joonistamise asju võtta 
riiulist kümneid?}

Ei, ikka oli. Sedasama Turbo Visionit oleks võinud pruukida ja seal oli 
igasuguseid teeke. Aga kuidagi, mis see siis on, nagu ametiuhkus ei lubanud 
teise mehe akna teeki kasutada. Tuleks ikka enda oma teha, sest et no mis 
mõttes, ma ei oska nüüd parimat akendusteeki teha. 

\question{Sellist suhtumist pannakse tänapäeval pahaks? Või ei panda?}

Seda tehakse teisel tasemel, eks ole. Tasemeid on juurde tulnud, seal 
nokitsetakse hoopis mingisuguste muude asjade juures, aga mina arvan, et see on 
nagu suht paratamatu, et see on hädavajalik, et inimesed heas mõttes 
leiutatakse jalgratast. Teeks asju, mis on juba tehtud, aga teeks teistmoodi, 
teeks paremini. Põhimõtteliselt olid ju opsüsteemid olemas, et mis mõte oli 
seda Linuxit hakata tegema, PC-Unix oli olemas. See oli olemas, et no mis siis 
häda oli sellel SCOl või millel iganes. 


\question{Jah, põhimõtteliselt oleks ju võinud olla, et siiamaani kõik 
kasutaksid sinu aknategijat}

Kindlasti need inimesed, kes on armunud kaheksakümmend korda kakskümmend viis 
teksti ekraanisse, need oleks olnud siiamaani selle andunud kasutajad. 

\question{Mäletan, FoxPro\index{FoxPro} joonistas lausa mingeid varjusid 
akende taha}

Ja, see on loomulik, varjud akendel pidid olema.

\question{Kas seda teie kiibikõrvetajat kasutati väljaspool 
Ektacot\index{Ektaco} ka?}

Need asjaolud muutusid nii kiiresti, et see, mis oli kallis ja kättesaamatu 
kaks aastat tagasi,  kaks aastat hiljem ei olnud enam seda. Ja ma arvan, et 
seda võib-olla tehti mingi üks või kaks eksemplari ja seda pruugiti Ektaco 
siseselt, aga sellest mingit edulugu ei tulnud. Ja see ei olnudki põhitegevus. 
Mina jälle ei tea, eks ole, et miks seda üldse tegema hakati, kas tõesti oli 
siis nii kättesaamatu või lihtsalt oli äge seda teha.  

\question{Jah, kui ma sind kuulan, see ei kõla suurepärase ärina}

Ektaco tegi ju  äri ka. Ja ma pean tunnistama ausalt, et  mind huvitas tollal 
programmeerimine. See, et mida  kolleegid nagu tegid, ma teadsin, aga ma väga 
ei süvenenud sellesse. See oli hästi selline fokusseeritud toimetamine.

\question{Kas tol ajal tekkis mingi kokkupuude arvutisidega ka juba?}

Seal Ektacos oli mul terve hulk toredaid kolleege. Olid 
Tarvi\index[ppl]{Martens, Tarvi}, Heiki Kask\index[ppl]{Kask, Heiki}, Jaak 
Niit\index[ppl]{Niit, Jaak}, Gunnar Valge\index[ppl]{Valge, Gunnar} oli seal 
minuga samas toas, kindlasti oli veel paar-kolm inimest. Ja siis meil oli 
Fido\index{FidoNet} \emph{point}, mis siis tekkis jälle seal Tarvi ja Heiki 
initsiatiivil, minu meelest ennekõike. Me olime alguses Lõvi point. 
Lõvi\index[ppl]{Lõvi|see{Lepp, Andres}}\sidenote{Lõvi, pärisnimega Andres 
Lepp\index[ppl]{Lepp, Andres}, on legendaarne TPI arvuti-mees, paljude meie 
põlvkonna inimeste sõber, teejuht ja eeskuju} oli siis TPI 
Arvutuskeskuses\index{Tallinna Tehnikaülikool!Arvutuskeskus}. Minu jaoks oli ta 
kunn, ma ei tea, mis ta seal tegelikult oli ja siis olime seal Lõvi 
\emph{point}. Jooksutasime seal FrontDoori\index{FrontDoor}\sidenote{FrontDoor 
oli üks populaarsemaid FidoNeti mailereid} ja mida iganes me jooksutasime. 

Ma arvan, et mingil hetkel me \emph{point}i staatusest \emph{upgrade}sime 
ennast \emph{node}ks. 71 oli meie number, julgeks arvata. Ja me helistasime 
kuhugi sisse ka, sest ma mäletan, et ma olen mingisuguse \emph{prompt}i otsas 
rippunud. Ja vaat seda jälle ei tea, et kust ma sain teada, mis käskudega seal 
Unixis\index{Unix} midagi teha. Ja kuidas mingi binaarne fail ära 
\emph{uuencode}da, selleks et ma saaks seda üle terminali endale 
\emph{dump}ida, selle \emph{dump}i salvestada, oma masinast \emph{decode}da ja 
mingit zipi sealt seest kätte saada. Kuidagi ma teadsin seda, kuidagi ma 
mingisuguseid asju imesin. Aga see on jälle niimoodi, et mingid asjad olid nagu 
õhus nagu mingisugused hallitusseene eosed laiali. Nii, kui kusagil pinnase 
sai, kohe läks kasvama. 

\question{Nii mitu sammu selleks, et midagi kätte saada, barjäärid olid jube 
kõrged toona.}

Info ikkagi liikus, see, ma arvan, ei olnud probleem. Küsimus oli ikkagi 
ennekõike riistvaras ja \emph{access}is ja  telefoniliinides ja niisuguses 
kraamis. Modemid olid ju roppkallid asjad, eks ole. Arvutid,kõik oli roppkallis 
välja arvatud aeg. Töö juures õnneks meil mingeid modemid olid, mitte küll 
kõige härjemad. Meil oli mingi 2400 ja MNP5\sidenote{\emph{Microcom Network 
Protocols (MNP)} on perekond (tähistatud numbritega ühest kümneni) 
veaparandusprotokolle, mida sageli kasutati varastes kiiretes (2400 bit/s ja 
rohkem) modemites} oli see meie lagi, millega me seal alguses toimetasime siis. 
Aga kõik olulised asjad liikusid ikka flopide peal, seda ei viitsinud keegi 
ära tõmmata, tõmmati mingeid pisikesi nublakaid. Tollal oli flopiga bussi peale 
minek reaalselt kiirem kui modemiga toimetamine.

\question{Mis sorti materjali te oma nodes hoidsite?}

Point oli meil puhas Fido point. Meil minu meelest küll BBSi ega midagi olnud. 
Meil oli ikkagi sõnumivahetus, \emph{Echomail} ja \emph{Netmail}, ehk siis 
privaatkirjad ja niisugused avalikud foorumid. See oli see, miks me nii-öelda 
suures pildis seda \emph{node}i pidasime. Kui keegi midagi tõmbas, siis ta 
tõmbas enda jaoks ja võib olla jagas  kolleegidega kuidagi midagi aga meil 
mingit sihukest varamut või niisugust ei olnud.

\question{Kellega te neid meile vahetasite, mis uudisgruppe lugesite? Kogukond 
ei olnud ju suur? Lõviga sai ju niisama ka juttu rääkida, ei pidanud kirja 
saatma?}

Mina lugesin põhiliselt \emph{Echomail}i, mul mingisuguseid kirjasõpru, kellega 
mingeid asju seal väga oleks olnud ajada, et tegelikult väga ei ei olnud. Minu 
jaoks oli see lihtsalt nagu foorum, kus sa saad huvitavat ja enamasti ka väga 
humoorikat  sisu. See väljendustase, see, kuidas inimesed, ükskõik mis teemal, 
viitsisid oma mõtteid sõnastada, need iroonia, sarkasm, huumor, kõik need 
tasemed, see oli niivõrd hea tekst valdavas osas, et seda oli  alati lust 
lugeda. Ükskõik mis oli, mingid autofoorumid, mul polnud  sooja ega külma 
nendest autodest. Aga lihtsalt need naljad, need vihjed, see oli lihtsalt hea 
meelelahutus, enamuses. Muidugi seal on ikka programmeerimised ja riistvara ja 
kõik muud teemad ka. See oli kasulik ja naljakas.

\question{No aga skaalal Tolkienist üle autode C++-ni?}

No kõike, absoluutselt. Kogu elu oli seal minu meelest. Seda jaksas tervikuna 
läbi lugeda sest inimesi oli vähe, palju sa ikka seda head kvaliteetset sisu 
suudad toota. Seda  oli vähe tegelikult, mis seal liikus minu meelest.

\question{Ühesõnaga, praeguses mõistes oli võimalik kogu sisuloomel silm peal?}

No sellel, mis Fido \emph{Echomaili} kaudu tuli, jah. Seal kuskil paralleelselt 
hakkasid arenema mingeid \emph{newsgroupid}, ka Eesti omad, millega mina 
alguses eriti ei puutunud  kokku. See oli natukene teine seltskond minu 
meelest, kes seal nii-öelda internetimaailmas hakkas toimetama. 

\question{Need olid kaks eri maailma, nende vahel mingit silda ei olnud?}

Nii ja naa, kontseptsiooni mõttes olid interneti uudisgrupid ja Fido omad 
samad, aga seal olid mingid ebamugavad erisused. Kunagi  hiljem, kui ma 
Ektacost Küberneetika Instituuti läksin\index{Küber|see{Küberneetika 
Instituut}}\index{Küberneetika Instituut|see{Cybernetica}} siis ma tegin oma 
\emph{node} Solarise\index{OS!Solaris} peale. Meil oli seal üks 
SPARC\index{Arvutid!SPARC}\sidenote{\emph{Scalable Processor Architecture 
(SPARC)} on Sun Microsystems'i poolt arendatud RISC-arhitektuur. Sun müüs 
sellele arhitektuurile tuginevaid, siinmail populaarseid, servereid ja 
tööjaamu} server ja siis ma ajasin seal peal käima kogu selle Fido softi. Üks 
venelane oli selle kirjutanud. Ja siis ma tegin \emph{news}i \emph{gateway}, 
mis nagu Fido Echomaili \emph{newsgroup}ideks köitis kahesuunaliselt ja siis 
ühtlasi ka Netmaili siis tavaliseks meiliks köitis. See oli päris popp, ma 
isegi ei mäleta, millal see maha sai võetud. Ma arvan, et seal juhtus see asi, 
et sellele Solarisele oli lõpuks vaja  korralik \emph{upgrade} teha ja siis ma 
ei viitsinud vist enam. Fido oli ära surnud selleks hetkeks ja siis ma tõmbasin 
ta maha. Aga mingil ajal  oli ta hästi popp, mul oli seal, ma arvan, ikkagi 
sadu sadu kliente oli oma personaalse \emph{account}iga seal minu \emph{news}i 
serveri küljes, kellel oli siis nii-öelda kirjutamisõigus Fido gruppidesse. 
Fidos oli see korrapidamine nagu olulisem, seal ei olnud sellist anonüümset 
kasutust, keegi vastutas alati kellegi eest. Keegi  kuskilt kaudu sai 
\emph{access}i ja kui see keegi oli nõme, siis see \emph{access} võeti talt 
ära. Kui ma hakkasin seda asja Newsi \emph{gate}ma, siis ma lubasin sedasama 
teha, eks ole. Ma ei andnud kellelegi Fido gruppidele  kirjutamisõigust, kui ma 
ei teadnud, kes ta on ja ma ei saanud seda \emph{access}i talt ära võtta. 

\question{Aga see on ju, ütleks, autoritaarne?}

See toimis. See oli nagu  endale olulise keskkonna  normaalsena hoidmise 
eeldus. Teistmoodi ei saa. 

\question{Aga mis on \enquote{nõme}?}

No, solvad teisi inimesi, trollid, ütled puhasti, eks ole. See ongi põhiline, 
et kui sa lähed isiklikuks, teed teisele haiget, halba emotsiooni, sihukest 
asja ei ole vaja. See kui sa vaidled, see on okei, seda peab olema, see ongi 
tähtis, eks ole. Aga sa ei tohi  teistele haiget teha. 

\question{Kõlab nagu lihtne, eluterve ja samas fundamentaalne definitsioon. Aga 
kui sa Ektacost ära tulid, kas sa veel õppisid?}

Ei, mu õppimised olid selleks hetkeks õpitud või noh, mitte päris lõpuni 
õpitud, aga ma olin oma inseneridiplomi kätte saanud, vist 1993 või 1992. Sain 
oma kraadiselt kätte. Magistrikraadiks \emph{upgrade}sin ma ta siin natuke 
hiljem. Mina õppisin, viis aastat, sain süsteemiinseneri diplomi, aga pärast 
hakati  kogu seda kraadi värki järjest lahjendama, eks ole. Kui nüüd õppeaastad 
järjest lühenesid, siis \emph{by default} oli mul bakalaureus, aga siis ma 
pidin veel natuke juurde õppima ja tegema magistritöö, ma saaks magistriks. See 
oli kunagi seal 2001. aastal umbes, kui ma selle  ette võtsin. 

\question{Aga tol hetkel sul ei olnud sellist tunnet, nutikas ja usin õppur 
nagu sa olid, et peaks teadusmaailma sukelduma?}

Ega ma seal TPIs ise teadusmaailma suurt kokku ei puutunud. Kuna ma sealt poole 
pealt hakkasin programmeerijana tööd tegema, eks ole, siis see  haaras  
enam-vähem täielikult. Ma arvan, et lõpus läks kas see õppimine natukene 
nigelamaks, sest töö juures oli palju huvitavam ja palju nagu väljakutseid 
pakkuvam. Viimased asjad mis seal Ektacos\index{Ektaco} sai tehtud, oli 
kontrollerite uue sideprotokolli disainimine. Ma olin hullult vaimustatud 
TCP/IPst ja  siis ma trükkisin välja kõik standardid, mis ma sain: TCP, IP, 
Etherneti. Aga kontrollerid on mingi 8051 peal, mis on umbes nagu, nagu väga 
väike. Aga siis ma lugesin need RFCd kõik läbi ja siis ma tegin mingi oma 
sideprotokolli, mis inspireerus siis kõigest: Ethernetist, IPst ja TCPst. Ehkki 
ta ei olnud nagu päris \emph{flow}le orienteeritud, aga pigem  selline 
\emph{datagram}i-põhine protokoll. Sihukesed vanad riistvaraässad Ektacos  olid 
väga nördinud ja solvunud, et mis mõttes ma kirjutan protokolli, mis ei ole 
deterministlik. Mitte \emph{master-slave}, vaid igaüks võib traadi peal 
lobiseda, kui mõte pähe tuleb, ja siis lahendatakse konfliktid ära ja tehakse 
re-transmissioon. Nad olid väga pahased minu katsetuste peale, aga ma arvan, et 
programmeerisin selle lõpuks sinna ära ja ta mingil määral töötas ka. See oli 
päris äge.

\question{Aga mille vahel see protokoll siis käis?}

Põhimõtteliselt oli see, et PC, mis siis juhtis neid tööstusarvuteid. Neil oli 
sihuke karp, mille nimi oli satelliit, mis oli siis tööstuskontroller, millel 
olid igasugused digi- ja analoogsisendid-väljundid, mis kuskil tehases keerasid 
mingit nuppu, et betooni teha või midagi. Ja siis sellel olid mingid
juht-programmid ja neid tuli konfida. Tüüpiline värk, eks ole. Sa pead teadma, mis 
sul tehases toimub. Sa pead käske andma, selleks on mingit võrku vaja. Ja neid 
satelliidikontrollereid võis seal korralikus tehases ikka palju olla. Ja siis 
ta tuli PCsse kokku tõmmata ja ma usun, et keegi kirjutas siis mingit softi 
sinna PC poolele, mis siis neid satelliite siis jälgis ja juhtis.

\question{See kupatus oli päriselt \emph{production}is ja Eesti Vabariigis 
tehti betooni niisuguste seadmetega?}

Jaa. Ma arvan, Palivere Ehitusmaterjalide Tehas\index{Palivere 
Ehitusmaterjalide Tehas} vist oli see, mis oli ära automatiseeritud Ektaco 
poolt ja ma millegipärast arvan, et midagi oli Tallinna 
Veepuhastusjaamas\index{Tallinna Veepuhastusjaam}.Aga seda ma väga kindlalt ei 
tea, aga seal oli neid veel. Neid objekte ikka oli.

Mida mina tegin, see oli järgmine generatsioon, need objektid juba töötasid 
mingisuguse muu protokolli ja mingi muu  tehnika peale, aga kõik see kasvas, 
eks ole. Ja siis algatati uue generatsiooni satelliidi väljatöötamise projekt, 
kus mina siis  protokolli kontributeerisin ja realiseerisin. 

\question{Kui sa võrgundusest juba nii palju teadsid, sind kuhugi interneti 
varasesse maailma ei tõmmatud kaableid vedama või midagi?}

Ei, mulle meeldis programmeerida. Nende muude asjadega ma tegelesin nii palju, 
kui nad olid kasulikud ja vajalikud selleks, et saaks midagi ägedat 
programmeerida. 


\question{Ja Küberis\index{Küber} sai ägedamalt programmeerida?}

Lõpuks jah. Jälle Tarvi\index[ppl]{Martens, Tarvi} kutsus mind sinna. 
Küberneetikasse oli tehtud infotehnoloogia osakond, mis peitis seda infot, et 
tegelikult tegeldi seal infoturbega ja siis oli seal mingi riiklik programm, 
mille eesmärk oli Eesti riigi infoturbe ja krüptograafia vajadusi rahuldada. 

\question{See oli juba enne, kui tekkis AS Cybernetica?}

Jaa, see oli enne seda. Mina läksin sinna 1994, aga see töögrupp tehti 1993, ma 
arvan. Ja siis seal oli terve hulk nutikaid inimesi koos, kes siis  selle 
missiooni elluviimisega tegelesid, et  kompetentsikeskust ehitada.

\question{Kes selle taga oli? Keegi pidi ju selle tellimuse formuleerima, et 
riiklikult on tarvis tegeleda krüpto ja infoturbega?}
Ülo Jaaksoo\index[ppl]{Jaaksoo, Ülo} oli siis Küberneetika 
Instituudi\index{Küberneetika Instituut} direktor. Minu vaates oli tema see, 
kes seda kõike lõi ja korraldas. Kuidas ja  kellega tema läbi rääkis või kust 
see mandaat tuli, seda mina ei oska küll öelda. Aga tema oli jah, kellel see 
visioon  oli, et seda on tarvis. 

\question{Arvestades, kui vähe vajas Eesti riik krüptot ja infoturvet praegu ja 
kui strateegiliselt oluline teema see praegu on, siis sellise visiooni jaoks on 
ju tarvis väga ägedat ettenägemisvõimet?}

No aga kaugemale vaatamine ongi teadlaste ja akadeemikute ülesanne. Kust mujalt 
see tulla saab? 

\question{Visioon visiooniks, mida see töö toona praktiliselt tähendas?}

Esiteks, ise õppida. Teiseks, teisi õpetada. Eestikeelne terminoloogia, 
standardid, profiilid, seminarid, koolitused mida iganes.  Ja  teistpidi 
hakkasid niisugused praktilised asjad tulema. Vaata, tollel ajal maailm oli 
nagu väiksem, ka krüpto ja infoturbemaailm oli väiksem ja mingil hetkel on 
ikkagi veel võimalik hoomata  kõike, mis oli oluline. Mitte küll päris üksi, 
aga sihukese väikese töögrupi sees nagu meil oli. Ma arvan, et mis meil  väga 
hästi läks, oli see, et meil olid inimesed, kes  tegelikult  huvitusid just  
sellest infoturbe süsteemsest poolest. Et mitte see, et mis on nagu see 
tehnika. Aga mis on see organisatsioon, need inimesed, need reeglid, eks ole, 
seadusandlus seal ümber. Ühesõnaga süsteemne lähenemine valdkonnale kui 
tervikule, mis on  väga tähtis ja  mis sellest meie grupist välja kasvas. 

Teiselt poolt oli see, et meil on seal sihukesed \emph{hardcore} häkkerid ja 
\emph{hardcore} krüptograafid, kes nagu olid valmis mida iganes tegema. See 
sümbioos oli minu meelest hästi lahe. Ma arvan, et minu esimene töö 
Küberneetika Instituudis oli see, et ma pidingi riigiasutustele kirjutama 
juhendi, kuidas KA9Q\index{KA9Q} otsas ehitada endale internetti ruuter. 

\question{Mille otsas?}

KA9Q on üks soft. \enquote{KA9Q} on mingi radistide kutsung, mis vastab mingile 
inimesele, kes selle softi kirjutas, on minu arusaamine. Ja see oli DOSi peal 
jooksev \emph{all singing all dancing} asi, mis realiseeris TCP, kõikvõimalikud 
sideprotokollid, võrgukaartide toed, SLIP, PPP, ruuterid, mida iganes. FTP 
deemonid. Täiesti müstilisi asju on tehtud maailmas.  Et kui sul oli üks  
üleliigne PC, modem, võrgukaart ja see soft, siis sa said teha endale ruuteri, 
millega oma organisatsioon kuhugi ära ühendada. Ja siis mina peksin selle käima 
ja kirjutasin eestikeelse lühijuhendi, kuidas seda asja  pruukida, hooldada ja 
nii-öelda käimas hoida. See oli mu esimene nii-öelda, ma ei tea, praktikandi 
töö või mis iganes töö seal Küberis. Aga siis hakkasid igasugused muud asjad
tulema.

Me olime mingis hästi varajases europrojektis, ma mäletan, see võis olla 1995. 
aastal. Ma tean, et ma käisin Darmstadtis\index{Darmstadt}. Sakslased olid 
kirjutanud sellise tarkvara nagu secu-d, mis oli,  ma ei kujuta ette, et ma 
pakun mingi kümme mega haljast C koodi väga halvasti kirjutatud, mis  
realiseeris kogu krüpto, mis tolleks hetkeks oli teada. Kõik sertide töötlus, 
särk-värk. Ja siis me üritasime seda secu-d'd kuidagi rakendada ja kuidagi 
käima peksta. Ütleme niimoodi, et selline \emph{cross-platform} arendus tollal, 
et sul on kood, mida sa kompileerid mingi UNIXi jaoks ja mingi PC jaoks ja siis 
tulid Windowsid, eks ole. Ja teha nii, et see kuidagi enam-vähem  töötab ja 
piisavalt vähe mälu lekib ja piisavalt harva sama mäluplokki kaks korda 
vabastab on  raske ülesanne. Ja siis ma selle secu-d najal ehitasin 
mingisuguseid asju. Turvalist meiliklienti näiteks ja sertifitseerimiskeskust. 
Sertifitseerimiskeskused olid lahedad,  seal mingisugusel  ajaperioodil oli 
see, et me seal Küberneetika Instituudis iga aasta programmeerime vähemalt ühe 
sertifitseerimiskeskus valmis softi mõttes.

\question{Miks?}

See oli mingisugune \emph{blend} sellistest praktilistest vajadustest ja 
teadustöö eesmärkidest. Et üks  sertifitseerimiskeskus, mille me näiteks 
programmeerimine oli näiteks selline. Tollal ei olnud ju mingeid kiipkaarte ja 
riistvaralisi turvamooduleid kätte saada. Ja see oht, et kui sul 
sertifitseerimiskeskuse võti ära 
 kompromiteerub, et siis keegi annab võltssertifikaate välja, see oli suur. Või 
et keegi annab sellele operaatorile altkäemaksu, et annaks võltssertifikaadi 
välja. Sul oleks vaja mitmesilma printsiipi ja sihukest  topeltkaitset. Ja siis 
me realiseerisime selle, et me võitsime selle RSA võtme tükkideks. See on 
seesama, mida praegu SplitKey\index{SplitKey} ja SmartID\index{SmartID} teevad. 
Meil ei olnud küll seda turvalist mitmes osas võtme genereerimist, me lihtsalt 
RSA võtme, jagasime ta osakuteks ja siis meil oli sihuke m-n-ist skeem. 
Selleks, et sertifikaati välja anda, siis viiest operaatorist kolm pidid  
allkirja andma ja siis me kombineeris neist korrektse sertifikaadi kokku. Selle 
nii-öelda initsialiseerimisprotsessi käigus tekitati viis flopit,  millega need 
 operaatorid ringi oleks pidanud käima. Selles mõttes oli ta praktiline, et ta 
töötas,  tegi täitsa korrektseid X.509  sertifikaate ja oli kasutajajuhendiga 
varustatud.  
 
\question{Tundub, et kui sa enne seal ISA siini peal tegelesid väga madala 
taseme asjade katsetamise ja läbi mängimisega, siis nüüd sa tegid sedasama 
krüpto jaoks põhiolemuses olulisi primitiive ja protsesse läbi realiseerides?} 

Jah, et seda võib öelda küll, et mingis mõttes me tegelesime selliste hästi 
\emph{basic} asjadega. Me jõudsime ka rakendusteni välja. Meil oli ka 
hästi-hästi praktilisi asju, aga me kontrollisime tegelikult kogu seda pinu 
ülevalt alla välja. Et sellel ühel hetkel me tegime tulemüüre, mis oli väga 
hästi müüv toode Eesti turul, Barrikaad\index{Barrikaad} oli selle nimi, mul 
siiamaani barrikaadi T-särk alles. Siis me tegime VPN toote, mis oli veel 
ägedam. Selle VPNi teine versioon oli igasugustes Eesti riigiasutustes 
väga-väga pikalt kasutusel ka peale seda, kui selle tugi ametlikult õnnetuseks 
ära lõppes. Ja selle põhieelis oli see, et ta oli projekteeritud hästi 
turvaliseks, keskelt administreeritavaks, eriti töökindlaks. Ehk et see, et sul 
on  harukontorid, kust sa ei taha üldse interneti väljapääsu, vaid tahad läbi 
keskse tulemüüri (mis oli kallis) neid välja juhtida, see oli meil sinna sisse 
ehitatud. Igasugused paralleelsed ruutingud üle erinevate kanalite, eks ole. 
Seal tekivad probleemid, kui sul on VPN tunnel, sul on  sisemised aadressid, 
välimised aadressid, kuidas sa neid majandad niimoodi, et see ruutingu info ka 
seal sisevõrgus korrektselt leviks ja tegelikult ka töötaks. Et kasutajad 
ei peaks  ootama, kuni nende seanss katkisest kanalist tervesse kolib, eks ole, 
et see lihtsalt töötakski. Ja kogu see administreerimine. Meil oli tehtud see 
tükk, mis võimaldas süsteemi konfiguratsiooni muuta, see oli eraldi, see võis 
offlainis olla, see suhtles  muu maailmaga floppide kaudu, see ei olnud võrgus. 
Ja siis oli meil võrgus olev tükk, mis ainult monitooris, kogu sealt infot ja 
täitis neid käske, mis võrgust väljas olev tükk talle  ette pani. Niisugune 
eriti kõrgete turvanõuete jaoks tehtud haldussüsteem. Ja, ja seal me muuhulgas 
siis, kuna tollal ikkagi see PC krüpteerimisvõime oli nõrk, siis me 
realiseerisime ise  šifreid. Tollal just MMXi laiendused tulid prosele välja, 
mis võimalused sul näiteks IDEAt\sidenote{\emph{International Data Encryption 
Algorithm (IDEA)} on esmakordselt 1991. aastal kirjeldatud sümmeetriliste 
võtmetega plokkšiffer} paralleelselt arvutada, mitu plokki korraga. Ja siis 
Helger Lipmaa\index[ppl]{Lipmaa, Helger} oli veel Küberis tööl, kes 
programmeeris siis Linuxi tuuma jaoks MMXi \emph{extension}eid  kasutava 
AESi\sidenote{\emph{Advanced Encryption Standard (AES)} on Belgia 
krüptograafide poolt välja töötatud Rijndael plokkšifri alamhulk. 1997. aastal 
teatas NIST (\emph{National Institute of Standards and Technology of the United 
States (NIST)}) plaanist asendada avaliku protsessi abil tolleks ajaks 
ohtlikult nõrgenenud DES algoritm. Vincent Rijmen ja Joan Daemen esitasid oma 
ettepaneku valikuprotsessi ja see standardiseeriti NISTi poolt 2001. aastal}  
realisatsiooni. Meil seal Linuxi\index{OS!Linux} tuumas olid oma draiverid, mis 
seda VPNi asja haldasid, seal peal olid  oma deemonid võtmete vahetuseks, konfi 
levituseks, kõigeks muuks  ja siis niimoodi hierarhiliselt üles välja.

\question{See, mis sa räägid, et see ei kõla enam nagu programmeerimine, see 
kõlab nagu arhitekti töö. Kas sa liikusid programmeerija rollist arhitekti 
rolli või mõtlesite te neid asju kambakesi välja, kuidas see käis teil?}

Selles mõttes, et välja mõtlesin kogu aeg lihtsalt enamasti oli see teine tüüp, 
kes asju realiseeris,  sellesama peakolu sees. Lihtsalt seal tulid inimesed 
nagu appi. Meil ei olnud  väga selgelt nagu defineeritud rolle, eriti alguses, 
eks ole. Arhitekt, projektijuht, projektijuhid olid üldse väga haruldased 
nähtused, Me ei teadnud isegi, mis projekt on, me lihtsalt programmeerisime 
mingi hetkeni. Meil oli seal, jah, ikkagi terve hulk inimesi, kes arutasid 
intensiivselt praktiliselt kõigil teemadel. Kui asjad olid selged ja siis 
igaüks natukene läks oma  valdkonnas  süvitsi sellega.

\question{Nutikatel inimestel on vahel oma nutikusele vastav ego ka, keegi nina 
püsti ei ajanud ja ennast arhitektiks ei kuulutanud?}

Ei, päris nii ei olnud. Aga ma ise kardan tagantjärgi võib-olla mina ise 
kippusingi see tüüp olema, kes oma  arvamust teistele peale surus. Aga ma tol 
hetkel ei tajunud seda kindlasti niimoodi. 

\question{Ma arvan, et ega teised ka ei tajunud ja soft ju lõpuks ikkagi 
töötas ju}

Absoluutselt. Nii see tulemüür kui ka see VPN, olid meil ikkagi lõpuks ikkagi 
ääretult stabiilsed ja, ma ütleks, kvaliteetset tükid. 

\question{Privador kasvas ka ju sealt välja?}
Jah, Privador\index{Privador} oli siis Küberneetika Aktsiaseltsi spin-off 
firma, mis siis sai need nii-öelda infoturbetooteid, eesmärgiga need laia 
maailma viia, aga see kahjuks ei õnnestunud. Seal oli  kindlasti ports 
probleeme ja üks probleem, mida mina nägin oli see, et tollal hakkasid tekkima 
standardid, et mis asi on VPN, mis asi on standardne VPN. Ja IPSec oli 
enam-vähem ära standardiseeritud, IKE oli ära standardiseeritud ja see oli 
tegelikult see, mida oleks tahetud osta. \emph{Vendor lockin}i juba päris 
mõõdukalt kuni palju kardeti. Ja ehkki meie olime oma asja ehitanud, eriti need 
alumised kihid, need olid  standardite põhjal ehitatud aga mudel,  kuidas me 
nägime seda võrgu tervikut ette ja mida me pidime tegema, selleks, et neid häid 
omadusi saada,  seal tekkisid konfliktid IKE või ütleme, IPSeci, ideoloogiaga 
natukene. Meil  tegelikult oli töölaua peal  versioon kolm VPNist, mis oleks 
siis olnud täiesti standarditega ühilduv, mis loodetavasti selle  firmapärasuse 
probleemi oleks ära kõrvaldanud, aga see kahjuks ei läinud realiseerimisele. 
Selle asemel me tegime digiallkirja tarkvara ja ajatembeldustarkvara ja 
Notariseerimistarkvara ja kõike muud. Me nagu natuke ennustasime valesti, et 
mis on see \emph{killer} rakendus krüptomaailmas järgmise kümne aasta jooksul. 
Olime nagu natuke ajast seest selles mõttes.

\question{See lähenemine, et võtame alumise kihi standardid ja paneme nad 
kuidagi täitsa uut moodi ülemise kihi standarditeks kokku on ju seesama, mis 
sai digiallkirja konteineriga tehtud ja X-Teega ka}

Absoluutselt. Aga vaat seal ongi see, et standardid on ja peavadki olema 
tegelikult geneerilised, eks ole. Nad peavad olema sellised, et nad lahendavad 
paljude inimeste paljusid probleeme, siis nad on elujõulised. Nii. Aga aga kui 
sa võtad ühe konkreetse riigiasutuse, kellel on konkreetsed vajadused, mis ta 
peab ära lahendama efektiivsel viisil, siis sa ei pääse lihtsalt sellega, et sa 
võtad standarditele vastavat tüki ja evitad selle. See ei ole efektiivne. Ja 
see oli siis see, mida meie tegime. Aga seal oligi vaata natukene see, et me 
võib-olla ei tajunud seda, et kui suur see maailm on ja kui võimas ta on ja kui 
suure massiga ja kui kiiresti ta liigub. Me mõtlesime, et me teeme ikka rajult 
ägeda asja. Ja noh, see on nagu \emph{way}  parem ja praktilisem väga suure 
hulga klientide jaoks. Aga see teadmine, et miski asi on hea ja praktiline,  
seda on väga raske efektiivselt ja kiiresti ühest peast teise viia.  

\question{Arvestades, et samast pundist tulid ju ka X-Tee\index{X-Tee} ja 
ID-kaardi kontseptsioon, siis kahest kolm ei ole üldse mitte paha edu protsent}

X-Teega on muidugi see, et X-Tee omab selles meie VPNi tootes väga selgeid 
juuri. Tegelikult, kui me seda X-Teed tegime, see oli 2001.  Mais või juunis 
hakkas asi pihta või isegi natuke hiljem ja detsembris läks tootesse. Eks ole. 
See oli võimalik ainult tänu sellele, et me võtsime oma selle VPN toote kui 
substraadi. Meil oli kõik see olemas, et kuidas me teeme ühe Linuxi purgi 
turvaliseks, kuidas me sellele  Linuxi purgile paneme peale oma tarkvara 
\emph{patch}id, särgid-värgid, kuidas me seda Linuxit konfime, kuidas me hoiame 
konfi niimoodi, et see on efektiivne, kuidas konfi jagamine käib, see kõik oli 
olemas. Me lihtsalt selle asja peale ehitasime ühe natukene teistsuguse 
protokolli vahenduse tüki, eks ole. 

\question{Aga see kõik on natuke hilisem lugu. Kui mina sinuga esimest korda 
kliendina kohtusin, siis sa ikkagi juba juhtisid vägesid. Mina rääkisin oma 
mure ära ja sina tegid nii, et asjad sündisid. Kuidas sul inimeste juhtimine 
rollina esile kerkis ja kas sa üldse mõtestad seda tegevust niimoodi?}

See tekkis Barrikaadi\index{Barrikaad} või VPNi või Privadori\index{Privador} 
programmeerimise käigus, kui meeskond läks suuremaks. Eriti selle VPNi juures, 
ma arvan,  koordineeriv funktsioon oli ikkagi minu peale, et kes nüüd mida 
programmeerib, eks ole, mis ajaks. Ja kes neid asju evitamas käis, ikka meie 
ise, sealt tuli ka see klientidega suhtlus, eks ole. \emph{Helpdesk}, 
projektijuht, arhitekt, programmeerija, testija, tarneinsener, et mu roll oli 
natukene nagu kõik koos. 

\question{Aga ometi kuidagi jäi see koordineeriv roll just sinu peale?}

No ju siis selles pundis see  kõige paremini  mulle sobis, ei oska muud midagi 
arvata. Keegi pidi selle ära tegema, eks ole. Kui see olin mina, siis olin see 
mina, nii see läks.

\question{Ma selle pärast küsin, et ega sul mingisugust kihu ei olnud inimesi 
juhtida?}

Ei. See pigem oligi sedapidi, et, ma nägin seda, mis see asi võiks olla, mida 
me teeme,  päris detailselt päris paljudes aspektides. Ja siis ma nagu tahtsin, 
et see nii läheks, siis ma olin sunnitud  inimestele  ülesandeid või siis 
eesmärke püstitama. See tuli pigem sedapidi, et üksinda ei jaksa kõik ära 
progeda.

\question{Aga see on jällegi arhitekti vaatenurk. Minu peas on olemas täiuslik 
mudel süsteemist ja siis ma teen niimoodi, et see saaks teoks tehtud. Mis sa 
praegu teed?}

Mis sa praegu teed? Väga paljusid erinevaid asju. Ma suhtlen hästi palju 
klientidega ja potentsiaalsete klientidega, et aru saada, mis on  nende  mured 
ja vajadused, kuidas me saame   neid aidata. See on alates müügitööst, projekti 
juhtimiseni. Teistpidi ikkagi see, ütleme, arhitektuurne töö. Kui probleem on  
arusaadav, et mis oleks see lahendus. Ja need probleemid on keerulisemaks ja 
mastaapsemaks läinud. Mõnes mõttes ka vastutusrikkamaks selles mõttes, et me  
ikkagi tegutseme suuresti turvavaldkonnas. Ja see keskkond on nii palju 
vaenulikum ja nii palju keerulisem ja need panused on nii palju suuremad, et sa 
pead lihtsalt palju palju paremaid asju tegema kui me kunagi tegime. Sedasorti 
arhitektuurne  mõtlemine ja siis inimestele nende ideede jagamine. Nõustamine, 
mõnes mõttes ka võiks öelda isegi natukene koolitamise moodi asjad. 

\question{Sa oled kogenud arhitekt ja tead, mida on vaja selleks, et projekt 
välja tuleks. Kuidas sa viid entusiastlikult pihta hakanud meeskonnale kohale 
selle, et sinu arvates projekt ei saa välja tulla? Ja seda nii, et sind pärast 
tuppa tagasi ka lastakse?}

Samm üks on see, et sa pead aru saama. See võtab tegelikult päris kaua aega ja 
see on nagu see koht, kus tihti suhtled väga vähe. Ega seda, et vaatad peale, 
saad kohe aru,  mis valesti on, kuidas peaks olema, seda ei ole. Kõigepealt 
pead probleemist aru saama. Ja võib olla, et  sellepärast see see olukord ongi 
võib-olla keeruline või halb,  et see ongi olemuslikult keeruline probleem. 
Seal on mingisugused mingisugused põhjused, keegi on teinud mingeid otsuseid, 
mingeid probleeme on lahendatud ja selle käigus on tekkinud niisugune asi. Sa 
pead sellest aru saama. Sa ei saa lihtsalt minna, et \emph{sorry}, vanad, et 
siin on jama. Sa pead kõigepealt aru saama, mida on tehtud ja miks on tehtud, 
need probleemid endale selgeks tegema. Ja siis sa tõenäoliselt marineerid nende 
otsas päris kaua ja see ei tule niimoodi, et hops, homme hommikuks on valmis, 
eks ole. Sa mõtled ja kirjutad ja räägid. Ja ehkki tihti on niimoodi, et  sulle 
endale võib tunduda, et lõpuks kui sa mingeid asju hakkad tegema, et selline 
lahendus oli algusest peale selge. Aga kui sa lähed kontrollima fakte, et mida 
sa tegelikult rääkisid, mida sa oled ise kirjutanud, mis sa arvasid, siis 
selgub, et tegelikult see lõplik lahendus on sinu juurde väga suure kaarega 
tulnud. Sa pead selle lihtsalt välja kannatama ja selle ära tegema. Aga, aga 
point on lõpuks see, et kui sa oled jõudnud mingisuguse asjani, millest sa 
näed, et see ongi okei ja lahendab ära  selle probleemi ja selle probleemi ja 
selle probleemi. Võib-olla see lahendus on keeruline ja on kulukas nagu 
realiseerida ja on isegi riskantne aga ta on õige, ta on juba olemuslikult 
õige. Sa saad aru, et mis see probleem olemuslikult on, kuidas seda asja  
tükeldada, kuidas seda keerukust peita, kuidas seda asja üldistada. Ja siis sa 
pead väga kannatlikult väga paljudele inimestele seletama, miks me võiks teha 
just nii. Seda jõuga ei saa teha. Sa pead neid julgustama ja sa pead olema 
valmis nende eest viskuma džotile, juhul kui on vaja. Aga ma ise muidugi usun, 
et ei lähe vaja, või siis sealt džotist ei tule midagi surmavat välja, eks ole.


\chapter{Ahti Heinla}
\index[ppl]{Heinla, Ahti}
\question{Kuidas sa sattusid arvutite juurde?}
Ma tulen sellisest perekonnast, et minu ema ja isa olid mõlemad 
programmeerijad. Nad olid  ülikooli lõpetanud ja  said tööl kokku ka, 
see oli kuskil kuuekümnendate lõpp. See oli  aeg, kus Eestisse tekkisid 
esimesed arvutid, mis sel ajal olid muidugi kapi suurused, aga ikkagi.

\question{Kuuekümnendate lõpus ei saanud neid programmeerijaid ju palju olla?}

Jah, kindlasti neid ei olnud palju, kuigi neid siiski ikkagi täiesti 
mingisugusel määral oli. Minul muide muide oli hiljuti selline asi, et  emal 
oli selline suur juubel, üle seitsmekümne ja niimoodi, ja ta kutsus enda 
kursusekaaslased külla. Ja kes need kursusekaaslased siis on, need on 
rakendusmatemaatikud, praegu siis sellised üle seitsmekümne aastaseid  
inimesed, nii mehed kui naised. Põhimõtteliselt kõik  
professionaalsed programmeerijad olnud, enam-vähem kõik, mehi ja naisi 
võrdselt. Ja, näed, meil on ikkagi asi juba nii kaugel, et meil on juba nagu 
suhteliselt kaugeid põlvkondi, kes on üles kasvanud programmeerijatena. Ja mina 
sündisin kahe sellise inimese järeltulijana.

\question{Kas see on pigem vedamine või vastupidi? Oleks võinud ju ka ära 
hirmutada?}

Mind see kindlasti ära ei hirmutanud, ma kasvasin üles perfolintide vahel. 
Vahetevahel, kuna arvutiaeg oli ju piltlikult öeldes talongidega 
jagatav, arvuti pidi ikka õhtuti töös olema, ema ja isa käisid 
õhtuti tööl, kui nad said arvuti aja kella kaheksaks õhtul. 
Ja siis nad võtsid vahepeal minu ja mu õe 
 kaasa, mina jooksin  arvutikappide vahel ringi ja vaatasin, kuidas seal 
magnetlindid vaikselt käisin nii ja naa ja see oli kindlasti hästi põnev. 
Hoopis teistsugune keskkond ja isegi helid on teistsugused. Vaatad, kuidas 
need masinad seal toimetavad, mingid magnettrumlid vaikselt vihisevad ja 
sahisevad ja kindlasti oli põnev.

\question{Legendid räägivad, et selle põlvkonna rahvas korraldas Ameerikamaal 
lindikappide võidujookse ja muud sellist, tolles sinu arvutiruumis midagi 
sellist ka toimus?}

Mina selliseid asju ei näinud. Ma saan aru, et ka Eestis  tehti selle sel ajal 
sellist  pulli, aga võib-olla  seda tegid natukene nooremad inimesed, kellel 
lapsi ei olnud. Minu isa ja ema olid ikka natuke sihukesed ontlikumad. Nad 
üritasid mingisuguseid konstruktiivseid asju arvutiga teha,  panna neid just 
ühel või teisel moel  käima, aga nad ei olnud sellised, kuidas öelda, häkkerid 
tänapäeva mõistes. Et  mismoodi arvutiga  pead pesta, näiteks, et selliseid 
asju nad ei mõelnud.

\question{Sind ju esialgu ei lastud linte perforeerima? Mis esimene asi oli, 
millele sa ise käed külge said?}

No mu vanemad olid programmeerijad aga mina ei olnud programmeerija, tavaline 
laps nagu ikka. Ma vist olin nagu natuke  matemaatiliselt  andekas, aga 
otseselt arvutitega minu  esimene kokkupuude oli tegelikult ikkagi sellest, kui 
ma olin kümne aastane. Lihtsalt järsku päevapealt  ühel õhtul tuli ema  koju ja 
ütles, et kuule, Ahti,  ma õpetan sulle midagi, istume maha. Istusime maha ja 
ta õpetas mind programmeerima. Kolm õhtut niimoodi õpetas. Ja ma sain selle 
kolme õhtuga tegelikult sellest oast aru, et mismoodi see asi käib. Sealt 
alates  siis hakkasin juba ise edasi mõtlema, proovima, katsetama, lugema, 
natukene lolle küsimusi küsima. Kolm päeva ma olen sellist süstemaatilist 
programmeerimise õpetust saanud.

\question{Mis ta siis rääkis, et see kümneaastasele huvitav oli?}

Eks mind huvitasid sellised asjad kindlasti. Sel ajal polnud ju ka niimoodi,  
aasta oli 1992, et  lapsel on tohutult palju mingisuguseid ahvatlusi 
ümberringi, et Facebookid ja Instagrammid hüppavad siia-sinna ja kõikvõimalikud 
muud asjad käivad. Sel ajal oli ikkagi niimoodi, et ega meil kodus ju telefoni 
näiteks ei olnud. Arvutit ka ei olnud, ma kirjutasin programmi  
alguses ikkagi paberi peale. Need kolm päeva õpet käis paberi 
peal. Ja kui  ema tuli õhtul koju ja sellist asja ütles, siis me ikkagi mitu 
tundi istusime maas, eks ole. Ei olnud niimoodi, et mul oleks kogu aeg telefon 
helisenud ja hüpanud, mingisugune asi, et \enquote{kuule, Ahti, tule nüüd sinna, teeme 
seda}. Selles mõttes võimalik, et ei olnudki nii väga vaja, et see oleks nagu 
hullult kuidagi põnev olnud kümneaastasele lapsele. Mul pigem oligi lõpuks  
põnev see, kui ma sain aru, kuidas see asi töötab.

\question{See peab olema päris korralik ettekujutusvõime, et sa paberi peale 
kirjutades saad aru, kuidas miski asi töötab. Sest paberil ei tööta sul midagi, 
seal on lihtsalt tekst}

Nojah, samas aga eks programmeerimise üks selline  võtmeoskus ongi tegelikult 
ju oskus ette kujutada, et mismoodi  masin  töötab. Lõppkokkuvõttes  
programmeerija ehitab ju masinat. Ja noh, piltlikult öeldes, ikka samasugust 
masinat nagu  mingisugused hammasrattad kuskil käiksid. Üks koodirida on 
piltlikult öeldes üks hammasratas, teine koodirida on mingi kangikene 
kuskil seal, eks ole. Ja kui sa ehitad sihukest füüsilist või mehaanilist 
masinat, siis sa näed, kuidas see töötab, et siin mingi ratas keerab 
ja siis kang liigub ja kuidas siis teine asi kuskilt midagi lükkab ja mingi 
lint või tross kuskilt midagi tõmbab. Sa näed seda kõike füüsiliselt. Ja 
programmeerija peab ka nägema. Aga ta peab nägema seda vaimusilmas, sest seda  
füüsilises maailmas  silmadega ei näe. Ja see vaimusilmas nägemise oskus on 
programmeerijale ülivajalik. Tagantjärele vaadates võib öelda, et eks ema mulle 
seda tegelikult õpetaski see kolm päeva.

\question{Kas \texttt{goto} käib nii- või naapidi või tehete järjekord on 
selline või teine, on teisejärguline.}

Just. Tegelikult oleks põhiliselt vaja teada, et mida \verb|goto| tegelikult teeb 
või et selles masinas, millise hammasratta, millise kujuga asja, see 
\verb|goto| seal teeb.

\question{See kolm päeva tekitas huvi, sa said enam-vähem aru, kuidas arvuti 
töötab, aga mis edasi sai?}

Siis läksime kuskil õhtul emaga  sinna arvuti juurde, ema tööle, ja  
tippisime selle programmi sisse. Kui ma õieti mäletan, siis seda sisse 
tippimist võis juba mitu päeva olla. See oli ikka mitu lehekülge, see minu 
programm ja mõnikord läks midagi valesti ka ja nii edasi. Ema aitas mul siis 
seal mõned vead ära parandada ja  tuli välja, et  tegelikult see programm 
töötas. See lahendas ühte väikest sihukest matemaatika  keerdülesannet, kus  
loogika oli  selles, et kui sul on  näiteks sada ühikut raha ja sa lähed 
raamatupoodi ja sa tahad seda sada ühikut raha ära kulutada. Siis sa pead 
kombineerima, et osta üks raamat, mis maksab viiskümmend seitse ja teine raamat 
nüüd maksab kolmkümmend, selline klassikaline  matemaatika keerdülesanne, 
kuidas kombineerida niimoodi, et kokku saada  summa, mis on võimalikult 
lähedane sajale aga mitte üle selles. Ja sellist ülesannet lahendas see minu 
programm. Ei ole nagu kõige triviaalsem asi, see ei ole nagu päris niimoodi, et 
vajutad nuppu ja programm ütleb lihtsalt \enquote{tere}. Tänapäeval ikkagi 
pigem kõik asjad üritatakse, ka heal põhjusel, ehitada niimoodi, et sul on 
selline nagu hästi kiire rahuldus või et sa nagu näed kaks minutit vaeva ja 
juba midagi hästi väikest nagu töötab ja siis sa näed veel viis minutit vaeva 
ja tuleb veel midagi. Mina pidin kolm päeva vaeva nägema, enne 
kui tulemust oli. Enne seda oli kõik ainult vaimusilmas.  Aga, tõepoolest, kui 
sul kogu aeg Facebook taskus ei hüppa, siis on nagu natuke lihtsam ka seda kolme 
päeva leida. 

\question{Mis tolle arvuti nimi oli?}

Ausalt öelda ma isegi ei mäleta, ei pruukinud isegi nõukogude masin olla,  seal 
oli tegelikult ka lääne aparaate.

Sedasama ühte programmi, mis ma kirjutasin, sai minu meelest 
 isegi  mitmel arvutil käitatud. Et see ei olnudki niimoodi, et 
\enquote{kuule Ahti see on nüüd sinu arvuti, millega nüüd sina  mitu päeva 
tegeled}. See isegi vist nägi niimoodi välja, et ma pool programmi 
tippisin ühel arvutil sisse, mis oli sihuke suur must kapp ja siis järgmisel 
õhtul läksime ühe hoopis  läänelikuma välimusega  nagu nõtkema 
välimusega moodsama asja taha ja tippisin teise osa sisse. Et ma juba sain ka 
natuke kogemusi sellest, et see programm on ikka hoopis midagi muud, see ei ole 
see füüsiline arvuti, millega ma tegelen. Ma võin istuda ühe arvuti taha ja 
siis ma võin minna teisele korrusele teise arvuti taha, mis on terve toa suurune 
ja seesama programm jookseb mõlemas.

\question{Mille peale sa vahepeal kirjutasid selle programmi? Kaartide peale?}

Siis olid ikka juba diskid olemas. Mitte need kolmetollised 
disketid, vaid sihukesed  kaheksa või viie tollised või mingid sellised asjad, 
pigem kaheksatollised ilmselt. Aga kindlasti see esimene programm oli ainult 
selline algus, eks ole, sellest tuli mingisugune  oskus ja huvi asja vastu. 
Edasi hakkasin ise vaatama ja  sattusin kokku juba teiste poistega, 
kes analoogse asja vastu huvi tundsid. Lähemate aastate jooksul hakkasid 
tekkima ka personaalarvutid ja enam ei olnud alati niimoodi, et sa pead õhtul 
tingimata ema töö juurde minema, vaid oli kuskil juba muid kohti ka olemas.

\question{Kust sellised tutvused tekkisid, internetti ju polnud?}

Internetti ei olnud, küll aga  oli olemas näiteks kaheksakümnendatel tekkinud 
selline asi, nagu Raaliklubi\index{Arvutiklubi!Raaliklubi}, mida vedas selline tegelane 
nagu Jaak Loonde\index[ppl]{Loonde, Jaak}. Mina olin ka vahepeal selle klubi liige ja see koondas sihukesi huvilisi poisikesi. Ega mul on raske öelda 
täpselt, ise nagu poisikesena tol ajal süstemaatiliselt ei pannud tähele ja ei 
jätnud meelde ka täpselt, mida täpselt Jaak Loonde tegi ja kas ta üldse midagi 
tegi peale selle, et lihtsalt need poisid kokku tuua. Aga täiesti võimalik, et 
sellest piisabki, et sama huviga poisid kokku tuua ja siis nad 
omavahel juba vahetavad kogemusi, kellel kuskil jälle ema või isa töötab 
kuskil. 

Minul oli näiteks üks niisugune reliikvia, mille ema mulle andis: ta õpetas 
mind kolm päeva ja siis ta andis mulle ühe ingliskeelse raamatu, mis oli 
põhimõtteliselt Pascali programmeerimiskeele õpik. See oli inglise keeles, ehk 
siis ma ei saanud sellest eriti midagi aru. Ma koolis õppisin saksa keelt, 
mitte inglise keelt\sidenote{Tol ajal jagunesid koolid kaheks: kas lisaks vene 
keelele õpetatakse läbivalt inglise või saksa keelt}. Küll aga ma sain aru 
nendest  programmi näidetest, mis seal oli, eks ole, ja seal oli asjad ikkagi 
mingisugused loogilises  järjekorras. Tegelikult, kuigi ma inglise keelt ei 
osanud, ma siiski suutsin sellest raamatust kindlasti midagi õppida ja sealt 
tuli ideid, mida katsetada. Seal oli kuskil mingi, piltlikult öeldes, mingi 
\verb|goto| käsk, oletame. Ma ei teadnud, mis see tähendas, aga ma sain 
selle \verb|goto| käsu  hiljem mingisse arvutisse sisse tippida ja 
vaadata, mis ta teeb ja küsida kelleltki teiselt, et mis see \verb|goto| tähendab. 
See on midagi muud kui lihtsalt öelda, et \enquote{õpeta mulle programmeerimist},  
sul on juba ka konkreetne küsimus. Niimoodi läbi nii-öelda lukuaugu see 
õppimine käis. Internetti ei olnud, aga  inimestel ei olnud ka internetti, siis 
kui nad  lennukeid ja autosid ehitasid, ja sai hakkama.

\question{Seal arvutiklubis sa käisid seepärast, et programmeerimine pakkus 
huvi?}

Jah, mulle pakkus see programmeerimise pool huvi. Mul tegelikult oli niimoodi, 
et isegi enne programmeerimist sattus kätte mingisugune lastele mõeldud 
elektroonika raamat ja ma  natukene nagu harjutasin või mõtlesin selle 
elektroonika peale ka, et kuidas näiteks transistorid töötavad ja muu selline. 
Ja see pakkus ka mulle kindlasti huvi. Aga tagantjärele vaadates  ma 
ütleks, et minu elektroonika tegemine sel ajal oli  ülialgelisel tasemel. Ma 
nii-öelda  kuidagi nagu hästi õrnalt natuke nagu kõditasin elektroonikat ja 
üritasin sellest midagi aru saada, aga samas programmeerimisega ma tegelesin  
tegelikult. Selles mõttes oli seal väga suur vahe ja loomulikult oli ka väga 
suur vahe siis minu  professionaalsuse tasemes, mis  tekkis.

\question{Kas sa oskad öelda, kas see oli pigem eeskuju või midagi muud, mis 
sind pigem programmeerimise poole suunas?}

Üks asi oli kindlasti eeskuju, aga teine asi oli ka kindlalt puhtalt ju see, et 
selle jaoks, et elektroonikaga tegeleda,  on vaja ikkagi mingisuguseid 
teatud füüsilisi asju. Sul on vaja elektroonikakomponente, sul on vaja 
tööriistu ja nii edasi, mida ju ei olnud. Isegi tänapäeval on ju poes  kõik 
olemas, aga sa pead minema ja üldse mitte vähe raha kulutama ja need 
endale ostma. Ma olen natukene ka, hobi korras, elektroonikaga tegelenud, 
tinutan üht-teist ja nii edasi. Ja noh, praegu, kui kõik on nagu 
justkui valla, kõik on olemas ja kahe päevaga tuuakse koju ära, raha ikka kulub 
selle jaoks. Mingi  üks jootekolb ja teine suurendusklaas, takistite 
komplektid, väiksed mikroprotsessorid, igasugused kivid ja sensorid ja andurid 
ja displeid ja nii edasi. Sellega ei ole lihtne alustada. Programmeerimine on 
niimoodi, et sul tegelikult on vaja seda kohta, kus nii-öelda arvutis käia, eks 
ole, paber ja pliiats ja kolm päeva on täitsa piisavad alustamiseks.

Nagu mul sõber ja töökaaslane Jaan Tallinn\index[ppl]{Tallinn, Jaan} on  
öelnud, et programmeerimine on selline naljakas asi, et  enamikes muudes  
valdkondades, kui sa hakkad   õppima, siis sa saad mingisuguse 
algse  taseme kätte ja  pead veel rohkem õppima, et saada järgmisele 
tasemele. Ja sa ei saa iseseisvalt õppida,  
on vaja kedagi, kes õpetab. Kui sa õpid näiteks klaverimängu, siis sul on 
vaja tegelikult, et keegi sulle pidevalt klaverimängu õpetaks, sa ei saa 
ise õppida klaverit mängima. On mingisugused teatud käelised asjad, et 
mismoodi sa seda teed, parimal juhul sa saad või mingist YouTube'i, videost või 
mingist õpikust õppida. Aga sul on vaja seda YouTube'i videod või õpikut. 
Programmeerimine, aga, on tegelikult selline asi, et kui sa oled selle algse oa 
kätte saanud ja sind siis suletakse üksikule saarele aastaks koos arvutiga, siis sa tegelikult suudad ise ilma ühegi õppevahendita õppida 
ennast väga heale tasemele, kui tahad. Puhtalt ise katsetamise,  ise mõtlemise  
teel. Ja eks tegelikult täpselt seda ma tegingi, kui ma teismeline 
olin.

\question{Kas sel ajal hakkas ka juba personaalarvuti moodi arvuteid liikuma?}

Jah, personaalarvuteid hakkas täiesti tulema ja meile koju tekkis ka üks Apple 
II\index{Arvutid!Apple II}. Sellega siis mina hakkasin toimetama, aga see oli 
üsna  kaheksakümnendate lõpus kuskil. Ma ei oska täpselt aastanumbrit öelda, 
aga ju ta võis juba 1988 olla või midagi niimoodi. Ma juba ikkagi  
nagu täiesti oskasin sel ajal programmeerida, ma ei olnud nagu päris enam kümneaastane, olin juba viisteist või kuusteist või midagi niimoodi. Inimestel, 
kes on seitsmeteist ja kaheksateist aasta vanused, on enamasti üsna 
kõvasti nagu meri põlvini  ja peod ja seltsielu ja asjad käivad. Aga minul on 
nagu paar asja teisiti. Esiteks ma olen üldiselt introvertne inimene ja mitte 
üliseltsiv, seltsielu mul kuidagi nii hästi nagu välja ei tulnud. See on 
üks asi. Teine lihtne tõsiasi oli see, et vist kuni viimaste aastateni, umbes 
kolmekümne viienda või neljakümnenda eluaastani oli mul elus selline asi, et kui ma 
joon kaks klaasi veini ära, siis hakkab mul pea valutama. Ma lihtsalt ei pea 
ühel  korralikul peol kaua vastu, ma lähen hiljemalt 
keskööks koju. Ja niimoodi on  kogu aeg olnud ja oli ka siis, kui ma olin 
kaheksateist, eks ole. Aga alates kella kaheteistkümnest ju nagu tegelik 
\emph{action} hakkab pihta, nagu mulle räägitud, ma nagu väga palju ise kogenud 
ei ole. Ja siis ongi, et kui  ülejäänud inimesed avastavad seal seltsielu 
ja pidusid, siis osad avastavad arvutiasju ja avastavad 
seltsielu natukene hiljem lihtsalt.

\question{Mida sa programmeerisid? Sellise jõukohase aga samas huvitava 
ülesande leidmine ei ole ju üldse lihtne?}

No eks poisikesi ikkagi mängud huvitavad üsna palju ja kindlasti ma arvan  
minul ja minu kaasvõitlejatel  kõigil oli ju üks esimesi unistusi, et 
kirjutada oma üks arvutimäng. Sel ajal olid juba  
Yamaha arvutid  tekkinud, eks ole, ja juba ka Apple II peal oli täitsa 
korralikke  mänge olemas. Need kommertsiaalsed mängud olid ikka sellisel 
tasemel, mida üks mingi hobistist poisikene ikkagi nädalaga valmis ei viska. Ja 
ega see tähelepanu ulatus  kolmeteistaastasel või viieteistaastasel ei ole väga 
selline, et suudaks midagi väga palju pikemat ette võtta. Selliseid väga 
lihtsaid mängukesi sai kindlasti ehitatud ja kindlasti ka proovitud üritada 
siis niimoodi häkkerlikult natukene läheneda sellele arvutile. Et mida on 
võimalik arvutit tegema panna, mis hääli on võimalik arvutit tegema panna ja  
igasuguseid lollakaid visuaalseid kujundeid ette manada, ja muu selline. 
See pool kindlasti ka huvitas. 

Aga hiljem, tegelikult teismeeas, sai igasuguseid asju proovitud. Ega täpselt ei 
teadnud, mida võiks  teha, aga valdkond kindlasti huvitas. Järjekindlamalt 
hakkasime mänge programmeerima Jaan Tallinna\index[ppl]{Tallinn, Jaan} ja Priit 
Kasesaluga\index[ppl]{Kasesalu, Priit} kui me olime keskkoolis. Siis oli 
juba niimoodi, et tähelepanu ulatus on inimesel juba nagu natukene 
kasvanud, eks ole, ja võtsime ette ühe mängu kirjutamise projekti. See sai 
natukene pikema vinnaga, et paneme nüüd kõik oma seni õpitud väiksed kogemused 
ja oskused kokku. Ja paneme kohe mitu inimest, mitte niimoodi, et igaüks oma 
nurgas pusib mingit oma mängu, vaid teeme ikka sellise tiimitöö. Jagame 
ülesanded omavahel ära ja kuude viisi töötame selle kallal. 

\question{Kust selline mõtte üldse tuli või arvamus, et selline asi üldse 
võimalik võiks olla?}

Kuskilt iseenesest tuli, ma ei oska täpselt mõelda. Meil isegi ei olnud  
mingit arutelu sel teemal. Lihtsalt sündis, et proovime midagi sellist. 

\question{Mis keskkool see oli?}

Mina õppisin Gustav Adolfi Gümnaasiumis\index{Koolid!Gustav Adolfi Gümnaasium}
\index{Koolid!Gustav Adolfi Gümnaasium|see{Tallinna 1. Keskkool}}
ja Jaan Tallinn\index[ppl]{Tallinn, Jaan} oli minu pinginaaber. Ja Priit 
Kasesalu\index[ppl]{Kasesalu, Priit} oli Jaan Tallinna pinginaaber eelmisest 
koolist, kus Jaan käis. Nii et me olime mõlemad Jaani pinginaabrid olnud. Viimase keskkooliaasta jooksul niimoodi kolmekesi kirjutasimegi ühe mängu, 
millel oli nimeks Kosmonaut\index{Mängud!Kosmonaut}. Mina küll  
kirjutasin seda kui hobiprojekti, aga Jaan ikka ütles, et see asi tuleb teha 
nagu äriks või see asi tuleks maha müüa ja selle eest  raha saada. 

\question{See oli nõukogude aeg ju veel, selle eest võis kinni minna ju?}

Peaaegu. See oli nõukogude aja lõpp küll, sel ajal, kus juba igasuguseid 
metalliärikaid juba juba käis ringi ja niisugune nagu väikene üle piiri  
kaubandus käis ja kooperatiivid ja asjad ja selline värk juba täitsa toimis. Me 
muidugi ei teadnud tuhkagi sellest, kuidas  see  ettevõtluse või selline maailm 
üldse käib. Ja ega tegelikult ei teadnud seda ka need suured inimesed, kes sel 
ajal ettevõtluses olid. Aga mingi tegevus toimus ja metalliäri alal 
mõningase kogemusega või 
sidemetega inimeste abil õnnestus meil tõesti see Kosmonaudi mäng müüa Rootsi. 
See oli selles mõttes muidugi pöördeline sündmus, et me saime selle eest 
lõppkokkuvõttes ikkagi, kui ma õigesti mäletan, siis oli see viis tuhat 
dollarit. See oli täiesti kosmiline number,  aasta oli mingi 1990 ja rubla 
kurss oli seal selline, et vist kui ma õieti mäletan, ühe dollari eest sai 
kolmkümmend rubla juba. Ja kui kokku arvutada, siis see viis tuhat dollarit 
oli ikkagi umbes selline summa, mis meie vanemad olid elu jooksul teeninud või 
midagi umbes sellist. Loomulikult inflatsiooniga võib korrutada ja 
korrigeerida, aga ikkagi oluline number, väga-väga oluline number. Kas 
nüüd mõelda, et kui õigesti me seda summat  kasutasime, kokkuvõttes sai ikkagi 
ka valuutapoes käidud ja Coca-Colat ostetud, selle peale kulus ka ikkagi 
märgatav osa sellest ära. Aga mina ja Jaan ostsime enda endale näiteks kahe 
peale arvuti. Selle peale läks pool minu ja Jaani osast sellest rahast 

Selle raha me saime kätte kuskil, see oli juba üheksakümnendate alguses,  Eesti 
kroon oli just tulnud või tulemas. Sellega on selline lugu, et see oli just 
täpselt see aeg, kus Eesti kroon tuli niimoodi, et minul oli see raha rubladena 
käes. Ja siis oli mingi kooperatiiv või firmakene, kust me siis olime kokku 
lepitud ja välja valitud mingi 386SX protsessoriga arvuti ja me olime seda 
ostmas. Ja ma mäletan, oli see hetk, kus meil oli teada, et järgmine 
päev on rahareform ja minul oli kümnete tuhandete 
kaupa neid arvuti jaoks mõeldud rublasid käes. Meil oli kokku 
lepitud, eks ole, et me anname nii palju rublasid ja saame  
arvuti. See kooperatiivitegelane, kellele helistasin, ütles midagi, et \enquote{too 
homme see raha} või midagi niimoodi. Aga mul ikka nii palju oidu oli, et ma 
ütlesin, et ei, on kokku lepitud, ma toon täna selle raha. Ja ma tõingi täna 
selle raha, ta võttis selle täna vastu ja me saime selle arvuti kätte. 
Nii et jah, ma ei tea, mis oleks 
juhtunud, kui me oleks tegelikult üritanud homme selle raha  maksta. 

\question{Eks ajalugu oleks läinud tonks teistmoodi. Aga see oli juba 386, mis 
oli juba päris korralik aparaat. Sinna vahele jääb ju õige mitu aastat 
puselemist mingite teiste inimeste arvutite juures. Kus te selle mängu 
kirjutasite? Kodus kellegi juures?}

Mängu me kirjutasime suurel määral tegelikult Jaani\index[ppl]{Tallinn, Jaan} 
ja Priidu\index[ppl]{Kasesalu, Priit} töökohas. Sest Jaan ja Priit keskkooli 
kõrvalt töötasid programmeerijatena ühes kooperatiivis. Mina tegelikult ka 
töötasin keskkooli kõrvalt programmeerijana poole kohaga minu vanemate töökohas 
ehk Küberneetika Instituudis\index{Küberneetika Instituut}. Aga  ütleme 
niimoodi, et ma arvan, et  minu vanemate tööandja oli selles mõttes mõistlik. 
Kui ma ise tööandjana mõtlen, et kui mingisugune seitsmeteistaastane poiss 
tahab tööle tulla,  alles õpib programmeerima või niimoodi, et ega esiteks ma 
ei maksaks talle väga palju või ma ei võtaks teda nagunii väga tõsiselt.  
Samuti ma võib-olla ei annaks talle nii palju mingeid võimalusi, ma vast ei 
annaks talle missioonikriitilisi asju. 

Aga Jaan ja Priit olid, olid tööl ühes kooperatiivis, kus nemad olid vaata et 
 peaaegu et juhtprogrammeerijad või midagi niimoodi.  Ja neil oli 
tunduvalt paremad võimalused  käes. Mis on noh, tänapäeval vaadates, ma ütleks, 
ikkagi küllalt ebamõistlik, aga need olidki  ebamõistlikud ajad. See tähendas, et 
nad ei saanud oma arvuteid nii-öelda töölt koju kaasa võtta, aga neil oli 
tegelikult töökoht, kus nad said päeval  olla koolis, aga õhtud-ööd said olla 
arvutis. Ja sel ajal,  kui sa oled kuusteist ja seitseteist, siis võid vastu 
pidada niimoodi, et magad kuus tundi päevas, siis kui vaja.

\question{Kui ma nüüd kokku loen, siis te käisite Gustav Adolfi Gümnaasiumis, 
mis polnud lihtne asi, te töötasite programmeerijatena ja takkapihta 
kirjutasite mängu, mille kannatas pärast maha müüa. Kõike samal ajal?}

Jah, peab ütlema küll,  et vähemalt siis, kui mina  programmeerijana töötasin, 
ma töötajana ei ole uhke tööpanuse üle, mille ma Küberneetika Instituudile 
andsin\index{Küberneetika Instituut}. Tõsi küll,  ma sain ikkagi midagi valmis 
ja mu tööandja oli sellega rahul. Ma ei olnud ka tegelikult ainus, oli natukene 
teisigi selliseid õppijaid ja mõni üliõpilane, kes oli seal niimoodi tööl ja ma 
sain isegi aru, et mu tööandja isegi oli pigem minuga rohkem rahul kui seal 
mõnede teistega. Aga ma arvan, et see ütleb rohkem nende teiste  kui 
minu kohta. Mina ikkagi kulutasin enamiku ajast selle mängu ja koolis käimise 
peale.

\question{Sel ajal hakkasid tekkima esimesed BBS-id ka?}

BBS-id hakkasid tekkima ja minu tutvusringkonnast siis Priit 
Kasesalu\index[ppl]{Kasesalu, Priit} oli see põhiline, kes meie kambas tegeles 
BBS-idega ja ühe ka püsti pani, mille nimi oli \emph{Dark Corner}\index{BBS!Dark 
Corner}, kui ma õigesti mäletan. Ja mille Fido, kuidas seda siis nimetati, 
\emph{node} number või midagi sellist, oli, kui ma õieti mäletan, neliteist. Ja 
teda tõmbas nagu see pool kuidagi rohkem või kuidagi väga palju ja eks 
kindlasti mind ka, sest BBS-iga tekkis järsku  võimalus  ekraani kaudu suhelda 
hästi paljude teiste inimestega, kellega sa võib-olla füüsiliselt ei istu koos. 
Teatud mõttes võiks isegi öelda, et järsku nendele inimestele anti natuke nagu 
Facebook kätte. Mitte taskusse otseselt, aga ikkagi kätte või niimoodi, et 
järsku tekkis hulk sõpru, kellega ma olin suhelnud ainult interneti teel. Fidos vahetati mõtteid  kõikide asjade üle, mitte ainult arvutite üle ja tekkis 
järsku üks mingisugune  täiesti isevärki sotsiaalne seltskond. Tolle aja 
kohta oli see väga isevärki sotsiaalne seltskond. Tänapäeval on niimoodi, et 
sotsiaalne seltskond, kes on mingi Facebookis  grupi, olgu mingi 
MMS-i klubi või ma ei tea mis, liige,  siis nad võivad aeg-ajalt kokku saada. 
Netiajastul on see tegelikult väga-väga tavaline. Aga  selline, kuidas 
öelda elustiil või tutvusringkonna ülesehitus järsku tekkis  meile kätte, kui  
aasta oli umbes 1990 või umbes kuskil sealkandis.

\question{See seltskond pidi siis olema ka teatavas mõttes homogeenne, sest 
Fido külge saamise barjäärid olid kõrged?}

Jah, eks muidugi oli palju ka inimesi, kes nii-öelda jõlkusid kaasas. Olid 
sellised entusiastid nagu näiteks Priit Kasesalu\index[ppl]{Kasesalu, Priit} 
või Tarmo Mamers\index[ppl]{Mamers, Tarmo} näiteks ja no nende muud sõbrad  
aeg-ajalt tekkisid ju ka sinna sisse, kellele siis  Tarmo või Priit võimaldasid 
ligipääsu. Ja see oli kindlasti väga huvitav. Tekkis selline  sotsiaalne 
distants-suhtlus. 

Ma mäletan ühte juhtumit, oli juba tegelikult siis, kui vaikselt Internet juba 
hakkas Eestisse tekkima. Internet kui selline tehniliselt oli ju olemas juba 
 kuskil seitsmekümnendatel või kaheksakümnendatel, aga  Eestisse  ta hakkas umbes sel 
ajal niimoodi natukene  tekkima. Mul oli selline sõber, siiamaani väga hea 
sõber, nimega Sulo Kallas\index[ppl]{Kallas, Sulo}, kellel oli ka BBS ja kes praegu 
töötab minuga koos Starshipis\index{Starship Technologies}. Tema andis 
mulle kasutada ühte oma kontot ühes Unixi arvutis. Ja Unixis oli olemas selline 
programm nagu \verb|talk|, kus omavahel ekraani kaudu said suhelda 
inimesed, kes olid samasse masinasse sisse loginud. Ja ma mäletan, et  minu 
jaoks oli üks ikkagi täiesti selline silmi avav elamus. Mul ei olnud sel ajal 
kodus telefonigi. Midagi ma toimetasin selle Sulo kontoga Sulo nime alt 
selles arvutis ja järsku selle \verb|talk|iga  hakkab minuga keegi 
rääkima.  Ütleb, et minu nimi on Epp. Nii, ja mina siis esimese asjana, kuna ma 
teadsin, et ma kasutan Sulo kontot, eks ole, keegi Epp tahab Suloga rääkida. 
Selgitasin talle, et kuule, mina ei ole Sulo, et mina olen hoopis üks 
teine inimene. Tema ütleb vastu, et  sellest pole midagi, räägime ikka. Ma ei 
saanud täpselt aru, mis värk on nagu, mis mõttes, ta ju tahab Suloga rääkida, 
eks ole. Aga siis ma sain aru, et ta tahab tegelikult lihtsalt kellelegi 
rääkida, et tal  tegelikult on täitsa okei, et ta räägib  minuga. Sihuke 
jutuajamine tekkis sealt, ja ma sain selle jutuajamise käigus teada, et  
tegemist on ühe Eesti tüdrukuga, kelle nimi on Epp ja kes hetkel füüsiliselt 
asub Ameerikas, ta oli ühe Ameerikasse  ülikooli üliõpilane. 
Ja mina istun Eestis, eks ole, ja reaalajas räägin arvuti 
ekraani vahendusel  temaga juttu. Me rääkisime maast ja ilmast 
mingisugune tund aega, see oli  väga-väga kummaline kogemus. Sa  suhtled 
kellegagi reaalajas, kes on  sinust väga-väga kaugel. Ma siiamaani ei tea, 
kes Epp täpselt oli, ta ütles oma perekonnanime ka, ma ei ole seda nime mitte 
kunagi hiljem kuulnud, mitte kunagi hiljem selle inimesega suhelnud. Aga see 
oli ikka väga kummaline kogemus minu jaoks. Ongi naljakas tegelikult, et 
tänapäeval ju selline asi on ju niivõrd tavaline, kõigil mingid Snapchatid ja 
asjad on kuskil taskus, eks ole. Ja tol ajal oli sotsiaalses mõttes see, et sa 
võid suhelda inimestega üle maailma,  oli nendele interneti häkkeritele 
võimalik ja teistele inimestele ei olnud.

\question{Sa ütlesid, et BBS-ides räägiti igasugustel teemadel. Näiteks, 
millest räägiti?}

\label{sisu!inimeseks}Kui ma õieti mäletan, seal oli igasugust, sellist elulist, nagu tänapäeva 
internetifoorumid, eks ole. Kõigest võidakse seal rääkida. Seal oli mingisugune 
filosoofiateemaline  vestlusgrupp, kus  inimesed olid ju enamasti sellised 
kaheksateistaastased, kes alles mõtestavad oma elu. Ongi selline aeg inimeste 
elus, kus kõik mõtlevad, mida  mingisugused asjad tähendavad ja kas ikka inimene 
peaks panustama sellele või tollele. Tänapäeval neljakümneaastasena väga 
võib-olla ei viitsi sel teemal juttu vesta, kõigil on juba oma elu 
tõekspidamised välja kujunenud, aga tol ajal minul kindlasti ei olnud ja enamik 
sellest ülejäänud BBS-i seltskonnast oli ka umbes sama vanad, eks ole. Siis oli 
seal igasuguseid psühholoogiateemalisi, neid vestlusi oli igasuguseid, see 
kindlasti ei olnud sugugi mitte ainult tehnoloogiateemaline. 

\question{See, mis sa ütled, kõlab väga oluliselt. Sest see tähendab, et 
mingisugune ports nutikaid inimesi mitte üksinda ja mitte juhuslike inimestega 
vaid koos sama moodi mõtlevate ja samade oskustega inimestega mõtestasid seda, 
mida tähendab olla inimene kõige laiemas mõttes}

Absoluutselt. See oli tegelikult üks niisugune virtuaalne sõpruskond.  Võib 
olla võib öelda, et see Fido seltskond oli kõige esimene virtuaalne sõpruskond 
Eestis üldse. Tänapäeval on  igaühel virtuaalseid sõpruskondi taskus sada tükki 
aga see võis olla võib olla täiesti esimene.

\question{Kas selle kõige juurde käis ka mingi spetsiifiline raamatu-, muusika- 
või filmihuvi?}

Ahaa, muusikakanaleid, muusikateemalisi  vestlusgruppe,  oli loomulikult ka. 
Aga huvi mõttes, minul ei käinud. Võib-olla natukene. Ma arvan, et  selles ringkonnas pigem olid 
populaarsed sihukesed elektroonilise muusika bändid. Nii, ja naa, ütleme. 
Kraftwerk mulle ei meeldinud ja ei meeldi siiamaani, Jean-Michel Jarre samuti 
mitte nii väga palju aga Tangerine Dream näiteks meeldis mulle väga ja 
siiamaani meeldib, mul on ikka mingi viisteist nende plaati ja nii edasi. Aga 
samas jälle ma olen inimene, kes ei ole kunagi vaadanud Star Warsi, ma ei ole 
kunagi lugenud \emph{Hitchhiker's Guide to The Galaxy}'t. Minu jaoks  on 
esteetiline subkultuur ja arvutid natukene lahus seisnud.

\question{Endal sul BBS-i ei olnud?}

Minul endal BBS-i ei olnud. Ma vist nagu kuidagi ei tahtnud ka, see oli ikka hull 
jahmerdamine, mis selle üleval hoidmiseks vajalik oli. 
Mul oli väga hea meel, et ma sain  Priidu BBS-i kasutada.

\question{Selge. Aga siis te müüsite selle mängu maha, mis edasi sai?}

Noh, kui üheksateistaastasele inimesele anda nii palju raha, nagu tema vanemad 
on kogu elu jooksul teeninud, eks ole, siis tal karjäärivalik on nagu selge 
kohe. Et noh, sellist küsimust nagu ei olnud, et mida ma siis 
tulevikus professionaalselt tegema hakkan. Loomulikult programmeerija.  Ja  mul 
oli ka selline mõtlemine, ma ei tea, kui õigustatud see oli, aga ma arvasin, et 
et noh, eriti üheksakümnendate alguses Eesti ülikoolides eriti midagi väga 
kasulikku sel teemal ei õpetatud. Ma ei tea,  kui õige või vale see on. 
Kindlasti vastas tõele see, et meil keskkoolis oli  arvutiõpetus ka ja üldiselt 
ikkagi meie klassist pigem paljud teadsid rohkem kui meie õpetaja. Ma 
miskipärast oletasin, et ülikooliga oli samamoodi, ei tea, kas see on tõsi või 
mitte. Tänapäeval see kindlasti ei ole tõsi aga  tol ajal  võib-olla pigem oli. 
Igatahes ma tegin selle otsuse, et ma ei lähe ülikooli õppima midagi 
programmeerimise või arvutitega seotut, vaid ma läksin hoopis õppima füüsikat. 
Füüsika oli kindlasti mul  niisugune teine selline huviala,  ma olin 
füüsikaolümpiaadidel käinud  ja mulle see kindlasti kindlasti väga meeldis. 

\question{Mis sulle füüsika juures meeldis?}

No võib-olla natuke sihuke filosoofiline aspekt, et ma sain kuidagi aru, kuidas 
nii-öelda maailm toimib teatud mõttes. See oli põnev. Mingid sihukesed asjad, kui 
 mingid tuumafüüsikad ja mingid planeedid, kuidas liiguvad ja niimoodi, see 
natukene andis võib just sellist filosoofilist mõõdet. Et mis see maailm meie 
ümber on ja kui suured või väikesed meie, inimesed, selles maailmas  oleme.  Ja pigem ikkagi väga väikesed oleme. 

\question{Kuhu sa läksid  füüsikat õppima?}

Ma läksin  füüsikat õppima Tartusse\index{Tartu Ülikool}, koos Jaan 
Tallinnaga\index[ppl]{Tallinn, Jaan}. Pinginaabrid läksid mõlemad õppima 
füüsikat. Sellega läks niimoodi, ma kindlasti  tegelikult ei väärtustanud seda, 
et piltlikult öeldes mul oleks paber taskus, et mul ülikool oleks   
edukalt lõpetanud. Ja kui ma olin ühe või poolteist aastat ülikoolis ära 
olnud, siis mulle hakkas veel rohkem kohale jõudma see, et tegelikult ma ju 
tegelen programmeerimisega kogu aeg, töötan professionaalse programmeerijana. 
Samal ajal tegime järgmist mängu, mille me kavatsesime maha müüa ja nii 
edasi ja nii edasi. Ja ma ei kavatsenud kunagi füüsikuna töötada, ma õppisin hobi 
korras. Kui esimese aasta sai hobi korras  füüsikat õppida, 
siis teisel aastal hakkad aru saama, et tegelikult  õppejõud ikkagi eeldavad, 
et sa tõsiselt tegeled selle asjaga, panustad  füüsika õppimisse enamiku oma ajast. 

Ja siis ma tulin ülikoolist ära. Ma sain aru, et see asi lihtsalt nõuab rohkem 
tööd, kui ma olen nõus sinna sisse parema ja siiamaani ma ülikooli lõpetanud ei 
ole. Jaan Tallinn\index[ppl]{Tallinn, Jaan} käis ülikooli lõpuni ja õppis 
füüsika lõpuni. Tegi oma oma lõputöö, kui ma õieti mäletan, 
relatiivsusteooriast. Sellest, kuidas ruumi painutada selle jaoks, et reisida 
valguse kiirusest suuremate kiirusega ühest kohast teise. Ma küll oletan, et 
tõenäoliselt  ta mingisugust väga suurt teadmist ühiskonnale sellega juurde nii 
väga ei lisanud selle nelja aastaga, mis ta õppis, aga sellise töö ta tegi. Ta 
on rääkinud, et ükskord, kui ta kuskil seltskonnas kirjeldas oma seda tööd, 
mida ta tegi, siis tema vestluskaaslane küsis  vastu, et kas see oli nagu 
rohkem teoreetiline töö või tuli seal ka mingeid praktilisi laboratoorseid 
katseajale.

\question{Selle asja nimi, mida te tol hetkel kampas pidasite, oli juba 
Bluemoon\index{Bluemoon}?}\label{sisu!bluemoon}

Jah. See mängutegijate punt, me hakkasime ennast nimetama nimega Bluemoon 
Software ja Bluemoon Interactive. Inimesed ikka tahavad panna mingisuguseid 
kõlavaid firmanimesid.

\question{Aga miks just Bluemoon?}

Lihtsalt oli üks nimi. Ma arvan, et me ei osanud nimesid üldse välja mõelda ja 
ma olen kogu aeg pidanud ennast väga halvaks nimede väljamõtlejaks ja et ma ei 
valda seda valdkonda üldse ja niimoodi, aga kui Starshipile\index{Starship 
Technologies} nime panin, siis ikkagi osalesin selles kõvasti ja  lõpuks oli 
ikkagi minu pakutud nimi, mis selleks lõpuks sai.

\question{Programmeerimise juures pidi olema täpselt üks raske asi, nimede 
välja mõtlemine}\sidenote{Eksin tsitaadiga. Täpne tsitaat on 
Netscape\index{Netscape} arhitekti Philip Karltoni\index[ppl]{Karlton, Philip} 
poolt ja kõlab nii: \enquote{\emph{There are only two hard things in Computer 
Science: cache invalidation and naming things}}.}

Ma olen täitsa nõus sellega, võib-olla nüüd neljakümneaastasena on juba 
natukene rohkem käppa seda saadud. 

\question{Mis sa praegu teed?}

Praegu ma olen sellises firmas nagu Starship Technologies ja ehitan 
pakiroboteid. Asutasime selle selle firma koos Skype'i\index{Skype} kaasasutaja 
Janus Friisiga\index[ppl]{Friis, Janus}  neli pool aastat 
tagasi\sidenote{Intervjuu Ahtiga toimus jaanuaris 2019}. Ja meil oli selline 
visioon, et asjad võiksid ju maailmas liikuda automaatselt samamoodi, nagu 
elekter tuleb meile stepslisse  ise  ja veevärk on olemas ja 
informatsioon tuleb läbi interneti. Aga asjad liiguvad ikkagi  läbi meie maja 
või korteri ukse, tulevad füüsiliselt kohale ja alati mingisugune inimene toob 
seda, kas sa ise tood või siis sa maksad kellelegi inimesele, kes toob. Ja see 
on hirmus raiskav ja asjad võiksid liikuda automaatselt samamoodi, nagu me 
lennukipileteid broneerime üle interneti nii öelda automaatselt, ilma et me 
läheksime füüsiliselt kohale kuskile reisibüroosse seda lennukipiletit ostma.

\question{Starshipi tegemine on ju juhtimise töö. Kuidas sa jõudsid 
programmeerimise juurest selle töö juurde, mida sa praegu teed ja kui erinevad 
nad sinu jaoks on?}

No need on ikka väga erinevad. Minu jaoks on see areng olnud selline, et ma 
olin programmeerija ja ma olin programmeerija üsna kaua aega, ilma et ma oleks 
üldse midagi kuskil juhtinud. Ja kui me hakkasime startuppe tegema koos Jaanus 
Friisi ja Niklas Zennströmiga\index[ppl]{Zennström, Niklas} siis ma olin 
neis tehnilise arhitekti rollis. Arhitekti roll on juba rohkem 
natukene nagu  juhtimisega seotud, aga sa ei juhi nii väga  inimesi või 
organisatsioone või protsesse, vaid sa juhid just tehnilist arhitektuuri. Et 
milline see masin niisuguses suures plaanis kokku tuleb, mida siis terve suurem 
tiim inimesi ehitab. Nagu maja ehitamisega: osad inimesed ehitavad ja panevad 
kive üksteise peale ja on ka teisi inimesi, kes vaatavad seda projekti 
suuremalt, et kus peaks olema aken ja mitu akent me üldse teeme ja kas 
me teeme rohkem ümmargused aknad või teeme kandilised  ja nii edasi ja nii 
edasi. Ja ma olin Skype'is alguses tehniline peaarhitekt ja 
mitmetest teistes startuppides samuti. Skype'is veel natukene pooleldi juhtisin 
ka ühte väikest tiimi, kus  ma tegelesin sellega, et mõelda umbes viiele 
inimesele välja seda, mida nad tegema peaksid ja koordineerida nende tööd. 
Mõtlesin välja, mis meie eesmärk peaks olema, kuhu poole me peaksime liikuma ja 
nii edasi. Sihukene viieinimeselise tiimi juhtimine oli selline 
nagu väike harjutus või  sissejuhatus, et mingisuguseid kogemusi natukene sain 
või natukene kujutasin ette. Hiljem olen juhtinud siis ka natuke suuremaid 
tiime, umbes kümneinimeselisi ja niimoodi. Aga Starship oli esimene koht, kus 
ma üsna kiiresti, esimese kahe nädalaga, võtsin tööle kümme inimest
ja esimese poole aastaga oli juba umbes kakskümmend inimest meil tööl ja 
nii edasi  läks juba natukene suuremaks see asi. Eks ma niimoodi käigu pealt 
natukene siis  õppisin, et  kuidas juhtimine käib. Ju ma olen kindlasti veel 
üsna  alguses, et me oleme siin Starshipis olnud sihukeses  naljakas 
olukorras, kus nagu juhtimises ikkagi üsna kogenematu juht on olnud sellel 
firmal. Neli aastat ma olin tegevjuht ja  nüüd jõudis pool aastat tagasi siis 
asi nii kaugele, et me palkasime  professionaalse tegevjuhi Lex 
Bayeri\index[ppl]{Bayer, Lex} Californiast. Ja mina olen CTO ehk 
tehnikadirektor, kus ka peab üsna palju juhtima, aga nüüd enam mitte kahtsadat 
inimest, vaid natukene väiksemat hulka inimesi.

\question{See on siis olnud pikk ja just vajadusest ja huvist kantud õppimine?}

Jah, absoluutselt.  Üldiselt ma ütleks niimoodi, et paljud programmeerijad, 
kaasa arvatud ka mina, meile programmeerimine meeldib nii palju, see on niivõrd 
tore  ja niivõrd äge tegevus, et selliseid masinaid ehitada, et tahaks  
muudkui ehitada neid masinaid. Inimeste juhtimine on pigem selline asi, mida 
enamik programmeerijaid väga ei taha teha ja ma ei ole päris kindel ka ise, kui 
palju mina seda tegelikult teha tahan. Aga küll on lihtsalt asi selles, et kui 
sa oled  üksikprogrammeerija ja sul kogemus tekib ja sa oled  arhitekt siis sa 
oskad juba rohkem  arvata, mismoodi me seda tarkvara peaksime ehitama ja mis 
asjad on selle juures olulised ja mis need ei ole. Siis on nagu on kaks võimalust. 
Kas sa  oled vait ja osaled selles protsessis  kellegi teise juhtimisel või siis 
sa üha rohkem nagu vaatad, et ei, ma teen ise, ma teeksin seda paremini 
kui see juht, kes meil on. Ma tahaks ise seda asja juhtida või mul on juba nii 
hea ettekujutus, kuidas seda teha, et ma ei suuda pealt vaadata, kui 
mingisugune teine inimene, kes on võib-olla väiksema kogemusega kui mina,  
kuidagi seda asja juhib ja mitte selles suunas, kus mina olen täiesti 
veendunud, et  õige oleks. Ehk siis see on tulnud justkui nagu vajadusest. Kui 
sa oled üksikprogrammeerija, siis sa aja jooksul ikkagi saad aru, et sa saad 
tegelikult lõppkokkuvõttes rohkem tehtud, kui sa piltlikult öeldes palkad 
endale tiimi ja hakkad juhtima mingisuguseid suuremaid seltskondi. 

Minu jaoks küll see nii-öelda raketiga lendamine nagu Starshipis, kümneinimeselise 
tiimi juhtimisest kuni selleni, et ma tükk aega juhtisin üle kahesaja inimesega firmat, võttis ikkagi pea ringi käima. Et ma kindlasti 
edutasin ennast  oma ebakompetentsuse tasemele. Aga eks kohati öeldaksegi, et 
starupid ongi asjad, mis  väga sageli on  klassikalise sellise juhtimise 
distsipliini ja teooria ja juhtimispraktikate mõttes väga halvasti juhitud 
organisatsioonid. Mis ei ole siiski tihti takistuseks olnud nende edule, 
sellepärast et nad on olnud nii piisavalt värske mõtlemisega, nende toode on 
olnud piisavalt selline värske ja revolutsiooniline, et sellest ei ole olnud 
hullu, et nad on olnud halvasti juhitud. Tegelikult ikkagi need kakssada 
inimest, kes meie Starshipis töötavad,  ma ikkagi vaatan nende peale küll nagu 
niimoodi, et palun vabandust nende ees, et nad on osalenud sellises loomkatses, 
et ma olen neid mitu aastat juhtinud. See ei ole võib-olla olnud aus nende 
suhtes. Aga samas nad ei ole ka sugugi mitte meil siin firmast minema jooksnud 
ja tunduvad olevat rahul, et võib olla väga hullusti siis ei olegi läinud.


\chapter{Madis Kaal}
\label{cptr:mast}
\index[ppl]{Kaal, Madis}
\index[ppl]{Mast|see{Kaal, Madis}}

\question{Kuidas arvuti Saaremaale sai?}

Arvuti ei saanudki Saaremaale. Minu esimene kokkupuude päris arvutitega oli 
Rahvamajanduse Saavutuste Näitusel\sidenote{Tänapäeva mõistes oli tegu 
messikeskusega, kus ajutistel või püsinäitustel demonstreeriti 
kas liiduvabariigi (nagu Tallinnas asunud näituse puhul) või kogu Nõukogude 
Liidu majanduslikku võimekust. NSVLi Rahvamajanduse Saavutuste Näitusest arenes 
välja Eesti Näituste Messikeskus.}, praeguses Pirita näitusehallis. Käisin 
seal koos oma emaga. Ühes nurgas olid üles pandud 
terminalid, mida manageeris kaks imeilusat tüdrukut. Seda siis ajaloolisest perspektiivist, tõenäoliselt oli tegemist üsna keskmiste 
operaatoritega, aga siis tundusid nad imeilusad ja targad. Terminalide peal oli 
nõukogudeaegne venekeelne raamatukogude otsingu andmebaas. Terminalid ise olid ka 
venekeelsed. See oli esimene kord, kui ma reaalselt nägin, et ekraanil olid 
tähed ja klaviatuuril sai kirjutada. 

\question{Mis aastal see oli?}

Arvatavasti 1983. Ja need terminalid jätsid kustumatu mulje. 

\question{Kas pärast seda tekkis sul selge soov terminalide 
juurde pääseda?}

Pärast seda tekkis väga selge mõte, et see asi huvitab mind. 
Seejärel sattusin Tartusse ja ostsin sealt venekeelse 
raamatu \enquote{Programmeerimine keeles PL/I\index{PL/I}} ning lugesin 
seda. Ma ei teadnud arvutitest veel midagi, aga tasapisi hakkas selgeks saama, 
misasi on programmeerimine ja näiteks \verb|for|. See oli mingi imeline struktuurkeel, mitte päris vene, 
vaid kõlas nagu piraatversioon.

Järgmine kord nägin arvuteid Tehnikaülikooli\index{Tallinna 
Tehnikaülikool}, tolleaegse TPI\index{TPI} lahtiste uste päeval, kus me käisime 
pinginaabriga, kellega koos pärast ka kooli sisse astusime. 
Meile tehti ekskursioon automaatikateaduskonna kõigis 
kateedrites\index{Tallinna Tehnikaülikool!Automaatikateaduskond} neljal 
korrusel ja mõnes kohas olid arvutid. Mäletan selgelt, et Indrek 
Saul\index[ppl]{Saul, Indrek}, kes oli minu meelest sel ajal tudeng ja hiljem 
kinnisvaraärimees, näitas meile analoogarvutit. Sellega
sai analoogpingete ja skeemiga diferentsiaalvõrrandeid lahendada.

\question{Vanasti sihiti ju õhutõrjekahureid analoogarvutitega.}

See masin võis täiesti olla sedalaadi projekti osa. Igatahes mul tekkis kindel soov seda valdkonda
õppima minna, aga pinginaaber veenis mind ümber, et lähme parem 
raadiotehnikasse, ikkagi sama maja.

\question{Kas esimest korda arvuti nägemise ja ülikooli sisseastumise vahele jäi veel 
midagi arvutitega tegelemise mõttes?}

Ainult see üks raamat. Otsus arvuteid õppima minna sündis esimesel korral ja raamat tuli 
pärast seda. Ainuke imelik asi oli otsus raadiotehnikasse 
minna, aga selle vea parandasin ruttu ära. Ülikooli teise korruse otsas oli arvutussaal, kus oli kaks või 
kolm SM-4\index{SM EVM!SM-4}. Need olid PDP-11\index{PDP-11} vene 
versioonid. Pärast seda, kui sain aru, kuidas sinna sisse saab, ma enam 
tundidesse ei jõudnud. Ja kuna olin maalt tulnud poiss ja raha ka üldse ei olnud, 
käisin lihtsalt kõik kateedrid läbi ja küsisin iga ukse vahelt, kas neil on tööd anda. Raadiotehnika kateedris\index{Tallinna 
Tehnikaülikool!Automaatikateaduskond!Raadiotehnika kateeder}\label{sisu!mast_raadiotehnikas} oli, 
mind võeti sinna laborandiks tööle ja nii see läks. Kool jäi pooleli, kateedrisse
jäin seitsmeks aastaks paika.

\question{Mitmendal kursusel kool pooleli jäi?}

Esimesel kursusel. Algul olin raadiotehnika kateedris laborant ja pärast 
tehnik. Sattusin tuppa, kus olid väga toredad inimesed: Mart 
Palmas\index[ppl]{Palmas, Mart}, kes õpetas mulle peaaegu kõike, mida ma 
programmeerimisest tean, ja Villem 
Vannas\index[ppl]{Vannas, Villem}, kes praegu töötab Datelis\index{Datel}. Tema 
õpetas mulle enam-vähem kõike, mida ma rauast tean.

\question{Siis ei jäänud ju haridus pooleli.}

Formaalselt siiski jäi. Tol ajal oli
laborant rohkem nagu abitööline. 
Parandasin seda, mida vaja, aitasin seal, kus vaja. Mu esimene töö oli 
kolikamber tühjaks tõsta.
Algusaegadel oli üsna suvalisi projekte, hiljem tekkis
suund kommunikatsiooni poole, mis tundus mulle sel ajal huvitav. 

1990. aasta paiku tekkis Eestis 
mitu huvitavat suunda. Kõigepealt hakkas tulema 
personaalarvuteid. Sinnasamasse, kus oli kunagi SM-4 arvutiklass, tekkisid 
personaalarvutite klassid. Neid oli mitu tükki ja erinevate portsudena 
toodi Austraaliast MicroBeesid\index{MicroBee}\sidenote{1982. aastal 
Austraalias algselt komponentide komplektina müügile tulnud koduarvuti. Tuntud 
huvitava videolahenduse ja patareitoitel mälu poolest, mis 
võimaldas arvutit teisaldada mälu seisu kaotamata.}. Kuskilt tuli terve klassi jagu MSXi 
arvuteid\index{Yamaha MSX} ja siis mõned 
Robotronid\index{Robotron}\sidenote{Robotron (originaalis VEB Kombinat 
Robotron) oli Ida-Saksamaa suurim arvutitootja.}. 
Raadiotehnika kateedris oli juba siis, kui mina sinna sattusin, olemas 
Apple II\index{Apple II} ja mõned aastad hiljem tekkis sinna IBM 
PC\index{IBM PC}. See oli omapärane kogemus. Apple II peal olid 
harjunud, et lülitad sisse ja pilt on ees. IBMi sisse lülitades ei juhtunud midagi. Ühel hommikul tööle tulles vaatasin, et uus 
arvuti, ja lülitasin sisse. Midagi ei juhtunud. Ootasin natuke aega ja lülitasin välja, ise 
tegin näo, et midagi pole toimunud. Hiljem selgus, et masin tegi \emph{self 
test}'i. Seal oli tublisti mälu sees ja testimine võttis palju aega -- ma ei 
suutnud nii kaua oodata. 

\question{Midagi pidi see ju ekraanil senikaua näitama?}

IBMil oli roheline long-fosfor\sidenote{Katoodkiirtel põhinevates monitorides suunati laetud osakeste kiir fosforühendiga kaetud ekraanile. Kasutatud ühendi tüübist sõltus nii elektronkiire mõjul tekkinud värv kui ka see, kui kauaks ekraan peale kiire edasi liikumist helendama jäi. Selle viimase järgi liigitataksegi ekraanides kasutatavaid fosforühendeid  \enquote{pikkadeks} ja \enquote{lühikesteks} (ingl. \emph{long} ja \emph{short}).} monitor, mis läks tükk 
aega käima, ja ma ei jõudnud esimese \emph{boot}'imise ajal ära oodata, millal 
midagi toimuma hakkab.

Üheksakümnendate paiku tekkis meile tuhande kahesajane modem, mis läks 
PC sisse. Sel ajal olid just tulnud esimesed BBSid ja umbes samal ajal otsustas TPI 
automaatikateaduskond\index{Tallinna Tehnikaülikool!Automaatikateaduskond} 
ehitada arvutivõrgu. Toodi kohale viiesajameetrine kaablirull 
kollast sõrmejämedust Etherneti kaablit ja umbes kümmekond komplekti 
kobakaid kaabli peale, mille külge käis teine jäme kaabel, 
mis läks võrgukaardi taha. See oli nagu esimene Etherneti tehnoloogia. 
Mäletan selgelt, et meile toodi ainult kaabel ja kobakad, ei mingeid tööriistu, pistikuid ega terminaatoreid.

Kateedris oli sel ajal eterniiditahvlitest lagi, mille peale me selle Etherneti kaabli tõmbasime. Et kobakad külge saada, tegime
naaskliga kaabli kesta sisse augud, ajasime nõela läbi ja ühendasime
arvutite külge ning tinutasime otsa terminaatorid ja takistid. 

\question{Tarmo Mamers\index[ppl]{Mamers, Tarmo} rääkis, kuidas te PC ja Maci 
vahele traati vedasite. Kas too kaabeldamine oli enne või pärast seda?}

See oli meil kahe PC -- sellesama raadiotehnika kateedri PC ja Tarmo oma -- vahel, Tarmol oli 
veidi vägevam AT arvuti. Ühendasime need 
kaabliga ja tegime väikese 
\emph{chat}'i programmi, et teineteisega suhelda. 

Arvutivõrk tekkis sellest hiljem. Tehnikaülikooli toodi Novell 2.15\index{Novell} 
server, mille ma installisin ja mis oli üks esimesi väheseid asju, millel oli manuaal 
olemas, nii et kõik oli justnagu ametlik. Novelli serveri peal panin käima Pegasuse 
Maili\index{Pegasus Mail}-nimelise asja, kuhu külge kirjutasin \emph{gateway}, 
millega sai UUCP meili, mida toimetati Küberi 
majja Soomest (ma ei mäleta, kas Soomest siiapoole lükates või siit 
üle telefoniliini tõmmates). Tõmbasime selle oma pisikese modemiga 
Tehnikaülikooli majja ja jagasime kasutajate vahel laiali.

\question{Siin tundub jälle suuremat sorti lünk olema selle vahel, kuidas sulle 
hakati programmeerimist õpetama ja kuidas sa naaskliga kaablit torkisid ja 
\emph{gateway}'sid programmeerisid.}

Mõned aastad tuli õppida asjade kirjutamist lihtsalt erinevaid asju tehes ja ehitades, aega katsetamiseks oli palju. 
Olin noor inimene, peret polnud ja praktiliselt elasin raadiotehnika 
kateedris\index{Tallinna Tehnikaülikool!Automaatikateaduskond!Raadiotehnika 
kateeder}. Meil oli seal omamoodi seltskond: arvutussaali 
kamp, Tarmo\index[ppl]{Mamers, Tarmo} kohe kõrval sama 
koridori peal ja mina üleval raadiotehnikas. Vana kooli mees 
Lõvi\index[ppl]{Lõvi} oli kõrvalkorpuses ja käis aeg-ajalt Apple II peal 
oma projekte arendamas.

\question{Kas meetodiks oli siis peamiselt katsetamine, mitte 
manuaalide tudeerimine?}

Manuaale ega dokumentatsiooni ei olnud üldse. Riiklikul 
tasemel tarkvara varastamise programm pakkus küll ägedat tarkvara, aga 
enamasti ilma dokumentatsioonita. See oli nagu infovaakumis 
tegutsemine ja disassembler\sidenote{Programm, mis teeb masinkoodist 
oluliselt loetavamat Assemblerit.} oli justkui sõber.

\question{Keegi pidi sulle ju ometi ütlema, et selline asi nagu disassembler 
on olemas.}

Jaa, seda tegid head vanemad kolleegid, kes hoidsid kätt ja 
juhendasid. Lõviga\index[ppl]{Lõvi} tegutsesime pikalt koos, temal oli kindlasti 
väga suur mõju minu arengule. Aga see lünk, kuidas ma BBSideni 
jõudsin, sai täidetud nii, et mul oli raadiotehnika kateedris\index{Tallinna 
Tehnikaülikool!Automaatikateaduskond!Raadiotehnika kateeder} 
arvuti, mille sees oli modem ja millega sai helistada. Lähim BBS
asus Küberneetika Instituudi otsas, kus tollal asus 
Proekspert\index{Proekspert} ja kus nüüd on Tehnopoli kontor. Andrus Suitsu\index[ppl]{Suitsu, Andrus} 
oli BBSi mees, käisin tema juures oskusteavet ja tarkvara 
hankimas. Panin algul BBSi ja peatselt pärast seda ka Fido, algul vist
\emph{point}'i, ja käitasin seda üsna mitu aastat. 

\question{Miks sa seda tegid?}

Huvist kommunikatsiooni vastu.

\question{Kas sa mõtled kommunikatsiooni masinate või inimeste vahel?}

Mõlemat. See moment, kui täielikust infopuudusest saab järsku täielik 
infovabadus, on väga ergastav. Tänapäeva inimestel, kellel on internet olemas, ei 
kujuta ette, kuidas saab olla ilma, aga ilma oli väga pime.

Üks asi oli tehniline info, aga Fidoneti ja Useneti grupid
(UUCP meiliga koos toodi ka Useneti gruppe) olid ka muidu väga 
huvitavad. Sealsed diskussioonid olid väga 
informatiivsed. Suurem osa 
juttudest olid muidugi tehnilised, sest seal käisid tehnikud ehk need, kes 
said kanalile ligi.

\question{Kas too kollase kaabliga võrk hakkas tööle ka?}

Ikka, see töötas uhkelt. Novelli server käis veel 1992. aastal, kui ma 
sealt ära läksin. Inimesed said omavahel meilida ja ka välismaailmaga 
suhelda. Ainukene probleem oli see, et arvuteid, millel oli see 
Etherneti äge \emph{interface}, oli suhteliselt vähe, paar tükki kateedri 
peale vist suudeti tekitada.

\question{Kas Etherneti kaart oli defitsiit?}

Tol ajal oli kõik defitsiit, siis oli veel rublaaeg. Millise projekti 
raames see toodi, ei tea. Avo Ots\index[ppl]{Ots, Avo} tegi minu meelest 
doktoritöö selle kohta, kuidas ehitada arvutivõrku. See oli 
oluline kogemus, et toimuks järgmine samm. Pärast tehnikaülikooli 
töötasin lühikest aega Microlinkis\index{Microlink}, kus ma olingi 
arvutivõrkude installeerija ja ühtlasi .EXE\index{.EXE} kirjutaja.

\question{Miks sa sinna läksid?}

Ühel päeval astus uksest sisse Margus 
Kliimask\index[ppl]{Kliimask, Margus}, keda ma teadsin Rainer 
Nõlvaku\index[ppl]{Nõlvak, Rainer} kaudu, ja tegi ettepaneku hakata 
tegema ajakirja. Sellest sai .EXE.

\question{Miks ikkagi? Jälle kõlab suure muutusena, et ühel päeval tõmbasid kollast 
kaablit ja järgmisel päeval tegid ajakirja.}

Täpselt nii oligi. Ma arvan, et Rainer tahtis Microlinki promo teha. 
See võis olla suur motivaator, aga seda peab Rainerilt endalt küsima.

\question{Kust sul üldse tuli mõte, et ajakirja tegemine võiks huvi pakkuda?}

Tundus huvitav. Mul ei olnud siis rohkem kõrgeid eesmärke kui see, et elu oleks 
huvitav.

\question{See on tegelikult kõige kõrgem eesmärk, mis üldse saab olla.}

Algul oli jutt, et teeme ajakirja, ja siis selgus, 
et mul oleks ka uut töökohta vaja. Nii sattusingi korraks Microlinki\index{Microlink}. Olin seal aga loetud kuud, sest 
siis hakati tegema Eesti Forekspanka\index{Eesti Forekspank}\sidenote{Eesti 
Forekspank sündis 1992. aastal ja ühines 1995. aastal Raepangaga\index{Raepank} 
1995.}. Pangal olid oma sidevajadused ja mind kutsuti sinna tööle.

\question{Üheksakümnendate algus oli Eesti panganduses ju hull aeg!}

Jah, ja Forekspank oli sel ajal pisikene valuutavahetuskontor, mis opereeris 
rubla-dollari börsi.
See tegutses tolleaegses hulgifirmas Abestok\index{Abestok}. Selle ühes toas olid 
inimesed, kes otsustasid panga teha. Margus 
Kliimask\index[ppl]{Kliimask, Margus} oli nendega seotud, vist IT-poisi 
staatuses. Temaga läksimegi Rävala puiesteele, istusime koos 
ehitusjuhiga ühte tuppa, mille ühes nurgas hoidsid
ehitajad oma tööriistu, ja ehitasime panka.

\question{Kust tekkis mõte, et panga tegemiseks ei piisa kilekottidega 
sularaha edasi-tagasi lohistamisest?}

Need mehed, kes panga tegid, olid piisavalt targad, mõistmaks, et pank käib 
teistmoodi. Kui palju teistmoodi, sai alles siis selgeks, kui 
Inglismaalt osteti pangatarkvara ja konsultandid rääkisid, kuidas panka 
tehakse. Aga see ei olnud kohe esimesel aastal. Esimestel aastatel ehitasime, 
tõmbasime kaablit ja panime laua alla püsti serveri. Ühel ilusal päeval lükkas
Margus Kliimask\index[ppl]{Kliimask, Margus} kogemata varbaga 
toite välja ja pank jäi seisma. Aga mitte kauaks. 

Nii Rein Usin\index[ppl]{Usin, Rein}, Ivar Lukk\index[ppl]{Lukk, Ivar} kui ka Margus Kliimask\index[ppl]{Kliimask, Margus} olid 
visiooniga inimesed. See pidi olema suhteliselt algusaastatel -- BBSid ja 
Fidonet olid siis veel kuum teema --, kui Margus Kliimask ütles, et teeme 
modemipanga. Tal oli kindel mõte, et see peab olema Norton 
Commanderi\index{Norton Commander} F2 menüüs\sidenote{1986. aastal turule 
tulnud ja 1998. aastal viimase versiooni saanud Norton Commander oli 
ülipopulaarne failihaldur MS-DOSi platvormile. Ekraanil oli korraga kaks 
nimekirja faile ja käsurida, allservas nimekiri saadaolevatest 
klahvivajutusega käivitatavatest käskudest. Nii oli kasutajal ilma suurema 
koolituseta kohe selge, mida ja kuidas teha. Ohtralt kasutati F-klahve 
ja neist olulisemate funktsioonid on inimestel siiani peas (F3 -- faili sisu 
vaatamine, F5 -- faili kopeerimine).}. Kõik kasutasid Norton Commanderit 
ja kõigil oli see olemas, aga keegi ei ostnud, sest tol ajal tarkvara ei ostetud. 

\question{Jah, ma mäletan poes karpe, aga ei mäleta, et keegi oleks neid kunagi 
ostnud.}

Hämmastav oli see, kuidas mõtte väljakäimisest 
modemipanga \emph{launch}'ini läks umbes kaks kuud.

\question{Tegite kahe kuuga nullist modemipanga?}

See oli programm, mis oli mingil määral Norton Commanderiga integreeritud: 
läks sealt menüüst käima, nägi välja nagu Norton Commanderi 
osa, võimaldas makseid ette valmistada, kontoväljavõtteid ja panga teateid 
saada ning enda makseid panka saata.

\question{Ja teisel pool võttis mingi asi kõned vastu, suhtles panga 
tuumaga ja tegi arveldused ära?}

Just. Panga tuumaga suhtlemine oli üsna traagelniitidega asi, kuna selleks 
ajaks oli juba toodud Inglismaalt panga tarkvara, millel ei olnud ühtegi head 
liidest peale terminali.

\question{Ja siis tegite terminali emulaatori?}

Mina jah kirjutasin terminali emulaatori ja üks kolleeg kirjutas programmi, mis 
lükkas emulaatorist maksed pangasüsteemi, ning see toimis 
aastaid niimoodi, enne kui tekkisid tehnilised vahendid, et seda 
natukene viisakamalt teha. \emph{Launch} toimus 
tolle aja kohta suure pressikäraga: tehti korralik meediaüritus, imekenad 
Hansapanga\index{Hansapank} tüdrukud istusid ka seal ja tegid märkmeid. Ja läks mööda vaid
mõni kuu, kui Hansapangal tuli välja
Telehansa\index{Telehansa}.

\question{See kamp, kes tollal
BBSides suhtles, võis olla kokku paarsada inimest. Kust tulid
kliendid modemipangale?}

Kliendid jagunesid umbes pooleks. Forekspanga klientuurist arvestatav 
protsent oli Venemaal, sest suur 
raha oligi tol ajal Venemaal, aga ka Eesti klientuur ei olnud sugugi kehv. Pank 
müüs seda suhteliselt suure summa eest ja Eesti firmad 
ostsid. Käisin seda ise Tallinnas installeerimas. Küsimus ei olnud 
selles, et inimesed ei saanud tulla maalt linna pangaasju ajama, vaid nad 
lihtsalt ei tahtnud kontorist välja tulla. Pangas sai mugavalt ära käia 
laua tagant püsti tõusmata.

\question{Ja see kõik tasus ära, et hakata isegi arvutiga 
makseid ette valmistama?}

Sel ajal oli igas firmas raamatupidamiseks arvuti olemas ja raamatupidajate 
arvutites maksed olidki. Ilmselt mugavus ja aja kokkuhoid tõukasid
Eesti firmad sinnapoole.

\question{Kui palju seal telefoniliine küljes oli?}

Alustasime kahega ja lõpus oli vist kuus. Kuna 
sideseanss oli nii lühike, mahtus enamik sideseanssidest paari minuti 
sisse. Kõik pakiti kohapeal kokku ja saadeti ühe portsuna ära -- Fidonetist õpitud tehnoloogia. Alguses tegin mina kliendipoole ja 
Margus Kliimask\index[ppl]{Kliimask, Margus} kirjutas serveripoole. Hiljem kirjutasin serveripoole veidi paremaks, et see oleks paremini eskaleeritav.

\question{Mida see tähendab?}

Ühe masina taha sai panna mitu modemit.

\question{Kas sa oma BBSi hoidsid siis veel püsti?}

Minu meelest oli meil pangas ka BBS veel mõnda aega, 
Microlinkis\index{Microlink} oli kindlasti. Kuna Forekspank asus Rävala puiesteel, siis kohe, kui 
üheksakümnendate alguses tekkis internet, oli selge, et meil on ka 
seda vaja. Tõmbasime koos Andrus Aaslaiuga\index[ppl]{Aaslaid, 
Andrus} oma valgete käekestega mööda majakatuseid Forekspanga kõrvale KBFI\index{KBFI} majja, 
kus sündis Uninet\index{Uninet}, Etherneti kaabli.

\question{Te olite siis otse Unineti küljes?}

Otse Unineti küljes, olime ühed esimesed kliendid, kodukootud 
ruuteri softiga, mis läks flopi pealt käima. Mõlemas otsas oli üks 
arvutikast ja nii me ennast internetti panime. Muide, ükskord 
lõi meil sinna välk sisse.

\question{Mida te internetis tegite?}

Algul õppisime, mis see on. Ja pangas oli hädavajalik meilivahetus, et suhelda. Üks esimesi asju, mis pangas sai 
ehitatud, oli teleksi \emph{gateway} Pegasus Maili\index{Pegasus Mail}. 

\question{Misasi on teleks?}

Teleks oli viiekümneboodine\sidenote{\emph{Baud rate}, eesti keeles lihtsalt \emph{boodid}, 
näitab, mitu korda sekundis signaal liinil muutub andes indikatsiooni side kiirusest.}  telegraafisüsteem. Kahtlustan, et paljud pangad maailmas kasutavad seda endiselt. Suhtlus ei käi telefoniliini 
pidi, vaid selleks on eraldi teleksivõrk, mis toimib mööda telefonitraate 
hoopis teistsuguste signaalidega kui tavaline telefon.

\question{Kas see oli \emph{circuit switched}\sidenote{Ahelkommuteeritud. 
On ju ilus eestikeelne sõna?}, eks? Siis see vajas eraldi keskjaama?}

Jah. Põhimõtteliselt tuli ikkagi kõne teha ja ühendus püsti seada. 
See ehitati veel sel ajal, kui olid teletaibid -- klaviatuur ja 
paberirull.

\question{See \emph{gateway} ei saanud siis ju olla ainult tarkvaraline, vaid
oli ka riistvara vaja?}

Jah. Seal oli üks kast vahel, mis tegi sellest jadapordi. Esimese kasti tegi minu meelest
Küberneetika Instituudi\index{Küberneetika Instituut} majas üks Sass, Aleksander\index[ppl]{Reitsakas, 
Aleksander}.

See oli väga keeruline kast, tegin hiljem sellest peopesasuuruse 
versiooni flopikarpi.

\question{Mind hämmastab see, et sa ehitasid järjest keerulisemaid asju, aga kust sul tulid selleks teadmised, seda ei selgu.}

See on nagu Youtube'i videot vaadates -- tundub, et kõik asjad juhtuvad ise. 
Vahepeale mahtus siiski kuude kaupa õppimist, häkkimist ja katsetamist.

\question{Sul pidi hirmus kihu seda teha olema.}

Kindlasti, peaasi, et oli huvitav. 
Pangas töötades hakkas esimest korda ka kohusetunne vaevama, sest kui pank hommikul ei toiminud, olin ju mina paha.
Töötunde kulus kõvasti, aga üksiku inimesema ei olnud mul eriti muid kohustusi.

\question{Lisaks rääkisid muudkui teistega juttu BBSides.}

Panga ajal enam mitte, siis võttis töö kogu aja ära. Varem toimus jah BBSides suhtlus, aga kui tuli internet, võttis meilindus asja üle. Meiliga tuli kohe ka  \emph{gateway} 
kohe panga serverisse. Pank oli selles mõttes väga hästi kommunikeeruv.

\question{Legend räägib, et sina kirjutasid esimese eestikeelse klaviatuuri draiveri, 
on see tõsi?}

Nii ja naa. Rainer Nõlvak\index[ppl]{Nõlvak, 
Rainer} leidis esimesena, et klaviatuuril võiks eestikeelne \emph{layout} 
olla. Veel enne, kui infotehnoloogid jaole said, tellis Rainer eestikeelse 
klaviatuuri ära.  Nii et pärast, kui kehtestati  uus standard (EVS 8:1993),  
olid olemas klaviatuur ja oli kirja pandud standard. Lisaks klaviatuurile oli aga vaja ka standardile vastavat 
lokalisatsiooni. Eriti hull lugu oli Windowsi fontidega -- sel ajal oli olemas
Windows 3\index{Windows}. Ja siis korraldati konkurss, kus kõik lähenemised 
olid lubatud.

\question{Kes konkursi korraldas?}

Ma ei mäleta organisatsiooni nimesidenote{Tegemist oli Eesti Informaatikafondiga\index{Eesti Informaatikafond}, sellest sai hiljem Eesti Informaatikakeskus\index{Eesti Informaatikakeskus}, Riigi Infosüsteemi Ameti\index{Riigi Infosüsteemi Amet} eelkäia.}, aga see oli riiklik 
konkurss, mille auhind oli tolle aja kohta täitsa korralik, vist kakskümmend 
tuhat krooni. Olime selleks ajaks Raineriga juba natuke sel alal 
koostööd teinud -- Microlink pani enda klaviatuure müües kaasa draiveri, mis seda 
\emph{layout}'i toetas ka, nii et osa tööd oli juba tehtud. Kui konkurss 
välja kuulutati, ütles Margus Kliimask\index[ppl]{Kliimask, Margus}, visiooniga mees,
et teeme nii, nagu Microsoft teeb. Me \emph{reverse 
engineer}'isime kogu selle DOSi lokalisatsiooni ja klaviatuuri draiverid ning 
tegime installeerimisprogrammi, mis paigaldas 
standadkomponendid: \verb|KEYBOARD.SYS|i, \verb|COUNTRY.SYS|i ja muud
sellised asjad. Kuskilt õnnestus hankida soft, mis tegi Windowsi 
fonte, ja ma joonistasin fondid ka. See ei olnud küll kuigi hea soft, 
ei teinud TrueType'i \emph{hint}'ingut; \emph{kerning} vist 
on see teine, mis teeb fondid ilusaks, kui need väikseks muudad. Eesti 
fondid paistsid ekraanil karvased, aga me ei saanud sinna kahjuks midagi parata. Igal juhul
oli meie lähenemine teistega võrreldes nii palju parem, et võitsime konkursi.

\question{Kas pank läks konkursile osalema?}

Ei, ainult meie Margus Kliimaskiga\index[ppl]{Kliimask, Margus}. 
Meil oli pisike OÜ, koos pangaga tehtud ühisfirma Forex Communications modemipanga müümiseks. 
Selle firma alt osalesimegi. 

\question{Ja osalesite seepärast, et tundus huvitav?}

Sinna läksime ilmselt raha pärast ja võibolla ka 
Näitusepaviljonis toimunud joomingu pärast, mille seesama riiklik asutus piduliku sündmuse puhul 
korraldas.

\question{Kas sul sellepärast saigi panga aeg otsa, et pank sai valmis?}

Pigem pean olema tänulik pangajuhtidele, kes andsid meile 
hämmastavalt vabad käed igasugust tehnoloogiat katsetada ja uurida ning mõelda 
uusi asju. Tänu sellele oli Forekspank ka üks esimesi internetipanga tegijaid -- meil oli olemas internetiühendus ja me juba mõistsime, mis toimub. 

\question{Millega tollast internetipanka tehti?}

Forekspanga esimene internetipank oli minu meelest 
IISi\sidenote{1995. aastal turule toodud \emph{Internet Information 
Server (IIS)} oli Microsofti veebiserver, mis üritas (mõnevõrra tulutult) 
konkurentsi pakkuda tol ajal domineerinud Apache'i veebiserverile.} peal ja töötas
Windowsis\index{Windows}. 

\question{Eksootiline valik tolle aja kohta ...}

Oli küll imelik valik. Aga sel ajal olid meil juba arendus- ja 
hooldusmeeskonnad eraldi. Margus Kliimask\index[ppl]{Kliimask, Margus} oli 
arendusmeeskonnas. 

\question{Ehk te olite \emph{DevOpsist}\sidenote{Arendusmetoodika, kus tarkvara 
ehitamine ja selle edasine käitamine korraga nime kaotavad ehk omavahel 
lahutamatult kokku saavad.} astunud sammu tagasi?}

Panga käigushoidmine ongi natuke omapärane tegevus. Margus 
\index[ppl]{Kliimask, Margus} juhtis internetipanga arendust, tema meeskonnas
oli ka Pronto\index[ppl]{Pronto|see{Raja, Tanel}}\index[ppl]{Pronto}\sidenote{Vt
 lk \pageref{sisu:pronto}.} ja veel paar 
hakkajat selli. 

\question{Kas sina olid ka sellega seotud?}

Mina ei olnud internetipangaga peaaegu üldse seotud. Sel ajal oli
modemipank veel põhikanal, kuna internet oli siis vähestel. Forekspank oli juba üsna suureks kasvanud, 
hooldusmeeskonnas oli kümmekond inimest.

\question{See kõlab juba nagu terve organisatsioon, kahe telefoniliiniga ei saanud enam 
hakkama?}

Sel ajal tekkisid teised probleemid. 
Pangale ostetud tarkvara käis kummalise IBMi platvormi peal, mida aeg-ajalt 
tuli \emph{upgrade}'ida. Selle tarkvara jaoks oli COBOL uus keel. 
Tarkvara oli kirjutatud imelikus keeles nimega \emph{Report Generator Language}, mis 
oli pärit System/36\index{System/36}\sidenote{System/36 oli IBMi poolt 
1983. aastal turule toodud väike mitme kasutaja jaoks mõeldud mitmetegumiline 
server, mida programmeeriti peamiselt platvormipõhises RPG II\index{Report Program Generator} (\emph{Report Program Generator - RPG}) keeles.} ajast. Sellest keelest 
kumasid perfokaardid ikka veel kõvasti läbi.

\question{Vähe sellest, et teil oli visioon, aga raha pidi ju ka olema, et brittide juurde 
minna.}

Server maksis sel ajal meeletu raha. Algul ei olnud pangal jaksu õiget masinat osta, hangiti üks karm 
PC ja selle peal käis System/36 emulaator, millel jooksis 
panga tarkvara. Õnneks kasvasime sellest üsna ruttu välja. Pärast oli meil selline unikaalne platvorm nagu
AS/400\index{AS/400}\sidenote{AS/400, hiljem tuntud kui 
\enquote{System i}, oli IBMi keskmise suurusega serveriplatvorm, mis 
toodi turule 1988. aastal.}, mida ka korduvalt uuendati.

Ilmselt sai pank tarkvara ostes 
ka teadmise sellest, kuidas panka teha. See oli võibolla 
rohkem väärt.

\question{Teil oli Margusega juba siis kahe peale pisike OÜ, aga mõni 
veedab terve elu oma huvi üksnes akadeemilistes sfäärides rahuldades. Kust sul tekkis
arusaam ärist?}

Nagu ma mainisin, siis OÜ sündis modemipanka tehes ja pean jällegi kiitma 
tolleaegseid pangajuhte, kellega koos me ühisfirma lõime. Otseselt äritegemist kui sellist ei olnud: meie 
kirjutasime tarkvara ja inimesed maksid selle eest OÜ-le, pärast 
jagasime pangaga raha ära. Klassikalise äri mõistes ei pidanud meie midagi 
müüma, pank müüs. Muidugi tekkis ettekujutus näiteks
raamatupidamisest, aga erilist ärisoont see minus ei arendanud.
OÜ käigushoidmine mingit tähelepanu ei nõudnud, kogu fookus oli tehnoloogial.

\question{Tõnu Samuel\index[ppl]{Samuel, Tõnu}\sidenote{Vt lk \pageref{sisu:tonu}.}  rääkis mulle, et Mastsidenote{Ehk siis käesoleva loo kangelane.} oli see mees, kelle juurde sai minna riskantsete 
asjadega. Kui oli vaja emaplaadi peal vaibanoaga radu lahti kratsida 
ja sinna relee vahele panna, siis Tõnu teadis, mida teha, aga ei 
julgenud. Seevastu Mast julges.}

Ilmselt oli abiks raadiotehnika kateedri kool. Kui saad aru, 
mida teed, siis sa ei karda lõigata.

\question{Nii et sul sellist aukartust masina ees ei olnud?}

See kadus suhteliselt vara ära, kuna raadiotehnika kateedri Apple 
II\index{Apple II}s oli mitu laienduskaarti sees. Kui sellel oli
kaas peal, siis kuumenes üle, aga kaas ei olnud kunagi peal. Seal võis 
vabalt näppupidi sees sobrada ja mitte keegi ei öelnud, et sa ei tohi seda kivi 
välja võtta. Kõik oli pesades, kõike võis välja võtta. Kui katki läks, siis 
tuligi võtta. 

\question{Kas läks katki ka?}

Ikka läks, aga Apple II\index{Apple II} oli 
lihtsa loogika järgi ehitatud, Vene kivid läksid sinna asemele ja taktsagedus oli üks 
megaherts. Seda sai parandada ja see oli väga õpetlik. Ka 
esimese IBM PC\index{IBM PC}ga tulid kaasa (meil olid 
kõik juhendid olemas) BIOSi \emph{listing}'ud ja skeemid. Kõik olid 
standardtükid, kõike sai parandada ja parandatigi. 

\question{Mida sa pärast panka tegid?}

ITd ühele väikesele investeerimiskontorile. Kirjutasin 
Exceli Visual Basicus\index{Visual Basic} väärtpaberite 
kauplemise programmi. Tol ajal tehti paljusid asju Excelis, näiteks arvutati intressi. Tegin suured Exceli makrod, millega sai 
väärtpaberiportfelle hallata ja tehinguid jagada. 

\question{Kas jälle selle pärast, et oli huvitav?}

See oli rohkem vajaduspõhine. Meie enda investeerimiskontoril oli seda 
vaja ja ühe koopia müüsin maha ka. 

\question{Nii et tegelesid siiski ka müügiga?}

Ma ei tegelenud müügiga. Enamasti oli nii, et keegi tuli ja ütles, et tal 
oleks ka vaja. 

\question{Kui on väärt asi, siis lõpuks ikka tullakse.}

Jah, kui hind sobis, siis miks mitte.

\question{Sa oled BBSummeri\index{BBSummer} kuulsa grupipildi peal. Kas käisid tolle seltskonnaga läbi, kuigi töö võttis enamiku ajast ära?}

BBSummerid algasid siis, kui olin alles tehnikaülikoolis, ja neid ei olnud üldse palju. See grupipilt, mida sina vist 
mõtled\sidenote{Memcpy podcast'i kaanepildiks olev foto, kus on peal 
hämmastavalt paljude suurte asjade toonased või hilisemad algatajad.}, ei ole esimesest BBSummerist, vaid teisest või kolmandast, kus käisid ka FidoNeti tublid mehed Soomest. Seal 
pildil on üks habemega mees nimega Ron Dwight\index[ppl]{Dwight, Ron}, kes 
oli FidoNeti kunn Euroopas, regiooni pealik. Ron 
oli väga tore mees, ma olen tal isegi paar korda külas käinud ja tema juures Soomes 
ööbinud, kui piirid lahti läksid. Ja ma ei ole Eesti kambast ainukene, kes tal
külas käis. 
Soomlased, kes FidoNeti Soomes vedasid, olid tol ajal üldiselt väga toetavad. 
Sa oled teistega rääkinud, kuidas te Soome helistasite, ja keegi ei ole 
maininud, et tegelikult algusaegadel helistasid soomlased siia. Ei olnud nii, 
et ainult sealt oleks tõmmatud. Hiljem, kui BBSid ja firmad said siin jalad 
alla, saime "rinnapiima" otsast lahti, aga algusaegadel 
soomlased toetasid meid tublisti. 

\question{Kas puhtalt missioonitundest? Hõimuvelled ja nii?}

Ma ei tea, kui palju hõimuvendlus rolli mängis, pigem arusaam, et tehnoloogiat tuleb huvitatud inimestega jagada. 

Mul on nendest aegadest väga head mälestused ja sellepärast kutsusimegi neid ka BBSummeritele\index{BBSummer}. Ron käis minu meelest kahel. Igatahes oli
soomlasi esimestel BBSummeritel palju ja ma mäletan, kuidas nad olid selle grupipildi aegsel BBSummeril äärmiselt
hämmastunud sellest, et kõik võivad õlut juua ja et teisel päeval ei toimunud mingeid 
kaklusi!

BBSummeri korraldamise juures oli veel tore see, et korraldustasu
tagas söögi ja joogi kõigiks päevadeks. Ja õlut pidi kõigile jätkuma. Ühele BBSummerile toodi küll õlut Fanta tünnides, nii et 
õllel oli kerge Fanta mekk juures.

\question{Tundub, et sul on inimestega vedanud.}

Mul on jah sõpradega vedanud. Kui ma üksi elasin ja 
tehnikaülikoolis\index{Tallinna Tehnikaülikool} 
vabakutseline olin, siis suhtlesin väga paljudega. 
Hiljem võttis perekond nii palju aega ära, et kahjuks ei jõudnud enam kõigiga 
kontakti hoida.

\question{Aga kriitilisel hetkel olid nad olemas?}

Nad on siiamaani olemas. Näiteks 
Lõvi\index[ppl]{Lõvi} kohtasin ma umbes viis aastat tagasi Selveri 
parklas, nüüd käisin tal hiljuti tehnikaülikoolis külas.

\question{Ahti\index[ppl]{Heinla, Ahti}\sidenote{Vt lk \pageref{sisu:ahti}.}  ütles 
väga targasti, et seltskond noori inimesi sai
omavahel suheldes inimeseks koos Eesti riigiga. Kas sul on ka selline 
tunne?}

Jah, me olime kõik suhteliselt üheealised. Täpselt selles 
vanuses, kui oli huvi teha midagi uut ja selleks tekkis võimalus ning ka omavaheline klapp. Oli ka erandeid, näiteks Henn Ruukel\index[ppl]{Ruukel, Henn} 
oli esimesel BBSummeril selgelt alaealine, aga õlletünni juures passi ei 
küsitud.

\question{Mida sa praegu teed?}

Pean pausi. Aitan ülikoolil satelliiti\sidenote{Masti panusega satelliit lendas 
2020. aastal ka edukalt kosmosesse.} ehitada. 

\question{Sellepärast, et on huvitav?}

Sellepärast, et on huvitav. Kosmos on huvitav.

\question{Kosmos on suur ka, seal ei ole karta, et huvitavad asjad 
saavad otsa.}

Praegu käib sebimine enamjaolt Maale väga lähedal. Orbiidid, kuhu 
väikseid satelliite lastakse, on viie- kuni seitsmesaja kilomeetri kaugusel.

\question{Kas sul üldse on kunagi juhtunud, et järgmist huvitavat asja ei ole 
silmapiiril?}

Ei.

\question{Kuidas see sul on õnnestunud?}

Isegi kui päevatööl ei ole huvitav, siis mul 
kodus käib kogu aeg mõni projekt. Kui üks saab valmis või läheb 
sahtlisse (sinna läheb enamik, sest huvi kaob ära), on 
järgmine kohe laual. Sellist asja ei ole, et mul ei ole midagi teha.

\question{Kas sul sahtel juba täis ei saa?}

Saab. Jube täis on. 

\question{Mida sa siis teed?}

Viskan ära. Suur osa neist on ju eksperimendid. Võtan ära tükid, mis lähevad  
järgmise eksperimendi peale, ja ülejäänu on prügi. Teadmised jäävad alles.


\chapter{Kain Kalju}
\index[ppl]{Kalju, Kain}
\question{Kuidas sina arvutite juurde jõudsid?}
See oli umbes aastal 1990--1991, kui sõpradel tekkisid esimesed arvutid, 
olime kaksteist kuni neliteist aastat vanad. Ühel sõbral oli selline 
imelik asi nagu Texas Instruments TI-99\sidenote{Täpsemalt Texas Instruments 
TI-99/4\index{Texas Instruments TI-99/4}. Ärilistel ja arhitektuursetel 
põhjustel lühikese elueaga koduarvutite perekond. Oli koos samal 1979. aastal 
turule tulnud Atari 8bitiste arvutitega esimesi omataolisi, millel oli 
audio- ja videoülesanneteks eraldi protsessorid.}, see oli Commodore'i ja Apple 
II sarnane riistapuu selles mõttes, et see oli 16bitise protsessoriga ja 
\emph{boot}'is otse BASICusse\index{BASIC}. 

See arvuti oli telekaga ühendatud ja seal olid mõned primitiivsed mängud, nagu näiteks 
Space Invaders\index{Space Invaders}, ning 
loomulikult ka BASIC. Kogu programmi kood tuli kassetilindilt nagu tol 
ajal kombeks, flopisid polnud olemas. See oli minu esimene kokkupuude 
arvutiga, millel oli klaviatuur, kuhu sai sisestada programmi koodi ja 
kus katsetasime ka esimest korda BASICus ise programme teha, toksides 
neid ajakirjadest ja mõeldes ka ise välja. 

\question{Mis linnas see oli?}

Ma olen pärit Keilast\index{Keila} ja mul ei ole kunagi olnud 
spetsiaalset ligipääsu arvutitele mõnes teadusasutuses või koolis. Võrreldes mõne teisega oli minu ligipääs arvutitele suhteliselt piiratud.

\question{Kas sul reaalainete vastu oli huvi?}

Koolis käisin reaalkallakuga klassis. Meil oli 
gümnaasiumis\index{Keila Gümnaasium} väga vahva lend, peaaegu kõik poisid olid 
arvutihuvilised ja nii palju, kui olen nende elukäiku jälginud, on 
pea kõik mingitpidi arvutimaailmas tegevad.

\question{Kuidas see juhtus? Kas teil oli koolis nii korralik tase?}

Gümnaasiumi esimestes klassides (olime just 
läinud üle kaheteistkümne klassi süsteemile) olid kooli arvutiklassis 
Jukud\index{Juku}, mis meid loomulikult absoluutselt ei huvitanud, sest
seal oli Pascal\index{Pascal}, aga meil oli siis juba ligipääs PCdele.

\question{Juku oli omal ajal igavesti äge aparaat!}

Jah, aga Juku tuli hilisemas faasis, kui meil oli 
juba PC ligipääs olemas ja mul endalgi kodus PC. Minu 
suur arvutihuvi läkski lahti sellest hetkest, kui vanemad otsustasid mulle 
PC osta. Seda lugu peab natuke tagasi kerima: seesama sõber, 
kellel oli Texas Instrumentsi imepill, sai aasta hiljem monokroomekraaniga
286, mille ta isa tõi Ameerikast. Nad elasid
kortermaja esimesel korrusel ja PC oli raudkapis luku taga. Oli suur hirm, et 
keegi murrab sisse ja varastab arvuti ära. Aeg oli selline.

\question{Arvuti maksis tollal ju rohkem kui korter. Mõni ime, et PC 
kappi pandi!}

Mu vanematele käis see kohutavalt pinda, et ma ei viibinud üldse kodus, vaid olin
kogu aeg hilisööni sõbra juures külas. Millalgi üheksakümnendatel, 
vahetult enne Eesti krooni tulekut oli aeg, kui rubla devalveerus väga
kiiresti. Olen isa käest küsinud, kuidas see täpselt oli, ja ta on meenutanud, et 
tema sai millegipärast palka Ameerika dollarites ja ostis
kuskilt kooperatiivist dollarite eest
ühe 286. Hinnaklass oli umbes tuhat dollarit. See oli 
VGA ekraaniga, täiesti uus ja väga äge, kuigi 286 oli
tolleks ajaks ilmselt natuke \emph{outdated}, kuna siis oli juba 386 
ajastu.

\question{Võrreldes XTdega, mille abil Tartu Ülikoolis 
programmeerimist õpetati, oli see ikkagi väga kõva sõna. Mida sa selle arvutiga tegid?}

Nagu noor poiss ikka, tõenäoliselt mängisin, aga mind huvitas ka kõikvõimalik 
tarkvara.

Meenub tore lugu, kuidas umbes aasta pärast arvuti saamist käisime sama sõbraga 1993. aastal Ameerikas. 
Meil oli kolmene punt, kes me 
elasime üksteise lähedal, ja meil kõigil oli kodus kas isiklik või vanemate 
tööarvuti. Kord rongiga Tallinnast Keilasse sõites
hakkasime rääkima, et jube lahe oleks minna 
Ameerikasse. Ühel sõbral elas tädi seal ja ta oleks meid hea meelega vastu võtnud, ainult et
kuidas sinna saada. Minu isa töötas Muuga 
sadamas ja tuli jutuks, et põhimõtteliselt saaks ka 
laevaga minna. Olime siis
viie-kuueteistaastased. Rääkisin sellest kodus, kuidagi hakkas 
pall veerema ja ühel hetkel taotlesime juba USA saatkonnas viisat. 
Järgmisel hetkel oli isal kokku lepitud, et saame minna kaasreisijateks 
suurele Ameerika kaubalaevale, ning me sõitsimegi Muuga sadamast laevaga üle Atlandi ookeani 
New Orleansi. Seal pani laevakompanii meid lennukile ja 
edasi lendasime JFK lennuväljale New Yorki, kus sõbra tädi meid vastu 
võttis. Kusjuures me saime laeva peal palka, sest laevafirmale oli palju odavam 
vormistada meid töötajateks. Muidu oleks olnud vaja tasuda suuri 
kindlustusmakseid. Selles mõttes täiesti kreisi käik.

\question{Kaua te sõitsite sinna ja kas te midagi kasulikku ka laeva peal tegite või sõitsite lihtsalt kaasa?}

Laevasõit üle ookeani võttis umbes kaks nädalat. Midagi kasulikku me ei teinud, hängisime nii-öelda ohvitseride
alal. Meile küll näidati, kuidas laev töötab, aga me näiteks ei koristanud tekki ega teinud muud kasulikku. 
Võibolla heal juhul saime ülevaatlikku õpet mootoriruumist justnagu muuseumis. Loomulikult ei lastud meil midagi teha, võibolla avaookeanil saime korraks rooli keerata ja nii-öelda
laeva juhtida.

\question{Millega te tagasi tulite?}

Tagasi tulime lennukiga. Aga miks ma sellest üldse räägin, on see, et 
Ameerika pinnale astudes oli meil päris palju raha, kuna saime laevast 
palka, umbes tuhat viissada dollarit, mis oli tolle aja kohta üüratu 
summa. Mina kulutasin raha ära loomulikult arvutipoes -- tõin endale Ameerikast elu ühe kõige tähtsama riistapuu, 
milleks oli modem. Pärast seda läks elu lahti. 

See oli 2400boodine modem, tüüpi ei mäleta. Lisaks 
tõin Sound Blaster 16\index{Sound Blaster} helikaardi, mis oli tollal täiesti 
tipp\sidenote{Sound Blaster oli Singapuri firma Creative 
Technology (tuntud USAs kui Creative Labs) helikaartide perekond. Need kaardid 
olid PC-maailmas \emph{de facto} standardiks, kuni Windows 95 vastavad liidesed 
standardiseeris ja PC audio muutus tarbeesemeks.}. See oli just paar kuud varem välja tulnud. 

Üks asi, mille ma hiljem avastasin ja mis levisid BBSides, olid helimoodulid. 
Mul oli neid päris palju, kogusin neid mõnda aega. Ilmselt toona 
tegid seda paljud. Need on helifailid, mida tollal pandi kokku
Amiga arvutites ja mis koosnesid sämplitest. Oli umbes kaheksa \emph{track}'i, kuhu sai miksida sämpleid niimoodi 
kokku, et tekkis muusika. 

\question{Ja need liikusid BBSides?}

Jah. Loomulikult sai üritatud neid ka ise teha, aga mul erilist muusikatausta ei olnud, nii et sellest ei tulnud midagi välja.

\question{Kas Ameerikast tagasi tulles panid kohe BBSi püsti?}

Ei, hakkasin siis alles avastama BBSi 
maailma. Vanu asju üle vaadates selgus, et üks mu lemmik BBS oli Dark 
Corner\index{Dark Corner}, mida vedas Priit Kasesalu\index[ppl]{Kasesalu, 
Priit}. Esmalt loomulikult üritasin alla laadida kõike, mida sain. 
Kõik oli ju puhas kuld, kogu tarkvara. 
Tol ajal veel
Kadaka turg\index{Kadaka turg}\sidenote{Aastal 1991 avatud ja 2002. aastal 
kaubanduskeskusega asendatud, Mustamäel asunud turg oli küllaltki metsik 
müügikeskkond, kust oli võimalik hankida kõike alates karvamütsidest ja 
Nõukogude aurahadest kuni kõikvõimaliku piraatkaubani. Sisuliselt oli tegu 
endise Nõukogude Liidu territooriumil toiminud varimajanduse väljundiga 
Eestisse. Turg oli turistide seas hinnas, parematel aegadel käisid sinna 
Tallinna sadamast eribussid.}, kus müüdi piraattarkvara. Nii et
väga palju sain ka sealt. Minu mäletamist mööda BBSides 
otseselt piraattarkvara väljas ei olnud, pigem 
häkkimise stiilis tarkvara.

\question{Windowsi sealt vist keegi endale ei laadinud?}

Jah, just, selliseid asju otse faililistides ei olnud, need olid taha 
nurkadesse ära peidetud. Aga seda mäletan küll, et meil oli kodus 
telefoniliin ja minutitasu ei olnud või siis 
oli see väike. Igal juhul oli meie koduliin enam-vähem
ööpäev ringi kinni, sisse ei olnud võimalik helistada, sest minu 
arvuti helistas ja laadis kogu aeg midagi alla.

\question{Kuidas te alguses rea peale saite? Kuidas sa teada said, mis numbri 
peale helistada?}

Võimalik, et see teadmine tuli .EXE ajakirjast\index{.EXE}. Aga kui oled ühte BBSi juba sisse pääsenud, siis avaneb kogu 
maailm. Üks teema, mida BBS levitas, oli teiste BBSide 
aadressidega failid. Ühel hetkel pani Priit Kasesalu\index[ppl]{Kasesalu, Priit} 
kogu oma BBSi viimase versiooni veebi üles. Laadisin selle alla ja 
avastasin selle kettalt igasugu huvitavaid asju. 

\question{Mida seal leidus?}

Kõikvõimalikke häkkimisvahendeid, C-programmide näiteid, raamatuid 
nagu \enquote{Terrorist Handbook}\sidenote[][-4cm]{Ilmselt peab Kain silmas William 
Powelli raamatut \emph{The Anarchist Cookbook}. Vietnami sõja vastaste 
protestide laineharjal 1971. aastal USAs ilmunud (ja mitmel pool keelatud 
olnud) raamat sisaldas kõikvõimalikku vastandkultuuriga seotud sisu 
termiidi ja LSD valmistamise õpetustest kuni telefonisüsteemide 
murdmise juhisteni. Raamat levis tekstifailina laialt ülikoolide serverite ja FidoNeti 
kaudu ning seda täiendati pidevalt; eriti kuulsad on anonüümse 
autori \enquote{\emph{The Jolly Rogeri}} täiendused.}. Igasugune 
selline kraam, mis pakkus noortele inimestele põnevust.

\question{Tuleme veel kord sinu arvutihuvi alguse juurde. Kas sa olid pigem 
seda tüüpi mees, kes mängis arvutiga, võrgutas arvutit või programmeeris 
arvutiga?}

Tagantjärele mõeldes on olnud mitu ajajärku. Koduse 286 ja BBSide ajal üritasin pigem 
sisse krahmata kõike, mida nägin. Leidus ka 
arvutimänge, aga ma ei mäleta, et oleksin väga palju mänginud. 
Kui mul endal veel arvutit ei olnud, siis sõbra juures mängisime loomulikult, 
mitte ei programmeerinud. Hiljem jäi mängimine tagaplaanile ja püüdsin
aru saada, kuidas arvuti töötab. Näiteks üks teema, mis mind 
kohutavalt paelus, olid viirused. Mul oli alati kõige viimane viirusetõrje 
tarkvara. Mul oli selleks hetkeks juba mitu kõvaketast ehk 
võimalus katsetada, mida viirused teevad. BBSides levitati ka nii-öelda 
viirusekollektsioone ja ma uurisin, kuidas viirus 
põhimõtteliselt töötab. 

Järgmine ajastu tuli siis, kui avastasin enda jaoks 
Linuxi\index{Linux}, samal ajal tekkis ka internet. Gümnaasiumi 
kaheteistkümnendas klassis sattusin tööle Riigi Elektriside 
Inspektsiooni\index{Riigi Elektriside Inspektsioon|see{Tehnilise Järelevalve 
Amet}}, mis on täna Tehnilise Järelevalve Amet\index{Tehnilise Järelevalve 
Amet}. Sattusin patsiga poisiks, kuigi patsi pole mul
kunagi olnud. Olin tavaline arvutipoiss, seadistasin arvuteid.

\question{Kuidas sa kooli kõrvalt sinna sattusid?}

Seesama sõber töötas Pennus\index{Pennu} ja temalt kuulsin,
et otsitakse arvutitüüpi, kes oskaks arvutitega midagi teha. Läksin 
kohale ja mind võeti poole kohaga tööle.

\question{Kas teil klassist töötasid mitmed keskkooli ajal?}

Jah, üks klassivend töötas näiteks Keila linnavalitsuses. Ta oli juba siis kõva programmeerija, kinkis mulle 
mu esimese programmeerimisraamatu \enquote{C Programming Language}\index{The C 
Programming Language}, autoriteks Brian Kernighan ja Dennis Ritchie.

\question{See on seesama salapärane väljaanne\sidenote{\phantomsection\label{sisu:richie_vene}Kainil oli raamat 
jutuajamisel kaasas, selles puudus igasugune märge väljaandja ning trükkimisaja ja -koha kohta. Raamat oli 
korralikult köidetud ja kopeeris isegi värvilist kaanekujundust täpselt. 
Paigas olid ka sellised detailid nagu indeksis mõiste \enquote{recursion}, mis viitas (nagu ka mõiste sisu nõuab) 
tagasi mõistele endale. Mart Palmas\index[ppl]{Palmas, Mart} mäletab, et raamatut 
olla trükitud Novosibirskis. }, mis minulgi oli.}

Just, Amazonis on täpselt seesama raamat müügil. 
See oli mu esimene programmeerimisraamat, aga leidis kasutamist ka
aastaid hiljem, kui tegin netit\index{neti.ee} ja mul tekkis praktiline 
vajadus programmeerida suurema jõudlusega otsingusüsteemi.

\question{Kuidas teil ikkagi juhtus olema selline klass, 
kus mitmed töötasid-programmeerisid juba keskkooliajal?}

Võibolla just sel ajal arvutid ilmusidki rohkem koju ja 
kontorisse ning oli tohutu puudus oskusteabest. Vanemad 
inimesed ehk ei julgenud arvuteid veel kasutada, samal ajal kui noored julgesid nendega 
igasuguseid asju teha.

\question{Igas keskkooliklassis ei olnud nii, et neli-viis 
poissi töötasid arvutispetsialistidena. Miks teil oli?}

Ma ei oska seda tagantjärele öelda. Küll aga mäletan sellist huvitavat seika, et 1995. aastal oli meil kaheteistkümnendas klassis
arvutieksam, mis aga ei seisnenud 
programmeerimises, vaid me seadistasime koolile
arvutiklassi. Kool sai Tiigrihüppe kaudu peaaegu 
klassitäie arvuteid ja siis R-klassi\sidenote{Reaalklass.} poiste ülesanne 
oli võrgutada klass 
füüsiliselt Etherneti kaabliga ning installeerida arvutid ja 
võrguserver, milleks oli Linuxi server. Server jäi minu peale, 
kuna ma olin tollel hetkel kõige suurem Linuxi\index{Linux} käpp võrreldes 
teiste poistega.

\question{Linux ei olnud selleks ajaks ju kuigi vana, kuidas sa selle otsa 
komistasid?}

Linuxi otsa komistasin siis, kui töötasin Riigi Elektriside 
Inspektsioonis\index{Riigi Elektriside Inspektsioon}. Kui ma sinna 
läksin, siis seal veel internetti ei olnud, aga tekkis paar kuud hiljem, 1994. aasta lõpus. 
Inspektsioon asus aadressil Ädala 4d, mis on ka 
legendaarne internetihoone. Meie allkorrusel oli 
Valitsusside\index{Valitsusside}, kus toimetas Taavi Talvik\index[ppl]{Talvik, 
Taavi}. Taavi andis Riigi Elektriside Inspektsioonile juhtmeotsa kätte, 
milleks oli kümnemegabitine koaksiaalkaabel, ja ütles, et palun, siin on 
internet. See koaksiaalkaabel sai veetud kõikidesse ruumidesse, ei mingeid 
\emph{hub}'e ega tähttopoloogiat.

Siis avastasingi enda jaoks interneti. Koolis loomulikult poistele rääkisin, et 
FidoNet on vana ja aeglane jama, toimib üle modemi, aga meil on 
üks palju uuem ja huvitavam asi. Kusjuures Valitsussidest edasi olid kanalid 
üsna kiired. Mäletan, et Tartu Ülikooli FTP-serverist sai kahemegabitise 
kiirusega faile alla laadida, see oli meeletu kiirus. Välislink oli 
loomulikult kuskil 64 või 128 kilobitti. 

\question{Mida Tartu Ülikoolist tõmmata oli?}

Seda ma täpselt ei mäleta, aga ju midagi oli, sest mul on väga selgelt 
meeles kadri.ut.ee\index{kadri.ut.ee}\sidenote{Tartu Ülikooli masinad kadri.ut.ee ja madli.ut.ee said Toomas Soome\index[ppl]{Soome, Toomas} andmetel nimed Otto Telleri\index[ppl]{Teller, Otto} tütarde järgi.} FTP-server. 

FidoNet oli selles mõttes tohutu kullaauk, et avas kõik oma 
\emph{echo}-kanalid. Internet aga avas meililistid ja ühes listis ma 
lugesin, et Anto Veldre\index[ppl]{Veldre, Anto} teeb 43. 
Keskkoolis\index{Tallinna 43. Keskkool} 
sissejuhatavaid kursuseid. Toona ilmus ka ajakiri .EXE\index{.EXE}, 
kuhu Anto artikleid kirjutas. Ma ei mäleta, kumb kummale täpselt eelnes, aga 
igatahes ühel hetkel olin 43. keskkoolis, et 
\enquote{siin ma olen ja ma tahan teadmisi saada}. Seal tegutsesid  
sellised legendaarsed koolipoisid nagu Indrek Mandre\index[ppl]{Mandre, 
Indrek} ja Heno Ivanov\index[ppl]{Ivanov, Heno}. Tagasi tulin  
juba kuue Slackware\index{Slackware}'i distributsiooni installeerimisflopiga. Installeerimisprotsess käis flopi kaupa. 

\question{Kas lisaks kõigele muule jäi Anto peale ka Linuxi-pisiku 
levitamine Eestis?}

Tal oli väga suur roll selles, et Linux 
Eestis käima läks. Igal juhul mina sain küll selle pisiku. Kuna olin tolleks 
hetkeks juba mõnda aega Elektriside Inspektsioonis\index{Riigi Elektriside 
Inspektsioon} töötanud ja ka palka saanud, oli mul päris korralik 
\enquote{taskuraha}. Müüsin oma 286 FidoNetis maha 
(FidoNetis käis ka suur riistvaraga hangeldamine) ja ehitasin endale uue arvuti,
486, kusjuures see ei olnud mitte lihtsalt 486, vaid 486DX4 
100 MHz\sidenote{Inteli nomenklatuuris oli DX-tähistusega protsessorite 
kiibil eraldi matemaatika kaasprotsessor, mis andis märgatava
jõudlusvõidu.} -- absoluutne tipp. 

See oli kõige kõvem 486, mis üldse kunagi tehti. Sel ajal oli mul juba \emph{node} registreeritud. Olin varem saanud Dark Corneri 
BBSist\index{Dark Corner} esimese FidoNeti \emph{point}'i, kust 
pääsesin ligi FidoNeti uudisekanalitele, aga ühel hetkel tundus, et 
\emph{node} oleks ägedam. Kirjutasin Tarmo Mamersile\index[ppl]{Mamers, 
Tarmo} (tema oli Eesti regiooni \emph{manager}, kes jagas aadresse), 
kas oleks võimalik registreerida \emph{node} number kuuskümmend kuus, ja 
Tarmo vastas, et \enquote{tehtud}.

Millalgi seadistasin Elektriside Inspektsioonis 
Linuxi\index{Linux} serveri, sest meil on praktiline vajadus kasutada
printerit, faksi ja faile. Linuxi server jagas 
faile üle Samba teenuse ja võttis vastu fakse. Mul õnnestus ka enda 
FidoNeti \emph{node} samasse serverisse sokutada. Kui muidu töötas FidoNeti 
tarkvara MS-DOSi peal, siis oli ka alternatiiv Unixitele 
Ifmaili\index{Ifmail}-nimelise programmi näol.

\question{Miks riigiametis üldse internetti vaja 
oli? Kas see oli puhtalt sinu huvi või tehti seal sellega midagi kasulikku ka?}

Jah, praktiline vajadus interneti järele oli olemas, sest 
inspektsioon\index{Riigi Elektriside Inspektsioon} tegi koostööd 
ITUga\sidenote{\emph{Rahvusvaheline Telekommunikatsiooniliit.}}, kes 
juhib sageduste jaotust, protokolle ja kõike muud sellist. 
Inspektsioonil oli ITUga tihe kirjavahetus, ilmselt meili teel. Ma 
ei suuda meenutada, kuidas meilivahetus enne kaabliga internetti käis, aga 
pärast oli seesama Linuxi masin ka loomulikult meiliserveriks. Meil
tekkis oma domeen rei.ee ja Linuxi server hakkas rei.ee kirju vastu 
võtma ning ka mina sain endale esimese isikliku ülilühikese meiliaadressi, mis 
oli tollal ülikõva -- kain@rei.ee\sidenote{Lühikesed meili- ja muud aadressid 
olid staatusesümbolid, mis näitasid kuulumist kas serveriadministraatorite 
kõrgesse kasti või neile väga lähedasse ringkonda.}.

\question{Nii et sa avastasid ennast suhteliselt õrnas eas Linuxi ruuduna 
riigiasutuses?}

Just. Ja kui külastasin Anto 
Veldre\index[ppl]{Veldre, Anto} arvutiklassi 43. 
keskkoolis\index{Tallinna 43. Keskkool}, jäi mulle sealt üks asi elu 
lõpuni meelde: kuidas kõik need noored tüübid, kes seal 
siil.edu.ee\index{siil.edu.ee}-nimelise SCO\index{SCO UNIX} masina 
taga istusid, olid tohutult kõvad häkkerid. Nad demonstreerisid,
kuidas suudavad \emph{exploit}'ida Tartu 
Ülikoolis olevaid masinaid\phantomsection\label{sisu!ylikooli_root} ja sealseid professoreid jälgida. See 
avaldas mulle nii suurt muljet, et mind hakkas lisaks 
viiruseteemale huvitama arvutiturvalisus.

Me olime kõik ühise Etherneti kaabli peal. Räägin sellest esimest korda avalikult, et ma 
\emph{sniff}'isin loomulikult ka meie võrgus, mida 
Valitsusside\index{Valitsusside} insenerid seal tegid. Sama kaabli otsas oli kaks ametit: Riigi Elektriside 
Inspektsioon\index{Riigi Elektriside Inspektsioon} ja 
\emph{Valitsusside}. Ja kui Valitsusside insenerid käisid oma ruutereid või 
keskjaamu üle Telneti konfigureerimas, siis levis liiklus lahtise 
tekstina võrgus. Nende tegevust oli päris huvitav jälgida. 
Loomulikult ei kasutanud ma seda kunagi pahatahtlikult ära.

\question{Eks see seik iseloomustab suurepäraselt toonast aega. Kui 
praegu kasutaks keegi lahtise traadi peal lahtist kanalit, korraldataks 
poole tunniga mingi jama.}

Jah, ma arvan ka. Siis oli kogu võrguvärk 
niivõrd ebaturvaline, et niipea kui Soomest keegi härrasmees\sidenote{Tatu 
Ylönen, Helsingi Tehnoloogiaülikooli teadlane} tegi 
\emph{secure shell}'i esimese versiooni, hakkasin seda kasutama kohe, kui teada sain. 

Kui nüüd Keila Gümnaasiumi\index{Keila Gümnaasium} juurde tagasi tulla, siis 
pärast kooli lõpetamist jäin ma seal edasi 
administreerima kooli serverit. Nagu tol ajal ikka, pidid 
kõikidel Unixi masinatel olema ilusad nimed. Kodus sellest rääkides pakkus 
isa välja, et Kratt oleks hea nimi. 
Vaatasin hiljuti nimeserverist järele, et Keila Gümnaasiumi serveri 
nimi on siiamaani kratt.keila.edu.ee\index{kratt.keila.edu.ee}.

\question{Hakka siis nime tagantjärele muutma \ldots Loodetavasti riistvara ei ole 
päris seesama?}

Riistvara ei ole kindlasti sama, sest seda koolimaja füüsiliselt enam alles 
ei ole. Keilas on nüüd uus koolimaja, kus mu enda lapsed käivad, sest ma elan 
siiamaani Keilas. Aga serveri aadress on sama.

\question{Seepärast ongi asjade nimetamine oluline, et nimed võivad pikalt 
kesta.}

Just. FidoNeti ajast on veel üks huvitav asjaolu osutunud 
hiljem väga kasulikuks. Nimelt töötasid modemid
AT-käsustikuga\index{AT-käsustik}\sidenote{Hayesi käsustik, tuntud ka kui
AT-käsustik, on käsukeel, mille Dennis Hayes lõi 1981. 
aastal omanimelise ettevõtte 300-boodise modemi Smartmodem juhtimiseks.}, mis 
oli selles mõttes universaalne, et seda kasutati hiljem 
erinevates muudes rakendustes. BBSidesse sissehelistamine toimus loomulikult
lihtsa terminaliga ehk pidid nagu häkker käsustikku teadma. Enne 
helistamist pidi sisestama ATDT, telefoninumbri ja nii edasi, võibolla ka
seadistama protokolli. Tolleaegsed inimesed teavad täpselt, 
missuguse protokolli heli on kuulda memcpy podcast'i avakõllis. 

\question{Kas BBSil oli kliendisoft ka?}

Ei olnud. Helistada tuli terminaliga, ainult FidoNetil oli kliendisoft 
nimega FrontDoor, mis helistas, ja teine soft, mis pakkis kokku FidoNeti 
\emph{echo}'d ja saatis selle paki edasi. BBSil kliendisofit ei olnud, tuli minna Telnetiga külge ja seal 
edasi tegutseda.

\question{Oleks ju olnud loogiline, et 
keegi oleks teinud BBSide ette, näiteks \emph{cache}'i jaoks 
mingi tarkvara.}

Jah, kui vaadata, mis toimus Ameerikas, kus olid 
\emph{online service provider}'id, nagu AOL ja 
CompuServe\index{CompuServe}\sidenote{Internetieelsel ajal domineerisid USA 
turul agressiivsete turunduskampaaniatega (ühel hetkel oli pool \emph{kõigist} 
toodetud CDdest AOLi logoga) teenusepakkujad, kes pakkusid kummalist segu 
BBSi-laadsetest ja internetiteenustest. Neist suurimad olid CompuServe, Prodigy 
ja America Online.}, siis neil oli tarkvara olemas. Mäletan, et 
kui olin USAst modemi ostnud, siis noorte poistena tahtsime seda loomulikult 
proovida. Kujutad ette, me keerasime kruvikeerajaga lahti 
ühe suure soliidse arvuti, vist Computer 2000\sidenote{Computer 2000 
oli küll ka siinmail tegutsenud arvutiäri, kuid ilmselt peab Kain silmas Gateway 
2000 nimelist ettevõtmist, mis tootis sama nime all personaalarvuteid.}, mis oli 
tol ajal väga kõva valge PC bränd. Sõbra tädimehel oli 
väike arhitektuuribüroo ning meil oli julgust 
omavoliliselt kruvikeerajaga lahti keerata üks nende suur \emph{tower} ja 
selle sees proovida seda sisemist modemit. Modemiga oli kaasas kas 
CompuServe'i või mõne muu sarnase teenuse CD-plaat või flopi, ja siis sai helistatud Ameerika BBSi. 

\question{Kui sa BBSides ringi kolasid, kas sulle jäi midagi muud peale tarkvara 
ka silma? Sa mainisid raamatuid ja MODe.}

Raamatud mind siis eriti ei köitnud, BBSidest laadisin ikkagi 
peaasjalikult tarkvara ja muusika MODe. Aga kogu infovoog 
tuli FidoNetist, see oli minu jaoks kullaauk. Nagu varem 
mainisin, ei olnud mul ligipääsu teadusasutustesse ja
ülikoolidesse ega ka mentorit. Meil oli kamp 
poisse, kes omavahel infot vahetasid, ja kõik käis katse-eksituse meetodil.

\question{Hea, et te selle kambaga paha peale ei läinud. Noored 
poisid, tont teab, mida oleks võinud teha.}

Ju siis olime piisavalt mõistlikud. Sellest ajast saadik on mul 
ise õppimise oskus. Võibolla see sai ka saatuslikuks, miks ma ei suutnud
tehnikaülikoolis kaua õppida, ainult ühe aasta nagu
paljud teisedki tollal.

Peale gümnaasiumi läksin tehnikaülikooli informaatikasse\index{Tallinna 
Tehnikaülikool!Informaatika}, aga kuna ma juba ka töötasin, siis tekkis
igasuguseid huvipakkuvaid projekte. Mina eeldasin, et 
saan hakata ülikoolis programmeerimist ja muid huvitavaid asju 
õppima, aga tuli välja, et kõigepealt tuli läbida füüsika ja 
matemaatika. Mul oli matemaatikast natuke kopp ees, kuna 
meil oli gümnaasiumis väga püüdlik matemaatikaõpetaja ja tegelesime 
matemaatikaga põhjalikult, nii et ülikooli sissesaamise probleemi 
ei olnud -- matemaatikaeksamist lihtsalt 
lendasime läbi.

Ja nii see ülikool järgmisel aastal pooleli jäi.

\question{Kuidas sul kaitseväega lood on?}

Seejärel tuligi kaitsevägi\index{Kaitsevägi}. Kui ülikoolis ei õpi, siis varem või 
hiljem leitakse sind üles. Kaitseväkke läksin 1997. aasta suvel ehk 
olin siis juba aasta otsa Netit teinud. 

Ahjaa, et kuidas ma sinna sattusin. Töö Elektriside Inspektsioonis\index{Riigi 
Elektriside Inspektsioon} hakkas pisut ära tüütama, tahtsin 
edasi areneda ja kuhugi huvitavasse kohta tööle 
minna. Mul tekkis soov kindla peale töötada arvutifirmas, et saada arvutitele väga 
lähedale.

Vanu \emph{backup}'e läbi kammides jäi silma Helmes\index{Helmes} ja ma isegi kandideerisin sinna, aga ei saanud. Õnneks, mõtlen ma nüüd tagantjärele. Keskkooli ja ülikooli vahelisel suvel töötasin poolteist kuud 
Tõnu Samueli\index[ppl]{Samuel, Tõnu} IT-firmas nimega Eramees\index{Eramees} 
ja maandusin samale kohale, kust oli just lahkunud Pronto\index[ppl]{Pronto}. 
Tõnu ütles mulle, et Pronto müüs Gravis 
Ultrasoundi\sidenote{Üheksakümnendatel väga populaarsed helikaardid, mis 
esimesena omataoliste hulgas suutsid toimetada pärisinstrumentide 
sämplingutega.} kaarte ja et hakkaksin ise sellega tegelema. Aga ma olin 
noor koolipoiss ega teadnud kaubandusest mitte midagi. Vaevalt
minust seal ettevõttes muud erilist kasu oli, kui et olin nii-öelda patsiga poiss. 

\question{Päris mitmed inimesed on ühel hetkel tegelenud müügitööga ja üldse mitte 
halvasti.}

Eramehest on mul veel üks asi eredalt meeles. Tõnu BBS oli tal 
kontoris, mis asus Eesti Talleksi majas, Mustamäe tee 1, kui ma ei 
eksi\sidenote[][-8mm]{Siiski Mustamäe tee 4.}. Ja see BBS kujutas endast aknalaual laiali laotatud arvutijuppe: seal 
oli USR Courieri\index{US Robotics!Courier} modem\sidenote[][-8mm]{US Roboticsi 
 Courier tooteliin oli oma töökindluse ja suurte kiiruste tõttu 
BBSide ja varaste internetipakkujate lemmik, ka Eestis.}, emaplaat, toiteplokk 
ning hunnik juppe ja juhtmeid. See siis oligi Tõnu BBS või \emph{node}.

Pärast Erameest kandideerisin Estpak Datasse\index{Estpak Data}, sest mulle 
tundus, et ISP on tegelikult veel huvitavam asi, ja nad 
tegelesid internetiga.

\question{Kas Estpak oli tol ajal juba Eesti Telefoni oma või veel eraldi?}

See oli eraldi. Kui ma õigesti mäletan, siis Estpak Data omanik oli 
Eesti Telekom\sidenote[][-2.8cm]{Eesti Telekom ehk pika nimega Riigiettevõte Eesti 
Telekommunikatsioonid oli Teede- ja Sideministeeriumi haldusalas töötav 
\emph{holding}-ettevõte, mis valdas Eesti Telefoni, Eesti Mobiiltelefoni, Eesti 
Kaugotsingu, EsData, Estpak Data ja TeleMedia aktsiaid. Hiljem viidi ettevõte 
börsile ja ainuomanikuks sai Telia.}, mitte Eesti Telefon. See oli Eesti Telefonist täiesti eraldiseisev ettevõte. Huvitaval kombel oli kellelgi 
tulnud idee edendada veebi 
virtuaalhostimist. Keegi oli välja mõelnud neti.ee\index{neti.ee}-nimelise 
domeeni, mille alt üritati müüa traditsioonilist 
veebihostingut. Tollal see veel traditsiooniline ei olnud, aga 
tänapäeva mõistes küll. Estpak Data palkas mu veebihalduriks,
kes pidi hoolitsema veebi hostinguserveri ja -teenuse eest. Muu seas tekkis 
neil mõte, et kuidas veebihostingu äri ikka muudmoodi 
edendada, kui et on vaja kataloogi. Inimesed peavad ju 
need veebilehed, mida kliendid sinna panevad, üles leidma.

\question{Kas tol ajal oli Meediamaa juba olemas?}

Meediamaa\index{Meediamaa} startis umbes samal ajal. Enne seda oli olemas 
Eesti veebisaitide nimekiri, mis oli nlibi ehk 
Rahvusraamatukogu\index{Rahvusraamatukogu} domeenis, kus tegutses Toomas 
Mölder\index[ppl]{Mölder, Toomas}. Ilmselt kolis tema
selle nimekirja Meediamaasse ja sealt www.ee\index{www.ee}-sse. Kuna 
Meediamaa üks tegelane oli Tarvi Martens\index[ppl]{Martens, Tarvi}, siis neil 
õnnestus EENetilt\index{EENet} välja meelitada domeen nimega 
www.ee\sidenote{Alates oma asutamisest 1993. aastal kuni 2013. aastani oli 
EENet .ee domeeni registripidaja ja rakendas mitmeid suhteliselt rangeid 
reegleid. Näiteks oli domeeni registreerimine küll tasuta, kuid ühel 
organisatsioonil tohtis olla vaid üks domeen.}. Ma arvan, et mitte kellelegi 
teisele kui Tarvile ei oleks sellist domeeni elu sees välja antud.

\question{Kas sa kataloogi tegid algul käsitsi?}

Jah, alguses käsitsi, see oligi väga 
algeline ja puine. Asi hakkas lendama siis, kui kutsusin appi Jaanus Vainu\index[ppl]{Vainu, Jaanus}, kellega tutvusime 
Riigi Elektriside Inspektsioonis\index{Riigi Elektriside Inspektsioon}. 
Jaanus on ka omamoodi huvitav tegelane. Inspektsioonis mõtles tema 
välja kogu meie FM 108 sageduse plaani ehk kõik Eesti raadiojaamade 
sagedusnumbrid on tema tehtud. Nõukogude ajal oli meil teistsugune FM 
sagedusala, et takistada raadiost 
välismaiste raadiojaamade kuulamist. Eesti Vabariigi alguses 
koliti lääne sagedustele üle. Jaanus oli üks nendest, kes käis mööda Eestit 
mõõtmas ja tegi sagedusplaani. Tal joonistas väga detailselt Corel Draw's\index{Corel 
Draw} kõik sagedusringid Eesti kaardile. Eesmärk 
oli planeerida sagedused nii, et saatjatel oleksid kogu Eestis sagedused, 
millel on võimalikult vähe häireid naaberriikidega ja omavahel. 

\question{Kas kogu seda teadust tehti Corel Draw abil?}

Jah. Jaanus on tohutu pedant ja suure töövõimega katalogiseerija. 
Tema enda isiklik huvi on \emph{bluegrass}-muusika. Mäletan, et tema oli esimene 
inimene minu tutvusringkonnas, kes välismaalt e-poest asju tellis, näiteks plaate 
CDNow'st\sidenote{CDNow oli 1994. aastal asutatud internetipõhine muusikamüüja, 
kes paraku ei elanud esimest dot.com-mulli üle ja sulges sajandivahetusel uksed.}, ja imestasin, kuidas selline asi üldse 
võimalik on. Ta tellib jumal teab kust CD ja see tulebki pakiga kohale.

\question{Jaa, isegi üheksakümnendate lõpus oli Amazonist raamatute tellimine 
suhteliselt eksootiline tegevus. Aga mis hetkel ja kuidas te 
neti.ee\index{neti.ee} automatiseerisite?}

Meie tandem Jaanusega töötas selles mõttes ülihästi, et mina olin 
programmeerija ja arendasin tarkvara ning Jaanus oli katalogiseerija. Kui ta 
projektiga liitus, siis hakkas see täielikult 
lendama. Meil läks paar kuud aega, kui olime 
Meediamaast\index{Meediamaa} igatpidi kõikide näitajate poolest mööda läinud. 
Olime tollal ajal võibolla isegi natuke liiga ebaviisakad noored mehed. 
Näiteks reklaamisime netit spämmides: tegime 
masspostituse, saates kõikvõimalikele meiliaadressidele teate, et nüüd 
on selline huvitav teenus olemas nagu neti.ee, tulge ja külastage. Kui vaatasin hiljuti enda \emph{backup}'e, siis 
avastasin, et nimetasin oma \emph{crawler}'it ehk otsingurobotit, kes mööda lehti 
ringi kolab, Nuhiks. 

Huvitaval kombel olin Nuhi programmeerimist alustanud juba mitu kuud 
varem ehk miski oleks mind nagu suunanud sellele teele, et seda võib vaja 
minna. Otsingumootoreid olin ka varem pisut teinud. Kui ma pärast 
Erameest ülikooli läksin, siis üks sealt saadud tuttavatest kutsus mind 
tegema üht ärikataloogisarnast teenust Bartanet. See 
asus EsData\index{EsData} Suni serveris Akadeemia tee 21 
teisel korrusel, samas majas, kus me hetkel viibime. Ja selles Suni serveris 
sain teha FTP-serverite otsingut. Panin püsti otsinguteenuse Filerix, mis töötas umbes 
kolm-neli kuud ja võimaldas väga hõlpsasti 
faile üles leida igasugustest kohalikest FTP \emph{mirror}'itest. 
Marek Tiits\index[ppl]{Tiits, Marek} hostis tollal IBSist\index{Institute of Baltic Studies} 
sellist asja nagu TuCows\sidenote{TuCows (The Ultimate Collection 
Of Winsock Software) keskendus oma algusaegadel tasuta tarkvarale. Kuna 
interneti kiirus sõltus veel väga suurel määral geograafiast, hoidis
ettevõte käigus skeemi, kus huvilised võisid jooksutada lehekülje TuCows.com 
lokaalseid peegleid. Ühte sellist Marek pidaski.}. Minu otsingumootor 
võimaldas kergesti failinimede järgi üles leida tarkvara tolleaegsele Windows 
95-le, vanadele Windowsidele ja nii edasi. Nii et tolle pooleaastase projekti kõrvalprojektina tegin failiotsingut.

\question{Suure hulga failide indekseerimine ei ole naljaasi, vaid 
eeldab programmeerimisoskust. Kust sa selle üles korjasid?}

Tol hetkel oskasin ma programmeerida Perli\index{Perl} ja kõike 
seda, mis Unixi \emph{shell}'is oli saada. See oskus tuligi sellest perioodist, 
kui uurisin, mis on nii-öelda Unixil kõhus.

\question{Kas sa korjasid algoritmika ja muu sellise ise üles?}

Jah, aga mis puudutab veebi \emph{crawl}'imist, siis selle peale tuli juba 
mõelda.

\question{Kaua su \emph{crawler}'il aega läks, et 
kogu Eesti veeb üle käia?}

Umbes ööpäev, veeb oli 
tollal väga väike. Kataloogi suurus võis olla paar tuhat linki, mitte rohkem. 
Keskmine koduleht oli ka kolm kuni viis lehekülge. Huvitavamaks läks pärast, kui linkide hulk ulatus juba 
miljoniteni. Ühel hetkel oli käigus selline \emph{crawler}, mis 
töötas paralleelselt paljudes \emph{thread}'ides, aga see oli 
loomulik evolutsioon. 

Estpak Datasse\index{Estpak Data} võeti mind ilmselt tööle seetõttu, et olin ühe sellise kõrvalprojektina 
teinud HTMLi tutvustuse. Pidin seda siis, kui Keila gümnaasiumis
serverit administreerisin, kellelegi õpetama, kuna 
eestikeelset materjali polnud ja tegin ise ühe esimese eestikeelse 
HTMLi tutvustuse, mis võttis läbi kõik üksikud elemendid.

\question{See tuleb tuttav ette, olen sealt ilmselt isegi infot otsinud.}

See HTMLi tutvustus on samal aadressil praegu ka üleval ja ma 
olen üsna kindel, et see on üks vanemaid veebilehti 
Eesti veebiruumis, mis on originaalkujul originaalaadressil. 

\question{Mis aastast see on?}

Aastast 1996. Olen muide ühe projektina teinud veel ka veebipokkeri. Nii et ei saa öelda, nagu mul poleks kunagi huvi olnud 
mänge teha, aga rohkem olen programmeerinud nii-öelda 
veebiasju kui \emph{desktop}'is või masinas töötavaid rakendusi. 

Nende teadmiste baasil mind Estpaki tööle võeti. Tõenäoliselt 
näitasingi neile veebipokkerit ja 
HTMLi tutvustust ning võibolla rääkisin ka seda, et olen 
\emph{crawler}'i teinud. Igal juhul mind võeti tööle.

\question{Kes teil tootepoolt tegi või polnud siis veel niisugust mõistet nagu 
tootejuht?}

Ei olnudki. Piltlikult öeldes pandi mind laua taha istuma, et palun tee. 
Tegelikult oli see mõnes mõttes ikkagi läbi mõeldud. Estpak Data\index{Estpak 
Data} tegi koostööd reklaamiagentuuriga PRC Nord Decor\index{PRC Nord Decor}, mis rentis ruume Kullo majas 
Mustamäe teel. Nii et minu füüsiline töökoht asuski seal. Mul oli arvuti, millel oli püsiühendus 19.2 
kilobitti sekundis. Noore mehena ei huvitanud mind, kuidas raha liigub, vaid ainult 
tehniline pool. Idee seisnes selles, et reklaamiagentuur aitas 
potentsiaalsetel Estpak Data klientidel teha kodulehti ja neile 
reklaami. Üks kolleeg, Tiit Sermann\index[ppl]{Sermann, 
Tiit}, kes Nord Decoris töötas, oli kunagise 
OK jutuka\index{OK jutukas}\sidenote{OK jutukas oli üks esimesi massidesse läinud 
sotsiaalvõrgustikulaadseid rakendusi Eestis. Jututube ehk kohti, kus sai üle 
telneti kaaskodanikega suhelda, oli teisigi, aga 1996. aastal käivitatud OK oli 
esimesi veebipõhiseid jutukaid ja tõenäoliselt omataolistest siin kandis 
suurim. Üheaegselt lobises omavahel kuni 300 inimest ja jutuka esimese 
aastapäeva pidu kajastas isegi toonane Päevaleht.} üks asutajatest. Teine oli
Kaupo Kalda\index[ppl]{Kalda, Kaupo}. Naljakas oli see, et Tiidu alias oli Ott \sidenote{OK tulenes asutajate nimedest Ott ja Kaupo.}, kuigi
pärisnimi oli Tiit. Praegu tundub, et kogu see 
maailm oli tollal nii pisikene, et kui natukenegi seal ringi 
käisid, siis puutusid paratamatult kõikide nende inimestega kokku, 
kes siis toimetasid.

\question{Räägi palun sellest, kuidas te Hoti tegite.}

Kaitseväest tagasi tulles oli Eesti 
Telefon\index{Eesti Telefon} Estpak Data\index{Estpak Data} ära söönud, see 
lakkas olemast. Ma töötasin Eesti Telefoni teleteenuste arenduse
allüksuses, kelle eesmärk oli välja töötada uusi teenuseid, ja neti.ee tegemisega sattusimegi sinna. 

Kontoriruumi jagasin ühe noormehega, kes arendas 
sissehelistamisteenust. Meil vedeles kapi peal üks pisike 
Ascendi sissehelistamiskeskus ja ma küsisin, 
kas võin seda uurida.

\question{Kas selle külge käisid tavalised modemid 
või oli see juba valmislahendus?}

Ei, see oli spetsiaalne sissehelistamiskeskus: see tuli 
installeerida \emph{rack}'i ja panna juhtmed külge, et see hakkaks numbreid 
kuulama ja teenust osutama. Keskust 
uurides avastasin, et see autendib ennast 
vastu sellist autentimisserverit nagu Radius. Edasi uurides sain teada, et Radius on lihtne sõnastikupõhine protokoll, 
ja nii ma programmeerisingi Radiuse serveri, mis suutis 
sissehelistamiskeskust juhtida. Avastasin ka, et sissehelistamiskeskuse 
\emph{firmware} võimaldas teha igasuguseid huvitavaid asju, näiteks sai kohe Radiuse serverist öelda 
sissehelistamiskeskusele, kui kaua konkreetne kasutaja võib ühenduses olla. 

Sellest teadmisest sündis näiteks selline toode nagu Atlas Surf\index{Atlas Surf}, mida 
Eesti Telefon ettemaksulise internetina\sidenote{Sarnane kontseptsioon nagu mobiiltelefoni kõnekaardid.} müüs. Ühesõnaga, see toode sündis 
sellest, et häkkisin väikest sissehelistamiskeskust, mis oli 
mõeldud mobiiliga sissehelistamiseks. Too keskus toetas V.35 protokolli, millest paljud pole ilmselt kunagi 
kuulnud, aga see oli \emph{wideband}-protokoll, mis töötas üle GSMi. 
Kui sul oli GSM-telefon, mida sai arvutiga ühendada, siis see võimaldas V.35 protokolliga
sisse helistada ja kiirus oli veidi suurem kui 
tavalise modemiga üle mobiili vilistades. 

Hüppan korraks veel minevikku. Oli aasta 2000, ilmselt kõik mäletavad 
Y2K\sidenote{Nagu kogenud programmeerijad ütlevad: \enquote{Sinu lapselapsed neavad päeva, mil sa otsustasid oma 
koodi optimeerida}. Kuna pikka aega optimeeriti koodi hoides aastaarvu  kahekohalise 
numbrina, kulutati aastatuhande vahetuse paiku üüratus koguses tööaega ja raha, 
tagamaks, et aasta 2000 ei oleks arvutite arvates võrdne aastaga 1900. Aastal 2038 ootab meid sarnane probleem, kui Unixi aeg oma andmetüübi jaoks liiga suureks kasvab.} probleemi: kardeti, et arvutid 
lähevad katki, sest nende kell lakkab aastatuhande vahetusel töötamast. Ka Eesti Telefonis 
kardeti seda, sest \emph{legacy}-süsteeme oli tohutult palju. Kõik süsteemidega seotud insenerid pidid jääma valvesse. Ma ei mäleta, kuidas 
mul õnnestus sellest kõrvale nihverdada, aga tol hetkel olin sõpradega Soomes 
suusatamas ja lumelauaga mäest alla laskmas. Paar päeva enne 
aastavahetust tuli mulle klienditeenindusest kõne, et enam ei saa sisse 
helistada. Läksin autosse, kus mul oli sülearvuti, panin telefoni arvuti külge, helistasin 
V.35 protokolliga meie privaatkeskusse sisse ja 
hakkasin uurima, miks Hoti kliendid ei saa sisse helistada. 
Tuli välja, et keegi oli viimasel hetkel Y2K hirmus peale laadinud ühe turvapaiga, mis muutis Radiuse serverile 
minevat teadet, mispeale Radius läks katki, kuna sellele tuli tundmatu 
sisuga \emph{dictionary}.

Surfist edasi juhtus nii, et Eesti Telefoni kontsessioonileping 
oli juba lõppenud või lõppemas ja turule tuli Tele2\index{Tele2} Rootsist. 
Tele2 idee oli korrata Eestis täpselt sama, mida Rootsis: nad
soovisid suurelt \emph{telco}'lt palju raha välja imeda. Kuna Eesti 
Telefon üüris ruume ja liine, oli meile teada, et Tele2 paneb oma 
sissehelistamiskeskuseid püsti. Eesti Telefoni juhtkond oli paanikas, 
ma ise ka külastasin laiendatud juhatuse koosolekut, kus 
seda arutati. Tulin sealt üsna mornilt tagasi -- mulle 
tundus, et vanemad kolleegid ei suuda midagi otsustada ega ära teha. Mina 
noore mehena oleksin tahtnud kohe tegutseda. 

Pidasin telefonikõne Priit Pirsoga\index[ppl]{Pirso, Priit}, kes oli selle valdkonna juht 
Eesti Telefonis, ja me otsustasime teha Eesti 
Telefoni osutatavale Atlas Starteri teenusele alternatiivse teenuse, sellepärast 
et Atlas Starter ei sobinud Tele2ga konkureerimiseks. Meil oli vaja 
teenust, kus kasutajate registreerimise protseduur oleks 
automaatne, st kasutaja registreeriks end ise. Kuna kuutasu poleks pärast Tele2 jampsi niikuinii enam olnud, siis 
ainukesed, mis maksid, olid kõneminuti hinnad. Tele2 lootis raha teenida sellest, et
termineerib kõnet ja Eesti Telefon on sunnitud talle 
vahendama kliendi käest küsitud kõneminuti hinda. 

Selle telefonikõne käigus me leppisime kokku, kes, mida ja kuidas teeb, ja et toode saab 
nimeks Hot\index{hot.ee}. Ma olin siis juba arvutist järele vaadanud,
millised huvitavad domeenid olid vabad. Tollal oli veel see 
aeg, kui EENet\index{EENet} ei nõustunud andma ühele ettevõttele mitut 
domeeni, aga ühel mu praegusel kolleegil, Guido 
Kõivul\index[ppl]{Kõiv, Guido}, õnnestus saada EENetist meile
hot.ee domeen, sama skeemiga, nagu 
Tarvi\index[ppl]{Martens, Tarvi} ilmselt kasutas www.ee jaoks. Igatahes kõik käis ruttu ja kaks nädalat hiljem olime \emph{live}'is: 
meil toimus teenuse \emph{launch} ja kasutajaid hakkas registreeruma 
tempoga tuhat tükki päevas.

Sealt saigi hot.ee alguse ja minu teha jäi Radiuse pool. 
Hoti\index{hot.ee} puhul oli meie huvi see, et 
inimesed helistaksid meile sisse. Tollal hakati juba
kasutajatele ka meiliaadresse andma. Kuna varem küsiti meili eest raha, siis 
meile tundus, et lihtsalt niisama meiliaadresse jagada ei tahaks. Siis sai 
tehtud sedasi, et kasutaja sai küll veebipõhiselt konto luua, aga 
meili- ja ka kodulehekonto ei hakanud tööle enne, kui 
registreeritud kontoga oli tehtud vähemalt üks telefonikõne
sissehelistamiskeskusesse. Seda loogikat võimaldas minu \emph{custom} 
Radius, kes kasutajatel järge pidas. 

\question{Ühel hetkel oli hot.ee-s ka veebimeil, eks?}

Veebimeiler oli suhteliselt algusest peale esimese 
kujunduse osa, aga see ei olnud minu programmeeritud, vaid internetist 
leitud vabavara. Me isegi ei \emph{rebrand}'inud enda värvidesse, vaid see oli lihtsalt meie lehelt 
lingitud ja me ise majutaasime teda.

\question{See seletab, miks meil mõned aastad hiljem veebimeileri tegemine 
Hansapangas\index{Hansapank} nurja läks -- meil miskipärast ei tulnud pähe mõtet 
seda lahendust internetist alla laadida.}

Mulle ei tulnud pähe seda ise teha. Küll aga mäletan 
sellist huvitavat protokolli nagu WAP\sidenote{Wireless Application 
Protocol (WAP) oli sajandivahetuse paiku tehtud 
katse luua toona kasinate sidevõimalustega mobiiltelefonide jaoks lihtsamaid 
internetiprotokolle 4. kuni 7. OSI kihini. Muu hulgas sisaldas WAP
erilist \emph{markup}-keelt toonase mobiiltelefoni mõnerealisele ekraanile 
sobivate kasutajaliideste loomiseks.}, mis kujutas endast interneti mobiilivarianti. Selle WAP-meili tegin Hotile küll täiesti 
nullist.

\question{Õnneks see ei olnud väga pika elueaga, sest ka WAP ei kestnud kaua.}

EMT tollane arendusjuht Ando Meentalolt\index[ppl]{Meentalo, 
Ando} kommenteeris minu WAP-meili nii: \enquote{Sa võid ju sinna suahiili keele ka panna, aga 
ilmselt pole sellest väga palju kasu.} Mul sai WAPiga tegelemine 
alguse sellest, et olin saanud endale WAPi-võimelise telefoni (kusjuures see oli vist ainuke telefon, mida ma olen iialgi tööandjalt saanud). See oli suure ekraani ja klapiga
Nokia 7110\sidenote{Tegu oli 1999. aastal uskumatult innovatiivse 
telefoniga: mitut tekstirida näitav ekraan, rullikuga kasutajaliides, T9 
ennustav tekstisisestus sõnumite puhul, vedruga uhkelt lahti hüppav klapp, WAP, 
ebamaiselt küütlev korpus jne. Oma iseäraliku kuju tõttu sai aparaat rahva seas 
hüüdnimeks \enquote{banaan}.}. 

\question{See telefon oli suurepärane põhjus Hansapangale WAPi-põhine 
internetipank teha, sest selle testimiseks pidi ju pank ometigi väljastama ka 
sobiliku seadme.}

Mul juhtus \emph{vice versa}: kõigepealt sain telefoni ja siis 
tuli idee, et äge oleks enda postkasti vaadata sellisel mugaval moel. Ja siis tegingi 
WAP-meili.

\question{Sellega algab juba uus sajand ja sellest räägime võibolla 
mõni teine kord. Lõpetuseks küsin, mida sa praegu teed?}

Praegu teen Bolti\index{Bolt}\sidenote{Endise nimega Taxify ja asutatud kui mTakso.} serveri infrastruktuuri. Minu üks kauaaegseid 
kolleege Eesti Telefonist Tarmo Kople\index[ppl]{Kople, Tarmo} on 
üks nendest inseneridest, kellega alustasime Bolti 
serverimajandust algusest peale. Ja kui algul oli kliente ja sõite tuhandeid kuus, siis nüüd juba miljoneid.


\chapter{Andres Kütt}
%!TEX TS-program = arara
% arara: myindex

Sündisin 1975. aastal Võrus. Millestki midagi aru saama hakkasin 
kaheksakümnendate teisel poolel. See oli mitmes mõttes üsna kole 
aeg. Noorukile kõige arusaadavam neist koledustest oli lihtlabane praktiline 
puudus. Päris nälga ei olnud, aga midagi vähegi leivast ja piimast edevamat 
saada ei olnud. Kui linnakeses levis kuuldus, et olla toodud kast jäätist, oli 
poes veerand tunniga saba ning poole tunni pärast kõik otsas. Muu hulgas oli 
kaubandusvõrgus saada kahte tüüpi meeste talvejopesid. Mitte kahtekümmet või
kahtesadat, vaid kahte. Ühed olid hallid ja neid said osta lihtsurelikud\sidenote{Huvitaval kombel oli jope põuetasku 5,25 tolli lai, sinna mahtus 
üks flopi täpselt sisse.} ning teised olid punase A-tähega ja neid said osta 
ainult inimesed, kes teadsid kedagi, kes teadis kedagi. 
Hämmastaval kombel käisid ka seda viletsust inimesed Pihkvast bussidega 
uudistamas ja viimastki kaupa ära ostmas. 

Kogu selle halluse keskel suutis Nõukogude Liit meie Võru \linebreak[4]\mbox{Kreutzwaldi} 
Gümnaasiumile\index{Võru Kreutzwaldi Gümnaasium} tarnida 
arvutiklassitäie arvuteid Agat\index{Agat}\sidenote{Agat oli 
Nõukogude Liidus valmistatud arvuti, mis oli küll Apple IIst\index{Apple 
II} inspireeritud, kuid siiski mitte täpne kloon.}. Kust need tulid ja kes 
seda asja ajas, ei tea. Küll aga mäletan, et nende saabumine oli pikalt oodatud 
ja edasi lükatud. Miks ja mida täpselt oodatud sai, ei oska öelda. Tean ainult, et 
kui klass tekkis, läksin sinna sisse ja enam välja ei tulnud. 

Ega tolle purgiga palju teha ei olnud. Olid mõned mängud ja programmeerimiseks 
BASIC\index{BASIC}, milles meid programmeerima õpetatigi. Esimese hooga ei õpetatud 
seejuures mitte kõiki käske, näiteks for-tsükkel oli tükk aega saladus. Kui aga 
nohikud said aru, et nende eest tarkust varjatakse, kadus igasugune respekt ja 
läks lahti suuremaks isepusimiseks. Kõik muutus, kui kooli saabus noor, vist
värskelt ülikoolist tulnud arvutiõpetaja Aivar 
Halapuu\index[ppl]{Halapuu, Aivar}. Temaga tekkis kohe 
poolkamraadlik side, mis sisaldas siiski alati suurt kogust meiepoolset lugupidamist. Tolleks ajaks oli meil väiksem seltskond poisse, kes seal 
klassis toimetas ja end kohe \emph{in corpore} Aivarile sappa haakis. Aivar 
viitsis meiega tegeleda ja kuigi ta meile suurt midagi arvutite mõttes ei 
õpetanud, sai tema käest midagi kultuuritaolist. Ta
üritas meiega bridži mängida, rääkis mänguteooriast ja nii edasi. Ega me väga palju aru saanud, kuid targa inimese viitsimine meiega tegelda tekitas soovi
tolle viitsimise vääriline olla.

Kuna me sisuliselt elasime arvutiklassis (peale kooli kohe sinna, õhtul 
hilja koju, nädalavahetustel käisime samuti Aivari käest võtit palumas), siis 
usaldati arvutiklassi võti üsna pea meie kätte. Aga \emph{kooli} võtit meie kätte keegi ei andnud. Seetõttu 
oli oluline hoida järjepidevust: keegi oli alati klassis olemas ja lasi hõikamise 
või kivikese viske peale tulija sisse. Mõnikord oli meie käes 
siiski ka välisukse võti, aga tihti roniti sisse-välja akna kaudu. 

Ühel hetkel avanesid kuskil kraanid ja hakkas saabuma humanitaarabi. Võrul oli vist seoses 
rahvamuusikaga põnevaid suhteid välismaa asutustega, kes hakkasid meile 
igasugust huvitavat kola saatma. Kord saabus klassitäis rootsikeelsete 
paberite ja tarkvaraga masinaid, millega me ei osanud mitte midagi teha. Käima 
nad läksid, rootsikeelseid veateid väljastasid, aga sellega asi piirdus. Millega tegu oli ja mis neist sai, 
ei tea. Aga tuli ka üks iidne aparaat, mille külge käis neli-viis 
terminali ja kaks kokku külmkapisuurust kettaseadet, mille sisse käisid 
hiigelsuured plastkarbis kettad. Tegu oli industriaalseadmega: kui tuurid sisse 
võttis, siis oli alla tänavale kuulda, et \enquote{arvuti töötab}. Tolle masina 
peal ei osanud ka keegi midagi tarka teha, sest tarkvara polnud. Kuna masinasse asjade saamiseks 
olid ainult nimetet hiigelsuured kettad, ei olnud tarkvara ka kuskilt võtta. Mängisime mänge 
ja oligi kõik. Mäletan siiski, et seal puutusin esimest korda 
kokku Zorki\index{Zork}-nimelise mänguga\sidenote{\enquote{Zork} on üks varasemaid 
tekstipõhiseid arvutimänge. Mängija sisestas teksti ja talle ka vastati tekstiga 
vastavalt sellele, mis mängus parasjagu juhtus. Kuna mängu alguses sattuti 
lagendikule valge maja ette, oli meie puhul ilmselt tegemist \enquote{Zork I-ga}.}.

Lõpuks tulid meile Jukud\index{Juku} ja üheksakümnendate algul ka 
PCd. Jukusid oodati väga, sest Agat oli päris jube aparaat.\sidenote{Ma ei ole 
kunagi hiljem kohanud arvutit, mis suudab flopikettasse füüsilised sooned tõmmata.} 
Jukud olid väga ägedad, ainus nõrk koht oli 
klaviatuur. 

Mis aga palju ei muutunud, oli tarkvara. Võru ei ole Tartu ega 
Tallinn. Meie seltskond ei suhelnud õieti kellegagi, nii et uut tarkvara ja
teadmisi ei tulnud eriti kuskilt peale. Ajakirjast \enquote{Arvutustehnika \& 
Andmetöötlus}\index{Arvutustehnika \& Andmetöötlus}\sidenote{\phantomsection\label{sisu:aa}Alates aastast 
1987 Eesti esimese infotehnoloogiaettevõtte Algoritm\index{Algoritm|see{Tallinna 
Teadus-Tootmiskeskus}} (sellest põnevast asutusest loe lähemalt lk \pageref{sisu:algoritm}) 
algatusel ja rahastusel ilmunud esimene regulaarne IT-ajakiri. A\&A ilmus Eesti Teadus- ja Tehnikainformatsiooni ning Majandusuuringute 
Instituudi\index{Eesti Teadus- ja Tehnikainformatsiooni ning Majandusuuringute Instituut} 
(lühendatult Eesti Informatsiooni Instituut\index{Eesti Informatsiooni 
Instituut|see{Eesti Teadus- ja Tehnikainformatsiooni ning Majandusuuringute Instituut}}) infoseeriana.} 
võis küll lugeda Unicode'i võludest, aga programmeerida tuli ikkagi 
assembleris või BASICus. Seejuures sain alles hiljem teada, et eksisteeris 
ka makroassembler. Tavalises assembleris pidi JMP-käsule andma argumendiks 
suhtelise aadressi (mis muidugi osutus kohe valeks, kui kuskile mõne rea 
vahele panid)\sidenote{See oli probleem vaid minusugustele surelikele. Inimesed, 
nagu klassivend Vallo Trell\index[ppl]{Trell, Vallo}, suutsid ka otse BIOSi 
prompti peal mällu baite kirjutades masinkoodis programmeerida.}, aga 
uuemas assembleris sai silte kasutada. Käisime ka mõnel üritusel Tallinnas (mäletan 
Pedas\index{Tallinna Pedagoogikaülikool} asunud MSXide\index{Yamaha MSX} klassi) 
ja tõime sealt ka tarkvara kaasa, aga üldiselt olime üsna omaette. Isegi 
flopisid käisime ostmas Tallinnas ühest komisjonipoest. Tavaliselt kasutasime ära mõnda klassiga organiseeritud käiku teatrisse, kui jäi 
ka paar tundi linnas kolamise aega. 

Olin ka üks õnnelikest, kellele lõpuks arvuti suveks koju usaldati -- muidu pidime 
suvekuud veetma arvutiklassi akna all kurvalt kiibitsedes. Esmalt lubati
Agat, siis Juku. Kuna ekraanid olid mõlemal nigelad, veetsin kaks-kolm suve 
ettetõmmatud kardinate taga arvutiga toimetades. Juku peal mäletan oma tegemistest kahte 
suuremat projekti. Esimene oli Norton Commanderi moodi failihaldur ja teine 
fondiredaktor. Jukul sai tähekujusid suhteliselt lihtsasti ümber teha, mälus 
olid vist kaheksabaidised bitimaatriksid ning teksti kuvamine käis kiiremini 
kui muu graafika. Mõlemat kirjutasin assembleris ja kumbki päris valmis ei 
saanudki, sest teatud mahust alates muutus kood hoomamatuks. Sel ajal omandasin 
ka hiljem palju vaeva põhjustanud kombe \enquote{tunde järgi} koodi kirjutada: 
teed muutuse, kompileerid, proovid ja muudad pikalt mõtlemata uuesti, kuni 
asi pigem juhuse kui mõistuse tahtel tööle hakkab. Kood oli
nii kole, et seda oli liiga keeruline iga kord uuesti läbi mõelda. Mingid 
\emph{off-by-one} vead olid sagedased, aga üldjuhul sai mõne konstandi ühe võrra 
nihutamise peale koodi käima. Sellest rumalast kombest pole ma paraku siiani lõpuni 
vabanenud. 

Juku peal sai ka andmebaase teha, dBASE\index{dBASE} oli täitsa olemas. 
Tänu sellele õnnestus maik suhu saada kellelegi arvuti abil kasulik olemisest, kui tegin 
koolivend Aini dieediteemalise uurimistöö jaoks andmestiku ja kirjutasin 
ka programmi kassetiümbriste trükkimiseks. Tollal käibis muusika kassettidel, 
mida ohtralt kopeeriti\sidenote{Eksisteeris ka tänapäeval mõeldamatu täiesti 
põrandapealne muusika kopeerimise asutus, näiteks Tartus. Läksid kohale, 
valisid kataloogist albumi välja, jätsid tühja kasseti maha ja mõni päev hiljem 
said sobiva summa vastu muusikaga kasseti tagasi.\phantomsection\label{sisu!kassetid}}. 
Seetõttu kirjutati lugude 
nimesid käsitsi ning see oli tüütu. Minu tarkvara võimaldas aga kiiresti 
kassetiümbriseid trükkida. Selle teenuse eest sai vist ühelt 
klassivennalt isegi raha küsitud.

Linna peal tegin erinevates kohtades ka PCdega tutvust. Kellelgi oli mööblivabrikus 
tutvusi ja seal toimus isegi mõned korrad mingisugune õpe. Istusime ilmselt 
raamatupidamise masinate taga ja meile näidati, kuidas FoxPros\index{FoxPro} 
vorme joonistada ja andmeid hoida. 

Keskkoolis õnnestus käia väga murdelistel aastatel 1990--1993. Võrus möllas 
punkar Saare Ain\index[ppl]{Saar, Ain}\sidenote{Kodanikunimega Ain Saar, asutas 
Vaba Sõltumatu Noortekolonni number 1 ja tegi muid tükke.}, Võru surnuaial 
taastati Vabadussõja mälestussammas ja miilits ajas koertega üritusi laiali. 
Ühe sellise intsidendi järel oli koolis näha kummalistes ülikondades 
seltsimehi, kes pingsalt vanemate klasside õpilaste nägusid jälgisid ilmses 
lootuses tuttavaid kohata. 

Tekkis ka äri. Leidsime sõpradega 
ajalehest kuulutuse, milles otsiti meie jaoks ulmeliste palkadega (umbes 
vanemate aastapalk paarinädalase projekti eest) meelitades C 
programmeerijaid. Kandideerimise tähtaeg oli kaks nädalat ja see 
tundus täiesti mõistlik aeg, millega omale C selgeks teha. Kuskilt sai hangitud 
klassikaline Brian Kernighani ja Dennis Ritchie \enquote{The C Programming 
Language}\index{The C Programming Language}, mida kambaga tudeerisime ja mis tundus 
loogiline. Kuna meil puudus juurdepääs C kompilaatorile, siis päris koodi 
kirjutada ei saanud. See meid ei heidutanud ja saatsime isegi mingid kirjad välja. Vastust muidugi ei tulnud. Hiljem olen mõelnud, kas tegu võis olla 
tollesama legendaarse lehekuulutusega, mis viis kokku Bluemooni\index{Bluemoon} 
poisid ja Stefan Obergi\index[ppl]{Oberg, Stefan}, aga ajastus siiski vist ei klapi. 

Kõik head asjad saavad kord otsa, nii ka keskkool. Tollal sai lõpueksamit valida ning oleks olnud kummaline, kui meie 
seltskond ei oleks valinud arvutieksamit. Aivarist olime arvutiteadmiste poolest juba kaugel 
ees, sest meil ei olnud sõna tõsises mõttes mitte midagi muud teha kui arvutit 
torkida. Laulsin küll ka kooris\sidenote{Kooriga välisreisile minek oli ka põhjus, miks ma ei ole kunagi vabariiklikul 
informaatikaolümpiaadil käinud. Tol ühel kevadel, kui sinna õnnestus välja 
murda, oli ka reis plaanis. Otsustavaks sai, et ma ei tahtnud koori hätta 
jätta, mitte et oleksin seal kandvat rolli mänginud.}, 
aga põhimõtteliselt kogu muu vaba aeg oli arvutite päralt. Isegi õppetöö ei 
seganud, sest põhikoolis tegin endale kõva põhja alla. Kõik see ei 
vähendanud sugugi eksami pidulikkust. Sisenesime ruumi, võtsime pileti, 
lahendasime, vastasime komisjonile -- kõik oli nii, nagu peab. Aivar oleks võinud 
meile kõigile pikalt mõtlemata viied välja kirjutada, aga ometi viidi eksam täie tõsidusega läbi. 
See tundub siiani oluline.

Kuna mul õnnestus kool nibin-nabin kullaga lõpetada, sain Tartu Ülikooli 
matemaatikateaduskonda\index{Tartu Ülikool!Matemaatikateaduskond} eksamiteta 
sisse. Sinna minek tundus loogiline, sest Tallinn oli kaugel ja tundmata ning 
arvutivärki tahtsin kindlasti õppida. Sõjaväega probleeme ei olnud. Esiteks 
olid segased ajad ning Eesti riik polnud veel päriselt välja mõelnud, mismoodi 
oleks mõistlik väeteenistust korraldada.\sidenote{Hiljem on selgunud, et ülikooli 
minek vabastas väeteenistusest ning meie aastakäik hakkas ülikooli lõpetama just 
siis, kui otsustati siiski enne ülikooli väeteenistuse läbimise kasuks.} 
Teiseks oli mu silmanägemine nii paha, 
et kaitseväe tohtrid ütlesid mulle: \enquote{Kui venelane peale tuleb, 
siis paneme su laipu vedama, seniks mine koju.} Nii veetsingi suve Võru ja 
Tartu vahel hääletades, käisin näiteks ka Steni\index[ppl]{Tamkivi, Sten} 
juures\sidenote{Meie suguseltsid sõbrustasid, 
Steni vanaisa elas Võrus ja saime juba üsna õrnas eas tuttavaks.} 
Primexis\index{Primex Data} külas. Kohtasin seal elus esimest korda Photoshopi-nimelist tarkvara, laserprinterit ning morni näoga, kuid
huvitavat tüüpi, kes osutus Tarmo Taliks\index[ppl]{Tali, Tarmo}. Temaga puutusime
hiljem veel korduvalt kokku. Tarmo on üks neid inimesi, kelle puhul olen veendunud,
et olen temalt kohutavalt palju õppinud, suutmata siiski midagi konkreetset sõnastada. 
Olen tänulik. 

Sügisest algas ülikool ja asusin püsivamalt Tartusse. Kuna jäin paberite ajamisega 
töllerdama, siis ei õnnestunud koos teiste matemaatikutega Tiigi ühikasse kohta saada. Ühe või kaks talve olin sugulase juures üüriliseks, ühe talve 
elasime kambaga Tartu Kurtide Ühingus\index{Tartu Kurtide Ühing}, mis üüris
tudengitele tuba välja. Küll aga sai külas käidud klassivendadel, kellest enamik 
läksid majandust õppima ja kelle ühikaks olid Narva maantee 
tornid. Nii õnnestus ühikaelust maik suhu saada selles siiski kõrvuni osalemata. 
Sellest mul ülemäära kahju pole, sest õlu mulle ei maitse. Tiigi ühikas 
tegid kaastudengid kord koridori lõkke ning Narva maantee tornides pudenes regulaarselt 
keegi rõdult alla. Minu jaoks ei ületanud kahtlemata elava seltsielu paleus 
kommunaalhorrori veidi ligast reaalsust. 

Ülikoolis sain piltlikult öeldes kohe ägeda laksu otse ego pihta. Esmalt 
selgus, et erinevalt keskkoolist oli ülikoolis vaja päriselt õppida, aga vastav oskus 
oli juba kadunud (keskkool möödus arvutite seltsis ja põhikooli seljas 
liugu lastes) ning tuli uuesti tekitada. Teiseks selgus, et 
ropust tööst enam heade hinnete saamiseks ei piisanud, vaja oli ka annet, aga 
seda on mul kogu aeg nappinud. Teistel seevastu annet jagus 
ning see tegi egole haiget. Näiteks Meelis Roos\index[ppl]{Roos, Meelis} 
ja Rene Prillop\index[ppl]{Prillop, Rene} seilasid igasugusest matemaatikast läbi 
ilma nähtava pingutuseta ja kirjutasid koodi nagu jumalad. Margus 
Sutt\index[ppl]{Sutt, Margus} teadis arvutitest nähtavasti kõike ja oli tolleks 
ajaks juba tegelenud täiesti müstilisena tunduvate asjadega. Asko 
Seeba\index[ppl]{Seeba, Asko} oli kõike seda \emph{ja} seejuures veel 
seltskondlik, mängis kitarri ning oli tüdrukute hulgas popp. Ei jäänud 
midagi üle, tuli tasapisi hakata inimeseks õppima. 

Lisaks inimeseks saamisele oli vaja saada tööinimeseks, sest ema käest ei 
saanud ju jäädagi raha küsima. Proovisin saada baarmeniks, vast avatud Atlantise ööklubi valgustajaks ja isegi 
arvutigraafikuks, aga asjata. Lõpuks sattusin ettevõttesse Korel 
IN\index{Korel IN} programmeerijaks, esimene tööpäev oli 1993. aasta detsembri alguses. Mind ja kamraad Veljot\index[ppl]{Hagu, Veljo} võeti palgale 
eesmärgiga luua firmale arvetega majandamiseks vajalik tarkvara. Keeleks oli 
Visual Basic\index{BASIC!Visual Basic} ja ei läinud palju aega, kui meil mõned asjad 
juba töötasid. \enquote{Programmeerija} kõlab märkimisväärselt glamuursemalt, 
kui asi tegelikult välja nägi. Tegime kõike alates kauba tassimisest (kontor 
asus viiendal või kuuendal korrusel ja kahekümnetolline CRT monitor on päris 
raske) kuni isegi mõningase müügitööni. Toonasele arvutiärile iseloomulikult 
ei teadnud eales, mis seisus töökoht kontorisse jõudes oli. Mõnikord oli ära 
müüdud mälu, mõnikord võrgukaart või monitor. Mäletan end kirjutamas koodi 
üheksatollise must-valge kassamonitori ees taburetil istudes\phantomsection\label{sisu:jupimyyk}. 

Tartu ei ole suur linn ja nii puutusime Korelis töötades kokku suure osaga 
toonasest arvutiseltskonnast. Tarmo Tali\index[ppl]{Tali, Tarmo} oli meil 
müügimeheks ja aeg-ajalt käis tal külas Asko Oja\index[ppl]{Oja, Asko}, keda kutsuti
hellitavalt \enquote{Tarmo blondiiniks}. Vahel astus Sorose sajalisi 
luhvtitades läbi Marek Tiits\index[ppl]{Tiits, Marek}, kellele õnnestus mingi ime läbi 
isegi üks Suni tööjaam müüa. Kui ütlen, et puutusime, siis tegelikult 
mina ei puutunud eriti kellegagi kokku, sest olin toona ja olen siiani küllaltki 
asotsiaalne. Igasugust toredat rahvast käis poest läbi ja enamasti kuulasin lihtsalt, 
silmad punnis peas, spetsialistide jutte ilma nende nimesidki teadmata. 

Kuidagi tekkis Korelisse aktiivne kodanik nimega Tanel 
Urbanik\index[ppl]{Urbanik, Tanel}. Ta pandi meile alguses ülemuseks, aga üsna 
varsti vedas ta meid Korelist minema, asutades uue ettevõtmise nimega HClub. 
Nimi tuli sellest, et meie tuba kutsuti Koreli päris ärimeeste hulgas veidi põlastavalt 
häkkeriklubiks. Tanel tahtis tarkvaraäri teha, küllap seetõttu tal 
Koreliga teed lahku läksidki. Meie peamiseks leivanumbriks sai kassasüsteemide 
ehitamine ja põhiklientideks erinevad tanklad, näiteks Favora omad. 
Kirjutasin muu hulgas ka Ravimiametile\index{Ravimiamet} nende ühe 
esimestest andmebaasidest. Selguse mõttes olgu öeldud, et toona mingist 
klient-server arhitektuurist juttu ei olnud. Kõik lahendused hoidsid andmeid 
võrguketta peal Microsoft Accessi\index{Microsoft Access} andmebaasis ja selle 
poole pöördumine käis kliendi juurde paigaldatud \enquote{paksu} kliendi abil. 

Tollele ajale tagasi mõeldes tundub hämmastav, et meie tarkvara töötas. Meid 
olid vähe ja testimisest või versioneerimisest ei 
teadnud keegi midagi. Kord pidin Tartust Võrru tanklasse tagasi 
sõitma, sest värsket versiooni flopi peal kohale viies olin midagi valesti 
teinud ja kriitiline toiming läks hilisõhtul katki. Vähemalt minu kood püsis 
kindlasti koos peamiselt nätsu ja teibiga. Veljo\index[ppl]{Hagu, Veljo} oli märkimisväärselt pädevam programmeerija, aga tarkvaratehnikast polnud 
ilmselt palju aimu temalgi. 

See kõik mind lõpuks HClubist ära viiski (päris suure tüliga, tuleb tunnistada). Ma 
ei jaksanud enam selle kokkupunutud ja päris kliente teenindava tarkvara 
eest vastutada. Põlesin läbi ja kõndisin Tanelit valjusti (ja mõneti teenimatult) needes minema. 
Mõnega toonastest seikadest kohtusin veel aastaid halbades unenägudes. Oma rolli mängis 
ilmselt ka see, et just tol ajal läks põhja mu 
unistus saada arvutialane haridus. Nimelt olid matemaatikateaduskonnas 
esimesed paar aastat kõigile ühised, seejärel tuli valida arvutiteaduse, 
statistika või rakendusmatemaatika suundade vahel. Valik käis seejuures õpitulemuste 
alusel. Minu tulemused võimaldasid napilt ennast tulevaseks arvutiteadlaseks pidada ja 
nii esitasin vajaliku avalduse ning asusin järgmisest semestrist hoogsalt 
arvutiteaduse aineid kuulama. Neid loeti enamasti Liivi tänava 
õppehoones\index{Tartu Ülikool!Liivi õppehoone}. 
Dekanaat oma teadetetahvliga asus aga Vanemuise õppehoones\index{Tartu Ülikool!Vanemuise 
tänava õppehoone}. Ja kuna ma ka oma 
ut.ee meiliaadressi ei jälginud, läks minust täiesti mööda dekanaadi mõte, et 
peaks tudengite käest nende suunavaliku kohta veel mingeid pabereid küsima. 
Kui ma ükskord jaole sain, olid 
arvutiteaduse õppekohad täis ja minust sai statistikaüliõpilane. 

See oli päris 
valus hoop. Kuigi arvutiteaduse ained olid minu jaoks rasked (mäletan end kolm 
korda kompileerimismeetodite eksamit tegemas), oli mul siiski mingi lootus 
sealtkaudu kuidagi paremaks programmeerijaks saada ning kamraadidele järele 
jõuda. Toonane ülikooliharidus oli tänasest väga erinev ja ei omanud reaalse eluga 
suurt sidet, aga lootus jäi. Statistikast huvitusin ma vähe ja 
ei näinud mingit võimalust sellest oma töises elus kasu saada. Masinõppe revolutsioonini 
jäi veel paarkümmend aastat. Seetõttu tegin 
edaspidi minimaalse, et kuidagi koolist läbi saada, ja keskendusin tööle. 

Kogu BBSindus läks minust üsna suure kaarega mööda. Võrus ei olnud kohalikku 
BBSi ja kaugekõne ei tulnud ei hinna ega kättesaadavuse mõttes kõne allagi. 
Sten\index[ppl]{Tamkivi, Sten} Primexis\index{Primex Data} küll vist näitas kuhugi 
helistamist, aga tuhka ma aru sain. Korelis oli väline modem ja aeg-ajalt 
sai kuhugi sisse helistatud, aga väga sporaadiliselt. Peamine 
selleteemalise info allikas oli kursavend Mati Muts\index[ppl]{Muts, Mati} ja 
põhiliselt käisin Luciferi \mbox{BBSis}\index{Luciferi BBS}. Küll aga oli 
ülikoolil tol ajal juba täiesti korralik internetiühendus ja palju aega kulus 
Vanemuise õppehoones\index{Tartu Ülikool!Vanemuise tänava õppehoone} terminali 
taga FTPd pidi ringi kolades. Mäletan, et tõmbasin kas ftp.funet.fi või 
ftp.sunet.se serverist tükk aega Metallica albumi kaanepilti ja olin väga 
rahul, kui see ka päriselt kohale jõudis ning ekraanile ilmus.

Selgelt mäletan ka kohtumist HTMLiga. See oli Liivi 
tänaval\index{Tartu Ülikool!Liivi Õppehoone}, kus asus Suni 
klass\sidenote{Need pidid olema Sunid, sest mäletan ruudulist hiirepatja, mis 
muidugi ei olnud mingi padi. Suni optiline hiir sõltus lihtsalt ruudulisest aluspinnast.} 
ning kus ma sukeldusin veebilehtede 
võrratusse maailma. Pärast pikka pusimist suutsin tekitada oma kodulehe, kus 
asju õiges kohas hoidis tabel! Ega sinna kodulehele midagi kirjutada ei olnud, 
aga tabeli ridade ja lahtrite saladuste lahtipusimine oli põnev.

Kõik see osutus kasulikuks, sest HClubi järel võttis 
klassivend Meelis Mäeots\index[ppl]{Mäeots, Meelis} mind enda juurde tehnikuks. Ta tegeles tol ajal 
igasuguste imelike asjadega, kuid muu hulgas asutas ka internetifirma. See koosnes 
alguses peamiselt minust ja temast. Firma tegeles Unineti\index{Uninet} 
\emph{dial-up}-ühenduste edasimüümisega, tegi kodulehekülgi ja pidas isegi 
Infomeistri-nimelist interneti infokataloogi. See viimane oli täiesti hämmastav 
äri. Meelis käis ja rääkis mingitele firmadele augu pähe, mina kirjutasin firma 
andmed kuskil serveris asunud staatilisse (!) HTMLi. Mis kasu sellest kellelegi 
ammu enne otsingumootorite laia levikut tõusta võis, on mulle siiani 
arusaamatu. Ma ei mäleta ka, et keegi seal lehel väga käinud oleks. Ometi maksti 
meile arved ära ja ma väga loodan, et tolle tegevuse käigus antud lubadused said
enam-vähem täidetud. 

Kuna teadsin Steni\index[ppl]{Tamkivi, Sten} juba varasemast ja Meelis vist ka puutus temaga kokku, 
lõpetasime ühel hetkel modemitega jantimise ja infokataloogi pidamise ning 
asusime Steni asutatud Halo\index{Halo Interactive DDB} nime all kodulehekülgi tegema. Kampa 
võeti ka mõned kunstnikud, näiteks väga andekas Oliver 
Reitalu\index[ppl]{Reitalu, Oliver} ja mitte vähem andekas Alar 
Koort\index[ppl]{Koort, Alar}, keda kutsuti ilmselt tema rajude elukommete tõttu 
Helbekeseks. Projektijuhiks oli Priit Sasi\index[ppl]{Sasi, Priit}, keda 
kõik tema joviaalse oleku ja suure habeme tõttu Sasuks kutsusid. Sasu õpetas 
mind briti punki ja Alar kurjemat sorti hiphoppi kuulama ning elu oli päris tore. 
Minu käe alt tuli Eesti esimene kommertsalustel tehtud 
(st ettevõte maksis kellelegi lehe tegemise eest raha) kodulehekülg, mis sai 
tehtud Tartu Raadiole\index{Tartu Raadio}, kui mälu ei peta. Kunstnik joonistas 
pildid valmis ja lõikas tükkideks, mina kirjutasin Notepadiga HTMLi ja nii see 
töö käis. 

Ühel hetkel hakkasime lehekülgede tekitamist automatiseerima, kirjutasime 
Perli skripte. Mõnda aega ei olnud meil ei oma serverit ega üldse kuskil Perli 
jooksutada. Siis sai programmeeritud nii, et skript läks meiliga
Unineti\index{Uninet} süsadminnile, kes kopeeris faili õigesse kohta, meie 
vajutasime brauseris nuppu, saime veateate, admin saatis meiliga konsooli 
veateated, mina parandasin koodi ja saatsin uue versiooni. Admini kannatus 
lõppes enne kui minu oma. 

Lõpuks jõudsime oma tegemistega siiski päris kaugele. Perli skriptid läksid 
järjest pikemaks ja kuna andmebaasi pidamiseks ei olnud meil serverites 
piisavalt õigusi, hoiti andmeid enamasti lihtsalt tekstifailis. Üllataval moel 
kattis see ära päris suure hulga vajadusi. Perlilt liikusime ühel hetkel PHP-le 
ja tekkis ka levinud, kuid seetõttu mitte vähem rumal mõte endale ise 
oma sisuhaldussüsteem kirjutada. See sai vist isegi valmis, aga konkreetsed 
mälestused tollest elukast puuduvad.

Ma ei mäleta, et see äri oleks kuidagi tänapäevases mõistes äri moodi välja 
näinud. Raha oli alati vähe ja seega tuli teha kõike, mille eest maksti. Kuidagi 
müüs Sten\index[ppl]{Tamkivi, Sten} Ühispangale\index{Ühispank} maha mõtte anda nende aastaraamat välja CD-l. Mis muud, kui
õppisime selgeks Macromedia Directori kasutamise ja video redigeerimise ning 
andsime minna. Ainus asi, millega me hakkama ei saanud, oli heli. Õnneks oli Sten 
hea sõber Lauri Liivakuga\index[ppl]{Liivak, Lauri}, kelle Forwards 
Studio\index{Forwards Studio} asus meiega sama koridori peal. Lauri 
tegi kenad kõllid ja plõnnid ning aitas selle kõik visuaaliga ära 
sünkroniseerida. Tulemus sai päris kena. 

Igatahes hakkas meile järjest rohkem Tallinna kliente siginema. Ühtlasi müüs Sten 
suure tüki ettevõttest Brand Sellers DDB-le\index{Brand Sellers DDB}, mis oli 
minusuguse Tartu nohiku jaoks täiesti müstiline kamp inimesi. Intelligentsed, 
säravad, jõukad (nii mulle tundus) ning andekad. Bruno Lill\index[ppl]{Lill, 
Bruno} oma terava ütlemise ja peene olekuga on siiani meeles. Nii tehti 
kampas otsus kolida kogu Halo Tallinna. 

Olin tegelikult ligi aasta üsna kahepaikne, pendeldades Tartu ja 
Tallinna vahel. Ülikoolis olid veel viimased sabad lõpetada ja 
Mari\index[ppl]{Kütt, Maria}, kellega peagi abiellusime, käis samuti veel 
koolis. Lõpuks sain oma lõputöö kaitstud ja kuna selliseks triviaalseks asjaks ei 
hakanud ju keegi Tartusse sõitma, käis Mari mu diplomit dekanaadist ära toomas. 
Prouad nõudsid allkirjastatud volitust, mis sai ukse taga kohe valmis tehtud, 
ning nii omandasingi oma esimese teaduskraadi. Tartu Ülikooli peahoone 
sammaste vahelt ei ole ma kunagi välja astunud ja kuigi toonaseid õppejõude 
hindan siiani kõrgelt, pean oma \emph{alma mater}'iks siiski Massachusettsi 
Tehnoloogiainstituuti. 

Tallinnasse kolimisega sai läbi üks etapp Halo kasvuloost. Senise boheemliku 
mis-ikka-valesti-võib-minna mentaliteedi asemel tuli hakata käibenumbritest 
rääkima. Samuti oli meeskond kasvanud. Veel Tartu päevil olin saanud omale 
elu esimese alluva, olles ühtlasi ka tema esimeseks ülemuseks. Vist veel 
keskkooli lõpetav noor nutikas tüüp aitas mul koodi kirjutada ja hängis niisama 
ringi -- ei mina teadnud, kuidas inimesi juhitakse või mida üks ülemus tegema 
peaks. Nimeks oli tüübil Taavet Hinrikus\index[ppl]{Hinrikus, Taavet}. 

Inimesi 
lisandus veelgi ja ma ei saanud enam aru, miks ja kuidas asju tehakse. Ühel ilusal päeval
leidsin kuulutuse, et Hansapank\index{Hansapank} 
otsib internetipanga meeskonda inimesi. Läksin intervjuule. Mäletan siiani 
seda tunnet, kui Liivalaia tänava pangahoone tolle aja kohta ülišiki lifti 
uksed kaheksandal korrusel avanesid ja minu ees laius hurmav vaade 
vanalinnale. Olin müüdud mees, õnneks arvas Vilve Vene\index[ppl]{Vene, Vilve}, 
kes seal majas tarkvara arendamist vedas, samamoodi. 

Nii sai minust veidi enne sajandivahetust 
hansapankur. Mul vedas kohutavalt, sest pank oli praeguses mõistes ulmeliselt 
dünaamiline asutus. Vägesid juhatas Indrek Neivelt\index[ppl]{Neivelt, Indrek}. 
Vaata Maailma programm oli just käima minemas ja sellega tegeles Tiit 
Pekk\index[ppl]{Pekk, Tiit}. Marketsi tiim eesotsas Erkki 
Raasukesega\index[ppl]{Raasuke, Erkki} pidas ülejäänud panka talumatuteks 
venivillemiteks ja tema jaoks toodeti Erik Jõgi\index[ppl]{Jõgi, Erik} juhtimisel imeilusat 
koodi. Ehitasime panga jaoks mõne aastaga mitu internetipanka ja panime ka e-riigile käed külge. 
Aga see, nagu öeldakse, on juba üks teine jutt.

Maria Klenskaja ütles ühes intervjuus ilusti umbes midagi sellist, et mõni inimene on lavale sündinud
ja mõni teeb kõvasti tööd ja, kui noot ees, hätta ei jää. Mõni on programmeerija ja 
mõni oskab koodi kirjutada. Kuulun kindlasti viimaste hulka. Võib-olla just 
seepärast ma hindan väga inimesi, kes erinevalt minust mitte ei \emph{tee}, vaid \emph{on}, ja mulle väga meeldivad päris asjad.
Samamoodi tunduvad näiteks Villu Tamme ja Freddy Grenzman päris -- minu kogemuse põhjal ei ole nad laval 
kuigi palju teistsugused kui elus. Vahest see seletab ka, miks 
seesinane raamat on sündinud.

\chapter{Jaanus Lillenberg}
\index[ppl]{Lillenberg, Jaanus}
\question{Kuidas sina said arvutite juurde ja arvutid sinu juurde?}

See sai alguse aastal 1983, kui Tartu 
Ülikoolis\index{Tartu Ülikool} tehti Nõukogude Liidu ja Jaapani koostöö 
tulemusena personaalarvutite klass.

\question{Mis arvutid need olid?}

Need olid Yamaha MSXid\index{Yamaha MSX}. Yamaha MSX kuulub samasse põlvkonda, mis Commodore 64, mõned vihasemad Sinclairid ja 
ka Apple II. Äge oli see, et nendes arvutites jooksis 
tegelikult Microsofti tava-kasutajatele mõeldud operatsioonisüsteem.

\question{Kas see oli Microsofti oma?}

MSX nagu Microsoft \emph{Extended} vist\sidenote{Lühendi päritolu kohta liigub mitmeid variante, ka asja juures olnud inimesed ei mäleta enam täpselt.}. Igatahes nägi see äge välja. Arvutiklass paiknes kooli peauksest kümne sammu kaugusel f
keldris, mille aken avanes täpselt kooliukse ette. Ühele üheteistaastasele, kes läks sellest igal hommikul ja õhtul mööda, 
oli see vastupandamatu. Selles mõttes valikuvõimalusi tegelikult 
ei jäetud. 

\question{Sa lihtsalt pidid sealt uksest sisse minema?}

Kõndisin ühel päeval otse aknast sisse, sest 
aken oli tänavaga samal tasapinnal. Küsisin, kas võib tulla vaatama, ja ära mind otseselt ei aetud. Kolmandal päeval 
andis keegi mulle MSX BASICu\index{BASIC!MSX BASIC} 
manuaali koopia. Ma küll ei saanud 
inglise keelest aru, aga mängude tegemine tundus huvitav. 
Arvutiklassis toimetasin kolm-neli aastat ja olin vahepeal ka abiõpetaja. Kirjutasin ise tekstiredaktoreid ja mänge ning loomulikult häkkisin 
lõputus koguses olemasolevaid mänge. Kirjutasin ka oma elu esimese viiruse, mis hävitas flopiketta. 

\question{Mis koolis sa käisid?}

Tartu 10. Keskkoolis\index{Tartu 10. Keskkool}, praegu on see
Tartu Mart Reiniku Kool\index{Tartu Mart Reiniku Kool|see{Tartu 10. 
Keskkool}}. Arvutiklass paiknes Vanemuise tänaval teatri vastas 
oleva õppehoone\index{Tartu Ülikool!Vanemuise tänava õppehoone} keldrikorrusel. 
Seal oli isegi kaks arvutiklassi. Teises olid 
Agatid\index{Agat}, mis olid venelaste pihta pandud 
Commodore'i või Apple II koopiad\sidenote{Agat kasutas küll sama 
6502 protsessorit, mis Commodore 64 ja Apple II, ning oli suuresti viimasest 
inspireeritud, kuid erines disaini poolest mõlemast ja otsese koopiaga tegu 
ei olnud.}. Kusjuures mul läks rohkem kui aasta, enne kui sain aru, et see oli tegelikult 
Apple II koopia. Klassis olid ka Apple 
II\index{Apple II} arvutid ja kuigi protsessori tasandil olid need sarnased, 
oli sisu väga erinev. 

\question{See nõukogude variant oli üsna industriaalse väljanägemisega.}

Jah. Kas sa tead näiteks, et kui inglise keeles on klaviatuuril vasakult paremale lugedes 
QWERTY, siis vene klaviatuuril tuleb sama moodi lugedes kokku \enquote{pidev \emph{lag}}? 

\question{Tol ajal ei teadnud veel keegi \emph{lag}'ist midagi.}

Kui sellest ajast kümmekond aastat edasi hüpata, siis olid Tartu Ülikoolis juba arvuti- ja 
terminaliklassid. Tol ajal olid arvutid nii võimsad, et neil oli 
hunnik terminale, mis moodustasid terminaliklassi. Siis oli juba ka
väga palju võrgutegevust. Arvutiklassi kõrval oli 
IBMi koopia või litsentsi alusel tehtud ES\index{ES 
EVM}\sidenote{ES EVM (\begin{russian}ЕС ЭВМ, единая система электронных 
вычислительных машин\end{russian}) oli sari IBM 
System/360\index{System/360} ja System/370\index{System/370} 
kloone. Nende riistvara põhines küll IBMi omal, kuid oli väheste eranditega 
siiski Nõukogude Liidus välja töötatud. Tarkvara seevastu oli IBMi tarkvara 
lokaliseeritud ja väheste muutustega koopia. Neid masinaid nimetati eesti 
keeles hellitavalt jessukesteks.}, Nõukogude arvuti vene klaviatuuriga ja \enquote{pidev \emph{lag}} nii klaviatuuril kui arvutis oli väga ilmne kontseptsioon.

\question{Kõik üheteistaastased, kes arvutiklassist mööda kõndisid, ometi ei roninud 
aknast sisse. Sul pidi järelikult olema tehnika- või elektroonikahuvi.}

Ei olnud, ma käisin hoopis ratsutamistrennis. Aga mõni 
asi on kohe visuaalselt uus ja lahe ning vastandub 
kõigele muule ümbritsevale. Kujuta ette, et lähed mööda näiteks
lendamistrennist, kus inimest õpetatakse lendama. Sa ei hakka ju arutama, et ma pidin minema malet mängima või telekat vaatama. 
Lendamine on universaalselt väga \emph{cool} asi, kõigist 
teistest asjadest kümme korda kõvem.

\question{Ja siis ei oskagi pärast hästi seletada, miks sulle 
lendamistrenn meeldis ja miks sa ei läinud malet mängima.}

Lihtsalt kaldusid teelt kõrvale. Muide, ma ratsutasin neli aastat, see ei seganud.

Võtsin ühe klassivenna ka arvutiklassi kaasa. Mäletan, kuidas me arutasime omavahel, kuidas mänge tehakse. Kuidas Assembler või 
masinkood näeb nii suvaline välja ja äkki on 
võimalik \emph{random} kombinatsioone katsetades saada 
lahedaid mänge. Mõtlesime küll, et see vist ikka ei ole tõsi, 
aga oleks äge, kui nii saaks! Katsetad kümmet tuhandet 
kombinatsiooni, kõikvõimalikke koodivariante ja vaatad, milline läheb käima 
ja milline mitte. Õnneks nädal hiljem olime juba \emph{Hello 
World}\sidenote{Siiamaani peetakse oluliseks, et uut programmeerimiskeelt katsetades luuakse esmalt programm, mis väljastab kuhugi teksti \enquote{Hello World}. Traditsioon pärineb kuulsast The C Programming Language raamatust\index{The C Programming Language}.} kirjutanud ja asjad läksid natuke selgemaks. MSX 
BASICust\index{BASIC!MSX BASIC} kasvas muide välja Visual 
Basic\index{Visual Basic}, nii et Visual Basicu õppimine oli meie jaoks 
\emph{what else is new}.

\question{Kes seda arvutiklassi vedas? Pidi ju olema keegi, kes sind aknast sisse 
lasi ja kohe välja ei visanud.}

Mind visati sealt mitu korda välja, aga nad olid oma väljaviskamises 
tunduvalt vähem veenvad kui mina sisseronimises. Ma ei osanud väljaviskamise peale
kuidagi solvuda või seda pahaks panna. 
Sain ju aru, et see klass ei ole minu jaoks tehtud. Näiteks ronisin 
mitu korda sisse õpetajate täiendkoolitusele, kus tegelikult ei õpetatud 
arvuti kasutamist, vaid seda, et maailm muutub ja et arvutiõpe on 
hea käegakatsutav asi seda muutust kirjeldama. Selle visiooni taga oli üks 
väga vinge inimene, Anne Villems\index[ppl]{Villems, Anne}.

Anne Villems on tohutult kirglik, tema kirg on maailma 
paremaks teha. Õpetame neid inimesi, kes õpetavad teisi! Näitame neile ja nemad
näitavad väikestele inimestele, milline maailm võiks olla! 
Arvan, et tema täienduskoolitustele jõudnud inimesed 
olid mõnes mõttes juba paremad. Nad suutsid endale 
sõnastada, et peaksid sinna minema, sest äkki maailm muutub sellest paremaks. Need, 
kes koolituse läbi tegid ja seal omavahel suhtlesid, 
olid üks suur rest kive selles vundamendis, mille peale meie IT-riik on 
ehitatud.

\question{Õpetajad õpetasid omakorda õpilasi, kellest said abiõpetajad, ja nii see teadmine levis.}

Bingo! Koolitusel oli näiteks
üks Tartu Kunstikooli\index{Tartu Kunstikool} õpetaja, kes hiljem tõi oma lapsed 
arvutiklassi tundi pidama. Kunstikooli õpilastel oli üks
joonistusprogramm – 64 värvi, maa ja ilm. Ja nad tegid arvutiga päriselt mingeid asju, olgugi et printerit kahjuks ei olnud. 
Igatahes said nad tunnetuse, kuidas kontseptuaalselt täiesti uuel viisil kunsti teha. 

Isegi mina, kes ma mõtlesin primitiivselt, kuidas saaks mänge teha ja 
mängida, jõudsin lõpuks kuhugi välja. Aga nemad võtsid 
graafilise \emph{editor}'i ja tegid sellega võimsaid
asju\ldots Samasugune trikk nagu iPhone'i tulek: me ei teadnud algul, kui kõva asi see oli, aga kindlasti 
sajal ägedal moel. Omal ajal oli sama lugu personaalarvutite ja MSXidega.

\question{Iga uudsus läheb ju lõpuks üle, kas sinu jaoks arvutite puhul ei 
läinud üle?}

Ei läinud, see liikus baastasandilt järgmisele tasandile. Toon ühe näite kaks aastat hilisemast ajast – Tõravere observatooriumi\index{Tõravere observatoorium} 
astrofüüsikud, kes olid maailmaga hoopis teistsuguses kontaktis kui 
koolipoisid. Üks selline oli minu alumine naaber Enn 
Kasak\index[ppl]{Kasak, Enn}. Ühest küljest olid kontaktid 
teadusmaailmaga, aga teisest küljest maailm sulas ja oli 
võimalik bisnist teha. Nad tõid endale Amiga 
500\index{Amiga!Amiga 500}\sidenote{Tuntud ka kui A500, oli Amiga 500 
koduarvuti 1987. aastal Commodore'i poolt turule toodud professionaalse Amiga 
2000 vaste. Tegu oli populaarseima Amiga mudeliga, eriti Euroopas.}, mis olid 
järgmine põlvkond Commodore 64st. Sellel oli kümme 
korda võimsam protsessor ja hoopis teisest klassist graafika. 
Kui MSXi Z80 protsessor võimaldas kolmehäälset muusikat teha, siis Amiga 
suutis pakkuda kuutteist kanalit. Tänapäeva mõistes oli 1986. aastal võimalik täielik MIDI-lahendus kodus püsti panna. Oi, kuidas 
ma seal nende Amigadega muusikat tegin! Täiesti häbitult ja ööde kaupa.

\question{Kas sul muusikahuvi oli enne olemas või tekkis koos Amigadega?}

Igal inimesel on mingisugune arvamus, kas talle meeldib muusika või mitte. Osaliselt on see seotud sellega, 
kas pead viisi või ei pea. Kui mind ei võetud esimeses klassis lastekoori, siis sain aru, et mulle meeldib muusika, sest ma olin väga kurb. 

Kui ma Yamahadega tegelesin, siis see ei olnud ainult mängimine. 
Meil oli täiesti mitteametlik arvutiring: 
kutid vajusid iga päev pärast kooli kohale ja enne ära ei läinud, kui välja 
visati. Klassis tegutsesid tegelikult 
üliõpilased, näiteks Ain Sakk\index[ppl]{Sakk, Ain}, Alar Pandis\index[ppl]{Pandis, 
Alar} ja mõned teised kutid, kes jätkasid pärast ülikooli lõppu vist ka pedagoogidena. Nad olid lastesõbralikud ja toetasid meid. Meil oli võimalik seal käia sellepärast, et
arvuteid oli klassis viisteist tükki, aga 
täiendkoolitustel enamasti seitse kuni kümme inimest, nii et 
alati olid mõned arvutid vabad. Asi toimis põhimõttel „kes ees, see mees“. Kui arvuti said, siis enam seda ära ei 
andnud. Koolituse ajal muud võis teha, aga mängida mitte, nii et ootasime, hambad ristis, mingi \emph{manual} 
kõrval, mille järgi proovisime asju teha. 

MSX BASICuga\index{BASIC!MSX 
BASIC} sai samuti muusikat teha: noote ritta seada, 
rütmi kiiremaks ja aeglasemaks sättida, oktaavi muuta ning vist ka näiteks 
kolm erinevat meloodiat kokku panna. Ühetoonilist muusikat sai 
kindlasti teha – kuulasin midagi ja proovisin 
järele teha. Amiga oli selle kõrval hoopis teine tera.
Erinevus oli sama suur, nagu panna endale papist 
tiivad külge ja mängida lennukit või minna päris lennuki peale. 
Ühel juhul paned teksti-\emph{editor}'is nooditähti paika, mängid selle ette ja kuulad. Teisel juhul on täisgraafiline muusika-\emph{editor} koos nootide ja digiklaveriga, mida saab arvutiklahvide peal mängida ja salvestada nii nagu tänapäeval. Pluss sadu pille, mille seast valida, mis 
kõlasid küll digipiiksudena, aga mis olid nii ära tuunitud, et viiul ja klaver kostsid kõrvale ikkagi erinevalt.

\question{Selleks et suuta kõrva järgi muusikat järele teha, peab kõrva olema. Kas sul oli muusikaline kuulmine olemas?}

Midagi oli jah. Ega noodid ju kõik õiged pruukinud olla, aga rõõm tegemisest oli suur! Iga kord, kui midagi 
natukenegi välja tuli, viskas see puid alla juurde ja leek läks suurema hooga põlema.

Tartus oli selline võimas 
organisatsioon nagu Tartu 
Tähetorn\index{Tartu Tähetorn}, aga infotehnoloogilise ajaloo prismas oli see ainult väike ripats Eesti 
Biokeskuse\index{Eesti Biokeskus}\sidenote{Eesti Biokeskus moodustati 1986. 
aastal Tartu Ülikooli\index{Tartu Ülikool} ja KBFI\index{KBFI} ühisasutusena.} 
küljes, mis oli tähetorni kõrval väikene kuut, aga kus toimusid 
ülisuured asjad. Tähetorni katusele oli hea panna \enquote{satipann}: sealt paistis kaugele, puid ümber ei olnud ja
signaal oli alati hea. Eesti kahest
esimesest internetiühendusest üks oli Tallinnas KBFIs\index{KBFI} ja teine, Tartu oma, paikneski 
tähetornis, õigemini biokeskuses, mille ruumid olid tähetorni 
lähedal. Biokeskuses tegutses Richard Villems\index[ppl]{Villems, 
Richard}\sidenote{Eesti Biokeskuse juht selle asutamisest 
alates.}, kes koos Lippmaadega üldse selle interneti-maailma Eestile avas.

Igatahes Amigad jõudsid tähetorni ja ühendati internetti, sest kõik, kes tee peal olid, istutasid ennast ka selle traadi peale, mis enne biokeskust katuselt 
alla tuli. 

\question{Kui veel ajas tagasi minna, siis sul pidi tublisti distsipliini olema – koolituse ajal 
taganurgas istudes tuli ju vagusi olla.}

Ma mõtlesin välja sellise asja nagu võtmeluba: mulle anti klassi võti. Kuna nädalavahetustel koolitusi ei toimunud ja valvur ärkas kell seitse üles, 
siis selleks hetkeks sai ukse taha mindud. Ukse avas väga unine valvur, kes alguses ei uskunud, et mul on mingi võtmeluba, ja ajas mind minema. 
Aga kui olin juba kell seitse hommikul kohale läinud, siis ega ma sealt ära ei 
läinud. Palusin süüdimatult helistada pühapäeva hommikul mingitele inimestele, et need võtmeloa olemasolu kinnitaksid. Üksikud uued valvurid ei lasknud sisse, aga paari-kolme kuuga 
olid nad kõik välja õpetatud.

\question{Sest jama ei tekkinud, keegi ei läbustanud ja midagi ei 
varastatud.}

Läbustamiseks polnud aega. Ainukene jama oli vaba arvuti saamine: inimesed ootasid arvutiruumi ukse taga koolituse lõppu, et äkki järgmisel koolitusel on auk ja pääseb sisse. 

\question{Miks see luba just sulle anti? Kas paistsid kuidagi silma? 
Olid eriti tubli, korralik, pealetükkiv?}

Kõik see kokku. Samas tegin ma tänapäeva 
mõistes vabatahtlikku tööd. Tahtsin nii väga olla arvutite juures, et olin nõus tegema koolituste ajal abiõpetaja tööd, 
oma vabast ajast ja ilma rahata. 
Seal õpetati ülilihtsaid oskusi, mille laps omandab paari-kolme 
päevaga, nagu arvuti käimapanek ja mis tähendab \emph{press any key}. Mul ei olnud probleemi näidata tädidele, kuhu tuleb 
vajutada. Tädidel oli ka hea meel, et lapsed oskavad seda teha. Ja Anne Villems\index[ppl]{Villems, Anne} ei visanud ka mind välja eriti. 

\question{\enquote{Eriti}...}

Ma ei tea, kui palju oli sealpool seda, et nad ei 
jaksa enam võidelda ja ei ole mõtet välja visata. Meid oli vist kolm, kellel oli võtmeluba. 

\question{Seda on ikkagi vähe.}

Kõik ei jaksanud kogu aeg käia ega mahtunud ka. Eks visamad lõpuks jäid. 

\question{Kas sa käisid seal kuni keskkoolini?}

Jah. Keskkooli läksin teise kooli, 
Treffnerisse\index{Hugo Treffneri Gümnaasium}. Seal olid küll
oma arvutiklassid, aga siis oli juba oluline tähetorn\index{Tartu Tähetorn}. Seal olid Amigad, seal lindistasime esimesed lood, mu naaber töötas seal, käisin astronoomiaringis
õppisin C-d\index{C} kirjutama. Seda õpetas mulle Kaur Virunurm\index[ppl]{Virunurm, Kaur}, 
ainuke tüüp, kes suutis, sest see, mida me seal tegime, on maailma kõige halvem 
õppimismeetod. Kujuta ette, et sinu kõrval on inimene, kes tahab mingit asja 
õudselt osata, aga ta ei oska mitte midagi ega viitsi \emph{manual}'i 
lugeda. Põhimõtteliselt sind muudetakse elavaks \emph{manual}'iks ja iga kahe-kolme minuti tagant küsib õpilane, et \emph{are we there yet}.

Tartus oli teine keskus veel, füüsikahoone\index{Tartu 
Ülikool!Füüsikahoone} Tähe tänava alguses. Seal toimetas Taavi 
Talvik\index[ppl]{Talvik, Taavi}, kes andis mulle ühe C 
\emph{manual}'i, mis oli vist fotoaparaadiga üles pildistatud.

\question{Kas see võis olla Richie kuulus sinise C-ga raamat\index{The C 
Programming Language}\label{sisu:richie}?}

Jah, aga see oli ainult must ja valge, sinist ei olnud seal midagi. Lehitsesin kümme-viisteist aastat hiljem neid 
üksikuid fotokoopiaid ja vaatasin, et päris hea kraam. 
Ega ma tollal väga palju asju C-s\index{C} ei kirjutanud, aga
hiljem küll. Igatahes Kaur\index[ppl]{Virunurm, Kaur} jaksas minu tohutut huvi taluda. 

Tol ajal olid ilmunud juba esimesed XTd\index{XT} ja tolle 
Assembler\index{Assembler} oli hoopiski teisest klassist kui Z80 Assembler – kaheksa-, mitte neljakohaliste koodidega! 

Ühel hetkel lõi aga murdeiga sisse ja arvutid ei võtnud enam sada, vaid seitsekümmend protsenti ajast. 

\question{Tavaliselt tekib inimestel keskkooliajal mingisugune kultuuriline 
kontekst – muusikat sa mainisid, aga raamatud? Veel midagi?}

Väga hea, et sa selle välja tõid. Me peame aru saama, millisele lavale 
idud kasvama läksid. Tartus oli akadeemiline keskkond ja see tähendas 
enamasti kõrget lugemust ning paremat kirjandusega kursis olekut.

Kasakul\index[ppl]{Kasak, Enn} oli tolle aja kohta täitsa hea raamatukogu, 
samuti mu tädil. Autoritest oli Asimov
kindlasti kõva sõna. 

Mu vanaema oli tõlkija, ta tõlkis kuuekümnendatel näiteks teose \enquote{I, 
robot} eesti keelde. Ta vist tõlkis kellegagi koos terve Asimovi 
kogumiku. Teine väga tugev liin oli sõrmused ja nende 
isandad. 
Võibolla mõtlen natuke üle, aga „Sõrmuste isand“ on tegelikult lugu suurest ja palju tugevamast kurjusest, mille 
vastu ei saa. Mõtle, mis aastad need olid, 1987–1989! See lootus! Need 
raamatud sobisid hästi sellesse aega.

\question{„Kääbik“ oli jah eesti keeles olemas, aga mina sain teada, et see on osa 
suuremast loost, alles üheksakümnendate lõpus. Kas sul olid ingliskeelsed 
raamatud?}

Jah. Need olid erilised, keskmise vene papitrüki kõrval läikivad ja ilusad. Kuidas tunda ära 
inimesi, kes olid tollal tegevad? Neil kõikidel on kodus nähtaval kohal kogu Tolkieni looming.

Ulme oli teine liin, näiteks Asimov ja Bradbury\index{Ray Bradbury}. Osa teoseid avaldati Mirabilia sarjas\sidenote[][-2cm]{Mirabilia oli 
kirjastuse Eesti Raamat aastatel 1973–2012 ilmunud raamatusari, mis 
keskendus peamiselt ulme- ja kriminaalromaanidele. Omal ajal oli tegu 
suurepärase võimalusega tutvuda üldjuhul väga hästi 
tõlgitud ulmekirjanduse klassikaga: Simaki, Lemi, Strugatskite, 
Asimovi, Bradbury ja paljude teiste romaanide ja lühilugude kogumikega. Paljuski 
kujundas just see sari terve põlvkonna ulmehuviliste maitse ja lääne klassikute 
hulgas ilmus ka Eesti, Soome ja Läti autorite loomingut.}. 

\question{Aga Strugatskid?}

Loomulikult nemad ka\index{Strugatskid} ja Stanisław Lem\index{Stanisław Herman 
Lem}\sidenote{Stanisław Herman Lem (1921–2006) oli Poola ulmekirjanik, kelle 
teosed olid ühekorraga nii filosoofilised kui ka satiirilised ja 
humoorikad.} ja teised. Oli selline ulmekirjanduse kogumik nagu \enquote{Lilled 
Algernonile}\sidenote{Ilmus aastal 1976 sarjas \enquote{Ajast Aega}.}, see oli ka suhteliselt kohustuslik kirjandus. Kui 
tütarlastel oli võibolla kotis Herman Hesse „Stepihunt“, siis poisid raamatut 
kaasas ei kandnud, aga kuskil oli neil natuke äranäritud nurkadega \enquote{Lilled 
Algernonile}. USA ulmekirjanduse sissevool lõi toimuvale tõepoolest 
kultuurilise tagapõhja.

Muusikaga oli teistmoodi. Suured inimesed kuulasid suurte inimeste 
muusikat, noored noorte muusikat. Tähetornis\index{Tartu Tähetorn} 
kuulati palju muusikat maki pealt. Seal olid koos naljakad 
kooslused: esiteks tähetorn ise, siis 
füüsikatudengid, kes pühendusid astronoomiasuunale (näiteks Kaur 
Virunurm\index[ppl]{Virunurm, Kaur}), ja teisalt tähetorni 
direktori lapsed, kes käisid Miina Härma 
Gümnaasiumis\index{Miina Härma Gümnaasium} ja vedasid sinna 
omi koolivendi ja -õdesid. 
Tartus oli tol ajal kaks kooli, kes defineerisid, mis on äge: 
Treffner\index{Hugo Treffneri Gümnaasium} ja Miina Härma ning mõlemad 
arvasid, et on parem kui teine. Tartu värk. Tähetornis olidki 
ka mõned Treffneri tüübid. Sedasi tekkis segu keskkoolist, 
ülikoolist ja internetist, millest ei saanudki tulla mitte midagi peale plahvatuse. Seal oli tüüpe, kes on tänapäeval Eestis kõik
väga asjalikud.

\question{Keskkoolinoorena astrofüüsikutega sammu pidada ja originaalkeeles 
Tolkieni lugeda ei ole lihtne. Sa pidid ikka nutikas 
inimene olema.}

Mul oli sõnaraamat kõrval, kuni ühel hetkel polnud seda
enam vaja. Treffneris\index{Hugo Treffneri Gümnaasium} 
oli gümnaasiumis bioloogia-keemia õppesuund, kus õpetati ladina keelt, 
tavalist inglise keelt, aga ka teaduslikku inglise keelt, mille 
kõrval tavaline inglise keel oli \emph{walk in the park}. Seda ainet andis 
muide bioloogiaõpetaja\sidenote{Õpetaja Tago Sarapuu\index[ppl]{Sarapuu, Tago} 
õpetas ka Meelis Roosile\index[ppl]{Roos, Meelis} bioloogiat.}, kes oli tugeva akadeemilise taustaga ja 
tegi hiljem pikalt akadeemilist karjääri. Sa 
mainisid sinise kaanega C-õpikut. Kujuta ette, et sul on näiteks geneetika 
kohta samasugune ning sa võtad ja närid ennast sellest lihtsalt läbi. Paberit lendab 
kahele poole, aga sa tõlgid selle kõik ära. See aitas hiljem väga hästi kaasa.

Meil oli saksa keel ka, ma sain saksa keele lõpueksamil kooli parima hinde. 
Aga miks? Sellepärast, et Kaur Virunurmel\index[ppl]{Virunurm, Kaur} oli samal 
ajal ülikoolis saksa keele eksam. Ta tõmbas sõnaraamatu arvutisse, tegi sellest baasi ja siis oli 
võimalik skoorida: õige vastus andis punkti, vale vastus miinuspunkti. Mina 
valmistusin saksa keele eksamiks niimoodi, et viimasel õhtu enne eksamit mängisin punktide peale saksa keele sõnade tõlkimist, ja 
see aitas mind väga hästi. See oli üks esimesi kordasid, kus 
võin kindlalt väita, et infotehnoloogiline tööriist parandas mu sooritust 
hüppeliselt. Lõpuklassis oli mul küll saksa keel vist ühel veerandil ka kaks, aga 
see oli ealine iseärasus, hinded ja teadmised ei ole alati 
omavahel lineaarses seoses.

Kui nüüd muusika juurde tagasi tulla, siis Miina Härma \index{Miina Härma Gümnaasium} kutid tõid tähetorni
The Smithsi, The Cure'i ja 
ka vene muusikat. Ilmusid välja kitarrid ja 
midagi ka lindistati, ilmselt kassettidele. 

Samas oli tähetornis nõukogudehõnguline teaduskultuur, mille juurde 
käis näiteks konjaki ja kohvi joomine. Keskkooli- ja 
üliõpilased ei saanud seda küll endale lubada, aga tubades oli see hõng üleval. Ma ei 
teagi, mis asjaoludel seal joodi, sest läbusid 
ei toimunud, aga see kõik tekitas erilise atmosfääri.

\question{Seal tehti ju teadust ja mitte halba.}

Just. Ma käisin isegi astronoomiaringis ja mind saadeti 
oma tööga kuskile rahvusvahelisele
õpilaskonverentsile esinema. Kõike sai teha.

\question{Kuhu sa pärast keskkooli õppima läksid?}

Tahtsin minna ülikooli ajakirjandust õppima. Ülikooli mitteametlik \emph{statement} oli see, 
et kui tahad tulla ajakirjandust õppima, siis pidi olema ette näidata portfoolio ehk
pidid olema midagi avaldanud. Ajakirjanduse või üldse meedia õpetamine on 
suhteliselt kallim tegevus kui näiteks keeleõpe ja nad tahtsid olla veendunud, et üliõpilane tõesti tahab ajakirjandust õppida, mitte ei astu 
juhuslikult sisse. Mulle tundus, et kultuuriajakirjanik on äge olla. Ilmselt selles vanuses 
arvab iga mees, et kultuuriajakirjanik on äge olla, sest on olemas oma arvamus maailmast, mille masstiražeerimine tundub 
veidral kombel teiste aitamisena.

Ühesõnaga, tegin ettevalmistusi: käisin 
kontsertidel, kirjutasin intervjuusid ja arvustusi. 
Aga samal ajal käisin ka näiteringis. Lõpuklassis olid veebruaris-märtsis 
Viljandis lavaka sisseastumiskatsete 
eelvoorud. Mõtlesin, et äge oleks minna Viljandisse trallima ja 
seiklema, aga sain eelvoorust edasi ja hiljem lavakasse sisse.

Ma ei pabistanud üldse ja eks see aitas. Teadsin, et mul on
ajakirjandusega plaanid ja head soovituskirjad 
toonastelt tuttavatelt Postimehe\index{Postimees} ajakirjanikelt. Õppisin 
lavakas ligi aasta, ent ühel hetkel sain aru, et see ei ole ikkagi see, mida ma teha tahan. 
Sellele otsusele jõudmist mõjutas kõik eelnevalt kirjeldatu 
väga tugevalt. See „lendamistrenn“ paistis kogu aeg aknast. 

Kooli kõrvalt sattusin sellisesse ägedasse kohta nagu Riigikogu 
Kantselei\index{Riigikogu Kantselei}.

\question{Kas sel ajal olid seal juba võrgud ja BBSid?}

Riigikogu Kantseleis oli täiesti adekvaatne kraam juba aastal 1992. 
Ühtlasi üks äge tekstipõhise kasutajaliidesega võrgumäng, 
põhimõtteliselt tolle aja Fortnite, mille nimi oli 
MUME\sidenote{Üks populaarsemaid MUD-tüüpi mänge, mille nimi tuleneb fraasist 
\emph{Multi-Users in Middle-Earth} ja mis põhines 1991. aastal loodud ja siiani 
aktiivselt arendataval DikuMUDi\index{Muda!DikuMUD} mootoril. Vt ka märkust \ref{sidenote!muda} 
lk \pageref{sidenote!muda}.}, tegu oli Tolkieni ainetel loodud 
Mudaga\index{Muda}. Tuletan meelde, et need ringkonnad olid kõik 
tugevalt Tolkieni usku.

\question{Kas seal oli server, kuhu mängijad külge läksid?}

Jah. Telneti pordi kaudu tõmmati sind külge, kõik istusid oma 
\emph{socket}'is\sidenote{St. omasid iseseisvat ühendust serveriga.}, aga nägid, mida teised teevad. Ja kuna see oli 
tekstipõhine, siis võrguühenduse kiirus ei olnud probleem.

\question{Kuidas adekvaatne kraam Riigikogu Kantseleisse\index{Riigikogu Kantselei} 
sai? 1992. aastal ei olnud Eesti Vabariik veel kuigi 
heal järjel ja oli muudki, mida korrastada.}

Väga hea küsimus. Tarvi 
Martens\index[ppl]{Martens, Tarvi} kindlasti teab seda. Samuti
Toomas Mölder\index[ppl]{Mölder, Toomas}, kes oli nlibi, 
tollase Rahvusraamatukogu\index{Rahvusraamatukogu} IT-juht tänapäeva mõistes. Ja eks KBFI\index{KBFI} rahvas aitas ka.

\question{Kas peale MUME'i mängimise tegite seal kasulikke asju ka?}

Ma käisin tol ajal lavakas ja teadsin küll, et nad teevad 
vingeid asju, aga tavaliselt siis, kui mina sinna saabusin, 
lõppes töö ära, sest tuli \emph{orc}'ideks kehastuda ja minna 
\emph{whiteskin}'e tapma. MUME'i tekitatav adrenaliinitase ei jäänud alla 
tänapäevaste arvutimängude omale. Näiteks olid seal haldjas ja sulle tuli 
teade: \enquote{\emph{An orc enters the room}.} Selle peale lükkab ka täna teatud 
seltskonnal vererõhu kakskümmend protsenti ülespoole. MUDe oli veel, aga 
MUME oli üks esimesi selliseid mänge, mis kestis aastaid. Esimesed 
eestlased, kes seal mängisid, tegid oma tegelased aastal 1991 või 1992 ja mängisid kolm-neli aastat järjest. Pronto\index[ppl]{Pronto} ehk Tanel Raja mängis muide 
väga kõvasti MUME'i. 

\question{Mäletan, et 1993. aastal sisenesid mingid inimesed Liivi tänaval 
VAXi klassi\index{Tartu Ülikool!Liivi Õppehoone!Vase klass}\sidenote{\label{sidenote!vaks}Sõna \enquote{vask} mitmesuguste 
variatsioonidega kutsutud ja ilmselt klassi toitnud arvuti tüübinime 
VAX\index{VAX} järgi 
nime saanud klass asus matemaatikateaduskonna Liivi tänava 
õppehoone esimesel korrusel ja koosnes vask.ut.ee\index{vask.ut.ee} 
külge ühendatud terminalidest.} ja kui mina ükskord ülikooli lõpetasin, siis nad 
olid ikka veel seal.}

Jah, \enquote{pidev \emph{lag}} oli muide sealsamas VAXi klassi kõrval olevas 
ES-klassis, mis oli alati tühi, kuna need arvutid olid jamad\sidenote{Ilmselt 
peab Jaanus silmas Raua\index{raud.ut.ee} klassi, mis käis päris IBMi 
riistvara, mitte ESi peal. Raul Tölp\index[ppl]{Tölp, Raul} meenutab: „Sattusin kas 
1996. või 1997. aastal Liivi 2 hoonesse, kus mul paluti IBMi esindajana teha raud.ut.ee 
serverile masina viimane \emph{power off}. Räägiti, et masin küttis 
vesijahutusega tervet maja.}. Liivi tänava 
VAXi klass on omaette peatükk, milleni kohe jõuame.

Kui nüüd õpingute juurde tagasi tulla, siis lahkusin lavakast esimese kursuse viimasel veerandil. Üheksateistaastane laseb end
välisest tugevalt mõjutada. Tollal erines
näitlejaamet sada protsenti sellest, mis see täna on. 
Oli nädalaid, kui jõin vähemalt pool pudelit viina päevas. 
Organism oli tugev, vedas ilusti välja, aga nägin kutte, kes olid 
seda kümme aastat teinud. Ühel hetkel küsisin endalt, kas ma jaksaksin 
ja tahaksin niimoodi elada. \emph{Hell no}! See ja „lendamistrenn“ akna taga aitasid äratuleku otsust teha. 

Siis läksin Tartusse\index{Tartu Ülikool} eesti keelt, täpsemalt 
arvutuslingvistikat õppima.

\question{Kes seda õpetas? See oli tollal mujal maailmaski suhteliselt uus ala.}

Nüüd jään vastuse võlgu. Seal oli üks lahe lühike vanamees, kes 
oli tõeline guru. Hästi viisakas, vaikne ja rahulik sell, nii palju kui 
mina temaga suhtlesin. Aga tema juurde jõudsin alles kolmandal aastal pärast 
spetsialiseerumist. Enne olime lihtsalt ühes toredas teaduskonnas, kus 
põhiliselt õppisid tüdrukud.

Selle taustal oli mul ikkagi tunne, et peaksin ka mingit tööd tegema. Samas olin
ainult „lendamistrennis“ käinud. Mängu tuli seesama Vase klass.

Mängisime seal \emph{StackMUD}i\index{Muda}. Stacken.kth.se\index{stacken.kth.se} oli Rootsi 
Kuningliku Tehnikaülikooli\index{Rootsi Kuninglik Tehnikaülikool} 
VAX\index{VAX}\sidenote{Virtual Address Extension – arvutisari, mille töötas välja DEC
seitsmekümnendate keskel. Siiani üks tuntumaid omalaadseid arhitektuure,
oli see PDP-11\index{PDP-11} edasiarendus, peamiselt mälu virtuaalse
adresseerimise suunas.}, mille peal 
jooksis BSD\index{BSD}, mille peal pandi käima Muda. 
Originaalne DikuMUD on muidu tehtud Taanis.

Mängimise tegi huvitavaks see, et mängijad ei olnud matemaatika üliõpilased, 
nagu oleks võinud arvata, vaid eesti keele ja usuteaduse 
üliõpilased. Näiteks praegune kirjanik ja usuteaduskonna õppejõud Meelis 
Friedenthal\index[ppl]{Friedenthal, Meelis} oli väga originaalne mudamängija. 
Oli teisigi tüüpe, kes käisid tõesti palju mängimas, mina 
sealhulgas. Suhtlus selle seltskonnaga ei piirdunud ainult 
mängimisega, me ka ehitasime seda maailma. Ma olin üks põhilisi ehitajaid.

\question{Kuidas see käis? Kas kirjutasid koodi või skripti?}

See oli väga lihtne: sain koodi koopia ja mul oli andmebaasi struktuur, kus 
täitsin väljad ära. Võtsin andmebaasi \emph{dump}'i 
teksti \emph{editor}'is lahti ja tegin näiteks ridadest sada 
kuni tuhat koopia ning kirjutasin sinna asjad teistmoodi. \emph{Editor} oli loomulikult vi\sidenote{Unixi spetsifikatsiooni osaks 
saanud, 1976. aastal kirjutatud tekstiredaktor, mis on siiani teatud 
ringkondades (ka käesolev tekst sünnib osalt vi abil) väga populaarne. Siin 
kontekstis on oluline, et erinevalt tänapäevastest tekstiredaktoritest ei 
olnud vi ainult tekstipõhine, vaid ka suhtles kasutajaga 
ähmaste käsu- ja klahvikombinatsioonide abil. Näiteks on 
legendaarsed algajate kasutajate tulutud katsed redaktorist väljuda, kuna 
selleks kasutatavad klahvikombinatsioonid \texttt{ZZ}, \texttt{:q!} ja veel kümmekond 
samalaadset ei ole just intuitiivsed.}. Kirjutamine käis tsoonide kaupa: ühes tsoonis oli 
sada ruumi ja iga ruumi kohta kirjutasin ingliskeelse kirjelduse. Mitmesuguseid asju sai 
ruumis olla vist kuni 255, kolle sai ka olla teatud
kogus. Täitsid kõigi ruumide kohta statistika ära, lisasin kirjeldused ja
tegevused ning postitasin. Kutid kompileerisid selle ära ja nii see tuli. 

Me saime Rootsi mängutegijatega üsna hästi läbi ning rääkisime 
vahel ka olmest ja inimlikest teemadest. Näiteks et meil on ainult üks klass ja 
seegi on kogu aeg pooltäis, ja kui mõni ahv FTPga tont teab mida tõmbab, ei 
saa üldse mängida. Nemad ütlesid, 
et neil on üks arvuti üle, kuna said uuema VAXi\index{VAX}, mille peal on 
BSD\index{BSD}. Ma küsisin, kas me vana arvutit kuidagi endale ei 
saaks, ja öeldi, et saate küll. 

\question{See masin oli ju Rootsis.} \label{sisu!jaanus_liivi_tn}

Jah. Anne Villems soovitas mul rääkida Otto Telleriga\index[ppl]{Teller, Otto}, kes oli vist 
arvutiteaduse õppetooli juht\sidenote{Tõenäoliselt toimetas Otto Teller siiski Tartu Ülikooli 
arvutuskeskuses\index{Tartu Ülikool!Arvutuskeskus}, mis oli 
eraldiseisev üksus.}, ja ütlesin, et ilmselt te mind ei usu, aga Rootsis on üks arvuti. Sellega tuleb kaasa 16 terminali, 
me saaksime teha terve klassi ja võin ise seda 
administreerida. (Kui muda sees mängid, siis oled kaelani porine, st 
süsteemide administreerimise oskus tekkis iseenesest.) 
Aga ma olen lihtsalt üliõpilane ega tööta siin, palun aidake. Lõpuks Otto Teller vastas, et see on küll kõik
väga imelik, aga olgu, ja ajas asja korda.

Siis kirjutasin igale poole kirju ja sain vastuseid. Ma ei tea, mida Otto 
Teller tegi, aga ta hüppas pea ees tundmatusse mingi kahekümneaastase 
kuti kätega vehkimise peale. 1993. sügisel sõitsime Rootsi klassi 
üle vaatama ja talvel oli korraga üks furgoon 
Laia tänava ukse taga. Tekkis Laia tänava arvutiklass\index{Tartu Ülikool!Laia tänava 
arvutiklass} ja mina sain selle administraatoriks. See oli minu 
esimene töökoht.

Tollal ei olnud administraator ainult tehniline töötaja, vaid ka 
administratiivne tegelane, kohalik jumal. Oleks mul võimuiha olnud, 
võinuksin seda väga hästi realiseerida, aga ma andsin hunnikule tüüpidele 
võtmed ja palusin, et nad serveriruumi ei läheks. Kuskilt 
veeti eraldi kaablid, Zyxeli\sidenote{1988. aastal 
Taiwanil asutatud Zyxel Communications Corporation tootis 
ülipopulaarseid ja hinnatud modemeid.}\index{Zyxel} modemid said üle püsiliini 
internetiühenduse ja seal me müttasime.

Olgem ausad, tänapäeva mõistes oli see klass totaalne õnnetus. Nii 
madala käideldavusega asja pole ma hiljem näinud. Arvuti oli väga 
vana, läks tihti katki ja ma ei tundnud nii professionaalset raudvara hästi. Mul oli abiks Viljo 
Soo\index[ppl]{Soo, Viljo}, kes oli tänapäeva mõistes \emph{sysadmin} ja kes 
aitas masinat palju kordi käima panna. Läks paar kuud ja 
saime klassi tööle. Klassi nimi oli Cure\index{cure.ut.ee}, 
The Cure'i järgi. 

Kunagi oli selline mäng nagu Nethack\index{Nethack}\label{sisu:nethack} ja sellest oli 
üks naljakas kloon tehtud. Otsustasin selle eesti keelde 
tõlkida. See käis põhimõtteliselt samamoodi: võtsin C koodi lahti ja 
hakkasin esimesest reast lugema.

\question{NetHacki lähtekood on niisamagi hea lugemine, see on üks kahest, 
mida ma olen oma elus lugemise eesmärgil välja trükkinud. Teine on Perl.}

Ühesõnaga ma tõlkisin kõik eesti keelde esimesest reast viimaseni, kaasa arvatud \emph{library}'d ja kõik muu, mis kaasas oli. Põhilise ekraaniteadete osani jõudsin 
kell neli hommikul. Teadupärast tekib väga suure väsimuse korral ühel hetkel veider pooleufooriline meeleolu. Mul saabus see hetk keset tõlget ja mäng kukkus
naljakas välja\sidenote{Mängus tembutanud \enquote{mõõkhambulisi varblasi} meenutatakse siiani hea sõnaga.}. Seda mängiti klassis väga palju, seda enam, et 
võrguühendus alati ei töötanud, aga NetHack oli kohalik. Senikaua, 
kuni keegi Viljo Sood\index[ppl]{Soo, Viljo} otsis, et ta 
modemitele restardi teeks, mängiti NetHacki. Ma ise ei pidanud seda suureks saavutuseks, lihtsalt tegin valmis. Kahjuks läks kood koos Cure'i 
masinaga kaduma.

\question{Nutikal inimesel oli tollal tüüpiliselt kaks suunda, kuhu kiskus: 
kas akadeemilisse maailma teadust tegema või äri suunas. Kas sind ei tõmmanud 
kumbki?}

Mind äri ei tõmmanud, sest olen pärit äärmiselt vaesest perekonnast. Raha ei 
olnud midagi erilist, mul lihtsalt ei olnud seda kunagi. Kuude kaupa elasin saiast ja piimast. Ja kui raha ei ole, siis ei teki sellega ka lähedast 
suhet. Mis puutub akadeemilise maailma, siis olin sel ajal
alles esimesel kursusel.

Cure'i klassi tegemisest mõni kuu hiljem toimus üks tüvikursus. Pildile ilmus taas Anne Villems\index[ppl]{Villems, 
Anne}, kes korraldas 1994. aasta alguses Eesti esimesed \emph{webmaster}'ite kursused. 
Liivi tänaval olid kuulutused üleval.

Mina olin siis oma arust juba kõva käpp. 
Tol ajal kasutati Gopherit\index{Gopher}\sidenote{Gopher oli varajane hüpertekstiprotokoll, WWW 
protokollistiku eellane. Erinevalt suhteliselt lõdvalt struktureeritud veebist 
surus Gopher sisu küllalt rangesse hierarhiasse ja oli navigeeritav 
menüüsüsteemi abil.}, HTML 1.0 standardi eelkäijat, millest oli lihtne aru saada: oli klient, server ja \emph{markup language}, mille 
põhimõttest sai kümne minutiga aru.

Kursusel selgus, et arusaamisega läheb natuke 
rohkem kui kümme minutit. Kursusel käinud seltskond oli 
kirju, sinna sattus igasuguseid karvaseid ja sulelisi erinevatest 
teaduskondadest. Teiste hulgas oli seal näiteks Anto Veldre\index[ppl]{Veldre, Anto}, aga 
ma ei mäleta, kas õpetaja või õpilasena.

\question{Mis seal ikka nii väga vahet on.}

Tollal ei olnud jah vahet. Üks oli asja läbi lugenud ja rääkis teistele 
edasi. Aga Anne Villems\index[ppl]{Villems, Anne} oli kursuse väga hästi ette valmistanud. Päris kohe selle järel otseselt midagi suurt küll veel ei juhtunud, aga osalejate nimekiri jäi alles. 

Kui EENet\index{EENet} tegi endale veebilehte, olid nad kuulnud, et on olemas
\emph{webmaster}'id, kes oskavad veebi teha, tänu millele tekib organisatsioonis 
avaliku infohalduse funktsioon. (Nüüd oskan ma seda niimoodi nimetada, aga 
tollal panime lihtsalt asju internetti, näiteks 
võtsime Eesti kaardi ja sellele vajutades juhtus midagi.) EENet otsustas teha endale täiskohaga \emph{webmaster}'i ametikoha. Võimalik, et seni oli seda tööd teinud mõni ülimalt nutikas 
sekretär või tollal seal töötanud Marek 
Tiits\index[ppl]{Tiits, Marek}, aga 
asi lõppes sellega, et pool aastat pärast kursust kutsus EENet mind
\emph{webmaster}'iks. Palk oli kolm korda suurem kui arvutiklassis, nii et raske oli ei öelda.

Sealt edasi läks elu väga ägedaks. Saime tuttavaks Tarvi 
Martensi\index[ppl]{Martens, Tarvi} ja Toomas 
Mölderiga\index[ppl]{Mölder, Toomas}, tegime EENetile korraliku veebi ning käisime Eestit esindamas. Tollal oli veeb küll korralik, 
aga hiljem tehti veel ägedamaks, kui liitus näiteks Pille 
Pruulmann-Vengerfeldt\index[ppl]{Pruulmann-Vengerfeldt, Pille}, kes on praegu
Rootsis meediaprofessor\sidenote{Intervjuu läbi viimise ajal oli 
Pille Pruulmann-Vengerfeldt Malmö ülikoolis meedia ja kommunikatsiooni professor} ja 
ERRi\index{Eesti Rahvusringhääling} nõukogu 
liige. Seal puutusin Marek Tiitsu\index[ppl]{Tiits, Marek} kaudu esimest 
korda kokku europrojektidega. Tollal küll veel ei olnud euro, vaid eküü\sidenote{Selle valuuta tähiseks oli ECU: 
\emph{European Currency Unit}.}. 

\question{Kas Marek oli see võlur, kes valdas unikaalset teadmist, 
kuidas fondidest raha saab?}

Just. Mina tulin lagedale veidrate ja üsna ebareaalsete
ideedega ja tema kasutas väikest osa neist hullustest, millel oli mingi point, projektides ära.

\question{Tol ajal liikus EENeti ja IBSi\index{IBS|see{Institute of Baltic 
Studies}}\index{IBS} kaudu tohutult palju põnevaid ja kasulikke projekte.}

Seal jooksis igasuguseid naljakaid teenuseid, aga see polnud kõik, millega tegeleti. 
Näiteks suutis Marek hankida mulle tööarvutiks Silicon 
Graphicsi\index{Silicon Graphics}\sidenote{1990. aastal asutatud ja 
2009. aastal pankrotistunud Silicon Graphics oli peamiselt 3D-graafikale 
keskendunud riist- ja tarkvaratootja. Mitmed varased arvutiabi kasutanud 
filmid, näiteks 1993. aastal linastunud „Jurassic Park“, kasutasid just Silicon 
Graphicsi tööriistu. Ettevõttele tegi lõpu odavate laiatarbe x86-arvutite 
võimsuse kiire kasv.} masina. Silicon Graphics oli väga kõva asi, tänapäeval 
võibolla võrreldav Mac Proga. Ulmeline aparaat ja milline disain! 
Korpused olid värvilised, kõvaketas 
käis lahti kangiga. See oli nagu automaailma 
Bugatti või Porsche 911, täiesti \emph{over the 
edge}. Mõni ime, et ma selle töö hea meelega vastu võtsin, olles tulnud 
totaalselt vananenud VAXi klassi administraatori kohalt.

Kõige selle taga oli Richard Villemsi\index[ppl]{Villems, Richard} pikk, aga heledat ja positiivset 
värvi vari. Tema seda kõike püsti hoidis.

\question{Hoidis püsti, aga vist ka natuke nagu joonte sees, et inimesed päris 
hullustega ei tegeleks?}

Jaa. Richard Villems on muide Anne Villemsi\index[ppl]{Villems, Anne} abikaasa. 
Tänu Richardi suurele 
mõjule toimusid EENetis\index{EENet} väga kõvade projektide arutelud, 
mis ei käinud tegelikult üldse selle organisatsiooni põhikirjajärgse tegevuse alla. 
Sealsamas Liivi tänaval oli ka ülikool, arvutiteaduse 
instituut\index{Tartu Ülikool!Matemaatikateaduskond!Arvutiteaduse instituut}, samuti füüsikamaja. Nii et püssirohtu jagus ja 
tekkis igasuguseid initsiatiive.

\question{Kas selles maailmas BBSid ka kuidagi figureerisid?}

BBSid olid kogu aeg taustal ja mõnes olid mul isegi kasutajad, käisin
isegi seal. Aga BBSi tehnoloogiline \emph{carrier} oli modemiga üle
telefoniliini ühendumine serverisse. Mina sain varakult väga kiire 
interneti juurde, 64 kilobitti sekundis on tuntavalt kiire. 

Edasi tulid väga kiiresti IRCd\sidenote{Internet Relay Chat 
(IRC) on kliendi-serveri arhitektuuril põhinev tekstipõhise kommunikatsiooni protokoll. 
Peamiselt disainitud suhtluseks suuremates gruppides, kuid 
võimaldab ka üks ühele suhtlust.}, mis võtsid BBSide funktsiooni üle. 

EENetis\index{EENet} toimus palju lahedaid asju, mõnes mõttes oli see 
ebaproportsionaalselt nähtav organisatsioon. Näiteks aitasid
nad korraldada koolidesse internetti.

\question{Lisaks sellele, et keegi kuskil poliitilisi otsuseid teeb, peab keegi 
suutma ja viitsima neid otsuseid ka ellu viia. Sõita talveööl 
Põlva kanti kooli modemeid installeerima ei ole palga eest tehtav asi.}

Seesama \emph{case} võtab kokku kogu tolle aja mentaliteedi. 
Seda üldse ei arutatudki, kui palju koolide internetti 
ühendamine maksab, kuna paljudes kohtades raha polnudki. 
Arutati ainult ühte asja: tuleb teha ja Marek\index[ppl]{Tiits, 
Marek} otsib, kust raha saab. Marek ei kantinud raha autodeks ega suvilateks, Marek tegi europrojekti ja tuli. Põmm, kümme 
Suni. Põmm, kakskümmend Zyxelit. Paar-kolm kutti viisid asja ellu ja 
keegi ei pahandanud. Aeg-ajalt käis mõni Antsla kooli mees 
küsimas, kuidas läheb ja kas oktoobris tuleb. Ja tuligi, kuigi vahel 
novembris. Keegi ei arutanud, kas teha. Prooviti vaid vaadata, et teenuse laienedes kvaliteet ei kukuks. 

\question{Mis seda kõike edasi vedas?}

Küllap iga agraarühiskonna tung harida maad, kus midagi ei kasva. Mõnes mõttes oli see lihtne: prioriteedi määras 
see, milline kool kõige rohkem ise huvi tundis. Alguses ei olnudki huvi suur, sest ei saadud aru, millest üldse jutt käis.

\question{See on väga õige lähenemine, et kõige suuremad hädalised, kes 
kõige rohkem oskasid internetiga midagi teha, said selle ka esimesena 
kätte.}

Ma ei tea, võibolla mõnesse kohta jõudiski internet alles 2000. aastal, aga 
vahet ei olnud, sest selleks ajaks oli klõps juba 
ära käinud. Nii naljakas kui see ka pole, aga üheksakümmend protsenti tööst oli 
veel tegemata, kui kümme protsenti internetiühendusega koole oli kaalu juba 
nii alla vajutanud, et ülejäänute puhul oli üksnes aja küsimus, millal juhe nendeni 
viiakse. Aga et seda kõike oli vaja, oli Anne Villemsi\index[ppl]{Villems, 
Anne} ja tema pundi sügav veendumus. 

\question{Mida sa praegu teed?}

Püüan avalik-õiguslikus meedias saada ühele poole transformatsiooniga, mille eraõiguslik meedia on kümmekond aastat tagasi ära teinud.

\question{Kindlasti vääriline töö, kus väljakutseid jagub.}

Kui avalik-õiguslik ringhääling\index{Eesti Rahvusringhääling} ise kaua aega ei 
tunnetanud, et peaks internetikasutajakeskse hüppe tegema, 
siis oli ka teistel institutsioonidel raske seda nende eest ära 
tunnetada. Selle tagajärjel tekkisid mitmed fundamentaalsed küsimused: 
kuidas te ütlete, et teil on sellise asja jaoks raha tarvis? Aga kus te siis 
olite, kui teised organisatsioonid sellega tegelesid? Riigi jaoks on olnud keeruline 
mõista, et kui avalik-õiguslik ringhääling niisuguse hüppe ette võtab, on see
ikkagi ookeani ületamine ja parvega seda ära ei tee.

\question{Ja kui terve riik on läinud teisele poole ookeani, siis on pisut
sandisti, kui ERR teisele poole maha jääb.}

Sellal kui suurem osa audiovisuaalsest meediast toimus kinoringvaate vormis (oli 
filmilint, mis ilmutati ja mida projektori abil näidati), oli 
televisioonil kuuskümmend aastat tagasi juba \emph{live}-signaali halduse kontseptsioon, mis töötas ja oli 
piisavalt lollikindel, et sellega Eestis eetris olla. 
Televisiooni tehnoloogia on arenenud oma kinniste protokollide ja 
signaalihaldusmudelitega ning oli internetist kaua aega signaali 
loogika poolest maas. Teleasi maksab muidugi ulmeliselt palju, aga 
see on olnud kogu aeg terviklik kinnine maailm, mis arenes teist 
evolutsioonipuu haru mööda. Mõni aasta tagasi jõudsid Euroopa 
Ringhäälingute Liit ja teised üleilmsed ringhäälinguorganisatsioonid oma standarditega nii kaugele, et on olemas IP-põhine 
signaalihaldusstandard, mis ei ole veel\sidenote{Jutuajamine Jaanusega toimus novembris 2019} valmis, aga millest mõned tükid 
töötavad. Aga nad tulid sellega lagedale aastal 2017. Mõtle, kui kaua aega on olnud 
normaalselt töötav internet.

Praegu saame öelda, et tegelikult ei ole mõttekas mitte-IP-põhist 
tehnoloogiat ehitada, aga tollal läks terve tööstusharu teist rada pidi kaugele 
edasi.

\question{Seega ERRi vaatepunktist mitte ainult ei ületata parvega ookeani, vaid parvel on 
känguru ja hobune, keda üritatakse panna kuidagi järglasi saama.}

Sealjuures on veel tugevad kogemused hundiga, kes puhub puust ja õlgedest 
maja ära. Järelikult on parv tehtud igaks juhuks betoonist. 

Meil on nii äraspidised kogemused, et puust paadi kontseptsioon 
tundub algatuseks lihtsalt ohtlik. Televisioonisignaali haldusloogika seisneb selles, et ehitame 
signaali nii, et see ei saaks katkeda. Kui palju see maksab? Nii palju, kui vaja! Teeme 
nii, et ei katke! IP-põhine paketihaldusloogika ütleb, et lükkame paketid läbi ja, kui vaja, parandame. Need on 
fundamentaalselt erinevad mudelid, aga pikas plaanis on parandamine 
odavam kui kohe hästi tegemine.


\chapter{Kaspar Loit}
\index[ppl]{Loit, Kaspar}
\index[ppl]{B'Knows}
\index[ppl]{B'Knows|see{Loit, Kaspar}}

\question{Kes sa oled?}

Mina olen Kaspar. Ja kunagi, kuna see võtame selle teema, et me peame tagasi kerima mingisugune miljard aastat, siis toona see aka oli B'Knows. 

\question{Aga kust sa said sihukese aka?}

Seda ei mäleta enam keegi. Seal on nagu kaks komponenti. Üks on nagu \enquote{B} ja siis on nagu \enquote{knows}, ehk siis see B peaks  midagi nagu teadma. Pronto\index[ppl]{Pronto} alati kutsus mind Buttknows.

\question{Kuidas sina arvutite juurde said või arvutid said sinu juurde?}

Mul on selge mälupilt, et mu tädi, kes on superkuul ja minust mõnevõrra vanem, töötas Tartus
vist Bioloogia Instituudis. Ja  talle oli kuidagi jäänud mulje, et mind võivad huvitada sellised asjad. Ma arvan, et ma olin mingi, ma ei tea, kaheksa-üheksa-kümme, \emph{something like that}. Ja siis, kui ma tal ükskord Tartus külas käisin, ta viis mind instituuti. Muidugi peale tööd, kõik oli juba pime ja  seal oli mingi kabinet lahti ja seal seisis laua peal mingisugune masin, mille nimi oli Apple II Europlus\index{Arvutid!Apple II}\sidenote{Apple II Europlus oli Apple Euroopa turule kohandatud versioon. Muu hulgas erines toiteblokk aga ka video osa tuli ümber teha, sest Steve Wozniaki trikid NCTS signaali genereerimisel ei toiminud enam keerukama PAL süsteemi puhul}. See oli nagu legendaarne selline nagu fakinossem. Ja, ja, ja seal sai, ta oli seal mingi laborandi käest küsinud, et kuidas seal midagi käima, enne seda üles kirjutades sai mingi plaadiga mingisugune maine, sellel paar mängu, mis olid teksti ekraanil, jooksid ülisuper ägedad. Ja eks see oli vist mingi trein Roveri, ma mäletan, ma olen seda asja, see pilt silme ees. Ja sellest hetkest ma arvan, ma olin, möödus ka, et ma, ma ei oska nagu meenutada, kas, kas ma enne olin kokku puutunud juba seal? Tõenäoliselt mitte, see oli ikkagi liiga liiga vara ja, ja ma arvan, et selliseid, esimene. Ja, ja kuna mul on mulle tohutult meeldisid koolis Nintendo väikseid Keymengu otsmängud. Võib-olla mäletate, kuidas ei, ei mäleta. Ühesõnaga siuksed, ma ei tea, tänapäeva telefonisuurused umbes. Ja, ja see oli LCD ekraan, millesse oli nagu ette programmeeritud ette joonistatud mingisugused tegelased ja siis nuppudega näoga täna kaugele neid noh, Janno paha nii, ja, ja kogu selle aja eest. Aga Nintendo oli see mingi põhiline tootja neile, neil olid siuksed väga-väga lahedad mängud ja siis ma või isegi kunagi mõtlesin, et oh, kui lahe oleks, kui saaks ise niisuguseid teha, aga no ma sain aru, et seal taga on mingi tootmine ja see ei ole nagu reaalne on ju. Ja nüüd järsku saavad aru, et tegelikult selliseid asju on võimalik sinna noh, nagu masina sisse programmeerida, firmad seda minid elektroonikaskeemi tootma või mingisuguseid noh, muide see võib üldse mingit tehast olema, et see oli nagu see oli üks niisugune realism on ju see muidugi noh, tõenäoliselt viis mind kunagi ka sellele, et me vaikselt tegi mänge. Aga niisugune päris toimetamine hakkasin tööle. Kuskil oligi seal Jaak Laande nimi, Eesti Põlev jookseb ta kindlasti nagu ülioluline tegelane, sest et sellistel noh, siis oli selge see, et arvutile ligipääs oli see asi, mis oli nagu oluline ka valutaks. Ja, ja ma mäletan, et ma just siia sõites tuletasin meelde, et ma tegelikult olin kaardistanud omale kõik kohad, kus üldse nagu tõenäoliselt Eesti Vabariigis arvutite ligi pääses. Nõo oli liiga kaugel selgelt. Aga Tallinnas neid kohti, noh, need ikka olid. Aga Jaak Londoni nagu selles mõttes oli nagu lahe tegelane, et minu meelest ema kaudu ma esimest korda sain progeda. Kas te siis nagu andis selle võimaluse või ta õpetas ka või toika nagu õpetas loomulikult, sest et ega alguses oli lihtsalt mingi mingil põhjusel ma ei tea, kuidas ma sattusin sinna selline neljas keskkoolist või, või kolmas kurat, ja see oli see suur klass, seal olid mingisugused masinad, tõenäoliselt ka need olid emmessiksid. Ma arvan. Ja, ja see kari tegelasi paar tükki, ma tundsin nende kaudu vist kuidagi nendest jadad. Ma mäletan, et üks mu koolikaaslane nägi siukest masinat esimest korda ja meil öeldigi, et natuke midagi tegema selle teisikali esseevõistlusel lõksid ainet emmessiks selle sellepärast tuli üldse pull masinaid ta tegelikult vist mõeldi välja tagantjärgi tarkusega võin öelda selleks, et ühtlustada koduarvutite standardit ja peidikuid, milles nad jooks, mida nad jookseksid ja see oli isegi tegelikult algatatud initsiatiiv mingisuguse Jaapani Microsofti executive poolt, okei butis Peisikusse lihtsalt otsa. Saab võistlus katkestati pea niimoodi, et ma eile just ostsin välja omale ühe ühe emulaatorit tahab selle ekraani ette ja sa võid seal kirjutada kümme ja siis kirjutada reha, paks kerkivad teisegi, antud listist näitab, mis sul on, sa kirjutad uuesti kakskümmend, kirjutad selle rea üle ja, ja kirjutada on nii, et, et see tähendab, see on nagu su kohe käitma. Mõnus. Ja siis ma mäletan lihtsalt seda ka, kuidas mul võiks sõber, kes läks sinna, nägi seda asja esimest korda läinud, et nüüd siin saab midagi teada, kirjutas Liis deroomiiesse.
Ja noh, ütleme, et NLP Janar veel, nagu ta on, äkki on seal midagi tunud?
Aga ja siis seal üldse nagu tegelikult paljudes Karel Kannel oli seal kuidagi toimetas ja, ja kõik need väiksed tattnokad õitseksid Kaasani. Aga palju huvitavam oli tegelikult see, et Jaak Loonder oli ka üks masin, nimi oli Mir, kaks mingi nõukogudeaparaat, mingisugune nõukogudeaparaat, mis oli noh, siukse põhimõtteliselt oli ikka pikem kui viis meetrit oligi pikap poisikesest kõrgem Marje ja Jaagu leida võib-olla ninani.
Ja Tein meeletute häält, sest et oli nagu noh, põhiline osa ilmselt oli jahutuseni. Just seda tegigi villa kraažinis masinaid ilge müra ja siis aeg-ajalt, kui inimestel viskas nagu Kopli ette, siis nad lülitasid selle välja selle jahutuse ja siis see oli, tšekk tuli muidugi tohutult rääkis sest, et see kokku ja ilmselt on. Ja, ja see oli, selles mõttes oli ka nagu geniaalne vaim, kus ta seal üldse kätte saanud, see oli nagu nad lahendust, lahendust, masin, tal oli ikkagi võimalik klaviatuurist sisestada käske kus klaviatuur oli elektronkirjutusmasin, mis need põhimõtteliselt oli nagu klaveriprinter ühekorraga. Nii et nagu kaaned kapitaliga sellest samast masinast ja siis tal oli
Mustvalge telekas aga tal ei ole, kas valguspliiats ehk siis sa said nagu ekraanil tabada mingisuguseid punkte ja see masin tundis selle ära, ilmselt ta luges seda kuradi kineskoopkiirt ja selle järgi pani selle kokku. Ja, ja ta suutis ka mingisugust Rudimentaarselt mingit graafikat, Kuvalehti tal ei olnud nagu ainult teksti ekraan, vaid ta suutis ekraanile kuvada mingit punkti, see oli selle arvuti, eks niisugune tohutu ülesanne see punkti püsinud hästi paigaldas, ikkagi õrnalt ujus asjadega hakkama. Ja siis mu esimene programm, ma mäletan, oli, see oli muidugi see programmeerimiskeel oli vene keeles vene tähe, et kõik olid mingid lühendid super lõkseni ja, ja siis ma progesin mingisuguse graafilise neli tipulised tähele, sest ma arvan, et ta koosnes, ütleme siis Megist umbes kuueteistkümnest punktist tulla ja see ikka tõmbas selle arvutiga täiesti koomas ja kõik see ekraan ujus selle, aga väga uhked. Aga meil on veel lisaks veel, meile õpetas, oli näiteks polindi lugemist nad seda masimiseks õi kahte moodi meediat, üks oli paber, perfolint, siuke õhukene, mis lasti läbiteooriasse džäki, kotid, teravamad vennad suutsid torkida nõelaga augurauaga oleks võinud tegelikult seda perfolindi peale seda proge kirjutada, ma arvan.
Mul on selline tunne, et äkki oskasin.
Aga aga siis olid seal veel mingisugused vahvad asjad, mingisugused magnetkaardid, mis olid umbes sellise magnetkaart, jah, ta nagu perfokaarte ma küll tean, aga ma usun, et magnetkaart oli selline jälle niisugune umbes tänapäeva või tänavale telefonist suurem, noh niisugune mingisugune. Ma arvan, et mingi kaheksa senti korda mingi viisteist senti sihukene troon latakas, mis on põhjust meenutada oma materjalid, seda, mis floppy diski sees on tegelikult. Ja siis on põhjust, panid selle mingist latist sisse, tõmbas seda solisti läbi ta luges sealt midagi mingisuguses koguses hästi, läks ja luges keelde. Ja see oli nagu noh, siuke sel juhul müstika seda enam lugeda ei saadud, see. Aga see niisugune noor nagamannid, see peab olema ikkagi päriselt märksa tahtmine, et sellest ju ekraani peale tähejoonistamine, huvid oleks või see nagu mästidesta ja sellest tõesti sa, sa nagu kirjutasid midagi ja see pilt tekkis seda sinna ekraanile. Siis see huvitav oli just see mina andsin käsu, mis ta tegi midagi või? Noh, see olnud oleks olnud noh, täpselt, et kui see oleks nii lihtne, et plii, Stroomi eeskõneleja, siis see ilmselt oleks kaotanud huvi, aga see oli ikkagi kombigeiti värke seri. Selles oli mingi algkeemiline element, selline ülikõva ja seal oli nagu niisugune noh, mis maistele poistele meeldivad igasugused Salagi keeled ja igasugune koodid ja ma ei tea lipukirja Eestis asjad, et noh, selline nagu see oli kõike seda. Ja veel, mida see oli ikkagi super hästi kukuvad ja, ja, ja sisena jällegi, et sealt tekkis, aga noh, see noh, selge oli see, et see oli nagu meeletult piiratud on, et kaua sa seal ikka seda seal ja, ja lisaks sellele, et seda merre, et üks noh, õnneks emmessiks, las see suurem, aga seal oli vist jälle midagi mingisugused piirangud, kuna see oli keska olla onju ja, ja ilmselt sellepärast säpizzewiski. Ma arvan, et see oli kuidagi ega vist säkiga seotud semis roopa tänavale selle võttega, mida siin ka teistest lugudest läbi jooksnud minema. Ja seal oli siis ka terve klass.
Kus siis oli üks niisugune nagu juhtarvuti, millel oli mingisugune draiv? Ma eeldan, et see oli mingi flopidraiv. Ja, ja siis oli terve klassides arvutid, mis said sellest nagu peaarvutist omale alla laadida asju, isa füüsika, eks ole seal lihtsalt kirjutada oma neid programme, aga kuna kuna Trai oli ainult üks, siis kui sa tahtsid salvestada või midagi, et siis sa viid selle sinna, saab vaid siis tavaliselt see asi. Tegevus oli üsna nagu lihtne, et.
Kõik tšekid ja aeglasel väga nagu hängida on ja siis oli seal kogu aeg oli mingi poiste karjed olid talle lisaks bar nutikamalt vend olid siis nad pannud seda vedama. Üks legendaarne tegelane Emmucats ehk linnade edesiis tänasest kuskil Soomes toimetamine. Aga noh, tema oli nagu selgelt minu esimene nagu niisugune guru, keda ma nägin, et ta oli, ta oli, ma arvan, oli umbes minuvanune, aga ta oli omandanud juba täiesti kõik need peenemad alged, ehk siis põhiline, mida ta oskas, oli see, et oskas kahest programmijupist panna kui ühe terviku ja selle nagu paketeerideni eestlase laadideks. Point oli selles, et väga suur osa seda softi levis tavalist magnetofoni kassettidele. Ja vist oli see kuidagi, ma ei tea, kes oli võrguprotokollist kinni või sellest kas endist kinni või enam-vähem niisugune šehhivad mängud olid kolmkümmend kaks, kilobaiti umbes pikalt, noh see oli ka mässimine. Ja, aga selleks, et nad sinna ka sätivad ära mahuks, nad olid tehtud pooleks kuus kas kuusteistpidi vahepeal nagu keerama ja ühesõnaga see kogu see kasseti majandusele keeruline. Aga, aga kui seal oli juba nii kõva asi nagu flopidraiv, et siis sa said sellega seti pealt lugeda selle kuusteist kilo sisse siis tõsta kuskil mälus mujale ja siis lugeda teise kuusteist kilo, saanud kokku panna asju, tekkis see tervik, mida sai nagu Rootsi või kuidagimoodi, ühesõnaga see oli kõik täielik, seal juba seal juba supermaagia siis oli seal tegelikult tekkis sul siis niisugune teadmise põhinenud nagu eeskuju, see oli keegi inimene, kes, kes vaatasid alt ülesse sellepärast et ta teadis rohkem kui sina, oi, ealiseks tegelasi veel, seal oli mingisugused, ometi võiks ainult käivaid Kont, toimetuste eesnimi oli ühtlasi ka natukene tüsedam lendude, tal oli väikene kohvrikene metallkohver mille sees oli kogu emmessikse manual, see oli fotokopeeritud käes täkazzbeeber, siis ta käis sellega renni väga uhked saajat, et see lahti siis seda midagi selle alusel kirjutas see nagu jälle superlux. Eks tegelasi oli veel seal. Jaa. Ja ega mul seal ka nagu selles mõttes ongi, et kui on, sa istusid seal seal nagu otsest läks, sassi sellele draivi ei olnud, võib-olla seal seal mängisid, mingeid tüütas ära, on ju sisse kirjutasite sama teisikut. Ja sellele tuli mingid mingid piirid ette. Loomulikult miljonites graafika pool. Ma üritasin sinna midagi mingisuguseid pilte manada ja kuna emmessiksil oli tegelikult tal ei olnud graafika ekraani ja seda aimuleeriti tekstiekraaniga, ehk siis põhimõtteliselt iga pilt, iga mäng, mis emmessiksin toona jooksis, oli tegelikult otse mälus tahe generaatori ümber programmeerida. Juku ja juku tehti sama lugu, et seal said laadida mingisuguse oma nii-öelda fondi kuskile mällu. Ja see oli ikkagi tähed, olid mingid, panin siin teha asemele pit, näppisin ennast või mida iganes kokku võtta, et põhimõtteliselt sama sama laks, et seal oli noh, ekraan põhimõtteliselt emmessiksil oli otse aadresseeritav, et kui sa teadsid, et nad selle režiimi ega algas aadressil, seal mingisugune eksas mingi jõuate värvata veel on ja siis sa lähed sinna järjest panna iga, iga bait oli üks rida on ja tal oli, kas tal oli läbipaistev taustavärv või esivesivärve, siis veel mingites režiimides said need iga rida-realt neid värve vahetada, selline Välitel väest. Ehk siis põhimõtteliselt niisugune mõiste nagu ma ei tea, kas sa ilmselt oled kokku puutunud, et mängudes on siuksed tegelikult nagu Spraynyden ja need on need, mis liiguvad, eks ole, tausta ees. Et MSI CD-lt spaitega emuleeri sellesama tähe genega ehk siis kohalik tähe Gene Programmeeriti jooksvalt ringi, eks ekraan kirjutati täis ABCD, ehk ei ole ma nädal aega kõik märgid, mis talle pähe tulid, mis olid, on ja, ja siis need kogu aeg adresseeritud ja kirjutajate ring ja selle jälle iga ekraanist oli alati üks mingis kolmeks osaks. Igas osas sa said eri eri nagu tähestiku väänata ja, ja see oligi nagu see minu jaoks oli võlu, kui ma sain selgeks, et on olemas Assembler see Marissa, plaanida ühele mälu aadressile üks Payton ja ja siis ma mõttelist veetsin suure osa oma ärkveloleku ajast millimeetri paberile, joonistades mingisuguseid tegelasi neid tõlkides paineriks või eksakse siis laanide kuskile mällu. Aga kui ma nüüd refereerin tagasi, siis see ega, ega tänapäeval arvutigraafika ei ole ka lihtne. Aga see keerukus tundub olevat nagu teises kohas, et tänapäeval sa pead aru saama, sest kolm t/a, kolm tee geomeetrilist ja sa pead nendest spetsiifilistest kepihoopidest teadma ja nii edasi, et noh, seal on ikkagi leierzazdafan seal, vahel ka sellele, kas sa põhimõtteliselt võiksid ja panin nagu toorendiselt toppisid otse ekraanile selline, aga kui see sinna toored toppimine, et ega see ei olnud nii et sa ütlesid talle nüüd joonista sellesse kohta mingisugune asi, vaid seal pidin ikkagi tegema allikast algoritmilise tööd. Et noh, mis, kuidas ma seda värski maja ees ja nõia Graszek nagu hea, et ta kähku majakaks ja nii edasi. Noh, ütleme, et seal oli igasugused trikid, et, et ta nagu töötaks, aga, aga kuna see, see keelest ja kõik, see asi oli nii lihtne, siis oligi nagu super elegantne, super nagu lihtne ja ma arvan, et ega minu progemis aeg jäigi sinna kaheksaga.
Keskel pärast ma olen võib-olla natuke mingist mingisugust HTML ja võib-olla see SS-i nokkinud, aga tegelikult ega ma sellesse Võru oleks nagu üle kohe, kui tal läks nagu, nagu läksid nagu keerulisemaks. Aga õnneks tuli tasemele igasugused graafikapaketid ja, ja muudes, aga millal see oli juba keskkooli ajal? No seal jah, sealt ütleme olles enda jaoks kaardistanud ära kõik kohad, kus sai midagi arvutitega näppida, jõudsin ma läbi. Väga tähtis on kusjuures see, et tippis oli ka üks klass, kus teeme siis. Aga seal olid juba igal masinal oli Drive tava. See oli ka juba super, et vahest ja, ja seal oli ka muidugi seal guru Guru staatus oli nagu juba järgmisele levelile, seal olid mingid laborandid.
Ühe nimi neli koma ekselgiga. Ja ei tule enam meelde. Aga aare, tali tuleb kuidagi ette ja ma ei ole kindel, kas see on õige nimi õige näo ees. Aga igal juhul olid seal mingisugust juba üliõpilased on juba nagu kõvemad vennad või isegi oma mingisugune maine postkäädavat või mis iganes ja, ja, ja loomulikult see nabade ja jada seal ukse taga neid selgelt tüütas ja siis nad seal tegid omaette mingisuguseid reegleid, olid suured jumalused ja näiteks mingi hetk oli, kui neil juba olid täiesti noh, infoga leviks, eks ole, et järjest rohkem Kunder äkki see on ju sinna värava taha ja ja siis, kui oli vaja reglementeerida, et keskmise, seda saab ligi, siis nad võtsid ühe kõige popima mängu, mis seal parasjagu oli. Kinks, väli, trükise välja kogu selle soos kooli. See oli nagu täkkov seda perforeeritud paberit ilusti nadid. Ja siis nad otseselt meie naha tavaliselt punase pastakaga ringe ja progesid selle mänguringis selguvad täiest seal juba minu jaoks oli see juba nagu Jumal, see tase. Ja, ja nad progress niimoodi ringi, et nad said vaikselt iseehitatud Sostikudega juhtida neid selle mängukolle. Ja, ja siis põhimõtteliselt oli umbes. Reegel oli see, et kui sa said nende jumalate vastu ühe leveli läbi siis sa said selle ühe päeva käia. Ja teha seda jah, jällegi see iseenesest mängufaktor ja see kõik oli põnev, oli see, et nad tõesti nagu nad võtsid selle mängu, mis minu jaoks tundus nagu superkeeruline nagu ja, ja lihtsalt nad kirjutasid, tuleb aina ringiga, et kirjutasin, mitte lihtsalt ei, ei teinud seal mingile tegelasele mütsi pähe, on ju vaid nad lihtsalt nagu tegidki, kõik ringid, käitumine muutus ja tänapäeval muidugi tagasi vaadates tundub, et see tegelikult
Väga lihtne.
Aga noh, see on seesama see kelleltki, kes ja mis praegu nagu üles tõuseb ja mille peale kõik asjad ehitatakse ja ja pluss see töötas veel ka ju sellel tasemel. Et see on just nimelt tõust testis näed veel nagu nende petnokkadel rivisi, mis veel nagu kõrgemale apteegi veel ihaldusväärsem sinna sissesaamiseks absoluutselt. Aga igal juhul lõppkokkuvõttes lõpetasin kuskil Kullos, kus oli ka üks klass, kus olid vist juba natukene kõvemad emmessisid. Seal oli juba nagu mingi graafika reziim ja, ja igasugused muud asjad, ehkki, ehkki ütleme, et kui seal midagi kahtlast, mis kiiresti liigutaks või riike kasutama seda teistega. Ja, ja seal oli selline legend nagu ränimeister, kes seda seal nagu majandas kes on üks tore toona, ta oli selgelt sihuke tore punkar, kes oli tulnud kuskilt Volgaga gaasianalüsaatorite tehasest temast ja seal vits poistega jahmerdada, aga ta vaikselt seal hakkas tegelema Komoderamiigadega mis oli juba sihukene, Superad väest. Raud. Ja, ja, ja kuidagi tali seda kõike ei seksivideo tootmise ja, ja sellise asja kuidagi tähe all. Ja tänu sellele oli ka loomulikult siis on välja arvestanud, et kus, kus niisugune asi veel toimub, on ju, ja Eesti Televisioon oli selgelt üks ja siis oli mingisugust vene metalliärikat. Kuidagi niisugune turundusharu oli tekkinud Eestisse ilmselt keegi vend oli piisavalt palju lobi teinud ja tal ei ole muud teha. Siis ta oli kuskil Kristiines kuskil keldris oli püsti pannud väikse mingisuguse nii-öelda reklaamistuudio kus ta siis tootis märki ja tal oli seal ka, eks saviga. See pidi olema siis üheksakümnendate algus, eks ole, jah, kuskil sealkandis. Aga ühesõnaga kogu sa selle Kullos nikerdamine sama jaoks või ka mingeid mänge tegema ju võimalus seksida ja Marcus klass Mandel toimetas ja, ja, ja. Raul Keller, kelle alkoholi killer? Ja, ja seal me midagi seal noppisime, isegi ta üritas seal mingisuguseid emmessiksi mängimist nagu pubitsee midagi, aga, aga see tundus kuidagi ikka väga niisugune kahtlane ja, ja naiivne tegevus mulle vähemalt toona, aga siis juba räni kuidagi nähes minus potentsiaali meelitas mu Eesti Televisiooni ja siis olin põhimõtteliselt ma olen isegi veel kestva viimases klassis, ma töötasin juba Aktuaalses Kaameras ja uudistetoimetuse kõrval oli siuke väike Kubrick. Kus me siis tegime Aktuaalse Kaamera infonurki, mis selle diktori taga oli nagu seina peal ja kuna abiga oli selline tore masinat sinna see lasta videosignaali sisse, sealt ise sinna välja, et sa said ega teha nagu tiimixi vist juba toona, eksole, kus on see ju, see oli ju jõhkralt kallis riistvaraga piisiide peal, et enam Keilasse see oligi, et Abigail oli, miks, miks need Amingat siinkandis selles vallas levisid, oli just see, et noh, PC jaoks mingisuguse tasemega videokaart seal ikka mingi, kas see oli mingi hollivuudi või, või sellise tasemega asjani ja, ja need masinad olid ka neil oli, eks ole, mingisugune Seega ja, ja neli värvi on ja samas kui amiga oli nagu Full videos on, põhimõtteliselt võisid talle panna selle võrra isegi arvuti monitori see võistkonna teleka taha ja see toimis ja see oligi ilmselt see point, miks tal oli videosignaal, et oli nagu seks, sest nagu kodukodutarbimisest arenenud selliseks nagu ja, ja seal tegime oma ilmakaarte ja panime need videopilte sinna ja põhilise osa ajast muidugi mängisime arvutimänge, sest jälle Amigo super šefimängudeni. Millega te tegite selleks teiega teie nullist ja kirjutanud kogu seda ka. Öelge, kas seal oli olemas täitsa viisakad graafikapaketid, De Lux Pent on nagu šefimaid graafikas ohte, mis oli nagu igasugustest asjadest, igasugustest Photo soppidest ja kurat teab millest ikka. Kümme aastat enne tuli. Ja, ja me siis ainult noh, me olime nagu selliselt nagu India nõu abiga vennad ja me vaatasime ikka kõikide pisi muude mängude peale ikka ülevalt alla, sest et nad ikka ei teadnud, milles nad seal Sarkisideni paraku lihtsalt saviga pisesena oli, oli kehva ja ta lõpuks jooksis, läks nurja ja aga iseenesest see tehnika oli nagu superäge. Ja seal meil tekkis mingi väikese kohaga selline punt tegelastest, kellel oli, kas oli kodusaami ja/või kuidagi tegelesid siis noh, Martin Rinne, kes täna teeb, direktor on ju tema juba tulid tekkis sealt kuidagi sinna telesse ja siis Margus kliimas Marx samamoodi oli seal, tema tegeles Eesti videos Siilatsi kuidagi sellele, noh, kõigil oli nagu mingisugune Äkses ja, ja siis jällegi loomulikult seal ka võlus mind pigem see, et et sa midagi seal nikerdasid ja sa tekitasid mingisuguse elava pildi sees selle pidanud olema minikaamera ja näitlejad ja mingi asi, sa võisid teha sõna, mingeid väikseid, mingid animatsioone või animatsioonide vedas välja. Noh, selles mõttes, et sa said seda teha põhimõtteliselt stop mausseniga. Noh, ütleme, et seda niisugust animatsiooni reaalajas
Bussiga Nõgisema juba ikkagi reaalajas täie, mis aga ei, me tegelikult ikka tegime tele, tegime reaalajas ka, sest keegi viitsinud stock mossega vastane, aga põhimõtteliselt sai seda teha ka Stognoosiniga, aga minu meelest isegi enamus sellised asjad käisid ikkagi reaalajas, et tal olid juba sellest eluks peenraid sisse ehitatud igasugused nutikad asjad, näiteks nagu liikumise aeglustamine või kiirendamine andsid talle põhjust ette, et siin on sul mingisugune see kast see kastab liikuma mingisuguse viiekümne kaadriga siia, siis ta automaatselt täitis need viiskümmend kaadrit ära. Ja vajadusel, kui sa ütlesid, et siin, siis ta tõmbas lõpus hoo maha ja kõik oli väga-väga fain näiteks. Mativeermet sellist naist, kes, kes pärast sellist naist Tallinna linna mingisugune disainer või, ja, ja ma mäletan, kui ma olles läksin sinna teles ja selle järgi võib muidugi mingi aasta paika panna ja tegime öölaulupeole, tegime mingisuguseid valgus, Kippe, et, et see oli juba ikkagi jällegi superväest, mäletan, et mingisugune nagu asi oli nagu televisioon, Jajah, seal seal olid väga ägedad ja, ja see oli ka nagu kogu teletegelikult siukseid asju tegime, sest et alternatiiv oli tiitri masin, eks ole, mis oli mingi räme puit ja mis oli nagu ikka eriti.
Oleme jälginud.
Sihukene pool analoog ja siis meil oli niisugune super, et väest animatsiooni ja värviline ja seal sain teha mida iganes ja siis me tegime seal vaid vaest Tegime mingit haltuurat mingite reklaamide jaoks ja igast lollusi rist puhas ning iseõppimise värk või kuskilt hakkas tulema, mingit informatsiooni ka juba ei, see oli ikka puhas iseõppimise teema, et selles mõttes, et oli noh need, need vahendid olid suhteliselt piiratud ja ega seal midagi nagu väga keerulistel kunagi mingi hiljem tulid ka, mida sellise koldepaketid sellega sai seal, kus sa rõhutad, meil olid noh, jällegi see tasemete vahe, et on ikka hoopis teine, et seal sa pidid ikkagi mingi punkt ajal konstrueerima mingisuguseid kolm teeb hindu ja siis neid seal kuidagi opereerima, et et tänapäeval vaatad, kuidas väänatakse mingit pump, Mäppinguid ja mingisuguseid asju pleierite kaupa ja siis see kõik kuidagi tuleb välja, et see on nagu täiesti müstika. Mis, mis sa siis tegid, kui sa Eesti Televisioonis enam ei olnud, sest ühel hetkel sa enam jootmist? Jah, seal oli kuidagi tundus, et see videograafika oli nagu väga põnev, aga tundus, et kuna siis oli, hakkas tekkima niisugune nagu Business. Et siis niisugused sõbrad, kes kuidagi olin rohkem sattunud trükigraafika peale, kes seal kujundas Eesti Ekspressi ja kes seal tegi midagi, et see tundus nagu kuidagi nagu rohkem piinas. Ja siis ma kuidagi sattusin, sain aru, et ahah, et videote abi, aga, aga selle sellele Business juurde peaks nagu pisside peale ennast kuidagi sebima ja siis sealsamas telemajas kuidagi tekkisid mingid potensiaalsed kliendid ja, ja ma pidin hakkama tootma mingisugust kujundust, mis on nagu trükikõlbulik mujalt kunagi näinud sellist programmi nagu korraldro on, mul oli see vaja nagu ära teha, siis ma istusin mingi öö läbi, tegin endale selgeks sõiduautot, frustreeriv, sest et ta oli täiesti teine maailm. Ja, ja tänapäeval on ikka see, et saab joonistada, siis see pilt on nagu ekraanil, mida see joonistada, aga siis oli niimoodi, et seal midagi konstrueerisid mustvalgelt mingisuguse vektormessi. Sa panid sinna mingid värvid peale ja siis vaatasin Bregioodist teisele joonistasin aeglasem sa oled. Siis sa läksid nagu uuesti selle kallal, kuidas me selle kohe gruppi, mis oli siis, kui mina üheksakümnendate keskpaigast mäletan korral troon, siis see, see rajakas kipun töötama. Ta tegi mingisuguseid asju nagu tennispalli katki ja lihtsalt, kui Sa salvestad selle valesti ja selles mõttes sa arvestasid ju noh, olles kasvanud, arvestada joast, harrastasid nad aeg-ajalt jooksid kokku ja aeg-ajalt nad aga võib-olla jälle seal mängis ka natuke see, et selles poisikesepõlves selles mõttekas õpitud arvutist ikkagi üleolek läbi ühe lihtsa fakti, et emmessiksil oli paremas nurgas oli port, mille sisse käis kas siis kettaseade või mingi mälukaart vits, pisikene, pisikene, üpriski suur sahtel. Ja nüüd selleks, et mitte seal midagi asja tuksi keerata, siis selle kvartalisse sisselükkamise hetkel seal sees oleks väike lüliti, mis tegi masinaga sätti. Ja loomulikult selle lapiti kiirelt ära, et selle asemel, et Poola väljas selleks, et teha mingi kord, kui sa oled midagi tuksi keerab, näiteks olid kirjutanud programmi, mis loopima ainesse, panid kohe nagu näpud sinna auku oli masinalis surnud, onju ehk siis see kontrollmasinale oli sellest, et selle ühesõnaga, et vaatasin, teadsid, et mingi valemiga saad jao palju tegijaid ülemasinast jah, et see, see teadmine on olnud ma ka siiani, et ma alati tean, et kui ma kuskilt heinast ikka lõpuks juhtme kätte saame nüüd siis on ta surnud on ju, võib ühendada, pelgab.
See on hea teadmine. Selle koha peal ma nüüd pean andma järgi kihule ja ja lõpuks küsin, ma arvan, need küsimused, mida ma väga tahan küsida, me jõuame Maiko ringi ja punkte, eks see, kuidas sa sinna jõudsid. See oligi selles mõttes, et kui ma olin juba selle prindiga alustanud, on ja ja, ja siis ma vahepeal kuidagi sattusin mingisse
Niisugusesse maailma, kus, kus nagu oligi nagu print, oli niisugune asi, millega ma tegelesin ja, ja kuna ma olin vargusega varem suhelnud seal televisioonis ja tema omakorda suhtles selliste tegelastega nagu lõvi, kes on muidugi kõige olulisem tegelane üldse, kes, kelle juurest ilmselt algab kogu Eesti arvuti pisest, kui Jaak Loondest algab kogu Eesti arvutiteadust, siis ma arvan, et lõvist algab kogu arvutipises, Alugete ise poleks pisest kunagi käinud. Aga seal Rainer Nõlvak ja kõik, kõik see nagu plekist kokku ja siis ma saan aru, et Rainer oli Margusele teinud ettepanek toimetada siis mingisugust ajakirja, mis siis nagu alguses toimetaja seal ja ta oli, ta oli noh, nagunii-öelda asutaja, toimetaja mis iganes alguses ja tema siis vits mul varrukast kinni ütles, et davai, et need on vaja teha seda ajakirja, onju mina muidugi, pigem oleks mänginud arvutimänge nagu ma olen harjunud teleselline ikkagi üheksakümmend protsenti meie tegevusest oli arvutimängude mängimine. Aga noh, seal oli ka täiesti. Ma sain omale väga korrektse neli, kaheksa, kuue, ma arvan, seal ja seal jooksis Ultima andev roll ja just asjad, et see oli täitsa tore. Ja siis.
Ma muidugi tegelesin sellise klassikalise nagu noh, toimetus tegevusega, mitte tuttava inimesena mõtlesin, et millest peaks alustama, peaks alustama ikkagi ajakirja esikaanest välja. Ja siis ma sellest korralisse, seda esikaant, sellel hiirega joonist sinu ma joonistasin minu arust praegu tagantjärgi mõeldes muidugi tuleb mõista seda, kui aga noh
Siis tõenäoliselt see nii ei olnud. Aga.
Aga siin oleks jah mingi tohutu aur, et õnneks järgmiste numbritega, kuna siin Brontoloogas ka kokku, et neid nii väga palju ei olnud ja nendega läks palju aega. Ja kuna see on nagu otseselt ka nagunii, kui äriline ettevõtmine vaid oligi sihuke nagu promo, siis keegi nagu väga ei survestanud seda ajaaja poolt ka nii et meil ei olnud nagu kohustust, mille tellija heidetele meil iga kuu ilmuma või vähemalt alguses algusest Olysikenena kuskil Võrus istus üks kuradi nohik siis miks ei ole tulnud, eks, et miks ei ole loobunud näeme, ei adunud, et.
Meil on selline efekt.
MP imper kindlasti oli individuaalne oma näitel võin kindlasti öelda see, et see, see punkt exe praegu foto tehtud niimoodi internetis on, et see on ka kindlasti nagu märk sellest, et ta on ikka päris oluline asi, millest ma ka küsib teise selles mõttes, et oli oluline igas plaanis, sest et tegelikult ta tõesti noh, jällegi, et olles selles asjas sees, siis noh, minu jaoks ei olnud nagu küsimus, et kas arvutid tulevad, muud maailma me siin ei mõelnud sellele, et nendega oli lihtsalt hea asja teha ja nad tõenäoliselt olid inimesed ikka täiesti rumalad, kes seda ei teinud, on ju midagi. Ja noh, kõrvalt vaadates ma isegi ei saanud aru, kuivõrd vähe tegelikult arvuteid kasutati toona, sest me istusime, eks ole, MicroLinki peakontoris seal, kes kauem mingisugune, see vilunge enne ju seal ka noh, telemajas igal pool, noh, mul oli Äkses arvutite oli päris hea. Aga ma mäletan, kas see, eks see esimeses numbris või lihtsalt ei marsi. Esimeses oli arhitekt Kalle Rõõmuse büroo.
Niisugune väikene tutvustus läbi selle, et nad hakkasid kasutama arvuteid projekteerimisel ja see oli see midagi täiesti siukest epohhiloovat ja, ja ma isegi toona ei saanud sellest aru, et kuivõrd imelik see on üldse, et keegi teen nagu paberil midagi ja seepärast tundus kuidagi ära noh, naisena väed hakkavad aga pärast noh, jällegi selle sellega on võib-olla nagu isegi tagantjärgi seda artiklit nagu lugedes üks kord, sest malakas siis sinnapaika ja panin pildid külge ja, ja noh, mind see võib olla vägagi huvitav, mis on kirjutanud, veeretanud näiteks ma ise kirjutasin. Aga, aga tegelikult tõesti, et et kui üks arhitekt käis, eks ole, Staseerimas kuskil Kanadas ja seal tegeleti just sellega, et osteti personaalarvutid ja see nagu jällegi muutis selle töö efektiivsust sellest, kui mingisugused arhitektid, konstruktorid päevad läbi joonistasid kaika peale midagi ja siis järsku maid Bach valitsusele arvutisse ja, ja kõik on nagu hästi, onju. See oli ka nagu väga-väga põnev mõte ja, ja tegelikult on huvitav vaadata seda teed, mis täna toimub, on see, et meil on see nii-öelda see pimm modelleerimine ja, ja, ja, ja, ja siis sa kuuled, mis on need nagu väljakutsed. Et tõesti, et, et et mu üks sõber tõotab start, tapmiseks, tegeleb pimm-mudelite konfliktide analüüsi, et, et kuidagi üritada aru saada, et näiteks ventilatsioonitoru ei tohi läbi akna minna näiteks siis ma vaatan, siis mõtlen, et issand jumal, millega need inimesed on tegelenud, et see noh, miks nad seda arvutit pole varem kasutusele võetud.
Kui raisatud aega, onju?
Seda saab lihtsasti teha programmiga jah, olen teinud jah, aga just see, et, et see, eks see, eks see tõesti ta nagu ta tõesti üritas tuua olu jutust jäi mulje, et see on ikka super, on väheste mingi häkkerite Räkani tegelikult ma arvan, et ikkagi inimestele andis pildi, et mis, mis tegelikult toimub. Noh, nagu üldse, et see arvuti seal nurgas ei ole nagu raamatupidaja kalkulaator ainult, või noh, mingeid muid asju ka teha. Aga kuidas sa selle kirjutamiseni jõudsid joonistamise juures? No ma ütlen, et seal oli vaja ju kontenti toota ja ega keegi toona ei olnud arvutiajakirjanik Ain ja, ja kuna mulle meeldis arvuteid, arvutimänge mängida ja ja ma arvan, et noh, kirjutamine on iseenesest tore tegevus. Siis kuidagi nagu kas sul kooli ajal juba oli see nagu kirjutab ise kirjandisoon oli kuidagi olema? Ei, ma olen võimeline kirjutama okeilt ja iganenud jah, mulle joonistada meeldib võib-olla rohkem, sest kirjutamine on sellin, raske asi, et sa pead nagu lause peaks läbi mõtlema ja siis sulle tundub, et nendele headele. Et, et on nagu liiga.
Kreetne.
Ja siis selleks, et teema jätkuks, mul ikkagi veel üks oluline küsimus.
Nüüd mõni aasta tagasi.
Tõnis Kahu seletas mulle pikalt, kui nad, minu arusaam sellest, mis asi on küberpunk on täitsa vale. Nii pikalt ja põhjalikult, ilmselt tuleb ära, härra Kahu selle koha pealt just uskuda, mis ta teab, millest ta räägib. Nüüd aga minu arusaam sellest, mis asi on küberpunk ühest väga konkreetsest exe artiklist, mille kirjutasid sina ja proto. Ma olen laid, oli seal ka kindlasti öelda, minu meelest olid DVD ka nimed olid seal all. Aga see jutt sellest, mis asi on küberpunk ja kuidas ta teab, mis asi on meie kaheksa. Ja tal on V8 mootoriga auto näiteks muuhulgas Eesti pik nimega. Neid asju, mida kübervähk tegid, räägib ka, kuidas see oli, kuidas te niisuguse sisu sisu suutsite provotseerida.
Raske.
Ta ainult äri, aga eks meil oli, eks meil oli mingi ettekujutus sellest, see on ju jälle, ega, ega küberpunk ei ole mingisugune geneetiline, mingisugune organism, mis on välja arenenud ja siis on, pärast on hea klassifitseerida, et vot see on pool on hüljes ja pool on mingisugune või veel paari ähvardamine või mis iganes. Ei noh, huvitav jah, et meeleolu selgelt olime. Jällegi ma eeldan, et me toona juba teadsime, et mis on, ma eeldan, et oli olemas juba Gibsoni nekromancer, onju, ja see oli kaheksakümnendate keskpõhjust ja see oli kui kõik nagu räägivad sellest Itšalker Kailist, mis oli muidugi ja see oli väga oluline teosena. Aga noh, minu jaoks Gibsoni pööning, kroom ja Nekromant selline ligi lasi ka ju täiesti välja, et see oli noh see oli aru saanud, tõsi, ma siiamaani loen seda regulaarselt üle. Ja härra Gibson kirjutas need raamatud kaheksakümnendate keskpaigast trükimasinasse paberi peale ja ja täpselt nii ongi aasta kaks tuhat üheksateist oled sa näinud uuemaid raamatuid ka lugenud? Mõnda need, need lähevad veel hirmuäratavalt nagu tõepäraseks. Ja aja ajahorisont tuleb lähemale, siis ma ei oska nüüd järgmine küsimus, tee pidi olema mingisugune nagu väikesed, ei mõtelnud selle kedagi nagu mõistet nagu välja kahekesi ei mõista seda saiti, see oligi nagu Gibsoni, kui öeldi, et kui tol hetkel juba mingisugused rahvusvahelised mingid Peebeeessid Internetis, kus te väikesed nagu ma arvan, et see oli kõik kuidagi klikkis tõenäoliselt kokku, et tegelikult noh, jällegi, et ma arvan, et ma ei oska prantslasest rääkida, aga noh, pleier on eraon, eks ole, eepiline nagu nurgakivi on ja siit Meier Olin ja see fotoroloog, kes joonistas ilusaid pilte, tsiklil Distopiline noh see kõik kujundas meil välja mingisugused Pästopilisem pildi tehnilisest maailmast, mis kõik on nagu külge ühendatav. Ja, ja, ja ma arvan, et see pättele ei saa liita, mis täitsa juhuslikult praegu hinna jõudis, onju. Et ma arvan, et vähesed inimesed Eestis teavad seda originaallugu. Ja mina olin selle toonane, mainin nagu totaalfänn. Ma käisin aeg-ajalt Helsingis akadeemilises kirja kaupas, siis sellega ma vaatasin, et kas uus osa on tulnud üheksa raamatut mul on kõik olemas ja see on kõik, on kuidas mingid metalltorude või jäävad su silmamuna sisse ja ajust on selle järgi ainult mingisugused džiibid ja natukene pudru, onju, ja noh, selles mõttes, et see, see, see kuidagi koerad, me elasime selle asja sees, ma arvan ja ja ka mingisugune jälle mingis mängumaailmas tõenäoliselt olid paar mängu jälle, mis, mis kuidagi sinna kontriculteerisid, pluss on see, et me muidugi üritasime ka siis teha mänge enne veel, kui me pruumooniga tegime midagi, nagu mängu ideed see sees kuidagi paralleelselt selles amiga maailmas me üritasime ka midagi teha, vaid ta ja Juhan Soonets
Me tegime Rockexi-nimelise mängu, mis pärast ka ploomul keeras pissi peale palju ägedama, võib-olla võib-olla vähem kena ja, ja siis selle intro, ma mäletan, oli väga selgelt kantud kõigest sellest vee kaheksatest ja, ja rakettidest jama ja kõikides väga tähtis oli kindlasti see, et päikseprillid olid õige kujuga ja siiamaani ja, ja muidugi Andrus Aaslaid veel kõrvale rääkis lugusid, kuidas või, või kirjutas või, või teleri, kuidas siis mingi prinki valgusega saab su aju ümber programmeerida. Ja noh, see, see, see kõik nagu absopeerusel tekitas mingisuguse omaette alternatiivse reaalsuse ja ma arvan, et see on see meie, meie arusaam küberpungist, aga enne tõu kohta küsida siis mind hakkas huvitama see, et kui sa räägid, et te kujutate, siis kuidas tootmine maailmas peast on. Ajust on ainult niisugune kodu ja natuke Tšiped. Ometi rohkem silpe, vähem rohkem sitta raamatut. Kõlab nagu väga hea teise maadel. Aga ometi see te nagu astusite pika sammuga toe tuleviku poole, ilma mingit kõhklust loomata, et see on nagu õige suund sinna, sinna tuleb minna. Seda nagu tagasi hoida on nagu, tõenäoliselt on mõttetu, sest et noh, Loviidid ka üritasid midagi aine, aga, aga noh, parem on olla seal enne teisi. Et sa paned juba õiget Tzipi taha ära ja võtad selle pudru osakaalu väiksemaks. See tahab ikkagi siukseid mõtteid mõelda, üheksakümnendatel see tahab ikkagi nagu visiooni saada. Plumbum, kuidas sa sattusid selle ahti ja jaanitule? Jällegi, et ma kuidagi, kuna ma olin kogu selle super häkkerid, noh, kes igalt ostult toona võib, igaüks mõtles, et on super häkker, see, kellel oli, kattis see mees seksi Vano All või ka see, kes oskas neid faile kokku panna, nii et ma olin üks väheseid tegelasi, kes joonistas pilte. Ja jällegi ma ma tegelikult oskan, oskan ka ilma arvutita päris hästi joonistada, et see on lihtsalt seal tundus see kuidagi nagu lahedam, et seda sai salvestada seal see on tuul ja see on nagu põhiline, et kui sa lihtsalt joonistad, siis saanud uut teha on väga raske. Isegi võiks öelda, et võimatu peaaegu ja kuidagi jällegi, et see seltskond ei olnud nagunii suur ja nad kõik nagu kollektisid kuidagi, eks ole, näiteks seesama nagu siin ka teised on rääkinud, et see sihuke välismaailma riitsimine, et, et see kõik oli nagu nii see sisemaa ja välismaale, et see kuidagi ikka klikist kokku, et et sellega ma teen korra kõrvalehüpe, kiire, tuli lihtsalt meelde, kuidas näited, kuidas Peeveeesside ja värkidega suheldi, et et meil oli sealsamas telemajas oli täpselt samasugune ambitsioon, meile lihtsalt oli see, et olid amiga, ta on ju ja mänge ei olnud, siis pidi neid mänge kuskilt jälle piirama on ju ikka. Ja ma loodan, et tagantjärgi mägi kahekümnendat pidamisasjatundjad ei hakka, ei hakka need peale lendama, aga igal juhul üks viis oligi see, et sa pidid noh, täpselt jõudma kuskile mingisugusesse Peeveeessi, onju ja, ja, ja, ja kuidagi sinna sisse pääsema ja ega see oli olnud niimoodi, et tasku sisse, et seal nagu ennegi on kuulnud satavastist mingid vennad, kes monitorist tegevust jälle Eesti tundus eksklusiivne, niisugune veider koht on ja see on sama hea kui eskimo naine. Ja, ja, ja mingeid, meil isegi tekkis mingisugune treeningpassid, et meil juba oli midagi, mida nagu vastu pakkuda, aga tavaliselt me ikka mängisime sellist vaest sugulast ja, ja siis oli ka, et me isegi noh, nagu bluuboksisime ennast sinna sisse ja noh Otto, tundub, et räägime sellest lähemalt, see tähendab siis seda, et sa pidid kuidagi toonase Telekomi keskjaamale midagi kõrva vilistama, mida too kuulda ei tahtnud tingimata. Kusjuures nüüd, kui ma hakkan mõtlema, et igasugused Margus oli, meil põhineb loobuksid spetsialist, aga kas me nagu reaalselt loobuksime sinna ka jõudsime, seda ma nüüd hea küsimus, sest ma mäletan, et punkteksidele küll manuaal, sellega ja, ja aga põhimõtteliselt samad asjad lugeda, see on ju iseenesest üsna lihtne, kuna need vidinad toona olid suhteliselt rumalad, on ju näiteks simkalle võimeta, kellelegi ma annaksin või kellega sa rääkisid, et et kui kiired olid modemine, mina mäletan, seda tõi kolmesaja Heyes. Ma mäletan ka seda, et minu meelest ma ei mäleta, kas oli mast või keegi oli väidetavasti suuteline händseigi ära vilistama sellele kes ta oli nagu piisavalt aeglane. Et kui sa nagu raagu suutsin talle suusõnaliselt selgeks teha, et et see on see legendaarne käpp, Francise vile, mida Ameerika moel jagati, mis kaks tuhat kuussada üldse välja vilistas täpselt ja, ja just-just-just, sest seda saab kasu kui teha, kui vaja. Ja aga jah, et see kõik oli nagu kuidagi jällegi, et võib-olla oli see, et igaühel meist oma fookus on ju, et kõik, kes oli, tahtis rohkem seal mingit Networki, äkki ta on ja kes tahtis rohkem lihtsalt häkkida häkkida, kes tahtis progeda? Jällegi mind huvitas, mida ütlesid selgelt, mind hoidsid mängud, liikuvad pildid, värvilised pildid, kuidas neid ise teha, kolm tee kõik niisugune värk, et ma nagu, nagu olin. Pigem nagu otsis neid võimalusi, eks see viis meid ka tegelikult kokku siis lõpuks pruumoni pundiga, kellele oli lihtsalt ta oli kindel soov, et nad tahavad teha mängu, onju. Ja kuna minu jaoks oli see lihtsalt natukene niisugune nagu nõme ülesanne, kuna neile oli, mul oleks amiga, seal oli kuradi Maidan miljonit värvid on ju, neil oli mingisugune VGA ekraan ei, alguses ühtegi C ja sinna pidi mingi nelja värviga midagi valmis hingeldama väga palju, et noh, teeme ära. Ja, ja, ja seal kõige naljakam on see, et see kosmonaut, mis sellest sündis. Jällegi see, kuidas sa turustasid, mis ma nagu lihtsalt vaatasin ja imestasin. Ja järelaeg mul nõrgurist meeles seekord nagu telgeitid, need vennad olid lihtsalt toona ja on vist siiamaani, eks. Et nad noh, olid nagu tõesti nagu Pakosta teeme seda asja, aga see, see graafiline pool oli, oligi nagu super lihtne, nokkisin valmis. Siis nad tegid oma selle mingi mingi musa, ehitage sinna marakesin ka mingisugused ikoonid, mingit kitarre ja mängi trummid ja väga vaheli. See tähendas seda, ma mäletan küll, et inimesed, kes oskasid arvutiga joonistada, et neid oli nagu vähe ja kas sul tekkis. Ühesõnaga, et kas sul kõigepealt tuli arvutiga joonistamine ei siis joonistamine või oli sul enne ka joonistada? Ei, ma ikka Ene Jaaniste, selles ma noh, jällegi kes ikka ennast kiidab, kui mitte ise, eks ole, ma nagu. No akadeemilist joonistamist ma valdan nagu suhteliselt väga-väga hästi okeid, et selles mõttes mul ei olnud nagu keeruline omandada enamus inimesi, vaatasin toona joonistatud tarvetitega ega mini, vaid oli seesama kuradi munaga hiir, mille, eks ole, muna aeg-ajalt jooksis mingit kuradit pahna täis, siis jälle küünega puhastama. Ja, aga noh, ma ikkagi endale võtsime kiire, mis nagu enam-vähem jooksis, on ju, et selles mõttes minu jaoks ei ole vahet, kas see on pliiats või, või, või Nov tablett või, või mad Evelin ja et see on nagu noh see arvuti oli sinu, see nii-öelda kunsti tegemise ja selle selle ande nii-öelda laiendus. Jah, ta oli lihtsalt ütleme nagu mingi teistsugune tehnika ja tunduvalt andeks andma kui näiteks mingi akvarell või mis iganes. Et selles mõttes oli.
Täna sa vaatad, eks ole, et et iga kõik kunstnikud kasutavad mingit Maidamegidzindiku tabletti või, või noh, neid noh, neil on kõik super ägedat toolid on ja ma siiamaani aeg-ajalt, kui mul on vaja omanikega midagi, kus täna maanikerd on tegelikult temale parim tants päri ja kõik vaatavad, et ma olen peast soe. See on nagu noh, mugav ja käe järgi tuua tegelikult, et kui seda raamandade keerata, keerad kursori viinud kiireks. Ühel hetkel sul tekkis see koht, kus tekkis mõte, et võiks hakata Weeki tegema. Ja see oli ka Pauluse mainib, et siin jah, see oli ka nagu pigem läbi selle, et kuna ma olin aru saanud, et ma ei ole piisavalt järjepidev ja, ja see sealt ütleme, see pragemise osa oli nagu kõik tundus toona liiga kui välja. Et noh, mul olid sõbrad, kes sellega tegelesid ja, ja miks see üldse nagu see tulemus ei olnud seksikas. Aga teeb järsku noh, ta jälle algusest oli ka mingi superpoorid on ju, et seal ei olnud nagu noh, mis seal oli see osaühing või mis esimene oli see sealt pikemasse? Adria seljaga siis kui ma sain aru, et kui sa said juba tabelite keelata ported maha ja sinna mingite üksikute ühe mikstiste tükkidega seal neid hakata mingit leiavadki tegema ja siis ma olin müünud mees, siis ma nagu hääletades ka siis ma tahtsin ühes reklaamibüroos ja midagi sealt katsetasin, nokkisin. Ja, ja siis mainitakse juba olemas. Ja, ja selle asutajad tulid siis otseselt andmed maja veel natuke reklaamimise, sest ka et paneme midagi mingit seljad kokku ja hakkame, hakkame seal nagu vaatama ja eks seal jällegi, et lihtsalt see, et ma sain mitte lihtsalt enam selle pildi oma käe seest ekraanile, vaid ma ise sain selle pildi nagu pauh kõigile nina ette, eks ole, paljudel ekraanidel jah, ja, ja, ja, ja toona oli noh poisikesed, mis ma tegelikult ikka nimega poisikesed olime, siis oli juba aasta oli siis kakskümmend viis või? Üheksakümnendate keskpaik, teine pool. Jah, jah, et üheksakümmend kuus, üheksakümmend seitse oli juba, et see siis oli juba, noh, Siis sa juba Business teha, siis. Panid jälle, tegid lõikest neid piksleid ja, ja mingisugused, ma ei tea, ma mäletan, meil oli klient, oli Reval Hotel Group, et noh, siuksed mingid nagad tulid ja võtsid siis kliendid ja tegid neile mingeid ägedaid asju. Sõber. Ja, ja see jätkuvalt oli see, et et sa said omale selle pildi panna inimestele silma, et eks ole seal sees ju liigutav faktor, ma olen isegi võib-olla see, et tegelikult mul ei oleks isegi vahet, on kas sa nagu inimestel silma ette vaid just nimelt see, et sa tegid mingisuguse, sul oli mingisugune distsipliin siis selle seal mingisugune HTML, onju. Ja, ja, ja, ja sa teadsid, kuidas optimeerida sa oled, seal oli mingisugune mingisugune tuulised ja saab alles seda suhteliselt hästi ja see tekitas sulle nagu rõõmu, et sa said sellega teha mingisuguseid asju, mida võib-olla teised nagu aga saab teha. Ja, ja see Jobs saatis, läksin värk, et tegelikult noh, ega ma kujutan ette, et muru niitmine on ka selles mõttes lahe, sa näed, kuidas on nagu Oru ja taga maha niidetud selle instan suhteliselt instanud prätifykeissimat. Võib ka nii mõelda, et, aga noh, samas kui sa muidugi ei oska, siis ei tule mingit hetke kehvade isenditega täisajaga ja see ongi, et sa tead, et sa oled sinna mingil määral panustanud, on ühe seal on nagu mingisugune Technical, et on ju, et sa saad. Et, et iga mats ei tule, ei tee seda, et sa saad nagu öelda, et Aak maisse, mis sa praegu teed. Praegu.
Ma olen kuidagi lihtsalt distantseerunud sellest disaineri rollist aga samas mitte, et jälle ajab ikka oma arusaamad, tegelikult selle pildi tegemine on, mõnes mõttes võiks nagu niisugune käsitöölise töö, et tegelikult need lõikelauad, mida minu lapsepõlves turul müüdi, kus olid need selle põletiga oli tehtud Nuubavaliibiale, onju ta nagu veits sarnane, et palju on Tšehhimaad tegelikult võtta ja aru saada mingitest äriprotsessidest või mingisugusest inimeste mõttemallidest ja disainida neist midagi ja, ja, ja jällegi, et see progemine minu jaoks on see, et kui see õigesti sõnastada, siis mingisugused vennad teevad selle valmis ja see, see muutub nagu päris, eks ole, et seal seal jällegi sellise protsessi toetav mingisugune asi seal masina sees mis toimetab täpselt nii nagu sa oled talle nagu öelnud, et toimetan ja et et seal olid need nüüd mitte ei ole, tule sinu käe seest sinna, see pilt vaid tuleb sinu pea seest mingi mõte, kuidas see kupatus võiks käia, Swing programmeerib, valavad selle valmis ja siis käibki niimoodi just just et mul oli lapsepõlves oli kuidagi ma mäletan selgelt, et mul oli mõnus mõte, et tehas on tore asi, sest et ta võtab mingisugused toorme ja see pannakse kokku mingite detailide, siis pannakse sellest kokku mingi asi. Ja jällegi, et me jõuame sedasama selle Nintendo Diva esijuurde, et, et füüsilisel kujul seda toota. Jõle tüütu, palju lihtsam oleks teha seesama asi, nii et oleks bittide jada, mis kõik liigrupeeruvad moodustavaid mustreid ja sellest nagu peaaegu nagu võluväel, eks ole, tekivad mingisugused asjad, mis inimestele tegelikult on tänaseks sama reaalselt tööriistad kui haamer ja, ja höövel, eks.
Nii on.
Aitäh ja sain küsida palju huvitavaid küsimusi, palju targemaks. No ma loodan ka, et ma nagu liiga ei läinud rändama, et see on juba kõik, on, kõik on väga hästi.
Aitäh sulle, aitäh Sulle.


\chapter{Tarmo Mamers}
\index[ppl]{Mamers, Tarmo}
\index[ppl]{MomraT|see{Mamers, Tarmo}}


Loodetavasti tekib siia jutt Oktroobrirajooni ÕTK kohta\label{content!OTK}, B'Knows viitab.



\chapter{Tarvi Martens}
\index[ppl]{Martens, Tarvi}

\question{Kuidas sina said arvutite ja arvutid sinu juurde?\sidenote{Kuna Tarviga rääkisime juttu mitmel 
korral, on jutulõng mõnevõrra hüplik. Katkemiskohad on tekstis markeeritud.}}


Ma olen pärit Pärnust ja seal arvuteid minu meelest tollal ei olnud, aga 
ma käisin olümpiaadidel, nii et matemaatika ei olnud minu jaoks 
mingi teema. Viiendas klassis
võitsin kuuenda klassi matemaatika linnaolümpiaadi, mille üle kõik olid suhteliselt 
jahmunud. Ühe riikliku olümpiaadi käigus viidi meid 
ekskursioonile Nõo Keskkooli\index{Koolid!Nõo Keskkool}, kus oli suur arvuti. See oli teistsugune maailm, aga kui mind sinna õppima 
taheti viia, siis ma ei tahtnud väga minna. Mul oli Pärnus oma bänd.

\question{Sul oli oma bänd?}

Jah. Tegime punki nagu ikka sel ajal. Käisin Pärnus muusikakallakuga koolis ja bänditegemine oli 
elementaarne. Kooliteater tegi ka oma esimesi samme. Kadunud Aare Laanemets\index[ppl]{Laanemets, Aare} ja Elmar 
Trink\index[ppl]{Trink, Elmar} tegid esimese kooliteatri, kus ka mina osalesin. 
Kõik see oli nii tore ja ma mõtlesin, et ei viitsi kuhugi kaugele 
kooli minna. Aga matemaatikaõpetaja käis mu vanemate juures, rääkis nad 
pehmeks ja nii see läks. 

\question{Kas sel ajal Nõo legend alles kujunes või oli see juba tuntud paik?}

Jah, oli kindlasti tuntud. Oli teisigi tugevaid koole, 
Tartus-Tallinnas, aga Nõo kool oli üle kõige. Põhiliselt 
sellepärast, et neile oli oma arvutuskeskus ehitatud, nii et sinna tuldi üle 
vabariigi kokku. Samas enamik olid ümberkaudsed maalapsed, kes ei olnud võibolla väga suured geeniused. 

Nõo Keskkoolis oli Nairi 3-1\index{Arvutid!Nairi!Nairi-3-1}, niisugune 
\emph{mainframe}, millele sai perfolinti sisse sööta ja laiprinterist 
tulemuse välja printida. Aga see ei tundunud väga huvitav. Umbes üheksanda klassi poisina 
avastasin Tartu Ülikooli Vanemuise õppehoone\index{Tartu 
Ülikool!Vanemuise tänava õppehoone} keldrikorruselt kabineti, kus oli 
kaks ja pool Apple II\index{Arvutid!Apple II}. Kaks ja pool sellepärast, 
et üks oli kogu aeg katki ja Andres Peiker\index[ppl]{Peiker, Andres}, kes oli 
selle keldri kunn, remontis seda.

Koolipoisina konkureerisin arvutiaja pärast tõeliste 
üliõpilastega nagu Tanel Tammet\index[ppl]{Tammet, Tanel}, Margus 
Liiv\index[ppl]{Liiv, Margus} ja teised. Sain ennast kuidagi 
vahele pista ja enamiku ajast ei käinud enam väga palju 
koolis, vaid olin rohkem Tartus.

\question{Ometigi oli Nõo kool mõeldud sinusuguste harimiseks süvendatult. Kas sul oli vaja veel rohkem süvitsi minna?}

Mis sa seal Nairi juures perfolindiga harid! Saatuse vingerpussina saabus 
aasta hiljem, kümnendas klassis Nõo kooli hunnik 
Agate\index{Arvutid!Agat}, mis olid Apple II kloonid, 
ainult värvilised. Kõige naljakam oli see, et kohalikud arvutiõpetajaid ei 
teadnud nendest midagi ja siis tuli välja, et on üks Tarvi, kes tunneb Agati
protsessorit läbi ja lõhki. Sel olid küll oma operatsioonisüsteem ja 
venekeelsed programmeerimiskeeled, aga sellest polnud midagi. Nii et ühel hetkel 
oli mul arvutuskeskuses oma kabinet ja arvuti. 

\question{Kas selleks piisas Tartus Apple II uurimisest? Kas said sahibide 
vahel noka piisavalt märjaks, et Nõos kunn olla?}

Täpselt nii, pärast õpetasin õpetajaid. 

\question{Kas Agat oli Apple II kloon kuni riistavara disaini ja arhitektuurini 
välja?}

Vähemalt protsessori mõttes oli see kindlasti sama. Ma ei ole väga suur riistvara 
asjatundja, kuigi assembleris\index{Keeled!Assembler} programmeerisin 
vabalt sel ajal. Küllap see oli üsna täpne kloon, aga 
värviline võrreldes Apple IIga. See tähendab, et pilt virvendas kogu aeg 
silme ees. 

\question{Kui sa omale kabineti said, kas siis oli uhke tunne?}

Mis seal ikka erilist oli. Hea oli see, et sain oma asja ajada ega pidanud enam Tartu vahet 
käima.

\question{Kas see õppimist ei hakanud segama?}

Ei hakanud. Mul ei ole sellega kunagi probleeme olnud. Tuleb 
lihtsalt kontrolltööd ja eksamid ära teha ja siis keegi ei õienda.

Nairi peal olid tõsiste inimeste keeled nagu Algol\index{Keeled!Algol}, aga 
lastele õpetati programmeerimiskeeli ROPS\index{Keeled!ROPS} ja 
KÕPS\index{Keeled!KÕPS}\sidenote{Vt ka märkust 3 lk
\pageref{sidenote:ROPS}.}, mis olid eestikeelsed. KÕPSis 
sai programmeerida joonistamist, näiteks kuidas plotter 
liigub: mine üles, mine alla, mine paremale; jäta joon, ära jäta. ROPS oli 
päris programmeerimiskeel. Ma tegin need keeled ka Agati peale ringi, et 
lapsed ei peaks Nairiga tegelema. 

\question{Matemaatika tuli sul lihtsalt, aga kuidas matemaatikahuvi läks üle nii suureks arvutihuviks, et käisid Nõost Tartus arvutis ja portisid programmeerimiskeeli? Mis 
sind selle puhul tõmbas?}

See on hea küsimus, aga mul ei ole head vastust. Arvuti oli selgelt täiesti 
teistmoodi, nagu praktiline matemaatika – rehkendusmasin, mis on kalkulaatorist intelligentsem. Mõtlesin vist
juba siis, et see on paratamatu tulevik ja teistmoodi ei saagi olla. 

\question{Huvitav, et sul on matemaatika ja arvutite seos algusest peale selge 
olnud. Mõnel tekib see seos palju hiljem kui üldse.}

Matemaatiline loogika on olnud kogu aeg üks minu lemmikdistsipliine, arvutid 
ja muusika on väga loogilised asjad. 

Ühel hetkel lõpetasin kooli ära ja läksin TPIsse\index{Tallinna 
Tehnikaülikool}.

\question{Miks sinna? Tartu Ülikool oli ju sulle juba tuttav.}

Mulle tundus, et TPI oli natukene praktilisema hoiakuga, ja aastal 
1987 räägiti Tartu Ülikooli informaatika kohta, 
et seal rohkem ikka joonistatakse tahvli peale. Ja päris matemaatikuks ma kindlasti 
ei tahtnud saada.

Tegelikult olin Tallinna vahet enne käinud. Seal oli Õpilaste 
Teaduslik Ühing\index{Õpilaste Teaduslik Ühing}, kus Peeter 
Lorents\index[ppl]{Lorents, Peeter} tegi matemaatikasektsiooni. Käisin 
Lorentsi juures aeg-ajalt, ta andis mulle kaelamurdvaid 
ülesandeid. Kahekordsete integraalidega 
elu oli huvitav, nii et TPIsse minek tundus loogiline.

\question{Mida sa õppima läksid?}

Automaatikateaduskonda ja eriala oli
LI\index{Tallinna Tehnikaülikool!Automaatikateaduskond!LI} ehk arvutid ja 
arvutitehnika. Seal juhtus kohe mitu asja. 

Kõigepealt ütlesin esimeses programmeerimistunnis, et siia tundi ma rohkem 
ei tule. Õppejõud ei solvunud, sest kirjutasin sissejuhatavas tunnis salaja
ühe programmi valmis ja näitasin seda talle.

Teiseks oli Teaduste Akadeemia Küberneetika Instituudi 
Erikonstrueerimisbüroo\index{EKTA} juhtimissüsteemide osakonnas\index{Teaduste Akadeemia 
Küberneetika Instituut|see{Küberneetika Instituut}}\index{Küberneetika 
Instituut!Juhtimissüsteemide osakond}\sidenote{Esineb ka nimekuju Arvutustehnika Erikonstrueerimisbüroo ja
Arvutustehnika Arendusbüroo, mis paistavad viitavat samale asutusele.} just
leiutatud kooliarvuti Juku\index{Arvutid!Juku}. Nad asusid sealsamas Küberneetika majas, kus olin juba käinud, ja 
septembri esimesel nädalal sadasin sinna sisse. Mul jäi 
õpilaste keskkondade pärast mure, et kui tuleb kooliarvuti, siis võiks olla ka 
õpilastele mõeldud programmeerimiskeeled, ja ROPSi\index{Keeled!ROPS} portimine 
Jukule oli tegemata. Rääkisin Juku tegijatele, et oleks vaja vastavasuunalist 
arendust. Nad lubasid mul enda juures hängida ja nelja kuu pärast 
olin tööle võetud. 

\question{Kas ülikool jäi kõrvale?}

Ei jäänud, käisin korralikult eksameid tegemas. 
Vahepeal, pärast esimest kursust, käisin Vene kroonus ka. Olin viimane 
lend, kes sai kroonusse minna, ja olen selle üle väga õnnelik. Meid viidi Leningradi lähistele, aga kuna
sain puhkpilliorkestrisse ja tegelikult tegin jälle bändi, siis polnud häda midagi. 
Jälle üks kogemus juures. 

Kroonust tulles paljud langevad ülikoolist välja, sest leiavad, et võiks 
midagi praktilist teha ja ennast targaks ajamine ei tasu ära. Mulgi 
oli teise kursuse poole peal kriis, kui mõtlesin, et mul on kohal 
käimata ja et kui eksameid ära ei tee, siis on kõik. Aga tegin 
eksamid ära ja võtsingi selle elustiili, et pühendasin ülikoolile umbes 
kolm nädalat poole aasta kohta. Imesin materjali sisse, tegin eksamid ära ja 
kõik töötas. 

\question{Minu puhul möödus keskkool mängides ja lauldes, sest 
kõik oli lihtne, kuid ülikooli minnes lõppes lihtsus ära. Kas sinul ei 
lõppenud?}

Lihtsus lõppes tõesti. Õigemini olid keerukad esimesed poolteist või kaks 
aastat, kui taoti pähe fundamentaalset kõrgemat füüsikat ja matemaatikat, mis lööb kaane pealt ära. Aga edasi läks erialasemaks 
ja inimlikumaks, õppimine ei olnud enam nii teoreetiliselt tappev. 

\question{Kas ülejäänud aja tegelesid Jukudega?}

Ei, kui kroonust tulin, oli kontorisse toodud juba esimene 286. Oli huvitav aeg, et käisin küll 
tööl, aga tööd oli vähe. Kui 
leidsid endale haltuuraotsi, oli suhtumine väga soosiv. Kõige suurema haltuuraotsa puhul, 
mida mäletan, tuldi koos arvutiga. Sain personaalse arvuti ja 
tööandja eraldas ka kabineti. 

\question{Kes need haltuurapakkujad olid? Kas oskad mõne näite tuua?}

Igasugused. Arvutiga tuli Soome laevaehitaja. 
Pean seda siiamaani kõige vingemaks programmiks, mille ma olen teinud. Ülesanne 
oli selline, et on kümne tekiga sõjalaev, mis vajab 
elektrivarustust; kuskil on jõuallikad ja kuskil tarbijad. Ja nüüd tuleb
hakata nende asjade vahele erineva jämedusega kaableid vedama. Kaablirennid 
on olemas, aga ühel hetkel saab kaablirenn täis. Mis me teeme? Veame 
teistpidi. Aga kes ütleb, et kaablikulu on sealjuures kõige optimaalsem? 

\question{Kas siis oli veel sügav Nõukogude aeg?}

Ei, siis oli juba sula ja hell aeg. See oli pärast kroonut, 1990 või 1991.

\question{Sel ajal ei tohtinud isegi mitte arvuteid 
Nõukogude Liitu tuua, aga sina arvutasid sõjalaevade kaableid.}

Kes seda ikka teadis. 

\question{Kuidas see haltuurapakkuja oskas sinu juurde tulla?}

See õppejõud, kellele esimeses tunnis ütlesin, et 
ma rohkem ei käi sinu juures, leidis mulle otsi. Inimesed 
teadsid mind ja oskasid soovitada. Just ülikooliajal sai väga 
eripalgelisi asju tehtud. Ma olin siis kõva programmeerija, kirjutasin muu hulgas 
oma andmebaasisüsteemi, mis oli FoxProst kordades kiirem. Vanasti oli 
kõvaketta poole pöördumine ränk tegevus, mis võttis 
aega, mitte nagu praegu SSD puhul. Ma kirjutasin andmebaasisüsteemi, millel olid 
fikseeritud pikkusega väljade asemel sujuva pikkusega väljad. See tähendab, et andmeid oli ketta peal täpselt nii palju, kui oli, mitte ei 
olnud eraldatud kindel hulk megabaite. Tõmbasin
keskmise andmebaasi umbes kaheksa korda kokku ja vastavalt sellele suurenes 
töötlemiskiirus.

\question{Kuidas sa kirjeid pakid ja mis saab siis, kui välja pikkus 
muutub? See ei ole ju lihtne.}

Miks see peaks lihtne olema? Mis see geniaalsele programmeerijale ja 
matemaatikule ära ei ole välja rehkendada? Nagu sõjalaevade kaalutud 
graaf, milline on kõige optimaalsem kaablikulu. 

\question{See tegevus läheb otsapidi teadusse, mujal maailmaski ei olnud
andmebaase teab mis palju. Kas sa teadlaseks ei tahtnud saada?}

Ei, mulle meeldis praktiline pool. Lõpuks läksin pika hambaga 
magistrantuuri ja virelesin seal umbes kuus aastat. Siis kui
ainepunktid hakkasid ära kustuma, tegin jõuga lõputöö. Mulle kuiv teooria ei paku 
eriti midagi, mulle meeldib maailma muuta. 

\question{Kas sa olid kuulus ka?}

Ei olnud. Eks ühe või teise tehtud töö tõttu renomee levis ja ka õppejõud
Peeter Lorents\index[ppl]{Lorents, Peeter} levitas sõna, nii et kõik käis
tutvuste ja sidemete kaudu. See ei olnud massiline, tegin umbes kümmekond projekti, aga need olid päris suured.

Tööasju tegin ka loomulikult, aga tööd oli toona vähe 
ja mentaliteet oli selline, et parem olgu inimene olemas ja valmis. Kui tööd 
tuleb, siis saab seda teha. Too kontor, mis on tänase nimega 
Ektaco\index{Ektaco}, oli fantastiline koht. Seal oli umbes 
viiskümmend inimest, tehti riistvara ja tarkvara, \emph{fifty-fifty}. 
Juku oli muidugi nende tehtud. Muu hulgas tegi Elleri-papi 
ehtekarbist valmis esimese hiire maailmas\sidenote[][-1cm]{Arvo 
Eller\index[ppl]{Eller, Arvo} oli Juku loomise eestvedaja (Ants Vill (2010). 
Meenutusi aegadest, kui arvuteid tehti veel käsitsi. Linnaleht (Tallinn), 
46). Kas tema loodud hiir just maailma esimene oli, aga ehtekarbi lugu kordab 
ka viidatud allikas.}.

Pooled inimesed olid \emph{cum laude} TPI lõpetanud, nii et sealne 
ajupotentsiaal oli nauditav. Näiteks kui ülemusel oli sünnipäev, siis
vennad mõtlesid, et teevad kingiks rääkiva papagoi. Tegidki. Seal oli 
briljantseid ja lahedaid tüüpe. 

\question{Mis see töö sisu seal ikkagi oli? Kas ise mõeldi projekte välja?}

Nii ja naa. Üks põhiline valdkond oli 
tööstuskontrollerid: ise mõtlesid välja, ise tegid, ise programmeerisid. Need olid 
\emph{rack}'i-suurused, täna saab samasuguse asja osta Hiinast 
kiibisuurusena. Kontroller koosneb analoogsisenditest ja 
-väljunditest, digitaalsisenditest ja -väljunditest ning nendevahelisest 
loogikast. 
Tollal oli vaene aeg ja Ektaco\index{Ektaco} tehti ühisettevõttena ühe Soome partneriga. Tänase 
päevani teevad nad kassasüsteeme Compucash, mida võib 
baarides aeg-ajalt siiamaani näha. Toona tuli soomlane ja ütles, et tehke mulle 
proovitöö – selline maatriksklaviatuur, et kui baarmen vajutab \enquote{õlu}, on 
kohe olemas. See tuli välja ja koostöö jätkus. Tollal ei olnud lihtne 
tellimusi leida, seetõttu suur osa
inimesi istuski pool aega jõude. 

\question{Ja sina muudkui programmeerisid?}

Mina muudkui programmeerisin. Ektacos\index{Ektaco} olin kokku viis aastat, enam-vähem kogu
ülikooliaja. Aastal 1992 läksin siiski tagasi
nii-öelda peamajja, Küberneetika Instituuti\index{Küberneetika 
Instituut}. Seal tekkis uus rakuke, mis esialgu alustas krüptograafia alusuuringuid. Seltskonnas 
olid mõned teadlase moodi ülikoolipoisid ka, näiteks Ahto Buldas. Ülo Jaaksoo\index[ppl]{Jaaksoo, Ülo} oli 
toonud välismaalt paksu raamatu krüptograafia aluste kohta ja seda me siis koos 
lugesime. Keegi luges peatüki läbi, proovis aru saada ja seletas 
teistele ka. Krüptograafia kui teadus Eestis puudus arusaadavatel põhjustel. Kui Eesti
iseseisvus, oli plats lage ja kuskilt pidi alustama.

\question{Kuidas mujal maailmas oli krüptoga? Mis tolleks hetkeks juba 
olemas oli?}

RSA oli olemas, aastast 1978. Ma täpselt ei tea, sest ei ole ennast 
kunagi krüptoloogiks pidanud. Minu eriala on rohkem nii-öelda 
rakenduskrüptograafia, mitte süvakrüptograafia.

\question{Miks sa sinna läksid? Sul oli Ektacos ju mõnus oma projekte teha.}

Pooled inimesed olid suurepärased insenerid, lõpetanud \emph{cum laude}, aga firmas ei saadud aru, et nende arenguga peaks tegelema. 
Oli väga selge seisukoht, et igaühe areng on tema enda asi. 
Interneti panek firmasse, ajakirjade ostmine või 
inimeste saatmine konverentsile ei tulnud kõne allagi. Pinge 
kogunes ja mingil hetkel, oma sünnipäeval, saatsin kohalikku võrku essee, mis firmas valesti on, mida tsiteeriti
pärast aastaid. Kümme aastat hiljem võeti see välja ja vaadati, et ikka on sama lugu. 

\question{Kuidas see kamp ülejäänud Eesti kogukonnaga kokku käis? Tol ajal pidas osa inimesi juba BBSe.}

Mu hea sõber ja kolleeg Heiki Kask\index[ppl]{Kask, Heiki} pidas ühte 
BBSi ja ma liitusin sellega. Sealtkaudu sattusin lõpuks fidonautide 
sekka ja hakkasin nendega läbi käima. 

\question{Kas see ei olnud sinu jaoks tähtis asi?}

Fidonet ei olnud minu jaoks tähtis, see oli lahe ja andis 
esialgse maigu suhu, aga nii kui tuli Internet, armusin sellesse.

\question{Mis interneti juures nii armastusväärset oli? Meile ja uudiseid 
sai Fidoneti kaudu ka.}

Meil oli esialgu UUCP ja modemiga helistamine mitu aastat, 1991–1993, kui ma 
ei eksi. Sai meili saata, mis oli väga tore, aga mulle jõudis kohale, et kuskil on 
olemas nii-öelda püsiühendusega internet ja suhelda saab reaalajas\sidenote[][-.8cm]{Mõiste \enquote{püsiühendus} oli tol ajal maagilise 
tähendusega: ei unistatud mitte kiirest, vaid pidevalt ühendatud 
internetist. Võimalus kaugete arvutitega vahetult suhelda tundus imeline.}. 
See oli minu jaoks nii võluv, et 
loomulikult tahtsin seda ühel või teisel moel uurida. Nii et UUCP 
aegadel mäletan ennast pühapäeviti kuskil modemi küljes rippumas ja RFCsid\sidenote{\emph{Request For Comments (RFC)} on juba alates 1969. aastast kasutusel olev standardne viis kõiksugu internetiga seotud standardite avaldamiseks ja kokku leppimiseks, RFCd on nummerdatud ja tuntudki oma numbrite järgi. Need sätestavad sõna tõsises mõttes kõike alates Interneti alusprotokollidest kuni tuvide abil side korraldamise (RFC 1149 — A Standard for the Transmission of IP Datagrams on Avian Carriers, D. Waitzman, 4/1/1990, 2 pp.) ja kohvi keetmiseni (RFC 2324 — Hyper Text Coffee Pot Control Protocol (HTCPCP/1.0), L. Masinter, 4/1/1998, 10 pp.).} 
alla laadimas, et need kõik algusest peale läbi lugeda.

\question{Kas see oli tol ajal võimalik?}

Oli küll. RFCde ülemine ots oli kuskil tuhande kandis alles, nii et see ei olnud 
probleem. Osad olid lühikesed ja osad mõttetud, ja oli ilmselge maniakaalsus 
koguda endale hästi palju materjali, et küll ükspäev loen.

\question{Kas seal uues üksuses oli internet sinu jaoks siis infoallikas?}

Eks jah. Sai meili kirjutada, lahe värk. Enne veebi olid 
põhilised FTP-saidid – ei pidanud mõtlema, mis \emph{node}'ist või kust 
mida saad. Mõnikord sai FTPst ka mõne mängu kätte, seal ikka liikus kraami. 
Seal sai ju samamoodi alla ja üles laadida, nagu Fidonetis. 

\question{Kas sa mängisid arvutimänge ka?}

Suur mängumees ma ei olnud, aga noorest peast midagi ikka õhtuti põristasin 
ja täristasin. See oli lõõgastumisviis, mitte huvi. 

\question{Sinu fookus oli matemaatikal.}

Programmeerimisel, mulle meeldis arvutit oma pilli järgi tantsima panna, mitte 
arvuti pilli järgi tantsida. Kui Windows\index{OS!Windows} tuli, 
siis ma kaotasin usu arvutitesse, sest ma ei suutnud enam igat 
bitti kontrollida. Kuni sinnamaani teadsin opsüsteemi, EEPROMi 
tasemel, mis sünnib, aga nii kui Windows tuli, siis kontroll kadus ja mul läks tuju ära.

\question{Kui tekkis Linux\index{OS!Linux}, kas siis tuli tuju tagasi?}

Linux aitas jah Windowsi aja üle elada, aga hulluks 
Linuxi kasutajaks ma ikkagi ei hakanud. Kui läksin 
Ektacost\index{Ektaco} Küberneetikasse\index{Küberneetika 
Instituut}, siis jätsin programmeerimise maha. Viimane asi, mille 
tegin, oli 1996. aastal mail.ee\index{mail.ee}. 

\question{Miks sa selle tegid?}

UNDP\sidenote{\emph{ÜRO Arenguprogramm}. Üheksakümnendatel läks Eesti veel üsna 
selgesti arengumaana kirja ja sai paljudest kanalitest igasugust abi. 
Tänaseks on humanitaarabi mõiste õnneks suuresti ununenud, kuid toona tuli 
seda kõikvõimalikul kujul päris palju ning oli tõesti abiks.} andis selle tegemiseks väikse grandi.
Kõigepealt tekkis hea mõte, et igal soovijal võiks olla meiliaadress. 

Pean alustama sellest, et 1994. aastal sai tehtud 
firma Teleport\index{Teleport} (mitte ajada segi selle sajandi 
Teleportiga!). Meid oli kaheksa tudengit, kellest kuus õppisid välismaal, sest 
neil oli raha. Eesti tudengitel raha ei olnud. Kaheksakesi panime rahad 
kokku, ostsime Soomest portsu modemeid ja tegime sissehelistamiskeskuse, kus 
sai ilma lepinguta 900-numbri\sidenote{Telefoninumbrid 
algusega 900, millele helistamisel kehtis eritariif. Tariifi 
jagati teenusepakkujaga ja see võimaldas tasulisi teenuseid osutada.} kaudu helistada. Saime 
tänu 900-teenusele kohe oma raha kätte. Kommertsiaalse interneti pakkumine oli sel 
ajal vaat et olematu ja laiadele massidele mõeldes täiesti 
puudulik. 

\question{Mis aastal Uninet\index{Uninet} meile tuli?}

Uninet oli juba olemas, aga selleks tuli leping sõlmida. 
EsData\index{EsData} oli ka olemas, me istusime tegelikult nende võrgu peal. 
Kuu hiljem tuli Microlink Online\index{Micolink Online} ja sõi meid massiga 
ära. Teleportist sai mõnesid partnereid kaasates 
Meediamaa\index{Meediamaa} ehk www.ee\index{www.ee}. See oli Eesti 
esimene veebiäri, kus proovisime inimestele rääkida, et kui sind pole 
internetis, pole sind olemas, ja et tulevikus pole sul oma kaubaauto peal muud vaja kui URLi. Nad vaatasid meid nagu idioote, aga nüüd ainult URLiga 
kaubaautosid näebki. 

\question{Miks teie kui programmeerijad firma tegite?}

Pigem olime ikka tudengid. Tarvi selgitas, et niisugust teenust turul ei ole, ja see 
tundus väga lahe, et inimesed saavad juurdepääsu internetile. 

\question{Kas see oli siis puhas missiooniüritus?}

Eks mõttes lootsime raha ka teenida, sest see tundus olematu bisness, 
kus on võimalik kanda kinnitada. Veebiga oli sama lugu. Samas oli see 
paljuski ka missiooni ja eestvedamise asi. Kirjutasin 1996. aastal internetist ka raamatu, mis oli esimene eestikeelne 
selleteemaline originaalteos\sidenote{Tarvi Martens, Vello Hanson. Internet. Ilo, 
1996.}. See oli interneti propageerimine. Samal ajal 
ehitasin riigile andmesidevõrkusid ja TCP/IP 
tehnoloogia laialdane levik tundus mulle sellel kümnendil väga tähtis.

\question{Miks?}

Saavutamaks seda olukorda, kus me täna oleme. 

\question{Kas sul oli peas olemas teadmine, et selline olukord peab ja hakkab olema ning see on hea?}

Ma teadsin, et see on hea. Ma ei teadnud, kui kiiresti ja kui massiliselt see levib, aga 
hüved olid ilmselged. 

\question{Kas su juttu keegi kuulas ka?}

Arvan, et jah. Me oleme näinud, et igasuguse uue tehnoloogia evitamine 
võtab palju aega. Siis on täitsa loomulik, et räägime kahekümne viie aasta tagusest ajast, mille järelmeid võib näha täna. Samamoodi
ei ole ID-kaardi ja e-hääletamise tulemused tulnud 
päeva, kuu või aastaga. Rääkisin kord 
ühele psühholoogile, mida ma teen, ja ta ütles: \enquote{Tarvi, sa oled 
hull. Need asjad, mida sa teed, on inimeste käitumise muutmine. Ühiskondliku 
käitumise muutumine võtab minimaalselt seitse kuni kaheksa aastat aega. Sa ei 
saa oma tibusid lugeda enne, kui jääd vanaks.}

\question{Vähe sellest, tagantjärele on too algne impulss sisuliselt tuvastamatu 
ja seega keegi aitäh ei ütle.}

Ma ei igatsegi seda, see on väga okei. Lihtsalt vaatan 
ringi ja naeratan. 

\question{Sa mainisid, et tegid riigile 
andmesideühendusi.}

Ojaa, see on üks tore lugu. Tegime sel ajal
riigiga palju koostööd standardite ja andmekogude 
vallas, näiteks disainisime Andmekaitse Inspektsioonile\index{Andmekaitse Inspektsioon}. 
Usun, et oli aasta 1993, kui Eesti toll\index{Tolliamet} ja piirivalve\index{Piirivalveamet} tulid Küberneetika Instituuti\index{Küberneetika Instituut} ja ütlesid, et 
oleks vaja piirivalve ja toll üles ehitada. Neil on 
ühised piiripunktid, kus pole mingit sidet, mõnikord isegi mitte 
telefonisidet, ja kas Küberneetika Instituut saaks aidata. 
Joonistasin projekti, klient tuli paari kuu pärast tagasi ja ütles, et mitte keegi ei 
suuda seda projekti ellu viia ja tehke see ise ära. 
Pidimegi hakkama paberimäärimisest tegudele üle minema. 
Koostöös Eesti Telefoniga\index{Eesti Telefon} 
said esimesed ühendused tehtud ja siis hakkas see tegevus mullina 
paisuma. Järgmisena tuli politsei ja riburada teised järel. Me 
tegutsesime Küberneetika Instituudi katuse all, mis oli väga hea ja
amorfne asutus: tahtsid, tegid teadust; tahtsid, tegid äri.

Raha hakkas liikuma, pidime ruutereid 
ostma (kulud jagasime tellijaga – näiteks ostsime piiripunkti ruuteri ja tegime piirivalvega kulud pooleks) ja seega oli vaja moodustada mingi juriidiline keha. Tegime midagi niisugust, mida 
ei tohtinud tegelikult seaduse järgi teha, põhimõtteliselt MTÜ riigiasutustest. 
See MTÜ oli Andmeside Osakond\index{ASO}\index{Andmeside 
Osakond|see{ASO}}, mida juhtis nõukogu, kus oli iga riigiasutuse esindaja.
Raamatupidamistoimkond nurises iga aasta, et sellist asja ei tohi teha, aga 
ülemused ja ministrid ütlesid, et ärme lõhu 
toimivat asja.

\question{See eeldas, et keegi riigi poolel kuulas sind ja 
mõtles kaasa. Kas need olid tippjuhid või IT-juhid?}

Kõigepealt kuulasid IT-juhid, kes rääkisid oma tippjuhtidele. 
Mäletan selgelt, kuidas 31. detsembril istusid toonase 
piirivalve\index{Piirivalveamet} ülema Kõutsi\index[ppl]{Kõuts, Tarmo} 
kabinetis kõik asjaosalised – politsei, piirivalve, toll ja Küberneetika 
Instituut – laua ümber ja kirjutasid lepingule alla. Kõuts veel ütles: \enquote{Ma saan aru, et meil on siin juhikandidaat ka laua taga.} Ma olin siis alles kahekümne viie aastane naga. 

Edasi läks väga huvitavaks, sest meil oli tegelikult olemas selline asutus nagu 
Valitsusside\index{Valitsusside}, kes tegeles erivõrkudega.

\question{Kas nad su peale kurjaks ei saanud?}

Teatav konflikt tekkis jah erinevatel põhjustel, sealhulgas 
koolkondade vastasseis – Jaaksood \emph{versus} Lippmaad. Vanemad inimesed 
teavad seda väga hästi.

Aga juhtus jah, et piirivalvel oli kagupiir täiesti lage, 
seal polnud mingit sidet. Ja selle asemel et minna 
Valitsussidesse, kes pidanuks seda tegema, tulid nad minu juurde ja ütlesid, 
et näed, Tarvi, siin on kümme miljonit\sidenote{Tegu on Eesti 
kroonidega. Arvestades valuutakursse ja  inflatsiooni, on tänases kontekstis 
tegu umbes 1,2 miljoni euroga. Arvutades protsenti riigieelarve (mis oli tänasega võrreldes väga pisike) 
kuludest, maksis too projekt tänases mõistes suurusjärgus 21 miljonit eurot.}, meil on seal lage 
plats, kaheksa piiripunkti on vaja ühendada, tee midagi. Ma ütlesin, et jaa, väga huvitav. Aasta oli 1994 või 1995.

\question{See oli tol ajal suur raha. Kõike oli ju vaja ehitada, kust tekkis 
idee see raha just sidele kulutada?}

Kui oled keset tühja platsi, kus ei ole 
mobiililevi ega mitte midagi, kuidas sa seda piiri pead? Jutt käib elementaarsest telefonisidest ja sõnumivahetusest, mitte suvalisest veebibrausimisest.

Minust sai projektijuht ja me ehitasime tühjale kohale 2,4gigase raadioside
kaheksa mastiga, taldrikud otsa.

\question{See ei ole raadioside disaini mõttes triviaalne ülesanne – kas
õppisid seda kuskilt raamatust?}

Mõtlesin kasutada 
kõrget sagedust ja seega pidi olema otsenähtavus. Aga kuidas seda 
kindlaks teha? Lõuna-Eesti maastik, mäed ja orud. Leidsin 
Maa-ametist\index{Maa-amet} ühe tuttava, kes oli hakanud 
Vene ohvitserikaarte (kõige täpsemaid, mis tollal oli) digiteerima ja oli 
selle kõige huvitavama osa ehk Võrumaa sisse saanud. Ta suutis mulle 
väljastada profiili: andsin talle otspunktid ja tema mulle arvujada. Kirjutasin ise programmi, keerasin maa kumeraks, panin mastid kasvama ja 
vaatasin, kas on otsenähtavus. Selle järgi sai mastide kõrguse 
rehkendada ning mida kõrgem mast, seda kallim oli. 
EMT\index{EMT} ei 
vaadanud mingit profiili, pani 80 meetrit igale poole. Mul aga oli 
52 ja 54 meetrit, mille puhul pidi 
lennutuled lisama ja jälle oli kallim. Sattusin ühe teadjamehe peale 
Eesti Telefonist\index{Eesti Telefon}, kes vaatas tehtut ja ütles: 
\enquote{Kuule, mees, kas sa tegid nädalaga sellise asja? Trassi projekteerimiseks 
läheb poolteist aastat, tuleb jala kõik läbi käia, puud ära kaardistada!} Aga 
mul olid juba mastid tellitud. Ta rääkis, et on olemas Fresneli tsoon – 
saatja ja vastuvõtja vahele ei teki mitte kiir, vaid vorsti moodi 
asi\sidenote{Fresneli tsoon on ellipsoidne tsoon, mida pidi raadiolained 
saatjast vastuvõtjani levivad. Tsooni võivad sattuda ja seega sidet segada 
ka otsenähtavusest väljapoole jäävad objektid.}. See võttis natuke jahedaks 
küll, kuid mastid olid tellitud ja side läks käima. Järgmisel aastal tegin Peipsi 
äärde sama viguri. 

\question{Ühesõnaga sa ei teadnud, et nii ei saa teha?}

Ei teadnud, mõtlesin inseneri mõistusega, kuidas see käib. 

\question{Miks sa üldse kulude optimeerimisega vaeva nägid, kui nii palju raha anti kätte?}

Vabariigi algusajal ei olnud raha palju. Igas vallas pidi olema optimaalne ja tegema parimat, mis teha 
annab. See ei olnud teab mis üleliia suur raha, kulus kõik ära. 

See oli väga tore aeg, kui sai tõesti käegakatsutavalt riigi arengut 
toetada, pealegi minu lemmiktehnoloogia ehk 
interneti osas. 

\question{Kui ma sind kuulan, siis sa olid programmeerija, kuni saabus internet 
ja leidsid, et tuleb hoopis sinna panustada, sest maailm läheb sellest 
paremaks.}

Jah. Programmeerida oskas sel ajal juba üha rohkem inimesi, ma ei olnud enam 
unikaalne ja kaua sa ikka programmeerid.

\question{Mõni programmeerib eluaeg.}

Arusaadav, aga kõrgemad ja üllamad mõtted tundusid 
järjest paremad. Võibolla see on ka isiksuse arenguga seotud. Ausalt öeldes, 
kui olin programmeerija, siis kartsin telefonihelinat, sest ma ei 
tahtnud inimestega suhelda. Ühel hetkel läks see üle. Linna peal teadsid kõik, et kui Martens tuleb jaurama, siis 
proovib kindlasti Küberneetikasse tööle meelitada. 

\question{Kas sa olid Küberneetika Instituudis\index{Küberneetika Instituut} juhtkonnas, et käisid teisi tööle meelitamas?}

Olin ASO\index{ASO} pealik, see sai üle antud 
Informaatikakeskusele\index{Informaatikakeskus}, mis oli RIA\index{Riigi Infosüsteemi Amet} eelkäija.\sidenote{Eesti Informaatikakeskus koos 
Riigihangete Keskusega liideti aastal 2003 Riigi Infosüsteemi Arenduskeskuseks, 
millest 2011. aastal sai Riigi Infosüsteemi Amet ehk RIA.}. 

Aastal 1997 toimus reformatsioon: instituudid kui eraldiseisvad institutsioonid 
kaotati ja pidid liituma ülikoolidega. Küberneetika Instituut jagunes kolmeks: kõige väiksem osa ehk 
andmesideosakond läks informaatikakeskusele, teisest osast sai aktsiaselts ja kolmas liikus
Tallinna Tehnikaülikooli alla. Kuna Küberneetika Instituudis oli 
praktilist tegevust hästi palju, siis kõigest praktilisest moodustati 
Küberneetika Aktsiaselts\index{Küberneetika AS}, mis on siiamaani alles. See 
asutati riigiettevõttena ja nüüd on vist erastatud. 

Küberneetika AS oli väga 
huvitav kombinatsioon. Oli osakond, kus programmeeriti Tolliameti\index{Tolliamet} 
infosüsteeme. Minu osakond oli keskendunud infoturbele nii teoorias, 
praktikas, konsultatsioonides kui ka analüüsides. Ja seal kõrval oli 
meremärgindus ja -navigatsioon ning valgusfooride tegemine. Lisaks
kinnisvarahaldus, aga seda enam pole. 
 
\question{Sinu jutu sisse sigineb tasapisi juhiroll. Mõned inimesed saavad selle maigu suhu ja siis ainult sellega 
tegelevadki. Kas sul ei olnud nii?}

Pidin jõuga maigu suhu saama, sest tegevust oli vaja laiendada ja 
töö tahtis tegemist. Inimesi oli vaja, neid tuli meelitada. 
Küberneetika ASi\index{Küberneetika AS} moodustamisel sai minust selle
arendusdirektor. 

Mõeldi küll, et vaatan laiemat asja ning tegelen ka meremärkide 
ja poidega, aga selle õnge ma ei läinud. Hakkasin arendama infoturbetooteid. 1996. aastal tegime esimese tulemüüri valmis, siis 
VPNi toote ja SSLi \emph{proxy}'sid. 

\question{Kas see oli pärast Meediamaad?}

Jah, see oli hiljem. Infoturbetoodete arendamine läks esialgu väga
hästi. Tegime Linuxi peale veebipõhise liidese 
jubinatele, millest osav insener saab ise tulemüüri teha. Tegime selle veebiliidese kaudu lihtsamaks ja oligi jämedas plaanis 
toode valmis. Eesmärk oli teha keskmisest viis korda odavam toode – keskmine 
tulemüür maksis tollal kolm tuhat dollarit. Ja tuli välja. 

Ilmselt siin oli seos, sest just 
riigiasutused ostsid meeleldi meie tehtud tooteid. \enquote{Tarvi 
tegi võrgud, nüüd müüb neile turva ka peale.}

\question{Enamasti tekib riikides soov teha omale privaatne turvaline
internet. Kas Eestis seda ei mõeldud või üritati teha ja ei tulnud välja?}

Loomulikult üritati, tegime 
VPNi toote, mis oli võrreldes praegustega unikaalne. Kui kast 
oli võrgul ees, siis ei saanud internetti, see lasi ainult teise omasuguse juurde. 
Näiteks igas maakonnas on kontorid, kus paned rohelise kasti võrgule ette ja kamba peale on üks tulemüür ka, näiteks Tallinnas, ja ainult läbi selle tulemüüri saab 
välja. Muidu on täielikult sisevõrk. 

\question{Sa kirjeldad ju X-teed. Arhitektuuri mõttes tundub 
väga sarnane.}

Ei ole, sellel pole andmete semantikaga mingit pistmist. 

\question{Kas see tähendab, et projektide vahel ei toimunud mingit risttolmlemist?}

Ei, see oli privaattorude ehitamine, X-tee on OSI tasemetes 
natuke kõrgemal.

\question{Kas sa tol ajal tegelesid interneti propageerimisega paralleelselt 
edasi või oli see lihtsalt üks faas?}

Siis oli turul juba piisavalt tegijaid ja ma ei tundnud vajadust 
sellega tegeleda. Pigem oli minu jaoks saabunud järgmine faas teha 
internet turvaliseks. Kolmas elementaarne faas 
oli osapooled internetis identifitseerida, et saaks ka
\emph{business}'it teha. 

\question{Kust tuli mõte, et internet peab turvaline olema?}

Hakkasime teoreetiliselt turvalisusega tegelema juba 
1992. aastal. Kontseptsioon, kuidas ja miks seda 
teha, oli mulle tuttav. Meie roheliste kastide puhul oligi 
eesmärk puhas ja turvaline andmeside, muud midagi. Minu sõnum oli see, 
et ärme teeme eraldi X.25 võrku, sest üle avaliku interneti toimetades on palju 
kuluefektiivsem.

\question{Kuidas sul ikkagi tekkis mõte, et interneti turvalisus on 
probleem, mida tuleb hakata lahendama? Kas keegi luuras või häkkerid kiusasid? Kust 
probleem tekkis?}

Probleem on olnud aegade algusest. Ja olles infoturbega algusest 
peale tegelenud, oli selge, et võrkudes on infoturve teemaks. See on 
elementaarne. 

\question{Kui mina oma ajaloo peale mõtlen, siis minu jaoks ei olnud. Ehitasin pikalt oma asju ja võrke, üldse mõtlemata, et need võiksid ka turvalised 
olla.}

Infoturve oli minu eriala, ükskõik mis 
ametis, ja see sai alguse tolle 
ühe raamatu kooslugemisest.

\question{Lisaks on sul matemaatiku, programmeerija ja antenniehitaja 
taust, nii et saad päris süvitsi minna.}

Jah, ma olen kirjutanud Jukule\index{Arvutid!Juku} püsimälu. 
Minu töö puudutas tähtede joonistamist ekraanile, EEPROMi tasemel 
sai ESC-käskudega aknaid teha. 

\bigskip
\noindent\rule{.3\textwidth}{.7pt}
\bigskip

Mõtlesin, mis lugusid veel võiks rääkida, ja mõned tulid meelde.

Ma ei olnud Tallinna poiss ja Jaak Loondet\index[ppl]{Loonde, Jaak}, keda 
mitmed varasemad rääkijad on maininud, ei tundnud. Küll aga kuulsin temast 
Fidoneti inimestelt. 

Juhtus niisugune lugu, et varajastel üheksakümnendatel, kui 
Eestis ei olnud isegi piisavalt leiba, oli talongide peal\sidenote{1980ndate
lõpust kuni umbes 1993. aastani, kui vaba turg hakkas enam-vähem 
toimima, müüdi elementaarseid toidu- ja tööstuskaupu, sealhulgas 
periooditi leiba, üksnes talongide esitamisel.}, otsustas 
Soome Rotary klubi Eesti koolidele natuke arvuteid kinkida. Ilmselt oli PC-aeg peale tulnud ja ühel tehaseinimesel jäi komptuureid üle. 
Need olid kummalised masinad, aga lahe oli see, et need olid võrgus ja emaarvuti ka. Soome Rotary tegi haridusministeeriumile
ettepaneku kinkida need Eesti koolidele. 
Minu mentor Peeter Lorents\index[ppl]{Lorents, Peeter} oli sel ajal 
ministeeriumis mingi tegelinski ja sattus selle peale. 
Läksimegi kolmekesi – autojuht, Peeter ja mina eksperdina – 
kohapeale vaatama, mis arvutid need on ja kuidas töötavad. Tõime need Eestisse ja siis tekkis küsimus, mida me 
nendega peale hakkame. 

\question{Kui palju neid masinaid oli?}

Kuus-seitse tükki, terve klassitäis. Eesti peale ei olnud palju, aga 
Rotary klubi sai endale linnukese kirja: Eestit aidatud, heategevus tehtud. Ja 
siis meenuski mulle Jaak Loonde\index[ppl]{Loonde, Jaak}. Sain temaga kokku ja Jaak 
oli kohe nõus sellega tegelema, silmad peas põlemas nagu ikka. Mõne aasta pärast saime kokku 
ja küsisin, kas masinatel pruukimist ka oli, ja tuli välja, et need olid väga 
hästi vastu võetud ja nendega igasuguseid vigureid tehtud. 

\question{Nii et Jaak toimetas edasi ka pärast seda, kui 
enamik temast rääkinuid olid koolipoisieast välja kasvanud?}

Jaa, ta oli legendaarne, toimetas arvutitega elu lõpuni. Tema põhiline soov oli, et lapsed saaksid näpud arvuti külge.


\bigskip
\noindent\rule{.3\textwidth}{.7pt}
\bigskip

1993. aastal tegin ma esimese 
jututoa, mille nimi oli Anna\index{Jutukad!Anna}. See oli umbes samasugune asi nagu praegu Messenger: hulk inimesi logib sisse ja hakkab omavahel suhtlema. 

\question{Kas see käis sinu enda tehtud tarkvara peal või said selle kuskilt?}

Sain kuskilt tarkvara ja tõlkisin käsud eesti keelde, käsk algas 
punktiga. Olin tollal Göteborgis neli kuud asumisel ja mul polnud 
seal suurt midagi teha, nii et putitasingi seda jututuba. 

Anna jututoas kaitsti isegi üks Tallinna 
Tehnikaülikooli\index{Tallinna Tehnikaülikool} diplomitöö ära – kaitsja asus 
Uus-Meremaal, õppejõud kogunesid jututuppa.

\question{Mis oli jutukate fenomen? Seal käis igasugust rahvast, mitte ainult tehnikud.}

See oli \emph{community building}, umbes samasugune grupp nagu Fidonet. Edasi 
tekkisid OK \index{Jutukad!OK} ja 
Cafe\index{Jutukad!Cafe}\sidenote{Cafe pärisnimi oli \emph{The Roadkill 
Cafe} ja see asus aadressil \texttt{ns.uninet.ee:5555}. Selle pani 23. 
veebruaril 1996 NUTSi (\emph{Neil's Unix Talk Server}) versiooni 2.3 
lähtekoodist püsti Indrek Siitan\index[ppl]{Siitan, Indrek}.} jutukad. Meil oli 
isegi Anna kasutajate kokkutulek Viljandi lähistel, mida 
Jüri Ruut\index[ppl]{Ruut, Jüri} veab siiamaani, nüüd küll ee.kevade nime all.

Jutukates käis suvaline rahvas, seal ei olnud õnneks üksnes tehnofriigid, vaid ka tütarlapsi. 

\question{See pidi siis olema väga vajalik teenus, sest 
mittetehnofriigile pidi see tehnika olema paras barjäär.}

See oli tegelikult lihtne, kui ainult terminalile ligi said. Panid 
\verb|telnet anna.ioc.ee|\index{Masinad!anna.ioc.ee} ja läks. 

\question{Kas sa hoidsid jutukat Küberneetika Instituudis\index{Küberneetika 
Instituut}?}

Pean tunnistama, et jah. Alustasime Küberis Unixi pruukimist aastal 
1992, kui tõime Soomest flopidega Linuxi\index{OS!Linux}. Teistmoodi ei 
saanud seda kätte. 

\question{Kas otse Linuse käest?}

Enam-vähem. Proovisin tollal Unixi kultuuri aretada. Kord ostsime hirmsa
raha eest ühe Suni. Kui küsiti, mis sellele nimeks panna, siis ütlesin suvaliselt 
\enquote{keeks} ja tekkiski igavesti kuulus FTP-server keeks.ioc.ee\index{Masinad!keeks.ioc.ee}. Pärast pidin \enquote{keeksi} lahti mõtestama ja 
arvasin, et see on Küberi Esimene Eestimeelsete Kasutajate Server.

\question{Tuleme korraks jutukate juurde tagasi. Selleks et sotsiaalvõrk 
lendu läheks, peaks olema algne seltskond. Kes need inimesed olid ja 
kuidas sa selle võrgustiku tekitasid?}

Ma täpselt ei mäleta, aga küllap rääkisin sõpradele, nemad oma 
sõpradele ja nii see vaikselt levis. Ühtegi erilist 
aktsiooni ei mäleta, piisas sõprade ringist, aga lõpuks läks 
ring väga laiaks – üle poole või rohkemgi olid 
täiesti tundmatud inimesed. 

Annaga\index{Jutukad!Anna} juhtus nii, et ühel hetkel vaatasin, et 
teised jutukad hakkavad ka tekkima, ning panin selle pidulikult kinni. 
Anna matused olid eraldi sündmus. Asja peab ära lõpetama, mitte laskma 
sel lihtsalt hääbuda. 

Kui Unixi juurde tagasi tulla, siis oli meil 
Eesti Unixi Pruukijate Selts ehk EUPS\sidenote{Selts asutati 1994. aastal ja sellel oli 62 
asutajaliiget. Asutavasse toimkonda kuulusid lisaks Tarvile Andres 
Bauman\index[ppl]{Bauman, Andres}, Margus Liiv\index[ppl]{Liiv, Margus}, Jaanus 
Pöial\index[ppl]{Pöial, Jaanus} ja Anto Veldre\index[ppl]{Veldre, Anto}.}. Teised tahtsid panna \enquote{Kasutajate Selts}, aga EUKS kõlab 
halvasti ja mina ütlesin, et peab ikka pruukima. Meil oli 
Tõraveres isegi kokkutulek.

\question{Miks te Soomest Linuxi\index{OS!Linux} tõite? Kas te ei tahtnud Sunile 
raha anda?}

Ühelt poolt ei tahtnud raha anda ja teiselt poolt oli see uus värske 
tuul, mis oli vaja ära proovida. Linuxi eelis oli see, et see käis 
PC peal. 

\question{Linux on praeguseni hädas oma kõrge sisenemisbarjääriga, inimestel on 
raske sellega liikuma saada. Kuidas toona oli?}

Me rääkisime Linuxist serveri kontekstis, tööjaama-Linux ei olnud teema. 
Tol hetkel pidi raha eest ostma mingi tarkvara, et failiserverit ringi 
ajada. Ma ütlesin, et ärme tee seda! Panen Linuxi püsti, kasutame 
seda. 

\question{Tol ajal taheti igasuguste asjade eest, nagu 
veebiserver, raha saada ja kommertstarkvara oli väga kallis.}

See oli ropult kallis, kuna kirjutajaid oli vähe ja see oli eksklusiivne asi. Kui hakkasime Küberis\index{Küberneetika 
Instituut} 1996. aastal tegema esimesi tulemüüre nimega 
Barrikaad\index{Barrikaad}, siis tol hetkel maksis keskmine tulemüür maailmas 
kolm tuhat dollarit. See on ju absurdne. Me võtsime Linuxi, tegime näo pähe ja 
müüsime viis korda odavamalt.

Seoses kogukondadega ei saa mainimata jätta 
sellist olulist \emph{community}'t nagu WC Fauna\index{WC Fauna}. 
Raske öelda, mis see täpselt oli või kes sinna kuulusid, see oli rohkem 
mõtte- ja eluviis. Selle liikmed tegid igasuguseid asju, pahatihti käisid 
lihtsalt kõrtsides või tegid niisama nalja ja ehitasid lumelinna.

Vanasti olid kompuutrimessid tähtsad\sidenote{Aastatel 1993–1999 
korraldati Eestis igakevadist arvuti-, side- ja bürootehnika messi 
\enquote{Kompuuter}. Tegu oli olulise kogukondliku ja 
müügiüritusega, mida Päevaleht tituleeris lausa infotehnoloogia laulupeoks.}. Ühel messil pakuti meile oma boksi ja pidime selle 
kuidagi sisustama. Boksis oli üks kompuuter, mis luges sekundeid tuleviku 
alguseni, ja WC Fauna leviala kaart, milleks oli punaste läbipaistvate 
vorstinahkadega kaetud Eesti kaart, politseilindiga ümber tõmmatud. 

\question{Tänapäeval läheks selline asi kunstiprojektina kirja.}

Jah, ilmselt küll. Eks see oli häppening, igasuguseid erinevaid asju sai tehtud. Näiteks oli 
IT-inimeste kokkutulek 
OK-fest\index{OK-fest}\sidenote{1994. aastast Eesti Infotehnoloogia- ja 
Telekommunikatsiooniettevõtjate Liidu\index{Eesti Infotehnoloogia- ja 
Telekommunikatsiooniettevõtjate Liit} korraldatud suvine kokkutulek.}, 
kus \emph{community} kokku sai. WC Fauna nimi sai alguse sellest, et ühel 
OK-festil oli vaja jalgpallimeeskond kokku panna. Mõtlesime, et FC Flora juba 
on, paneme siis WC Fauna. Aga see oli ka vist viimane kord, kui jalgpalli 
mängisime. 

\question{Sinu jutust kumab läbi palju 
ühistegevust, aga tavaliselt ei tegeleta arvutitega sellepärast, et 
meeldib teiste inimestega suhelda. Kuidas sul arvutite ja inimeste suhe 
kokku käib?}

Ma olengi imelik loom, kellest pole kunagi aru saadud. Üks tuttav 
psühholoog ütles: \enquote{On olemas insenerid ja on olemas kunstiteadlased, 
aga kumb sina oled, aru ei saa.} 

Inimene areneb vaikselt. Nagu ma mainisin, siis algusaegadel olin 
introvert, kes istus nurgas ja programmeeris ning kartis, kui telefon 
helises. Hiljem hakkasin inimestega suhtlema, seejärel ühiskonda nägema ja sealt tulid ka riigi- ja 
vaat et maailmalaiused asjad. 

\label{sisu:everyday}Üks lugu, milles maksab kindlasti rääkida, on see, kust Skype\index{Skype} tegelikult alguse sai ja kus see kamp 
kogunes. Ilmselt nii mõnigi mäletab, et umbes 1994. või 
1995. aastal oli lehes kuulutus \enquote{otsime programmeerijat, maksame 
viis tuhat krooni päevas}\sidenote{Teiste allikate alusel oli kuulutus lehes 
1999. aastal, mis on loogilisem – muidu jääb Skype'i asutamise ja 
Bluemooni Tele2-seikluse vahele liiga pikk paus.}. Viis tuhat krooni oli kaks kuupalka. 
Kuulutuse tagamaa oli see, et Tele2\index{Tele2}, kes oli juba Eestis olemas, ja 
Bonnier Media\index{Bonnier Media} sepitsesid Rootsis 
nii-öelda uue põlvkonna portaali
everyday.com\index{everyday.com}. Niipea kui nad uudise välja lasid, et 
niisugune portaal tuleb, tõusis nende turuväärtus poolteist miljardit. 
Absurdne, aga nii see oli. Eestisse tuldi jutuga, et meil on tiimid Itaalias, Rootsis ja Taanis ning kõik on juba tükk aega programmeerinud. Kahte 
programmeerijat Eestist on veel vaja, siis saab kõik korda\sidenote{Eestis 
töötas toona Tele2s Stefan Öberg\index[ppl]{Öberg, Stefan}, kes hiljem täitis Skype'is 
mitmeid juhtivaid rolle. Tema juhataski viimase kahe tegija otsijad 
Eestisse.}. 

Mina sattusin seda otsingut nõustama ja lõpuks projektijuhiks, kes 
pidi need inimesed välja valima ja asjad ära tegema. Valisin välja 
Bluemooni\index{Bluemoon} poisid. Sõitsin kõik need Itaalia, 
Taani ja Rootsi kontorid läbi ning sain aru, et peale Rootsi, kus oli tehtud väike 
andmebaasimootor, olid kõik teised tiimid tootnud täielikku kräppi. Nii ei 
jäänudki projekti päästmiseks muud üle, kui kogu värk ise teha. Bluemooni
poistel ei olnud probleem see käsile võtta ja nädala-paariga 
portaal kokku veeretada, kuigi nad PHPd\index{Keeled!PHP} ei tundnud.

Tulevane miljardär\sidenote{Tarvi peab silmas Niklas 
Zennströmi\index[ppl]{Zennström, Niklas}.} oli Tele2s projektijuht ja talle 
hakkasid need poisid meeldima. 

\question{Sina olid portaalis projektijuht. Kui tegid portaali valmis, kas siis 
ei tekkinud mõtet, et peaks suures Rootsi kontsernis kosmilist karjääri tegema?}

Absoluutselt mitte, see oli kõrvaltegevus – aitamisprojekt ja raha 
maksti ka.

\question{Mis su põhitegevus oli?}

Ehitasin riigivõrku ja juhatasin neid 
vägesid. Sinna kõrvale mahtus veel üks kõrvaltegevus, 
mail.ee\index{mail.ee}, mille omanikuks sai ka lõpuks Tele2. 

\question{Kas mail.ee all oli standardne SMTP-server?}

Täpselt nii. Alustuseks oli ilma näota 
meilboks. See tähendas, et igaüks sai endale aadressi luua, aga pidi enda 
meilerit kasutama. Teine arengufaas oli sellele veebi nägu pähe teha, seal oli veebimeiler ka. See sai täitsa ise kirjutatud, all 
oli loomulikult standardne kompott. 

\question{Nii et sa ei läinud ise sinna maailma midagi leiutama, vaid võtsid 
tükid ja ladusid kokku?}

Jaa, see on mul kogu aeg veres olnud. Ühel hetkel sain aru, et 
programmeerimine on üldse kurjast, sest kõik on juba ära tehtud. 
Tegelikult on kunst tükid üles leida ja oskuslikult kokku panna. 
Tänapäeval on tükkide arv muutunud hoomamatuks ja väga raske on neist midagi kokku panna. Ilmselt on 
tekkinud kildkonnad ja voolud. Kunst on muutunud.

\question{Mis üldse on tänapäeval sinu jaoks programmeerimine?}

See kipub olema järjest igavam asi, sest vanasti oli 
see selgelt loometöö. Nii kui hakkasid tulema igasugused 
mudelid ja RUPid\sidenote{\emph{Rational Unified Process (RUP)}. RUP oli 
1990ndatel suurorganisatsioonides levinud tarkvaraarenduse raamistik, 
mis keskendus arendusprotsessi keerukuse vähendamisele läbi standardiseeritud 
rutiinide. Et samal ajal üritati keerukat tarkvara tarnida harva ja suure 
pauguga, võis RUP küll teha projektid paremini kontrollitavaks, kuid ei vähendanud kuigivõrd arendajate frustratsiooni.}, siis hakkas see
järjest rohkem tunduma kraavikaevamisena. Arhitektid joonistavad asja ette ja sina lihtsalt täidad 
funktsiooni. See ei ole eriti keeruline. 

\question{Ometigi ehitatakse igasuguseid hullusi, nagu tekstiterminalis 
video mahamängimine.}

Loomulikult, nalja pärast saab ikka teha. Ma räägin raha 
eest või tööstuslikust programmeerimisest, kus tuleb konkreetset asja teha. 
Vanasti olid mees nagu orkester ja mõtlesid ise välja, kuidas arhitektuur 
võiks välja näha. Tegid oma äranägemise järgi ja keegi ei kobisenud. Nüüd 
on arhitektid. Loovust on 
programmeerijatele jäänud kindlasti vähemaks. 

\question{Kui me juba selle teema juurde jõudsime, siis küsin ka sinu käest, 
milline on ilus kood?}

Ilus kood on loetav kood, siin ei ole kahtepidi mõtlemist. 

\bigskip
\noindent\rule{.3\textwidth}{.7pt}
\bigskip

\question{Kuidas sündis ID-kaart?}

Küberis\index{Küber} tegutsesin ma kahel rindel. Ühelt poolt ehitasin võrke, 
aga olin ka kogu aeg infoturbe ja krüptograafia keskel. Lisaks 
võrguturbele, mis oli sel ajal väga oluline, tundus avaliku võtme 
krüptograafia huvitav ala ja pakkus oma rakenduste poolel pinget. 
Küberis sai jälgitud, kuidas 1995. aastal vist Rootsi Post alustas oma 
ID-kaardi väljalaskmisega ja avaldas ID-kaardi profiili. Päris vara, 
üheksakümnendatel, toodi mulle Ektacosse\index{Ektaco} Schlumbergeri 
kiipkaardid ja paluti vaadata, mis elukad need on. 
Kirjutasin sinna peale programmi nimega \emph{Clevercard}.

\question{Kas see oli Java kaart?}

Javat polnud veel väljagi mõeldud, 
krüptokaarte ka mitte. Mälukaart see ei olnud, protsessor 
oli sees. Sellele kiipkaardile sai käske anda, näiteks „tee fail“. Kõige all oli 
kaardi operatsioonisüsteem. Baidid ajasid sisse, baidid tulid vastu ja ma kirjutasin 
PC-le programmi, millega seda sai mõnusalt teha. 

Aeg läks vaikselt edasi ja see oli umbes 1996.
aastal, kui tegin Äripäeva lahti ja esimesel leheküljel oli pildil Kaja 
Kuivjõgi\index[ppl]{Kuivjõgi, Kaja}, keda ma tundsin ja kes 
oli siis Kodakondsus- ja Migratsiooniameti\index{Kodakondsus- ja 
Migratsiooniamet} asedirektor. Pildi juures oli kirjas, et riik 
planeerib uut dokumenti ja et esimesed passid, mis võeti kasutusele 1992. aastal, saavad 2002. aastal läbi. 
Sinna on viis aastat aega ja KMAs on moodustatud töörühm, kes 
uurib variante millegi uuega välja tulla. 

Võtsin Kajaga ühendust ja ta 
näitas mulle töörühmas arutatud materjale. Kui olin need 
läbi vaadanud, sain aru, et nende tehniline teadmus on üsna allpool 
nulli. Seal räägiti kiibiga varustatud vöötkootidest. 

\question{Mida nad teha tahtsid? Uut ja paremat passi?}

Nad mõtlesid ikkagi kaardi suunas, aga milline see võiks olla – 
kas kiibiga varustatud vöötkood või mis – ei olnud selge. 

\question{Mina olen kogu aeg arvanud, et kaardi pakkusid
välja tehnikud, mitte ametnikud.}

Soov oli tol hetkel väga hägune ja igasugused 
variandid olid laual. Aga oli selge, et kuna tekib suurem 
passivahetus, siis on võimalik inimesi üllatada millegi uuega ning vaadata, mis 
maailmas tehnoloogia vallas toimub. 

Oli päris selge, et KMA\index{Kodakondsus- ja Migratsiooniamet} 
töörühmal ei ole mõtet jätkata. Tehti ettepanek moodustada 
laiem töörühm ja võtta laua taha ka eksperte: pangad, 
telekomid, riigisektori ja Küberi\index{Küber} inimesed.

\question{Kas tänapäeval tundub veider, et riik võtab pangad ja 
telekomid laua taha sellist dokumenti arutama?} 

Absoluutselt mitte. Ei tundu praegu ega tundunud ka tol ajal. 
Laiapõhjaline koostöö riigi- ja erasektori vahel on meile alati edu 
toonud nii ühes kui ka teises. 

\question{See on haruldane asi, mida mujal sageli ei näe.}

Eesti on nii väike riik, et põhimõtteliselt tead kõiki, kes midagi teavad, ja 
ei ole mõtet kedagi kõrvale jätta sellepärast, et ta on parasjagu erasektoris. Me räägime ikkagi eksperditeadmisest ja 
ekspertide kogumist, mitte institutsionaalsest asjast. 

Tuligi töörühm kokku ja arutas asju. Telliti kaks tööd, 
KMA\index{Kodakondsus- ja Migratsiooniamet} maksis. Ühe töö viis läbi 
aktsiaselts Aprote\index{Aprote}, kes uuris, milleks kõigeks 
võiks seda kaarti kasutada. Nad läksid näiteks tanklaketti ja küsisid, 
mida nemad tahaksid. Tulemus oli muidugi väga ulmeline, aga turuootuste uurimine oli 
vajalik tegevus, vaat et kohustuslik samm. 
Teine töö, mida tegime meie Küberis\index{Küber}, oli 
tehnoloogiline ülevaade, milleks kiipkaardid on suutelised, kaasa
arvatud see, mida on Rootsis ja Soomes tehtud. 
Millised on profiilid ja tehnoloogiad, sealhulgas Microsofti 
PC/SC\sidenote{\emph{Personal Computer/Smart Card} – spetsifikatsioon 
tarkade kaartide integratsiooniks arvutustehnikaga.}. 

1996. aastal joonistasin projektiplaani, et neljateist kuuga 
toome kaardi välja, kaasa arvatud pilootprojekt ja muu säärane. Võttis see siiski viis 
aastat, sest see oli väga oluline samm ühiskonnas 
ja vajas pikemat kaalumist. Peale selle tuli seadusi juurde ja ringi teha. 

\question{Kas kõike seda vedas KMA?}\index{Kodakondsus- ja Migratsiooniamet}

Ei, kindlasti mitte. Digiallkirja seadust näiteks vedas 
Majandusministeerium\index{Majandusministeerium}.

\question{Kuidas nii? Asi ju algas 
dokumendi väljastamise vajadusest ja siis äkki tahtis Majandusministeerium digiallkirja 
teha?}

Kindlasti oli suunanäitajaks Saksamaa, kes võttis esimesena vastu 
digiallkirja seaduse, mille pealt Eesti oma on paljuski maha viksitud. Meie 
digiallkirja seadus või vähemalt selle kavand nägi ilmavalgust enne, kui 
oli olemas Euroopa 1999. aasta direktiiv\sidenote{Euroopa Parlamendi ja nõukogu 
direktiiv 1999/93/EÜ.}. Seetõttu oli meie seadus mõnevõrra erinev. Euroopa 
direktiiv lubas igasuguseid lahjasid allkirju ja sellist koledust nagu 
näpuga ekraanile kirjutamist ning ei läinudki tööle. Seepärast tuli ka 
lõpuks eIDAS\sidenote{\emph{electronic IDentification, Authentication and trust 
Services - eIDAS} – Euroopa Parlamendi ja nõukogu määrus 910/2014 e-identimise ja e-tehingute kohta.}, et direktiiv 
oli väga lahja. Kehitati õlgu ja ei kasutatud, tehti lahjasid allkirju ja 
öeldi, et nüüd ongi kõik hästi. 

Meie seadus ütles algusest peale, et ainult 
kvalifitseeritud allkirjad\sidenote{Lihtsalt öeldes on kvalifitseeritud 
elektrooniline allkiri selline allkiri, mida võib pidada võrdväärseks 
omakäelise allkirjaga. Keerulisemalt on öeldud eelviidatud eIDASi direktiivis ja 
selle rakendusaktides.} on aktsepteeritud, ja mingeid lahjasid allkirju ei 
tunnistatud. Neid seadus ei käsitlenudki. 

\question{Kas seaduse väljatöötamises osalesid ka eksperdid või oli 
see Majandusministeeriumi tehtud?}

Eksperdid olid kaasatud. Oli töörühm, kus osalesid
inimesed krüptoloogist kuni juurateadlasteni. Nad tegid seda tööd ligikaudu
kaks aastat, nii et see ei tekkinud niisama, vaid mõeldi väga põhjalikult 
läbi. Kuskilt mujalt kui Saksamaalt ei olnud šnitti võtta. Nii mõnigi 
seadusepunkt oli inspireeritud nii-öelda krüptograafide mõtlemisest. 

\question{Sinu jutust ei kõla läbi kõikehõlmav õilis visioon 
sellest, kuidas ühel päeval sünnib Eesti digiühiskond ja kõik saab e-teenuste 
abil uueks loodud.}

Eks see võibolla kuskil ajusopis oli, aga mis sellest ikka rääkida, asju 
tuleb teha. 

\question{Seda ma peangi silmas, et liikumine toimus samm-sammult ja tegeldi 
konkreetsete asjadega.}

Jah, kasvõi seesama ID-kaardi väljatoomine. Võib ju digiallkirja seaduse vastu 
võtta (mis aastal 2000 ka vastu võeti), aga kui inimestel ei ole vahendit, 
millega digiallkirja anda, siis pole seadusel suuremat mõtet. 
Euroopas valitses ka selle direktiivi tegemise ajal nägemus, et 
kommertsfirmad hakkavad sertifikaati müüma ja seetõttu on vaja neid 
reguleerida. Kuidas see võiks käia? Teed turule putka ja hakkad sertifikaate 
müüma: suured ja väikesed sertifikaadid, punased ja kollased? 

See ettenägemisvõime oli meil küll, et niisugune visioon, et müüme inimestele 
sertifikaate, neid ostetakse ja kuidagi tekib kasutus, on üdini vale. Selles mõttes oli näiteks Soome, kes tegi ID-kaardi 
mittekohustuslikuks ja pani kohe hinnaks nelikümmend eurot. Siis juhtub see, et teenusepakkujad ei hakka ID-kaarti toetama, sest 
nad teavad, et inimestel ei ole seda (viiel protsendil võibolla on). Ja inimestel ei ole kaarti, sest teenuseid ei 
ole. Siis ongi nokk kinni, saba kinni ja mudel, mis ei toimi. 

Kõigepealt peab elektroonilise identiteedi 
taristu looma ja siis võibolla hakkavad asjad juhtuma. Samamoodi nagu ei 
saa proovida kuskil metsa sees müüa kilomeetrit maanteed kohalikule 
metsaelanikule. 

\question{Su jutust kõlab läbi üsna suur usaldus ekspertide vastu. Poliitika eest vastutava inimese ja krüptoloogi 
vahel pidi olema usalduslik vahekord, et viimasel lasti seadusesse punkte kirjutada.}

Skepsis on väga raske tekkima, kui laua taga on Eesti paremad pead – misasja sa ikka kahtled või kõhkled. Mida targemaks inimesed saavad ja 
mida rohkem eksperte on, seda rohkem tekib diskussiooni. 

\question{Kui suur see ekspertide ring oli, kes töörühmades käis ja seda 
ideed kujundas?}

Viis kuni kümme võtmeinimest.

\question{Ja kogu tarkus tugines tollele salapärasele raamatule, mis 
Küberis oli?}

Oo ei, see raamat oli lihtsalt algus. Me ei räägi ainult 
krüptoloogiast või infoturbest. Näiteks ID-kaart ei puuduta ainult 
infoturvet, vaid väga paljusid rakenduslikke ja isegi sotsiaalseid aspekte. Ei saa rääkida, et krüptograafia päästis maailma. 

\question{Sagedasti inimesed arvavad, et kui asjad saaks 
ära krüptida, siis olekski maailm päästetud.}

Ma jään selle juurde, et ei saa kilomeetrit maanteed müüa 
külaelanikule, sest ta küsib: \enquote{Mis ma teen selle maanteega?} – 
\enquote{Hakkad autoga sõitma.} – \enquote{Aga mis see auto on?} See on 
ufo müümine, täiesti mõttetu tegevus. Sa ehitad teed valmis, lased autod müüki, 
paned sõidukoolid püsti ja siis ühel päeval võibolla inimesed avastavad, et 
transpordist on kasu. Aga kui hakkad sellest pihta, et proovid 
igaühele juppi maanteed müüa, siis see ei toimi. 

\question{Küberneetika Instituut\index{Küberneetika 
Instituut} ja selle järelmid on väga pikalt Eestis olulist rolli mänginud. 
Sina oled olnud seal sees ja, mis veel olulisem, ka sellest väljas. Kas sa oskad 
öelda, mis maagiline asi selle asutuse nii võimsaks teeb?}

Nagu ma mainisin, siis 1997. aastal jagunes Küberneetika Instituut kolmeks. Osa 
läks ülikooli alla, osast moodustus aktsiaselts ja andmeside osa läks riigile. 
Võibolla kõige nähtavam osa IT-inimeste jaoks ongi Küberi instituudi või 
aktsiaseltsi ehk infoturbe ja programmeerimise osa. Seal tehakse meremärke 
ka, aga need on merel ja ei paista välja.

Mul on olnud au omal ajal umbes kolmkümmend inimest sinna tööle võtta ja 
Tartu labor\index{Cybernetica!Andmeturbelabor} asutada, mis on nüüd inimeste arvu mõttes nüüd isegi suurem, kui Tallinn. Need olid väga toredad ajad. Aga fenomen seisneb selles, et Küberit peetakse põhimõtteliselt ainukeseks firmaks 
Eestis, kes oskab turvaliselt programmeerida ja teab midagi infoturbest. Seetõttu 
on neile ka sattunud niisugused tegevused ja projektid alates X-teest ja 
lõpetades Smart-IDga\index{SplitKey}, kus turvalisuse ja krüptograafia komponent on omal 
kohal. 

Lisaks on seal tõsiseid inimesi, kes tegelevad puhtalt teadusega ja koodi ei 
kirjuta. Küberis on oma teadusosakond ja teadusdirektor. Ühtlasi käivad 
teadusega tegelevad inimesed Küberi ja ülikoolide vahet. Sellist sümbioosi 
otsitakse nagu spunki mööda Eestit taga ja ka Teaduste 
Akadeemia president ei väsi rääkimast, et Küber on fenomen ja suur erand. 

\question{Miks ei ole näiteks Helmes võtnud endale teadusdirektorit 
tööle ja hakanud sama tegema?}

Asi on selles, kas teed kõigepealt teadust ja siis hakkad seda 
rakendama ühiskonnas või proovid vastupidi teha: kõigepealt oled kõva
programmeerija ja siis mõtled, et teeks teadust ka kuidagi. Päris 
nii see ei käi, need juured on natuke sügavamal.

Juhtusin hiljuti nägema Küberi töökuulutust: otsitakse 
projektijuhti, nõutav CISA\sidenote{\emph{Certified Information Systems Auditor} – sertifitseeritud infosüsteemide audiitor.} sertifikaat. 
Halleluuja!

Omal ajal kõik teadsid, et Martens tuleb jälle jaurama ja Küberisse tööle 
meelitama. Mul oli väga lihtne äriidee: ajan kõige targemad 
inimesed ühte suurde ruumi ja annan neile teema kätte. Nad asuvad 
plaksti tööle, ise panen jalad seina peal. Töötas! Väga hästi töötas! Pärast 
lugesin kuskilt raamatust, et niimoodi tuleb käituda, ja täpselt nii ma 
olengi käitunud.

%--------------------------


\question{Kas sa oled siis asjade käimalükkaja ja visionäär?}

Kui sa nii ütled.

\question{Ma ei ütle, ma küsin. Sa ütlesid, et programmeerija sa enam ei ole. Kes 
sa niisugune oled?}

Ma ei oska ennast sildistada. Mul on see häda küljes, et mõtlen kogu aeg kuidagi 
laiemalt.

\question{Miks see häda on?}

Võiks ju midagi näpu vahel teha, kaltsuvaiba või midagi. On vähemalt füüsiline tükk taga, 
suurtest sõnadest ei jää midagi\ldots

\question{Mida sa praegu teed?}

Olen endiselt elektroonilise hääletuse juht, juba aastast 2003. 
Hiljuti olid meil kümnendad valimised, kohe on algamas üheteistkümnendad, 
Europarlamendi valimised\sidenote{Jutuajamine Tarviga leidis aset 2019. aasta mai algul, Europarlamendi valimised toimusid 26. mail, edukas 
elektrooniline hääletamine 16.–22. mail.}. Aga valimised võtavad võib-olla 
kaks-kolm kuud tähelepanu. Valimistevahelisel ajal ma palju 
suurt ei teegi. Jõudumööda, nii nagu kutsutakse, käin maailmas ringi ja proovin 
inimesi aidata nende arengus erinevates riikides – nii 
elektroonilise identiteedi teemal kui ka IKT rakendamise alal 
nii-öelda valimismajanduses.



\chapter{Peeter Marvet}
\index[ppl]{Marvet, Peeter}
\index[ppl]{Tehnokratt|see{Marvet, Peeter}}

\question{Kuidas ja millal sa jõudsid arvutite juurde?}

See oli umbes täpselt aastal 1985, pidin siis olema 15 aastat 
vana. Eelnevalt olin arvuteid näinud Soome televisioonist, 
seal reklaamiti ilmselt Commodore 64 ja Spectrumi masinaid. 

\question{Sa oled järelikult Tallinna poiss?}

Jah. Ma olen sündinud Tartus, aga pikalt Tallinnas elanud. 

Kui ajas veel tagasi krutin, siis üks kokkupuude arvutitega oli 
veidi varem, papsi laboris. Ta töötas TPI Veekvaliteedi 
Laboris\index{Tallinna Tehnikaülikool!Veekvaliteedi labor}, mis asus selle koha peal, 
kus keset Järvevana teed on praegu Maru Ehituse maja. Seal oli olemas üks 
terminal, mis käis Datasaabi\index{Datasaab}-nimelise 
arvuti\sidenote{Datasaab oli Rootsi lennukitootja Saab arvutustehnoloogia 
eraldi ettevõtteks kasvanud divisjon, kus toodeti nii tsiviil- kui ka
militaarkasutuseks mõeldud arvuteid.} külge, mis asus kusagil Mustamäe teel. 
See masin oli ostetud ühelt rahvamajandussaavutuste näituselt, kus 
vahetevahel käisid ka välismaalased kohal. Kuskilt oli saadud valuutat ja ostetud selle
eest välismaa arvuti, mille külge käisid modemitega oranžid terminalid. 

Datasaab osteti väidetavasti ilma 
operatsioonisüsteemi ja igasuguse rakendustarkvarata, sest rohkem 
rutsi sellel ajal ei olnud. Aga Nõukogude insenerid olid vinged, kirjutasid sinna ise operatsioonisüsteemi peale. Sellest tarkvarast õnnestus veel mingisugune jupp Datasaabile 
tagasi müüa ja saada ilmselt vastu mälu või 
lisakomponente. 

Datasaabi terminalil õnnestus 
lihtsalt, ilma ühenduseta \emph{backspace}'i ja tühikuga 
\enquote{ronge kokku haakida}. See on minu esimene mälestus arvutiga suhestumisest. 

Järgmine mälestus on samuti aastast 1985, kui olin 
ilmselt seitsmendas klassis. Toonases Pedas toimus Tallinna koolide 
füüsikavõistlus ja meid viidi ka arvutisaali, kus oli Minsk\index{Minsk}. Seal sain tuttavaks ühe aasta vanema
koolivennaga, 
kellel oli kaasas isiklik perfolint programmiga. See ei olnud
keegi muu kui Sulo Kallas\index[ppl]{Kallas, Sulo} ja tema perfolindi peal oli 
üks mängulaadne asi, mis arendas mingisuguseid 
organisme\sidenote[][-2cm]{Tõenäoliselt oli lindil Briti 
matemaatiku John Horton Conway välja mõeldud rakuautomaat, mida tuntakse 
nime all Game of Life. Tegu on mängijateta mänguga, mis ainsa sisendina 
vajab algseisu määratlemist. Automaat on ühest küljest levinud 
programmeerimisülesannne ja teisalt põnev uurimisobjekt, seetõttu võis selle 
realiseerimine olla noorele arvutihuvilisele nii huvitav kui ka jõukohane.}. 
Sulo vend oli Raadiomaja 
Arvutuskeskuses\index{Raadiomaja Arvutuskeskus}, nii et selleks ajaks oli Sulo 
juba mõnda aega arvutitega tegelenud. 

Ja see oligi esimene kord, kui sattusin arvutiga kokku ja mõtlesin 
\enquote{oo, vinge!}.

\question{Mis seal vinget oli? Mis konksu külge sa jäid?}

Tol hetkel oli see rohkem \enquote{ahaa, vau, teebki 
mingisugust asja!}. 

\question{Ja Sulo oli kõva mees oma perfolindiga?}

Nojaa, ikkagi kaheksandik, vanem koolivend, kellel on, kujutad 
ette, isiklik perfolint! Vau! Sellised kutid on ümberringi! Siis peab 
ikka ise ka vaatama, mida seal tehakse. Küllap ka Soome televisioonist 
arvutitega seoses nähtu tekitas soovi, et olgu või 
Nõukogude oma ja perfolindiga, aga ikkagi arvuti. 

Mõni kuu hiljem tulid kooli paar djuudi, kes tegid arvutiklubi ja 
kutsusid mind osalema. 

\question{Kas see oli legendaarne arvutiklubi Ahhaa?}

Ei, see oli legendaarne arvutiklubi Juta\index{Juta}, mida vedas 
juudi papi Lev Moišeejevitš Šoroht\index[ppl]{Šoroht, Lev}. Klubi 
tegutses Raua ja Kreutzwaldi tänava nurga peal, ühe maja keldrikorrusel, kus on kaarega 
aknad. Nende akende taga asuski arvutiklubi Juta. Kui 
Ahhaa puhul võiks ette kujutada, kust see nimi tuleb, siis 
Juta nimi tuleneb vene keelest: \begin{russian}Юный Техник 
Автомат\end{russian}. Ma eeldan, et Lev Moišeejevitš Šoroht ei satu seda 
lugema, aga kui kellelegi meenub, et aastal 1985 vedas ta TPI või Peda 
üliõpilasena noori arvutiklubisse, siis ma suurima hea 
meelega saaksin kokku ja teeksin väiksed õlled välja, sest sealt see suurem arvutihuvi alguse sai. 

Klubis õpetati meile programmeerimist PL/I\index{PL/I} keeles.

\question{Kas klubisse käidi kutsumas, mitte ei joostud ust maha, et arvuti ligi 
saada?}

Ma arvan küll. Ma täpselt ei mäleta, aga sõnum jõudis 
meieni vist kooli või õpetajate kaudu. Küllap Sulo oli ka seal, 
sest kui oli võimalik kusagil veel arvutisse saada, siis loomulikult seda võimalust 
kasutati.

\question{Seda ma mõtlengi, et tol ajal ju otsiti tikutulega kohti, kus 
\enquote{arvutisse saada}.}

Seitsmenda klassi lõpupoole tuli arvutuskeskusest ühe minu 
programmi väljatrükk laia aukudega paberi peal, \emph{line}-printeril 
välja lastud. Keegi klassivendadest, kes oli ilmselt meiega seal 
koos käinud, tõi väljatrüki klassi ja siis kõik vaatasid, et 
oo, Soome telekavad. Tollal oligi inimestel kõige 
üldisem seos arvutitega see, et arvutuskeskustes trükiti välja Soome 
telekavasid, kus olid näiteks tabuleeritud kujul eraldi väljavõtted 
seriaalide kohta. Parimatel vendadel olid olemas nädalakavad. Aastal 1985 
keskmine teadlikkus arvutitest umbes selline oligi.

\question{Kust need kavad saadi?}

Need liikusid arvutuskeskuste vahel ja vähemasti millalgi oli üks selline 
koht Postimaja Arvutuskeskus\index{Postimaja Arvutuskeskus}, mis on 
üks väheseid kohti, kus ma ise pole käinud. Seal oli SM-4\index{SM EVM!SM-4}, mille külge oli ehitatud teksti-TV 
vastuvõtja\sidenote{Kavade allikaid oli rohkem kui üks. Mäletatakse, et vastav 
riistvara oli olemas TPI raadiotehnika kateedris\index{Tallinna Tehnikaülikool!Raadiotehnika kateeder} 
Apple II küljes. Räni Meister\index[ppl]{Meister, Räni} olla selleks otstarbeks kasutanud 
ka Eesti Televisiooni\index{Eesti Rahvusringhääling!Eesti Televisioon} Amigat.}, 
ja sealt see kava tuli. SM-4 küljes oli 300boodine modem, millega 
pumbati kavasid mööda linna laiali. Mäletan üht
etappi, kui minu päralt oli üks PC, välja arvatud vist kolmapäeviti, 
kui lõuna paiku saabus Postimajast üle modemi Soome telekava, mis trükiti maatriksprinteril enam-vähem nähtamatuks kulunud lindiga välja. Siis 
pidin endale muud tegevust leidma, aga muudel pärastlõunatel sain seda 
arvutit kasutada.

\question{Ma katkestasin sind seal, kus sa PL/I keeles programmeerisid \ldots}

Meile õpetati natuke programmeerimist ja seal oli palju segaseid ja 
täiesti arusaamatuid asju. Oli programmeerimiskeel mingisuguste muutujatega, 
mis mõnevõrra koitis. Ja ilmselt esimene programm, mida meile õpetati, oli ruutvõrrandi lahendamine. Annad paar muutujat sisse ja siis 
trükitakse tulemus paberil välja. Alguses meid 
päriselt arvuti juurde ei lastudki --- keegi toksis  meie 
programmid sisse ja pärast saime väljundi kätte. 

Hiljem leiti meile võimalus arvutitega tegeleda veel kahes kohas. Üks oli 
Tihnikus, kus asus ETKVLi Arvutuskeskus\index{ETKVLi Arvutuskeskus}. Järgmised 
põlvkonnad teavad seda kohta kui esimest Maksimarketit, seal ühes majas oli üks 
vinge ES\index{ES EVM}. Teine koht oli Endla tänaval, kus asus Maksuameti maja\sidenote{2013. aastani asus 
Maksu- ja Tolliameti teenindussaal aadressil Endla 8.}. Seal kolmandal 
korrusel olid Ehituskomitee\index{Ehituskomitee} ESid\index{ES EVM}. 

Minuga läks edasi umbes niimoodi, nagu õpetatakse tööõpetuses, et on 
oluline anda lastele midagi, mille nad saavad valmis voolida, näiteks puulusika,
et nad saaksid tulla koju ja seda perele näidata. Siis laps saab kiita, tal läheb edaspidi väga 
hästi ja ta teeb paremaid puulusikaid. Kui ma olin teinud 
oma esimese kolmeteistrealise programmi ruutvõrrandi lahendamiseks, laekusin 
selle väljatrükiga koju ja köögis näitasin vanematele, et 
näete, sihukese raha eest tegin sihukese asja. Paps, kes oma 
teadustegevuses tegeles elektrokeemia ja hapnikuanduritega ning teisalt 
oli džässpianist, vaatas mu tööd ja ütles: \enquote{Mul oli 
just üks tudeng, aga ta kadus ära ja temast jäid ainult mingid listingud 
järele. Kas sa saad nendest sotti? Mul oleks vaja teadusandmeid 
töödelda.} Ülesandeks oli anduri toimekõverate kokkuajamine 
matemaatiliste valemitega, et õnnestuks digitaalseid mõõteriistu teha. Ja nii juhtuski, et olles programmeerinud oma 
esimesed kolmteist rida esimeses mulle täiesti tundmatus keeles, 
läksin kohe üle järgmisele.

Nii ongi mu pea nagu puder ja kapsad selles mõttes, et ma suudan kirjutada 
ainult dokumentatsiooni abil, kaasa arvatud keeli, mida ma igapäevaselt 
kirjutan, nagu PHP\index{PHP} ja JavaScript{Keeled!JavaSctipt}. Need 
süntaksid on peas nii segi, et ma ei mäleta kunagi täpselt 
PHPs \verb|for|-tsüklis parameetrite järjekorda. Õnneks on tänapäeval 
olemas kõikvõimalikud IDEd, mis teevad mõningase töö ära ja aitavad \emph{auto 
complete}'ida. 

\question{Nii et su puulusikas mitte ainult ei saanud kiita, vaid pandi kohe 
ka tööle!}

Puulusikas võeti kohe tööle, sealt edasi olin lapstööjõud. Ühel
hetkel hakkas papsil kahju --- laps võiks lisaks 
ekspluateerimisele natuke ka raha saada. Mind pandi ametlikult kirja 
veerand kohaga laborandina. Tänu sellele oli mul ligipääs kõikide papsi sõprade arvutuskeskustele. 
Ja kuna paps tegeles oma hapnikuga TPIs, siis loomulikult olid nende 
sõprade hulgas TPI ja teisi seltskondi, kes olid seotud mingisuguste 
anduritega. Näiteks Pirital Masti tänaval 
arendati sportlaste mõõtmise lahendusi ja selle teine ots asus 
kiirabihaigla arvutuskeskuses\index{Kiirabihaigla arvutuskeskus}. Seal saingi 
pidevalt üht arvutit kasutada, välja arvatud kolmapäeviti. 

Arvutiks oli Sanyo PC ja see oli väga vinge. Seal oli muuseas olemas ka Apple II\index{Apple II}, mille peal sai mängitud Karatekat\index{Karateka}. Ja kui ma 
õigesti mäletan, oli seal ka üks Labtami\sidenote{Austraalia arvutitootja aastatel 1972--1990, kellel 
olid Nõukogude Liiduga head suhted. Aastal 
1984 disainis Novosibirski Riikliku Ülikooli tudengite Kronos Research Group 
neile emaplaadi. URAL-LABTAM OOO tegutseb Venemaal siiani ning nende arvuteid 
leidub lisaks Austraalia arvutimuuseumidele Tartu Ülikooli omas. Labtami 
arvuteid osteti naftadollarite eest ka Küberneetika 
Instituuti\index{Küberneetika Instituut}.}-nimeline \emph{kone}. Lisaks oli seal
suurte trumlitega andmetöötlus-\emph{kone}, millega mina ei suhestunud ja mille  
nime ma ei mäleta. 

\question{Kas andurid ja elektroonika ei pakkunud sulle huvi?}

Mitte eriti. Progemine oli huvitavam. 

TPI santehnika kateedris\index{Tallinna 
Tehnikaülikool!Santehnika kateeder} oli olemas SM-4\index{SM EVM!SM-4}. Ehituse all oli 
selline kateeder, vee kvaliteet ja kõik selline kuulus sinna alla. Mingil 
hetkel tekkis sinnasamasse Järvevana teele, kus asus ka Läänemere 
Instituut\index{Läänemere Instituut}\sidenote{Ei ole selge, mis asutust Peeter 
silmas peab. Eestis on Läänemere Instituut tegutsenud eelmise sajandi 
kolmekümnendatel ja praegu tegutseb sellenimeline asutus Soomes.}, ka 
SM-4\index{SM EVM!SM-4}, mille ma kohale 
minnes lülitasin ise sisse ning pärast töö tegemist jälle viisakalt välja. 

\question{Arvuteid oli siis ikkagi piisavalt?}

Kui sattusid õigesse kohta ja 
oskasid õigel ajal vait olla ning mitte liiga palju täiskasvanuid segada nende 
tähtsas töös, siis üldiselt jagus. 

Tulles korraks veel tagasi alguse ehk Juta\index{Juta} juurde, 
siis selle asutajal oli endal ka paar huvitavat projekti, millega ta 
üritas Nõukogude Liidu tasemel kuulsaks saada. Üks neist võiks olla võrreldav 
Facebookiga: kirjasõbrad kogu Nõukogude Liidust 
saadaksid oma andmed, mis sisestatakse perfokaartidel 
\emph{mainframe}'i ja see teostab \emph{match-making}'u ning 
siis saadetakse kirjad leitud \emph{match}'idele laiali. 

Mina käisin algusest lõpuni Reaalkoolis ja keskkooliajal tekkisid arvutid ka meie kooli. Saime klassitäie Yamaha MSXe\index{Yamaha MSX}. Kuna erinevate koolide vahel oli masinate saamiseks 
konkurents, olla Reaalkool saanud ka ähvarduskõne, mille peale 
vaprad raadiorufi ja füüsikaklassi tagaruumi noored organiseerisid 
öö läbi valve koolimajja. Arvutikastid olid vist direktori kabinetis, me ööbisime koolis ja valvasime neid. Pärast sai
arvutiklass meie teiseks koduks.

\question{Mis seltskond seal raadiorufi ja füüsikaklassi tagaruumis koos käis?}

Seal olime mina, Sulo Kallas\index[ppl]{Kallas, Sulo}, Heiki 
Savitš\index[ppl]{Savitš, Heiki}, Vallo Veinthal\index[ppl]{Veinthal, Vallo} 
ja Reimo Mesipuu\index[ppl]{Mesipuu, Reimo} --- kindlasti jätan kedagi 
ebaviisakalt mainimata. Avo Nappo\index[ppl]{Nappo, Avo} tiirles meie ümber 
rohkem arvutiseltskonna poolest, raadioruumis olime põhiliselt vist mina, Sulo 
ja Reimo.

\question{Mille alusel seltskond moodustus? 
Klassivennad? Tehnikahuvi?}

Otseseid klassivendi oli vähe, jäime paariaastasesse vahemikku. 
Sulo oli kõige vanem, Vallo ja Heiki olid meist aasta nooremad. Tegu oli pigem kooli aktiiviga, keda huvitas tehniline pool. 
Füüsikaklassi juures oli raadioruum, kus me hängisime, sest seal sai 
nuppe keerata. Sellest pundist tekkis hiljem suurem seltskond arvutiklassi ümber. 

\question{Kas programmerimine ja raadioruumitamine koolitööd ei hakanud segama?}

Lõpetasin kooli neljade-viitega, nii et selles mõttes probleemi ei olnud. Medaliga lõpetajat poleks minust nagunii saanud, see ei olnud minu maailmavaates.

\question{Eks see ongi tunnetuse küsimus, kumb oli tol hetkel primaarne.}

Eks arvutipool oli põhiline. Keskkool möödus üleüldse enam-vähem 
niimoodi, et vahepeal sai käidud kohvikus ja vahepeal 
olümpiaadidel. Kui olid olümpiaadid, olid hinded head, sest õpetajad ei 
saanud ometi olümpiaadil esinejatele halbu hindeid panna. Aga kui olümpiaadil 
ei käinud, siis kippusid hinded kehvemaks minema, sest kooliskäimine ununes. Näen siiamaani unenägusid sellest, et eksam on 
tulekul ja ma olen unustanud terve veerandi tunnis käia. 

Sellest tekkis mõtteviis, et ma ei pea olema midagi õppinud. Kui läksin
TPIsse ja mataeksam tehti koos raamatutega, siis jõudsin eksami 
käigus alati ära õppida selle, mida oli eksamiks vaja. Ma ei pidanud 
eelnevalt liiga palju loengus käimisele pühenduma, vaid võisin 
lihtsalt tulla ja eksamid ära teha. Ülejäänud semestri sai arvutitega 
tegeleda. Ma kindlasti ei soovitaks seda noortele, aga minul juhtus
niimoodi. 

Selline lähenemine tekitas mul teistsuguse 
arusaama ümbritsevast tehnikast. Ma ei karda midagi selles 
mõttes, et kui on vaja asi ära teha, siis tuleb võtta \emph{manual} või 
kood ette. Loomulikult võtab see aega --- läksin eksamile 
esimesena sisse ja tulin viimasena välja, aga sain kolme tunniga 
õige asjaga hakkama. Tundus nagu efektiivne lähenemine. Võibolla 
oleksin saanud targemaks, kui oleksin süsteemsemalt õppinud. 

\question{Mida sa TPIsse õppima läksid?}

See oli TI ehk majandusinfo töötlus\index{Tallinna Tehnikaülikool!TI}. Linnar 
Viik\index[ppl]{Viik, Linnar} lõpetas sama ala mõned aastad enne mind. 
Aasta sattus olema 1989, kui päris pol-ök-i ja kompartei 
ajalugu\sidenote[][-2.5cm]{Nõukogude ajal kuulus ülikoolihariduse juurde kohustuslikus korras \enquote{punaste ainete} 
läbimine: kaheksakümnendatel olid nendeks kommunistliku partei ajalugu, dialektiline ja ajalooline materialism, kapitalismi ja sotsialismi poliitökonoomia, filosoofia ajalugu ja teaduslik kommunism. Lisaks veel teaduslik ateism ja marksistlik eetika. Ka kooli lõpetamisel tuli üks riigieksamitest sooritada mõnes nendest ainetest} ei oleks tahtnud õppida, aga nad ei olnud veel välja 
mõelnud, mida nende asemel õpetada. Oli ka muid asju, mille vajalikkusest ma 
päris täpselt ei saanud toona aru. Näiteks
miks ma peaksin tegema transistoritest valmis 8080 protsessori paar käsku, 
eriti kui normaalsed inimesed kasutavad vähemasti Z80-t, mitte 8080-t. 
Teismelise värk --- ei olnud piisavalt \emph{cool}. \enquote{Intel 8008? 
Zilog\sidenote[][-3mm]{Zilogi toodetud 8-bitine Z80 protsessor oli Inteli 
8080 protsessoriga ühilduv, aga märkimisväärselt odavam.} on normaalne!} 
Täpselt sama lugu, nagu täna on hipsteri habe või muud välised 
tundemärgid. 

Pean tagantjärele tunnistama, et kuigi olin algul selle suhtes 
kriitiline, siis hetkel käin koolitusel, kus räägitakse sellest, kuidas 
\emph{fuzz}'imisega\sidenote{\emph{Fuzzing} on tarkvara (turvalisuse) testimise 
meetod, kus programmile söödetakse süsteemselt juhuslikku sisendit.} 
mälukorruptsiooni juhtumeid leida. Kui lektor ütles, et see on maru keeruline, 
räägime hästi aeglaselt ja mitu korda nagu miilitsatele, siis minu arust midagi 
nii rasket seal polnud, \emph{stack} on \emph{stack}. Protsessoril on 
registrid, ma olen neid transistoritest teinud. Kui 
pead protsessori arhitektuuritasemel läbi mõtlema, kuidas käskude 
töötlemine toimib, kuidas pointereid inkremenditakse ja 
kuidas see mäluga on seotud, siis saad aru, kuidas arvuti 
masinkoodi tasemel töötab. Mul on väga tore kuulata, kuidas mu vanem poeg räägib, et
Tartu Ülikoolis sunnitakse neid ka aru saama protsessori siseehitusest. 
Tõsi küll, raamatu tasemel, aga nad programmeerivad ka assemblerit ja see on väga oluline. 

\question{Kas sind akadeemiline maailm ei tõmmanud, kuigi servapidi olid
juba selle sees?}

Ei, sest ma sattusin keskkooliajal sellisesse seltskonda nagu 
vabariiklik õpilasstaap\index{Vabariiklik õpilasstaap}, mis oli 
komsomoli keskkomitee juures tegutsev mittekommunistlik vastupanuliikumine. 
Tiina Tšatšua\index[ppl]{Tšatšua, Tiina} oli näiteks üks selle eestvedajaid. 
Sellest sai vabariigis üks toonaseid orgunnitiime, kes korraldas 
suurüritusi, milleks alustuseks olid komsomoli ja EKP kongressid. Organiseerimise
mõttes on ju savi, kas tegu on EKP kongressi või Eesti Kongressi või Rahvarindega. Inimesed tulevad kohale, neid tuleb 
registreerida ja toita. Kui on dokumentidega üritus (mida 
tänapäeval eriti ei toimu, aga kõik Eesti Kongressi ja Rahvarinde kongressid 
olid sellised), siis on olemas näiteks redaktsioonitoimkond. Meie olime 
arvutitiim, kes organiseeris seda, et registratuur toimiks 
listide alusel, ja samuti toetasime redaktsioonitoimkonda kõikvõimaliku 
tekstitöötluse, väljatrükkimise ja vormistamisega. 

Kui keskkool sai läbi 1989. aastal, siis oli mul suveks üks tööots. Tallinnas 
toimus ÜRO invaekspertide tipptasemel kokkusaamine. Tallinnas lõigati sel puhul esimesed äärekivid faasi ja minu 
arust Jack Lippmaa\index[ppl]{Lippmaa, Jaak}\sidenote{Peeter peab ilmselt 
silmas Jaak Lippmaad} isiklikult ehitas ümber paar 
Ikaruse bussi\sidenote{Ungari tootja Ikarus bussid olid Eestis laialt kasutusel 
liinibussidena.} nii, et neisse kuidagi ratastooliga sisse saaks. Kuidas see 
võimalik oli, ma ei kujuta ette. Meie ehk Reaalkooli tiim toetas ürituse 
redaktsioonitoimkonda, kes vormistas ÜRO-le kõigis põhikeeltes 
dokumente. See tähendas, et oli posu tõlkijaid, aga aastal 1989 ei olnud ilmselt ükski tõlkija näinud arvutit rohkem kui 
võibolla Soome reklaamides. Meid oli piisavalt palju, kümmekond inimest, ja hoidsime tõlkijatel kätt ja jalga. Kui kellelgi tekkis kivistunud pilk, siis 
keegi meist tuli ja \emph{reboot}'is tõlkija arvuti taga või arvuti enda, kumb 
parasjagu oli rohkem kinni jooksnud. 

Minu enda hilisemas eluloos on see episood huvitav sellepärast, et 
olles parasjagu keskkooli lõpetanud, õnnestus mul tolle ürituse jaoks lihtsalt 
omaenda sõna peale linna pealt toatäis PCsid kokku laenata. 
Paar tükki siit, paar tükki sealt ja kokku sain umbes kaheksa arvutit. Kõige kihvtim 
tuli surnukuurist --- üks PC, mille peal oli 
Xerox Ventura Publisher koos Xeroxi graafilise kasutajaliidesega, milleks oli 
GEM ja mis nägi välja nagu MacOS\sidenote{Graphics 
Environment Manager oli üks varastest graafilistest 
kasutajaliidestest, mille liigne sarnasus Apple'i tarkvaraga viis ka kohtuasjani.}. GEM 
sai DOSist üles \emph{boot}'itud, läks ilusti mustvalgeks ja halliks 
kasutajaliideseks ning seal peal jooksis minu esimene küljendusprogramm. Lisaks
saime neilt ühe laserprinteri kasutada, mis ei olnud küll PostScript, aga siiski laserprinter. 

\question{Siis oli ju veel Nõukogude aeg!}

Ilmselt meditsiin oli saanud üht-teist valuuta eest 
osta. Tegelikult Kivilo\index[ppl]{Kivilo, Ago} plaanis kesklinna oma 
diagnostikakeskust\index{Diagnostikakeskus}\sidenote{1988. aastal asutatud Diagnostikakeskus oli omal 
ajal märgilise tähendusega. Ühest küljest pakuti kõrgtehnoloogilisi 
teenuseid, keskuse algusaegadel asus seal Eesti ainus kompuutertomograaf. 
Teisalt oli tegu väga innovatiivse organisatsioonilise 
konstruktsiooniga, mis viis hiljem mitme keskust ümbritsenud 
kõrge profiiliga afäärini.}, meditsiinis olid 
väga kõvad tegijad. Eesti arvutinduse arendusest teatakse rohkem Tartu
seltskonda, kes on seotud geneetikaga, ja võibolla 
Küberit\index{Küber}. Mina sisenesin meditsiiniliini pidi, selles 
valdkonnas tegeldi päris kõvasti teadus- ja arendustegevusega. 

\question{Kas sealt said ka oma küljendamiskonksu?}

Jah. Otse loomulikult sai hunnik flopikettaid Venturaga ära kopeeritud, mis toona oli igati tavapärane, \emph{standard operating 
procedure}: kõigest, mis kätte sattus, tehti koopia. Ja nii juhtuski, 
et aastatel 1989-1990 oli minu jaoks ülikoolis käimisest palju 
huvitavam arvutiga küljendamine ja kujundamine. 

\question{Kas sul on muidu ka joonistamise soon?}

Ei ole. Kahtlustan, et inimesed, kes on pidanud minu küljendatud raamatuid 
tarbima, on kindlasti selle all kannatanud, nii et ma väga vabandan. Näiteks
Avita\index{Avita kirjastus} kirjastuse algalgusaegade raamatutest oli suur hulk 
minu tehtud.

\question{Mis sind küljendamise juures köitis, kui sul muidu ei olnud
visuaalkunsti huvi?}

See oli hoopis teistmoodi arvutiga 
tegelemine kui programmeerimine ja andmetöötlus, mis olid ka toredad. 
Mind tõmbas see, et õnnestus asju ekraanil teha. 

Pärast 1989. aasta suve üritust läksin ma ülikooli. Ja siis 
Mart Siilmann\index[ppl]{Siilman, Mart}, kes oli äsja lõppenud ürituse orgunni 
pealik, ütles, et kuule, järgmisel suvel on ka üks üritus, kus on 
arvutiabi vaja, tule ka. Aastal 1990 toimuski European Nuclear Disarmament Convention ehk suur rahuvõitlejate ja roheliste üritus. 
Sellega seoses tekkis meil ühte kontorisse, mis asus nüüdseks 
lõpetanud NO-teatri ruumides, üks PC, vist Sanyo. Selle küljes oli 1200boodine või -bpsine 
modem. Mingi koha peal lähevad boodid ja bpsid vist lahku\sidenote[][-3.6cm]{Bps (\emph{bits per second}) on sekundis edastatavate bittide hulk. \enquote{Boodid} (\emph{baud rate}) näitavad aga, mitu korda 
sekundis signaal muutub. Kuniks kasutatakse tavalist jadaporti, kus signaalil 
on kaks taset, on väärtused võrdsed. Keerulisemate skeemide korral võib ühe 
signaalimuutusega edastada rohkem kui ühe biti ning kiiruseühikud lahknevad.}. 

Meie ametlik tegevus oli suhtlus orgkomiteega ja selleks 
sai helistatud kaugekõnega Tallinnast Helsingisse. Eestis oli otsevalimine --- 
meil oli selles mõttes väga vinge positsioon, et mujal Baltikumis välismaa 
numbreid otse valida ei saanud. Ka Tallinnas ei olnud seda võimalust igal pool, aga meil oli, sest see 
oli ürituse jaoks oluline. Mart Siilman, endine Fila direktor\sidenote[][-3cm]{Eesti NSV Riiklik Filharmoonia\index{Eesti Riiklik 
Filharmoonia}, mille järeltulija on alates 1989. aastast Sihtasutus Eesti 
Kontsert. Tegu oli mõjuka asutusega, mille korraldada oli kogu Eesti
kontserdielu, sealhulgas levi- ja jazzmuusika ning estraad. Seega 
oli \enquote{Fila endine direktor} äärmiselt mõjukas inimene, kelle jaoks Soome 
otsevalimise korraldamine oli kindlasti võimalik.}, organiseeris, mida vaja. Kuidas, ei tea. 
Igatahes saime helistada Datapakki X.25 võrku, mille kaudu oli võimalik 
suhelda ühe Rootsi serveriga, teine server oli Kanadas. Sealtkaudu 
suhtlesime ürituse orgkomiteega, aga hakkasime ka vaatama, kuhu veel 
õnnestub helistada.

\question{Kuidas te seda \emph{bootstrap}'isite? Mida kliendi poolel vaja 
oli, et võrku saada?}

Tavalist modemit ja tavalise modemiga suhtlevat terminaliproge. 
Modemiga helistasime Datapaki liidestuspunkti, kust edasi läks asi 
pakettvõrguks või X.25ks. Terminali peal oli nagu ikka: lehekülg skrollib ja menüüst tuleb valida 
\enquote{üks, kaks, üksteist}. Lisaks oli meiliboks, kus sai kirju vahetada, ja
jututubade või listide alajaotus. Ja siis, parafraseerides Heinleini: 
\enquote{\emph{Have modem, will find BBSs}}\sidenote{Robert A. 
Heinleini 1958. aasta jutustus \enquote{Have Space Suit --- Will Travel}.}. 
Loomulikult leidsime üles ka selle, et on olemas BBSid. 1989. 
aasta lõpus tekkis Lembit Pirnil\index[ppl]{Pirn, Lembit} esimene 
PirnBoxi\index{PirnBox}-nimeline BBS, mis asus praeguse SEB taga, kus trammid toona kõva kriginaga keerasid, 
Autotranspordi Arvutuskeskuses\sidenote{Eesti NSV 
Autotranspordi Arvutuskeskus (ATAK).}. Nii et 
alguses helistasime kõik sinna Pirni BBSi sisse. 

Peagi tekkis HNS ehk \emph{Hackers Night 
System}\index{HNS}. Kolmas oli Goodwin BBS\index{Goodwin} meil 
Suloga\index[ppl]{Kallas, Sulo}, mis ilmselt jooksis sellesama 
väljahelistamise liini otsas. Öösel jätsime arvuti sisse ja kõik said 
sisse helistada. Kui tahtsid kuhugi sisse helistada, aga liinid olid kogu aeg 
kinni, siis ainuke võimalus olukorda parandada oli panna ise ka üks
\emph{box} püsti. 

Sealt tekkis siis ka Fido pool ja jällegi sissehelistamise küsimus --- 
kui meil oli võimalik e-post ja jututoad omavahel 
kuidagi sünkroniseerida eri masinates, siis polnud ju vahet, kuhu me sisse 
helistame. Masinad käivad päeval ja ööl ning vahetavad omavahel 
sõnumeid. Fido oli selles mõttes korralikult distributeeritud nett. Seesama, 
mille kohta nüüd öeldakse \enquote{veeb kolm}. 

\question{Võrgustike \emph{bootstrap}'imine on keeruline just inimeste 
mõttes. Selleks, et kuhugi külge minna, peaks seal olema huvitav, ning selleks omakorda 
peaksid seal olema inimesed. Mida te näiteks PirnBoxis huvitavat
tegite?}

Ilmselt lämisesime niisama. Pean tunnistama, et ei mäleta, 
aga väga huvitav oli igal juhul. Oletan, et kuskilt
pääses ligi faili kujul \emph{sci-fi} raamatutele ja 
laiematele uudisegruppidele, mis kuskil liidestusid 
Fidonetiga, nii et informatsiooni liikus. Lihtsalt kirjutada
oli ka huvitav, et vau, kõik liigubki traadi kaudu! 
See oli tollal nii \emph{amazing}. Sellest ma sain aru, et arvutiga saab
programmeerida ja midagi kujundada, aga et ka reaalselt suhelda!

\question{Mis tegi ühe BBSi populaarsemaks kui teise? Goodwin ja HNS 
olid pikka aega populaarsed, kuigi PirnBox oli esimene.}

See oli esimene jah, aga jooksis toona vähem 
levinud softi peal. Meil oli vist Maximus. 

Sulo oli omamoodi arvamusliider, kuna tal olid 
kõikvõimalike asjade suhtes väga toredad ja tugevad seisukohad. 
Mina olin niisuguse tutu-lutu taustaga, olles olnud muu hulgas 
Reaalkooli\index{Tallinna 2. Keskkool} viimane komsomolisekretär.
Arvestades et enne mind oli komsomolisekretär Karl Martin 
Sinijärv\index[ppl]{Sinijärv, Karl Martin}, siis me ilmselgelt ei võtnud seda 
asja väga tõsiselt.

Kuidagi me sattusime seda asja vedama, kuna meil oli tänu sellele 
tuumaüritusele ressurssi käes. Ühel hetkel tekkis meil igatahes kaks 
telefoniliini, võibolla aastake hiljem, kui üritus läbi sai ja 
olime juba Eesti Instituudi\index{Eesti Instituut} ruumides, veidi enne 
seda, kui Eesti Instituut osutus tegelikult Eesti välisesinduste ja iseseisvuse 
ettevalmistuslavaks. Näiteks kui kuulutati välja iseseisvus, 
tuli järsku välja, et Jüri Luigel\index[ppl]{Luik, Jüri} ja kõigil teistel, kes 
mööda maailma laiali olid, olid juhuslikult kaasas ka pruunid ümbrikud 
esitamiseks kohalikule võimupealikule küsimusega, kas teie ekstsellents 
lubaks meil asutada suursaatkonda. 

Eesti Instituudis olid meil ka oma arvutid, aga ma ei mäleta, kas meie enda või instituudi omad. Saatsime Suloga\index[ppl]{Kallas, Sulo} öösiti 
fakse. Mitu toredat kolleegi 
oli, vähemalt huumoriga pooleks, sügavalt veendunud, et faks ongi selline seade, et kui sinna 
peale panna paber koos kollase post-itiga, kus on telefoninumber, siis on see 
hommikuks ennast ära saatnud. Tollased liinid toimisid öösiti oluliselt paremini kui päeval. 

Tingituna sellest, et välisühendust oli meil läbi modemi helistades 
suhteliselt piiramatult käes ja liine oli ka mitu, siis oli meil kaks 
modemiühendust. Ühel hetkel hakkas meie ja Fidoneti kaudu väljapoole 
ühenduma Läti.

\question{Ma teadsin, et Vene Fidonet käis läbi meie, aga et ka Läti?}

Venemaa tekkis ka jah millalgi. Läti oli Fidonetis Eesti all, aga leedukad loomulikult ei oleks millegi 
selliseni laskunud, et nad on mingi Eesti regioon kusagil 
võrgustruktuuris. Nemad selle asemel helistasid kord nädalas ja tõid e-posti enam-vähem 
nagu ämbriga, välja arvatud Kaunase 
Ülikool\index{Kaunase Ülikool} ja Leedu parlament\index{Leedu Seim}, kes olid 
Goodwin BBSi pointid. Seal oli hädasti vaja ja uhkus jäeti kõrvale. 

Lätlased käisid meil külas ka. Panid raha kokku ja tõid selle meile 
ühenduse eest. Investeerisime need kakssada dollarit kahte modemisse --- ostsime US Roboticsi\index{US 
Robotics} HSTd, mille kiirus oli vist neliteist kilobaiti. Väga väärt aparaadid, nii et lätlased panustasid Eesti neti 
arengusse.

Samal ajal ametlikku postivahetust internetiga pidas 
Küberi\index{Küberneetika Instituut} seltskond, aga neil oli Mustamäel suhteliselt sant jaam, mis krabises ilmselt rohkem, kui sidet läbi 
lasi. Ühtlasi olid akadeemilised tüübid millegipärast suured
UNIXi sõbrad ja kasutasid PEPi TrailBlazereid\index{Telebit TrailBlazer}, 
mis esiteks olid 9,6 kilobaiti ehk mõttetult aeglased ja teiseks suutsid HSTd 
postsovetlike liinidega paremini sidet vilistada. Olime 
sügavalt veendunud, et need olid ka oma reaalselt võimekuselt pikki seansse ja 
kiirust üleval pidada märksa paremad. 

\question{Kas sa mõtled \enquote{liini} all ikka telefoniliini?}

Jah, need olid tavalised analoogliinid, mille otsa käis kettaga telefon. 
Keskjaamas numbrit valides jooksid releed kontakte mööda ringi. Kui modem valis, oli kuulda klõbinat, kui see
releega katkestusi tekitas. Kõik oli elektriline, sellepärast ma kujutangi 
ette, kuidas andmeside tegelikult toimub. Aga kuidas on võimalik, et mingid 
vennad panevad läbi ADSLi kümme megabaiti? Meie panime enam-vähem samasugust asjast 
läbi neljateistkümne kilobaidi, nii et täiesti arusaamatu. Wifi täpselt samuti. Ma ei 
kujuta ette, kuidas see põhimõtteliselt saab üldse toimida. 

\question{Kas kujundamisega tegelesid kõige selle kõrval?}

See käis jah kõrval. Umbes 
samal ajajärgul sattusin seltskonda, kellel oli arvuti ja printer ning vajadus 
midagi trükkida. See oli poistekoor\index{RAMi poistekoor}, mida juhtis Venno 
Laul\index[ppl]{Laul, Venno}\sidenote{Venno Laul asutas 1971. aastal Riikliku 
Akadeemilise Meeskoori juurde poistekoori ning oli kuni 1990. aastani selle 
kunstiline juht ja peadirigent.}. Neil oli esimene PostScripti printer, mille 
ostus ma osalesin --- kas Tektronix või muu säärane A4-formaadis 
300 dpi laserprinter. Ühendasin Ventura selle külge.

Põhimõtteliselt kõik toonased kujundusprogrammid olid sellised, et arvutis olid 
\emph{bitmap}-fondid, mis saadeti juhet pidi printerisse, ja see kõik võttis kaua 
aega. Aga PostScriptiga sai lehekülje nagu programmi saata 
printerisse, mis siis oma tarkusega joonistas. See oli Adobe ja Apple'i 
vendade poolt väga mõistlik valik, kui nad kord Silicon Valleys kokku said ja 
otsustasid, kes mis osa maailmast vallutama hakkab. Tõepoolest, kontoris ei pruugi 
igal vennal printerit olla ja selleks, et inimesed saaksid printida, 
võiks olla printer ka tark. Väga spetsiifiliselt tark, et suudaks 
joonistada lehekülje endal mälus valmis ja siis välja trükkida. 

Millalgi samal ajal puutusin kokku ka Sirbiga\index{Sirp} (ma ei tea, ka 
see toona oli juba Sirp või veel Sirp ja Vasar), mis oli üks esimesi 
ajakirjandusväljaandeid, mis läks üle digitaalsele töövoole. Alguses oli protsess umbes 
selline, et toimus tinaladu, millega tehti kas siis üks tõmme paberi peale ja 
see vist pildistati üles. Nii et ofsettrükk toimus 
veel läbi tinalao. Ja nüüd, kui oli võimalik minna üle sellele, et arvutist 
saaks välja trükkida, siis see oli megaraju.

\question{Kas PostScripti printerist lasti trükitavad 
asjad kilele?}

Põhimõtteliselt jah, peegelpildis. Üks asi, mille koos 
Suloga\index[ppl]{Kallas, Sulo} Eesti 
Instituudis\index{Eesti Instituut}tegime, oli PostScripti \emph{pre-header}, mille nimi oli vist Preambul. 
PostScripti puhul tuli programmiga kaasa 
koodijupp, mis kirjeldas programmeerimiskeskkonna, defineeris 
täiendavad funktsioonid ja muud oma käsud. Seejärel tuli kood ise ja lõpus 
koristusfunktsioon või midagi sellist, mis välja trükkis. PostScript oli tore selles 
mõttes, et see oli nagu \emph{open source}. Mitte küll vabatarkvara, aga 
nähtava lähtekoodiga. Ehk oligi võimalik võtta sama Ventura ette, 
mis kusagilt laadis selle Preambuli, mis oli tekstifail ja mida oli võimalik 
muuta. Ette sai kirjutada transformatsiooni, mis 
keeras pildi peeglisse. Meil õnnestus see Sulo PostScripti preambul maha müüa kooliraamatute kirjastusele
Avita\index{Avita kirjastus}, mille eesotsas oli Tiit Aunaste\index[ppl]{Aunaste, 
Tiit}. Hiljem, vist 1991. aastal sattusin ise ka Avitasse tööle, asjad liikusid tollal
väga kiiresti.

\question{Jah, sest umbes viis aastat hiljem mäletan mina sind Eesti 
vaieldamatu autoriteedina teemal, kuidas arvutist värviline asi trükki saada.}

Eks ma olin seda piisavalt praktiseerinud. Tegime Ventura peal ka 
haltuuraotsasid. Liikus muudki tarkvara, näiteks Arts \& Letters, millega oli võimalik paigutada tähti ringikujuliselt. 
Toona asutasid kõik aktsiaseltse ja börse ning 
neil oli vaja pitsateid. See oli meeletu innovatsioon, et oli võimalik 
arvutist ühe matsuga pitsat valmis teha ja ei pidanudki 
kujundajatädi fotolao tähti välja lõikama ja kleepima. 

Sirbi toimetus andis välja Jutulehte\sidenote{Ilmus AS Kodamu väljaandel 
aastatel 1990--1992.}, mille \emph{layout}'i tegin mina. 

Nii ma tasapisi sattusingi selle ala peale. Käisin ka Helsingis vaatamas, 
kuidas Helsingin Sanomati\index{Helsingin Sanomat} tehakse. Neil olid 
miniarvutid ja rohelised terminalid ning 
Linotronic\sidenote{Mergenthaler Linotype Company toodetud 
kõrgekvaliteediline printer. Tegu oli kalli seadmega, mis võimaldas trükkida 
resolutsioonis kuni 2540 dpi.}, millega trükiti veerge 
fotopaberile. Fotopaber oli 30 cm lai ja sellele lasti välja üks 
ajaleheveerg. Siis lõigati veerud kääridega välja ja pandi suure 
maketi peale, mis oli vaha või millegi säärasega koos. Rastreeritud fotod 
pandi veergude vahele, leht sai kokku ja tulemus saadeti faksiga 
trükikotta. Faks ei olnud loomulikult tavaline faks, vaid 
\emph{industrial-grade} ajaleheformaadis kõrgeresolutsiooniline masin, mis 
skännis ühelt poolt sisse ja teiselt poolt trükkis välja 
filmi, millega valgustati trükiplaadid. 

\question{Mis selle juures köitis? Kas tehnoloogiline keerukus või 
see, et protsessil oli palju samme, või veel midagi?}

Kõige huvitavam on tegeleda asjadega, millega teised parasjagu ei tegele. Või 
ka asjadega, mis toimivad teistmoodi, kui olen 
siiamaani arvanud. Niisama Pascalis programmeerida ei olnud väga huvitav, seda õpetati 
koolis. Aga kuna mul oli ilmselt olemas arusaam, kuidas asjad töötavad 
ja mis masinates on, siis ma suutsin asju efektiivsemalt tööle panna. Näiteks kirjastuses siduda küljendus- ja 
tekstitöötlusprogramm. Tekstitöötluses oli levinud WordPerfect (Perfect? 
Prefect? Ford Prefect ja Word Perfect!\sidenote{Peeter viitab tõenäoliselt 
Douglas Adamsi loodud tegelaskujule, mitte omaaegsele populaarsele 
automargile.}), meie küll kasutasime rohkem Volkswriterit\sidenote{Volkswriter oli üks esimese PC-platvormi tekstitöötlusprogramme, 
mida arendati 1980ndatel. 
Volkswriter oli saadaval ka eespool mainitud GEM-platvormile, sellest ilmselt 
selle kasutamine kirjastustöös.}. Tekstitoimetaja toimetas 
WordPerfecti faili, kus olid stiilid juba ära märgendatud. Küljendaja luges
selle oma küljendusprogrammi tagasi ja teksti korrektuuri oli võimalik teha ilma kallima arvuti või 
keerulisema programmita. Seda õnnestuski meil väga efektiivselt juurutada. 

Enne rublaaja lõppu, 1992. aasta alguses, tekkis Prisma 
Printi\index{Prisma Print} esimene Linotronic ehk filmiprinter. Seniajani
peeti 600 dpi laserit väga heaks, nüüd tekkis järsku 1200 dpi 
filmiprinter. Prisma alumisel korrusel olid suured Crosfieldi\sidenote{Crosfield Electronics oli Briti firma, mille toodetud 
skännereid peetakse siiani ühtedeks paremateks, mis iial tehtud.} trummelskännerid, millega sai 
teha värvilahutusi, filmi peale, juba rastrisse. Ja kogu montaaž 
toimus endiselt nii, et tekstikile ja pildikile või film 
valgustati või lõigati füüsiliselt kuidagi kokku. 

Mul oli keskmisest parem ettekujutus, kuidas need süsteemid 
töötavad. Kui mina jõudsin oma failidega kohale, nägid enamasti kõik vaeva, kuidas QuarkXPressist midagi välja printida. Minul olid kaasas oma flopid ja hiljem 
magnetoptilised ja ma trügisin vahele, et 
minu omad vahepeal välja lastaks, kuna ma ei viitsinud teiste järel oodata. Ja lastigi, sest minu asjad käisid tõesti kiiresti läbi, kuna ma 
kujundust tehes kujutasin ette, kuidas see PostScriptiks läheb. Nii 
küljendusprogramm kui ka seesama Ventura või graafikaprogrammid, nagu 
Illustrator või Freehand, aitasid mul visuaalselt 
valmistada ette PostScripti. Teadsin, kuidas see koodis 
välja näeb, ning sain võtta faili ette ja näha, kus miski on. Tänu 
sellele teadsin ka, mis on printeri jaoks keerulised asjad, 
oskasin neid lihtsustada ja mitte liiga keerulisi asju kasutada, sest 
see prose, mis seal taga oli, oli suhteliselt vaene. Kui suudad 
tekitada olukorra, kus programmil on tsükkel tsüklis (tänapäeval tuleb 
sinna otsa veel SQLi päring), siis üldiselt on see asi suhteliselt ebapädev. 

Tänu arvutitaustale ja kujundamisele tekkis mul hulk disaineritest sõpru ja teisi sel alal tegutsejaid, keda ma 
Prisma Prindi väljatrükijärjekorrast teadsin. 

Edasi sattusin tööle Uniprinti. Algul töötasin andetu disainerina, aga 
siis leidsin tasapisi võimalusi vähem disaini nõudvaid asju 
teha, kus mängis rolli just see, et ma suutsin võtta ette näiteks Eesti Näituste 
näitusekataloogi andmebaasi (see oli vist Microsoft Accessis) ja 
genereerida väljundiks tekstifaili, mis oli stiilidega märgendatud ja mida 
oli võimalik küljendusse sisse lugeda. Jällegi asi, mida tollal ei olnud 
\emph{desktop publishing}'is tavaks kasutada: valmistasin 
stiilid ette ja tekst kasutas neid, kohapeal midagi tegema 
ei pidanud.

\question{Nii et põhimõtteliselt oli tegu CSSiga?}

Täpselt. Põhipõhimõtteliselt nagu CSS, ainult et tekst ja paber. Veeb töötab siiamaani niimoodi, aga see oli minu meetod, mis võimaldas teha huvitavaid 
töövoogusid. Minul oli andmebaas käes, näitusetüdrukud, kes müüsid bokse ja korraldasid üritusi, tõstsid asju 
ümber, täiendasid firmade andmeid ja parandasid telefoninumbreid. Mina trükkisin \emph{layout}'i välja, viisin neile 
ja nemad tegid andmebaasi korrektuuri, samal ajal kui mina joonistasin logosid 
puhtaks. Niimoodi ma õppisin. Nagu ma ütlesin, siis ma olen andetu 
disainer, aga tehniliste protsesside ja töökorralduse poolt 
teadsin tollal rohkem kui keegi teine. 

\question{See klapib kenasti sellega, mida sa praegu tundud tegevat. 
Millega sa üldse tegeled?}

Jah, seoseid on igasuguseid. Kui ma veel trükialal tegutsesin, oli mul palju vaba aega tänu sellele, et olin suutnud oma 
tööd optimeerida. Ja ka tänu sellele, et Uniprindi pealikud Sirje ja Andrus 
Reinsoo\index[ppl]{Reinsoo, Sirje}\index[ppl]{Reinsoo, Andrus}, kes on just 
mõlemad lahkunud\sidenote[][-2mm]{Intervjuu Peetriga leidis aset märtsis 2019.}, jätsid 
mulle piisavalt vabadust. Käisin ja kolasin Ameerikas 
konverentsidel. Sel ajal oli valdav suhtumine, et mis 
mõttes väljamaa ja konverentsid? Me oleme Eestist ja teame väga hästi. Mina 
käisin \emph{cyber publishing} seminaridel, mis olid 
seotud just trükipoolega, mis mulle huvi pakkus: plaaditrükk ja 
muu selline. Ja kuna ma olin põhimõtteliselt nagu ajakirjanik, siis oli
mul võimalik möllida ennast konverentsidele, mis muidu maksid 
paar tuhat taala (tolle aja mõistes röögatult palju), 
ajakirjaniku passiga sisse. 

Aastast 1994 hakkasin ka kirjutama. Paralleelklassivend Peeter-Eerik 
Ots\index[ppl]{Ots, Peeter-Eerik} oli Äripäevas ajakirjanik ja kirjutas 
tehnoloogiateemalisi lugusid. Mul reaalikana hakkas 
mõnevõrra piinlik, sest kirjutatu ei tundunud olema piisavalt pädev. Post-BBSi ajastu inimesena olin kindlasti ka võrdlemisi 
\emph{opinionated} noor inimene oma kindlate 
eelarvamustega. Kirjutasin teisele Peetrile paar lugu ette, et avalda parem 
neid, vähemasti kirjutatud kellegi poolt, kes enam-vähem saab aru, millega 
tegu. Peeter ütles, et võiks need ikka minu nime 
alt avaldada ja saaksin honorari ka. Nii sattusingi Äripäeva 
kirjutama, hiljem ka mujale, näiteks
Arvutimaailma\index{Arvutimaailm}. 

\question{Tehnoloogia tehnoloogiaks, aga mis sind kirjutamise juurde tõmbas?}

Kirjandite kirjutamisega sain suhteliselt hästi hakkama juba kooliajal. Minu 
esimene avalikustatud töö oli Pikri\index{Pikker} noorte 
huumorivõistluse võidutöö. Olin üht-teist ka lugenud, sõprus 
sõnaga oli olemas, lisaks olin teinud kooli omavalitsust ja muud 
sellist. Ilmselt olin parasjagu jutukas ka. Kirjutamine ei 
olnud keeruline ja võibolla meeldis mulle ka õpetada --- läbi kirjutamise on 
võimalik teisi õpetada ja panna midagi teisiti tegema. Olen Peeter-Eerik Otsale väga tänulik, et 
ta tegi midagi valesti. See on ka interneti puhul tüüpiline: 
\enquote{\emph{Wait, somebody is just wrong on the Internet!}}. Kirjutamine 
ilmselt sai alguse sellest, et \emph{somebody was wrong} ja mul oli vaja 
kaitsta oma seisukohta ning loomulikult ka Reaalkooli au. 

Sedasi see algas ja hiljem palusid ka teised mul kirjutada. 
Ma siiamaani ei oska ei öelda ja eks edevus mängis ka rolli. 
Pealegi oli see valdkond suuresti katmata. 1995. või 1996. aastal kutsus Avo Raup\index[ppl]{Raup, 
Avo} mind kui juba kirjutanud ja tuntud inimest Raadio 2 saatesse \enquote{Võrgutaja} külaliseks. Meil klappis nii hästi, et minust sai resident-saatekülaline. Esimene inimene, keda 
ma sattusin saates üksinda intervjueerima (Avo oli vist haige), oli Kaido 
Saarma\index[ppl]{Saarma, Kaido} Abobase Systemsist\index{Abobase Systems}. 

1999. aastal tuli minu juurde 
Sarvik\index[ppl]{Sarvik|see{Sarv, Henn}}\sidenote{Legendaarne IT-mees Henn 
Sarv\index[ppl]{Sarv, Henn}.} ja ütles, et Kukust kas Lang või Tiido oli öelnud, et on vaja teha arvutisaadet. Istusime sealsamas Uniprindi 
lähedal Pärnu maanteel Westmani poe vastas keldris Hollandi 
õlletoas ja mõtlesime välja saate Tehnokratt. Juba esimesel hooajal sattusime Kukus kokku 
tegelastega, kellel oli mõte ka ETVs\index{Eesti 
Rahvusringhääling!Eesti Televisioon} midagi sellist toota. \emph{Whatever}, toodame! Nii sattusingi 
telesaatesse, kus pidin olema korraga toimetaja ja saatejuht ning panema kokku ka montaažiriba (mis tuli muidugi mõnevõrra üllatusena).

\question{Ja nüüd oled ringiga tagasi\ldots} 

Kas nüüd tagasi või edasi, aga praegu olen Zone'is\index{Zone}, mis on täiesti 
juhuslikult ajaloos esimene kord, kui töötan mingit otsa pidi 
IT-firmas. Olen küll vahepeal olnud reklaamiagentuuris digitiimi juht, 
mis on ka natuke IT, aga ikkagi reklaamindus. Nii et olen töötanud trükinduses, teisi koolitanud ja kõike muud teinud, 
aga see on esimene kord, kui mingid IT-tüübid mõtlesid, et palkaks Marveti siia 
tuututama. Ametlikult on mu müts seotud turunduse ja kommunikatsiooniga, aga 
tegelen ka sellega, et kui keegi ütleb, et midagi ei tööta, ja kõik 
väidavad, et töötab ju, siis kuidas saada aru, mida inimene 
tegelikult tahab. Äkki tal on õigus, et tal ei tööta. Äkki on võimalik, 
et see asi, mida meie oleme nunnutanud ja silunud ja teinud maailma kõige 
paremaks, tema kontekstis ei tööta. Ja täiesti üllatavalt selgub, 
et kui on piisavalt keerulised süsteemid, siis olukordi, kus tuleks 
kõige suurematele ja parematele püüdlustele vaatamata midagi 
teisiti toimima panna, on uskumatult 
palju. 

\question{Küll sa turunduse ka ära optimeerid, nagu sa kõik asjad 
ära oled optimeerinud!}

Jah, ma üritan. Mul see lootus on natuke teistpidine. 

Kunagi tuli Andres Kulli\index[ppl]{Kull, Andres} ja Kroonpressi\index{Kroonpress} 
seltskond küsima, kuidas panna reklaami ajalehte. Mina rääkisin, et on 
olemas PDF. Teeme parem nii, et kõik teeksid korraliku PDFi, leheküljendaja 
tõstab selle küljendussofti sisse ja kõik töötab. Kull, ikkagi
suure trükikoja juht, ütles seepeale: \enquote{Väga hea, nii teemegi. 
Kõik peavad saatma oma asjad PDFina Postimehesse}. Ja üllatus-üllatus, nii läkski. 

Mu enda roll selle kõige juures oli, et olin olnud pikka aega Prisma Prindis ja 
muudes reprodes selline majasõber, kes sageli tolknes seal ja üritas endale 
tegevust leida ning saada aru, kuidas asjad käivad. Näiteks võtsime
Eesti esimese Linotronicu pulkadeks lahti ja jootsime seal midagi, sest masin otsustas töö lõpetada parasjagu, kui oli vaja midagi välja 
lasta. 

Teadsin, millist roppu vaeva kõik mu repropealikest või -tehnikutest sõbrad olid näinud kehvasti ette valmistatud 
originaalidega. Kui PDFindus hakkas meile endale majja tulema, siis 
mõtlesin, et mina küll ei hakka selle ussipurgi avamist enda peale võtma
(tänapäeval räägitakse rohkem \emph{surströmming}'ust kui Pandora laekast). Ainuke 
asi, mida ma saan teha, on õpetada kliendid paremaid originaale saatma, mis 
loomulikult tundus äärmiselt lihtne:
ütlen neile, et seal on vaja mõned linnukesed panna ja siis kõik 
lähebki nii, nagu vaja. Aga tuleb välja, et ei. Olen õppinud, et päris 
kõva pingutus on aru saada, mida teised inimesed teavad, ja 
panna nemadki aru saama millestki, millest mina aru saan, 
seejuures ise liigselt masendumata või 
nende peale kurjaks saamata. Nii sattusingi õpetama Pagemakerit, 
InDesigni, Photoshopi ja muud säärast just töökorralduse poole 
pealt. Hetkel Zone'is näen ma, et kui vaadata kogu veebiga 
seonduvat, siis ilmselt tuleb proovida selle kõigega veel rohkem edasi minna. 

\chapter{Andres Peiker}
\index[ppl]{Peiker, Andres}

\question{Kuidas ja umbes millal sa jõudsid arvutite juurde?}
Äkki äkki äkki?
Tere See siin on Memm copy ammu enne seda, kui Silicon Valley maailmale näitas, kuidas suuri infosüsteeme skaneeritakse oli Eestis seltskond inimesi, kelle hallatav infosüsteem kahekordistas oma ärimahte iga üheksa kuu tagant ja tegi nii kümme aastat järjest. See kõik oli väga suuresti meie tänaseid külalisi. Teine külas on Andres Peiker. Head kuulamist. Memm labi.
Tere.
Sinu nimi on Andres Raid. Raili. Alustame asjade algusest, sealt, kust kõik asjad on pihta hakanud. Kuidas sinu arvutite juurde said?
Ta oli mingi kaheksakümne neljas aastake ja osta, et.
Suht juhuslikult tegelikult selles mõttes, et ma õppisin siis keskkoolis ja ma käisin mingisugustel füüsika loengutel Tartu Ülikoolis ja ja ühe tolle loengu lõpus mees nimega Otto talle asendus auditooriumite ütles, et, et aga, et kes on arvutitest huvitatud, et võivad natukene siia veel jääda. Ja noh, siis mingisugune seltskond jäi, Otto Teller viis meid siis seal tähe neli olevas õppehoones. Seal oli kaks maili arvutit ma ka ja naine kaks vist olid nood viis näitas nõid kes ütles, et noh, et põhimõtteliselt siin nagu mingitel õhtustel aegadel on või on võimalik käia nagu programmeerida proovida asju.
Millest, millest ma kohe järeldad, sa Tartu boss absoluutselt esimesed kakskümmend viis eluaastat vanad ja, ja siis, millest ma järeldan, on, on see, et kui sa keskkooli ajal kuskil Tartu Ülikoolis mingites loengutes kestis, siis kui pidi olema mingi matemaatika, kui reaalainete või niisugune huvi
No ma õppisin Tartu esimesest keskkoolist tuli matemaatika-füüsika eriklass.
Noh, ma käisin olümpiaadidel või ei mäleta täpselt, kui Kustu ülikooli loengute teema üldse tuli, et noh, füüsikas tüli ja füüsika tundus mulle nagu kõige põnevam asi üldse, et et siis siis saigi seal käidud.
Ja seal fäski loengu lõpus. Mind paneb imestama, et keegi üldse nagu ära, eks, et selles mõttes, et kõik sõnavahed ja rahvas tundus nagu arvutite vastu huvi tundnud või siis nii olnud.
Ei, ikka ei olnud selles mõttes, et ma arvan, et ikkagi pooled läksid ära ja ja ta on grupp jäi tegelikult noh, esimesel korral käisime neid arvuteid vaatamas, siis öelda, et noh, et järgmine kord saaks nagu sel päeval tulla, siis tuli juba vähem inimesi ja lõpuks jäi mingisugune, ma arvan, mingi kolm-neli inimest võib-olla alles, kes on nagu rohkem käi käima hakkasid.
No see oli mingisugune ring või lihtsalt.
Ma enam nii täpselt ei mäleta, et, et ma arvan, et Otto Teller ikkagi seal natukene juhendas ka alguses, et, et mis, mis ja kuidas, et kuidagi me tolle AB programmeerimiskeelega, mis on mägede peal, oli kuidagi me tuttavaks saime, et ma arvan, et läbi läbi talle tema vahet
Mingeid raamatuid niisugust kraami.
Ei, seda ma küll ei mäleta, et oleks olnud.
See on huvitav asi. Tänapäeval me pöörame palju tähelepanu selleks, et õpetada inimesi programmeerima ja see on täiesti läbivalt, mitte keegi suuda meenutada, kurjasid, õpsid, programmeerib, kuidagi sündis täiesti lihtsalt tuli. Aga mida te tegite sinna nairidega?
Noh, see on, seal sai ikkagi teha väga lihtsaid mingisuguseid arvutust või samme selles mõttes, et tollel arvutil ju koha pealt tuli elektroonilise kirjutusmasinaga, nii et sa kirjutasid programmi ja ta tuli paberi peale ja ta oli ainukene eksemplar tollest programmist, mida, mida sa pidid siis alles hoidma tuleb, eks needki saab parandada, tahtsid, siis sa pidid vaatama today prinditud paberit. Et et noh, kõige kõvem asi, mille ma seal valmis tegin, olis dollareid, tähtsad asjad, biorütmid, et noid Arvutuskeskuses tehti ja, ja siis ma tegin, talle meeldib ka peale tegin ka biorütmide programmi, et.
Et ma saan auto, osutus seal populaarseks, et tollest minu perfolingvist keegi tegi koopia ja siis lasti toda seal talle Tähe neli töötajatele usinasti välja, ilma et ma midagi tean.
Biorütmide arvutamine ja see oli kuidagi naljakas, sest tõrvist need algoritmid, mis liikusid, olid vist mõeldud käsitsi arutamiseks ja seal olid minu meelest kuidagi arvutati Siimust.
Kui ta skandaali Taligi lihtne siinus selles mõttes, et lihtsalt sa pettur sünni sünniaja ütlema ja siis tolle elatud päevade arvu pealt noh, tosiinusega ka häbilainepikkus oli lihtsalt nendel emotsionaalne ja füüsiline ja seksuaalne, mis neljas oli, ma ei mäletagi, oli, oli lihtsalt erinev mustajoonistest olla neli siinus sisuliselt välja, tegelikult et noh, täiesti triviaalne asi iseenesest taheti, rohkem oligi too, et noh, kuidas paberi peale toda sinust joonistab, et elektroonilise trükimasinaga
Jah huvitav asi, mis oli nagu oluline meil tänaseks täiesti varaga biorütme hakata joonistama inimestele jumalaid.
Tuli siis mingisugune väga popp asi ja ja tundus tolle arvuti jaoks nagu niisugune jõukohane ülesanne pärast, et ma arvan, et see oli mingi, kas neli kilobaiti oli, oli mälu tollele arutelule ja noh, ta oli sama suur kui mul, ma ei tea, kodus köögimööbel.
Et selline suur asi, milleks ta tänu füüsikud kasutasid
Samamoodi kasutamiseks.
Aga mida mingid?
Ei tea sellest sellest nagu ei olnud juttu selles mõttes, et on ai kaali nagu väiksem masin teises toas oli null kaks. Põhiliselt ma saan aru, kasuti Duda meie sinna masinale nagu eriti pidi ei saanud. Et oli nagu rohkem hõivatud ja Toljad vaatasin ka selles mõttes, et lindiseadmed need suured lindikapid, kus siis seda magnetlinti keerutate ja tol ajal ei olnud mitte see tavaline elektrooniline kirjutusmasinaid, tol oli niisugune trummeliga printer. Noh, mis suutis ikkagi toda paberit nagu päris kiiresti välja lasta, et.
Printimise tehnoloogiaid rublale kuidagi väga oluline ka absoluutselt. Kas sealt see pusimine oli lihtsalt puhtalt niisugune nõu põne, arvutiga möllamine või seal mingisugune nagu sügavam asi ka tagavad, sulle tundus, et kuidagi, et seal mingisugune asi, mis seda teha tahan.
Siis tuli puhtalt seotud tegelikult sellega, et et kuna keskma olin matemaatika-füüsika eriklassis, siis meil oli seal Andres Jaeger, ülikoolist andis, andis programmeerimist ka kolm aastat, aga ta programmeerimine oli sisuliselt ainult mingite blokk skeemide joonistamine paberi peale, ühesõnaga noh, me arvuti ligi ei saanud ja, ja noh, siis too Tähe tänaval oli võimalus nagu ise järele proovida, siis seda, mida sa olid tegelikult nagu paberi peal teinud. See, et toda nagu tavakooli õppeprogramm ei võimaldanud kooliprogrammile joonestuspaberi peale, algas.
See paberi peal paks geimid joonistamine, see võis ju huvi ära tappa, aga sul või tapnud seal millegipärast läksid. Pärast seda loengut jäid sinna ja millega see käsitsi Emmy eid.
Ei, ta ei tapnud kindlasti tolle pärast, et, et ka too blokk, skeemide joonistamine, et näed, kui sa talle ülesande said, noh siis siis ta siis ta ütles ka, et umbes, et noh, et kes suudab nagu mingisuguse
Kolme Ifiga teha, et on, on hea kaheffiga on väga hea, et noh, et ühega noh ma ei tea, ja noh, siis siis sul oli nagu eesmärk olemas, et sa pidid ühe väga tegema talle tulemast, et noh, too asi kõnetas mind ja noh, siis sa said õpetaja käest kiita ka murde on tõesti, et noh, et ma mäletasin, võib-olla viis aastat tagasi oli meil ka üks õpilane, kes suutis nagu selle algoritmi nüüd selliselt ära teha, et ja väga hea
Ühesõnaga sinu jaoks ajalise oskab joonistada niisugune nagu ülesanne, kui naguniisugune pusle või.
Jah, absoluutselt selles mõttes, et noh, et ikkagi ütleme, et toob üles loomulikult jõuga, sa suudad tolle mängu lihtsalt ära teha, aga et noh, et nüüd, kuidas ta nagu kõige optimaalsem saaks kõige parem. Et, et noh, tooli huvita.
Okei ma ütlen, et, et selle koolikoolid olla ühtegi nagu arvutitel on Tartu linna peal ju arvuteid oli nagu küll tootjad.
Meie koolis ei olnud, ei olnud, siis ei olnud ikka mingisuguseid arvuteid kuskil selles mõttes, et.
Tähe tänaval oli, olid kunud kaks naid. Loomulikult seal ülikooli arvutuskeskuses oli jeeess. Lahkus, kes üldse nagu ise arvuti ligi saanud operaatorid lõid programmidesse. Ja siis oli füüsika instituudis Riia maantee lõpus, seal oli ka mine niisugune.
Pidipi üksteist äkki.
Ja ma arvan, et umbes tollel ajal kuskil Anne Villems sebis need Apple kahed ka tegelikult siis Vanemuise tänavale uhke, et noh, tooli nagu.
Asi, kuhu ma, kuhu me järgmisena jõudsin peale vaid naistega sõid? Ma arvan, et tuli ka Otto Teller, kes meid sinna viis ja ma isegi tundsin nagu pärast natukene piinlikkust, et, et tema näitas, no täplid. Ja siis ma tegelikult hülgas endale Tähe tänava ja ei käinud enam tema juures, ainult vahtisin seal Apple'it ei oleks käinud sellepärast et ta oli palju nagu ägeda rahast tol ajal oli ikkagi monitor ja, ja tal oli nelikümmend kaheksa kilogrammi sinu elu ja ja ta oli ikkagi nagu ulmeliselt kiire.
Päris päris asju teha jah. Kes mängimine kaaslasi papi peal tuli ülesse teemaks või?
Ja, ja absoluutselt tooli tooli päris hull selles mõttes, et too oli kindlasti minu elu kõige suurem arvutimänguperiood. Et ma oleks peaaegu kahe klassivennaga keemiaeksamile hiljaks jäänud, tavapärast, et noh siis salvestada seisu ei saanud, sa pidid lihtsalt nii kaugele mängima, kus said ja juhuslikult juhtus ta hästi nii minema, et oleks pidanud juba minema, aga tuli järgmine level ja pidid edasi hängima.
Apple'i peal sihuke standartne mäng on nagu Batman mis on mänguautomaati siis igal pool Apple'i peal nimetatud super Pukmeriks. Ja tuli ma siiamaani pean teda kõige lahedamaks mänguks, mida ma olen kunagi mänginud tolle pärast, et toda Pakmenit oli kõigi teiste arvutite peal ka. Aga, aga seal oli nagu mingi katastroof, haalne erinevust olles Algo Hitmis, kuidas tund neli kolli liikusid tolle pärast, et kõigi ülejäänud arvutite peal, nii palju kui mina olen mänginud liikusid rändumiga. Aga Apple'i peal oli neil oma kindel algoritm. Ja tulemuseks oli see, et kui sa ise tegid täpselt ühtemoodi siis tus, situatsioon kordusmänguks mängu ja me tegelikult tööd meil olid välja töötatud esimese kuue leveli jaoks tegelikult sisuliselt algusest lõpuni, et sa teadsid täpselt, kuidas sa terve tolle ekraani puhtaks mängisid ja jäime sellele noh, sealt edasi oli, oli sisuliselt mingisugused paar avangut, mida sai erinevatel serveritel kasutada.
Kuidas toimub põhimõttest Bäckman kui male?
Natukene natukene, absoluutselt selles mõttes ja, ja noh, selle tõttu olnud võimalik teiste peal mängida sellepärast et nad lihtsalt rändama ka liikusid. Et jah, arvutimängud jah, absoluutselt, no seal oli teisi teisi veel, aga super Uponor'i kindlasti.
Ja ja on programmeerimine.
Nojah, muidugi selles mõttes, et seal teisik. Jaa noh, ütleme, et ma alguses ma kirjutasin ikkagi teisikus, aga, aga pärast pärast sai ikkagi valdavalt Assembler kirjutatud tolle pärast, et noh, programm teeb su oluliselt kiiremini kui Assad, et noh, see on tõesti olnud.
No kuidas teisikustessembrisse hüppamine käis, sest Peisikust ma saan aru, et seal see käsud on inglise keeles, eks ole see korrektsele luks lihtsasti üles. Aga selleks sa pead teadma ikkagi väga täpselt, mida sa teed ja miks sa nii töötavad. Harilik protsessor, arhitektuurist ja nii edasi.
Noh, kapeisiku puhul sa pidid ikkagi tollest arvutiarhitektuurist aru saama, et,
Et noh, kust olles neljakümne kaheksas kilobaitides nüüd paiknes ekraan tekstiekraan, kus paiknes graafiline ekraan, kus paiknes see programm, kus oli opsüsteem et, et tegelikult talle arvutiarhitektuuris arusaamine tekkis tulle peesiku kõrvalt ka suhteliselt kiiresti.
Ja aga, aga noh, too Assembler tuli ikkagi tänu sellele, et et osad asjad olid väga aeglased. Et üks asi, mida ma seal tegin, oli orienteerumisneljapäevakute protokollid.
Tollega alustas tegelikult Peep Paabel, kes rakendusmatemaatikat ülikoolis õppida, aga ta lõpetas ülikooli ja siis ta andis mulle tolle kogudule programmi komplekti üle aga tuli minu jaoks liiga aeglane. Andmemaht oli tol ajal neljakümne kaheksa kilomeetri jaoks natuke liiga suur, seal oli ka mitmeflopiga mängimist, et, et need andmebaasid ära mahuksid ja mõtlesin, et teen, ma kirjutan from spetsiifika sõbraks. Ja siis ma kirjutasingi kõik Assembler start oli kõik palju kiirem vahe.
Ja noh, peale toda sai veel siis, siis sai kogu too opsüsteem tegelikult Ribes Insineeritud, tissassembleerituda sellega, et kogu too täna kommenteeritud siis sai imestatud, kus päris mitmes kohas Steve Wozniak oli hämmastavaid trikke teinud talle talle opsüsteemi kirjutamise juures, et nagu noh, tolleks ajaks kui hakkasin Tizzasolveerima, siis siis ise ka juba arvasid nõutud Assembler tean nagu väga hästi. Aga siis ikkagi paar sellist asja, mida avastasid, et Vao, kuidas teha saab, et et noh, nagu niisugune pisut Hipp Trek tegelikult. Et noh, et Assembler siis lihtsalt olid ühe poidilised, kahe maitside, kolme pallised käsud.
Ja see trikk oli see, et et ühte kolme potist käsk oli võimalik kasutada siis selliselt, et kui sul programmi oksis otsele, niisiis ta kolme baitine käske ei teinud midagi, aga sa said tolle kolme baidi viimast kahte potti kasutada selliselt, et saab kuskilt eespoolt hüppasid tolle teise baidi peale, mis oli siis teine Command, et sa sinna kolme baidise käsu viimasesse kahte Balti paigutasid tegelikult teise Assembleri käsu.
Et siis üks elegantsed tekkele tehtud ja siis pärast püüdsid ise ka nagu mõnes kohas mõelda, et kas ma saadada nagu efektiivselt kasutada.
Kust, kust see teadmine tuli, need nii teha saab, et vett saab, lisab, et saab lahti võtta selle noh, mingi teadmine pidi kuskilt jällegi lekkima, eks.
Ei, no kui me tolle koodina kudissassembleerisime, noh siis, siis sa pidid kogu tollest algoritmist arusaamad, mis, mismoodi ühtset töötab. No tegelikult oli, ta on opsust noh, mitte nüüd bioss selles mõttes, vaid vaid opsüsteem, et kettaga suhtluskett nagu suhteliselt aeglane. Ma tahtsin seda kiiremaks saada, mu püütud, siit taim oli seal üks asi, mis, mis välja mängisid lõppes sellega, et ma tegelikult kirjutasin Assembler ise nagu ketaste kopeerimise programmi, mis töötas siis nagu tollest opsüsteemist nagu mingi kümme korda kiiremini. Kümme kurd, noh sa pidid arvestab optimeerima, lihtsalt tuleb pealiikumised, kui sa tahtsid kogu ketta ära kopeerida, siis, siis too sa pidid, ma ei mäleta, kas seestpoolt väljapoole või väljaspoolt sissepoole sõitma, selleks et siis ta siis ta tegi ühe liikumisega kui kirjutamise ära, mitte ei käinud edasi-tagasi. Et muidu standardselt käski edasi-tagasi alatia.
Jällegi, et kust sa üldse niisugune arusaam, et kettaseadmega saab nii üksi trikke teha, et nagu võiks ette võtta siukse asja seda teadmist ja julgust, niisugust pealehakkamist, pluss natuke ekstravagantse ka, et nagu, mis see vastus ikka teab, kuidas ketast kopeerida?
Noh, noh, too opsüsteem on ikkagi universaalne tehtud, et, et selles mõttes see ketta kopeerimist programm sai tehtud nagu Detiteedid siis optimeerituna mingisuguse konkreetse asjaaegselt noh tol ajal oli oluline, et noh, et mingisugune kuskilt kas või Moskvast mingisugune tüüp tuli, tal oli mingeid kettaid, kassi pidid kiiresti suutma kopeelemmitse sai Jokute seal nagu tund aega kopeerida, vaid et sa saad nagu kiirelt endale ära tõmmata näoga Süüria rohkme munade, kust, kust sa neid programme saab Internetti jõudnud, et too liikus ikkagi nagu ma käisin isegi tegelikult koos ühe klassivennaga korra Moskvas puhtalt sellepärast, et et mingisuguseid arvutimänge saada Su.
Moskva suurlinn, kus huviline
Ei, lihtsalt vend käis ise Tartu Ülikoolis ja, ja me saime ta ju kokku, tal oli mingit rõmmyndokupeedisinud näha ja siis me saime temaga kontakti ja siis ta ütles, et noh, et ta, et umbes, et Moskvasse siis alati, et very welcome. Ja, ja siin me nüüd lihtsalt läksimegi.
On ju rongiga?
Kuid mitte, mis too andmeside kiirus siis tuleb, kui arvestada, et sa sõidad rongiga sinna, siis kupeedia flopid ära jäetud ära, siis see tuleb.
Ma ei julge öelda, mis kolmsada kuuskümmend kilobaiti oli üks ketas või? Nii-öelda päris päris soolikaid, ütles, et, aga, aga eks ta oli vast kõige kiirem viis ikkagi.
Et.
Iimil tuli ikkagi mingisugused aastat hiljem ja too käis ikkagi kord päevas, helistasid modemiga sisse ja tõmbasid meilida.
Aga ühel hetkel sai keskkool otsa siis oleks siit õppima midagi.
Tartu Ülikooli saatikat sõjaväkke võtma. Õnnestus ära viilida, noh, selge. Et Tartu Ülikooli rakendusmatemaatikat, aga tollest õppimisest tegelikult palju välja ei tulnud, tuleb jah, ma istusin ikkagi see lätete juures edasi, nii, nii nagu.
Kooli ajal, et.
Jah esimese kursuse ma.
Tegelikult tegin ära kõik matemaatikaeksamid olid viied, aga aga inglise keele ka kukkusin välja. Kuna hõbemedaliga lõpetasid siis siis sisseastumine uuesti väga lihtne pidi matemaatikaeksamitega, mis minule oli triviaalne. Aga, aga noh, siis ma enam ei viitsinud üldse loengutesse minna, sellepärast et noh, kõik matemaatika eksami tehtud, me oleks pidanud Ants inglise keelega seal esimese kursusega.
Ja siis.
Siis ma istusin seal äplitud, aga nüüd ma tegin loomakasvatuse ja veterinaariainstituudile mingisuguse dolla direktor olnud kolonn, tegi, tegi doktoritööd ja ja tal oli terve bussitäie tädisid, kes olid valmis andmeid sisestama, annan talle ei olnud kuhu neid andmeid sisestada, mis valmis arvuti. Ja siis ma tegin talle talle programmi, mis, mis nüüd on meil võimalus sisestada. Noh, seal oli siis oluline, kui uus NTFS teha, selline, et tädid eksida ei saaks kuidagi. Et noh, tooli kõige keerulisem kindlasti teha.
Ja noh, torudsus oli lihtne tegelikult.
See on vist midagi arvates seal mingisuguseid mingi statistikat lihtsalt.
Nad olid mingid piimaproovid kus siis laktoosi, valgu, igasuguseid hulk kahte eestikud ja noh, ma ei mäleta seal mingit korrelatsioonianalüüsi, tuli teha mingisuguseid noh, tema ütles ikkagi olnud algoritmid ette, mida tuleb teha selles mõttes, et oma või siis on matemaatiliselt nõu anda, aga aga üldiselt ta ikkagi teadis ise mida ta tegi. Mis tähendab seda, et siin kuskil palgal siis juba.
Ma olin poole kohaga palgal Tartu Ülikoolis, jah, seal arvutiklassis ma küll insenerina ja, ja piimaga see loomakasvatuse Veterinaaria Instituut. Kuna kuna ta mulle kuidagi nagu ühekordselt maksta ei saanud, siis mind võeti sinna tööle. Aga ma ei käinud seal kunagi lihtsalt selles mõttes, et ma olin seal mingi aasta või, või, või kaks olin tööl lihtsalt selleks, et saada nii-öelda tule programmi eest tasu, siis ma ei viitsinud palka ka minna välja võtma, noh siis pangakontosid eraldi. Ehk siis siis ta direktor tuli mulle tagajärgi ja viis mind sinna sellepärast, et ta ei olnud kassapidaja kisa ära kuulata.
Tuldi autoga järgi riigi raha saama. Täpselt noh, programmeerija magus elu.
Jah ei too oma vastuses, et direktor valdkonnad oli väga-väga nagu lõbus sell, et.
Et nende oma inimestega, ta oli hirmus kuhi. Alati kui me sinna läksime, siis ta kõigepealt sõimas kõigil näo täis, aga aga, aga väga ettevõtlik tüüp selles mõttes, et ma mäletan kunagi ma olin kodus isaga saunas. Siis ema tuli sõnul, et kui on mingi mees, tuli. Ja noh, sama olgu need siis tuli, tal oli midagi kiirelt vaja. Ja mul ema ütles, et ta ka nagu enam-vähem minna tutvust ta Montreali uksest sisse astunud ja läinud kohe elutuppa ja maha istunud. Teie eelarve vaadata, et kui on probleem, et vahet ei ole, kus, kus ma Alementaator.
Et noh, selles mõttes väga sihikindel
Aga see, et sa nagu loengutesse jõudnud, siis mingi asi võltsimisele arvutite juures kinni. See oligi see Assembler ja pusimise huvi või, või mis sa siin-seal võidis?
Nojah, selles mõttes, et ma tegin Assembler, siis ma kirjutasin tekstiredaktori.
Kuhu sai ikka päris ohtralt igasuguseid kitsesid. Tehtud too oli kindlasti kõige-kõige nagu keerulisem masin, mul peaks vist isegi too paberi peal väljatrükituna kood alles olema, tuli.
Kas kas kas viis tuhat või kuus tuhat, keda Assembler sa seda üldse nii palju, et üksjagu eco asembri koodi mõttes on seda palju, aga arvestada, et nagu tekstiredaktor viie tuhande reaga pole paha.
Et jah.
No maitsesin seal ühel tütarlapsel, kes mulle väga meeldis, hoidsin tal ka kursusetöid teha ja tuleks selle tekstiredaktorit nagu vaja. Et läheb ainult muidu, muidu oleks pidanud kirjutusmasinal trükkima. Et noh, et, aga noh, arvutis ühtegi kohalikku tekstiredaktorit ei olnud noh, oleks ka saanud üht või teistviisi teha seal mingisuguseid hädiseid, asjad olid aga aga noh, selleks, et kõik suured-väiksed, tähed, sellised asjad, noh.
Ei olnud lahenduste, siis ma keetsin.
See.
Jaan Tallinn kirjutas ka Prangli endale ühe esimese asjana kirjutatud tekstide tahtma.
Ka seminarist mis tekitab mõte, et kas see tähendab siis seda, et igasugune kuramuse interneedus ja muud niisugused asjad on teinud nagu hoopis karuteene. Et varsti, kui sa tahtsid niux tekstide traktorist ise kirjutama ja nüüd võtavad Internetist täpselt sellise nagu vaja võtta Tii või noh, mis iganes.
Noh, eks ta siis oli ka natukene lihtsalt see, et sul ei olnud neid programme kuskilt saada, eks, eks Ameerikas olid Apple'i jaoks ilmselt kõik programmid olemas. Aga, aga nad ei olnud lihtsalt Eestisse ja eksis.
Töö olnud ja siis tegi teiseneks.
Noh, aega ka oli ja.
Mis sul see tol ajal sinu ettekujutus oli? Et kuhu see kõik nagu viibekas istubki, nagu järgmised kakskümmend aastat nagu Apple'i Apple kahtede juures Vanemuise tänavas või?
Mul mul ei olnud mingisugust, väga konkreetset plaani küll ausalt öeldes, kuhu see viib, selles mõttes, et,
Noh, siis siis tulid, on mingisugune meilinduse käimapanek seal Vanemuise tänavas ja noh, ta oli kuskil üheksakümnes aasta siis tolle pärast, et siis siis Taavi Talvik kutsus mind Postimehe toimetusse.
Sinna ta oli mingisuguses kuu juuniks valmis pannud ja mingisuguse hulga terminale, mille kaudu siis sada ajakirjanikku artikleid sisestasid. Emaks oli, vist on tekstiredaktor saa. Ja, ja eesmärk oli siis teha eesti keele õigekirjakontrolli programm soo sinna peale ja tollega ma siis seal tegelesin, ühesugused läksid sealt loengus, tänavalt, Postimehesse jah. No no ma käisin seal Vanemuise tänaval ta ikkagi tolle pärast, et et no tädikesed, kes arvutiklassi seal nagu haldasid. Tehniliselt liiga võimekad ei olnud ja, ja noh, siis oli kõige kasulikuma sel õhtul läbi käisin ja meelega mingite asjadega nõu andsin, aga aga jah, siis tuli ikkagi Postimees nagu.
Mullu Taimsoo.
Ja see jällegi, et see eesti keeles Belleri või õigekirjakontrolli tegemine ei ole nagu triviaalne asi seal keelest ka ei ole, ei ole triviaalne asi.
Ja ja siis siis ma avastasingi, kui neetult keeruline see eesti keel ainult nagunii ka, et iga teine sõna veel mingi erand olevat. Väga tüütu oli, et ega me teda valmis ei saanud.
Tegelikult et me ei saanud teda valmis. Jah.
Sest umbes üheksakümne kolmandal aastal kuskil hakkas tekkima filosoftide niisugused asjad, et nad tegid vöödilise eesti keele spellerit. Ja see oli ka ikka päris suur tükk pusimist ja seal oli selleks ajaks riismeid sedasi läinud arvutusvõimsus ka, eks ole.
Jajah absoluutselt aga, aga noh, üheksakümmend kolm oli juba see aeg, kui ma tulin.
Härra Tallinnasse Anna kohta.
Sa tulidki otse Postimehest.
Et seal oli mingisugune lühikene periood Postimehe ja Hansapanga vahel ka tegelikult kus ma olin mingisuguse sulgi Birmas
Saan ma ka kirjutasin mingit programmifaile. Aga, aga ta oli nagu vägagi selline kaootiline koht selles mõttes, et too bisnis läks tollel hulgifirmal nagu hirmus hästi ja iga kuunud kolm kutti, kes, kes talle omanikud olid, ostsid, ostsid igaüks endale uue BMW. Tolmutasin Tõndega ümber tolle maja sõita, et et ei olnud liiga motiveeriv keskkond. Tegelikult.
Eks neid kahjuks, mis sa teed? Kuidas, kuidas see kuidas see sinna Postimehe kutsumine ees pidid siis tolle Taaviga kuidagi tuttav olema, kui kus tahes üles leidis.
Ega ma nüüd ei julge öelda, ausalt öeldes peast, kus, kus ma Taaviga tegelikult tuttavaks sain selles mõttes, et.
Sel ajal, kui mina äplite taga istusin, istus Taavi tegelikult sealsamas Tähe neli kus ma esimest korda nairidega kokku puutusin, istus Tähe neli keldris, kus oli mingisugune IBM PC.
Mugi.
Ja kas Taavi tegi midagi äkki Tartu Ülikooli Raamatukogule ja mina olin ka tolle kuidagi seotud ja kas me äkki seal Tartu Ülikooli raamatukogus tissid, aga saime kuidagi kokku?
Jaa.
Ja ja siis noh, vanetsi näplethaavile ja dominets mulle toda PC-d. Ja siis oli sihukene mände King's kvast ja ja siin me Taaviga mängisimetada Kings kosti seal Tähe tänaval
Ja noh, sealt me tuttavaks saime me tolle King's mästyle
Kings, kes on ju King's Quest, on ju metsakas. Just.
Et tollega tollega läks ikka aega, et lõpuni mängida, et me istusime ikka palju.
Niisiis ühel hetkel abi oli see Postimehes ja siis tal oli abi vaja, siis ta kutsus sind täpselt. Aga vaat nüüd seejuures Hansapanka sattusid. See on huvitav lugu.
Hansapanka ma sattusin suutnud aduda vedelikule. Taavi Talvik ütles valitsussides mahvan tol ajal. Ja mitra lingist Rainer Nõlvak. Mahvan oli see, kes küsis Taavi käest, et et Tõnis Sildmäe otsib kedagi, kes juuliks tunneks.
Panka.
Ta ütles, et tema küll ei taha minna. Ja küsis minu käest. Ja mõtlesin, et ah suva, et ma võin ju rääkida ja kuulata, et mis, mis seal siis. Deemon. Ma tulin Tallinnasse Tõnis silmaga rääkima, Sildmäe küll jättis mulje, et tal on terve bussitäis juunitsi mehi ukse taga. Jah, ega kass, keda kõiki integreerub, aga, aga vist tegelikult selle minu ühtegi ei olnudki. Igatahes igatahes ma sain nagu sinna tööle.
Jaa.
Ja siis tus, kool, juuniks oli sinna juba ära installitud ja, ja Tarmo Pajumets püüdis päädistada vaakled sinna skoobiale, inste skoori UNIXi peale installida. Aga ega nad tulles skoori unistust ka midagi ei teadnud, nii et esimese päeva lõunaks nad mõlemad läksid sealt konsoolid ära, täna ja rohkem sinna tagasi ei tulnud, et kui nad vaatasin, et ma, ma vist tean natuke rohkem.
Noore inimese hästi aru, kust nende teadmiste piirid on. Aga tol hetkel oli pank kui selline oli ju juba olemas. Ja, ja muidugi noh, mille peale käis nüüd, mis infosüsteem oli Oracle'i talle seal kaks paigaldama, siin?
Too käis paradoksi peal. Paradoks on siis oli paha, tooks andmebaas, aga aga eks too Oracle'i andmebaasi majja toomine oli nagu üks väga paljudest. Maa on nagu Hansapanga. Edu aluseks olevates strateegilistes otsusest, kuidas tollased juhid suutsid ette näha, vähem tooks, töötas tol hetkel täiesti normaalselt. Ei olnud häda midagi. Aga, aga juba oli Tõnis Sildmäe nagu välja raadio tegelikult me peaksime nagu mingisuguse tõsisema andmebaasi mootori sinna alla panema. Et esialgu tulekski novelli peal toovacle. Ja aga siis me saime tule Skujunitsiverda nii kaugele, et tsime, leidsime, turnisime skoojunud.
Ja vaat räägingi sellest korra lähemalt, et see tahab niukest. Ühesõnaga see tähendab seda, et kellegi teise peas siis ilmselt oli arusaam sellest, et ummik, arhitektuursed, otsuseid, info, arhitektuursed, otsused on kuidagi seotud nagu äriga või nagu äriedu aluseks. Üheksakümnete alguses see ei kõla nagu üldse, nagu triviaalne teadmine. Kusjuures oli.
No ma usun, et et too seltskond, kes seal oli tol ajal, oli ikkagi ka selge arusaamine tolle paradoksi tehnoloogilistest piirangutest ja samal ajal oli oli ka arusaamine, kuhu poole see pank liigub. Ehk siis ma arvan, et tolleks hetkeks, kui mina sinna tulin, äkki oli, oli seal minus kõrvallaua peal juba esimene sularahaautomaat tegelikult oli niisugune suhteliselt pisikene, mis mahtus laua peale IBMi oma kahte eile käinud sisse, vaid tuli magnetiga lihtsalt läbi tõmmata. Et noh, ma arvan, et Aadeeemmide asi oli üks ka, mis, mis nagu tolle paradoksi andmebaasi piirangud välja tõi noh, samamoodi kuna klientide arv kasvas plahvatuslikult noh, ilmselt ka tolle pealt nähti, et et too paradoks ei suuda tegelikult kui selline kasv jätkub, ära teenindada.
Teine asi, mis mind on ikka huvitanud, on see, et, et samal ajal toimetati päris mitmes pangas ka nagu valmis softiga vastati lihtsalt kuskilt Briti maad, mingisugune pangas oht ja tehti panka, tuleb, miks, miks, miks Hansa teistmoodi?
Ei oska öelda selles mõttes, et vabalt võib ka olla see, et et too seda oskavad need öelda, kes päris algusest olid tolle pärast, et etas. Kuid kuidas ta, kas ta.
Kuidas ta spinn Development sinna Hansapanka tuli? Et noh, nimi dispinda hakanud, et ilmselt oli seal siis mingeid Developer häid. Ja ilmselt ta esialgne ülesanne, mida teha tuli, oligi mingisugune väike tükikene ja kui oleks hästi, siis sealt hakkas asi arenema. Ma, ma ei oska öelda.
Udud sprindi lood, need on mingisugune Greriti algus.
Jah, selles mõttes, et ole, kui minagi tööle läksin, siis esimene palga maksta oli tegelikult spinn Development siis Sistus pin Development minu meelest nimetati lihtsalt grebitics ringi. Ja, ja mingi aeg hiljem siis ma saan aru, Londoni kindlustusfirma ütles pangale, et kuulge, et teine, kuid, ja IT-st mitte midagi, et kogu asi on nagu väljas mingisugusest täiesti iseseisvad ettevõtted, et kuidas ta nagu Magnitski ei huvita ennast lõppes sellega, et et Hansapank ostis siis Tõnise käest, need rebiti aktsiad ära ja kõik me tulime siis Hansapanka tööle, et Nabala Kerbitit jäi siis noh, ta juriidiline keha jäi alles ja kuni tänase päevani siis praeguseks swedbank Support OÜ nime all olemas.
Huvitav on see, et see kultuur oli ikka jätkuvalt nagu kleebiti oma, sest kui mina tulin pangast ära aastal kaks tuhat kaks ma pakun, siis mina sain viimase särgi, mis mulle väljastas, oli kleebiti looga. See oli oi kui jube elujõu vesi.
No kindlasti oli selles mõttes, et on tega ega ta noh, mitte ainult mitte ainult Kebit, vaid toob pank ise tervikuna oli tegelikult äärmiselt elujõuline.
Et noh
Ütleme, vähemalt kuni tolle hetkeni, kui kui Hansapank Hoiupangaga liituti et siis toimus ikkagi suundub tundeline muutus, esimene soojem kultuuridanud tuli lihtsalt väga palju teisi inimesi juurde.
No, mis ta ta kultuuri nagu püsti hoidis, kust, kust see tuli?
Ma ma olen Duda mõelnud, et ma ei tea, ilmselt ühelt poolt ilmselt oli kõigile inimestele, kes seal olid, oli ikkagi väga selge saavutusvajadus, et ikka oma asja väga hästi teha. Sellepärast, et seal isegi ei pidanud minu meelest need, kes võib-olla ei performinud piisavalt hästi lahti laskma, vaid nad läksid ise ära. Mil puhul, et, et noh, seesama et näed, kui ma ütlesin, et Pajumets seda oleks installis. Tegelikult oli üks mees seal veel kõrval kes tolles kuu juunis sinna Collins, Tallinn tegelikult kui ma tulin sinna siis ma saan aru, et too mees ise läks paari nädala pärast ära, tegelikult teda ei lasknud keegi lahti, et ta isegi ütles lahkumisavalduse oleks hea, sellepärast et ta sai aru, et, et noh, et noh, temal ei ole midagi teha selleni, et tatud Asko püüdis kuidagi midagi teha, aga kuskil võib-olla. Ja noh, tuli nagu absoluutselt kõigile olid ühesugune kultuur, et sai, sai seal pidanud kaks korda kellelegi ütlema, sa teadsid, et asi on tehtud.
Jaa. Üks asi, mis mind on, kui ma jutu puhul on veel huvitav, et alustasid pihta siis võid kirjutada Selveris PIN koodi, see on ju puhas nagu arendaja. Aga vallas sa läksid kuidagi kohe selle asja nii nagu opereerimise peale. Kuidas see toimus? Ja miks oskad sa?
Ja ega seal mingisugust nagu teadlikku valikut väga ei olnud, selles mõttes, et ta töö tundus huvitav aga Mogakla baasi olnud varem näinud ja selles mõttes, et noh
Ma ma kuidagi ei mõelnud, et ma olen programmeerija selles mõttes, et noh, tegelikult ju ka seal arvutiklassi säplite juures noh, ta tööülesanne oli tegelikult kõigi nende inimeste Assisteerimine, üllad, hoid, et nood arvutid töötaksid, et probleemid nende arvutitega oleksid lahendatud. Programmeerimine oli puhtalt nagu hobi tolle töö kõrval. Kui kuigi noh, kogu asi algas loomulikult programmeerimisdetailide programmeerime Apple'it meeldima. Et aga kuidagi ei mõelnud. Ja ma arvan, et tollal ka ju liiga palju konteksti, et need on arendajad ja need on ülal hoidnud hiljem et, et ma arvan, et, et ta tuli hiljem, et et siis kõik lihtsalt
Sest siis, kui siis kui mina uksest sisse saabusin, siis oli see juba ammu olemas.
Aga ana ja Saldseks muidugi.
Aga kuidas sulle too kuue uniks sai külge juba Postimehes kujunes see testimine ja Oracle'i, kuidas see sulle külge sai?
No see saigi külge sealt Hansapangast, selles mõttes, et toetlejali
Jah, lihtsalt tegema. Ja mitte väga palju aastaid hiljem oli see üks maailma suurimaid oraakli koodibaas, et minu meelest käis mingi praakspetsialisti, mulle meeldis see jutt sellest, kuidas nad seda ei ole kuskil näinud, et kellelgi on sihuke asi tehtud oraakli peal.
Võis olla küll sellepärast, et,
Vilve Vene auto ostja, kes seal kirjutasin, et ettuda Piia Sikk välja, kirjutati seal seal usinasti noh, toda oli tõesti väga võetud, osa oli väga palju noh, kuna too alati fookuses olid ju tehnoloogia vaid ikkagi selles mõttes, et ilmselt ilmselt oli nii teda kõige efektiivsem teha, jääks. No baasi protseduurid kestlik kiiremini kui mingisugune noh, klient-server asi, et.
Ja sul ei tekkinud seal tunnet, et no see, et, et see Oracle'i püstipanekul padjal püsti pandud, no las ta siis nüüd käib Promeeriks parem.
Noh, eks eks programmeerima pidi veel ikka natukene selles mõttes, et,
Skripte tuli kirjutada, mis mis siis kogutud asju üleval, et kas või seesama, et, et kuidas too, kui, kui sa tolle skool juuniksi masina üles poodi, et kuidas tuhat läheb, aeg käima pannakse. Ega tollel Oracle'i installi juures mingeid skripte ei olnud. Ainult et sa kirjutasid ise need skriptid, mis tollebaasi käima panid, vist enese käima panid. Kogu tolle asja tegid kogu päkapike tegemine tolleks pidiskepsid kirjutama, noh, lisaks sellele kõik need Päts protsessid, mis, mis olid tehtud siis kirjutatud kogu nende käivitamine, et oleks tulist skriptid teha. Et noh, nüüd metsa retk sai tegelikult kirjutatud ikkagi päris päris palju.
No see, mis sa kirjeldad, on päris nagu keeruline asi mis peab olema siis nii arenda, noh, arendamise mõttes, et see on mingi peeles kull, mingi kuramuse mindi, siis pätsid ja Oracle'i baasides nende seas see kõik kuidagi nagu terviku moodustama, et see koos püsiks ja oleks nagu tehtud praegu, kuidas te tervik tekkis, kes seda juhtivi?
Kes arhitekt oli?
Ega siis kedagi arhitektiks ei, ei nimetatud, aga.
Noh, ma julgeks siis arvata vast.
Vastu tulles tantsu kontekstis arhitekt oli ikkagi Vilve Vene. No vähemalt ütleme mulle selline mulje jäänud, et tema oli, tema oli siis tänapäeva mõistes arhitekti keegi selliste terminitega.
Ega tänapäevalgi väga-väga lihtne kasutada.
Et see kontseptsioon, kuidas, kui too kõik kokku töötab tarkvaraliselt, et ma arvan, et ta tuli ikkagi ennekõike ilmelt mingisugune ütleme, non saksa heli laiem võtsid need tekitasin mina. Et NATO samas skriptide asi, et noh, et iga too Siibäam, mis pingid Pätsi tegi, et too ei oleks erinev, et on kuidagi ära standardiseerida, siis ma pidin mingisugused, et mitte funktsionaalsed nõuded esitama, et et nad kõik oleksid ühetaolised, saaksin kasutada mingit ühte skripti paljude asjade käivitamisest.
Ja see on see jõud või mitte, funks aasta nõuet juurde sealt järgmine asi, et kuidas noh, Postimehes on ka ikkagi suur ajaleht, aga see ajaleht ilmub ka siis, kui need ajakirjanikud oma trükimasinaga kirjutavad, oma lood valmis. Aga pank trükimasina peal nagu enam ei käi. Mis tähendas seda, et ühel hetkel see naguniisugune pusin ise ja vaatan, kuidas mossel teinud pidi nagu maad andma sellele, et on mingisugune struktuur ja mingisugune niisugune formaalsemisel kuidas, kuidas see sündis, kas see oli mingisugune otsus, et nüüd hakkame nagu korralikuks või see sündis kuidagi sujuvalt?
Noh, tolle tolle igapäevase ülalhoiu kõrvalt sa pidid tegelikult ikkagi paratamatult noh, igaüks kuna kuna ta kasv oli nii kiire, siis igaüks pidi tegelikult vaatama, aga mitte ainult seda, kuidas see asi täna ära rullib, vaid ka, milline see asi nagu aasta pärast välja näeks. Ja, ja noh, kindlasti Tõnis silma ka Fassilteeris seda, et, et tuleksid igasugused erinevad kontaktid kes, kes mingisuguseid uusi lahendusi pakuksid ja, ja nendega sai, sai konsulteeritud ja äkitudeni. Nii need asjad arenesid edasi ka selles mõttes, et noh Cuzco juuniks ju samamoodi tollel tuli tehnilised piirangud ette.
Üheksakümmend kuus, üheksakümmend seitse kuskil sai ju happe uksi vastu välja vahetatud, enne toda sai, oli mul nii HP kui Sammy serblase laua peal ja sai võrreldud siis kumb, kumb nagu kiirem on noh, tolleks ajaks oli oli ta panka piisavalt suured, siis, siis oli selge, et, et me tegelikult ei pane ühte masinat, vaid me paneme klastri. Sain nende Vendoritega nad klastri lahendused läbi räägitud.
Jah.
Et ega, ega sellist, nagu et mingisugusel hetkel oleks mingisugune Mäppe toime, Tennoli on Ahja, siis, siis tehti kõik asjad korda. Et kõik, kõik see arenes tegelikult ikkagi evolutsiooniliselt, neid asju on lahendusi vahetati iga aasta välja, tegelikult tuttav on vastu tolle pärast, et testi ei oleks lihtsalt toda. Kümme aastat kestnud olukorda, kus, kus iga, ma ei tea, üheksa kuud on olnud kahekordsus, klientide arv.
Käive kasum mis iganes. Kõik numbrid kahekordsest kümme.
Jah, ega last kenasti sellist kasvu ei kujuta tänapäeval nagu väga ette enam kui sa just kuskil Skype moodi asja sõidelt.
Nojah, ega, ega nüüd ettevõtted ongi maailmas väga vähe vist, kes, kes nii kiiresti nii pikalt suudavad kasvada, noh, oli, oli mingisugune substes of Estoniat.
Ma mäletan, aastal just sajandi lõpuks oli sinna panka tekkinud mingisugune niisugune üsna ike spetsialiseeritud tiim nagu inimesi, kes opereeris seda kupatust seal. Kuidas te tiim tekkis, kes igalühel oli oma mingisugune valdkond, millega tegeles vahetanud ja kolmekesi moodustasid te niisuguse asja, millest Veeber seisneb kogu maailm püsti. Et kuidas tuli, kuidas ta kolmik tekkis.
Tohutu tekkis ka aja jooksul selles mõttes, et noh, Madis oli, saar oli enne mind olemas ja on olemas, muutmist, ma siis oli, seal on praegu ka olemas. Et.
Ma ei teagi päris täpselt, mis tema roll päris alguses oli, aga aga ikka ikkagi sel ajal, kui kui mina seal tolle Orkla Masingu toimetama hakkasin, siis minu asi oli nagu toit, tehniline pooletu, andmebaasi ilmsin, töötaks ja Madise asi oli siis luua sinna uusi tabeleid ja teha indekseid ja vaadata, et päringut hästi käivad nii-öelda see tagasi. Ja, ja noh, too, too hall jätkus tal edasi. Toomas Soomets tuli.
Ma pidin peaaegu ütlema, et ta tuli koos Hoiupangaga liitumisega ka tegelikult ei tulnud. Tegelikult tuli kaks aastat enne seda, kui ta töötas Hoiupangas, aga ta tuli kaks aastat enne seda, kui juba kärastati. Et, et tolleks ajaks, kui avastati, istus teha õigel pool lauda juba.
Et ja, ja noh, toomas toomas siis oli, oli selles mõttes nagu täijendastuda seltskonda, et kui Madis oli nagu kõige üle meie nii-öelda tahta, leiab mina, teadsin toda köögiandmebaasi Encini osa siis siis Toomas oli see mees, kes, kes nagu met vöögistest olid, siis täiesti jagas. Noh, siis siis too Kukkondiski nagu kogu tuletehnoloogilisest äkki, et põhjest tiim töötab hästi. Noh, mõistus minu arust toas ühes infoväljas kogu aeg alati on võimalik nagu öelda, mis toimub.
Kas sul oli juba tol ajal, ma tean, sul on Vaarmani huvi? No see oli juba tol ajal olemas. Sest ma mäletan, et kapi otsas oli makk ja sealt tuli aeg-ajalt tuli niukest eepilist klassikast muusikat.
Jajah see oli, seda ma ei oska öelda. Jää ei.
To klassic, jah, ausalt öeldes ma isegi päris A aastaarvu jälle julge öelda tolle pärast, et ei olnud veel omas on, et esimesed siin viibima Internetist ostsin tolle firma nimelist siidi, no punkt kommu. Et ja siis sai tunda klassikalist muusikat, seal mängitud? Jah, mitte küll vaagne, et põhiliselt tegelikult ma julgeks arvata Mozartit tol ajal põhiliselt Mozartit. Et jah, ma ostsin ka mingisugused päris alguses, ma ostsin mingisuguse siniHenrikuga Ruusa plaadid. Aga, aga kas ma ostsin Mozartit ja siis me mängisime seal meid neto ka kuidagi ise kanda? Me, eks me tegime erinevaid asju. Mingi periood oli, kus, kus välja tuldi.
Öeldi, et.
Teatud lõhnad on teie toas igapäevaselt tunda et.
Mingi periood oli tõesti, kus, kus meil oli alati konjakipudel kapis ja ja päeva sai alustatud difitsiga konjakiga. Et loomulikult mingit joomist ei olnud, aga, aga noh, eks tollest ühest pitsist juba noh, sul oli ka klaaslaua peal, võib olla kuni lõunani seal, et ega keegi ei joonud, aga lihtsalt natukene. Ja WRC ralli oli ka, mille Toomas siis püsti pandi, esindatud on FC hallit, mängisime ka seal mingisugune periood ikkagi, et noh, jälle, et ta tahtis väga palju network. Aga kuna Toomas selline löögi põhjal siis siis siis ta bänd vist kellelgi oli siis meie toas ennekõike.
Või akustilise klassiku pidule.
Klassiku huvi tuli sealt Enrico kohustus tegelikult, et mul oli vanematel kodus, oli Vituaalse salto hälli raamat Totu välja oli vist kaugelt sugulane ja Mehhikoga osales, ta kirjutas nagu Henrikuga ruudust raamatu noh loomulikult itaallane ja kuna ta oli sugulane veel, noh, siis ta on ülimad ülistav, aga, aga ta oli huvitav lugeda ja jättis väga sügava mulje. Ja siis, kui internet siis oli võimalik tellida, siis ma tellisin huvi pärast tule siit-sealt Sindi nafta, Edgar Ruusa plaate. Ja teine asi, mis häiris, oli ikkagi kaheksakümne viienda aasta Milos farmani Amadeus. Mida ma kindlasti soovitan kõigil vaadata, aga suurepärane film. Et Mozart ja sealt tuli ta Mozarti või. Ja noh, kui midagi sellist oleks, siis ma tellisin mingisuguseid raamatuid Mozarti eluloost mingi neli-viis ja mis mul on kodus üle tuhande lehekülje paks, et jah, sealt edasi. Ma ei tea, mis, nagu Beethoven, Schubert, Schumann, Tšaikovski. Mis projektid?
Töötan G4S-i, turvalise Everesti baasteenuste arendusjuht, noh sisuliselt siis vastutan õlahoiu eest. Et et kõik asjad oleksid püsti ja valvatud. Jah, selles mõttes, et noh see ongi, et mitte mitte siis ainult IT, vaid, vaid ka see tehniline valve, kuhu puutub, siis ka see raadiosidevõrk on meil hästi, on et kõik need signaalid jõuaksid siis siia keskele kokku.
Selge see ka huvitav ameti, nagu nagu Viksu need seiklused alates sellest tassemberist seal äppidel.


\chapter{Jaan Priisalu}
\index[ppl]{Priisalu, Jaan}

\ldots ütles eestlaste kohta väga hästi. Et eestlased otsused saunas 
teevad, eks ole, on \emph{no-brainer}. Aga kuidas seda tehakse? 
Inimesed käivad saunas, on paljad, räägivad midagi. Ja kui ära 
lähevad, kõik nagu teavad, mis otsus on. Seda ei hääldatud mitte ühtegi korda 
välja 
ja ei ole aru saada, kes on liider, kes selle \emph{move}-ga välja tuli. 
Sul on pikka aega olnud mingid võõrad sellid peal, 
kelle eesmärk on mingi muu, kui kohaliku rahva eesmärk. Ja sa tead, et võimu 
peale loota ei saa. Aga kui sa teed otsuseid sellise \emph{mode}-ga, et sa oma 
liidreid välja ei näita, on see tegelikult liidrite kaitsmise süsteem. Mis 
muuhulgas tähendab, et me oleme projektirahvas. Kui sa Vabadussõda vaatad, 
siis see oli ka selgelt projekt.

\ldots Mulle on seda küsimust mitu korda esitatud. Ameeriklased tulevad ja 
küsivad, et kui me ringi vaatame, siis need \emph{challenge}'d, mis te 
välja käite, on nagu pooltel maailma riikidel. Aga miks siin välja tuleb?

\bigskip
\noindent\rule{.3\textwidth}{.7pt}
\bigskip

\question{Kuidas sina sattusid arvutite juurde?}

Arvutite juurde sattusin neljandas või viiendas klassis. Meil oli pioneerisalk, kellega tegime igasuguseid asju, ja mind pandi seda juhtima. Salgas oli ka klassivend Kermo 
Jaaksoo\index[ppl]{Jaaksoo, Kermo}, kes pakkus välja, et võiksime 
minna tema isa töö juurde. Ülo\sidenote{
Kermo isa, akadeemik Ülo Jaaksoo\index[ppl]{Jaaksoo, Ülo}} töötas sel ajal 
Estonia puiesteel ja tal oli seal ES-1010\index{ES EVM!ES-1010}\sidenote{1010 oli ES EVMi esimese alaseeria esimene mudel.}. Ega me seal arvutiga midagi muud teha ei 
osanud kui Kuule maandumise mängu mängida. 

Ma käisin 1. keskkoolis\index{Tallinna 1. Keskkool}. Kui lastelt küsitakse, kelleks nad saada tahavad, siis tavaliselt 
öeldakse, et tuletõrjujaks, politseinikuks, autojuhiks. Mu vanemad 
väidavad, et kui minu käest seda esimest korda küsiti, siis mina tahtsin saada
inseneriks. Seepeale otsustasid
vanemad, et poiss tuleks panna matemaatikat ja füüsikat õppima. Isa leidis, et 1. keskkool on õige koht, see 
oli elukohajärgne kool ka --- me elasime vanglahoovis, Suur-Patarei 29. 
Läksime kooli katsetele, tegin katsed ära ja siis direktor küsis isalt, 
miks nad tahavad mind sinna panna. Isa rääkis inseneriloo ära ja ütles, et poisil on vaja 
matemaatikat ja füüsikat teada. Direktor mõtles natuke ja ütles, et see kõik on ju väga tore ja tõepoolest, poiss tegi katsed 
edukalt, aga meil on see häda, et matemaatika-füüsika klass hakkab üheksandast 
klassist, mis ta seni teeb? Seepeale pandi mind prantsuse keelt õppima.

\question{See on ka ilmselt väga tarviline olnud!}

Prantsuse keel on paarikümne aasta pärast
kõige kõneldum keel maailmas. Kui mõelda, kui levinud see on Aafrikas, siis on sellel seal sama funktsioon, mis Indias inglise keelel. Ja 
kui 
vaadata, missugune rahvastikuplahvatus neil on ja palju neil maad käes on, siis 
Aafrikal ei ole India inimeste tiheduse probleemi, vaid paisumisruumi on
palju.

\question{Kas pioneerirühmas kohtusite esimest korda essukesega\sidenote{ES EVM seeria masinate levinud hellitusnimi.}, maandusite 
Kuule ja mis veel?}

Põnev oli vaadata, kuidas lindid käivad ringi. Sel ajal olid juba 
vahetatavad kõvakettad. Nad näitasid meile ka trummelsalvestit, kuigi see 
ei olnud käigus, vaid oli lahti ühendatud.

\question{Mis asutus see oli Estonia puiesteel?}

Üks Teaduste Akadeemia asutus, ma täpselt ei mäleta. Ilmselt 
praegune Küber\index{Küber}, sest Ülo oli kunagi Küberi 
direktor.

\question{Kas tollest ajast jäidki seal oma rühmaga käima?}

Ei, mitte päris. Järgmine kord oli siis, kui meil olid arvutiõppe tunnid 
ja õpetaja Loonde\index[ppl]{Loonde, Jaak} viis meid Pedasse\index{Tallinna Pedagoogikaülikool}. 
Teine arvuti mu elus oli sealne Nairi-2\index{Nairi!Nairi-2}. Nairi on 
transistorite peal arvuti, perfolint on viierealine. Kui 
matemaatika-füüsika klassi läksime, siis Jako Bergson\index[ppl]{Bergson, Jako} tõi 
Kirovi kalurikolhoosist\index{Kirovi Kalurikolhoos} MIR-2\index{MIR-2} 
ära. Mir-2 on tegelikult maailma esimene personaalarvuti, mis oli tehtud inimese aitamiseks. Selle protsessor kaalus pool 
tonni, aga ideoloogia oli selles, et inimene saaks 
oma rehkendused tehtud. Muu hulgas olid integreerimine ja diferentseerimine 
rauas\sidenote{St. riistvaraliselt.} realiseeritud.

\question{Sest inimesel oli ju vaja integreerida ja diferentseerida, mille 
jaoks talle muidu üldse arvuti!}

See oli Ukrainas tehtud arvuti, seal arvutati gaasiturbiine, 
rakette ja muud säärast. Programmil oli 
Algoliga sarnane keel ja see algas käsuga \verb|RAZR|, mis tähendas 
\begin{russian}разрядность\end{russian} ehk kui pikad arvud on. 

\question{Kui pikaks arve sai keerata?}

Me keerasime kas 300 või 400 peale, kümnendkohtades. Panime programmi piid arvutama ja arve ritta ajama ning see lõppes sellega, et 
kuigi oli talveaeg ja me tegime aknad lahti, kuumenes arvuti ikkagi üle. 
Pooleteist tundi saigi sellega tegutseda.

\question{Kust teil selline mõte, et võiks piid arvutada?}

See lihtsalt tundus lahe.

\question{Ja kust te matemaatika saite?}

Mul oli üks paks venekeelne matemaatika õpik või 
entsüklopeedia. Seal olid igasugused read ja pii rida oli 
üks nendest.

\question{Järelikult oli teil koolis tolleks hetkeks matemaatika juba pihta 
hakanud?}

Jah. Meie koolis oli nii, et kui käisid 
matemaatika-füüsika eriklassis, siis esimese asjana jagas õpetaja 
Uudelepp\index[ppl]{Uudelepp, Helgi}\sidenote{Gustav Adolfi 
Gümnaasiumi legendaarne matemaatikaõpetaja Helgi Uudelepp.} klassi pooleks. Pool klassi hakkas õppima tavalise 
keskkooliprogrammi järgi ja ülejäänud kambale anti olümpiaadi ülesandeid. 
Kamp oli väga kõva (näiteks Mati 
Pentus näiteks\index[ppl]{Pentus, Mati}\sidenote{Mati Pentus on Eesti 
matemaatik, alates 2003. aastast Moskva Riikliku Ülikooli professor.}), probleem oli koolist üldse välja jõuda. Rajoonist sai 
niikuinii vabariiklikule edasi.\sidenote{Toonased olümpiaadid olid organiseeritud kooli-, rajooni- ja 
vabariiklikeks olümpiaadideks. Vabariiklikult olümpiaadilt oli võimalik pääseda ka 
üleliidulisele olümpiaadile.}. 

\question{Kas teil oli koolis arvutitund ka ja kasutasite MIR-2?}

Jah. See asus küll kooli spordihoones. 

Selge see, et 1. keskkool seisis direktor Viikholmi\sidenote{Helmi 
Viikholm\index[ppl]{Viikholm, Helmi} oli kooli direktor aastatel 1962---1982.} najal ja kui 
Viikholm läks pensionile, juhtus nagu ikka organisatsioonidega juhtub, et 
hakatakse rootsi keelt või muud sellist õpetama.

\question{Kas selleks ajaks olid sina sealt koolist juba läinud?}

Ma käisin siis veel koolis, kui Viikholm ära läks. Organisatsioon toimis vana energiaga 
veel natuke aega edasi, tavaliselt võtab lagunemine 
kaks-kolm aastat aega.

\question{Kas keskkooli ajal pusisite Jaaguga MIRi peal või oli juba muid 
võimalusi ka?}

Mina sain oma esimese palga progemise eest aastal 1984, 
üks suvi enne keskkooli lõpetamist. Paldiski maantee 1 asus termo- ja 
elektrofüüsika instituut\sidenote{Eesti NSV Teaduste 
Akadeemia Termofüüsika ja Elektrofüüsika Instituut 
(TEFI).\index{Teaduste Akadeemia!Termofüüsika ja Elektrofüüsika Instituut}}, nende käes oli näiteks Arnold Veimeri nimeline laev.
Mina olin akadeemik Krummi\sidenote{Akadeemik Lembit Krumm\index[ppl]{Krumm, Lembit} (1928---2016).} juures, 
kes arvutas elektrivõrkude staatilisi režiime. Neil olid ka 
arvutid ja mina tegin oma esimese töö arvutil 
Iskra-226\index{Iskra!Iskra-226}, mis on sisu poolest Wang 2200 koopia ja millel on muu hulgas videoprotsessor
8080. Üks vend tegeles sellega, et 
pani selle videoprotsessori peale käima CP/Mi. 

\question{Kust nad su leidsid?}

Ma ei mäleta, aga ilmselt tutvuste 
kaudu või siis lihtsalt läksin ise sinna.

\question{Ja seal arvutati elektrivõrke?}

Esimese asjana pidin tegema rehkenduse ühe
ministeeriumi aruandluse jaoks --- tabelarvutuse Basicus\index{BASIC}. Tegin programmi valmis ja nägin esimest korda 
päris \emph{user}'i probleemi ka. Korraldasime piduliku üleandmise, 
komisjon tuli 
kokku ja naisterahvale, kes pidi seda 
programmi kasutama hakkama, öeldi, et istu arvuti taha ja proovi 
midagi toksida. Naine keeldus. Keegi ei saanud aru, 
mis juhtus --- kas programm ei meeldi või milles asi. \enquote{Ei, ma kardan 
elektrit!} vastas tema. Ma ei mäletagi, kuidas see tsirkus lõppes. 
Siis tuli juba järgmine töö. Neil oli programm, millega nad suuri jakobiaane 
arvutasid, ja see käis essukese\index{ES EVM} peal, mis oli vist 1055. Neid 
oli Küberis kaks tükki: üks oli 360 ja teine 370 koopia\sidenote{ES EVMi 
esimene alaseeria oli IBM System/360 ning teine ja kolmas System/370 koopiad. 
Mudel 1055 kuulus teise ja 1066 kolmandasse seeriasse.}. Nende terminal ei olnud 
mitte VT100 nagu mujal maailmas, vaid IBMi VT52, mis näeb üsna 
hüperteksti moodi välja. Kirjeldad ära sisendi ja väljundi väljad ning terve 
ekraan saadetakse korraga, töötab nagu veebileht. Sinna peale 
ma tegin ühe programmi, mis võimaldas sisendandmeid mõistlikult sisestada ja tulemusi vaadata. Kuna essuke oli \emph{patch}-arvuti, siis pidi 
vahele tegema programmi, mis \emph{patch}'i tulemused 
sisse söödaks või välja võtaks ning 
terminaliga suhtleks. Tolle vahejupi tegi vist Tarmo Mere\index[ppl]{Mere, 
Tarmo}. See kõik oli üsna aeglane, ringitõstmine võttis
aega ja käis läbi ketta.

Teine Küberi essuke oli 1066. Selle peal nad andsid mulle ühe
assembleri makro, et paneksin asja käima, aga protsessor oli juba 32bitine. Olin Inteli assemblerit vaadanud, aga no üldse ei olnud 
sarnane. Kõiki registreid sai kõikides funktsioonides kasutada ja kõige 
hullem, millest ma ei saanud aru, oli see, et kõiki aadresse võeti 
baasregistri suhtes. Sealjuures eeldati, et sa lihtsalt 
tead seda. Aga kuidas teha kindlaks, milline on baasregister ja kust see 
algab? Leidsin baasregistri valimise käsu üles, kuid 
midagi oli ikka puudu. Tuli välja, et seal, kust baasregister hakkas aadresse 
lugema, oli vaja lihtsalt kõikide käsuridade ette panna üks 
tärn.

\question{Programmeerimisoskus oli sul järelikult olemas. Kas õppisid seda Jaagult, 
pusisid ise või kust see tuli?}

Jaak\index[ppl]{Loonde, Jaak} õpetas jah, MIR-2 peal me mingit nalja 
tegime. Kui juba piid arvutad, siis peaksid ka paar rida oskama kirjutada. Järgmisena tuli Basic TEFIs. VT52 puhul ma ei mäleta, 
milles ma programmi tegin, võibolla oli Fortran. 

\question{Kuidas VT52 puhul progemine käis, kui ekraanil olid lihtsalt 
mingisugused väljad?}

Ei mäleta, interaktsiooni kirjeldus on selline, et saadad 
terve 
andmepaketi korraga minema ja saad terve andmepaketi korraga tagasi. Andmete 
töötlus ja esitus on eraldatud. 

\question{Naljakaid aparaate on olemas!}

Mäletan, kuidas vennad presenteerisid modemit, millega me Küberisse 
helistasime. 1200boodine modem oli külmkapisuurune. Kui tuli 2400boodine modem, siis see oli tükk maad väiksem, pool külmkappi.

Ja millega vennikesed veel tegelesid?! 1984. aastal olid olümpiamängud LAs. Mängude 
ajal nad jälgisid elektrivõrgu parameetreid, peamiselt sagedust, ja selle põhjal 
ütlesid, palju tootmist seisab ja mitu inimest vaatab olümpiamänge. 

\question{Kas nad ütlesid seda ka ametlikult kellelegi?}

Ei, nad vaatasid oma lõbuks. Neil oli kihlveokontor --- vedasid kihla, palju järgmisel päeval 
vaatajaid on.

\question{Sa olid matemaatikas ja füüsikas tugev, aga arvutiasja 
pidid suuresti ise pusima. Mis sind selle juures tõmbas?}

Äge on see, kui saab oma kätega midagi teha. Üks liik inimesi armastab teooriaid 
välja mõelda, teised asju kokku 
ja käima panna. Täna ma olen 
mõtlemise ja teooria poole peal.

\question{Eks asjad ole maailmas tasakaalus. Nii et tol ajal meeldis sulle 
vajutada arvutiklahve ja arvuti muudkui tegi?}

Automatiseerimine --- see, et asjad ise juhtusid --- oli väga äge. Näiteks et
auto sõidab ise. Sel ajal oli natukenegi targem 
juhtimisalgoritm haruldus.

\question{Ja see hoidiski sind nii palju arvuti taga, et õppisid programmeerima 
ja õigetesse kohtadesse tärne panema?}

Jah.

\question{Kas selle juurde käis ka mõni laiem valdkondlik huvi? Mõni on 
rääkinud, et raamatud, muusika ja muu selline suunas arvutite poole.}

See oli sügav Vene aeg, mis mõttes \enquote{raamatud ja muusika}? 
Loomulikult lugesin ma Asimovit, robotivärgist räägiti seal päris palju. Lugesin kõiki 
raamatuid, mida kätte sain. Neid, mida ei saanud, lugesin 
juba Prantsusmaal, kui läksin Toulouse'i õppima. Ma panin 
kooliminekuga natuke puusse --- kui ma poole septembri pealt kohale läksin, 
polnud koolis veel kedagi. Nii ma siis 
istusin raamatukogus ja lugesin matemaatikat ning Asimovi jutte, et keeleoskust parandada.

\question{Enne kui Toulouse'i juurde jõuame, küsin, kas 
töölkäimine keskkoolis õppimist ei seganud?}

Ei seganud, see oli eluviisi osa, ma olen alati kooli kõrvalt tööl 
käinud.

\question{Mida sa ülikooli õppima läksid?}

Automaatikat, automaatjuhtimissüsteeme tehnikaülikoolis\index{Tallinna 
Tehnikaülikool!Automaatikateaduskond}.

\question{Kuidas see otsus sündis? Oli see loomulik valik?}

Oli küll loomulik valik. Kuulsin oma sugulaselt Jaan 
Võrgult\index[ppl]{Võrk, Jaan}, mis 
see automaatika üldse on. Meie rühm oli väga äge, klassivendi ja -õdesid
oli umbes kümme. 

Gibbs\index[ppl]{Kübbar, Heiki} ütles mulle hiljem, et seda oli 
vastik vaadata --- ise higistad matemaatika- ja füüsikaloengutes, aga 
need vennad tulevad kuskilt, teevad pulli, lähevad eksamile ja saavad 
kõik viied. 

\question{Kas ta oli su kursavend?}

Me oleme kindlasti koos loengus käinud. Ta on aasta noorem, aga ilmselt läksid meil loengud minu sõjaväest tagasitulekuga kuidagi sünki --- ma istusin oma kaks aastat sõjaväes ära.

\question{Kas sind võeti enne ülikooli kroonusse?}

Ülikooli esimeselt kursuselt. Arvasin, et ma ei pea minema, aga tolle aasta viimane võtmine oli detsembris ning mind pandi rongi peale ja läksin.

\question{Kus sa need kaks aastat veetsid?}

Põhikoht oli Rostov Doni ääres sisevägedes ehk siis vangivalvurid. 
Alguses olin mingisuguses isolaatoris, kus oli kolm varianti, mida saab teha 
veest ja hapukapsastest. Esimene oli hapukapsasupp ehk vesi hapukapsastega. 
Teine oli praad ehk hapukapsad ilma veeta. Kolmas oli kissell ehk 
vesi ilma hapukapsasteta. Ma vaatasin, et suren sinna ära, kui pean seal väga pikalt 
olema, ja munsterdasin ennast \emph{utšebka}'sse\sidenote{Õppeväeosa.}, 
natuke luuletasin 
ka. Seepeale saadeti mind Galatši valveseadmete inseneriks õppima. Galatš on see koht, kus Stalingradi kott kokku murti. Nii et ma olen
Volgogradi Venemaa Ema Mamajevi kurgaani peal päriselt lähedalt näinud. 
Hirmus roostes kolakas oli --- kaugelt vaadates ilus, aga lähedale minnes roostes.

\question{Mina küll Vene kroonusse napilt ei jõudnud aga, nohik nagu ma olin, 
kartsin, kuidas ma seal füüsilise koormuse ja keelega hakkama saaksin. Kas sul seda 
hirmu ei olnud?}

See, et üritad ellu jääda, oli igal juhul. Ja selge see, 
et vene keelt koolis ära ei õppinud. Seal aga ei olnud valikut.

\question{Klassikalise haridusega vene proua ju kolme- ja neljatähelisi sõnu
ei õpeta.}

Jah, need on sõnad, mille abil õpid tõepoolest kõiki asju ära ütlema. Seal 
tehti naftast viina. Läksin kord brigaadi töökotta, et juhendada 
järgmist venda, üht Leedu poolakat, kes teadis elektroonikast 
tegelikult rohkem kui mina. Prapporid\sidenote{Praportšik ehk lepinguline 
allohvitser Nõukogude armees.} tulid oma viinapudeliga sinna ja tahtsid, et me selle 
treipingis ära tsentrifuugiksime. Panin pudelile rätiku ümber, treipinki ja 
pöörded peale. Seesama major, kes mind \emph{utšebka}'sse vajas, vaatas kõrvalt 
ja ütles: \enquote{\begin{russian}ты уважай русский язык, ты хот \ldots\ 
скажи!\end{russian}}. See oli päris hull keel, kindlasti mitte tavaline.

\question{Mõned inimesed on rääkinud, et neil oli kolmetäheliste maailmast
keeruline tagasi teadusmaailma tulla. Kas sul seda probleemi ei olnud?}

Kindlasti oli. Sõjaväest tagasi 
tulnud loobiti kõik eraldi kursusele, neid ei lastud puutumatute 
inimestega kokkugi.

\question{Kas sa läksid Tehnikaülikooli tagasi sama asja õppima?}

Jah, sain isegi sama töökoha tagasi, aga siis ühel hetkel läksin
EKTAsse\index{EKTA}\sidenote{Arvutustehnika Erikonstrueerimisbüroo oli 
Eesti NSV Teaduste Akadeemia Küberneetika Instituudi autonoomne osakond}. 
Ektaco\index{Ektaco} on EKTA \emph{spin-off} ja EKTA direktor oli 
Märtin\sidenote{Kaarel Märtin\index[ppl]{Märtin, Kaarel} oli siiski EKTA 
tarkvaraosakonna pealik, tema alluvuses Jaan ilmselt töötaski. EKTA direktor 
oli Kalju Leppik\index[ppl]{Leppik, Kalju}, Ektaco oma Rein 
Haavel\index[ppl]{Haavel, Rein}.}. Ma hakkasin seal FoxPros andmebaase kirjutama.

Üks huvitav kogemus oli käia putši ajal Moskvas. Tegime sealsele 
juveelitehasele 
väärismetallide arvestusprogrammi, aga sel ajal pidi softile autor kaasa 
minema, sest need asjad ei olnud väga töökindlad. Tehase 
osakonna juhataja oli tiba juudi verd. Kui ta kuulis, et 
erakorraline komitee on võimu üle võtnud, ütles ta mulle kohe, et see kõlab 
halvasti, \enquote{vedur teise otsa ja kohe koju tagasi!}. Mina aga olen eluaeg lollustega 
maha saanud või hirmus otse öelnud. Arvasin vastu: 
\enquote{Mis sa jamad, vaata kui hästi teevad
Levitani\sidenote{Juri Borissovitš Levitan oli 
Nõukogude diktor, kelle kanda oli peamiste oluliste uudiste edastamine Teise 
maailmasõja ajal, tema iseloomulikku häält tunti hästi.} järele, 
nagu oleks sõjaajast pärit.} Läksime tänavatele ja need olidki 
BTRe\sidenote{Nõukogude Liidus valmistatud soomustransportöör.} täis ning madin käis. Seal olid 
suured seitsmerealised tänavad, mis olid kõik autodest tühjad. 
Inimesed korjasid sillutisekive ja ehitasid nendest barrikaade. Vennikesed 
hüüdsid mulle veel uhkelt, et vaata, kui kõvad mehed me oleme, ehitame nii kõrgeid barrikaade. Barrikaadid olid aga põlvekõrgused.
Ütlesin neile, et tank T-72 tehnilises 
spetsifikatsioonis on kirjas, et see sõidab 70 kilomeetrit tunnis, kui maapinna 
ebatasasus ei ületa meetrit. 
Juveelimessil rääkis üks korralik proua, kuidas see kõik on nii 
kohutav, ja küsis, mis mina sellest 
arvan. Ma ütlesin, et kõik on ju hästi. Proua imestas: \enquote{Mis mõttes hästi?} 
Ma siis seletasin: \enquote{Vaadake, seni on venelased tapnud kõiki teisi rahvaid, nüüd 
tapavad venelased venelasi.} Populaarsust ma sellega muidugi ei võitnud.

\question{Kas sul igav ei olnud andmebaase treida, tulles matemaatiliselt keeruliste 
asjade juurest? See on ju rutiinne töö.}

Ei olnud, seal oli tegelikult sisendit ja väljundit palju ning pusimist piisavalt, et erinevatele 
inimestele vaated teha. Ja mul olid väga lahedad 
töökaaslased Jüri Freiberg\index[ppl]{Freiberg, Jüri} ja Ülle 
Heinla\index[ppl]{Heinla, Ülle} --- Ahti\index[ppl]{Heinla, Ahti} ema, kes näitas uhkusega poja
tehtud mängu. 

\question{Kaua sa neid andmebaase tegid?}

Ma ei mäleta. Kui ma läksin Ektacosse\index{Ektaco}, tegin seal 
baase edasi. Olin siis just Prantsusmaalt tagasi tulnud ning tegin juba 
niisuguseid baase, kus olid füüsilised asjad ka taga, nagu lukud ja kassad.

\question{Kuidas sa Prantsusmaale sattusid?}

Nõukogude Liidule oli eraldatud 300 stipendiumit ja kui liit 
lagunes, 
siis kuus stipendiumit tuli Eestisse. Kuna sel ajal oli prantsuse keele 
oskajaid suhteliselt vähe, korraldati 
avalik konkurss. TPIst korjati ka inimesi ja sealne prantsuse keele õpetaja 
pani mu naise (kes oli ka 1. keskkoolist) kirja. Aga dekanaat tõmbas ta 
maha, et naine on rase ja kuidas ta sinna läheb. Naine oli suhteliselt kõva 
iseloomuga, et mis see dekanaadi asi on, kas ta on rase või mitte. Otsustasime
saatkonda minna, aga tee peal jäi tal samm järsku 
aeglasemaks ja ta ütles, et kuule, ma olen tõesti rase, mine sina.

Saatkond teatas, et stipendiumi saamiseks 
peab paar tingimust täitma. Esiteks, võimalikult kõrgel õppima. Kuna mul oli 
neli aastat ülikooli seljataga, siis soovitati kohe magistrisse minna. Teiseks soovitati mitte 
Pariisi minna. Kuna Leo Mõtusel\index[ppl]{Mõtus, Leo} oli Toulouse'is 
üks tehisintellektiga tegelev tuttavtegelane, siis läksin 
sinna.

\question{Üheksakümnendate algus oli huvitav aeg tehisintellektiga tegelda, 
see oli ju enne riistvara läbimurret.}

Selle aja peale oli juba igasuguseid asju tehtud: 
produktsioonisüsteemid\sidenote{Produktsioonisüsteem on tüüpiliselt tehisintellekti pakkumiseks rakendatud arvutiprogramm, 
mis koosneb formaalsetest reeglitest, mehhanismidest nende reeglite järgimiseks 
ning süsteemi olekut säilitavast andmebaasist.}, esimesed teoreemitõestajad, otsustuspuud ja muud asjad. Ma ei mäleta, millal 
Rete algoritm\sidenote{Charles L. Forgy poolt 1974. aastal maailmale tutvustatud 
algoritm efektiivseks formaalsete reeglite rakendamiseks.},
produktsioonisüsteemide indeks tehti. Tolleks ajaks oli 
selliseid põhialuseid laotud juba päris palju. Masinõpet vist väga 
ei tehtud ega osatud, nii palju jõudu käes ei olnud.

\question{Kui kaua sa olid Prantsusmaal?}

Aasta. 

\question{Kas sealt hakkas su võrguhuvi tekkima?}

Jah, see oli esimene koht, kus ma internetti nägin. Naine oli veel Tallinnas, 
tema käis Küberis\index{Küber} interneti küljes. Oli niisugune programm nagu talk: 
ühel pool Unixi masinas kirjutad sina ja teisel pool teine. Kuna sel ajal pidi
kaugekõnesid tellima ja see oli keeruline protseduur, siis
talk võimaldas paremini suhelda.

\question{Kas sul sellist mõtet ei olnud, et hakkaks teadust tegema?}

Oli. Aga naine käis mul Prantsusmaal külas ja siis sündis meil teine 
laps ka ning tulin koju tagasi.

\question{Eestis saab ju ka teadust teha.}

Sel ajal ei saanud. Oli üheksakümnendate algus ja lihtsalt raha ei olnud. Pere jaoks oli vaja raha teenida ja kuidagi korter saada. 
Korterihinnad olid naeruväärselt madalad. Sain 
Prantsusmaal kõrget magistristippi ja pool sellest hoidsin kokku 
ning ostsime korteri.

\question{Mida sa Prantsusmaalt tagasi tulles tegema hakkasid? 
Kas programmeerisid jätkuvalt?}

Jah, ikka. Läksin Ektacosse\index{Ektaco} ja tegin lukkude juhtimist. 
Muu hulgas tuli seda teha ühes pullis kohas, Viimsi 
Talveaias\sidenote{Viimsi Talveaed asub Pringi külas ja valmis 1973. aastal 
Kirovi-nimelisele näidiskalurikolhoosile. See kolhoos (ja 
sealsed kolhoosnikud) olid tolle aja mõistes põhjatult rikkad ning Talveaiast 
kujunes Tallinna 
peenema rahva peokoht. Hulludel üheksakümnendatel oli tegu populaarseima 
paigaga, kus 
kiiresti ja kõikvõimalikel viisidel rikastunud inimesed käisid oma rikkust 
demonstreerimas.}. Seal garderoobis oli püramiid, kuhu korjati numbri vastu 
relvad ära. Avamispeo ajal oli lukkudega mingi jama, need ei töötanud. Läksin sinna ja saunapõrandal oli kiht paljaid purjus 
naisi. Väga imelik koht.
 
\question{Mis lukuprogrammeerimises huvitavat oli?}
 
Seal on pusimist, et kõik asjad paika saada. Mul oli ka see 
probleem, et kui Prantsusmaal õpitu kokku võtta, siis oli põhimõtteliselt tegu diskreetse matemaatikaga. Õppisin, kuidas kompilaatoreid tehakse, 
kategooriate teooriat, eri liiki semantikat (loogiline, denotatsiooniline ja 
operatsiooniline). Aga kus seda vaja 
läheb? Tuled tagasi ja raha saad ikka selle eest, kui kellelgi mõne päris 
probleemi lahendad. See lukuprojekt läks hulluks kätte. Kõigepealt 
pidime tegema kassasüsteemi, mille külge tulid lukud, ja nii see pintsaku 
nööbi ümber õmblemine käis. Tellija tahtis hästi palju muutusi, aga ma suutsin andmemudeli kohe niimoodi paika panna, et ei pidanud seda 
pärast enam muutma, ainult juurde tuli panna. Seepeale sain järsku 
aru, et olen midagi õppinud ka.

\question{See vajab päris head rakendusvõimet, et nii abstraktset 
teemat kohe baasi mudelis kasutada. Kategooriate teooria ju ei ütle sulle, 
millised tabelid olema peavad.}

Jah ja ei. Kategooriate teooria õpetab seda, kuidas maailmas 
asjad on korrastatud. Matemaatika point on selles, et see korrastab 
mõtlemist.

\question{Üheksakümnendate lõpus tegelesid juba infoturbe ja -riskidega, 
kuidas sa lukkude juurest selleni jõudsid?}

Mul hakkas igav. Enn Lakspere\index[ppl]{Lakspere, Enn}
läks Küberisse\index{Küber} tööle. Monika\sidenote{Monika 
Oit\index[ppl]{Oit, Monika}} ja Ülo Jaaksoo\index[ppl]{Jaaksoo, Ülo} olid 
teinud turvaseltskonna enne, kui Eesti Vabariigi iseseisvus paistma 
hakkas, sest nad arvasid, et see on strateegiline oskus, mida on igal riigil 
vaja. Ja neil on selles suhtes õigus.

\question{Kas neil oli selline visioon juba tol ajal?}

Jah, neil oli enne iseseisvumist visioon olemas, et iseseisvus 
ühel hetkel tuleb ja selleks ajaks peab kompetentsi olema. Riigi 
infoturve on riigi jaoks strateegiline asi ja see tuleb korda saada.

\question{Mõnes kohas ei ole sellest siiamaani aru saadud!}

Ilmselt ei saadagi. Need olid kindlasti väga suure visiooniga inimesed. 

\question{Ja sa läksid nende juurde tööle?}

Enn Lakspere viis mind sinna. Kokkulepe oli, et mina teen uurimusi ja tema otsib 
tööd. Minu eriala olid kiipkaardid ja teda kaardid huvitasid, kuna ta tuli 
Ektacost, kus kassasüsteemide külge käisid ka kaardid. Mina pidin kiipkaarte 
uurima. Kirjutasingi raha eest uurimusi, kolm lehekülge puhast teksti 
päevas, seda on päris palju.

Vello Hanson\index[ppl]{Hanson, Vello} õpetas mind kirjutama. Osa tema 
õpetustest olen tänaseks küll ära unustanud, aga Vello Hanson on tõsiselt kõva 
vend. 

Kirjutasin näiteks Pankade Kaardikeskuse\index{Pankade 
Kaardikeskus} arhitektuuri. Keskpank tahtis sellele asjale litsentsi anda ja 
menetleda. Aga ma kirjutasin sinna ühe asja, mis oli 
Sildmäe\index[ppl]{Sildmäe, Tõnis} jaoks uudis. Ütlesin, et ärge \emph{settlement}'i ja 
raha liigutamist üldse sinna 
keskusse pange. Võtke lihtsalt info, kes kellele kui palju võlgu on, ja tehke 
bilateraalne \emph{settlement}. Ta käis üle küsimas, 
kas nii saab. Nad hoidsid sedasi paar aastat 
puhast regulatsiooni kokku. Grupivend Margus Aun\index[ppl]{Aun, 
Margus} 
läks seda värki juhendama. 

Ühel hetkel küsis Ülo Jaaksoo minult, kas
kaartidega mässamisest ühiskonnale ka midagi kasulikku teha saab. Kõrval 
oli Ahto Buldas\index[ppl]{Buldas, Ahto}, kes rääkis mulle asümmeetrilisest 
krüptost ja et sellega saab digiallkirja teha. Lugesin selle kohta veel 
kuskilt juurde ja ühel siseseminaril 1994. aastal pakkusin välja, et 
kiipkaardid võiks inimestele kätte anda ja nendega digiallkirja teha. 
Esimene avalik esinemine sel teemal oli Küberis 1995. aastal. Lõpuks müüs
Tarvi\index[ppl]{Martens, Tarvi} selle idee
riigile maha ja nii see läkski. Tarvi oli tegelikult selle asja juurutaja 
ja innovaator.

\question{Kust tuleb legend, et tegu on soomlaste tehnoloogiaga? Või on tehnoloogia
soomlastelt ja idee teilt?}

See ei ole tõsi, soomlaste tehnoloogia ei ole ka originaalne. Kui 
digiallkirja seadust hakati tegema, tellis Tarvi minu käest profiili, missugune 
see kaart peaks olema. Vaatasin ringi, mis kuskil tehtud on, ja rootslastel oli 
kaardi profiil kirjeldatud, nad tegid kolm võtmepaari. Soomlased kopeerisid 
rootslasi ja panid kaks võtmepaari kokku. Eri võtmepaare on vaja 
seetõttu, et neil on täiesti erinev poliitika. Autentimise ja 
krüpteerimise võtmeid ei tohiks kokku panna, sest krüpteerimise võtmel peaks 
olema taaste, kui tahad seda pikaajaliseks säilitamiseks kasutada. Allkirja 
võtmel aga ei tohi olla taastet. 
Autentimisvõtme jaoks ei ole mingit põhjust taastet tahta. Need on erinevad 
poliitikad, mis tegelikult ei sobi hästi kokku, aga nii on lihtsam inimesi 
õpetada. Turbe põhimõte on see, et ainult lihtsad asjad töötavad. 
Jaapanlased võib-olla saavad keerulise asjaga ka hakkama, aga meie ei saa. Ja 
kuna ükski inimene krüpteerida ei oska, siis talle tuleb anda arvuti, mis teeb
seda tema eest. Kiipkaardist lihtsamat arvutit ei ole olemas.

\question{Ja nii jõudsidki infoturbeni?}

Jah.

\question{Infoturve tundub olevat sinu juttu kuulates ideaalne kombinatsioon: 
sai asju ära teha, oli palju matemaatikat ja arvuteid, kõik kenasti koos.}

Küber oli üsna selge teadusasutus: väga palju 
tehti teooriat ja natuke kirjutati ka programmi. Arne\sidenote{Arne 
Ansper\index[ppl]{Ansper, Arne}} oli juba sel ajal kodeerimises kibe käsi.

Ma läksin sealt ära sellepärast, et tekkis tunne, et kirjutan 
igasuguseid plaane ja siis teised mehed plaanide põhjal ehitavad. 
Ühispank\index{Ühispank} oli viimane pank, kellel ei olnud oma 
kaardiserverit, nii et ma läksin seda 
tegema. Ja kuna turvainimesi ka ei olnud, siis pidin olema turbe ja 
maksekaartide peal. Edasi läksin Hansasse\index{Hansapank}. 

Tore oli pangasüsteemi ringitõstmine, kui tuli ühendada kõik 
maapangad\sidenote{Ühispanga asutasid 15. detsembril 1992 kaheksa 
maapanka, Viljandi kommertspank ja Nordpank}, ning selleks oli vaja süsteemi. 
IT-direktor oli 
Novelli-mees. Ma rehkendasin talle \emph{roundtrip} aegadega, mitu 
transaktsiooni ta tänu lukustamistele jõuab üldse päevas teha, see oli kas 30 
000 või 40 000. See tähendas, et tal Tallinnas oleva 
panga jaoks jätkus, aga terve Eesti peale oli vähe. Siis sai Unix sinna alla 
valitud, et lukustamine käiks ühes masinas ära. Tema valis millegipärast HP, aga
tegelikult oli HP-UX\index{HP-UX} väga äge Unix. Inimesed arvavad, mis 
nad arvavad, aga väga robustne riist. Solaris, millest tavaliselt 
räägitakse, oli tükk maad hellem. 

Toona oli Windowsiga selline õnnetus, et TCP \emph{stack}'i eest 
pidi eraldi raha maksma, Linuxi käimapanek oli kaks korda odavam. Seepärast ühendatigi
kõik maapangad niimoodi ära, et pandi Linux \emph{front}'i ja 
ühe ööga keerati kontor ringi, süsteemi vahetus ja \emph{front}'i vahetus. 

\question{Mida sa praegu teed?}

Praegu uurin kriitilisi sõltuvusi, see teema on pärit 
RIA\index{Riigi Infosüsteemi Amet}\sidenote{Jaan oli aastatel 2011---2015 Riigi 
Infosüsteemi Ameti peadirektor} ajast. 
Kui pead vastutama niisuguse asja eest nagu massiivne küberrünnak, siis 
selle juhtimiseks lükkad kokku staabi. Ja esimene küsimus on muidugi, kas see, mida sa näed, on õige. Niisugust infosüsteemist 
sisse tulevat müra, kus pead süsteemi enda käitumist rünnakuks, on 
päris palju. Teine küsimus on, mis edasi 
juhtub ja kust rünnak peale hakkas. Tavaliselt ei oska inimesed kummalegi 
vastata. Minu algne mõte oli, et äkki nendel sõltuvustel on mingi 
võrestruktuur, võre moodi osaline järjestus. Kui leiad
selles osalises järjestuses miinimumi, siis võib see olla algpõhjus. Ja kui võtad transitiivse 
sulundi, saad kõik tuleviku asjad kätte. Nii tekkis mõte hakata pakkuma 
planeerimisabi. Selleks aga on vaja 
need sõltuvused kuidagi kirja panna. Nüüd olen saanud nii palju targemaks, 
et mu arvamus, et seal ei ole tsükleid, ei ole õige. Tsüklid on ja neid on väga 
palju ning väga lühikesi. Kogu majandus on tegelikult tsükleid ja tagasisidet 
täis ning ma ise olen dünaamilisi süsteeme õppinud ja näinud, mida sellised asjad teevad. Ühesõnaga, nüüd üritan neist asjust aru saada.

\question{Kõlab, nagu oleksid asjad jätkuvalt huvitavad ja see on üks väheseid 
olulisi asju. Või eelistad sa igavaid?}

Ei, seda kindlasti mitte, aga mõnikord võivad need liiga huvitavaks minna. 
Keerukus kipub kasvama ja seda peab jõudma
jõuga maha võtta --- \emph{refactoring} on alati töö.


\chapter{Tanel Raja}
\index[ppl]{Raja, Tanel}
\label{sisu:pronto}

\question{Miks sind Prontoks kutsutakse?}

Jäi lihtsalt külge, ma täpselt ei mäleta, mis asjaoludel. Tol 
ajal pidi olema igaühel oma hüüdnimi ja minu nimi oli lõpuks see.

\question{Kuidas sa arvutite juurde sattusid?}

Arvutite juurde sattusin ma enne, kui minust sai Pronto.

Kas sa oled lugenud sellist raamatut nagu „Professor Lillepooli 
kroonika"\sidenote{Herta Laipaiga ulmelugu, mis ilmus kirjastuse Eesti Raamat väljaandes 1982. 
aastal. Raamatu peategelased kohtuvad muu hulgas arvutiga nimega Kunigunde.}? 
Seal toimus kohutavalt põnev tegevus: tüübid tegid oma Musta Kassi 
ordu ja selle käigus käisid vist TPIs ning tegid mingi skeemi 
valmis. See on tagantjärele mõeldes naeruväärne, aga väikse poisina tundus 
tohutult põnev. Sealt tekkiski arvutihuvi. 

Edasi oli kaks liini. Mu onu oli IT valdkonnas tegev 
tükk maad varem kui mina, ta oli juba sügaval 
nõukaajal arvutite kallal. Käisin Tartus tema juures ja see kõik oli kohutavalt põnev.

\question{Mis selle põnevaks tegi?}

Põnevad olid igasugused nupud ja see, et asjad toimusid. 
See oli väikese poisi jaoks unistus, et mingit masinat saab täielikult kontrollida ja et jõud võiks sellest üle käia.

Teine liin oli Tallinnas, Jaak 
Loonde\index[ppl]{Loonde, Jaak} Luise tänava\index{Tallinna 
Oktoobrirajooni Õppetootmiskombinaat} klass, kus olid Yamaha 
MSXid\index{Arvutid!Yamaha MSX}\sidenote{Seal asus Tallinna Oktoobrirajooni Õppetootmiskombinaat, mida on vahel paigutatud ka Roopa tänavale ja mille kohta öeldakse ka lihtsalt Luise tänava klass.}. Sama tüüpi aparaat, mis mul siin külje all 
seisab.

\question{Mis klass see Jaagul oli? Kas see asus mõne kooli juures?}

See oli pigem huvialamaja, ma täpselt ei mäleta. Igatahes sai seal klassis käia arvuti taga istumas ja nikerdamas. Noored nagad tahtsid 
loomulikult hullupööra mängida, aga kurjajuur Jaak Loonde
ütles, et ei-ei, tuleb hirmsal kombel ikka programmeerida. Nii oligi 
tasakaal nende kahe asja vahel üsna hästi paigas, sest niipea kui 
Jack kõrvale vaatas, olid poisid kohe mingid asjad käima tõmmanud. Kui 
keegi lasi võrku mängu, siis said kõik seda endale laadida. 

\question{Kust mängud tulid? Poest neid osta ju ei saanud.}

Klassis oli õpetaja arvuti ja õpilaste töökohad. Nende vahel oli võrk, mis 
oli naljakal kombel üles ehitatud MIDI kaabli otsa. MIDI on küll rohkem mõeldud muusikariistade juhtimiseks – Yamahad olid tegelikult algselt 
muusikaarvutid ja olid läbi muusikasüsteemi kohapeal võrku pandud.

Mängud liikusid kassettidel ja ketastel. Aeg-ajalt käis keegi välismaal. Mäletan, kuidas TTÜs tuli keegi välismaalt 
mingi teadustööga seotud ettevõtmiselt tagasi ja pani lauale kolm kolmetollist flopit. Kõik seisid kõrval ja ootasid, mis 
nende peal on. See oli kohutavalt harras hetk.

\question{Tol ajal ju osta ega alla laadida kuskilt midagi ei olnud, kõik 
käis käest kätte.}

See oli keeruline jah, sest oli sügav nõukaaeg, 
kaheksakümnendate keskpaik. Servast hakkasid vabaduse kiired juba 
terendama, aga need ei olnud kuskilt otsast veel materialiseerunud. Põhiline 
värk oli see, et inimestel lasti teha asju, mille eest varem 
oleks türmi pistetud.

\question{Kas Luise tänaval käisid keskkooli ajal?}

See oli enne keskkooli, olin umbes 13-14aastane – kael juba kandis, aga mitte väga. 
Täiskasvanuks veel ei peetud, selline ebamäärane aeg, kui veel ei
tea, mis sust saab.

\question{Kas ühel hetkel hakkasid tulema BBSid ja Fidonet?}

Need olid tükk aega hiljem. Me kasvasime suuremaks ja sisi hakkasid ka täiskasvanud meiesuguseid nolke 
tõsisemalt võtma. Enne rääkisime põgusalt, et kui nõukaaeg lõppes 
ja Eesti aeg algas, siis esimene palk oli 300 
krooni, mis praegusel ajal on 20 eurot. See oli kuupalk ja sellest elas kenasti ära. Mitte küll nii, et oleks midagi hullupööra huvitavat selle eest 
saanud osta, aga elas ära. See võtab tegelikult nõukaaja 
elatustaseme üsna hästi kokku: kuna asjad, mis meil siin ringi liikusid, olid 
valmistatud oma Normas, Salvos või Kommunaaris, siis olid nii hinnad kui ka sissetulek väiksed. Asjad 
olid tasakaalus. 

\question{Arvutit 20 euro eest ei osta.}

Arvutit tõesti ei osta, aga need vahendid olid olemas asutustel. 
Teine asi oli see, et Nõukogude Liitu oli keelatud eksportida kõvemat 
arvutustehnikat. Et meile tekkisid siia näiteks MSXid, oli osaliselt 
tingitud sellest, et tegu oli suhteliselt alumise otsa masinatega, rohkem 
mänguasjade kui päris tööriistadega. Näiteks kui 
Soomest veeti Eestisse üks 386, kui need oli just äsja välja tulnud, siis sündis
sellest kohutav rahvusvaheline skandaal. Ühendriigid võtsid soomlastel kõri 
pihku ja ütlesid, et mis mõttes te veate Nõukogude Liitu niisugust 
tehnoloogiat, millega on võimalik rakette arvutada ja mida iganes teha! Meie 
mõistes oli see masin tol hetkel väga kõva sõna. Praegu on see muidugi 
naeruväärne, suvaline kell on ka võimsam.

\question{Ehk et arvutile ligi saada, pidi mõnele asutusele külje 
alla pugema.}

Jah. Asutustel olid arvutid, millega nad üritasid oma asjatoimetusi läbi viia.
Arvuteid oli mitmesuguseid, näiteks klassikalisi nõukaaegseid Uralitel 
põhinevaid süsteeme, kus olid terminalid ja suured kastid.
Ajapikku tekkisid ka muud, välismaise päritoluga masinad, peaasjalikult 
286d ning koos nendega võrgud. Seega oli vaja 
inimesi, kes seda kõike haldaksid. Aga inimestest oli põud, kuna 
keegi ei teadnud, mida nende kastidega peale hakata. Ja kui seal kõrval 
nikerdamas käisid, siis muutusid päris kähku kasulikuks. Noore poisina oli 
aega, ei mingeid perekondlikke kohustusi ega muud säärast, huvi oli suur ning 
saidki seal eksperimenteerida. Läksid õhtul pärast kooli sinna, nemad läksid 
töölt ära ja said seal istuda kuni üheksa-kümneni. Ja kokkulepe oli see, et üritad kuidagimoodi kasulik olla, ning 
lõpuks sai arvuti taga istumiskohast töökoht, kui kool läbi sai.

\question{Mis asutuses sina käisid?}

Minul oli alguses linnavalitsus\index{Tallinna Linnavalitsus} ja hiljem 
riigikantselei\index{Riigikantselei} – kogu selle 
huvitava perioodi, kui Eesti Vabariik välja kuulutati, töötasin ma
riigikogu majas. Seda nimetati 
vist peaministri kantseleiks, Stenbocki maja siis veel ei olnud. Seal olid ka 
Uralid\index{Arvutid!Ural}\sidenote{Nõukogude Liidus Pensas aastatel 1956–1964 toodetud arvutite sari.}, 
üüratud kapid, mille sees olid viiemegabaidised trummelkettad, mis tuli 
hommikul käima lükata.

\question{Kas sul akadeemiline haridus jäi pooleli?}

Mul on jah lõpetamata kõrgharidus. 
Üritan seda seniajani lõpetada ja loodetavasti paari aasta 
jooksul seda ka teen.

Tol ajal oli valida, kas tegeled arvutitega või õpid. Suuresti oli mu valik 
väga selge: ma õppisin arvuti taga oluliselt rohkem.

\question{Millal Fidonet Eestis jalad alla võttis?}

Ma täpset aastaarvu ei oska öelda, aga sellega tegutsesid 
Tõnis Reimo\index[ppl]{Reimo, Tõnis}, Tarmo 
Ausing\index[ppl]{Ausing, Tarmo} ja Virko Püss\index[ppl]{Püss, Virko}. Lisaks
jõlkusime seal mina ja Miko Raud\index[ppl]{Raud, Miko}.

\question{Kus te tegutsesite?}

Erinevates kohtades, näiteks Narva maanteel. Eks grupi tuumik 
teab nüansse paremini kui mina. Vahepeal tekkis seal 
võimalusi peaasjalikult soomlastega asju ajada, kui avastati enda jaoks BBSid ja aduti, et tarkvara peab ka kusagilt tulema. Sel ajal
hakkas lisaks flopidele tekkima võimalus modemiga asju alla laadida ja tulid 9600sed modemid.

Asjad hakkasid jumet võtma. Tol ajal olid tarkvarapaketid maksimaalselt paari 
mega baidised ja olid tõmmatavad umbes päevaga.

\question{Seega helistati Soome BBSidesse sisse?}

Jah. 

\question{Kuidas see käis? Jaan Tallinn\index[ppl]{Tallinn, Jaan} on 
rääkinud läbi inimoperaatori arvuti külge helistamisest.}

Igasuguseid imeasju tehti. Näiteks selgus, et lifti 
telefoniühendusest oli võimalik välismaale helistada, sest keegi polnud taibanud 
seda sealt välja lülitada. See tähendab, et liftist sai helistada ja öelda, et 
appi-appi, olen siia kinni jäänud, aga sellesama ühendusega sai helistada ka Soome. Keegi ei olnud nõukaajal kindel, kes 
selle kinni peab maksma, ja seetõttu jäigi see kuidagi ripakile. 
Loomulikult olid ligipääsud erinevatele keskjaamadele ja raha 
tekkis kusagil süsteemides ning kadus kuhugi, nii et 
tegelikult kasutati seda ühte- või teistpidi kurjasti ära. See oli üks viis Nõukogude süsteemi õõnestada.

Sellega seoses tekkisid kontaktid. Näiteks BBSi sisse logides vaatas
\emph{sysop}, et ohoo, Eestist 
mingid tüübid, ja tahtis paar 
sõna juttu ajada. See oli üsna tavaline, et BBSi operaator rääkis külalistega.

BBS ei olnud väga erinev tänapäeva 
sotsiaalmeediast. BBS pandi püsti kahel põhjusel: esiteks, et kontakte luua ja
\emph{networking}'ut teha, olla nii-öelda elu pulsil. Teiseks millegi
propageerimiseks, näiteks oli BBS mõne firma juures või oli 
mingisuguse demo grupp enda oma püsti pannud. See BBS, kust me esimese 
kontakti saime, oli Poison Door\index{BBS!Poison Door}.

\question{BBS võis ka mingi demo grupi juures olla, soomlaste 
\emph{demoscene}\sidenote{Demo on arvutikunsti teos, mis kujutab endast 
terviklikku, sageli väga väikest arvutiprogrammi, mis esitab 
audiovisuaalset vaatemängu. Demo eesmärk on demonstreerida (nagu nimigi 
ütleb) autorite programmeerimise, visuaalkunsti ja arvutimuusika oskusi. Demode 
ümber tekkis kogukond, \emph{demoscene}, mis sai kokku demopidudeks kutsutud 
festivalidel. Üks kuulsamaid on siiamaani regulaarselt Helsingis 
toimuv Assembly.} oli tol ajal väga kõva.}

Mul endal oli kontakt \emph{Future Crew}\index{Future 
Crew}\sidenote{Soome demogrupp, mis peamiselt tegutses aastatel
1987–1994. Nende tehtud oli tõenäoliselt kõigi aegade mõjukaim demo 
„Second Reality“ (avaldati Assembly demopeol 1993. aastal). See tegi 
tänapäeva mõistes olematu riistvara peal reaalajas asju, mis tundusid täiesti 
võimatud, nägi üliäge välja ja sisaldas muusikat, mis siiani kananahka tekitab. 
1999. aastal hääletasid Slashdoti lugejad selle demo kõigi aegade kümne 
vingeima häki hulka.} tüüpidega. Ma laadisin nende BBSist alla niisuguse 
toreda mängu nagu „Wing Commander“\index{Mängud!Wing Commander}. 
Omadele anti asju, mis tegelikult ei olnud
päris ametlikult väljas. Peaaegu kõikidel BBSidel olid tagatoad, 
kus hoiti nodi, mida kasutati vahetuskaubana. Tarkvara oli sel 
ajal kõva valuuta. Me panime püsti kahepoolse ühenduse: mina 
laadisin üles mingi muu asja, mille olin kusagilt saanud, ja 
sealtpoolt tõmbasin vastu „Wing Commanderit“ ning samal ajal sai rääkida ka. 
See tarkvara võimaldas kahepoolset sidet ja samas ka 
\emph{chat}'ida, mis ei võtnud väga palju ühenduse mahtu.

\question{Iga klahvivajutus oli üks sümbol, fondi või värvide 
informatsioon kaasa ei liikunud.}

Just, see oli tavaline tekst. Kogu mängu allatõmbamine võttis 
aega tunde. Selleks ajaks olin endale ise ühenduse sebinud, Riigikantseleil oli
selline võimalus nõukaaja lõpus. 

Ühesõnaga, tutvuti ja info liikus. Ja oli aja küsimus, millal lõpuks siingi 
oma BBS püsti pandi ja Fidoneti kontakt saadi. Ma just hiljuti uurisin selle 
kohta ja paistab, et Fido on nüüdseks lõplikult hinge heitnud.

\question{Üsna kaua võttis aega!}

Võttis küll, aga võibolla on mõttekas see uuesti üles tõmmata. See eksisteerib endiselt ja 
tänapäeval on retroasjad moes, nii et ehk ärkab see kunagi uuesti ellu.

\question{Mis oli Eesti üks esimesi suuri BBSe, kus rahvas hulgakaupa sees 
käis?}

Esimene tõsiseltvõetav BBS, just nimelt Fidoneti mõistes, oli Hackers Night 
System\index{BBS!HNS}\index{BBS!Hackers Night System|see{HNS}}. Nagu 
nimigi ütleb, oli tegu häkkerite öösüsteemiga. Päeval olid telefoniliinid muuks 
otstarbeks, öösel käis nende peal BBSidesse helistamine. 

\question{Miks sel ingliskeelne nimi oli?}

Et oleks rahvusvaheline ja äge.

\question{Kes HNSi käigus hoidis?}
Seesama kamp: Reimo\index[ppl]{Reimo, Tõnis}, Ausing 
\index[ppl]{Ausing, Tarmo} ja Virk\index[ppl]{Püss, Virko}.
 
\question{BBSi jaoks oli ju mingit riistvara ja modemeid vaja?}

Oli jah, sinna juurde käis paras sebimine. Tol ajal olid vahendid suuresti riigi rahakotis. Selle küljes siis 
istuti ja kui oldi juba kasulikud, siis sai alati ka neid ressursse juhtida õiges suunas. 

\question{Mis aastal see oli?}

Kaheksakümnendate lõpus, mitte 1989, vaid varem. Ma täpselt ei mäleta, 
vanus oli selline, et keegi ei olnud veel täiskasvanu, aga ka mitte enam laps. Aeg omas siis teist tähendust ja nüüd hiljem on
raske mõõtkava peale panna.

\question{Kas tol ajal oli BBSil üks modem ja üks liin?}

Ojaa. Tegelikult oli muid ka, paralleelse side katseid, näiteks 
PirnBox\index{BBS!PirnBox}. Fidoneti mõistes klassikalistest BBSidest oli HNS esimene ja sealt läks 
asi krõbinal laiali.\sidenote{Pronto ise pidas BBSi New Age 
System\index{BBS!New Age System} Fidoneti aadressiga 2:490/12.}

\question{Kui palju neid BBSe tipphetkel 
oli?}

Tipphetkel oli 20–30. Süsteem nägi ette \emph{point}'e, mis olid nii-öelda pool-BBSid. Täpsemalt olid \emph{point}'id ja \emph{full 
node}'id. \emph{Node}'il olid kohustused: meile tõmmata, hoida ja 
jagada. \emph{Point}'iga sai lihtsalt tõmmata. Paljud 
BBSid otsustasid \emph{point}'iks olemise kasuks puhtalt sellepärast, et need 
ei saanud ennast kogu aeg käimas hoida. \emph{Node}'idel olid \emph{point}'id, keda nad varustasid 
informatsiooniga, ja \emph{node}'i käimas hoidmine eeldas ühte- või teistpidi 
võimekust olla teatud hetkedel üleval.

\question{Seega oli Eestis tol ajal 
paarkümmend inimest, kellel oli võimekus sebida liin ja riistvara ning ka 
süsteemi käigus hoida.}

Tipphetkel küll jah. Vahepeal sai nõukaaeg otsa ja tuli Eesti Vabariik ning ühel 
hetkel hakkas asi selles mõttes käest ära minema, et raha hakkas omama 
tähendust. Enam ei saanud lihtsalt kusagil ettevõtte küljes istuda ja oma
asju teha. BBS koos telefonikõnedega tekitas kulusid ja peod
hakkasid vaikselt kinni minema. Inimesed vahetasid töökohti ja uutes 
kohtades ei vaadatud selle peale enam lahke pilguga.

\question{Kuidas sellest ürgsupist Eesti arvutifirmad tekkisid? Kas BBSide 
seltskond läks sujuvalt üle teenuste pakkumisele?}

Osaliselt küll. Need inimesed olid ühte- või teistpidi 
arvutifirmadega seotud, aga tihtipeale ei olnud need päris samad 
inimesed. Teatavasti on sogases vees kõige parem kala püüda, seal on kõige 
suuremad purikad. Sogasel ajal leiti erinevaid viise, kuidas endale 
raha teha. Näiteks Peterburist veeti autoga Tallinnasse igasugust IT-tehnikat. Peterburis olid punktid, kust sai asju 
osta ja Eestisse tuua. Nii see elu vaikselt edenes.

\question{Kas sina olid sel ajal veel Riigikantseleis\index{Riigikantselei}?}

Jah, aga oli näha, kuidas hakkasid tekkima esimesed firmad, mõned edukad, 
mõned vähem edukad. Ühel hetkel läksin Riigikantseleist 
minema, sest ka seal toimusid struktuurimuudatused.

\question{Mida sa tol ajal peamiselt arvutiga tegid? Kas kirjutasid 
koodi?}

Nüüd tundub see ehk naljakas, aga siis oli see nagu eluviis. Ega see väga ei erinenudki praegusest eluviisist, vahe on ainult selles, 
et nüüd ei pea näiteks Facebookile ligipääsu saamiseks kulmulihastel ringi roomama. Tollal ei olnud see kõikidele kättesaadav. 

Arvuti kasutamisel oli siis küllaltki kõrge lävi, mis eeldas teatud ülevaadet tehnikast ja võimalustest. Praegu on internet ise ennast 
sõlme tõmmanud, aga varem pidi täpselt teadma aadresse, 
kuhu minna, sest polnud otsinguid. Siis alles hakkasid tekkima esimesed 
otsingumootorid: WebCrawler, AltaVista ja lõpuks Google. Need tõmbasid läve madalaks. 

\question{Mida BBSiga teha sai?}

Sai faile jagada ja kirju vahetada. Fidonet oli tänapäeva mõistes suuresti
interneti meilisüsteemi sarnane. Olid ka uudisegrupid ja \emph{usenet}'i grupid, mis on asendunud näiteks Facebooki ja Redditiga, kus käib 
info vahetamine.

\question{\emph{Usenet}'i grupid olid tollal hierarhilised, aga praeguseks on see struktuur laiali vajunud.}

Jah, olid hierarhiad ja etiketid, mida võhikul oli väga raske 
aduda. Tihtipeale inimesed tundsid küll üksteist üsna lähedalt, aga 
teinekord mõnd jutuajamist jälgides tekkis täieliku \emph{outsider}'i tunne, kui ei saanud aru, millest jutt käib. Kõikidel oli oma taust.

\question{Kus inimesed tuttavaks said? Kas nendes gruppides?}

Oli kaks varianti. Keegi tutvustas ja aitas ree peale või siis kiibitsesid mõnda 
aega ja ühel hetkel hakkasid aru saama, mis toimub. Kui üldse hakkasid, see ei olnud lihtne.

\question{Kui tihedalt Eesti Fidoneti seltskond omavahel läbi käis?}

Seltskond pidi paratamatult läbi käima, sest Fidoneti tekkides moodustusid ka grupid, kus tuli sisu 
tekitada. Ja kuna esialgu oli inimesi vähe, siis paratamatult ei olnud ka
kommunikatsioon meeletult tihe. Fidonetiga tegeles 
paar-kolmkümmend inimest ja isiklikult tuttavaks saamine ei olnud keeruline.

\question{Kes need inimesed olid?}

Enamasti samasugused IT valdkonna inimesed nagu mina, kellel olid
sarnased huvid – meil oli, millest rääkida. Olid ka 
teemad, mis siis olid parajasti \emph{zeitgeist}. 
Näiteks „King's Quest 
IV“\index{Mängud!King's Quest} mängides ei olnud võimalust minna 
veebi ja otsida \emph{walkthrough}'d. Inimesed üritasid omal jõul 
läbi närida ja aeg-ajalt vahetati kogemusi. Muide, sellest ajast pärineb 
ka Habichti raamat \enquote{Selles mängus ei hüpata}\sidenote{Juhan Habichti novellikogumik, mis ilmus 1993. aastal kirjastuse Katherine väljaandel.}. 
See mäng oli 
„Larry“\index{Mängud!Larry}\sidenote{„Leisure Suit Larry“ oli Al Lowe'i
loodud seiklusmängude sari, mis ilmus aastatel 1987–2009 ning oli 
tuntud omapärase huumori ja alaealistele sobimatu sisu poolest. 
Näiteks katsus mängija riiulil seisvat kopratopist ja Larry teatas: \enquote{\emph{I've always 
liked the feeling of a good beaver}}.}.

Samuti räägiti võimalustest ja nende 
vahetamisest. Ühel oli üks asi, teisel teine ja pandi seljad kokku. Kuna 
inimesi oli vähe ja üksteist teati, siis ei olnud ka väga 
suurt kanakitkumist.

\question{Kas trollimist või muud säärast ka toimus?}

Kui auditooriumi ei ole, siis inimesed jäävad tavalisteks inimesteks. Kui 
annad normaalsele inimesele anonüümsuse ja publiku, siis saab tast igavene 
tõpranahk.

\question{Isegi kui sul oli \emph{handle}, siis sa ju tegelikult ei olnud anonüümne.}

\emph{Handle} oli lihtsalt nimi, tegelikult kõik teadsid, kes on kes. 
Isegi kui olid anonüümne, siis ei 
kasutatud laest võetud nimesid. Kui olid oma nimele
feimi tekitanud, siis sa ju ei tahtnud sellega 
uisapäisa ringi käia.

\question{Sa oled siiamaani Pronto ja teistele tähendab see senini midagi. 
Kui keegi hakkas sigatsema, kas ta visati siis välja?}

Jah, juhe tõmmati seinast ja olid kohemaid \emph{persona 
non grata}. Kuna see oli seotud sinu enda huvidega ning
mineviku, oleviku ja tulevikuga, siis ei saanud seda omale lubada.

\question{Seega käitusid kõik viisakalt?}

Kõik olid seal paadis võrdsed. Kui keegi hakkas paati 
kõigutama, siis ta kõigepealt kõigutas seda enda all ja kui ta seda jätkas, siis ta lihtsalt eemaldati paadist ja pidi ise vaatama, kuidas 
veekogus hakkama saab.

\question{Kas seda juhtus ka?}

Otseselt mitte või kui juhtus, siis juba hilisemal ajal. 
Alguses oli ikkagi tihe seltskond ja kuigi kõik ei saanud omavahel 
ideaalselt läbi, mõisteti, et selles paadis ollakse koos. Seetõttu tüli põhjustada võivaid teemasid
lihtsalt välditi.

\question{Seega saadi aru, et teatud asjadest ei tasu rääkida.}

Jah. Trollimine ju ongi rääkimine asjadest, mis teisele 
inimesele peavalu valmistavad.

\question{Tuleme sinu juurde tagasi. Kui sa 
Riigikantseleist\index{Riigikantselei} ära tulid, mida sa siis tegid?}

Töötasin sellises kohas nagu Marvin-Ekspert, sain seal ostmise ja müümisega käe valgeks. Tegelesin selliste toodetega nagu Gravis Ultrasound ja 
IOMega\sidenote{Gravis Ultrasound oli toona PC-maailmas tipp helikaartide tootja ja IOMega tegeles väga innovatiivsete andmesalvetuslahendustega.}.

See oli selles mõttes huvitav aeg, et Gravis Ultrasound maksis väikse 
varanduse, aga samas oli see tükk maad parem kui mõni teine toode. Müüsin neid umbes sama palju kui kõiki 
ülejäänud asju kokku müüdi, kuigi see oli kallis. Mõnes mõttes oli sellega sama lugu nagu 
Apple'iga: kallimat asja on alati lihtsam müüa, sest kalliduse taga on tavaliselt 
väärtus, toode ei ole kallis niisama.

\question{Mis aastal see oli?}

Ilmselt 1994. või 1992. aasta kanti.

\question{Kas sel ajal hakkas tasapisi Microlink tekkima?}

Microlink tekkis tegelikult üsna aegade alguses. See oli üks nendest firmadest, kes 
alustas sellest, et hakati kotiga Peterburist asju tooma. Esialgu müüdi 
arvuteid firmadele, sest nendel oli raha, kuigi 
omandisuhted polnud veel päris paigas.

\question{Kapitalismi oli veel vähe?}

Kapitalismi oli jah vähe, olid veel nõukaaja jäägid --- keegi oli kuskil käpa peale 
pannud. Oli niisugune aeg, kui ma paljusid asju ei teadnud ja 
paljusid teadsin, aga ei tahtnud teada. Asju, mille kohta võib öelda, et mis juhtus Vegases, las jääb Vegasesse. Sel ajal tehti 
igasuguseid asju, mis praegu võivad näida küsitava 
eetilise ja moraalse taustaga, kuid siis olid 
tegelikult õiged ja vajalikud.

\question{Tol ajal ju kujuneski välja, mis on õige ja mis mitte.}

Jah. Loomulikult tehti sel ajal igasugust erastamist ja ärastamist, aga ka see oli 
hädavajalik puhtalt sellepärast, et tookord tehtud otsused eristavadki meid 
tänapäeva Moldovast, kus omal ajal tehti teistsuguseid otsuseid. Isegi 
need, kes meil siin ärastasid, tegid seda teataval määral 
\enquote{eesmärk pühendab abinõu} kaalutlustel.

\question{Räägi pisut ka ajakirjast .EXE\index{Ajakiri!.EXE}.}

See ajakiri oli osaliselt Microlinki püüd ennast nähtavaks 
teha. Eestis oli tollal kaks arvutiajakirja: 
Arvutimaailm\index{Ajakiri!Arvutimaailm} ja .EXE. Arvutustehnika \& 
Andmetöötlus\index{Ajakiri!Arvutustehnika \& Andmetöötlus} ei olnud klassikalises mõistes ajakiri, vaid rohkem 
vihik. Nõukaaja lõpus ja Eesti aja alguses anti välja vihikuformaadis 
erialaväljaandeid, mis ei olnud mõeldud laiaks tarbeks.

.EXE tekkis umbes samal ajal kui Arvutimaailm, Microlink 
püüdis tekitada endale laiatarbeväljundi.

\question{Selles ei olnud palju laiatarbeasju, 
vaid stiilipuhas \emph{hard core} küberpungijutt!}

.EXE oli selles mõttes \enquote{laiatarbeväljund}, et sel ajal ei olnud inimestel 
raha arvuti soetamiseks. Pidi olema ikkagi väga suur tahtmine ja vastavalt sellele kujunes ka ajakirja sisu. 

Sel ajakirjal oli ajastu hõng juures. Mida inimesed arvutiga parasjagu tegid, 
see sealt ka läbi kumas. 

\question{Kuidas sa selle juurde sattusid? Kas kirjutasid juba enne .EXEt?}

Tol ajal kirjutati näiteks naljaviluks mängudest, sisu toodeti vabatahtlikult. 
Gruppidesse postitati dokke, häkiti ja nii edasi. Kuna ma olin mängudest kirjutamisega silma 
paistnud ja ka kirjaoskus enam-vähem olemas, siis nii ma ajakirja sattusin. 

See oli päris 
naljakas aeg – ajakirja koostamine oli omamoodi 
häkkimine. Tavalist kogunes kolleegium (seltskond, kes sisu kokku pani) 
kokku, lükati ette kaks kasti õlut ja enne toast välja ei 
lastud, kui ajakiri oli kokku pandud. Igaüks võttis endale mingid kohustused 
ja kadus nendega tegelema.

\question{Kaua .EXE üldse ilmus?}

See ilmus umbes poolteist aastat.

\question{Nii vähe?}

Jah, ma ühel hetkel korjasin kõik numbrid kokku\sidenote{Aadressil 
\url{punktexe.ee} on kõik ilmunud numbrid täies mahus olemas.}. Esimene number ilmus aprillis 1993 ja viimane 
1995. aastal. Nii et kaks aastat, vahepeal läks ilmumine eklektiliseks.

\question{Kes neid ilusaid kaanepilte joonistas?}

Kaspar Loit\index[ppl]{Loit, Kaspar} alias BKnows.

\question{Arvestades, milline mõju ajakirjal oli, palju seda loeti ja kuidas fännati, 
siis oli ilmumise lühidusest hoolimata tegu väga mõjuka asjaga.}

Jah, numbreid oli kokku vist kaheksa. Igaüks oli omaette šedööver, kuna see oli südamega tehtud, eriala inimestelt eriala inimestele. .EXEt anti välja selleks, et skenet juurutada, mitte et selle pealt 
üüratut kasumit teenida. 

\question{Millist skenet? Arvutiinimeste oma?}

Jah. Inimesele tänavalt
oli see ajakiri ehk pisut raskevõitu. Tol ajal oli 
arvutiajakirjandus teistsugune kui praegu, mil
igaühel on arvuti ja loetakse, kuidas oma mobiiliga 
ühte, teist või kolmandat teha. Arvuti oli siis suur asi, 
seda polnud kaugeltki mitte kõigil. Praeguses mõistes üks-kaks protsenti inimestest tabas tegelikult reaalselt arvutit ja oskas seda 
igapäevaelus kasutada.

\question{Seevastu inimesi, kes tahtsid kasutada, oli rohkem. Ja nii nad lugesidki hardalt, kuidas Pronto seikleb „Day of the 
Tentacle'is“\index{Mängud!Day of the Tentacle}\sidenote{Legendaarne mäng, mis ilmus 1993. aastal LucasArtsi väljalaskel ja uuendatud graafikaga 
2016. aastal ning mille \emph{walkthrough} avaldati .EXE teises numbris 
novembris 1993, autoriteks BKnows\index[ppl]{BKnows} ja 
Pronto\index[ppl]{Pronto}.}}.

Eks see kõik hakkaski pisitasa tuult tiibadesse võtma. Sel ajal toimus 
jõhker inflatsioon ehk räägitud kahekümnest eurost said päris kiiresti 
sajad eurod. Arvutid muutusid jõukohaseks ka teistele ja Rootsist veeti siia humanitaarabi korras pruugitud tehnikat.

\question{Kas kogu selle ajal jooksul müüsid sina muudkui Gravist?}

Gravist ja Iomega Bernouilli draive\sidenote{1992. aastal turule tulnud, oma aja kohta suure mahutavuse ja 
eemaldatava kettaga salvestussüsteem Bernouilli Box 
oli Iomega esimene laialt kasutust leidnud toode.}, QIC-80 
teipe ja muud säärast.

\question{Huvitav, et mitmed inimesed on teatud faasis tegelnud just 
arvutustehnika müügiga.}

Kuskilt tuleb raha teenida. Kätte jõudis aeg, kui varad said laiali 
jagatud ja sa pidid oma tegevust põhjendama, näiteks miks sul on 
BBS. Ainuke võimalus seda asja edasi edendada oligi müügi 
egiidi all.

\question{Kas koodikirjutamisega ei saanud elatist teenida?}

Tol ajal ei olnud eriti mingeid koode, mida kirjutada. Väikseid asju loomulikult oli, aga valdavalt käis koodikirjutamine 
andmebaaside ümber, näiteks olid FoxBase ja DBase, kus tehti 
ettevõtete raamatupidamist ja inventuuri. 

\question{Kas iga ettevõte pusis endale ise tolle rakenduse kokku?}

Kas ise või osteti firmadelt, aga süsteem koosnes tavaliselt 
mõnest andmebaasilahendusest. Oli ka muid asju, 
näiteks meditsiiniga seotud lahendusi, millel olid juba infosüsteemid, aga 
need olid väga spetsiifilised ja neid arendati enamasti väikses mahus.

\question{Eestlane üldiselt ei ole suurem asi müügiinimene, 
aga IT-asja on meil õnnestunud rahvusvaheliselt päris hästi müüa. Kas ehk
seetõttu, et kriitilisel hulgal inimestel on olnud müügikogemus?}

Kindlasti. Tol ajal oli see paratamatu, sest kui tahtsid 
saada ligipääsu, pidi juba siis ennast müüma. See on üks asi, mis on muutnud 
vana kooli IT-vennad teistsuguseks – sa pidid paratamatult suutma müüa. Kui ei suutnud, siis polnud sul IT valdkonda asja. Kõige 
tähtsam kaup olid sa ise.

\question{Sest muud sul ei olnud?}

Muud ei olnud, isegi mitte kogemusi, sest kogemused tulevad töö käigus. Sa pidid suutma endast teha väga vajaliku tegelase.

\question{Nii et kui enesemüügi oskus on olemas, siis võib igasuguseid 
asju juhtuda.}

Kui tähelepanelikult vaadata, siis IT valdkonna müügis ongi 
läbimurrete taga tihtipeale ühed ja samad inimesed ning just 
vana kooli kaader, kes enamasti on oma läbimurde ehk müügi saavutanud mitte 
tänu avalikkusele, vaid vaatamata sellele. Teatavasti tunneb avalikkus 
kohemaid muret, kui keegi teenib paremini või tunneb ennast kuidagi paremini. 
Hari läheb kohe kadedusest punaseks.

\question{Tihti öeldakse, et meil on vedanud, sest õiged inimesed on sattunud 
õigetesse kohtadesse. Sinu jutust tuleb välja, et tollest seltskonnast tulidki 
inimesed, kes sattusid õigetele kohtadele.}

Täpselt nii. Need inimesed on siiani 
alles, osa neist üle viiekümne, osa alla selle, aga üks või teine on suuremate 
läbimurrete taga.

\question{Oskad sa öelda, mis 
hetkel kaotas see maailm oma süütuse? Kui romantilisest õllekasti abil 
toimetamisest sai raha teenimine.}

Ma ei oska seda niimoodi paika panna, sest tegelikult on see 
ikkagi suuresti väljaspool loodud kuvand. Kui on mingisugune grupp, 
siis paratamatult tekivad autsaiderid, kes tunnevad pahatahtlikku 
kadedust. 

\question{Ja nimetavad inimesi häkkeriteks?}

\enquote{Häkker} hakkas omandama lihtsalt teistsugust tähendust.

\question{Viidates ühele .EXE loole, mis on 
küberpunk\sidenote[][-2cm]{Allkirjastamata, kuid BKnowsi\index[ppl]{BKnows} piltidega 
lugu \enquote{Kes sa selline oled, küberpunk?} ilmus .EXE kolmandas 
numbris 1994. aasta aprillis. Seejuures tuleb tunnustada artikli asjakohasust: nii ilmumise (eba)regulaarsuse ja lühiduse kui ka kultusliku staatuse poolest .EXEga sarnane, kuid suurema levikuga ajakiri MONDO 2000 (aastatel 1984–1998 ilmus USAs 17 numbrit) avaldas oma samateemalise satiirilise artikli \enquote{R.U. A CYBERPUNK?} oma 10. väljaandes 1993. aastal.}?}

Kõik asjad, mis on punk, nagu aurupunk, küberpunk või diiselpunk, on lihtsalt 
žanr, mis läbib mitut asja; valdavalt seda, kuidas siduda teadvus 
tehnikaga. Mõnes mõttes on meie ühiskond praegu nii-öelda küberpungi jaoks 
esimesel tasemel, sest see, kui inimesed istuvad ninapidi telefonis, on 
lihtsalt liidestamise küsimus. Inimesed on ennast tegelikult arvutiga juba väga 
intiimselt liidestanud.

\question{Nagu sa mainisid, siis algas see juba kaheksakümnendate lõpus, kui 
kogu sinu elu oli arvutis. Lihtsalt liides oli kandilisem.}

Liides oli kandilisem ja olemas vähestel inimestel; seetõttu polnud see elu, vaid 
mu \emph{alter ego}. See ongi üks põhjus, millepärast valiti omale sellised
tunnused, nagu mul on Pronto – et teha vahet sellel, mis toimub arvutis ja mis 
niisama. Põhimõtteliselt loodi endale identiteet. 

\question{Just nimelt loodi, mitte ei valitud!}

Ja sellega elati osaliselt tulevikus, aga ka muu elu jäi alles. Pere, sõbrad ja see õlu, mida joodi, jäi kõik teise ellu.

\question{BBSi rahvas käis ju koos ka.}

Käis küll. Kõigepealt olid \emph{sysop}'ide saunad ja muud üritused, kust kasvas välja Fidonet. Hiljem tekkisid
BBSummerid\index{BBSummer}.

\question{Kui palju neid toimus?}

Need said alguse nõukaaja lõpus ja neid toimus üksjagu. Üks 
BBSummeritest, vist teine või kolmas, lükati edasi sellepärast, et tankid sõitsid Eestisse 
sisse.

\question{Olen näinud BBSummeri pilte, mille peal on kõik Micolinki, Skype'i, Unineti ja 
teiste hilisemate suurte asjade alustajad. Kas tol ajal, asja sees 
olles, ei olnud niisugust tunnet, et oi, küll me oleme ägedad?}

Muidugi oli! Me olimegi hullult ägedad! See oli ka üks põhjus, miks me sellega tegelesime.

\question{Tulles meie jutu alguse juurde tagasi, kas selle ägeduse tuum oli 
jätkuvalt see, et sai masina mõne näpuliigutusega oma tahtele allutada?}

Kindlasti. Teiseks ei piirdunud elu enam oma õuega, vaid koos Fidonetiga tekkis ka ülejäänud maailm sinna otsa. See ei 
olnud väga erinev tänapäeva Redditist, Facebookist või Twitterist, kus ei
saa suhelda mitte ainult paari lähema tuttavaga, vaid kogu ülejäänud 
maailmaga. See andis 
näiteks võimaluse keeli omandada ja suhelda erinevates keeltes, mis omakorda aitas edasi.

\question{Nii et see tekitas maailma avardumise tunde?}

Maailm avardus kindlasti. See oli mõneti samasugune tunne nagu 
kosmonaudil, kui ta atmosfäärist väljub. Eriti kui see pind, millelt üles 
tõusti, oli tükk maad madalamal kui enamiku maailma jaoks --- me tegime
otse nõukaajast sammu tulevikku.

\question{Ühel hetkel olid Nõukogude pioneer ja pisut hiljem 
vestlesid California kuttidega keskjaamadest.}

Jah, absoluutselt. Tekkisid võimalused ja kogemused. Näiteks mõnes mõttes 
positiivne nähtus oli see, et Eestis puudusid \emph{legacy} süsteemid, meil 
polnud IT valdkonnas mineviku taaka, vaid asi oli lihtsalt poolik. 
Mineviku taaga puudumine võimaldas Eestil kihutada päris kiiresti 
päris kaugele võrreldes ülejäänud maailmaga, kes pidi oma asju käimas hoidma. 
Me oleme nüüd jõudnud sinnamaani, kus meil on oma taak tekkinud ja peame 
sellega tegelema.

\question{Lõpuks ikka saab inerts otsa, aga seni on see meid päris kaugele 
vedanud.}

Seda sai üsna hästi ära kasutatud just nimelt sellepärast, et õigel hetkel sattusid õiged inimesed pumba juurde ja saagi tõmmati 
käima nii kaua, kui jõuti, enne kui ärimehed jaole jõudsid. 

Kuna oli hulk inimesi, kes tegid midagi, mis oli 
müstiline, keeruline, käsitamatu ja ilmselt ka veidi elitaarne, siis loomulikult hakkasid 
tekkima needki, kes hakkasid kaikaid kodaratesse pilduma. Inimesed, 
kes tahtsid ka löögile pääseda ja tundsid ennast halvasti, et neid ei 
võetud jutule puhtalt sellepärast, et nad ei saanud aru paadi mittekõigutamise 
mentaliteedist. See oligi mõnes mõttes ajastu lõpp, kui igaühel 
tekkis ligipääs, lävi läks palju madalamaks ja ka lühemate pükstega mehed said 
paati astuda.

Tekkisid inimesed, keda keegi ei teadnud, kes olid anonüümsed ja kellel olid 
ambitsioonid, aga puudusid võimekus ja soov panustada. 

BBSummerid hakkasid samuti kasvama ja kihistuma. Ürituste lõppu tähistas see, kui hakkasid toimuma BB-üritused BB-ürituste sees. 

\question{Mida sa praegu teed? Kuhu see tee sind on toonud?}

Praegu olen juba viimased kümme aastat tegelenud veebipoodidega. Minu eriala on 
veebiarendused, täpsemalt veebipoed ehk e-kaubandus. 

Olen nüüd rohkem programmeerimise peal, sest tänapäeval on 
peaaegu kõik ühte- või teistmoodi seotud tarkvaraarendusega. Tol ajal ei 
olnud firmadel internetilehte nagu praegu. Tol ajal ei pakutud teenuseid 
interneti kaudu, aga nüüd pakutakse. Ja seega on tekkinud vajadus tehnilise võimekusega inimeste järele. Üks võimalus on värvata nad 
endale või siis palgata firma, kes sellega tegeleb.

\chapter{Meelis Roos}
%!TEX TS-program = arara
% arara: myindex

\index[ppl]{Roos, Meelis}
\textbf{\enquote{Kuidas sina arvutite juurde said?}}

Mina sattusin arvutite juurde isa töö juures kaheksakümnendate lõpus. Füüsikud ostsid omale mõned arvutid elektromeetria laborisse, eksperimendi juhtimiseks. Arvutid olid CAMAC\sidenote{\emph{Computer-Aided Measurement And Control (CAMAC)}. Elektroonikastandard andmete kogumiseks ja seadmete kontrolliks. Kasutusel (osakeste) füüsikas aga ka tööstuses} kontrolleriga vene DVKd\index{Arvutid!DVK}\sidenote{\begin{russian}ДВК, Диалоговый вычислительный комплекс\end{russian}. Nõukogude personaalarvuti, ühilduv DECi PDP-11\index{PDP-11} perekonnaga. Varasemad mudelid on tuntud ka kui  Elektronika MS-0501\index{Arvutid!Elektronika} ja Elektronika MS-0502}.

\textbf{\enquote{Kus see kõik sündis?}}

See juhtus Tartus\index{Tartu}, Tartu Ülikooli\index{Tartu Ülikool} juures. Isa oli Tartu Ülikooli füüsika institituudis\index{Tartu Ülikool!Füüsika instituut} füüsik. Nad tegelesid elektroonika mõõteseadmete välja töötamisega ja näiteks said mingisuguse auhinna elektromeetri eest, mis eriti väikesi laenguid registreeris. Näiteks visati pastaka kuul, millel oli mingi laeng kusagilt läbi ja mõõdeti selle laeng mööda minnes ära. Neil oli lahe töögrupp, elektromeetria sektor, mida vedas üks mees, kes sellesse ilmselt uskus. Noored ülikoolist tulnud mehed tegid koos lahedaid asju, minu arusaamise järgi. 

Et katseid juhtida ja mingeid andmeid töödelda oli spetsiaalne lisablokk, mis käis arvuti külge. Seal sees oli analoog-digitaaal muundur (võibolla vastupidi ka aga igatahes nii pidi neid kasutati). Nad õppisid programmeerida, et suuta oma eksperimendi andmeid reaalajas kätte saada. 

\textbf{\enquote{Aga mis arvuti see selline oli, mis suutis andmeid niimoodi reaalajas kätte saada?}}

DVK-2M. Vene LSI-11\sidenote{DECi PDP-11 perekonna liige, tuntud ka kui PDP-11/03. Masinat tutvustati 1975. aastal ja ta oli oma sarjas esimene, mille CPU oli integreeritud. Mitte küll ühele, vaid neljale Western Digitali poolt toodetud \emph{Large Scale Integraton (LSI)} kiibile). Meelise sõnul: \enquote{PDP-11 oli legendaarne DECi masin iidsel ajal enne meie aega}} analoogid. Peaaegu täpne kloon aga natuke kohapeal ka täiendatud. Programmide poolt ühilduv aga mitte identne. DVK peal jooksis näiteks DECi originaal opsüsteem RT-11\index{RT-11}. RT-11SJ oli igapäevane opsüsteem, see oli \emph{single job} ja RT-11FB sellel oli \emph{foreground} ja \emph{background}, millega sai taustal jooksutada mingisugust teist tegevust. 

\textbf{\enquote{Kui vana sa olid, kui su isa need arvutid omale hankis?}}

Põhikooli teises pooles. Ega mul ei olnud põhjalikku teadmist, mida selle arvutiga teha saab. Minu jaoks sai arvutiga teha kahte asja. Kõigepealt, kui ma tegin isale teksisisestustööd, näiteks sugupuu andmete sisestamiseks, siis sain ma pärast seda kuni õhtuni mängida. Lemmik mäng oli \emph{wall}\index{Mängud!Wall}, seina pommitamine mingi reketiga. Mind pandi kohe tööle, et miskit kasu oleks. Mis ma niisama aega raiskan. Programmeerima õpetati ka, eks nad ise ka õppisid. Basicus\index{Keeled!Basic} ja Fortranis\index{Keeled!Fortran} ja CASICus\index{Keeled!CASIC}. See viimane oli CAMACi kontrollerite programmeerimiseks mõeldud Basicu ja Pascali\index{Keeled!Pascal} vaheline keel\sidenote{Ilmselt peetakse silmas keelt formaalse nimega \emph{ANSI Standard Real-Time BASIC}, mille spetsifitseerib IEEE standard \enquote{726-1982 - IEEE Standard Real-Time BASIC for CAMAC}}. Selles mina ei sattunud programmeerima. Aga ma õppisin Basicus programmeerima. Minu parim programm oli programm, mis ajas inimesega eesti keeles juttu. Ütleb sulle ühe lause, sina ütled lause, tema ütleb lause, ütled lause ja tema valib juhuslikult ühe lause. Aga ta suutis mõnikord teemas püsida. Näiteks ütleb \enquote{Osta elevant ära} ja siis järgmised kaks lauset olid, et \enquote{Kõik ütlevad nii aga osta elevant ära}. Enne ta ei läinud järgmist lauset valima kui ta oli kaks vastust saanud. Ja teiste töötajate lapsed mängisid seda ja neil oli lõbus. See oli lahe emotsioon, et ma tegin midagi, mis teistele lahe oli. 

\textbf{\enquote{Huvitav, et sa kohe hakkasid mängu tegema ja seejures kohe midagi AI-sarnast}}

See tundus kõige lahedam asi mida teha! 

\textbf{\enquote{Need füüsikud pidid ju kähku õppima, sest reaalajas riistvarast andmeid lugeda on ju keeruline}}

Neid oli vähemasti kolm meest, kes õppisid programmeerimist ja neil oli üks natuke noorem pundis, kes oli nende põhiline arvuti-mees ja kes seda vist paremini jagas, kui teised ja kelle juures CAMAC kontroller oli. Arvuteid oli vist vähemasti kolm tükki selle labori peale aga üks oli see põhiline eksperimendi juhtimise oma. See, mida mina kasutasin oli niisama masinakirjutaja toas. Seda sai kasutada siis näiteks programmide sisestamiseks ja muidu andmetöötluse jaoks. Näiteks isa tegi sugupuu üles joonistamist arvutisse. Oli puukujuline puu, puu läks vasakult paremale ja siis sai rull-paberile välja trükkida ja siis oli pärast mitmeid rulle. Kui koolis tulid mingid tudengid ja andsid igaühele paberi, kuhu oli natuke templeiti ette tehtud isa ema ja lapse kohta, et joonistage oma sugupuu üles, siis mina palusin isal ühe koopia välja trükkida. 

\textbf{\enquote{Aga miks sa lasid ennast sellesse suhteliselt igava andme-sisestaja rolli suruda? Lihtalt, et saaks mängida?}}

Algul selleks, et sai mängida aga kui selgus, et ise programmeerida saab ka ja see on täitsa lahe, siis ma pigem mängimise asemel keskendusin sellele rohkem. Ma ikka mängisin ka vahel, ma ei jätnud mängimist päris maha. 

\textbf{\enquote{Mis selle programmeerimise juures lahe oli? Mis sind köitis?}}

Ma tegin ka mingeid asju käsitsi. Näiteks ESC koodidega printerile õigeid asju saates\sidenote{\emph{Epson Standard Code for Printers, ESC/P\index{ESC/P}} on Epsoni poolt maatriksprinterite jaoks välja töötatud (ja termoprinteritel siiani kasutusel olev) keel, mis võimaldab juhtida rastrivõimekuseta printerit. Keel sai oma nime sellest, et tema käsud algasid sümboliga ESC (ASCII 27). Näiteks ESC E lülitas sisse ja ESC F välja rasvase trüki} trükkisin oma õpiku silte, kus oli boldis ja suuremas ja väiksemas kirjas kõik erinevatel ridadel asjad kirjas. Ja siis üks ema tuttav tahtis oma firma logo visiitkaartidele. See firma logo tuli siis teisendada EPSONi printeri graafika ESC-jadadeks. Ma joonistasin selle \emph{bitmap}ina üles aga siis leidsime, et ei tasu vaeva ja seda logo ma ei teinud. Aga programselt oleks selliseid asju lihtsam teha. See oli näiteks koht, kus ma leidsin, et programmist võiks oluliselt kasu olla. 

Oli firma nimega Tensiid, mille logol oli mingi kolmeharjuline neljast aatomist koosnev molekul visualiseeritud. Keskel üks lömmis ja kolm tükki külgede pealt sees. Nad tegelesid keemiliste mingisuguste ühendite sünteesiga. Ülikooli keemiahoonest välja kasvanud firma minu arusaamist mööda. Muu hulgas käisid aegajalt reisidel. Sellest kasvas välja Tensi Reisi, keemia asemel tegeletigi reiside korraldamisega. 

Aga mina üritasin alguses Tensiidi logo arvutis joonistada ja mõtlesin, et programm võiks seda minu eest teha. 

\textbf{\enquote{Kuidas õppimine käis?}}

Ma sain mingisuguseid venekeelseid raamatuid. Osalt raamatukogust isa tõi, osalt oli ehk mõni raamat tal töö juures olemas.  Need olid enamasti kusagilt laenatud ajutiselt. Näiteks mul oli segadus ASCII koodi ja \emph{escape} koodidega. \emph{Escape} koode tuli terminalile saata ja printerile sai saata ja. Siis ma mäletan, et küsisin isalt nõu, et mis neil vahet on et kas see on seesama asi. Ja siis oli raamatuid erinevaid. Näiteks oli üks raamat Basicu kohta. Mingisuguse käsu kohta on mul siia maani meeles kirjeldus, mis minu meelest ei sobinud niisugusse raamatusse \begin{russian}\enquote{эта команда работает хорошо}\end{russian}. See käsk töötab hästi. Minu meelest oli see lati liiga madalale laskmine. Minu meelest peaks kõik hästi töötama. Asjad tuleks nii teha. 

\textbf{\enquote{Sul oli ju siis päris korralik vene keele oskus?}}

Jah, ma olin üheksandas klassis umbes kui ma programmeerimist õppisin ja kannatas vene keelset raamatut lugeda küll. Meil oli põhikoolis selline vene keele õpetaja, kellega pidi õppima niiet mul tõenäoliselt oli üsna normaalne vene keele oskus selle vanuse kohta. Ma käisin Tartu 12. Keskkoolis\index{Koolid!Tartu 12. Keskkool}. Meil oli üks ukrainlanna, vana Zinaida Tovkatš vene keele õpetajaks kelle kohta meie kirusime et ta on väga range ja isegi haige ei ole kunagi, et muudkui peab õppima ja muidu ei pääse. 

\textbf{\enquote{Kas keegi sind õpetas ka või käis ainult raamatu järgi see asi?}}

Isa õpetas mulle neid asju, mida tema teadis ja õpetas blokk skeeme ja see keskis kuni keskkooli ajani välja, et kui mina tegin programmi ja see ei töötanud, siis oli kask viisi debugimiseks. Üks oli see, et ma trükin ta rullpaberil välja ja loen õhtul kodus. Teine võimalus on see, et ma joonistan selle asja blokkskeemiks ja lähen näitan isale. Sealt pealt tema oskas vigu leida küll. Ja blokkskeemiks joonistamisel leidsin ma tihti vead ise ka üles. Et blokkskeemid oli asi, mida isa mulle õpetas sest tema õppis programmeerimist nendega. 

Ja isegi kui ma Pascal keeles kirjutasin, mida isa ei osanud, ma sain ikkagi isalt abi blokkskeemide tasemel. Sest isal oli hea loogiline mõtlemina ja ta seletas mulle minu vead ära küll. Mis veel lõbus oli, ma süütasin kodus katelt. Minu ülesanne oli keskkütte katla alla tuli teha. Ja süütamiseks oli toodud vanapaberit füüsika osakonnast. Ja seal oli teinekord mingeid arvuti väljatrükke, mida ma lugesin. Panin paberi kõrvale ja ajalehed ja muud läksid katla süütamiseks. Näiteks ma leidsin Minsk 32\index{Arvutid!Minsk-32}-e\sidenote{Minsk-32 loodi kuuekümnendatel, nagu nimigi ütleb, Minskis. Tegu oli mitmest mudelist koosneva Minsk suurarvutite sarja kõige võimekama esindajaga. Oli laialdasel kasutusel, kuni asendati seitsmekümnendatel IBM 360 kloonidega} mingisugused crash dumpid või mälu dumpid kolmekümne kahe bitised. Ma olin üllatunud, et vau, minul on 16-bitised PCd (see oli tol hetkel hiljem vist kui ma juba PC taga olin) aga nende oli juba siis 32-bitine arvuti. Ja seal olid Fortran programmid, mida ma huviga lugesin. Isa kõrvalt ütleb, et ah, need ei ole suurt midagi väärt eriti et see mees, kelle programmid need on ei oska veel eriti programmeerida et tema Fortran programmide pealt pole eriti mõtet eeskuju võtta. Aga põnev oli neid lugeda sellegi poolest. Fortranit õppisin keldris katla kütmise juures!

\textbf{\enquote{Miski pani sind tulehakatust lugema, mis see oli?}}

Seal olid uued põnevad asjad!

\textbf{\enquote{Kas selle asja juurde mingi kirjanduse või muusika huvi ka käis?}}

Ulme huvi natuke oli. Mul õnnestus saada venekeelsed Asumi seeria raamatud, mida oli rohkem kui kaks esimest. Asumid mulle meeldisid ja ühe isa sõbra käest laenasime venekeelsed ülejäänud Asumi raamatud. Ja mul õnnestus vene keeles raamatut lugeda ja ma olin selle üle sügavalt üllatunud. Isa algul luges neid ise ja mina lugesin ka vist midagi neist. Niiet ulme huvi oli küll aga see ei olnud niimoodi väga sügavavalt. Seda oli valdavalt nii palju, kui kodus sattus Mirabili ja mida iganes seal ulmekaid olema. Need said kõik läbi loetud aga see ei olnud esialgu kuidagi eriti seotud arvutitega. Arvutid olid asi, mis tuli reaalsest maailmast. Näiteks sõitsin bussiga koju ja ükskord Pärmivabriku peatuse juures mööda sõites parajasti ema seletas mulle arvuti viiruste kohta mida ta oli lugenud kuskilt Horisondist või mõnest niisugusest kohast. Ja väga põnev oli. Parajasti sõitsime Pärmivabriku peatusest mööda, kui ma esimest korda arvutiviirustest kuulsin. Seda ma mäletan. 

\textbf{\enquote{Kas sa olümpiaadidel ka käisid?}}

Jaa, käisin. Neljandast klassist saadik käisin matemaatikaolümpiaadil. Seal nägi natukene tuttavaid. Oli naljakas korrelatsioon: need lapsed kellega ma olin koos käinud ülikooli töötajate lasteaias, neist nii mõnigi oli seal olümpiaadidel. Järgmine laine sellega oli minnes keskkooli. Miks ma läksin vanast koolist ära? Vanas koolis oli nii, et keskkoolis pidi tulema kaks klassi. Reaalkallakuga ja humanitaarkallakuga. Ja humanitaarkallakuga pidi see \enquote{A} ja eliitklass tulema kuhu paremad õpilased lähevad ja ülejäänud võiks minna sinna reaalkallakuga klassi. Ma leidsin, et see on lati alla laskmine, et ma tahaksin ikka paremat. Mind kutsuti Nõkku\index{Koolid!Nõo Keskkool}. Hilisem ülemus Cyberneticast\index{Cybernetica} toonane Nõo kooli direktor Uuno Puus\index[ppl]{Puus, Uuno} saatis laiali kõikidele olümpiaadikutele Nõo kooli kutseid. Sain ka. Kaalusin. Oli kaugel. Raske. Siis selgus, et esimene keskkool Tartus\index{Koolid!Tartu 1. Keskkool} on ka täitsa kõva. Helistasin kooli ja küsisin, et kas teil arvutiklass on. Direktori võttis vastu ja reklaamis, et neil on väga hea arvutiklass. Selle peale ma otsustasin, et ma lähen esimesse keskkooli. Viisin paberid esimesse keskkooli, kui sisse astusin 1990 oli juba Hugo Trefneri Gümnaasium\index{Koolid!Hugo Trefneri Gümnaasium|see{Tartu 1. Keskkool}}. Olid väga head arvutid. Oli arvutiklass ja Juku\index{Arvutid!Juku} klass. 

\textbf{\enquote{Sul oli siis selge arusaam, et sa just sinna kooli tahad minna?}}

Jah, ma läksin nimelt sinna. Selle kohta tegi ajaloo õpetaja meil kunagi pisikese kiire küsitluse üheksanda klassi kevadel. Et paljud teist siia jäävad ja paljud lähevad kuhugi mujale. Ja mina olin see, kes leidis, et ma tahan ise oma tulevikku kujundada, et mulle sobib see asi paremini. Ja siis ta küsis kolme tema nina all oleva tegelase käest. Esimeses pingis sattusin mina istuma ja minu tagant kahe tüdruku käest, kes olid ka kätt tõstnud, et lähevad mujale küsiti, mis nad teevad. Need oli täpselt need kolm, kes läksid esimesse Keskkooli. Niiet kõik, mis ta küsis, sai vastuseks, et lähme ära esimesse keskkooli. Nemad läksid teise paralleeli, sinna bioloogia-keemia harusse. Aga see tundus olevat sinnakanti, et umbes see vanus oli koht, kus mõned hakkasid ise mõtlema oma tulevikule ja planeerima ja mõned lasid isevoolu teed minna. Et mõned olid need, kes planeerisid. 

\textbf{\enquote{See oli see aeg, kui ühiskonnas hakkas juba muutus tulema, eks ole}}

Natuke oli juba varem selles mõttes, et koperatiivid olid varem ja asjadest tohtis rääkida varem. Selle sama üheksanda klassi jooksul ma jõudsin kaks korda kirjutada mingisuguseid referaate millest võib olla aasta varem oleks vanematel pahandus tulnud. Aga siis juba tohtis. Sellesama õpetaja kohta oli teada, et ta on üks paras punane. Aga temale ma need referaadid kirjutasin ja sain kiita, mis oli üllatav. Ma olin üllatunud, ma mõtlesin, et tuleb kaitsta kuidagi oma seisukohti. 

\textbf{\enquote{Kas sind keskkooli ajal tööle ei tõmmatud kuhugi?}}

Ainult natukene käisin. Tiražeerisin isa töö juures elektromeetrite trükkplaate. Joonistasin ahjulakiga ja risti ära lõigatud otsaga süstlaga rajad, söövitasin plaadi ära, tinatasin ära ja jootsin sinna peale kõik elemendid vastavalt skeemile. 

\textbf{\enquote{Aga see tahab ju käelist oskust ja elektroonikahuvi, kust sul see?}}

Seitsmeaastaselt oli mulle vist isa töö juures kolb kätte sattunud esimest korda kui ma suvalisi tükke kokku jootsin. Eks ma oskasin kolbi hoida ja elektroonika huvi mul oli. Aga elektroonikat ma ei osanud, ei ole kunagi ära õppinud analoogelektroonikat. Üldisi põhimõtteid tean aga ise midagi teha ei ole osanud. 

Digielektroonika oli seal kõrval. Kui keskkool hakkas läbi saama ja ülikooli oli vaja minna, siis mina olin neljandast klassist peale kindel olnud, et ma lähen füüsikat ja nimelt elektroonikat õppima. Aga siis mingid arvutid tulnud, kah põnev elektroonika värk. Aga arvuteid sai matemaatika poolt ka õppida. Mul oli kuhugi ilma eksamiteta sisse saamisesd, äkki matemaatikasse ja füüsikasse olümpiaadi tulemuste pärast või midagi ja ma otsustasin matemaatika kasuks sest füüsika osakonnas ma olin kogu aeg kohal ja mulle ei meeldinud see. Tundus nihukene, et kui midagi ära tahta teha siis peab ise muudkui tegema. Oli nihukesi saarekesi, kes tegelesid oma kitsa erialaga aga laiemat kandepinda ma ei märganud seal. Oli töögruppe, kes olid vingel tasemel ja tegelesid oma asjaga. Aga võibolla ma ei sattunud õigete inimestega kokku aga tundus, et pigem nihukene nagu oleks seisev konnatiik et igaüks on seal kinni, kus on ja nii on. 

No seal oli huvitavaid ja põnevaid asju ka. Näiteks olid füüsika päevad, kus rääkis Undo Uus\index[ppl]{Uus, Undo}, keda mu isa käis kuulamas, rääkis materialismi ümber lükkamisest filosoofiliselt. Isa tuli koju, jutustas. Panin kõrva taha. Selliseid asju oli sealt ikka päris mitmeid. Sellist füüsikalist maailmapilti tuli vanemate kõrvalt üksjagu, see oli mul olemas. 

\textbf{\enquote{Kuidas sa siis ikkagi matemaatikat sattusid õppima? Lihtsalt seepärast, et sai eksamiteta sisse?}}

Füüsikasse ma oleks vist ka saanud ilma eksamiteta. Eksamid ei oleks probleem ka olnud, ma arvan. Lihtsalt laisk. Laisad me olime kõik. Keskkoolis klassijuhatajal tuli üheksandas klassis üritada meile ikka auku pähe rääkida, et poisid olge tublid ja võtke tehke need eksamid ikka ära, siis saab medalile pretendeerida, muidu ei saa. Vaja oleks ju medaleid ka. Siis me tegime vist kolm medalit klassi peale või midagi. Mina sain hõbeda. Ma täpselt ei mäletanudki. Kunagi hiljem kooli koduleheküljelt lugesin, et ma hõbemedali sain. Seda ma mäletasin, et medal oli aga mis medal, seda ei mäletanud. Polnud oluline, see tuli iseenesest. 

\textbf{\enquote{Ühesõnaga, matemaatikasse sa läksid seepärast, et füüsika tundus natuke seisev vesi olevat?}}

Jah. Ja ma olin kuu aega enne paberite sisse andmist kindel, et matemaatikasse ma küll ei lähe. Me käisime Moskva\index{Moskva} lahtisel olümpiaadil matemaatikas koolist tiimiga. Seal olid mingid doktorandid, kes tegelesid meiega. Seal ühtlasi toimus \begin{russian}Международная конференция старшикласников "Наука, природа, человек"\end{russian} kus keskkooli õpilased said ise asju esitada, mis nad olid teinud. Keegi oli teinud kiiret vektorgraafikat, et voldime siin kuubikut kiiremini, kui AutoCAD või mis iganes wirefreimis. Ja ägedaid asju oli tehtud. Seal oli mingit Hollandi rahvast, oli rahvusvaheline küll. Seal need doktorandid, kes meiega tegelesid, olid nihukesed parajad uhuud. Näiteks tuleb tegelane hommikul tahvli ette triiksärk on lükatud alukate sisse, alukad ulatuvad kümme sentimeetrit pikkade pükste peale välja ja tuleb niimoodi tahvli ette. Ma leidsin, et vot matemaatikuks mina küll ei lähe. Aga siis ma mõtlesin ikkagi ümber. Matemaatikuks ma ei tahtnudki, ma läksin neid arvuteid õppima Matemaatikateaduskonna\index{Tartu Ülikool!Matemaatikateaduskond} poolt. Mitte eleketroonika poolt aga programmeerimise poolt. 

\textbf{\enquote{Kuidas sulle ülikooli üleminek tundus? Sa ütlesid, et olla laisk olnud aga minu mälu järgi pidi ülikoolis kohe hakkama tööd tegema?}}

Jaa. Keskkoolis ma sain endale lubada laisk olemist isegi seal eliitkoolis, no mingil tasemel vähemasti. Ja ma sain keskkoolis arvutimängude mängimise isu täis mängida. Ostsin omale üheksanda klassi lõpus ZX Spectrum-i\index{Arvutid!ZX Spectrum}\sidenote{ZX Spectrum oli Sinclair Research'i poolt 1982. aastal Ühendkuningriigi turule lastud 8-bitine personaalarvuti, mõeldud peamislt koduseks kasutamiseks. Selle kloone liikus Nõukogude Liidus hulganisti, skeemid olid kogunisti hobiajakirjades avaldatud} Leningradi turu klooni 1500 rubla eest kui rubla juba kukkus. Siis oli suur rahanumber aga ma sain mängida täis oma mängimise isu. Joystick sai peeneks mängitud ja plastmassi paigatud alumiiniumiga. Tuttav treial tegi sinna uue varre, pärast kippusid kontaktid läbi põhja tulema. Aga Spectrum oli nii hea arvuti, sellest sai aru igat pidi! Basicus sai programmeerida, Assembleris\index{Keeled!Assembler} sai programmeerida Z80 peal. Sellest sai täiesti aru saada. Ja elektroonikast võib peaaegu üleni aru saada, välja arvatud videopildi kildi genereerimise osa, see ULA kivi või selle realiseerimine niisama lause-elektroonikana nagu selles vene variandis oli kui seda kivi ei olnud kloonina võtta. Niiet ma sain sõbra Sinclairi diagnoosimisega hakkama, et sul on ROMi see ja see jalg lahti ja ei anna kontakti. Et sellest tulevad tähtedel vertikaalsed kriipsud läbi nagu dollarimärgid. Sest ROMis oli see tähtede tabel ja kui seal bitt oli maas, siis on vertikaal. Seal oli kaks ROMi kivi et sellel ROMi kivil see jalg peab järelikult mitte kontaktis olema. 

\textbf{\enquote{See tähendab, seda, et sa pidid neid asju põhjalikumalt uurima?}}

Skeeme ma ikka kuskilt raamatutest ja niimoodi nägin. Keskkooli lõpus, kui Venemaal käisin, ostsin metroost omale raamatu \begin{russian}Введение в схемотехники IBM PC / AT\end{russian}. Venelased olid 286 skeemid välja ajanud arvuti järgi ja üles joonistanud. Neil oli seal viga minu meelest. Mingi reset signaal, selles oli aktiivne null versus aktiivne üks kusagil vist segamini, mul on nihuke mälestus. See oli lõbus igatahes avastada niisugust asja trükitud raamatus. See oli seesama kord, kui me olümpiaadil käisime ja konverentsil, millest me enne ei teadnud midagi, kui me sinna kohale sattusime. Meil ei olnud mingeid ettekandeid, me kuulasime niisama, mis räägitakse. Ja vaatasime, mihukesed on kenamad tüdrukud. Üks vene Maša oli kõige kenam. 

Olümpiaadil me eriti hiilgavaid tulemusi keegi ei saanud. Mina sain meie pundist kõige parema tulemuse, sest ma ei joonud eelmisel õhtul alkoholi. Aga seda oli seal saada ja seda käis ringi ja siis järgmine hommik oli pohmakas ja siis inimesed ei esinenud oma võimete tasemel. Ja mina olin meie omadest parim kuigi seal oli meil vähemasti üks nendest meestest, kes veel oli kaasas, oli parema peaga, kui mina. Minu jaoks oli õppetund, mida rõõmsalt teistele edasi jagada, et näe, olümpiaadi tulemus sõltus selgelt sellest. 

\textbf{\enquote{Räägi palun ülikoolist. Me sattusime seal 1993. aastal kokku, kuidas sulle see matemaatika tundus, mida me kohe esimese semestri alguses saama hakkasime?}}

See oli üks suur kukkumine. Ma mõtlesin ülikooli tulles, et ma tean, mis on reaalarv näiteks. Siis tuleb Matemaatilise Analüüsi esimene loeng, kus hakatakse neid defineerima. Ja kõike pidi algusest hakkama defineerima, ainult nende definitsioonide otsa ehitati kogu seda kõike. See tahtis palju harjumist ja palju tööd, mina ei olnud harjunud tööd tegema. Mina mõtlesin, et ma oskan programmeerida, kui ma ülikooli tulin. Aga asi, mis mulle näitas, et on veel palju, oli Rein Pranki\index[ppl]{Prank, Rein} mat. loogika õppeprogrammid, kus tehti tõestuspuu layouti ja ma mõtlesin, et vau, puu layouti niimoodi teha ma ei oska. Me õppisime seda küll alles hiljem umbes kolmandal kursusel Varmo Vene\index[ppl]{Vene, Varmo} funktsionaalses programmeerimises, kus me mingi minimaxi ülesande tüübi näite ülesandeks tegime puu layouti. No seda oleks saanud rekursiooniga esimese kursuse järel ka ehk kuidagi tehtud saanud. Aga see oli jah näide sellest, et ei ole kõik ikka triviaalne jõuga peale ja teeme ära. 

\textbf{\enquote{Matemaatiline analüüs, eriti Matemaatiline Analüüs II, võttis meil kursuse peal palju rahvast hõredamaks, see tahtis harjumist saada}}

Ja algebra tahtis ka. Kogu see matemaatiline lähenemine, et me ehitame asju üles mingite definitsioonide ja aksioomide millegi otsa. Kogu see asi tahtis kõvasti tööd. Ja ma kukkusin esimesel kursusel haiglasse. Et sessi ajal ma ei jõudnud mõnesid eksameid tehtudki, tegin neid alles järgmise semestri sees. Käisin dekaanilt küsimas sessi pikendust, sest vanemad õpetasid, et nii tuleb teha. Siis dekaan ütles, et meie ajal enam niisugust asja pole, lihtsalt tehke need eksamid ära, kuidas saate. 

\textbf{\enquote{Mis hetkel oli võimalik minna arvutiteadust õppima?}} 

Selleks oli kas esimese aasta järel spetsialiseerumine. Mingid põhimoodulid oli vaja ära teha ja siis vist sai. Kuna ma sain need vist kokku, siis mina kaldusin üldisest õppekavast kõrvale sellega, et mina läksin võtsin koos aasta vanematega põnevaid arvuti aineid. Käisin aasta vanema rahvaga koos kuulamas asju, mis olid lahedad. Peast ei mäleta, aga igasugu aineid, mis meil seal oli. Ja siis järgmine aasta tuli mul võtta siis need asjad ka, mis õppekavast puudu olid. Minu oma kursus oli need ära teinud, mina tegin neid siis koos aasta noorematega. Mingeid tõenäosusteooriaid ja mingisuguseid matemaatika aineid. 

Juhtus ka seda, et ma kirjutasin maha kodutöö programme teiste pealt. Meil oli mingisugused algebra ja analüüsi numbrilised meetodid, kus me arvutusmeetoditega numbriliselt tegelesime. Ma sain algoritmidest aru, nad ei pakkunud mulle algoritmi tasemel pinget ja ma ei viitsinud neid teha. Kui ma olin aru saanud, mis seal tehakse, siis sellest piisas. Siis oli üks lahke kaastudeng Jane, kelle programme ma kasutasin selleks, et neid esitada. Muutsin vist natuke treppimist ja muutujad nimesid mõnes. Mäletan, ma kirjutasin ühele kommentaaridesse üles \enquote{Viimati modifitseerinud Meelis Roos}. Eks see praksi juhendaja, et neid üksteise pealt üksjagu maha võetakse ja kuna ta lasi endale ette seletada, mida see programm täpselt teeb ja algoritm, siis sellega polnud probleemi, ma sain kõik asjad ilusti tehtud. Kirjutasin programme tüdrukute pealt maha, sest ma ei viitsinud programmeerida. 

\textbf{\enquote{Kas see ülikooli arvutuskeskus seal Liivi tänaval ei neelanud sind kuidagi endasse, nagu ta nii mõnedki neelas?}}\index{Tartu Ülikool!Matemaatikateaduskond!Liivi tänava õppehoone} 

Neelas ka mind aga natuke teistel viisidel. Mina ei kadunud ära Muda\index{Mängud!Muda} mängima. Muda oli küll tore, aga siis kui ma kirjutasin oma telneti klienti, siis sai seda Muda serveri vastu testida näiteks. Selleks oli Muda tore. 

\textbf{\enquote{Miks sa kirjutasid oma telneti kliendi?}} 

Võrguprogrammeerimise harjutamiseks. Tahtsin osata sokliühendusi igasuguseid teha. Ma kirjutasin oma netcati laadset mingit asja, mis telneti handshake'i ei teinud mingisugust ja ei osanud echo offi ja selliseid advanced featuure vaid lihtsalt sokli kuhugi ühendas. Sellise asja kirjutasin endale, et torkida igasuguseid asju. Seal olid mingid mured stiilis kui pikkade pakettidega asju saata ja vastu võtta ja TCP võis selle suvalisel koha pealt ära hakkida. Ei saanud eeldada, et kui teiselt poolt rida sisse kirjutakse, et sa täpselt rea suuruste tükkidena kätte saad. See oli põnev.

Aga mind neelas see arvutuskeskus natuke teistmoodi. Teisel korrusel Ülo Kaasiku\index[ppl]{Kaasik, Ülo} kabineti kõrval oli magistrandide arvutiklass, kus olid värvilised Sunid. See oli ette nähtud magistrandidele aga kellelgi ei olnud eriti probleeme, kui mina ka sinna imbusin. Aegajalt seal ei olnud kohti ja tuli ette, et ma kellelegi kohta pidin loovutama mõnikord aga enamasti töötas. Aasta vanema Raul Tölbiga\index[ppl]{Tölp, Raul} istusime seal koos ja seal sai õpitud ära Unix. 

Ja lõbus oli omakorda see, kuidas ma üldse sinna Unixit kasutama sattusin. Seda ma võin lausa rääkida, kust on pärit minu kasutajanimi mroos. Minu esimene online konto oli masinas vask.ut.ee\index{Masinad!vask.ut.ee}. See oli VAX\index{Arvutid!VAX}\sidenote{Arvutisari, mille töötas DEC välja seitsmekümnendate keskel. Siiani üks kõige tuntumaid omalaadseid arhitektuure, oli ta PDP-11 edasiarendus, peamiselt mälu virtuaalse adresseerimise suunas. \emph{VAX - Virtual Address Extension}} tüüpi arvuti VMS\sidenote{VAX arvutite \enquote{kohalik} operatsioonisüsteem} opsüsteemiga. Nihuke umbes kuupmeetrine kast pluss kettad kõrval. Teine VAX oli rubiin.physic.ut.ee\index{Masinad!rubiin.ut.ee} füüsika majas. See oli MicroVAXm sahtlitumba suurune masin ainult. Vot need olid VMSid. Esimesel kursusel, selle asemel, et sessi ajal õppida, mina olin raamatukogust võtnud omale VAX VMSi raamatu ja õppisin VMSi. Seal oli huvitavaid asju! Näiteks olid struktuursed failid. Sa võisid tekitada tühja faili, millel on ette antud kirje struktuur. Opsüsteemi tasemel oli record management system, RMS, millega mingis keeles kirjeldati struktuur ära ja tekitati selle kirjelduse järgi fail. Fail võis olla ka tühi aga tal oli struktuur olemas. 

Õigusi oli seal jõle palju ja keeruliselt. Kogu see õiguste süsteem op süsteemis oli keeruline tokenite süsteem. Windows NT\index{Windows NT} on selle sisemiselt pärinud või umbes niimoodi. Nii keerukas ei ole minu meelest kui VMSis aga kui ma nägin Windows syscalli create process koos portsu argumentidega, siis tuli tuttav ette sest VMSi sys\$createprocess oli umbes samasuguse listi argumentidega. sys\$ käis syscallide funktsioonide nimede ette lihtsalt. 

Sealt ma käisin näiteks Lynxiga veebis surfamas, tõmbasin mingeid faile FTPga, mida ma sain kuskilt kolmandat teed mööda kuidagi flopi peale. Käisin internetis veel midagi lugemas. Ma eriti ei programmeerinud VMSis. Kui vaja oli kursaõele Pascalis programmeerimist õpetada aga ainult VAXu klass vaba oli, siis ma näitasin talle Pascalis programmeerida VAXu peal. Ta oli väga üllatunud, et saab seda arvutit ka programmeerida. Aga sai. Aga seal oli lahe programm nimega swim, mis lasi ühe terminali peale multipleksida mitu akent. Akende suurusi lausa sai muuta. Ja see oli lahe ja sellega ma kasutasin kolme rakendust korraga. Aga swim kippus ajama terminali hanguma, kõditas vist mingit VMSi terminali draiveri bugi või mida iganes ja siis tuli leida administraator, keda tihti majas ei olnud või siis keegi sõber tudeng logis kuhugi üle võrgu rubiini\index{Masinad!rubiin.physic.ut.ee} ja talk-is Ville Hallikuga\index[ppl]{Hallik, Ville}, kes oli sealne VMSi admin ja kellel oli juurdepääs vaske olema ja kes sai tulla ja terminali päästa sest selle terminali tagant ei saanud keegi enam midagi kasutada, terminal oli hangunud. Tappis swimi ja mingid asjad ära seal niiet terminal sai jälle vabaks. Ja swim oli tülikas. Ja siis keegi rääkis, et arvutiteaduse instituudi SUNides on Unixis programm nimega screen, millega seda sama teha saab. Ja siis tekkis mõte, et kasutaks seda. Ma olin Unixit seni juba korra kasutanud. Math.ut.ee-s\index{Masinad!math.ut.ee}, kui tekkis online võrk, tuli 386BSD. Ja see upgreiditi 93. aasta lõpus mingile uuele tundmatule opsüsteemile. Sinna osteti 486 arvuti asemele, suure kahekigase scsi vindiga ja selle scsi kaardi jaoks ei sobinud enam 386BSD vaid pandi asemele mingi uus tundmatu asi nimega Linux. Versioon 0.99PL midagi. 

\textbf{\enquote{Kust selline asi sattus Tartu linna?}} 

No aga kust 386BSD sai? Internet oli ju olemas. Ja kasutajad koliti 386BSDst Linuxisse siuhti üle ja mul oli mingis Linuxis kasutaja. Jaanuaris umbes apgreiditi see Linux ära versioonile 1.0.2, kerneli versioon. Mul oli siis Linuxis kasutaja. Ma olin natukene nuusutanud Linuxit. Kui ma tahtsin seal Liivi tänaval Unixi screeni, siis math.ut.ee ühendus oli aeglane, lagis päris kõvasti. 9600ne ühendus jagatud paljude kasutajate ja meilide ja muude vahel. Siis ma küsisin sinna cs1, cs2, cs3 olid masinad ühise loginiga. Küsisin omale cs3-e (hilisem romulus.cs.ut.ee) konto ja põhjendasin seda, et tahaksin näppida mõnda mitte-Linux Unixit. Seal oli Solaris. Ja see tundus Toomas Soomele\index[ppl]{Soome, Toomas} piisavalt hea põhjendus. Toomas Soome kasutajanimi oli tsoome, ma mõtlesin, et ahaa, et eks Unixis käib see niimoodi. Küsisin siis omale tema süsteemi sama skeemi järgi kasutajanimeks mroos. Antigi. Seda ma olen sellest ajast edaspidi kasutanud igal pool. Isegi kui mul on kodus testarvuti  siis seal olen ma ka harjumusest mroos. Et tsoome mulle kasutajanime teeks tuli öelda, et ma tahan Solarist kasutada ja kasutajanimi peaks ka samas formaadis olema, et võimalikult vähe küsimusi oleks. 

Mul möödunud aastal oli väga sürr kogemus, kui kevadel võttis minuga ühendust Toomas Soome, kellel oli siiamaani magister tegemata. Ta tahtis, et ma juhendaksin tema magistritööd. Ma mõtlesin, et muna õpetab kana, et mida mina siin teen. Aga tal oli korralik tehniline töö olemas ja mina teadsin, mismoodi üks magistritöö peab enam-vähem välja nägema. Sellest teadmisest oli kasu, niiet see töö sai tal vormistatud magistritööks ja ta tegi selle edukalt ära. Aga algul lihtsalt oli väga sürr reaktsioon. Arvutiteaduste Instituudis\index{Tartu Ülikool!Matemaatikateaduskond!Arvutiteaduste Instituut} oli terve hulk rahvast, kes tegid hiljem magistrit. 

\textbf{\enquote{Kas sind teadust ei tõmmanud tegema?}} 

Ei, vot teadust tegema ei ole mind kunagi eriti tõmmanud ja keegi ei suutnud mulle ka auku pähe rääkida sel teemal. Väga ei proovitud ka. Meelitati erinevate viisidega, mingeid materjale ette söötes. Materjalid olid nii teadusega kui mitte-teadusega seotud. Näiteks Jaanus Pöial\index[ppl]{Pöial, Jaanus} jagas mulle omal algatusel kunagi Java Language Specificationi, et näe üks uus moodne asi. Et selliseid asju ülikoolist ikka sattus. 

Ma mäletan, ma olin rebane. Ma ei olnud veel spetsialiseerunud Arvutiteaduse Instituuti informaatika erialale. Aga mul oli vaja kusagil välja trükkida viietollise flopi pealt mingit tekstifaili. Ma lihtsalt vajusin kohale Liivi tänavale ja käisin mööda uksi koputamas. Äkki oli mingi laupäev ka või midagi või muidu õhtune aeg et seal ei olnud palju rahvast. Sattusin Mati Tombaku\index[ppl]{Tombak, Mati} ukse taha, kes lahkelt lasi trükkida. See oli raamatukogust mingisuguse kataloogi otsingu tulemus välja trükitud mingi raamatute otsimiseks. Äkki oli laupäevasel päeval vaja välja trükkida. Ja siis Mati Tombak oli see, kes lasi mul trükkida. Ja sellest tekkis nihukene tänutunne kogu selle ATI vastu, et siin on lahked inimesed. See oli minul nihukene esimene sedasorti kontakt. 

\textbf{\enquote{Millal sa tööle läksid?}} 	

Minu esimene ametlik töökoht oli Tartu Ülikooli Täpisteaduste Koolis\index{Tartu Ülikool!Täpisteaduste Kool} metoodik. See oli postmasteri töö tegelikult. Aga postmasteri nimelist ametinimetust ei olnud, oli metoodik. Korraldati programmeerimis kursust e-mailitsi koolides. Ja mina olin see, kes pidas arvet selle üle, kellel olid mis ülesanded lahendatud ja saata neile järgmisi. Arvutiõpetajad, kellele vastused saadeti ja kes parandasid saatsid minule seisu ja mina siis selle järgi saatsin edasi. Mina olen laisk inimene. Mina esimesel tööpäeval võtsin nägin pool päeva vaeva ja kirjutasin skripti. Panin kuhugi tekstifaili valmis nimed. Programm võttis sealt järjest nimesid ja saatis neile ära ja pidas arvestust, et kellele on juba saadetud, et kellelegi topelt ei saaks. Ja kui ma selle skripti rubiin.physic.ut.ee tollane Füüsikamaja Unixi server kõristas umbes pool tundi. Pärastpoole ma õppisin nice käsu ka ära. Aga see tähendas, et kogu minu edasine töö pärast selle skripti kirjutamist oli copy-paste meili seest sinna sisendfaili ja skript tööle lükata. Automatiseerisin oma töö lihtsalt ära. 

\textbf{\enquote{Aga kuidas sa sinna sattusid?}}

Ma arvan, et Indrek Jentson\index[ppl]{Jentson, Indrek} Täpisteaduste koolist kutsus mind, kes mat teaduskonnas oli vanem tegelane ja olümpiaadidega tegelenud. Tema kutsus mind. Ma läksin Täpisteaduste Kooli ukse taha, tuli Viire Sepp\index[ppl]{Sepp, Viire} vastu, kes juhataja oli, ütlesin, et tere, tulin töö lepingut tegema. Mis töölepingut? Ma siis seletasin, et Indrek Jentson saatis postmateri töölepingut tegema mind siia. Kuskil 95 või 96 algul, täpselt ei mäleta. 

\textbf{\enquote{See oli üsna vara ju? Tuleb häbiga tunnistada, ma läksin 93. aastal tööle juba}}

Te olite Veljo Haguga\index[ppl]{Hagu, Veljo} Korelis\index{Korel IN}, eks? Ma käisin Veljo töö juures vahel. Seal olid mingid mängud. Dune'i\index{Mängud!Dune} mängis Veljo näiteks õhtul näiteks millalgi kui ma sinna sattusin, vaatasin, kuidas see käib. Mängimisega ei olnud mul erilist suhet. Ma sain keskkooli ajal oma mängimise isu täis mängida Sinclairi peal ja lülituda juba programmeerimisele juba sellega, et ma tean, et see on palju põnevam asi. Ma kirjutasin näiteks oma binary editori näiteks, millega mängudest järgmiste levelite paroole välja nuuskida ja muid nihukesi asju. See oli juba keskkoolis, et sai igasugustel arvutiturva teemadel nuusitud ja huvi tuntud. 

Arvutiturva teema on mul keskkoolist saadik sees tõesti. Meil olid keskkoolis väga põnevad võidujooksud arvutiõpetajaga. Väga harivad. Näiteks oli õpetaja arvuti klaviatuur parooli all. Aegajalt tehti sellega meilivahetust niiet masinal klaviatuur oli lukus aga muidu masin töötas edasi IBM PS/2\index{Arvutid!IBM PS/2}\sidenote{PS/2 oli IBMi kolmas personaalarvutite põlvkond, mida tutvustati 1987. aastal. Paljud tolle masina innovatsioonid nagu näiteks VGA video muutusid \emph{de facto} standardiks pikkadeks aastateks}tedel oli mingi selline keyboard locki feature. Küll ma üritasin leida meetodeid sellest mööda hiilimaks. Kui ma sain mingeid skeeme kuskilt näha, siis mul tekkis idee, kuidas i8042 klaviatuurikontrolleri kaudu teha masinale sobivat warm booti, et sealt mööda hiilida aga klaviatuuri kontroller oli lukus edasi. Kirusin, et IBMi omad on kavalad olnud. See oli algul. 

Lõpuks selle arvuti parool saadi teada lihtsal viisil. Vaadatai üle selle arvutiõpetaja õla, kes aeglasemalt tippis. Kui see oli teada saadud, ega me sellega midagi ei teinud, see ei olnud eesmärk. Aga minul oli edasi põnevam see, kui keskkoolis viimasel aastal oli 386d kohale jõudnud ja nende C ketas, kõvaketas pandi kirjutuskaitse alla nii, et mingi spetsiaalne draiver laaditi config.sys-ist, mis tegi virtuaalse D draivi ja keeras kogu C ready-only-ks. Ja ma avastasin selle niimoodi, et mul oli mingi enda softi katsetamiseks see asi autoexec.bati või config.sysi panna või sealt midagi välja kommenterida, et minu asi ära mahuks või täpselt ei mäleta mis. Igatahes oli mul vaja sinna sekkuda. Kui ma sekkutud sain, siis ma pärast taastasin endise olukorra alati. 

\textbf{\enquote{Ka tol ajal mingit võrgu häkkimist ei toimunud?}}

Anto Veldre\index[ppl]{Veldre, Anto} rääkis jah, kuidas tema poisid ülikooli adminidel ruutusid käest ära võtsid. Tema jagas oma poistele modemeid ja terminale mis tulid kuskilt humanitaarabina. Meil oli üks modem. Õpetaja arvuti modemit ei puutud ja ühel poisil oli oma modem korra koolis kaasas, mida ta näitas aga vot me ei osanud nendega midagi teha ja LANi meil ei olnud. LAN tekkis meile alles 12. klassi kevadel, kui ma enam väga ei tegelenud sellega. OK, ma häkksin selle Lantasticu lahti social engineeringu meetodil. Sügisel pärast minu ära minekut oli kellelgi vaja saada Lantasticule juurdepääsu ja oli server masinas oli nihuke koht nagu network control directory. Seal olid andmebaasid binaarsena. Ja vot minu programm oskas käia ja binaarselt andmebaasi modifitseerida ja tekitada ühe administraatori juurde või panna kellelegi õigusi juurde või midagi. Ehk siis tuli meelitada noorem arvutiõpetaja flopi pealt ühte programmi käivitama seal masinas, viisakalt tänada ja puha. Tema poolt oli ka kõik OK. 

Aga varem oli see C-ketta kirjutuskaitse. Algul me käisime Nortoni Disc Editoriga kuskil seal config.sys algust ära sodimas, et seda ei loetaks. Aga siis oli paremini kaitstud järgmisel softil ja ei saanud sellega ka ligi ja siis oli vaja ikka flopi pealt bootida. Aga BIOS oli parooli all. A ja C, C ja A. Noh, siis järelikult muugime BIOSi paroolid lahti. See on obfuskeeritud kujul kirjutatud kuhugi CMOS mälusse. Selle sai sealt obfuskeeritud kujul välja lugeda. Ja masina ROM oli välja loetav. Ma võtsin ja disassembleerisin selle sourcereri nimelise disassembleriga ja matemaatika tunni ajal kirjutasin omale matemaatika vihikusse kõrval lehe peale programmi, mis seda obfuskeeritud asja lahti võtab. Järgmine tund oli ajaloo tund. Läksin ajaloo tunnist ära arvutiklassi ja realiseerisin selle ära ja muukisin BIOSi paroolid lahti. Mul tuli suur pahandus, sest see oli ajaloo tund, kust väga paljud olid puudunud ja õpetaja oli väga kuri ja keeras käkki. Mul oli pärast vaja järgi teha ja õnnestus ikkagi. Põhjendasime ikka kui väga hea programmi me tegime spetsifitseerimata, mis see oli. Et väga hea idee oli ja tuli lihtsalt minna arvutiklassi ja kohe ära teha. Parool oli obfuskeeritud striimsšifrina või bait haaval võibolla isegi, et otsast proovides järjest täht haaval sai selle ära arvata. Ma kunagi arvutiõpetajalt küsisin, et miks teil nii imelik parool on. Ja siis ta lahendas selle turvaprobleemi niimoodi, et delegeeris osa vastutust arvutiklassi haldamises ja võttis nigu appi arvutiklassi haldama, natuke. Väga hea pedagoogiline meetod, töötas. Ei häkitud enam, ei olnud enam huvi edasi jagada paroole, mida ma kätte saan. 


\textbf{\enquote{Aga kust sul see krüpto huvi?}}

Seda läks sealsamas kandis ka vaja. Näiteks meie õpetaja ässitas Norton Diskreet'i\sidenote{Diskreet oli tarkvarapaketi Norton Utilities 6.0 osa ning sisaldas paljuski kurikuulsat (Kevin Mitnicku\index[ppl]{Mitnick, Kevin} andmetel kasutati väidetud 56 biti asemel 30 bitist võtit, ka teised uurijad on osundanud mitmetele olulistele nõrkustele) DESi implementatsiooni} DESi\sidenote{\emph{DES - Data Encryption Standard} on sümmeetriline algoritm andmete krüpteerimiseks. Algoritm on oma väikese võtmeruumi tõttu tänapäeval kasutamiseks sobimatu (murti avalikult jaanuaris 1999), kuid oli siiski alates 1977. aastast USA föderaalse andmetöötlusstandardi (FIPS) osa.} kallale. DESist ma ei saanud jagu, ma ei saanud DESist arugi tol hetkel. Aga tema suunas. Ta oli üldse sedasorti kaval mees, et kui ta näiteks kunagi kui meil Veljo Haguga\index[ppl]{Hagu, Veljo} oli plaan kirjutada viirus. Me olime mingeid olemasolevaid viirusi disassembleerinud ja vaadanud, kuidas need käivad. 302 oli kõige lühem vist. Veljo oli mu pinginaaber. Õpetaja sattus pealt kuulma, kui me rääkisime viiruse tegemisest ja ütles, et kui teha, siis teha kohe selline stealth-viirus. Me olime väga nõus aga seda me ei viitsinud teha ja jäi tegemata viirus. 

Ta leidis meile muidu ka rakendust. Keskkoolis üldine taustaülesanne oli midagi arvutada. Minu arvutusülesanne oli arvutada arvu $e$ kahe tuhande komakohaga 30 sekundi 10Mhz 286 peal. Üks klassivend arvutas $\pi$-d tuhande komakohaga 60 sekundiga sest see koondus aeglasemalt. Ja kust tulid ajapiirangud? Õpetaja oli vaadanud, kui kiiresti temal vastus tuleb selle arvuti peal. Ma sain 35 sekundilise programmiga juba viie kätte, sest vastus oli õigem, kui õpetajal. Kuna need erinesid, siis ta võttis targa raamatu ja siis selgus, et minul oli õige. Mul oli selleks hetkeks 21 sekundiline proramm, mis käigu pealt suurendas mingi hetk arvutüübi pikkust. Algul tegi lühema tüübiga ja hiljem pikemaga, et kiiremini saaks. Aga see oli veel bugine. See veel ei töötanud õigesti. Ma kontrollisin oma enda programmi vastu. Ma olin minut aega töötava programmiga algul arvutanud tulemuse välja ja faili kirjutanud. Siis oli mul ka näiteks variant programmist, mis küsis, et kui mitme sekundiga oli vaja arvutada ja siis ütles hard-coded vastuse. Aga see ei sobinud õpetajale. Aga 35 sekundiline juba sobis, kui vastus oli õigem tema oma.  Minu 21-sekundiline ei läinud tööle aga õpetaja seepeale võttis ja kirjutas ise asja haljas assembleris ja sai kolme sekundiga. Muidu me kirjutasime Pascalis. 

Teine asi, mida me tegime, millega oli keskkooli ajal hulga nuputamist, oli interferentsi simuleerimine arvuti ekraanil. Kaks punktlaineallikat ringlainetega, kuidas lained liituvad, et tuleb interferentspilt. Seal ma nägin ka vaeva, arvutasin ruutjuurt assembleris Newtoni meetodil. Ma arvutasin iga ekraani punkti kohta pimesi selle faasi välja nii et ühtegi punkti näha ei olnud aga ma sättisin pixelite väärtused nii et palett oli seatud üleni mustaks. Arvutasin kõik väärutsed ära assembleris optimeeritud arvutusvalemiga ja õpetaja õpetas Newtoni meetodit sinna juurde. Oli abiks. Assembleris sai Newtoni meetodit! Oli väga hariv. 

Ja lõpuks ma siis ketrasin VGA paletti. Tehnilise dokumentatsiooni failid liikusid, seal oli kirjas, kuidas VGA paletti muuta ja ma seadsin siis paletti niimoodi, et need värvid, mis mul on liikusid sujuvalt heleduse järgi. Ja siis tulemus oli see, nagu oleks liikunud lained ekraanil. Ja see oli minu meelest tippsaavutus, see oli väga ilus sujuv liikumine selle kümne megahertsi juures, punkte üle arvutada poleks kuidagi jõudnud. Ja siis näidati mulle ühe teise õpetaja tehtud programmi. Tema ütles, et minu ideest see alguse saigi, et interferentsi simuleerida. Tema tegi Juku peal circle käsuga valgeid rõngaid üksteise ümber viie millimeetrise vahega. Need läksid mida edasi seda aeglasemaks ja minu reaalajalise sujuva pildi vastu ei olnud see midagi ja mul oli tükk tegu, et mitte naerma hakata. Aga kiitsin siis takka. Õpetaja suutis anda sellise ülesande, mille peale mul kulus ikka kaua ja sain palju targemaks. Õpetaja oli Tarmo Ainsaar\index[ppl]{Ainsaar, Tarmo}. Seesama, kes suunas meid viiruse kirjutamiselt ära ja kes lahendas selle BIOSi paroolide haldusteema probleemi meiega nii et probleemi ei tulnud. Väga hea õpetaja. Ta suutis meid suunata tegema õigeid asju nii, et me seejuures õpime ja paha peale ei lähe. 

\textbf{\enquote{Kuidas sa Cyberisse sattusid?}}

Ma töötasin HClubis ja mõtlesin, et mida võiks magistriks teha. Seal tegeldi hajusate andmebaasidega. Me saatsime SQL käsk haaval andmebaaside diffe üle võrgu mitmes suunas. See oli põnev, me saime selle tehniliselt lahendatud. Algul käis see mul üle UUCP, hiljem üle PPP ja POP3 ja SMTP. Mina ehitasin internetti sinna alla, oli ka põnev. Ja neid diffe siis saatsime ja tekkis küsimus andmebaaside konsistentsusest, mis tingimustel jääb ja mis tingimustel ei jää konsistentseks. Et kas me saame mingi eventually consistent mudeli sealt või mitte. Ma mõtlesin, et ma hakkan sel teemal magistrit tegema. Aga siis kõrval oli tulnud ... Ma sattusin HClubi tööle seoses sellega, et ma installisin sinna Linuxi serveri, gateway. Ja selle peal käis veeb ja meil ja kõik. 

Jah. Ma kusjuures mõtlesin, et ma võtan sul nööbist kinni ja küsin, miks sa sealt ära läksid. Kas ikka tasub. Aga ma vist ei sattunud sind tol hetkel tabama ja siis ma ei käinud sinu käest küsimas. Või küsisin kunagi tagantjärgi, mul on nagu mälestus olemas. Jaa, küsisin, sain vastuse ka. 

Ja siis HClubis interneti teemal, mis mind huvitas tol hetkel, ei olnud mul eriti kuhugi areneda. Seal ei olnud kellegi teise käest sedasorti asju õppida. Kui siis, ise õppida ja ehitada.  Asju, mida oleks võinud pisi ISP-na veel ehitada sinna ISDN sissehelistamiskeskusi kui oleks leitud raha ja et see rentaabel on. Nihukesi asju oleks ehk saanud aga siis samal ajal ma käisin mõnes koolis abiks Linuxit installimas ja käisin laenamas RedHati install plaati suvel Elmer Joandi\index[ppl]{Joandi, Elmer} käest Tartu lähedalt maalt. Tal oli see plaadina kohe olemas ja ei pidanud flopidega mässama. Ja Elmer ütles, et muide, Tarvil\index[ppl]{Martens, Tarvi} olla plaan Tartusse meiesuguste jaoks pesa teha. Ja siis mina käisin juunikuus umbes Tallinnas Cyberneticas\index{Cybernetica} Helger Lippmaa \index[ppl]{Lipmaa, Helger} juures, et tuleks magistrit tegema hoopis krüpto teemal. Ma mõtlesin, et näiteks pordiks Open SSLi Windowsile, sest mul oli Windowsi all krüptot vaja olnud aga ei olnud. Sellest konkreetsest ideest arvati kehvasti, et näe keegi vist juba on portinud ka midagi. Aga tule meile niisama progema, mitte-krüptot. Ütles Tarvi kõrvalt. Oleks peaaegu Küberisse tulemata jäänud. Aga Helger kutsus mu ikka turva asju tegema. Ja siis kutsuti mind Küberi väljasõiduistungile ehk \enquote{kvartalnajale}. 1997. aasta sügisel Arula motelli. Seal oli kutsutud kogu tulevane Küberi Tartu Andmeturbe Labor. Ja Viljar Tulit\index[ppl]{Tulit, Viljar} oma habemesse diktsiooniga ütles, et seda sa pead ikka ise suutma ära otsustada millegi järgi, kas sa tahad siia või ei taha, kui ma ütlesin, et segane on veel, kas ma tulen või ei tule. Ja siis räägiti ka tehnilistest teemadest ja mina läksin Küberisse tarkade inimeste juurde. Seal olid Arne Ansper\index[ppl]{Ansper, Arne} ja Viljar Tulit, kes oli kogenud süsadmin (kogenum, kui mina). Kui mina tegin näiteks tükk aega FTP otsingumootorit Nuuskur koos teiste tudengitega, siis Arne oli selle stiilis nädala otsa õhtutega ära teinud või niimoodi. Arnel oli ka Vosa nimeline FTP otsingumootor Eesti FTP serverite kohta. Vosa nagu \enquote{Vanaisa Oli Sulle Archie\sidenote{Archie oli üks esimesi interneti otsingumootoreid, mis võimaldas otsingut üle FTP arhiivide}}. Tal oli ainult veebiliides, meil oli muid liideseid ka. Meil oli telneti liides ja archie prospero protokolli\sidenote{Archie kataloogides navigeerimiseks loodud protokoll, mida võib pidada tänapäevase www protokolli eellaseks. Prosperot kasutades võis terve internet välja näha, nagu üks suur ühine kataloogipuu} liides, millega vana archie klient töötaks ja meililiides ja. Meil oli võimas vinge süsteem tehtud kamba peale. Kõike ei teinud mina, teised tegid ka. Ma olin lihtsalt üks vedajaid lõpuks, kes tegi kõige rohkem tükke. Ja sellega selgus, et Arne on tark. Seal oli veel asju, millega see selgus. Näiteks tal oli Fido ja Interneti vaheline gateway. Ma olin selle kaudo Fido lugejda. Ma pole päris Fidonetti kunagi näinudki. Minu jaoks Fido oli just another NNTP server stiilis keeks.ioc.ee. Sinna tuli kasutajanime ja parooliga läheneda ja sai tavalise newsreaderiga lugeda ja kirjutada. Minu jaoks oli Fido teenus üle interneti, mida vahendas Arne tehtud süsteem. 

\textbf{\enquote{Mis sa praegu teed?}}

Praegu ma olen Küberis turvainsener ja praktikas ka tarneinsener, kes pakendab asju ja ehitab mingeid keskkondi automatiseeritult nende otsa. Õpetan ülikoolis, olen ülikoolis hajussüsteemide külalislektor, õpetan operatsioonisüsteeme baaskursusena, andmeturvet baaskursusena ja magistrandidele õpetan turvalist programmeerimist. Kuidas teha nii, et auke poleks koodis. Mõni ikka kuskilt leidub aga eks seda ole aja jooksul endale piisavalt vastu tulnud. Andmeturbe kursus sai tehtud siis, kui ma olin magistrand Helger Lippmaa juhendamisel. Helger ütles, et kuule, et sa võiks teha sellise andmeturbe kursuse ülikooli. Mõeldud tehtud. Tegingi. Kellegagi eriti nõu ei pidanud. Küberi turvaraamatu võtsin vihjete jaoks aluseks. Infosüsteemide Turve Esimene köide, oli vist esimene valdavalt, võibolla esimene ja teine. 


\textbf{\enquote{See tundub olevat nii sinu moodi, et võtad, teed ja saab väga hea}}

Parim kiitus, mis ma aineturbe ainele kuulnud olen oli kunagi kui hakati küberkaitse magistrikava tegema. Oli Tallinnas sel teemal siis koosolek. Ja oli häda, et kui me tahame neile õpetada seda, seda, ja kõiki asju, et see ei mahu meil ainetesse ära. Ja selle peale oli vist Enn Tõugu\index[ppl]{Tõugu, Enn}, kes ütles, et kuida, et Meelis jõuab andmeturbe kursuses neist kõigist asjadest rääkida, et mahutame ikka magistrikursusesse ka ära. Mis sest, et põhjalikumalt aga küll me mahutame. Et see oli hea kompliment kursusele, et Meelis räägib neist kõigist. 




\chapter{Jaan Tallinn}
\index[ppl]{Tallinn, Jaan}

\question{Kuidas ja umbes millal sa jõudsid arvutite juurde?}

Ma mäletan  seda aega, kuskohas mul isa hakkas kaheksakümnendatel Soome vahet 
käima, seal mingisuguseid
filmi ja videorežiitöid tegemas. Ja ta tõi mulle erinevaid ajakirju, neid oli 
hea, odav ja võib-olla isegi tasuta tuua. Neist päris mitmed olid 
arvutiajakirjad ja see tundus kohe olema väga põnev. Armumine esimesest 
pilgust. Algkooli kas  viimases või eelviimases klassis juhtus selline asi, et 
üks kooli lapsevanem valis mind ja mõningaid mu klassivendi (sealhulgas näiteks 
Priit Kasesalu\index[ppl]{Kasesalu, Priit}) eksperimentaal-katsejänesteks, et 
viia  meid õhtuti  kuskil Kopli servas asuvasse Sideministeeriumi 
Arvutuskeskusesse\index{Sideministeeriumi Info- ja Arvutuskeskus} ja lasta seal 
suurte \emph{mainframe}-de peal lahti ja vaadata, mis juhtub. Nii et inimkatse 
tulemus. 

\question{Mis kool see oli?}

See oli Lasnamäel kuuekümnes keskkool\index{Koolid!Tallinna 60. Keskkool}.

\question{Kust selline mõte tuli, et peaks inimestega niimoodi tegema?}

Seda ma ei tea, aga sa võid ta enda käest küsida. Ta nimi on Jüri 
Malsub\index[ppl]{Malsub, Jüri}, talle meeldib sellest väga pikalt rääkida. 
Seal seltskonnas olin mina, Priit Kasesalu ja veel kaks klassivenda, kelles 
ühest (Mikk Orglaan\index[ppl]{Orglaan, Mikk}) sai ka arvuti-ettevõtja. Neljas 
oli Martin Kruusvall\index[ppl]{Kruusvall, Martin}, kellele sai selgeks, et 
numbrid teda väga ei paelu, et ta on rohkem nagu luuletaja tüüp.

Keskkoolis liitus selle seltskonnaga Ahti Heinla\index[ppl]{Heinla, Ahti}. Siis 
ma olin juba läinud Tallinnas Gustav Adolfi Gümnaasiumi, toona esimesse 
keskkooli\index{Koolid!Tallinna 1. Keskkool} ja hakanud tõsiselt tegema 
olümpiaadidega. Meie füüsikaõpetaja, kadunud Vilma Kukrus\index[ppl]{Kukrus, 
Vilma} ühel hetkel (peale seda, kui Ahti oli vabariikliku füüsika olümpiaadi 
kinni pannud) rääkis Ahti pehmeks, et mis sa seal Õismäel passid, tule parem 
Gustav Adolfisse. Nii, et ta tuli meile mitte esimese keskkooli klassi, vaid  
teise  ja me saime suhteliselt kiiresti headeks sõpradeks. Ilmselt mina, kes 
see muu võis olla, kutsusin teda sellesse seltskonna, kellega me olime juba  
mõned aastad seal Kopli piiril tegutsenud. 

\question{Ja kogu selle aja te käisite \emph{mainframe}-i näppimas? Mis te 
tegite nendega?}

No \emph{mainframe}-d said muidugi kiire lõpu, kuna arvutustehnika arenes. 
Esimene mitte-\emph{mainframe} platvorm, kuhu me kolisime, oli sealsamas 
keskuses õhtuti meisterdatud riistvaraplatvorm nimega 
Entel\index{Arvutid!Entel}, mis oli selline CP/M masin. Ta kasutas mingisugust 
Intel 8088 protsessorit, või mingit Vene klooni sellest kuulsast 
kaheksabitilisest protsessorist. CP/M tarkvara oli, aga midagi sellist 
spetsiifilist tema jaoks kirjutatud ei olnud ja  siis oligi nagu koht, kus sai 
hakata mitte-\emph{mainframe}-de peal kätt proovima. 

\question{Aga mis te nende arvutitega siis tegite? Noorel inimesel on ju see 
probleem, et kui valid liiga raske ülesande, ei saa hakkama ja on halb ja kui 
liiga kerge, siis on igav ja ka halb?}

See on üks väga relevantne küsimus, sellepärast et mõnes mõttes meie 
generatsioonil on arvutitega vedanud. Sel hetkel, kui arvutite juurde 
sattusime, olid nad sellised, et nagu midagi väga huvitavat ei toimunud. 
Arvutite peamine köitlus oli potentsiaal, mis neis selgelt sees tuksus. Versus 
see, et sul on Youtube ja Minecraft ühe kliki kaugusel. Ükskõik, kui palju sa  
pingutataksid, midagi  ligilähedastki sa võimeline tegema ei ole. Ja teine asi, 
et arvutid olid toona aeglased,  umbes  miljon korda aeglasemad kui praegu. 
Mistõttu, kui tahtsin midagi ägedat teha, siis pidin kohe kiiresti selle 
hingeelu endale põhjalikult selgeks tegema, et pigistada välja viimane 
efektiivsusepiisk.

\question{Sa jooksid kohe mingitesse riistvara piirangutesse sisse ja isegi 
mingi lihtsa asja ekraanil liigutamiseks pidi hoolega mõtlema, et kuidas see 
ikka täpselt käib!}

Täpselt. Mistõttu suhteliselt kiiresti läksime 
assembleri\index{Keeled!Assembler} peale. Kõigepealt siis kodukootud 
Entel-arvutite peal ja siis aasta-paari pärast tekkisid Eestis esimesed IBM PC 
kloonid. 

\question{Assembleri peale kolimine eeldab siiski, et programmeerimisest on 
mingi aimdus olemas. Kust see tekkis?}

See tekkiski nende \emph{mainframe}-de peal. Robotron või mis ta oli.

\question{Aga kuidas? Lugesite raamatuid või\ldots?}

Lugesin läbi, mis selle nimi oligi, Programmeerimine 
Pascalis\sidenote{Tõenäoliselt R. Jürgenson Programmeerimine Pascal-keeles.} 
või midagi sellist. Mul on seal siiamaani mingisugused esimeste programmide 
väljatrükid  vahel, raamat on raamaturiiulis. Kirjutan 
Basicus\index{Keeled!BASIC} programmi ja kirusin, et keel on ikkagi erinev kui 
Pascal. Basicut ma ei osanud aga Pascalit natuke siis teoreetiliselt oskasin ja 
kahe peale siis hakkasin avastama. Esimene programm vist oli ruutvõrrandi 
lahendaja.

\question{See on klassika, ilmselt seetõttu, et teda on praktiliselt vaja. Aga 
ikkagi, sealt Assemblerisse minna on pikk samm, juba arusaam, et kuskil on 
Assembler, on küsimus. Kust te infot saite? Keegi õpetas? Raamatud? Ajakirjad?}

Jah, seal \emph{mainframe}-de peal ma isegi jäin Basic-usse, tegin seal isegi 
oma esimese mängu. Ja kui me kolisime  \emph{mainframe}-de pealt ära nende 
kodukootud kaheksabitiste arvutite peale, oli näha, et seal on lihtsam  
riistvarale ligi saada, eks. Ja üks asi, mis kohe ahvatlema ja paistma hakkas 
oli C programmeerimiskeel\index{Keeled!C}.  Mäletan, et samas grupis aeg-ajalt 
näitas oma nägu selline sell nagu Hannu Krosing\index[ppl]{Krosing, Hannu}, 
endine Skype kolleeg, kes  otseselt samas seltskonnas ei olnud. Ja tema oli 
selleks hetkeks  kirjutanud Assembleri õpiku, oli mingi selline pisikene 
brošüür põhimõtteliselt.  Ja ta kas pistis selle mulle pihku või, ma ei tea, 
igal juhul ma lihtsalt lugesin selle läbi, et \enquote{ohoo, mingi päris 
huvitav asi}. 

\question{Oot, mis aastal see võis olla?}

See võis olla 1987 äkki? 1988?

\question{87. aastaks oli Hannu kirjutanud Assembleri õpiku!?}

Jah, mingi sellise brošüüri vormis, 
samizdat\sidenote{\begin{russian}Cамиздат\end{russian}, tõlkes umbes 
\enquote{iseavaldamine} oli Nõukogude Liidus levinud keelatud või põrandaaluse 
kirjanduse levitamise viis. Teksti trükiti läbi mitmete kopeerpaberite 
õhukesele paberile ümber, tulemused levisid käest kätte ning neid paljundati 
omakorda. Mäletan, et ka minu vanaema tegeles sellise toksimisega, ning lapsena 
ei mõistnud, miks sellest väga rääkida ei tohi. Kuna kõik klahvidega asjad mind 
väga huvitasid, nuiasin välja võimaluse ka ise tekste ümber lüüa, miskipärast 
olen veendunud, et olen aidanud paljundada mingit budistlikku teksti.} umbes. 
Oli ikka Hannu õpik? Temaga ma mäletan, ma sellel teemal arutasin,  üsna kindel 
et tema oli selle autor.

\question{Ja mis te tegite selle Assembleriga?}

Mäletan, üks nagu selliseid korralikumaid projekte, võib-olla tegin mingeid 
väiksemaid asju ka, oli tekstiredaktor\label{sisu!jaani_tekstiredaktor}. 
Sattusime Priit Kasesaluga\index[ppl]{Kasesalu, Priit} sellisesse 
võistlusrežiimi. Mõtlesime, et mida oleks hea sellele uuele kodukootud 
platvormile kirjutada ja leidsime, et siin ei ole korralikku tekstiredaktorit. 
Hakkasime mõlemad tegema, kõigepealt Basicus\index{Keeled!Basic}. Ja üritasime 
üksteist üle trumbata, et kellel tuleb parem. Mäletan, et 
Hannuga\index[ppl]{Krosing, Hannu} rääkisime  mingist tehnikast, kuidas 
tekstiredaktoris teha  mingisugust \emph{split buffer} arhitektuuri, et 
liikumine ja \emph{insert}-imine kiired oleksid.

Tuli koolivaheaeg ja meil jäi võistlus pooleli. Aga mina panin nagu edasi, suvi 
otsa kirjutasin paberi peal tekstiredaktorit, assembleris. Ja kui tagasi tulin, 
polnud  Priit muidugi suvega midagi viitsinud teha ja sellega oli võistlus 
läbi. Siis kirjutasin selle Assembleri paberi pealt arvutisse.

\question{Töötas ka?}

Esimene kord muidugi ei töötanud, eks ole. Aga tööle ma ta igal juhul sain, asi 
toimimis ja  vaatasin, et \enquote{oo, see on ikka päris äge}. Kiire, mugav ja 
palju parem, kui ükskõik milline tekstiredaktor sellel arvutil. See andis mulle 
väga positiivse tagasiside. Peale seda hakkasin mänge kirjutama.

\question{Reflekteerides siit tundub, et sul pidi olema oskus päris suuri ja 
keerulisi abstraktseid struktuure peas ette kujutada, et sa suudaksid selle 
koodi kõik paberil asmi valada. Kust see oskus tuli või on see sul kogu aeg 
olnud või oskad sa natuke selle juuri avada?}

Ma ei tea, mulle tundus see suhteliselt loomulik. Sellised instruktsioonid 
lihtsalt, nagu sammude kirjeldus. Et mida sa tahad, et arvuti teeks, eks. On 
vaja täpselt üles kirjutada, mida sa tahad. Alguses Basicus sain esimese 
tagasiside, et kuidas tsükkel käib, ja ühel hetkel assembleris nägin, et see on 
lihtsalt natuke tülikam, aga teisalt jälle rohkem positiivset tagasisidet 
pakkuv, kui sa ta käima saad. Käib nagu väga muljetavaldavalt võrreldes 
Basicuga. 

\question{Kas see tekstiredaktor kuskile jõudis ka või sai lihtsalt oma lõbuks 
tehtud?}

See seltskond, kes siis seda Entel\index{Arvutid!Entel} arvutit tegi, 
vormistasid niipea, kui eraettevõtlus seaduslikuks muutus, kooperatiivi ja 
hakkasid  neid arvuteid tootma ja müüma. Muu hulgas käis selle arvuti juurde 
tema jaoks toodetud tarkvara: CP/M ja  lisaks see minu tekstiredaktor.

\question{Ehk esimene suurem projekt, mis sa kirjutasid, läks kohe kliendile 
müüki?}

Jah, ma ei tea, palju seda kasutati, aga kui sa endale kaheksakümnendate lõpus 
selle Eestis toodetud arvuti ostsid, siis oli seal minu tekstiredaktor kaasas. 
Selle tõttu me saime esimest palka ja tekkisid esimesed mingisugused 
sissetulekud

\question{See ju tahab tarkvaraarenduse mõttes küpsust, et sa mõtled kõik 
nurgatagused juhtumid läbi ja võib-olla kirjutad abiteksti?}

Mõnes mõttes mitte väga optimaalne, aga ma olen märganud, et mul on selline 
OCD, \emph{obsessive compulsive disorder}. Et kui midagi alustan, tahan selle 
kindlasti lõpule viia, panna i-dele punktid peale. Seetõttu paljudele 
projektidele, kus ei ole ajasurvet taga, läheb kole palju aega. Alates sellest 
tekstiredaktorist. Ma tahtsin, et kõik oleks väga ilus, kõik funktsionaalsus 
oleks olemas ja pusin senikaua, kuni oli. 

\question{Ehk siis kombinatsioon täiuslikkuse soovist ja võimekusest see ka ära 
täita. Mul võib olla soov täiuslik teemant lihvida aga ma lihtsalt ei oska seda 
teha.}

Ja, jällegi, millega mul vedas, oli see, ma astusin arvutite juurde sellisel 
hetkel kuskohas kogu tarkvara, mis seal arvutites juba oli, oli väga lihtne. 
Mistõttu see ei olnud selline nii-öelda hingemattev kogemus, et ma olen nii 
pisikene selle tarkvara kõrval vaid \enquote{ahah, okei, ma saan enam-vähem 
aru, kas tehtud on, ma teeksin paremini}.

\question{Seda on mitmed öelnud, et oma esimest arvutit nad tundsid põhjani.}

Ka auto-entusiastidel, uunikumide austajatel, on samasugune lugu, neil on 
ikkagi väga lihtsad riistapuud.

\question{Aga see annab sulle kontrolli tunde, eksole.}

No mul täna läks Tesla katki. Ja midagi ei ole teha, tuleb Soome saata.

\question{Ma korraks võtaks kinni noist alguses räägitud arvutiajakirjadest. 
Oskad sa takkajärgi öelda, oled sa mõelnud, mis sind nende juures paelus?}

Asjad, mis seal kohe väga prominentselt silma paistsid, olid mingisuguste 
arvutimängude reklaamid, mingid Atari reklaamid ja sellised asjad. Noh, nagu 
ikka, reklaamidel muidugi joonistati natuke ilusamaks, kui päris maailm, aga 
nad andsid vaate mingisugusesse oma seaduste järgi toimivasse fantaasiamaailma, 
mis tohutult paelus. Mingisugused ekraanitõmmised, kus mingid tegelased on peal 
ja ma vaatasin, et \enquote{ahaa, see on vist väga äge asi!}.

\question{Seda on ka räägitud, et see oli nagu täitsa teine maailm, kuhu sai 
sisse minna.}

Ja veelgi enam, sa said neid maailmu ise luua, see teadmine tekkis mõne aja 
pärast. Et sa ei ole passiivne tarbija vaid aktiivne looja.

\question{Ja selle aktiivsusega sa kirjutasid selle tekstiredaktori valmis ja 
sind võeti palgale?}

Ma ei mäleta, kuidas see järjekord täpselt oli, võimalik, et meid võeti palgale 
seal alguses, kui ma seal niisama katsetasime. Aga võimalik, et see tõesti oli 
pärast seda, kui me esimesed asjad ära tegime.

\question{See oli keskkooli ajal veel?}

Jah, see oli vist keskkooli alguses, ma arvan. Kaheksakümmend seitse oli 
keskkooli algus. Kaheksakümmend kuus võis olla see aasta, kus ma üldse sinna 
sattusin ja siis 87-88 oli see, kus palka hakkasin saama. 

\question{Kas arvutis käimine olümpiaade ei hakanud segama või käis see 
õppimisega kuidagi lihtsasti kokku?}

Üldse ei seganud. Arvuti-värk on mul ikka suhteliselt kogu aeg põhiline asi 
elus olnud, ülikooli lõpetasin ka nii-öelda kõrvalhobina ära. Aga juba siis 
olid kool ja olümpiaadid arvuti taustal.

\question{Kas juba siis hakkas moodustuma seltskond, mis pärast sai 
Bluemooniks\index{Bluemoon}?}

Just. Bluemooni süda tegelikult oligi see seltskond, mõned klassikaaslased. 

\question{Kas tollest arvutikooperatiivist eraldusite kohe eraldi ettevõtteks 
või oli seal vahepeal mingi faas veel?}

See oli niimoodi, et ühel hetkel meil Ahtiga\index[ppl]{Heinla, Ahti} tekkis 
mõte, et teeks ühe arvutimängu. Korraliku mängu, mis jookseb PC peale, mitte 
ainult seal kodukootud arvutite peal. Meil oli mingi eeskuju ka, mille järgi  
mängu teha, mis oli Yamaha MSX-ide\index{Arvutid!Yamaha MSX} peal, mis oli 
palju vähempopulaarsem platvorm kui, PC. Oli näha, et PC-d hakkavad juba jõudma 
sinnamaani, kuskohas saab juba midagi huvitavat teha. Ja väga sellise sügava 
mulje jätsid toona Ahti matemaatikuvõimed. Kuidas ta jagas ära, et 
\enquote{siin tuleb tangensit kasutada, et  perspektiivi luua}. Mingisuguseid 
esimesi eksperimente tegime tema vanemate juures Küberis\index{Küber}, kus tal 
oli arvutitele ligipääs. Ahti hakkas palju varem programmeerima, kui mina. 

Üks tõuge selle mängu tegemiseks oli see, et meil keskkooli viimases klassis 
(oli vist ikka viimane klass?) tekkis võimalus minna klassiga Rootsi. Esimene 
välisreis üldse, aastal 1989 suhteliselt unikaalne võimalus. Läksime sinna läbi 
Leningradi, siis oli vaja teha mingisuguseid imelikke trikke väljamaale 
saamiseks. Seal onutütre mees ütles, et \enquote{Väljamaal nad õudselt tahavad 
softi, kirjutage mingisugune lahe soft. Lähed sinna, müüd maha ja mingit 
probleemi pole!}. Mõtlesin, et \enquote{aga teeme} ja hakkasime tegema. 
Leidsime, et teeme mängu, teeme korraliku mängu. Hakkasime tegema ja muidugi ei 
jõudnud valmis. Softiga, nagu ma nüüd hiljem tean, tuleb kõik ennustused  umbes 
piiga läbim korrutada, kulub umbes kolm korda rohkem aega kui alguses arvad. 

Muidugi me seda valmis ei saanud, aga samas oli juba piisavalt suur hoog sees. 
Et ühel hetkel võtsime sinna juurde korraliku kunstniku, Kaspar 
Loit\index[ppl]{Loit, Kaspar} ehk BKnows. Ja muusika ning heli-inimese Ott 
Aloe\index[ppl]{Aaloe, Ott}. Ja tegime mitte ainult ühe mängu vaid täitsa 
sellise mängude seeria. Millega meil vedas, oli see, et see esimene mäng 
õnnestus Rootsi müüa hoolimata sellest, et meil Rootsis käisime ajal seda mängu 
kuhugi kellelegi pakkuda ei olnud isegi, kui ta oleks valmis olnud.

\question{Kuidas? Meil oli ju veel Nõukogude Eesti?}

Selle Sideministeeriumi arvutuskeskuse\index{Sideministeeriumi Info- ja 
Arvutuskeskus} juhataja Jüri Malsubi\index[ppl]{Malsub, Jüri}  tuttav oli üks 
sell nimega Tiit Vasli\index[ppl]{Vasli, Tiit}, kellel oli väljamaal suhteid, 
ta vahendas metalli, mingeid sihukesi asju. Ma isegi ei teadnud, mida ta 
vahendas. Ta oli selline mees, keda oli kaugelt näha, sellepärast et tal oli 
üks Eesti esimesi mobiiltelefone, mille antenn oli mingi kolm meetrit kõrge. 
Oli kaugelt näha, et tema läheb seal kuskil tänaval. Tal oli Rootsis sidemeid 
ja ta mõtles, et \enquote{noh, ma vaatan, räägin} ja müüski. Ta äripartnerid 
olid huvitatud sellisest eksootilisest asjast nagu raudse eesriide taga 
toodetud mäng. 

Selle mängumüümise tulemusena me teenisime rohkem, kui mu vanemad kunagi oma 
elu jooksul teeninud olid. Mis oli vist mingi viis tuhat dollarit. Arvestades 
muidugi inflatsiooni, mitte reaalväärtuses, vaid nominaalväärtuses. 

Kui see mäng nii-öelda müüki läks, tekkis meil tõsine küsimus, see oli siis 
keskkooli lõpp, ülikooli algus, et kuidas me nüüd seda administratiivselt 
korraldame. Oleme selles kooperatiivis  ametlikult tööl, eks, aga on tegelikult 
näha, et meie plaanid võivad suuremaks kujuneda, kui see kooperatiiv. Rääkisime 
läbi. Mäletan pingelist läbirääkimist Jüri Malsubiga\index[ppl]{Malsub, Jüri} 
kuidas seda mängu tulu jagada. Nemad on ühelt poolt panustanud ja meie oleme 
teiselt poolt panustanud, tahaks nagu oma asja teha. Lõpuks saime meie poolt 
vaadates väga mõistliku kokkuleppe ja leidsime, et nüüd on aeg vormistada asi 
mingiks oma ettevõtteks. Mida me siis ka tegime, aastal 1990, ma arvan. 

\question{Kas te mõtlesite nullist välja, et teil on vaja kunstnikku ja 
muusikut ja kuidas nende töö programmeerimisega siduda või oli teil eeskujusid 
ka?}

No me olime teisi mänge näinud ja nägime, et nad näevad paremad välja kui see 
meie katsetus ilma kunstniketa. Ma ei mäleta, kes meid  
BKnowsiga\index[ppl]{Loit, Kaspar} tutvustas, see võis olla isegi Tanel 
Hiir\index[ppl]{Hiir, Tanel}, ei mäleta. Kaspari kunstniku-võimed toona jätsid 
mulle väga sügava mulje. Teda oli raske tööle saada, ma mäletan, tihtipeale 
pidi selja taga istuma, et \enquote{tee nüüd}, aga kui ta tööle sai, oli väga 
äge. 

\question{Ma just mõtlengi seda, et kindlad viisid graafikat kasutada, 
töödelda, laadida on ju tänaseks välja kujunenud, kas teie mõtlesite need ise 
välja?}

Üsna, jah, sest, jällegi, need platvormid olid miljon korda aeglasemad, kui 
praegu. Mistõttu tööriistad olid Turbo Pascal\index{Keeled!Turbo Pascal} ja 
Borland C\index{Keeled!Turbo C}. Kaspar tegi asju Amigal, seal olid tal oma 
tööriistad.

\question{Mind on painanud see küsimus, et te ju tegite muusikaprogrammi. 
Kuidas te sinna valdkonda sattusite, te pole ükski muusikainimene ju, nii 
palju, kui ma tean?}

Ükskord ülikoolis oli sihuke lahe hetk, kus olin arvutiklassis ja mingid tüübid 
istusid arvuti taga ning komponeerisid  SoundClubis\index{SoundClub} muusikat. 
Kiibitsesin natuke ja ütlesin, et see on minu programm. Nad ei uskunud. 

Ma ei mäleta, kuidas see algtõuge sattus. Tänu sellele, et me olime juba mänge 
teinud, oli meil kindlasti kokkupuude sellega, kuidas teha taustamuusikat. Ja 
toona, üheksakümnendate alguses, oli väga suur trend trackerid, ehk mingite 
sämplite baasil muusika kirjutamise väga sellised platoonilised riistapuud. Ja 
sealt tuli mõte, et  heli on väga hea,  aga kasutajakogemus tundus  vähemalt 
harjumatule silmale väga-väga ebamugav. Mõtlesime, et kuidas kasutada sedasama 
tehnilist võimekust, aga teha  kasutajaliides, mis oleks äge eriti inimestele, 
kes ei ole pidevalt muusika kirjutamise juures.

Teema hakkas järjest huvitama, kuna seal on väga mitmeid nüansse, nagu UI 
disain, muusika pool asjas (kuigi ükski nendest autoritest ei olnud muusikud), 
kuidas tehniliselt teha aeglastel arvutitel head heli. Seal ma puutusin esimest 
korda kokku mingisuguste matemaatiliste teoreemidega, mida ma siis 
Ahti\index[ppl]{Heinla, Ahti} abil üritasin lahendada. Üks huvitav asi oli see, 
et kuna me need instrumendid korjasime endale kuskilt BBS-idest kokku, olid 
need õudse kvaliteediga. Mäletan, et Ahti kirjutas mingisuguse tarkvara, kus ta 
tegi Fourier analüüsi, et nad häälde viia. Ükskord Tartus istusin ja 
häälestasin pille niimoodi, et endal väga suurt muusikaharidust ei olnud, 
natuke olin pilli õppinud. Aga Fourier analüüsiga sai ikka väga hea häälestuse. 

\question{Selle tarkvaraga on ju tehtud igasugu asju Vennaskona Diskost alates 
ja ta käib ka siin mõnest loost läbi. Küll aga ma ei mäleta, et keegi oleks 
rääkinud selle tarkvara ostmisest?}

Jaa!  

Siiamaani võib-olla vähem kui kord nädalas, aga kord kuus vähemalt saame mingi 
fännikirja, et näete ma olen on SkyRoads-i\index{Mängud!SkyRoads} peal üles kasvanud. On isegi mõned 
kloonid tehtud,  teda saab tänapäeval veebis mängida. Ja SoundClub oli teine 
suurem projekt. Meil oli siis juba Bluemoon firmana ja meil oli kaks toodet 
SkyRoads (mis tegelikult oli järg tollele esimesele Rootsi müüdud mängule, 
mille nimi oli Kosmonaut\index{Mängud!Kosmonaut}) ja SoundClub. 

Ja nüüd oli küsimus, kuidas neid müüa. Mäletan, et see oli mingisuguste 
telefonide ja faktidega ja tšekkidega jamamine. Mõlemad olid \emph{shareware}, 
osalt saadeti lihtsalt ümbrikus sularaha aga tavaliselt saadeti tšekke, mida ma 
käisin Eesti Maapangas või Rahvapangas lunastamas, selline kogemus. 

Teine asi, mis oli tegelikult väga äge kogemus, oli läbirääkimiste pidamine 
olukorras, kuskohas teisel poolel ei ole mingit juriidilist motivatsiooni 
lepinguid järgida. Mistõttu tuli tihtilugu tekitada selline olukord, kuskohas 
sa nagu lood sellise helge tuleviku, et partnerlusel oleks jumet. Mõnes mõttes 
selline \emph{iterated prisoner's dilemma}\sidenote{Mänguteoreetiline 
konstruktsioon, mille abil uuritakse osapoolte koostööstrateegiaid. Selle 
valdkonna üks teadustulemusi on, et (eriti mängu iteratiivses, korduvalt 
mängitavas ja eelmisi tulemusi \enquote{mäletavas} versioonis) indiviidile 
annavad pikas perspektiivis parema tulemuse altruistlikud, mitte egoistlikud 
strateegiad.}, sa pead looma olukorra, kus teisel poolel, hoolimata sellest, et 
mingit sundmehhanismi ei ole, on lihtsalt huvi olla osa sinu tulevikust ja 
seeläbi lepinguid järgida.

Alguses oli meil \emph{shareware} aga inimesed hakkasid kirjutama, et tahaks 
seda mingisugusesse ajakirja panna või tahaks seda kuskil levitada. Mingi väga 
lahe sell tekiks meil Saksamaale, kes hakkas mitmeid meie asju levitama, aastal 
1996, käisin tal lõpuks isegi külas. Samuti üks väga lahe omaette kogemus oli 
müük Taiwani telefoni ja faksi abil, kuskil Tartu Estiko\index{Estiko} 
kontoris. 

\question{Kuidas sa Tartusse sattusid?}

Ülikooli läksin.

\question{Mida sa õppima läksid?}

Füüsikat. Nii mina kui Ahti läksime füüsikat õppima aga Ahti kukkus sealt juba 
teisel aastal välja. Mina punnitasin lõpuni. 

\question{Miks just füüsika? Ahti rääkis, et see ala tundus talle mõnes mõttes 
kõige puhtam?}

Ma arvan, et ta on vähem puhtam kui arvutiteadus või matemaatika, eks. Kaks 
põhjust oli füüsika valikuks, Ahti põhjused olid ilmselt korelleeritud. Üks oli 
see, et ma tundsin, et  arvutites ja matemaatikas olen ma juba piisavalt sees, 
et füüsika oleks nagu silmaringi laiendav. Ja teine oli see, et meie füüsika 
õpetaja, Vilma Kukrus\index[ppl]{Kukrus, Vilma}, oli ikka väga väga äge õpetaja 
ja tekitas füüsika vastu sügava huvi. Või vähemalt süvaga austuse. Mul on väga 
hea meel, et ma füüsika lõpetasin.

\question{See ilmselt mõjutas päris olulisel määral noore inimese maailmapilti 
ka?}

Absoluutselt. Füüsika on selles mõttes optimaalne teadus, et sa suhtestud 
reaalse maailmaga niimoodi, et kui matemaatikud võivad minna niivõrd 
abstraktseks, et nad kaugenevad reaalse maailma piirangutest, siis füüsikas 
reaalne maailm tõmbab su alati maa peale tagasi. Sõna otseses mõttes, 
tihtilugu. Ja seetõttu sul tekib intuitiivne arusaam sellest, misasi on teadus. 

\question{Ahtiga\index[ppl]{Heinla, Ahti} oli ka nii, sinu puhul on samasugune 
muster, seepärast küsin. See, mis ma kuulen ei kõla nagu keskmine 
\emph{teenager}. See kõlab nagu üsna küpse inimese jutt?}

No praegu ma enam \emph{teenager} ei ole!

\question{Nüüd jah, aga need otsused ja see viis, kuidas toona asju aeti on 
üsna kaine, arutlev lähenemine. Kust see pärit on?}

Üks oluline asi oli ikkagi, ma arvan, et Ahtile ma võlgnen väga palju tänu. 
Meil oli super hea koostöö. Priit ka, eks, aga praktiliselt kõiki selliseid 
probleeme lahendasime tiimiga. Minu  ja Ahti vahel tekkis väga tihti selline 
asi, et Ahti on nupukas ja ta mõtleb väga erinevalt, kui mina. Mistõttu 
koostöös temaga sündinud otsused olid just nimelt ägedad, kuna nendes oli kaks 
väga erinevat vaatepunkti, mida see otsus pidi rahuldama.

\question{Ma olen alati tahtnud küsida. Võib olla ruttame natuke ette, aga kui 
me vaatame kasvõi Bluemooni kodulehekülge, on seal loetletud üksjagu edukaid 
asju aga ka päris mitu asja, mis ei ole ühel või teisel põhjusel välja tulnud. 
Inimesed ei suuda mõnikord isegi läbi suure edu tiimina toimima jääda aga teie 
olete koos läbi nii suure edu kui mitmete ebaõnnestumiste. Kuidas te seda 
teete?}

Vahemärkusena, mäletan, mõni aasta  tagasi sain Sean Parkeriga\sidenote{Sean 
Parker\index[ppl]{Parker, Sean} on Napsteri kaasasutaja ja, muu hulgas, 
Facebooki esimene president. Ta esineb ka tegelaskujuna Facebookist rääkivas 
filmis, kus kujutatud intriigidest ja tülist tõukub ka eelnev küsimus.} kokku 
ja meenutasime  Napsteri ja Kazaa aegu, tema tegi Napsterit. Selline lahe 
kogemus.

Ma arvan ikkagi, et sellised kohatised eduelamused olid piisavad, et  läbi 
suruda ka sellistest mitte õnnestunud projektidest. Ja mõned hetked olid ikkagi 
jube rasked. Konkreetselt mäletan  sellist hetke, kus kogu mänguarendus läks 
üles-suunas ja siis ühel hetkel lõpetas meie kirjastaja Ameerikas, Interactive 
Magic, \emph{milestone}-de maksmise. Raha jaoks oli meil Exceli tabel, kus 
\emph{runway} oli kogu aeg kirjas, mitmeks kuuks  meil raha on põhimõtteliselt. 
See \emph{runway} hakkas siis kahanema ja ühel hetkel oli selge, et nad on 
pankrotis, sealt enam midagi ei tule. Oli tõsine küsimus, et mis nüüd edasi 
saab. Ja Ahti\index[ppl]{Heinla, Ahti} oli just see, kes ütles, et \enquote{ah, 
küll me välja ujume!}. Ujusimegi.

\question{Ühel hetkel te läksite mängu kirjutamise juurest ära ja kirjutasite 
Everyday. Kas see legend, et see käis kuidagi lehekuulutuse kaudu vastab tõele?}

Vastab tõesti, jah. See oligi just see raske hetk, kuskohas ma mõtlesin, et mis 
me nüüd teeme.

\question{Mis aastal see oli?}

See oli aastal 1999. Ühel reede hommikul vaatasin, ise olin Tartus, oli 
lehekuulutus, et pakutakse inimestele mingisugust ulmelist palka\sidenote{Teist 
perspektiivi sellele loole vaata Tarvi jutust leheküljelt 
\pageref{sisu:everyday}.}. Everyday portaal oli arendushädas ja Tele2 oli 
börsile lubanud, et kohe tuleb sihuke asi välja. Nad olid juba mingi aasta või 
paar arendanud ja olid välja tulekust kaugel. Kuulutuses oli  pikk nimekiri  
nõuetest, mida arendajad peaksid olema osanud. Pikk nimekiri asjadest, millest 
ma elu sees kuulnud ei olnud. IMAP ja POP3 ja PHP ja SQL ja mingid niisugused 
asjad. Tähtaeg oli esmaspäev, oli reede, mina olin Tartus ja teised olid 
Tallinnas. Mäletan, et helistasin Ahtile\index[ppl]{Heinla, Ahti},  rääkisime 
läbi, mõtlesime, et proovime, vaatame, mis juhtub. Tegime kohe  nädalavahetuse 
plaani ja põhimõtteliselt esmaspäevaks oli valmis prototüüp sellest portaalist, 
mida nad tahtsid. Kui me esmaspäeval intervjuule läksime, oli meil dilemma, et 
kas me ütleme, et see oli ühe nädalavahetusega kirjutatud või mitte. Ta nägi 
väga hea välja, kuna meil oli palgal arvutimängudega karastunud kunstnik, 
näiteks. Ma arvan, et prototüüp nägi parem välja, kui lõpptoode. Ja toimis 
täitsa, võisid  sisse logida, erinevaid paneele ringi lohistada, võisid emaili 
kirjutada, uudiseid lugeda, ilmateateid, mida iganes. 

\question{See tähendab ju, et tolle nädalavahetusega pidi sigima päris hea 
arusaam sellest, kuidas HTML ja brauseri renderdus ja muu selline töötab?}

Andmebaasid. Mäletan, et Priidule\index[ppl]{Kasesalu, Priit} jäi 
andmebaasidega tegelemine. Ta sattus hätta, ei saanud  loogikast aru. Ja ma 
mäletan, et ta võttis telefoni, helistas mingile andmebaasieksperdile, kahjuks 
ei mäleta, kes see võis olla. Oli laupäeva hommik. Kuulsin seda kõnet kõrvalt, 
et \enquote{kuule, mul on üks niisugune kogemus, on sul nagu hetk aega? Aa 
okei, okei.} ja pani toru ära. Ei olnud aega. Hea küll, tagasi uurima, kuidas 
SQL  käib uuris välja. Sai tehtud.

\question{Kõlab üsna ulmelisena, seal peab ju olema mingi meetod taga, kuidas 
seda teadmist omandada?}

See oli väga äge kogemus, jah. Ega tegelikult tehnoloogiad toona ei olnud super 
keerulised, nad olid meile lihtsalt võõrad. Ja meil oli tiim tõesti äge tooma 
ning saime tööjaotuse tehtud: igaüks pidi mingi kindla aspekti välja uurima. 
Magasime natuke, mitte eriti. 48 tundi tundi tööd.

\question{See tähendab, et te pidite kuidagimoodi oma tööd ka koordineerima, 
kes seda kampa teil juhtis?}

No mina olin nii-öelda ametlik juht. Samas see tiim töötas ise ka päris hästi. 
Välja arvatud, jah, võib-olla kunsti pool, mis põhjustas võib olla kõige rohkem 
meelehärmi, et kuidas Kasparilt\index[ppl]{Loit, Kaspar}  lubatud asjad kätte 
saada. Kunstniku asi, rohkem boheemlane, kui teised.

\question{Arvutades leiame, et kui kuulutus oli 1999 ja keskkool 
üheksakümnendate alguses, siis te kogu kümnendi kirjutasite mänge?}

Jah, päris mitmeid mänge tegime, kutsusime ennast Eesti mängutööstuseks. 

\question{Kui suur see tiim oli?}

1999. aastaks ega ta väga palju suuremaks ei läinud. \emph{Core} tiim oli siis 
mina, Priit\index[ppl]{Kasesalu, Priit}, Ahti\index[ppl]{Heinla, Ahti}. Artur 
Vill\index[ppl]{Vill, Artur}, kes oli 3D-kunstnik ja kes muide on teinud 
sellise filmi nagu Happy Feet mingid \emph{landscape}-d ja maastikud. Ta kolis 
pärast Bluemooni kokku kukkumist Austraaliasse ja seal tõusis tähelennuna, väga 
kõva vend 3D modelleerimises ja kunstis.

Ja kõrvalt Kaspar\index[ppl]{Loit, Kaspar} tegi kunsti, Ott\index[ppl]{Aaloe, 
Ott} ja Glen Pilvre\index[ppl]{Pilvre, Glen} tegid muusikat. Juhan 
Soomets\index[ppl]{Soomets, Juhan} tegi ka nagu poole kohaga 3D-graafikat ja 
vist oligi kõik. Kui bluemoon.ee lehele minna, siis see tiim on seal siiamaani 
üleval.

\question{Isegi toonase tehnoloogia lihtsuse juures pidi teil siis ju selle 
väikse tiimi peale tööd palju olema?}

Tööd oli päris kõvasti jah. Põnev oli ka muidugi.

\question{Mis see põnevus oli? Kui ühe mängu valmis olite teinud, kas siis 
igavaks ei läinud?}

Mängude tegemine ongi selles mõttes äge, et see on  niivõrd palju rahuldust 
pakkuv, kui mingi asi tööle läheb. Kirjutad mingisugust andmeanalüüsi. Kui asi 
tööle läheb, tuleb ekraanile õige number. Aga kui mängus asi tööle läheb, tuleb 
vägev plahvatus, näiteks. Või tulevad mingid väga, sellist, rahuldust pakkuvad 
stseenid,  efektid või lood või midagi sellist. 

\question{Nojah, vaade mingisse teise maailma, millest sa oled nüüd järgmise 
tüki loonud.}

Just, jah. Ja nüüd sa saad seal testimise käigus  ringi käia ja mingisugused 
kohati väga vapustavaid vaateid, sündmuseid, mis on toimunud\ldots

\question{Ühte asja ma tahtsin veel küsida. Kui sa rääkisid, et sa tegid üksi 
tekstiredaktori ja seal pidi kõik asjad ilusti reas olema, sellest ma saan aru. 
Aga kui meeskonnana softi kirjutada, siis see vajab ju \emph{software 
engineering}-u protsesse ka, kust teil need tulid?}

Üheksakümnendatel olid lihtsalt zip-failid ja \emph{backup directory}-d. 
Versioneerimist või selliseid  asju me üldse ei teinud. 

\question{Aga kuidas te siis tagasite, et see kupatus teil kokku ei kukkunud?}

Me olime väga ettevaatlikud! Üks põnts, mis meil juhtus, oli see, et meil murti 
kontorisse sisse ja varastati arvutid ära. Sealt läkski mingisuguse SkyRoadsi\index{Mängud!SkyRoads} 
või mingi asja mingi versioon. Meil olid diskide peal \emph{backup}-id ja 
midagi jäi alles, aga mingisugused asjad Bluemooni ajaloost läksid lõplikult 
kaduma. 

\question{Siiski, kas te oma töökorralduse mõtlesite lihtsalt jooksu pealt 
välja?}

Istusime telefoni otsas, põhimõtteliselt\sidenote{Huvitav, kust see Skype idee 
küll sündida võis?}. Mina olin Tartus, Ahti\index[ppl]{Heinla, Ahti} kolis ühel 
hetkel Tallinna  tagasi. Istusime telefoni otsas, koordineerimine käis ka meili 
teel. Eks meil tekkis spetsialiseerumine ka. Mina manageerisin tiimi, 
kunstnikke, kirjutasin mingisuguseid tarkvaralõike. Priit\index[ppl]{Kasesalu, 
Priit} spetsialiseerus operatsioonisüsteemi asjadele, graafikale,  Windowsi API 
ja sihukesed asjad. Ahti\index[ppl]{Heinla, Ahti} tegi sellist rohkem 
teadusmahukat asja, kuskohas oli vaja midagi AI-laadset või siis mingisugust 
matemaatikat.  Kui vaja, tal oli võtta. Et \enquote{ahaa, ma tean, selle jaoks 
on siin sellel leheküljel Knuthi Art of Computer Programming-us\sidenote{Knuth, 
Donald E. Art of computer programming (TAOCP). Tuntud ka kui Knuthi Piibel. 
Tegu on monumentaalse teosega, mille seitse köidet pidid katma kogu teadaoleva 
arvutiteaduse. Praeguseks on ilmunud kolm köidet ja esimene osa neljandast. 
Kuna teise köite küljenduse kvaliteet lugupeetud autorit ei rahuldanud, lõi ta 
oma raamatu ilusaks tegemiseks süsteemi TeX, mille derivaatide abil kirjutab 
praegu oma artikleid kogu teadusmaailm ja mille abil on kujundatud ka käesolev 
tekst.} on õige algoritm, teeme selle!}. 

\question{Tundub siis, et kuna meeskond töötas tiimina hästi, siis see lahendas 
ka üksiti ära \emph{software engineering}-u probleemid. Ei tekkinud mingeid 
merge konflikte ega probleeme, sest te töötasite inimlikult nii hästi koos.} 

Tiim oli väike ka, räägime kolmest programmeerijast. 

\question{Kui sa vaatad tagasi enda kui programmeerija peale üheksakümnendatel, 
oskad sa kuidagi kirjeldada enda arengut?}

Põhiasi, mis, ma arvan, domineeris seda arengut, oli see, et arvutid läksid iga 
kahe aasta tagant kaks korda kiiremaks. Mistõttu oli vaja kogu aeg hoolitseda 
selle eest, et sa ajale jalgu ei jää. Lõpuks me ikkagi jäime, mängutööstuse aeg 
sai läbi. Kümme aastat, arvutid läksid selle aja jooksul mingi, mis see siis 
on,  kolmkümmend korda kiiremaks. Ja  võimalused: heli läks rikkalikumaks,  
mälu läks suuremaks, graafika ägedamaks, võrgundus tuli juurde. Kogu aeg tuli 
ennast hoida aja tasemel. 

\question{Aga programmeerimise kunsti mõttes? Mitte see, et kas ma tean üht või 
teist API-t vaid kas ma olen programmeerijana täna parem, kui eile?}

See on huvitav küsimus, ma väga palju ei ole selle peale mõelnud. Kindlasti  
kogemus õpetas. Ma ei oska praegu  tagantjärgi seda kuidagi kompresseerida. Ma 
tean, et programmeerijana arengud on mul pigem nagu hiljem olnud, võib-olla ma 
aga mäletan hilisemaid arenguid paremini. See, kus ma kolisin rohkem 
funktsionaalse programmeerimise peale,  juhtus peale Skype'i. Kuni Skype'i 
lõpuni ma ikka kirjutasin oma vanade tööriistadega.

\question{Pärast Everyday intervjuud, kui kiiresti te tolle \emph{production} 
versiooni välja lasite?}

Ma hästi ei mäleta, aga see võis olla nii, et  suvi otsa kirjutasime ja kuskil 
sügisel või umbes nii tuli välja. Mingi esimese versiooni jaoks võis kolm-neli 
kuud minna.

\question{Sellesama väikse tiimiga?}

Põhimõtteliselt küll. Kuigi nüüd oli nii, et seal oli juures mingeid rootslaste 
tehtud asju ja see väike tiim oli osa  palju suuremast organisatsioonist. 
Mistõttu läks asi ka oluliselt aeglasemaks. Mingeid asju oli vaja rootsi keele 
tõlkida. Ma mäletan ükskord öösel sain Niklaselt\index[ppl]{Zennström, 
Niklas}\sidenote{Niklas Zennström, hilisem Skype asutaja.}, meili 
rootsikeelsete vastetega inglisekeelsetele fraasidele ja all oli 
\enquote{midnight translation services by Niklas Zennström}.

\question{Sellised teenused siis. Mis Niklas tegi seal projektis?}

Niklas oli everyday.com-i CEO. 

\question{Niklas oli siis see mees, kes ei suutnud kogu oma rootslaste tiimiga 
tarnida?}

Seda ma ei tea täpselt, kuidas see atributsioon seal täpselt  oli, aga Niklas 
oli põhimõttelist see, kelle lõplik otsus oli see, et Eestist arendaja otsida. 
Linnar Viik\index[ppl]{Viik, Linnar} vist oli pakkunud, et võtaks Eestis 
programmeerijaid ja Niklas oli see, kes otsuse langetas, et Bluemooni 
seltskonda kaasata.

\question{Kuidas tiimi skaleerumine tundus? Kui te olite kõik see aeg 
kirjutanud kompaktses kõgproffide tiimis keerulist softi, siis veebiarenduses 
on rõhk ju mujal?}

Ma hästi ei mäleta, et seal mingisuguseid olulisi probleeme oleks olnud peale 
selle, et kohe tuli  kommunikatsiooni ülesanne juurde. Nagu ikka, kui on kaks 
programmeerijate tiimi, siis esimene reaktsioon kõigil on, et \enquote{see on 
teise tiimi bugi}. Neid asju  tekkis kohe kõvasti. Aga ma ei mäleta, et oleks 
mingi tohutu külma vee kaela saamine olnud. Saime hakkama küll. 

\question{See kõik toob meid 1999. aastasse ja seega ka otsapidi väljapoole 
meie ajahorisonti, milleks on kaheksa- ja üheksakümnendad. Mitte, et pärast ei 
oleks igasugu põnevaid asju veel juhtunud.}

Enamus asju juhtus hiljem!

\question{Aga inimeseks said sa ju varem. Mis sa praegu teed?}

Peamine ja kõige olulisem tegevus on hoolitsemine selle eest, et juhul, kui 
inimajastu peaks mõne aastakümne (loodetavasti mitte mõne aasta) jooksul 
lõppema,  inimesed siia planeedile alles jääksid.

\question{Ära pane pahaks, aga ma hästi ei näe mõtteliini ekraani peale 
plahvatuse joonistamisest selle teemani. Palun selgita!}

Sinna mahub üks kuni kaks aastakümmet veel, ehk see, millest me ei ole 
rääkinud. 

\question{Sa oled lihtsalt jõudnud selleni, et see on sinu jaoks oluline 
probleem?}

Jah, selleni jõudsin ma aastal 2008 või midagi sihukest, kui Skype'is juba hoog 
hakkas raugema, sealt enam väga väljundit ei olnud. Sattusin rääkima 
inimestega, kellega me viimased kümme aastat olen ehitanud sellist 
\emph{community}-t, kes üritavad teha ära AI-uurijate kodutöö. Ehk siis teha 
asju, mida on vaja selle jaoks, et AI-ga tulevik oleks inimestele soodne, aga 
millega AI-arendajad ise ei ole näidanud mingit motivatsiooni tegeleda peale 
selle abstraktse motivatsiooni, et nad on ka inimesed.

Üks võimalus probleemi kirjeldada on see, et meil on fundamentaalne 
\emph{trade-off}, ma isegi tea, kuidas seda eesti keeles öelda. Et sa ei saa 
nagu kahte asja korraga. Super kompetentset süsteemi ja sellist süsteem, mille 
üle sul on täielik kontroll. See ei ole isegi arvutite spetsiifiline probleem, 
inim-juhtidega on sama probleem: mida rohkem ta delegeerib, seda 
kompetentsemaks muutub süsteem või suuremaks kasvab organisatsiooni võimekus. 
Aga tema isiklik kontroll selle üle, mis toimub, väheneb. See on fundamentaalne 
printsiip. Ja mida inimkond praegu teeb, iga päevaga järjest rohkem, ta 
delegeerib oma otsuseid masinatele. Mistõttu sellise delegatsiooniga tegelikult 
väheneb inimeste kontroll tuleviku üle. Võib juba öelda, et praegu on inimeste 
kontroll tuleviku üle väiksem, kui see oli näiteks viiskümmend aastat tagasi. 
Ja see tendents tõenäoliselt jätkub. Nüüd on küsimus see, et kuidas me siiski 
säilitaksime kontrolli mingisuguste oluliste aspektide üle. Näiteks atmosfääri 
koostis, mis on meile oluline. Temperatuur, mis tundub juba praegu keeruline. 
Räägime siin veebruarikuus, väljas on kolm kraadi sooja, sajab vihma. Juba 
inimestel on raske keskkonna üle kontrolli säilitada. Lisame siia entusiastliku 
delegatsiooni arvutitele, kellel on keskkonnast täiesti ükskõik! Sellepärast me 
saadamegi roboteid radioaktiivsetesse aladesse või kosmosesse, et neid keskkond 
ei huvita. Probleem on selles, et AI arendajatel on motivatsioon aretada just 
nimelt delegatsiooni poolt, et delegatsioon oleks võimalikult lai ja tulemus 
oleks mingi meetrika järgi võimalikult kompetentne. Ja palju vähem on 
motivatsiooni selle jaoks, et mõelda selle peale, et kuidas kogemata mitte 
delegeerida selliseid asju, mida meie elus olekuks on vaja.

\question{Ma südamest loodan, et sul tuleb välja, sest muidu on pahasti!}

Ma tihtilugu ütlen inimestele, et \enquote{wish me luck, you are going to need 
it!}



\chapter{Võrgustiku analüüs}
\begin{itemize}
	\item Eemaldame kõik ühe seosega tipud
	\item Suured asutused konsolideerime prefixi alusel (TTÜ Arvutuskeskus -> TTÜ)
	\item Kesksed tipud (eraldi organisatsioonid ja inimesed)
	\item Degreede jaotuse graafik
\end{itemize}

\chapter{Kokkuvõte}
\begin{itemize}
	\item Paljud läksid keskkooli kõrvalt tööle, eriti kaheksakümnendatel
	\item Läbivaks jooneks on iseõppimine - ka näiteks Anto ei oska öelda, kust ta elektroonika-alase teadmise üles korjas. Andrus väidab, et \enquote{Programmeerimine sünnib vajadusest}
	\begin{itemize}
		\item Arne ütleb, et see oli \enquote{katsetamine ja kõlakad}. Kuna materjale ei olnud, siis õpiti üksteise käest ja inimsuhted muutusid lihtsalt praktiliselt hädavajalikuks: kui kellegi pealt ei olnud šnitti võtta, siis sa lihtsalt ei arenenud. Järelikult tippudel \emph{pidi} olema väga hea suhtevõrgustik
		\item Ahti ütleb Jaani tsiteerides, et programmeerimine, erinevalt muudest distsipliinidest on ise õpitav: pärast algse iva kätte saamist on edasi kõik võimalik ise katsetamise ja mõtlemise teel kätte saada
	\end{itemize}
	\item Mitmel puhul öeldakse, et "ei programmeeri, sest nägin väga vara tõeliselt häid programmeerijaid" Järeldus: arvutihuvi tuleb teisiti realiseerida?
	\item Tundub, et mingil hetkel oli ajalehekuulutuse kaudu omale tech-teami hankimine täiesti levinud praktika (vt. Vilve, Bluemoon, Kütt)
	\item \enquote{Kõik, mis on võimalik inimese peas, on võimalik ka päriselt}. Priit Raspel, Jaan Tallinn
	\item \enquote{Kitsendused sunnivad inimesi tegema õigeid otsuseid} Sergei. Ka Meelis ja tema interferents
	\item Vene kogukond on täiesti tundmata ja, tundub, elas mingit täiesti oma elu. Kahju, nutikad inimesed, nii palju, kui näha oli
	\item Mis asi see oli, et nii Jaan kui Andres P. tegid esimese asjana tekstiredaktori ja ka Jaanus mainib tekstiredaktori kirjutamist\sidenote{\url{https://corecursive.com/058-brian-kernighan-unix-bell-labs/} on sama lugu: alguses läks väga palju auru tekstiredaktori peale. Lisaks Knuth ja \LaTeX}? Seotud mitmel puhul läbi käinud mõttega sellest, et vanasti oli arvuti ühest küljest kättesaadavam ja teisalt vaesem. Kogu soft tuli ise kirjutada aga selle ka \emph{sai} ise kirjutada
	\item Õpetajatelt (Meelis, AK, Anto jt.) sai pigem üldist juhendamist ja suunamist kui konkreetset õpet. Aga ka lihtsalt kriitilise massi huviliste kokku toomist (Jaak Loonde Ahti jutus)
	\item Plokkskeemid jooksevad väga mitmest kohast (Priit, mroos, Ahti, Jaan jt.) läbi kui oluline vahend vigade leidmiseks. Paberil programmeerimine, kui selline samuti Jaan, Sergei, Ahti
	\item Raadiosõlm kui maagiline paik (Andrus Aaslaid jt.)
	\item Segavate faktorite puudus: Ahti, Arne
	\item Lihtsus
	\begin{itemize}
		\item Arne: ma sain oma Atarist lõpuni aru
		\item Mast ja Margus kirjutasid kahe kuuga modemipanga
		\item Jaani tekstiredaktor
		\item Kaspari kontroll MSXi üle (näpud auku ja masin suri)
	\end{itemize}
	\item Ülim usaldus
	\begin{itemize}
		\item Andrus (kuidas tööle mindi, kuidas Internetti jagati)
		\item Kuidas arvutiturve toimis (Kain ja Valitsusside)
	\end{itemize}
	\item Meedia (sh. trükimeedia) oluline roll. Ilmselt liikus raha ja tehnoloogia oli kohe rakendatav selle teenimiseks
	\begin{itemize}
		\item Tehnokratt
		\item Sten ja mina ja BrandSellers
		\item Kaspar
	\end{itemize}
	\item Üllatavalt sage seos näitekunstiga
	\begin{itemize}
		\item Tarvi ja näitering
		\item Jaanus
	\end{itemize}
	\item Võtmeinimesed
	\begin{itemize}
		\item Isikute kaupa
			\begin{itemize}
				\item Kõik seisid hiiglaste õlgadel (vt. kasvõi Anne juttu), aga selle jutu kontekstis on rida inimesi selgelt esile tulnud
				\item inimesed, kellel tundub olevat olnud väga oluline mõju (mitte tingimata inimesed, kes paljudest juttudest läbi käivad, vaid inimesed, kes on olnud oluliste (subjektiivselt) asjade edasi lükkajad)
				\begin{itemize}
					\item Anne Villems
					\item Tarmo Mamers
					\item Tarvi Martens
					\item Jaak Loonde
					\item Anto Veldre
					\item Sulo Kallas
				\end{itemize}
			\end{itemize}
		\item Äkki hoopis kategooriate kaupa?
		\begin{itemize}
			\item Asjade käivitajad. Anne, Anto
			\item Õpetajad. Lõvi, Jaak
			\item Visionäärid Tarmo, Margus
		\end{itemize}
	\end{itemize}
\end{itemize}


%%
% The back matter contains appendices, bibliographies, indices, glossaries, etc.



\backmatter


\printindex[ppl]
\printindex


\end{document}

