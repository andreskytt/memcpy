%!TEX TS-program = arara
% arara: latexmk: { clean: partial }
% arara: xelatex: { shell: true, synctex: true} 
% arara: makeindex
% arara: xelatex: { shell: true, synctex: true} 
% arara: xelatex: { shell: true, synctex: true} 
% arara: xelatex: { shell: true, synctex: true}
% arara: latexmk: { clean: partial }

\documentclass[a4paper]{tufte-book}
\usepackage[
    type={CC},
    modifier={by-nc-nd},
    version={4.0},
]{doclicense}


\ifxetex
  \newcommand{\textls}[2][5]{%
    \begingroup\addfontfeatures{LetterSpace=#1}#2\endgroup
  }
  \renewcommand{\allcapsspacing}[1]{\textls[15]{#1}}
  \renewcommand{\smallcapsspacing}[1]{\textls[10]{#1}}
  \renewcommand{\allcaps}[1]{\textls[15]{\MakeTextUppercase{#1}}}
  \renewcommand{\smallcaps}[1]{\smallcapsspacing{\scshape\MakeTextLowercase{#1}}}
  \renewcommand{\textsc}[1]{\smallcapsspacing{\textsmallcaps{#1}}}
\fi


\usepackage[T1]{fontenc}
%\usepackage[utf8]{inputenc}
\usepackage{polyglossia}
\setmainlanguage{estonian} 
\setotherlanguage{russian}
\newfontfamily\russianfont[Script=Cyrillic]{Linux Libertine}

\hypersetup{colorlinks}% uncomment this line if you prefer colored hyperlinks (e.g., for onscreen viewing)


%%
% Book metadata
%\title{print(memcpy[])\thanks{Thanks to Edward R.~Tufte for his inspiration.}}
\title{print(memcpy[])}
\author[Andres Kütt]{Andres Kütt}
\publisher{TeamConsulting}

%%
% If they're installed, use Bergamo and Chantilly from www.fontsite.com.
% They're clones of Bembo and Gill Sans, respectively.
%\IfFileExists{bergamo.sty}{\usepackage[osf]{bergamo}}{}% Bembo
%\IfFileExists{chantill.sty}{\usepackage{chantill}}{}% Gill Sans

%\usepackage{microtype}

%%
% Just some sample text
\usepackage{lipsum}

%%
% For nicely typeset tabular material
\usepackage{booktabs}

% Veiko keemia jaoks
\usepackage[version=4]{mhchem}

%%
% For graphics / images
\usepackage{graphicx}
\setkeys{Gin}{width=\linewidth,totalheight=\textheight,keepaspectratio}
\graphicspath{{graphics/}}

% The fancyvrb package lets us customize the formatting of verbatim
% environments.  We use a slightly smaller font.
\usepackage{fancyvrb}
\fvset{fontsize=\normalsize}

%%
% Prints argument within hanging parentheses (i.e., parentheses that take
% up no horizontal space).  Useful in tabular environments.
\newcommand{\hangp}[1]{\makebox[0pt][r]{(}#1\makebox[0pt][l]{)}}

%%
% Prints an asterisk that takes up no horizontal space.
% Useful in tabular environments.
\newcommand{\hangstar}{\makebox[0pt][l]{*}}

%%
% Prints a trailing space in a smart way.
\usepackage{xspace}

%%
% Some shortcuts for Tufte's book titles.  The lowercase commands will
% produce the initials of the book title in italics.  The all-caps commands
% will print out the full title of the book in italics.
\newcommand{\vdqi}{\textit{VDQI}\xspace}
\newcommand{\ei}{\textit{EI}\xspace}
\newcommand{\ve}{\textit{VE}\xspace}
\newcommand{\be}{\textit{BE}\xspace}
\newcommand{\VDQI}{\textit{The Visual Display of Quantitative Information}\xspace}
\newcommand{\EI}{\textit{Envisioning Information}\xspace}
\newcommand{\VE}{\textit{Visual Explanations}\xspace}
\newcommand{\BE}{\textit{Beautiful Evidence}\xspace}

\newcommand{\TL}{Tufte-\LaTeX\xspace}

% Prints the month name (e.g., January) and the year (e.g., 2008)
\newcommand{\monthyear}{%
  \ifcase\month\or January\or February\or March\or April\or May\or June\or
  July\or August\or September\or October\or November\or
  December\fi\space\number\year
}


% Prints an epigraph and speaker in sans serif, all-caps type.
\newcommand{\openepigraph}[2]{%
  %\sffamily\fontsize{14}{16}\selectfont
  \begin{fullwidth}
  \sffamily\large
  \begin{doublespace}
  \noindent\allcaps{#1}\\% epigraph
  \noindent\allcaps{#2}% author
  \end{doublespace}
  \end{fullwidth}
}

% Inserts a blank page
\newcommand{\blankpage}{\newpage\hbox{}\thispagestyle{empty}\newpage}

\usepackage{units}

% Typesets the font size, leading, and measure in the form of 10/12x26 pc.
\newcommand{\measure}[3]{#1/#2$\times$\unit[#3]{pc}}

% Macros for typesetting the documentation
\newcommand{\hlred}[1]{\textcolor{Maroon}{#1}}% prints in red
\newcommand{\hangleft}[1]{\makebox[0pt][r]{#1}}
\newcommand{\hairsp}{\hspace{1pt}}% hair space
\newcommand{\hquad}{\hskip0.5em\relax}% half quad space
\newcommand{\TODO}{\textcolor{red}{\bf TODO!}\xspace}
\newcommand{\ie}{\textit{i.\hairsp{}e.}\xspace}
\newcommand{\eg}{\textit{e.\hairsp{}g.}\xspace}
\newcommand{\na}{\quad--}% used in tables for N/A cells
\providecommand{\XeLaTeX}{X\lower.5ex\hbox{\kern-0.15em\reflectbox{E}}\kern-0.1em\LaTeX}
\newcommand{\tXeLaTeX}{\XeLaTeX\index{XeLaTeX@\protect\XeLaTeX}}
% \index{\texttt{\textbackslash xyz}@\hangleft{\texttt{\textbackslash}}\texttt{xyz}}
\newcommand{\tuftebs}{\symbol{'134}}% a backslash in tt type in OT1/T1
\newcommand{\doccmdnoindex}[2][]{\texttt{\tuftebs#2}}% command name -- adds backslash automatically (and doesn't add cmd to the index)
\newcommand{\doccmddef}[2][]{%
  \hlred{\texttt{\tuftebs#2}}\label{cmd:#2}%
  \ifthenelse{\isempty{#1}}%
    {% add the command to the index
      \index{#2 command@\protect\hangleft{\texttt{\tuftebs}}\texttt{#2}}% command name
    }%
    {% add the command and package to the index
      \index{#2 command@\protect\hangleft{\texttt{\tuftebs}}\texttt{#2} (\texttt{#1} package)}% command name
      \index{#1 package@\texttt{#1} package}\index{packages!#1@\texttt{#1}}% package name
    }%
}% command name -- adds backslash automatically
\newcommand{\doccmd}[2][]{%
  \texttt{\tuftebs#2}%
  \ifthenelse{\isempty{#1}}%
    {% add the command to the index
      \index{#2 command@\protect\hangleft{\texttt{\tuftebs}}\texttt{#2}}% command name
    }%
    {% add the command and package to the index
      \index{#2 command@\protect\hangleft{\texttt{\tuftebs}}\texttt{#2} (\texttt{#1} package)}% command name
      \index{#1 package@\texttt{#1} package}\index{packages!#1@\texttt{#1}}% package name
    }%
}% command name -- adds backslash automatically
\newcommand{\docopt}[1]{\ensuremath{\langle}\textrm{\textit{#1}}\ensuremath{\rangle}}% optional command argument
\newcommand{\docarg}[1]{\textrm{\textit{#1}}}% (required) command argument
\newenvironment{docspec}{\begin{quotation}\ttfamily\parskip0pt\parindent0pt\ignorespaces}{\end{quotation}}% command specification environment
\newcommand{\docenv}[1]{\texttt{#1}\index{#1 environment@\texttt{#1} environment}\index{environments!#1@\texttt{#1}}}% environment name
\newcommand{\docenvdef}[1]{\hlred{\texttt{#1}}\label{env:#1}\index{#1 environment@\texttt{#1} environment}\index{environments!#1@\texttt{#1}}}% environment name
\newcommand{\docpkg}[1]{\texttt{#1}\index{#1 package@\texttt{#1} package}\index{packages!#1@\texttt{#1}}}% package name
\newcommand{\doccls}[1]{\texttt{#1}}% document class name
\newcommand{\docclsopt}[1]{\texttt{#1}\index{#1 class option@\texttt{#1} class option}\index{class options!#1@\texttt{#1}}}% document class option name
\newcommand{\docclsoptdef}[1]{\hlred{\texttt{#1}}\label{clsopt:#1}\index{#1 class option@\texttt{#1} class option}\index{class options!#1@\texttt{#1}}}% document class option name defined
\newcommand{\docmsg}[2]{\bigskip\begin{fullwidth}\noindent\ttfamily#1\end{fullwidth}\medskip\par\noindent#2}
\newcommand{\docfilehook}[2]{\texttt{#1}\index{file hooks!#2}\index{#1@\texttt{#1}}}
\newcommand{\doccounter}[1]{\texttt{#1}\index{#1 counter@\texttt{#1} counter}}

% Generates the index
\usepackage{imakeidx}
\makeindex[name=ppl, title={Nimede register}]
\makeindex[title={Indeks}]

% See also
\makeatletter
\newcommand{\gobblecomma}[1]{\@gobble{#1}\ignorespaces}
\makeatother

\usepackage{csquotes}


%% Versioneerimine

\newcounter{run}
\InputIfFileExists{\jobname.runs}{}{}
\stepcounter{run}

\usepackage{atveryend}
\usepackage{newfile}
\AtVeryEndDocument{%
  \newoutputstream{runs}%
  \openoutputfile{\jobname.runs}{runs}%
  \addtostream{runs}{\string\setcounter{run}{\number\value{run}}}%
  \closeoutputstream{runs}%
}

\usepackage{needspace}
\raggedbottom
\addtolength{\topskip}{0pt plus 10pt}
%% Küsimuse vormistus
\newcommand{\question}[1]{\begin{samepage}\needspace{3\baselineskip}\textbf{#1\\}\end{samepage}}
%\newcommand{\question}[1]{\begin{minipage}{\textwidth}\textbf{\enquote{#1}}\end{minipage}}

% Reset the sidenote number each chapter
\let\oldchapter\chapter
\def\chapter{%
  \setcounter{footnote}{0}%
  \oldchapter
}


\usepackage{multicol}
\usepackage{pdfpages}
%\usepackage{verse}
\begin{document}

% Front matter
\frontmatter

% r.1 blank page
\blankpage
%\includepdf[fitpaper=true, pages=-]{mock_kaas.pdf}

% v.2 epigraphs
\newpage\thispagestyle{empty}
\openepigraph{%
Design and programming are human activities; forget that and all is lost.
}{Bjarne Stroustrup%, {\itshape Design, Form, and Chaos}
}
\vfill
\def\svgwidth{6cm}
%\input{barcode.pdf_tex}

\begin{fullwidth}
\sffamily\large
\begin{doublespace}
%\noindent\allcaps{Ärge valetage isad }\\ % The quote
%\noindent\allcaps{ära hoia kinni ema mind}\\ % The quote
%\noindent\allcaps{Need ei ole halvad sõbrad}\\ % The quote
\noindent\allcaps{\ldots}\\ % The quote
\noindent\allcaps{see on minu Vennaskond ja ring}\\ % The quote
\noindent\allcaps{Vennaskond. \enquote{Jumal kaitse vennaskonda}} % The author
\end{doublespace}
\end{fullwidth}
%\vfill
%\openepigraph{% 
%Ärge valetage isad ära hoia kinni ema mind Need ei ole halvad sõbrad see on minu Vennaskond ja ring}{Vennaskond. \enquote{Jumal kaitse vennaskonda}}
%\vfill
%\openepigraph{%
%\ldots the designer of a new system must not only be the implementor and the first 
%large-scale user; the designer should also write the first user manual\ldots 
%If I had not participated fully in all these activities, 
%literally hundreds of improvements would never have been made, 
%because I would never have thought of them or perceived 
%why they were important.
%}{Donald E. Knuth}


% r.3 full title page
\maketitle


% v.4 copyright page
\newpage
\begin{fullwidth}
~\vfill
\thispagestyle{empty}
\setlength{\parindent}{0pt}
\setlength{\parskip}{\baselineskip}
Copyright \copyright\ \the\year\ \thanklessauthor

\par\smallcaps{Toimetanud Kadri Põdra}
\par\smallcaps{Välja andnud \thanklesspublisher}

%\par\smallcaps{tufte-latex.googlecode.com}

\par \doclicenseThis 

\par\textit{\monthyear. Version V0.\therun}
\end{fullwidth}

% r.5 contents
\tableofcontents

%\listoffigures

%\listoftables

% r.7 dedication
\cleardoublepage
~\vfill
\begin{doublespace}
\noindent\fontsize{18}{22}\selectfont\itshape
\nohyphenation
Toivole, Meelisele ja teistele.
\end{doublespace}
\vfill
\vfill


% r.9 introduction
\cleardoublepage
\chapter{Sissejuhatus}
Tere. See siin on memcpy. Nende sõnadega olen sisse juhatanud suurt hulka 
intervjuusid oluliste inimestega ja nüüd on teie ees see tekst. 

Aga miks? 

Põhjus, tuleb tunnistada, on lihtne. Nagu ütleb Villu Tamme loos 
\enquote{Paneme punki}:

\begin{verse}
Tahan kord saada selliseks, nagu \\
on Villu või Freddy või Rott või Striit\\
\end{verse}

See raamat räägib inimestest, kes on mulle oma tarkuse, oskuste ja olemusega 
eeskujuks olnud. Kui Toivo Annus\index[ppl]{Annus, Toivo} mu kunagi Skype'i 
tööintervjuule kutsus, kõndisime piki toonase kontori koridori, mille ühele 
poole avanesid töö- ja teisele nõupidamisruumid. Kõigist ustest paistis ja 
koridoris tuli järjest vastu inimesi, kellega mul kas oli alati olnud rõõm koos 
töötada või kellega olin alati tahtnud koos töötada. Memcpy on mingitpidi 
katse toda tunnet uuesti kogeda.

Siiski ei ole isiklik emotsionaalne heaolu tingimata hea põhjus inimesi 
tülitada või veeta tunde teksti transkribeerides ja toimetades. Sügavam põhjus 
memcpy taga on vajadus dokumenteerida inimesi, kelle 
väikeste näppude alt on välja tulnud kõik suuremad Eesti IT-edulood. 

Riigi Infosüsteemi Ametis\index{Riigi Infosüsteemi Amet} töötades pidin aastate 
kaupa peaaegu igal nädalal rääkima riigi infosüsteemist, selle ülesehitusest ja 
ajaloost ning vastama küsimustele. Mul ei olnud pikka aega vastust sagedasele 
küsimusele \enquote{miks Eestis ja mitte mujal?}. Meil ei ole objektiivselt 
vaadates erilisi põhjusi olla oma naabritest edukamad, me isegi ei tee midagi 
eriti innovatiivset, aga ometi oleme suutnud kiiresti edasi liikuda ja meil on 
kogu maailmas selge positiivne IT imago. Miks? 

Vastust otsides jõudsin ikka ja jälle usalduse küsimuseni. Mingil põhjusel Eestis usaldatakse 
IT inimesi, neid kaasatakse oluliste otsuste juurde ja nad suudavad 
selle usalduse ka välja teenida. Sedalaadi suhetel on juured ja nende üle 
% TODO: Kas siin ei oleks parem "ja nad suudavad seda usaldust ka õigustada"?
% TODO: Minu keeletunde järgi on välja teenimine midagi, mis juhtub enne usaldamist, samas kui õigustamine või mitte õigustamine selgub pärast.
juureldes jõudsin aega natuke enne ja pärast vabariigi taassündi. Ühtäkki 
hakkas toona meile jõudma arvuteid, kuid keskmisel inimesel puudus võimekus 
neid kasutada. Teisalt oli tekkinud toimekas seltskond kodanikke, kes oskasid 
arvutit kasutada, kuid kellel ei olnud neile ligipääsu. Ja nii sündiski 
arusaam, et koos on mõistlik. Et ITst on kasu. Et kuskil on kellegi 
lahendamist vajavad elulised probleemid. Ja, mis peamine, et seda suhet ei ole 
mõistlik lõhkuda.\sidenote{Vt näide lk \pageref{sisu:andrus_usaldus} 
Sealsamas ka haruldane näide usalduse kuritarvitamisest.} 

Kust too seltskond tuli, kuidas toimis, kes sinna kuulusid? Nendele küsimustele 
otsib memcpy vastust määrani, mida \emph{fanboy} roll vähegi võimaldab. Seetõttu ongi 
fookus inimestel, mitte ettevõtetel\sidenote{Vähemalt Dateli, 
Proeksperdi ja MicroLinki kohta on väikesetiraa\v{z}ilised ajalood ilmunud.} 
või kurioossetel seikadel. 

Ometi ei ole ma ajaloolane ega folklorist, nii et kas memcpy-t ei võiks teha 
professionaalid? Kõik katsed leida keegi asjatundja asja läbi viima luhtusid 
sel lihtsal põhjusel, et kellelgi ei olnud teema vastu piisavat isiklikku huvi 
ja katsed ettevõtmist kuidagi rahastada jooksid liiva. Pärast esialgse 
idee formuleerimist veetsin umbes aasta, üritades edutult leida tegijaid ja 
rahastust. Seejärel veetsin umbes pool aastat ennast veendes, et memcpy ei pea 
olema täiuslik. Intervjuude ettevalmistamine, salvestamine, toimetamine ja 
järeltöötlus on tehniliselt keerulised protsessid, mida ma ei vallanud tööga alustades ja 
ei valda ka praegu. Siiski oli selge, et enne läheb issanda päike looja, kui ma 
neis valdkondades mind ennast rahuldava taseme saavutan. Nii tuli süda kõvaks teha ja teha memcpy-t 
mitte nii hästi, kui peaks, vaid nii hästi, kui suudan. Seetõttu on 
eriti esimeste memcpy episoodide helikvaliteet päris kole ja see häirib mind 
siiani.

Sügisel 2018 sai purki esimene intervjuu Prontoga\index[ppl]{Pronto} ja 
kevadeks veel kaheksa. Suvel on inimesed rohkem liikvel ja nii jätkasin 
2019. aasta sügisel juba märksa parema planeerimisega, saades napilt enne COVID-19 
pandeemia Eestisse jõudmist 2020. aasta märtsis linti ka teise hooaja 
intervjuud. Neid kokku lõigates jäi häirima, et need ei ole mugavalt 
otsitavad. Inimesed, ettevõtted ja kohad jooksid läbi eri lugudest, aga 
millistest täpsemalt? Väga raske on öelda midagi võrgustiku kohta, kui seda võrgustikku 
saab uurida vaid tipphaaval. Kuna pandeemia tõttu uusi intervjuusid salvestada ei 
saanud, siis oli loogiline samm võtta aega olemasolevate intervjuude
transkribeerimiseks, toimetamiseks ja indeksiga varustamiseks, et luua
seda teksti siin. 

Seega on memcpy igati isiklik projekt ning sellisena paratamatult 
piiratud. Ma ei saa ega kavatsegi toota kiretut ajaloodokumenti\sidenote{Mu 
enda peatükk on lisatud just võimaldamaks paremini mõista filtrit, mille ülejäänud 
sisu on läbinud.} ning teisalt ei saa lootagi, et võiksin suuta rääkida kõigi 
huvitavate ja oluliste inimestega. Kõik lihtsalt ei mahu raamatusse, mõned ei 
soovinud (minuga) rääkida ja mõned ei tulnud pähe. Andestust! 

Mõningased piirid intervjueeritavate valikule seadis ka projekti ajaline määratlus just 
kaheksa- ja üheksakümnendatega. Seetõttu on enamasti välja jäänud näiteks 
Mainori\index{Mainor} ümber tegutsenud seltskond ning natuke vanema põlvkonna, 
näiteks kadunud Ahto Kalja\index[ppl]{Kalja, Ahto} ja Monika 
Oiti\index[ppl]{Oit, Monika} tegemised. Samuti on praktilistel põhjustel 
valikus vähe Tartus tegutsenud ja venekeelse taustaga inimesi. Üle ega 
ümber ei saa ka asjaolust, et kunagi IT-rahva kohta laialt kasutusel olnud 
mõiste \enquote{patsiga poisid} leiab ka memcpy-s otsest peegeldust. Enamasti 
on tõesti tegu poistega. Kahju küll, aga uuritav kogukond paraku oli 
ebaproportsionaalselt maskuliinne\sidenote{Lk \pageref{sisu:tydrukud} on 
natuke lähemalt juttu tolle nähtuse põhjustest.} ja selle teistsugusena kujutamine ei oleks 
päris õige. Samas olid Vilve Vene\index[ppl]{Vene, Vilve} ja Anne 
Villemsi\index[ppl]{Villems, Anne} intervjuud ühed kõige huvitavamad salvestada.

Inimeste mälu on erinev. Seetõttu lähevad inimeste lood -- ja memcpy eesmärk on just nimelt lugude 
talletamine -- omavahel detailides aeg-ajalt vastuollu. Otsesed 
kõrvalekalded teadaolevast\sidenote{Teadmine paraku laieneb kogu aeg ja kindlasti 
jääb osa kaheldavaid detaile märkamata.} reaalsusest on osundatud ning pisemad vead 
parandatud. Samas ei maksa oodata, et järgnevatel lehekülgedel saaks näiteks vana 
Tartu ja Tallinna koolkondade vastuolu kuidagi objektiivselt lahendatud. 
Tegu on lugudega ja neid tuleb paratamatult võtta tera soolaga.

Samuti tuleb arvestada, et suuri asju tegevad huvitavad inimesed on harva 
lihtsad isiksused. Olen üritanud kunagisest küllaltki keerulisest suhete taagast 
oskust mööda üle olla. Seetõttu on intervjuud järgnevatel lehekülgedel 
tähestikulises, mitte intervjuude toimumise või olulisuse järjekorras. 

Transkribeeritud ja \emph{podcast}'i eetrisse läinud juttudest on üksikud 
detailid ka välja jäetud, sest mõnest asjast ei taha inimesed väga rääkida ja 
mõnda asja ei ole paslik tiražeerida. Üheksakümnendad oli päris hull ja 
tänasest täiesti erinev aeg. Tegu on siiski nüanssidega, mis suurt pilti ei 
tohiks mõjutada. Muidu on intervjuud enamasti täies mahus\sidenote{Erandiks on 
intervjuu Tarvi Martensiga\index[ppl]{Martens, Tarvi}, kellega meil oli väga paljust 
rääkida ja salvestasime kaks episoodi, mis teemade poolest osaliselt kattusid. 
Seega tuli kirjalikus tekstis selguse huvides asju natuke ümber tõsta ja 
tihendada.} ja küllalt originaalilähedase keelekasutusega edasi antud. 
Sellest ka anglitsismide ja võõrkeelsete terminite suur hulk. 
Aga kuna keelekasutus annab huvitava akna inimesse, eelistasin vahel autentsust 
ilusale emakeelele.\sidenote{Seesinane eelistus põhjustas toimetajale kahjuks palju meelehärmi.}

Tekst on mõeldud olema ka arvutikaugematele inimestele üldjoontes arusaadav: 
konteksti mõistmiseks olulised terminid on lahti seletatud ja tänaseks ehk 
ununenud asjad viidatud, kuid detailid otsib huviline ise välja. Eesmärk ei ole 
olnud anda struktureeritud ülevaadet arvutustehnika ajaloost või vanade 
tehnoloogiate toimimisest. Intervjuudes üritasin küsida võhiku 
positsioonilt, mis oli seda lihtsam, et seda ma paljuski olengi.

Kuigi fookus on inimestel ja nende lugudel, olen mõneti ajaloo säilitamise ja 
mõneti oluliste inimeste äramärkimise eesmärgil lisanud ka esimese Eesti regiooni
sisaldanud Fido \emph{nodelist}'i ja kõige varasema Eesti BBSide nimekirja, mille 
leidsin.\sidenote{Vt lk \pageref{sisu:nodelist}.}

Järgnevaid lehekülgi ei ole juba kasvõi nende mahu pärast ehk mõistlik kaanest 
kaaneni lugeda. Targem on lapata, kasutades kas indeksit, 
alustada mõnest huvitavamast loost või lihtsalt alates juhuslikust leheküljest 
pea ees minevikku hüpata.

Raamatu kujunduses on toetutud Edward Tufte tööle läbi vastava \LaTeX-i klassi.\sidenote{\url{https://github.com/Tufte-LaTeX/tufte-latex}.}
Tegu on teadliku valikuga: nii on lisaks ääremärkustele lehekülgedel piisavalt ruumi 
ka lugeja märkuste, arvamuste või joonistuste jaoks. Rõõmsat sodimist!



\chapter{Tänusõnad}
Memcpy on, nagu öeldud, isiklik projekt. Ometi on põhjust tänulik olla. 
Eelkõige kaasteelistele, kes on minuga ja minu ümber olnud. Neile, kes võtsid 
oma tihedast päevast tunni, et minuga juttu rääkida on eriline tänu. Neile, keda saame tänada
kunagi ühes kohvikus, on pühendus.

Aga on ka konkreetseid inimesi ja ettevõtmisi, kelleta see raamat ei oleks sellisel kujul sündinud. 

Esmalt tänan transkriptsioonitarkvara autoreid\sidenote{Alumäe, Tanel; Tilk, Ottokar; Ullah, Asad. Advanced 
Rich Transcription System for Estonian Speech. Baltic HLT 2018.}, kelle tööta see raamat kindlasti ei oleks sündinud. Samuti tänan Kadri Põdrat, kes kogu teksti eestikeelseks tegi. 

Teiseks tahan tänada inimesi, kes panustasid ühel või teisel viisil lugude täiendamisse märkuste ja selgitustega ning olid muidu abiks:

\begin{description}
	\item[Tarmo Mamers]\index[ppl]{Mamers, Tarmo} aitas palju nimede ja muude detailidega
	\item[Ott Köstner]\index[ppl]{Köstner, Ott} tegi memcpy podcast'i kaanepildi
	\item[Vootele Voit]\index[ppl]{Voit, Vootele} kommenteeris 
 ZX Spectrumi kiibistikku puudutavat
	\item[Mart Palmas]\index[ppl]{Palmas, Mart} aitas Soome telekavade teket mäletada 
	\item[Gert Silling]\index[ppl]{Silling, Gert} aitas jõuda Ants Roose ja Algoritmini 
	\item[Ants Roose]\index[ppl]{Roose, Ants} andis infot legendaarse Arvutustehnika ja Andmetöötluse ning Algoritmi kohta 
\end{description}

Hulk inimesi panustas Hooandja platvormil selle raamatu ilmumisele paberil ka rahaliselt. Aitäh teile kõigile!

\begin{multicols}{2}
Martin Lillepuu\\
Hanno Sirkel\\
Urmo Rae\\
Maarja-Leena Saar\\
Robert Laursoo\\
Priit Pääsukene \\
Risto Hinno\\
Ardi JüRgens\\
Vahuri Voolaid\\
Kalle-Rasmus Volkov\\
Taavi Tiirik\\
Joonatan Samuel\\
 Rudolf Osman\\
Renee Trisberg\\
Karmo Kristjan\\
Aldo Mett\\
Aivar Naaber\\
Andrus Kanter\\
Tarmo Tali\\
Maris Kütt \\
Neeme Praks\\
Raido Aarop\\
Asko Seeba\\
Lauri Väin\\
Jaanus Kase\\
Maili Mahlapuu\\
Argo Mändmaa\\
Elen Leigri\\
Allan Poola\\
Diana Poudel\\
Siim Sikkut\\
Sille Arikas\\
Vallo Veinthal\\
Peeter Marvet\\
Kaarel Jõgi\\
Kristjan Kuhi\\
Anti Veeranna\\
Kristi Küppar\\
Janno Teelem\\
Tarmo Luumann\\
Vasli Ekke\\
Renno Veinthal\\
Kitty Mamers\\
Jüri Kaljundi\\
Alar Mäerand\\
Martti Kuldma\\
Meelis Lang\\
Gristel Tali\\
Andres Ääremaa\\
Reimo Tõnis\\
Toomas Vaks\\
Klemens-Augustinus Kasemaa\\
Taavi Tamkivi\\
Els Kütt\\
Kaur Kullman\\
Hasso Tepper\\
Jüri Laur\\
Hillar Leoste\\
Aho Augasmägi\\
Elo Lindi\\
Allan Kändmaa\\
Tikan Tarko\\
Andrei Reinus\\
Peeter Russak\\
Viljo Marrandi\\
Indrek Siitan\\
Sten Tikerpe\\
Kalev Pihl\\
Jaagus Tõnis\\
Tobias Johannes Koch\\
Madis Kaal\\
Alvar Soome\\
Enn Sutting\\
Anu Käver\\
Ragnar Toomla \\
Indrek Rebane\\
Laas Vahur\\
Tarmo Mamers\\
Madis Pink\\
Ülle Kroon\\
Raul Allikivi\\
Mari-Liis Lind\\
Silver Salla\\
Mihkel Solvak\\
Evelin Kasenõmm\\
Edgar Kivit\\
Kaia Kalberg\\
Mart Parve\\
Rainer Kuhi\\
Risto Hansen\\
Margus Holland\\
 Priit Siilaberg \\
Toomas Rand\\
Sven Varkel\\
Joonathan Mägi\\
Reet Prii\\
Ivar Zarans\\
Janek Metsallik\\
Holger Rünkaru\\
Kerti Alev\\
Teet Vaher\\
Taavi Tamm\\
Lauri Antalainen\\
Kristo Mägi\\
Alek Kozlov\\
Raimo Reiman\\
Lauri Tarend\\
Mart Oruaas \\
Villu Teearu\\
Toomas Viirsalu\\
Fredi Dorbek\\
Priit Haamer\\
Tristan Krass\\
Sten Tamkivi\\
Toomas Lepik\\
Taavi Tänavsuu\\
Agur Jõgi\\
Martin Raag\\
Kristjan Rebane \\
Ivo Mägi\\
Tarvi Martens\\
Priit Potter\\
Argo Roots\\
Martin Villig\\
Marek Tiits\\
Katrin Laas-Mikko\\
Marko Jõemets \\
Jan-Erik Moon\\
Sutermäe Urmo\\
Martin Paljak\\
Ehouse OÜ\\
Tanel Vakker\\
Kalle Tabur\\
Margus Ernits\\
Dmitri Mihhailov\\
Marko Kivimäe\\
Erik Suit\\
\end{multicols}
 
%%
% Start the main matter (normal chapters)
\mainmatter


\chapter{Andrus Aaslaid}
%!TEX TS-program = arara
% arara: myindex

\index[ppl]{Aaslaid, Andrus}
\textbf{\enquote{Kuidas sa arvutite juurde jõudsid}}

Tihti on nii, et meil on elu muutvad otsused aga me ei mäleta, kuidas me neid 
tegime. Aga neid seda juhust ma mäletan täpselt. Mul oli juba toona 
raadio-hobi. Olin selline põhikooli juntsu ja mulle meeldis hirmsasti mööda 
lühilainet ringi kammida. Meil oli kodus selline Melodija 101 stereo, Riia 
raadiotehase\sidenote{A. S. Popovi nimeline Riia Raadiotehas, alates 1951 Rigas 
Radio Rupnica} toodang. Sellega ma siis seiklesin suviti, kui midagi targemat 
teha ei olnud, mööda eetrit. Mul oli tegelikult kaks raadiot: esimene juba 
mainitud Melodija, sellele lisaks veel detektorvastuvõtja, mille mu poolvend 
oli mulle ehitanud. Selle viimasega ma istusin pööningul, kus oli muu seas mitu 
aastakäiku ajakirja \begin{russian}Техника - 
молодёжи\end{russian}\sidenote{Aastast 1993 ilmuv algselt Nõukogude ja nüüd 
Vene populaarteaduslik ajakiri.}. Kuna perekond tegeles põllumajandusega ja oli 
üks konkreetne põllumajandusnipp. Nimelt raamatukogudest toodi vanu ajakirju, 
need rebiti lehtedeks, need lehed keerati sisuliselt ümber sellise õõnsa 
põhjaga pudeli sellisteks väikesteks pottideks ja  sinna sisse istutati taimed. 
Paber lagunes mulla sees, ära taim pääses põllul vabaks. Ka neid ajakirju oli 
seal tohutu hunnik ja nende ajakirjade juures oli mitu aastakäiku 
\begin{russian}Техника - молодёжи\end{russian}'t. Lappasin siis neid ajakirju, 
detektoriklapid peas. 

Igatahes ükskord astusin ma tuppa, lülitasin Melodija sisse ja sealt öeldi, et 
Tallinna 43. Keskkool\index{Koolid!Tallinna 43. Keskkool} on otsustanud hakata 
eksperimentaalseks Tehnikaülikooli\index{Tallinna Tehnikaülikool}\sidenote{Tol 
ajal oli ta veel Tallinna Polütehniline Instituut, TPI} ettevalmistuskooliks ja 
nad võtavad kümnendasse klassi vastu õpilasi, kes tahaksid edasi õppima minna 
TPIsse. Kuulasin uudise ära, lülitasin raadio välja, läksin vanemate juurde ja 
teatasin, et ma lähen Tallinnasse kooli. Ma olin siis 14.


\textbf{\enquote{Aga kust sa siis pärit oled, et niimoodi Tallinna kooli pidi 
minema?}}


Pärit olen ma tegelikult kahesaja meetri kauguselt sealt, kust ma täna elan, 
ehk siis Tallinnast. Aga mul perekond otsustas evakueeruda 
Muhusse\index{Muhumaa}, kui ma olin kahe- või kolmeaastane, siis mind 
deporteeriti sinna. Nii et oma põrsapõlve olen kõik Muhus  veetnud ja siis ühel 
hetkel sealt siis tagasi putku tulnud tehnoloogia juurde. 

\textbf{\enquote{No mõni ime, et te Mastiga\index[ppl]{Kaal, 
Madis}\index[ppl]{Mast|see{Kaal, Madis}} hästi läbi saate!}}

Me oleme Mastiga ühe kooli poisid tegelikult selles mõttes, et Mast oli samas 
koolis keskkoolis, kui mina olin seal põhikoolis. Me oleme isegi sama 
raadiosõlme väisanud mõnda aega. Aga tollel ajal noh, nagu ikka, eriti veel 
maakohtades, et ega nooremad ja vanemad väga läbi ei käinud. Aga Mast oli hea 
poiss, ei peksnud nooremaid ega midagi. 

\textbf{\enquote{Mis seal lühilaine pealt kostis, mida sa kuulasid? Muusikat?}}

Ei muusikat kuulati raadio Luksemburgist. Sealt lühilaine pealt tulid erinevad 
hääled. Tuli morset, tuli mingeid huvitavaid kahinaid ja sahinad ja siis keegi 
luges numbreid ja. Ega tegelikult lühilaine on siiamaani päris hea tervise 
juures, et seal on  eeter siiamaani on  maast laeni sodi täis, et ega ta olemus 
ei ole, väga palju muutunud. Võib-olla mingeid propagandasaateid on vähemaks 
jäänud ja Hiina raadiojaamu on vaikselt kinni pandud seoses sellega, kuidas 
Internet peale tuleb. Aga, üldiselt on see lühilaine samasugune, nagu ta oli 
nelikümmend aastat tagasi, ma arvan.

\textbf{\enquote{Kas seal sinu ajakirjade hulgas juba arvutiajakirju ka oli?}}

Esimest arvutit ma nägin tegelikult just tänu sellelesamale poolvennale, kes 
mulle selle detektori ehitas. Ühel hetkel ta ütles, et Guido 
Tammissaar\index[ppl]{Tammissaar, Guido}, tema oli Eesti Energia 
Arvutuskeskuse\index{Eesti Energia Arvutuskeskus} üks tegelastest. Ja ühel 
hetkel tuli sinna maale ja ütles, et tule kaasa paariks päevaks, näed, mis asi 
see arvuti on. Et sind see tehnika, asi huvitab. Ja lubati mind siis maalt 
linna paariks päevaks, Estonia puiestee arvutuskeskuses olid veel põhiliselt 
ESM-id tollel ajal\sidenote{Kas \url{https://en.wikipedia.org/wiki/BESM}? }. Ja 
üks mingisugune CP\textbackslash Mi\sidenote{CP\textbackslash M oli 1974. 
aastal Inteli 8080/85 protsessorisarja tarvis turule toodud 
operatsioonisüsteem, mille 1980ndatel asendas mitmes mõttes sarnane MS-DOS} 
masin, mis tagantjärgi tundub jube kosmiline selles mõttes, et ta tundus 
mingisugune sotsmaa disain, eks ta, mingi Bulgaarlane oli. Olen mõelnud, et 
peaks üles otsima, et  masin see selline võis olla, aga ma siiamaani täis sada 
protsenti kindel ei ole. 

Selle peale ma niisama natuke klõbistasin. 
SM-4\index{Arvutid!SM-4}\sidenote{SM-4 oli PDP-11/40\index{PDP-11} ühilduv 
Nõukogude päritolu ja terves Idablokis toodetud arvutisüsteem} peal ma 
kirjutasin oma esimese BASICu\index{Keeled!BASIC} programmi sellel samal 
päeval. See oli derivaat mingist asjast, mida mulle näidati, et näed, umbes nii 
käib. Ja edasi ma olin \emph{hooked}. Sellest ühest päevast piisas, et sõltlane 
tekitada. 

\textbf{\enquote{See oli siis enne seda, kui sa otsustasid, et nüüd sa oled 
neliteist ja lähed Tallinnasse kooli?}}
Ma ei oskagi öelda, ma ei ole sada protsenti kindel, kumb oli enne, kumb oli 
pärast. Et et kas, kas see huvi, et tulla Tallinnasse, mängis rolli. Ega nad ju 
arvutikallakut tegelikult ei propageerinud.  Suurema rõhuga oli  elektroonika, 
tarkvara osa, seda nad väga ei reklaaminud. Minust pidi ikka elektroonik saama 
tegelikult, mis minust nüüd ka vahepeal sai, aga tollel hetkel ikkagi arvutid 
tundusid nagu see päris asi. 

\textbf{\enquote{Kas selles 43. keskkoolis valmistati päriselt ette ka 
ülikooliks? Oli sellest kasu?}}

See oli selline kahe teraga mõõk selles suhtes, et valmistati ette ja 
valmistati väga hästi. Sellepärast et see keskkooliprogramm oli pandud kokku, 
tolle aja inseneri õpetajate poolt, kes teadsid suhteliselt hästi, mida oleks 
vaja õpetada selleks, et põhi alla tuleks. Mis tähendas seda, et me saime 
läbisegi  tavalisi keskkooliaineid ja siis ühel hetkel tuli härra 
Tiidemann\index[ppl]{Tiidemann, Tiit} ja hakkas meile rääkima võllide 
epüüridest\sidenote{Epüür (prantsuse sõnast épure) on teatava suuruse asukohast 
olenevate väärtuste graafiline esitus}. Sisuliselt me tegime käsitsi võllidele 
rakendavate jõudude arvutusi, et kust kohast läheb võll katki, kui ta on 
sellise jämedusega siit tollase jämedusega sealt. Siis vahelduseks loeti meile 
mingisugust teise kursuse elektrotehnikat. Sinna vahele me saime mingisugust 
inseneripsühholoogiat, mida Toomsalu\index[ppl]{Toomsalu, Arvo} luges, mis ei 
olnud vist üldse TPI õppekavas sees. Ta oli nagu kõikidele eksperiment. 
ETEK\index{ETEK}  oli tema lühikene nimi, Ants Reili\index[ppl]{Reili, Ants} ja 
Peeter Krosberg\index[ppl]{Krosberg, Peeter} teda tegid. Ta oli selles mõttes 
väga äge üritus, et ta oli ikkagi täiesti \emph{green-field}. Eriti kuna me 
olime esimene lend. 

Meil olid muidugi seal veel omaette sellised lahedad asjad nagu see näiteks, et 
enne meid oli keskkool tühjaks löödud ja me olime kolm aastat keskkooli kõige 
vanem klass. Mis tähendas seda, et me olime sisuliselt nagu jumalad koolis. 
Tänu sellele mitmed probleemid jäid olemata, mida muidugi tavalistes 
keskkoolides tol ajal veel eksisteeris. Keegi kedagi väga ei toginud ega 
nüginud ja samal ajal kuidagi see areng toimus, nii et, et mingisugune väärikus 
tekkis kõigile. 

Kahe teraga mõõk oli ta sellepärast, et ta andis nii kõva põhja, et väga paljud 
läksid otse tööle. Me saime ju kõik, kes keskkooli lõpetasid, see on 
automaatselt TPIsse sisse. Meil ei olnud vaja sisseastumiseksamit teha. Mis 
tähendas, et kogu see vist kaheksateist õpilast marssis otse TPIsse. Nendest 
kooli lõpetas nominaalajaga vist  kas üks inimene või kaks või kolm, ma päris 
ei mäletagi. Ikkagi käputäis. Hästi paljud läksid otse tööle. Kuna aeg oli ka 
selline, et et kuidagi see, mida TPIs tollel ajal arvutiteadusena õpetati 
ikkagi päris elule veel järgi ei jõudnud. See pidi olema aasta 91-92. Siis, kui 
see kambriumi plahvatus siin Eestis toimus.

Ühest küljest  mina istusin arvuti taga ööd ja päevad ja kirjutasin 
mingisugusele suurele autopargile ihuüksi  mingisugust tarkvara, mis pidi 
üleval hoidma tervet autoparki. Ja samal ajal siis üritasin kuidagi nügida 
ennast läbi SuperCalci\index{SuperCalc}\sidenote{Varajane tabelarvutussüsteem, 
algselt loodud CP\textbackslash M operatsioonisüsteemile} arvestusest TPIs kus 
aegajalt tuli nagu õppejõule näidata, et ära nii tee, nii päris ei käi see asi. 
Mitte, et nad oleks rumalad olnud, aga nemad õpetasid seda, mida nad olid kogu 
aeg õpetanud. Aga nüüd tekkis ühel hetkel selline seis, kus reaalne elu liikus 
palju kiiremini kui õppekava.

\textbf{\enquote{Aga kuidas sa selle programmeerimise juurde ikka jõudsid? Sa 
pidid ikka kuskil harjutada saama seda?}}

See oligi  tänu sellele 43. keskkoolile, noh, tänane siis 
Tehnikagümnaasium\index{Koolid!Tehnikagümnaasium|see{Tallinna 43. Keskkool}}, 
et, see eksperiment kestis ja mõnes mõttes kestab tänaseni. Seal oli 
põhimõtteliselt  esimest korda selline päris arvuti-inimese elu. Kuna 
IT-spetsialiste  liiga palju ei olnud, siis juhtus selline hämar lugu, et meile 
Eero Tohvriga\index[ppl]{Tohver, Eero} ulatati arvutiklassi võtmed ja hakati 
koolist palka maksma kümnendas klassis. See natukene vist oli tegelikult seotud 
sellise koolipoolse kerge kaastundega. Sellist otseselt tööstuskooli peale 
kaheksandat klassi tulemise traditsiooni juba mõni kümmend aastat ei olnud 
vahepeal olnud ja kõigile tundus see, et laps tuleb üksi Tallinnasse kangesti 
hirmus. Ja kuidagi ma arvan, et see oli pigem selline koolipoolne stipendium. 
Aga jah, põhimõtteliselt meile maksti kahe peale täis õppejõu palk välja, mis 
oli põhimõtteliselt ma kahtlustan, et mitte palju väiksem, kui need õpetajad 
ise seal tollel hetkel said. Nii hästi ei ole ma kunagi oma elus ei varem ega 
hiljem elanud nagu keskkooli ajal. 

\textbf{\enquote{Mis te siis tegite selle raha eest?}}

Käisime restoranis söömas ja mida ikka lapsed rahaga teevad. Aga kool sai 
selle, et nad rohkem ei pidanud selle arvutiklassiga tegelema.  Kõik, mis seal 
oli, neid 
Iskraid\index{Arvutid!Iskra}\sidenote{\begin{russian}Искра\end{russian} oli 
mitmel pool Nõukogude Liidus eri modifikatsioonides toodetud IBM/XT kloon} oli 
siis vist kolm või neli (alguses oli kolm, siis tuli üks uuemat tõugu juurde), 
seda me siis seal niimoodi püsti hoidsime, et tunnid seal toimusid. Meie asi 
oli hoolitseda, et masinad töötaksid ja seal midagi saaks õpetada. Mingil 
hetkel, kui me juba ise natuke vanemad olime, hakkas sinna juurde tekkima 
mingisugune kamp nooremaid huvilisi, kes seal ka siis hängisid. Ta oli selline 
täitsa tüüpiline arvuti ökosüsteem. Kunagi suvel käisime, remontisime sellesama 
arvutiklassi ära: värvisime ja panime uued põrandakatted. Ühesõnaga käitusime, 
ma loodan,  heaperemehelikult temaga. 

\textbf{\enquote{Ei ole kuulnud, et kellelgi oleks heaperemeheliku käitumisega 
probleeme olnud sarnastes situatsioonides}}

Tead, aga ajad olid sellised, inimeste usaldus oli suur. See arvuti oli nagu 
selline ühtepidi eriti müstiline, teistmoodi teda nagu kardeti vanema 
generatsiooni poolt. Kujutad ette, et oli (ma siiamaani ei tea nende inimeste 
nimesid naljakal kombel, ega ma vist pole ka kunagi teadnud), aga kunagi oli 
Eestis selline turismibüroo nagu Sarved ja Sõrad\index{Sarved ja Sõrad}. Minu 
meelest just sedapidi, mitte Sõrad ja Sarved. Asus Rävala puiesteel seal, kus 
täna on NO-teater. Ja mina läksin selle aknalt seal Sakala 
tänavat\sidenote{Külgneb Rävala puiesteega} mööda ja nägin, et inimestel on 
arvuti, see oli aastal 1991 või midagi. Igal juhul ma veel ei töötanud 
Skriiningus\index{Skriining}. Aga arvutit tahtsin hirmsasti kasutada. Keskkool 
oli läbi, sinna enam sisse ei lastud ka no sõltlane käis mööda linna, eks ole. 
Järsku näed, akna taga arvuti. Ja marsid sama hooga sinna sisse täiesti 
tundmatusse firmasse, täiesti tundmatu värske keskkoolilõpetaja, et 
\enquote{teil on siin arvuti, ma tahaksin seda kasutada}. Ja ilma mingisuguse 
tänapäeval heaks kiidetud taustauuringu või millegita ütleb firma omanik sulle 
oma kirjutuslaua tagant \enquote{Jah, ta on meil siin küll, me tahaksime teda ka 
kasutada, loomulikult}. Ja ilma mingi töövestluse ja ilma mingisuguse sellise 
pikema jutuajamiseta antakse sulle kontorivõtmed, öeldakse, et \enquote{tee 
nii, et meie saaks seda arvutit kasutada, tee ta korda}. Ja sa avastad ennast  
arvuti tagant. Ilma et keegi oleks isegi su dokumenti vaadanud, et kas sa oled 
varas, või sa ei ole varastega või kas sa tahad terve firma ära varastada või 
ainult arvuti. See usaldus, mis tollel ajal valitses inimeste vastu, kes 
oskasid arvuti sisse lülitada ja sellega midagi teha, see oli \emph{enormous}. 
See oli selline, mida tänapäeval ei ole võimalik ette kujutada. See oli 
selline, et need värsked keskkoolilõpetajad, kes seal tulid, nendel 
põhimõtteliselt oli võimalik küsida ükskõik millise firma ükskõik millise  
arvuti võtmed, sest see kõik töötas. 

Noh, see lõppes muidugi sellega, et lõpuks tuli Imre Perli\index[ppl]{Perli, 
Imre}\sidenote{Imre Perli oli pehmelt öeldes raju elulooga Eesti 
arvutispetsialist, kes sai kuulsaks \enquote{Perli andmebaasi} koostajana. 
Kasutades ära ligipääsu mitmetele andmebaasidele, lõi ta üheksakümnendate keskel 
\enquote{superandmebaasi}, mis sisaldas isikustatud andmeid autode, (toona üsna 
haruldaste) mobiiltelefonide, aadresside jms. kohta. Andmebaas levis althõlma 
laialt. Imre Perli hukkus segastel asjaoludel 15. aprillil 2000 
politseioperatsiooni käigus} ja kopeeris ära kellelegi andmebaasid, eks iga 
aeg saab lõpuks otsa. 

\textbf{\enquote{Kuidas sul ikkagi see programmeerima õppimise protsess käis?}}

See on sihuke  kuidagi viimasel sajandil tekkinud paradigma, et 
programmeerimine on midagi, mida peab õppima ja see on midagi, millega tuleb 
nagu spetsiaalselt vaeva näha. Programmeerimine juhtub. Vajadusest. 
Programmeerimine juhtub tahtmisest. Keegi ei ole mulle mitte kunagi õpetanud 
ridagi Cd, keegi ei ole mulle kunagi õpetanud ridagi Assemblerit. 

\textbf{\enquote{Kuskilt saju ometi said teada, kuidas \texttt{malloc} käib?}}

Aga see on see tahtmine teha. No mina hakkasin õppima 
Pascalit\index{Keeled!Pascal}, sellepärast et see oli ainukene raamat, mis 
mulle kätte sattus. Seesama õudne Jürgensoni Pascali pruunide kaantega 
raamat\index{\enquote{Programmeerimine Pascal-keeles}}\sidenote{R. Jürgenson 
\enquote{Programmeerimine Pascal-keeles}, 1985. Legendaarne raamat, mis 
miskipärast huviliste hulgas laialt levis}, mis on ikka tagantjärgi vaadates 
päris õudne algus programmeerimisele. Aga sellega sai alustatud. Ja siis, kui 
Turbo Pascal hakkas ära tüütama, sellepärast et see tegelikult oli ka niisugune 
keel, milles midagi normaalset teha oli väga keeruline. Siis ühel hetkel ma 
leidsin, et ikka Assembler\index{Keeled!Assembler} on see päris asi. Kuna tol 
ajal oli popp kirjutada igasuguseid demosid ja häkkida kõiki tarkvarasid, mis 
kätte sattus, siis siis \ldots No küsi, kuidas õpetad inimesele nagu x86 
Assemblerit? No võtad raamatu ühte kätte ja AT86e teise kätte ja hakkad tegema.

\textbf{\enquote{Aga kust sa said selle raamatu? Neid ju ei liikunud?}}

Liiklus küll selle koha pealt tuleb anda tõenäoliselt varem või hiljem 
presidendi medal Tarmo Mamersile\index[ppl]{Mamers, 
Tarmo}\index[ppl]{MomraT|see{Mamers, Tarmo}}, kes tollel ajal 
TTÜ-s\index{Tallinna Tehnikaülikool}? Ma ei teagi, kes ta seal oli. No ta oli 
seal üks paljudest nendest, kes seda arvutiasja püsti hoidis esimesena. Aga 
Tarmo kaudu põhimõtteliselt kõik see asi liikus. Minu minu varane mentor oli 
raudselt Tarmo ja no oli seda tegelikult veel pikka aega ka peale seda, kui ma 
juba tegelikult tööl käisin. Kõik, see materjal käis käest kätte. Hiljem tuli 
juba FidoNet\index{FidoNet}. Kui ma oma esimese esimese FidoNeti \emph{point}i 
püsti panin, siis oli juba kõik palju lihtsam, sest siis oli nagu aken maailma 
olemas. \emph{Point}i püstipanemine käis ka loomulikult läbi TPI. Tarmo 
istus natuke eraldi, et Tarmo oli teises teises ruumis samal korrusel. Aga siis 
Aare Tali\index[ppl]{Tali, Aare} ja Tõnu Raimla\index[ppl]{Raimla, Tõnu} olid 
seal, kus käis elu nii-öelda.  Tarmo juures oli selline natuke rahulikum 
õhkkond. Seal käis igatahes kogu aeg \emph{action}. Ja siis mul oli nagu ühel 
hetkel kinnisidee, et ma tahan endale teha nüüd FidoNeti \emph{point}i, et 
ikkagi lõpuks olla maailma osa. Siis ma töötasin juba 
Skriiningus\index{Skriining}. Läksin siis Aare juurde, et \enquote{noh, Aare, 
sa oled siin \emph{sysop} ja värk} ja Aare talle omase abivalmidusega ütles, 
jah, masin on seal. Mille peale leidsin ma ennast BBS-i masina tagant ja pidin 
endale sinna valmistama FidoNeti \emph{node}. Ma kardan, et Tõnu või keegi 
lõpuks halastas mu peale ja näitas, kuidas seda päriselt teha. 

Aga siis edasi oli materjal juba palju kättesaadavam, siis said juba kõiki 
igasuguseid dokumente risti-rästi laadida alla. 

\textbf{\enquote{Mis sa sinna TPIsse õppima läksid?}}

Ma läksin LIsse. Ma arvan, et selle tolle aja nimi oli informaatika, äkki?. 
Kuna ma suhteliselt ruttu sain aru, et ma ei ole võimeline hommikul loengutes 
käima, siis mina ja üks väikene punt teisi, kes olid otsustanud, et nemad 
peavad õhtuõppes käima, läksime dekanaati ja nõudsime, et tarvis on õhtust 
vahetust sellele üritusele. Sest  õhtust vahetust tol hetkel konkreetsel alal 
ei olnud. Läksime kateedrisse, kateeder ütles, et jaaa, väga tore mõte. Aga 
meie kogemus ütleb, et kui te juba sihukese jutuga tulete, siis mitte keegi 
teist ei kavatse seal õhtuses ka käia. Mis tähendab, et me ei hakka teie jaoks  
eraldi rühma püsti panema. Käite ehitajatega esimese aasta koos koolis. Ja kui 
teisel aastal veel siin olete, siis vaatame seda asja. No kas nüüd osalt selle 
pärast või sellepärast, et dekanaadil oli õigus, nii või teisiti kukkusime 
sealt kõik robinal kolmanda kuu lõpuks välja ja läksime tööle igale poole. Nii 
et TPI on mul siiamaani lõpetamata. 

\textbf{\enquote{Sa mainisid, et sa kirjutasid mingit autobaasi softi. Kuidas 
sa seda tegema sattusid?}}

Siis ma juba töötasin. Me kõik läksime ju ka suhteliselt kiiresti ikkagi päris 
tööle. Tol ajal mingid startupi kultuuri ja ettevõtluse ehitamist veel ei 
eksisteerinud. Me ka lõpetasime sellisel ajal, kui need esimesi 
arvutikooperatiive oli siin väike käputäis. Minu esimene ametlik töökoht pidi 
olema tegelikult Noorsooteatri\index{Noorsooteater} valgustaja. Kuna mulle juba 
tollel ajal meeldis audioga tegeleda, siis ma tahtsin sinna helimeheks minna, 
aga helimees oli juba värskelt tööle võetud ja valgustaja koht oli vaba. Aga 
siis minu meelest päev või kaks enne seda, kui ma pidin lepingu alla kirjutama, 
tuli Tarmo Mamers\index[ppl]{Mamers, Tarmo} küsis, et kas ma ikka päris tööd ei 
taha teha, et Skriining\index{Skriining} otsib programmeerijat. 

Siis ma sattusin Skriiningusse Kalle Lotamõisa\index[ppl]{Lotamõis, Kalle} 
tööle. Minu esimene siis ülesanne oligi see, et autopark on sellel aadressil. 
Neil oli mingi eriti eksootilise asja peal jooksev andmebaasisüsteem, see ei 
olnud isegi \emph{mainframe}, see oli mingi mini. Ja  see oli vaja siis 
moodsale vahendile ümber kirjutada. Moodne vahend tähendas tol ajal siis Novell 
Netware'i\index{Novell} ja värskelt oli Paul Leis\index[ppl]{Leis, Paul} toonud 
Eestisse asja nimega Dataflex\index{Keeled!Dataflex}. Mis oli selline päris 
päris korralik objektorienteeritud kõrgkeel tol ajal. Ma hakkasin selle 
Dataflexiga siis ühest otsast õppima, kuidas Dataflexis programmeeritakse ja 
teisest otsast õppima, kuidas autopark töötab. 

\textbf{\enquote{Ahhaa, läksid kohe äriprotsessi ka sisse!}}

Äriprotsessid olid seal paljuski ees olemas selles mõttes, et töötav tarkvara 
oli olemas. Pigem oli seal äriprotsesside seisukohast hea lastetuba, et ära 
kunagi eelda midagi. Näiteks mina oma IT-inimese mõistusega tegin seal oma 
arust mõned asjad paremaks ja siis selgus, et päris nii ei saanud hea, nagu 
mina olin mõelnud. Sest raamatupidaja vaatasid mind nagu idiooti ja küsisid, et 
\enquote{sa ikka saad aru, palju ma neid numbreid pean siia päevast sisestama 
ja mitu korda ma seda enterit, mille sa siia vahele toppisid, peal vajutame 
lihtsalt niisama. Need arvud on neljakohalised. Ma sisestan neli numbrit ära, 
ta läheb ise järgmisele väljale, mitte ma ei pea vajutama. Ja eriti ma ei pea 
vajutama tabi, mis on teises klaviatuuri otsas. Saad aru, ma ühe käega kasutan 
pabereid teise käega vajutan klaviatuuri. Kuidas ma sinna tabi juurde sinu 
meelest saan, kui mul on teises käes paber?} 

Nad olid väga innovatiivsed seal tegelikult selles mõttes, et nad olid 
kasutanud sedasama andmetöötlust selleks ajaks juba aastat kuus-seitse. See oli 
 meditsiinitehnika autobaas, Termak\index{Termak}, siiamaani elu ja tervise 
juures. 


\textbf{\enquote{Nad siis juba Nõukogude ajal alustasid arvuti-asjandusega?}}

Nad olid juba sügaval nõukogude ajal end täiesti ära automatiseerinud. Selleks 
ajaks, kui mina aastal 92 sinna jõudsin, oli nendel juba esimene IT-süsteemi 
jõudnud kätte moraalselt nii ära vananeda, et see tuli PCde peale ümber 
kirjutanud. Neil oli aastal 1992 juba \emph{legacy}. Nad olid nii palju ajast 
ees.


\textbf{\enquote{Kuidas Skriining jõudis selleni, et neil on programmeerijat 
vaja? Lihtsalt kasti võis ka ju edukalt müüa?}}

Kalle\index[ppl]{Lotamõis, Kalle} hammustaski läbi selle, et kuna nad olid kogu 
aeg seal meditsiinitehnika ümber sebinud ja meditsiinisüsteemi neid arvuteid 
proovinud müüa, siis nad avastasid ühel hetkel, et seal on arendusvõimalused 
ka. Siis tegelikult Skriiningust\index{Skriining} saigi sealsamas 
üheksakümnendate alguses  arendusfirma. See arvutimüük käis ka, aga mina tema 
tollal noore inimesena väga ei süüvinud sellesse, kust raha tuleb. Aga mulle 
tundub, et see suht palju sellest tuli arendusest puhtalt.


\textbf{\enquote{Kas sa Tehnikaülikoolis ka veel ringi hängisid?}}

Ma hängisin seal pikalt aga ma kunagi õppinud seal. Ta oli ikkagi niisugune elu 
elu epitsenter, kuna seal töötasid kõik olulised inimesed. 
Mast\index[ppl]{Kaal, Madis} ülemisel korrusel Tõnu\index[ppl]{Raimla, Tõnu} ja 
Aare\index[ppl]{Tali, Aare} ja Tarmo\index[ppl]{Mamers, Tarmo} alumisel 
korrusel. Hiljem oli seal siis epitsenter siis, kui sinna läksid veel tööle 
Martin Rinne\index[ppl]{Rinne, Martin}, ja Merle Alliksoo\index[ppl]{Alliksoo, 
Merle} ja kõik teised, kes hiljem Microlinkis\index{Microlink} lõpetasid. Ta 
oli selline  sotsiaalse elu keskus. 

\textbf{\enquote{Mulle see variant, et sa ei õpi aga hängid, tundub palju 
mõnusam, kui see, et sa õpid aga ei hängi}}

Nojah, eks ma ise ikka soovitan inimestel reeglina, et  proovige oma kool kohe 
ära lõpetada, et pärast osutub see palju raskemaks. Nüüd mina ja mu ja sõbrad, 
kõik on sisuliselt neljakümnendates hakanud oma haridusega lõpuks tegelema. On 
tekkinud natuke rohkem vaba aega uuesti ja ka mingisugune moraalne vajadus, et 
kuidas sa oled kõige väiksemate pagunitega mees ruumis.

\textbf{\enquote{Tol ajal, kui tagasi mõelda, ülikool palju praktiliselt 
kasulikku ei andnud. Tänapäeval on teistmoodi}}

No nii nagu kõik ütlevad, et  ta oli selline ta ei olnud mitte tempel selle 
kohta, et sa tuled sealt välja targemana, vaid on tõestus selle kohta, et sa 
oled võimeline, järjepidevalt mitu aastat asjaga tegelema. Pigem ikka 
vastupidavuse ja hoolsuse proov kui koolitus.

\textbf{\enquote{Räägi palun BBSidest. Kuidas sa selle node ikkagi püsti said, 
selle jaoks oli vaja ju ennast kuskil registreerida?}}

BBS, kes ei tea, oli varane  arvutivõrk, mille mõte oli selles, et sa helistad 
kuhugi oma modemiga ja seal teises otsas on modem, kes sulle vastab. Nad saavad 
omavahel andmeühenduse püsti ja siis sa saad sellest teises arvutis, mille 
küljes modem oli, saad ringi sobrada. Kusjuures tõepoolest selles mõttes 
sobrad, et ega tollel ajal arvuti turvalisus oli selline noh kokkulepete 
küsimus. Ma arvan, et suvaline suvaline üks BBSi omanik oleks võinud teise 
BBSi omaniku BBSi lasta neljaks tükiks kaks korda tunnis ilma mingite 
probleemideta, aga seda lihtsalt ei tehtud. Sellepärast, et see oli nagu 
saarlase ukselukk. Et kui sa oled ta paika pannud väljapoole ukse ette, siis kõik 
teavad, et sind ei ole kodus ja nii on. Et ei ole vaja katsetada, et kas uks 
on lahti või kinni, kodus kedagi ei ole. Ja BBSidega oli  turva umbes sama. 
Nüüd BBSi teine ja tegelikult palju kasulikum omadusi oli see, et kui sul juba 
oli modem ja juba oli arvuti, siis sa said ennast FidoNeti\index{FidoNet} 
\emph{node}ks registreerida. Et BBS iseenesest ei eeldanud midagi sellest, lihtsalt 
modemi ja vastava tarkvara olemasolu. Kuskil mingeid hämaraid teid pidi levisid 
need telefoninumbrid, et kus see telefon on, kuhu helistada, seal kohapeal sa 
said ennast ära registreerida.

Nüüd FidoNet oli ikkagi juba esimene selline ülemaailmne arvutivõrk selles 
mõttes, et modemeid helistasid üksteisele automaatselt. Ja ta oli kaunikesti 
hästi toimiv,  tolle aja kohta elektronpostiteenus, mille üks  eriline omadus 
oli veel see, et ta kuidagi liikus väljaspool KGB huviala. Eks küll 
kahtlustati, et teda kuulatakse pealt ja aeg-ajalt mingid imelikud modellid 
üritasid sinu modemiga poole jutu pealt rääkida ka, aga üldiselt teda väga ei 
monitooritud vist. Mingisuguseid probleeme ma ei tea, et kellelgi oleks 
kaheksakümnendatel olnud nende modem-modemiga sidepidamisega ei Eesti ega 
välismaaga. Mis on selles mõttes eriti huvitav, et et kui läksid 
kaugekõneliinid nii palju lahti, et ta oli juba võimalik kuidagi automaatvalida 
kuhugile, siis ju me helistasime igale poole välja. Ja see 
FidoNeti\index{FidoNet} \emph{mail} oli tegelikult  esimene võiks öelda vaba 
demokraatlik sidekanal tegelikult väljapoole juba kaheksakümnendatel. See oli 
kaheksakümmend üheksa, kaheksakümmend kaheksa, umbes niimoodi ta Eestisse tuli.

Mina olin siis keskkoolis, esimese \emph{node} panin püsti hiljem, jah, aga 
noh, see süsteem ise töötas varem. Ma olingi vist Aare \emph{point}. 
\emph{Point}i numbrit ma enam ei mäleta, mis mul oli. \emph{Node} number mul 
lõpuks endal oli kolmkümmend viis, aga, aga kas \emph{pointi} number, mis oli 
kaksteist-kaksteist. Mis \emph{node} all, ka ei mäleta. Mina panin selle jah 
püsti minu meelest vist üheksakümmend üks. Aga siis oli ka veel see aeg, kus 
ikkagi veitsa oli see asi nagu hämar selles mõttes, et me ju veel päris 
vabariik ei olnud. Me olime selline üleminekuvabariik. Ja selline 
registreeritud postiaadress andis sulle  võimaluse mingite siis foorumites juba 
kaasa rääkida. Eestis endas oli kümmekond gruppi, kus käis jutt erinevatel 
teemadel. Ja selles mõttes ta oli ikkagi päris selline elu, nagu me täna oleme 
harjunud, kuigi natuke teistsuguste tehniliste vahenditega. Ta oli aeglasem ja 
ta ei olnud reaalajas selles suhtes, et see post saabus sulle paar korda 
päevas. Ta ikkagi ei olnud selline, et kirjutan oma kirja ja see läheb kohe 
kõigile laiali. Aga ta täitis kõik need ülesanded, millega me täna tegeleme 
ära, nii et tegelikult  kaheksakümmendate lõpus, üheksakümnendate alguses see 
ökosüsteem, mida me täna oleme harjunud nägema, oli tegelikult täiesti olemas. 
Ja väike käputäis inimesi Eestis omasid seda privileegi, et seda kasutada. 

\textbf{\enquote{Kas see väike käputäis olid pigem entusiastid, akadeemilise 
seltskonna inimesed või kes?}}

Need olid ikka sada protsenti entusiastid. Ma arvan, et akadeemilised inimesed 
sellel hetkel, kes läks ärisse, üritas sellest raha teha ja panid püsti 
esimesed arvutifirmad kes oli lihtsalt tegevuses  ellujäämisega, kes õpetas 
seda, mida ta kogu aeg õpetanud oli. See ökosüsteem minu meelest koosnes sada 
protsenti entusiastidest.

\textbf{\enquote{Kas eksisteeris mingi spetsialiseerumine ka, et siit ma saan 
tarkvara ja seal on huvitavaid jutte, seal raamatuid?}}

BBSidel väike spetsialiseerumine oli. Aga ma arvan, et mitte eriti suur. Eks 
kõik enam-vähem proovisid korjata, mida nad vähegi endale seal kõhu alla said. 
See oli see aeg, kus juba esimesed need sellised suuremad kõvakettad tekkisid. 
Mis tähendas seda, et tegelikult lühikest aega valitses olukord, kus tarkvara 
oli vähem kui ruumi. Ruumi mõiste oli ka muidugi tollel ajal huvitav. Kõige 
rohkem ruumi võtsid Sierra\index{Sierra Entertainment}\sidenote{1979. aastal 
asutatud Sierra Entertainment (varasemalt On-Line Systems ja Sierra On-Line) 
oli vastutav paljude toonaste hitt-mängude eest. Eriti populaarsed olid nende 
seiklusmängude sarjad \emph{King's Quest}, \emph{Space Quest} ja \emph{Leisure 
Suit Larry}} mängud, mis olid flopiketaste peal. Neist suuremad, Space 
Questid\index{Mängud!Space Quest} ja muud, tulid mingisuguse viie-kuue flopi 
kaupa. Mäletan, et me istusime Eeroga\index[ppl]{Tohver, Eero} ja arutasime, et 
kui oleks võimalik panna kokku oma unelmate masinat siis kui suur kõvaketas 
peaks tal olema. Jõudsime sinna, et kui oleks umbes kaheksakümmend megabaiti, 
siis ilmselt jätkuks eluajaks, sinna saaks kõik mängud peale panna, kõik 
tööasjad ka ja jääks veel umbes pool jääks üle.

\textbf{\enquote{Sierra oli omaette fenomen, tema asju mängiti ikka palju. Kas 
neid müüs ka keegi?}}

No küsime siis laiemalt, kas Eestis üldse keegi tarkvara müüs tollel ajal. 
Äritarkvara, nagu Novell, oli võimalik osta. Eks teoreetiliselt oli kusagilt 
Windowsi või DESQview'd\index{DESQview}\sidenote{DESQview oli kaheksakümnendate 
lõpus ja üheksakümnendate algul populaarne tekstipõhine mitmetegumiline 
keskkond. Ta käis DOSi peal ja võimaldas korraga mitut programmi eri akendes 
käimas hoida}  kindlasti võimalik osta. Aga peale Novelli serveri ja DataFlexi 
litsentside, ma ei mäleta, et me oleks üheksakümnendatel näinud mingisugust 
legaalset tarkvara kellelgi. 


\textbf{\enquote{Tuleme tagasi selle BBSinduse juurde. Kas selle sisu hulk, 
mida enda kõhu alla õnnestus kokku kuhjata, oli ka mingit pidi staatuse 
sümboliks?}}

Ma ei oska öelda. Ma oskan ainult enda BBSide kohta nagu rääkida. Mina 
sisuliselt korjasin kokku kõik, mida ma kätte sain ja pakendasin ringi. See 
oli selline kultuuriküsimus, et sa skaneerisid selle tarkvara viiruste vastu 
kõige värskema viiruste skanneriga, mis sul parasjagu käeulatuses oli. See käis 
automaatselt muidugi. Siis sa lisasid sinna mingi väikese faili sinna arhiivi, 
mis sisaldas mingit sinu \emph{header}it. See oli siis niisugune väike 
failijupp, kus oli graafiliselt või siis tollel ajal pseudograafilised, sinu 
logo sisse punnitatud. Ja siis sa panid ta välja ja panid oma faili listi siis 
mingisuguse sellise nupukese, mis asi ta on. 

See oli nagu \emph{basic housekeeping}. Et kui see sinu fail läks mingisse 
järgmisse BBSi siis järgmine BBS viskas sinul logo välja ja pani enda oma 
endale  asemele. Et nagu \emph{tag}iti ära nagu \emph{graffiti}ga, et see on 
minu käest tulnud asi. Ja minul vähemalt on küll tunne selline, et välja läks 
kõik, mida sa ise olid endale mingil põhjusel hankinud. Et see ei olnud nüüd 
nii, et sa läksid ja tõmbasid öösel mingisuguse HNSi\index{BBS!HNS} tühjaks ja 
panid enda lehekülje peale välja. Aga mingid asjad, mida sina olid kätte 
saanud, sa panid üles. Neid duplikaate ei olnud väga palju üllataval kombel.

\textbf{\enquote{Ma tahtsingi küsida, et nii oleks pidanud üks hetk ju kõigil 
kõik asjad olemas olema, seda siis ei tekkinud?}}

Seda ei tekkinud, sest kuna need BBSid olid väga stabiilset üleval, siis 
tõmbasid ära mingeid asju, mida sina pidasid enda jaoks vajalikuks ja panid nad 
siis ka omakorda enda juurde üles. Aga sellist mõttetut \emph{leach}imist  väga 
palju ei olnud. Püüet iga hinna eest oma faili andmebaas kõige suuremaks saada, 
ma ei mäleta, et seda oleks nagu eraldi eesmärgina keegi järginud. 

\textbf{\enquote{Too mõni näide, mis laadi asjad sulle toona huvi pakkusid?}}

No mina olin juba tollel ajal vihane \emph{nerd}, minu jaoks igasugu 
programmeerimismaterjalid ja igasugused käsiraamatud ja igasugused 
tööriistakesed ja  programmeerimisvahendid ja need olid minu spetsialiteet. 
Kahjuks mul ei ole seda vana faililisti alles, sest kui ma Skriiningust ära 
läksin, siis suhteliselt lühikese aja peale lendas see vana SCSI ketas õhku, 
mille peal see BBS jooksis ja sellest ei olnud \emph{backup}i ja sinna ta jäi. 
Läks kogu FidoNeti \emph{node} koos failibaasiga hingusele.

Ma ise teda järgmisse kohta kaasa ei võtnud, sest ma läksin Skriiningust panka 
ja seal olid kõvad mehed nagu Mast\index[ppl]{Mast} ja  Marx\index[ppl]{Marx|see{Kliimask, Margus}}\index[ppl]{Kliimask, Margus}  ees, kes olid oma ökosüsteemi püsti pannud, siis ühele BBSile seal rohkem ruumi ei olnud. 

\textbf{\enquote{Mis panka sa läksid?}}

Mina läksin sellesse panka, mille lõpupidu nüüd siin kohe nädala-paari pärast 
kätte jõuab\sidenote{Intervjuu Andrusega toimus 2019. aasta novembri algul} 
novembris, mis lõpetas siis Danske\index{Danske Pank}\index{Danske 
Pank|see{Forekspank}} nime all aga alustas Forekspangana\index{Forekspank|see{Eesti Forekspank}}. See 
oli jälle omaette selline innovatiivne pangandustoode. 

\textbf{\enquote{See oli väga äge pank omal ajal. Miks sa sinna läksid? 
Skriiningus said programmi kirjutada ja BBSi pidada ju?}}

Nagu ma paljudesse kohtadesse oleme läinud, läksin sellepärast, et kutsuti. Ja 
teiseks, kuna parasjagu Eestis jooksis teleseriaal \emph{Capital City}, mis 
näitas panganduse elu väga glamuurse \emph{highroller}ina, siis mulle tundus, 
et mina tahan ka nii elada. Tuleb tunnistada, et üheksakümnendate panganduses  
väga ei pidanud pettuma, elu oli täitsa täitsa lill. Ütleme nii, et nii nagu 
selles eesti teleseriaalis Pank päris elu ikkagi meie majas vähemalt ei käinud. 
Päris hulle pidusid sai peetud, aga seda, et keegi oleks kuskile kokaiinise  
ninaga ringi käinud, seda mina ei tea. Meie meie kandis  kokaiin oli täiesti ma 
ei tea, kas tundmatu või seda tehti salaja või midagi, aga igal juhul ma neid 
narkootikumidega pidusid ei tea. Aga pidutsetud sai hästi.

\textbf{\enquote{Aga kas ma õigesti mäletan, et tollal te panka tõmbasite ikka 
püsiühenduse\sidenote{Enamik varasest internetiühendusest Eestis käis kuhugi 
sisse helistades. See tähendas, et pidev side puudus ja muul ajal sõltus side 
kvaliteet suuresti analoogtehnoloogial põhinevatest telefonikeskjaamadest. 
Püsiühenduseks kutsuti seda, kui asutusest füüsiline kaabel Interneti külge 
jooksis ja selle olemasolu oli IT-inimeste unelmates kesksel kohal} sisse?}}

Püsiühenduse tõmbasime me sisse väga konkreetsel päeval. Ühesõnaga, meil 
modemitega see nii-öelda poolpüsiühendus oli juba pikemat aega olemas. Kuna 
Forekspank asus Rävala puiesteel ja Rävala puiesteel asus seal suhteliselt 
lähedal ka juhtumisi KBFI\index{KBFI}\sidenote{Keemilise ja Bioloogilise 
Füüsika Instituut (KBFI). 1979. aastal Endel Lippmaa\index[ppl]{Lippmaa, Endel} 
poolt loodud teadusasutus. Tuntud ka kui \enquote{Lippmaa Instituut}. Just 
Lippmaade perekonna aktiivse ning laiahaardelise tegutsemise tõttu mängis 
instituut paljudes toonastes olulistes protsessides (sealhulgas kohaliku 
Interneti arengus) olulist rolli.}. Baumaniga\index[ppl]{Bauman, Andres} ning 
oli läbi räägitud, et kuidas seda Internetti saab ja meil olid  suhteliselt 
rivitu ligipääs. Aga mingil hetkel tundus, et see asi võiks ikka päris 
permanentne olla ja siis me võtsime Mastiga\index[ppl]{Mast} kaablirulli ja 
hakkasime siis üle Rävala puiestee katuste KBFI poole liikuma. Mis oli selle 
juures tähelepanuväärne oli see, et see juhtus päeval, mil esimest korda paavst 
Eestit väisas. Kõik katused olid snaipreid täis, oli mingi tohutu 
\emph{lockdown} et keegi paavsti siin käigu pealt ära ei tapaks. Meie sellel 
samal päeval, kui paavst sõitis ringi oma mobiiliga, käisime kaablikeraga. 
Seletasime kõigile, et meil on vaja kaablit vedada, meie paneme Interneti 
püsiühendust. Ja see oli selline maagiline valem, mis  võimaldas ligipääsu 
kõikidele kesklinna katustele, ilma et keegi oleks küsinud sult midagi 
rohkemat. Me otseselt snaiperitega samale katusele ei sattunud, aga üldiselt 
jah, midagi ei küsitud ka. Natukene oli seal vist vaja mingit häkkimist ka, et 
mingist koodlukust pidime vist ikkagi läbi minema kogemata. Aga see oli tollel 
ajal mehaaniline ja seda tehti nuppude kulumise järgi, et see ei olnud kõige 
suurem takistus. 

\textbf{Millest ma siis järeldan, et toona oli maailm teistsugune. Internet ei 
olnud veel kommertsiaalne vaid pigem kogukondlik nähtus?}

Selle eest vististi ikka keegi maksis ka kellelegi lõpuks midagi, aga palju, 
seda ma jällegi ei mäleta. Eks ta ju paljuski käis inimsuhete baasil ikkagi. Et 
kuna me Andres Baumanni\index[ppl]{Bauman, Andres} tundsime siis kuidas see 
sealt tegelikult  liikus ega mina ka ei tea. Mast\index[ppl]{Mast} seda asja 
ajas. Millegi pärast ma arvan, et  me maksime KBFIle mingit mingit raha ka 
selle eest. Jällegi. Aasta oli siis minu teada üheksakümmend viis ja 
üheksakümne viiendast aastast alates tegelikult oli meil 
Forekspangas\index{Forekspank} elu, nagu me seda täna tegelikult näeme. 
Suhteliselt samal ajal tuli Mosaic'i\index{Mosaic}\sidenote{NCSA Mosaic oli üks 
esimesi internetibrausereid ja mängis WWW populariseerimisel olulist rolli. 
Sama meeskond lõi hiljem Netscape\index{Netscape} Navigatori, mis omakorda on 
Firefoxi eelkäijaks} brauser, suhteliselt samal ajal hakkas veeb arenema, 
suhteliselt samal ajal tekkisid meile kõigile e-maili aadressid (mis olid 
natuke küll varem juba KBFI kaudu olnud korraks, aga siis tekkisid nad päris 
meie oma foreks.ee domeeni külge). Kõik see ökosüsteem, miinus Facebook, olid 
üheksakümne viiendal aastal tegelikult meil käes. Sealt edasi,  tegelikult me 
elasime täpselt sellist elu nagu inimesed infotehnoloogiliselt täna ette 
kujutavad. 

See elu oli natuke tillukesem selles suhtes, et me täitsa tõsimeeli arutasime, 
et see ühendus, mis meil KBFIsse on, on ikkagi nii aeglane, et äkki peaks kogu 
veebi kohalikku serverisse ära kopeerima. Siis me isegi vahepeal arutasime, et 
kuna see mahub ühele DVD-le tõenäoliselt ka ära, ehk siis äkki peaks tegemiseks 
äri, et hakkaks müüma Internetiga DVD-d, et siis oleks saanud kohalikest 
\emph{cache}dest tõmmata. Kogu see WWW oli tollel hetkel selline, tõsimeeli sai 
arutatud, et paneks ta ühele DVD-le ära. 

\textbf{Ka teistest lugudest käib läbi, et toonane maailm käis väga suuresti 
inimsuhete peal. Aga ometi inimesed ei hakka arvutitega tegelema, kuna neile 
meeldib inimestega tegelda. Siiski tunduvad Eesti arvuti-inimesed olema küllalt 
suhte-altid ja neis osavad. Kuidas see nii on?}

Ma arvan, et see on sellepärast, et kuna sul on sellised huvid, siis sa oled 
terve keskkooli ja pool ülikooliaega olnud sotsiopaat ja kellegagi väga ei ole 
sul rääkida millestki olnud. Ja nüüd sa ühel hetkel leiad omasugused, 
omasuguste huvidega. Täitsa puhas \emph{nerd}i ja nohiku käitumine, eks ole, et 
kui sa paned  nohikud  kõik ühte tuppa kinni, siis nüüd ühel hetkel leiavad 
üksteist ja siis on kõigil järsku lõbus sest kõik naeravad samade naljade üle 
lõpuks ometi. Ja peod täpselt samamoodi, eks. Ega kõige karmimat peod, kus ma 
olen  osalenud on ikkagi olnud inimestel, kelle igapäevatöö on kaunikesti 
\emph{boring}. Tahtmata anda hinnangut mingisugustele inimgruppidele, aga   
raamatupidajad ja  andmesisestajad, kui nad ikka käima lähevad, siis see on 
ikka ikkagi täiesti teine tase. Et siis lõbusad inimesed on keskmiselt lõbusad 
kogu aeg. Aga kui sellised nohkarid lõpuks lõbusaks lähevad, siis juhtub asju.

See ökosüsteem toimis tänu sellele, et inimestel oli hea meel üksteist leida. 
Ta oli tollal tõesti väike ka, ta oli ikkagi alla saja inimese kindlasti,  
võib-olla isegi alla viiekümne inimese alguses. Need olid siis sellise uue 
laine  arvutitegelased, kust hästi palju meie tänast  start-up ettevõtlust 
tegelikult ju välja on kasvanud. See kamp juba tollel ajal oli väga tihedalt 
koos ja väga õnnelik üksteise leidmise üle. Ja tänu sellele ju hakkasid siis 
toimuma legendaarsed BBSummeri\index{BBSummer} nimelised üritused. 

\textbf{Räägime lõpetuseks sellest ka, et mis sa praegu teed?}

Ma ei tea, see on võib-olla masendav tõdemusega, ega ta pole mind väga palju 
sellest nagu kaugemale ega kuhugi mujale viinud. Ikka väga laias laastus 
tegelen täna täpselt sama asjaga, millega ma tegelesin kakskümmend viis aastat 
tagasi. Olen pendeldanud elektroonika ja tarkvara vahel siin edasi-tagasi ja 
olnud mitme firma CTO ja  asutanud firmasid ja neid kihva keeranud ja töötanud 
teiste juures ja töötanud endale. Ja kui keegi küsib, et millega sa tegeled, 
siis ma tavaliselt ütlen, et ma annan masinatele hinge. 


\textbf{See on ilus ütlemine ja läheb kokku küsimusega, mis mul enne jäi 
küsimata. Tavaliselt inimesed tegelevad kas riist- või tarkvaraga aga sinul 
tundub olevat üks jalg ühes ja teine teises?}

Vaadates oma elu, siis ma muidugi tahaks, et tarkvara oleks mu tõmmanud endasse 
sellepärast, et see on nii palju  lihtsam ala, mõnes mõttes. Vigu on palju 
lihtsam parandada ja asju ära visata peaaegu üldse ei tule, mis katki lähevad. 
Kettaruum ei maksa täna eriti palju erinevalt elektroonika valmistamisest ja
utiliseerimisest.

Mul on kuidagi juhtunud niimoodi, et mul on see reaalne maailm, et kui ma panen 
tule vilkuma ja näen, kuidas mu tehtu  manifesteerub päris päris asjades, et 
siis mul kuidagi läheb tuju paremaks. Kuna mul see elektroonika disaini puhul 
tundub, et tuleb ka välja, samal ajal ma ikkagi taustalt oled nagu 
programmeerija, siis ma olen nagu sattunud sinna sidemeheks. Ma suudan tõlkida 
riistvara tarkvara jaoks ja vastupidi. Selle kõige konkreetne töönimetus on 
\emph{embedded engineering}. Mis on, tundub täna, vaadates, mis meil koolidest 
saabub, siis täiesti väljasurev kunst. Neid tegelasi, kes suudavad nii 
riistvara valmistada kui sellele tarkvara peale kirjutada, neid üritatakse 
nimetada mehhatroonikuteks või kelleks iganes, aga aga fakt on see, et nende 
juurdekasv on järsult pidurdunud ja see varem või hiljem hakkab meil 
probleemiks muutuma. Tõsi küll, töömeetodid ka muutuvad. Me kasutame täna
võibolla töövahendid, mis iseenesest annavad näiteks tarkvaratiimile parema 
ettekujutuse riistvarast kui see vanasti oli. Kirjeldused ja mingisugused 
\emph{markup language}d, millega  seda tehakse, on paremad. See, mis ma enne ka 
ütlesin oma tööd kirjeldades, et ma annan masinatele hinge, et see on 
tegelikult see osa, et kui sa lülitad oma pesumasina sisse, siis mida ta sinu 
heaks teha oskab või ei oska. Kui hästi see  raua ja tarkvara vaheline kooslus 
on välja mõeldud, sellest tuleb ka see kasutajakogemus. 

\textbf{Sa ütlesid enne, et sa oled ka CTOna toimetanud. See tähendab ju, et 
kolmas element juurde, sa pead selle kõik suutma ka äriks tõlkida?}

No vot seda CTO ametit on kaht sorti. Tavaliselt väikestes firmades tähendab 
CTO olek seda, et koosolekule on vaja kedagi kaasa võtta ja kuidas sa võtad 
kaasa ja ütled, et on mul programmeerija, eks. Sa pead talle lihtsalt andma 
visiitkaardi, millega ta näeb välja presentaabel. Tihti väikese firma CTO 
tähendabki lihtsalt seda, et sa teedki kõike, millel on tehnikane maitse 
küljes. Suurema firma CTO tähendab seda, et sa oledki see\ldots. Täna on 
startupi maailmas \emph{customer fit} ja \emph{market fit} hästi kõva teema. 
Kui vanasti väga ei tegeletud sellega, siis nüüd, kus on tohutu kuhi 
investorite raha põlema pandud, ilma et sellest isegi sooja oleks saadud, et 
siis kõik on hakanud rääkima sellest, et su toodetut kellelegi nagu päriselt
tarvis peaks ka olema. See tundub olevat mingi uuem asi, viimase paari aasta 
paradigma. Juba mingi kaks-kolm aastat tagasi hakkas Silicon Valley poole pealt 
pihta see kultuur, et laste kätte ei taheta raha enam hästi anda. Ehk siis 
nende kaheksateistaastaste ime-ettevõtjate, kes suudavad  väga suure kuhja raha 
korraga põlema panna, nii et sooja ei saa, aeg sai seal mõned aastad 
tagasi läbi. Nüüd on siis  selgunud uus innovatiivne lähenemine, et toodet peab 
kellelegi tarvis ka olema, teine selline paradigma muutus. Mis tähendab muidugi 
seda, et erakordselt raske on olnud hakata raha saama projektidele, sest kõik 
on hirmus järsku pirtsakas muutunud ja nõudnud, et kust raha tagasi tuleb. 

\textbf{See jutt läheb ju kokku sinu kunagise ettevõtte uksest sisse minekuga: 
seal sa pidid ju ka kohe hakkama kasulik olema ja ei tohtinud asju tuksi 
keerata}

Selle kasumlikkus, see on tegelikult õudselt valus teema. Riistvaraga on see 
asi  selgem selles mõttes, et riistvara ei eskaleeru kui keegi teda ei osta. Sa ei 
saa valmistada sedasama \emph{recorderit}, millega me siin praegu salvestame, 
miljon tükki, kui keegi seda ei osta, sellepärast et sa lähed lihtsalt 
pankrotti. Tarkvara tiražeerimine ei maksa midagi. Ja täpselt samamoodi võib ju 
juhtuda, et tarkvara, millest mitte kellelegi mitte pennigi raha ei teki, on 
tegelikult väga kasulik. Ehk siis kasulikkus ja ärimudel ei tähenda veel mitte 
midagi omavahel. Ja mis  dotkommi mullidega tavaliselt kipub juhtuma ja 
igasuguste tarkusemullidega, on see, et piir selle vahel, kus asi ei teeni 
raha, sellepärast et ta on väga hea mõte, mida veel ei ole õpitud raha panna 
teenima ja nende asjade vahel, mis ongi täiesti mõttetud, on väga raske tõmmata. 
Seetõttu on väga palju tegelasi, kes suudavad maha müüa  täiesti kasutu idee 
öeldes, et see ongi enne monetariseerimist faas ja see peagi midagi tootma. 
Unustades ära selle, et see on ka ühtlasi täielik kräpp, eks ole. Siin  
viimasel ajal on tekkinud paar niisugust suuremat skandaali, üks neist on 
muidugi see õnnetu Theranose \emph{case}, kus sa suudad nii veenvalt endale 
valetada. Et sul ongi ehitatud üles terve ökosüsteem väga kasulikkudest 
asjadest, mille ainus viga on see, et see fundamentaalne eeldus, millele ta 
rajatud oli, oli täiesti vale. 

\textbf{Tundub, et selle kahekümne viie aastaga maailm väga teistsuguseks 
saanud ei ole aga siiski natuke toimib teisti?}

Üks asi on oluliselt erinev. Tollel ajal tarkvara valmistati kahel põhjusel. 
Üks oli see, et teda oli tarvis, mis tähendas, et  oli tugev kliendipoolne 
tõmme. Ja teine oli see, et ma tahtsin, et midagi sellist eksisteeriks 
maailmas, mis tähendab, et ma lihtsalt võtsin kätte ja kirjutasin ta kas enda 
või teiste rõõmuks. Ja lasin ta lihtsalt maailma. Hästi palju mingeid väikesi 
utiliite, mis midagi kasulikku tegid, olid ju tegelikult kirjutatud kellelgi 
enda jaoks ära, pakendatud ja saadetud laiali. Eestis seda, et tarkvaraga 
õnnestuks mingit raha teha, et mina kirjutan mingisuguse vidinaga ja keegi 
maksab selle eest, seda kontseptsiooni polnud olemas. \emph{Corporate} maailmas 
küll seal igasuguseid  raamatupidamissüsteeme osteti-müüdi juba tol ajal väga 
edukalt ja see kõik töötas. Mujal maailmas tegeleti mingisuguste utiliitide 
pealt raha teenimisega ka väikesel viisil. Aga Eestis üldse mitte. Nüüd 
tänapäeval eks ole, see tarkvara tootmine on läinud niimoodi, et mul tuleb mingi 
ilgelt hea idee. Ja ma tahan sellest nüüd teha raha tootmise masina, mis 
tähendab, et sa teed nagu teistpidi. Et see ei ole mitte nii-öelda 
vajaduspõhine vaid selline unistus-põhine. Et mina tahaksin, nagu me siin 
aeg-ajalt  Ivar Sarantsiga(Kas on õige nimi?) naerame, et tänapäeva 
maailmas  inimesed  otsivad probleeme neid vajavatele lahendustele. Et kui 
vanasti otsiti probleemidele lahendust, siis nüüd otsitakse vastupidi ja see on 
 kõige suurem paradigma muutus selle kahekümne viie aasta jooksul.


\chapter{Sergei Anikin}
%!TEX TS-program = arara
% arara: myindex

\index[ppl]{Anikin, Sergei}
\question{Kuidas sina arvutite juurde sattusid?}

See oli päris huvitav lugu. Ma olin nii-öelda \emph{entitled}, mu isa oli
elektroonikainsener ja töötas Kalinini tehases\index{Kalinini
tehas}\sidenote{Algselt Balti Raudtee Peatehased, mis ehitati 1870. aastal ja
kandis aastatel 1902–1903 seal töötanud Nõukogude riigitegelase järgi 1940.
aastast alates M. I. Kalinini nime. 2007. aastast asub sellel
territooriumil ja osalt samades hoonetes Telliskivi Loomelinnak restoranide,
kohvikute, kontorite ja loomeruumidega}. Nüüd on seal kõige popim koht noorte seas, seesama Kalamaja ja Lendav Taldrik.
Lapsena käisin koos isaga tehases. Isa projekteeris rongidele
elektrimootoreid ja jõuelektroonikat. 
Hobi korras on ta teinud igasugust raadiotehnikat ja ma ise olen proovinud
väikest raadiot kokku panna, kuigi olin täielik võhik. Käisin küll raadiotehnikaringis. Minu esimese arvuti aga pani kokku isa.

\question{Kust ta vajalikud jupid sai?}

Isal oli selline Vene ajakiri nagu 
\begin{russian}Радио\end{russian}\index{Radio}\sidenote{Igakuine populaarteaduslik 
raadiotehnika ajakiri, mida andsid välja Nõukogude Liidu Siseministeerium ja 
DOSAAF (\begin{russian}Добровольное общество содействия армии, авиации и флоту 
России\end{russian} – vabatahtlik Vene armee, lennunduse ja mereväe 
abistamise selts). Ilmus eri nimede all alates 1925. aastast, 1975. aastal oli 
ajakirja tiraaž 850 000 eksemplari.}. Aastal 1986 avaldati seal 
kõigepealt arvutiskeemid ja siis kokkupanemise juhend. See oli 
Nõukogudemaal välja töötatud arvuti, aga skeemid võtsid nad ZX Spectrumi 
pealt\sidenote{\begin{russian}Радио-86РК\end{russian}\index{Radio-86RK}, populaarne
Nõukogude liidus loodud koduarvuti. Kuigi Nõukogudemaal 
kopeeriti ZX Spectrumit usinasti, oli see arvuti siiski väidetavasti 
originaalse disainiga, autoriteks Dmitri Gorškov, Juri Ozerov, Gennadi 
Zelenko ja Sergei Popov.}. Isa korjas komponendid kokku, 
joonistas ise plaadi, tegi tehases plaadi valmis ja 
pani arvuti kokku. Mäletan, et tal läks paar kuud, enne kui kõik 
vigased kohad ostsilloskoobiga välja juuris. Siis pani ta selle 
teleka külge, mis asendas monitori. See oli mustvalge telekas, 
värvitelekat meil ei olnud. Ega ma selle arvutiga midagi väga teha ei 
saanud, sel ei olnud isegi opsüsteemi. Oli küll \emph{interface} 
kassettmakiga, aga meil ei olnud kassette, mille pealt 
opsüsteemi laadida. Sellessamas ajakirjas oli trükitud baitkoodis opsüsteemi kood – 
kakskümmend lehekülge bait baidi haaval. Istusin kaks nädalat pimedatel talveõhtutel arvuti taga 
ja trükkisin kõik need koodid sisse.

\question{Miks sa seda tegid? Normaalne laps ju ei toksi niimoodi 
pimedatel õhtutel baitkoodi?}

Ka sellel on eellugu. Isa sõber tõi mulle umbes aasta varem lasteraamatu, kus tegelased õppisid programmeerima 
BASICus\index{BASIC}. Lugesin raamatu läbi, sain aru, kuidas 
programmi kirjutada, ja kirjutasin BASICus umbes kümnerealise programmi, mis 
midagi arvutas. Kuna aga paberi peal ei saanud ju
kompileerida, siis näitasin seda isa sõbrale, kes kontrollis ja  
ja ütles, et töötab küll.

Koodide sissetoksimine käis plokkide kaupa. Seal oli 
umbes poole leheküljeline plokk, millel oli kontrollkood. Sain seda 
valideerida ja kui see klappis, siis salvestasin makile. 
Kui ei klappinud, siis pidin viga otsima, mis oli väga 
keeruline. Ilmselt sellest ajast tekkis mul esiteks 
kannatus ja teiseks tähelepanu detailidele. Koode sisestades sain
lõpuks aru, et hästi oluline on need õigesti ja õiges 
järjekorras sisse toksida, sest ümbertegemine oli nii piinlik.


\question{Sulle tehti väiksest peale selgeks, et võid küll üle jala 
lasta, aga siis toksid ise neidsamu asju kolm korda.}

Jah, aga enamiku ajast veetsin loomulikult arvutiga mängides. Tollal 
oli olemas tavapärane maomäng ja ka tennis. Isale meeldis arvuteid kokku panna, sealhulgas sedasama 
ZX Spectrumi\index{ZX Spectrum}. Tegelesime ka 
selle väliskorpusega. Eestis on ju talviti kuiv õhk ja 
meil olid siis plastist õhuniisutajad, mis käisid radiaatori peale. Sellest sai väga 
hea korpuse arvutile: see oli õige kujuga ja selle sisse sai lõigata 
klaveri, toiteploki, plaadi ja kõik muu vajaliku. Makk oli eraldi.

\question{Miks sulle elektroonikaosa huvi ei pakkunud?}

Mul ei olnudki tegelikult arvuti vastu suurt kirge, siiamaani ei ole. Minu arust on see ikkagi vaid vahend. Tänapäeval on ju
teada, et arvutitega tegelejad teenivad päris korralikult raha. Tol ajal oli see ka mõnes mõttes staatuseküsimus, kui peres oli
arvuti. Kui paljudes peredes
üldse oli? Alles aastaid hiljem tekkisid arvutiklubid või
arvutimängukohad. Aga mul oli kodus olemas, kuigi me ei olnud
jõukas pere, kel oli raha niisugust asja osta. 

Arvuti on jah pigem vahend, ka meeldiv hobi, aga mitte ainuke. Mõnda aega ei tegelenud ma üldse arvutitega, 
mängimine enam kirge ei tekitanud ja programmeerida lihtsalt enda jaoks ei
tundunud väga huvitav. Aga mul oli üks sõber, kellega koos me mängisime. Tema
mainis: \enquote{Hoo, ma käin nüüd arvutiklubis. Me õpime seal programmeerima,
kuigi mina käin muidugi enamasti mängimas}. Siis ma mõtlesin, et tema ju
tegelikult ei oskagi midagi, aga mina küll, ja et peaksin koos temaga minema. Sa
ilmselt oled rääkinud paljude inimestega Eesti kogukonnast, aga mina sattusin
siis Vene kogukonda. Selle arvutiklubi nimi oli Interface\index{Interface}.

\question{Kes seda klubi pidas ja kus?}

Seda vedas Nina Botina\index[ppl]{Botina, Nina}, kes töötas vist bioloogiainstituudis Mustamäe teel. Me käisime 
Reaalkoolis\index{Tallinna 2. Keskkool}\index{Reaalkool|see{Tallinna 2. Keskkool}} tundides,
seal olid arvutiklassid.

\question{Mis koolis sa ise käisid?}

Koolis number kakskümmend kuus\index{Tallinna 26. Keskkool}. 
Viimasesse klassi läksin Tõnismäe Reaalkoolis\index{Tõnismäe Reaalkool} 
kus oli väga tugev matemaatika. Tegelikult seesama Nina Botina õhutas mind ja 
veel ühte klassiõde teise kooli minema ja 
matemaatikaklassi lõpetama. Tema pärast läksimegi sinna, seal oli hästi palju  
tuttavaid arvutiklubist.

Hiljem kasvas sellest arvutiklubist venekeelne tehnikakool või
arvutitehnikakool, mis asus Erika tänaval. 

\question{Ma teadsin, et Tartu ja Tallinna vahel on erinevus. Aga 
selgub, et ka Tallinna sees on kaks täiesti isesugust Tallinna.}

See on huvitav jah. Kusjuures minu huvi arvutite vastu 
vaheldus. Ühe aasta olin klubis, aga uude kooli minnes ei olnud mul 
selleks aega. Siis kutsus Nina mind appi, arvutiklassi instruktoriks, ja see 
tekitas uuesti huvi. Kui ma lõpetasin kooli ja läksin
ülikooli majandust õppima\index{Tallinna 
Tehnikaülikool!Majandusteaduskond}, tekkis seal esimese aasta lõpus 
võimalus spetsialiseeruda majanduslikule andmetöötlusele. Meil oli pisike 
grupp, seitse inimest. Kui kõik, kes olid majanduses, õppisid majandusaineid, siis enamik meie tunde olid arvutitehnika gruppidega.

Ma läksin küll venekeelsesse majandusteaduskonda, 
aga grupp oli eestikeelne. Huvitaval kombel ei pidanud me õppima arvutitehnika baasaineid. 
Esimese aasta arvutitehnikas õpiti nimelt füüsikat-keemiat, kõiki üsna 
keerulisi aineid. Ma olen kuulnud õudseid lugusid, kuidas inimesed ei saanud ülikooli 
lõpuni neid tehtud. Aga meie õppisime mikro- ja makroökonoomikat ning 
inglise keelt. Alates teises aastast hakkasime niisiis koos arvutitehnika omadega õppima ja
erilist jõudluse vahet ei olnud.

See, kus ma praegu olen, on ilmselt  
põhjustatud ka sellest, et ma ei läinud väga süvitsi arvutitehnikasse, vaid pigem 
oli arvuti alati vahend mõne probleemi lahendamiseks.

\question{Sa mainisid, et matemaatika tuli sul hästi välja. Kas käisid 
olümpiaadidel ka?}

Käisin, aga ma olin keskmiste seas. See sõltub palju õpetajast. 
Mäletan, et olin kas viiendas või seitsmendas klassis\sidenote{Selle põlvkonna inimestel jäi 
nii vene kui ka eesti koolides üks klass vahele, sest koolid läksid 
kaheksakümnendate teisel poolel üle aasta võrra pikemale õppele}, kui hakkasid geomeetria ja muud sellised ained. Ja siis mul 
klikkis, et iga teoreemi kohta, mida meile räägiti, tekkis mul teine 
viis, kuidas seda tõestada. Ma sain aru, et asjad ei ole alati ainult 
ühtemoodi, saab ka teisiti. See omakorda klikib õpetajaga: kui 
õpetaja näeb, et õpilane mõtleb, siis pöörab talle rohkem tähelepanu. Paraku läks see õpetaja ära ja järgmised ei olnud nii head.

Meil oli üks väga hea füüsikaõpetaja, kes tegi palju kontrolltöid. 
Tema juures õppisin seda, et valemeid ei pea üldse meelde jätma. Piisab, kui oskad neid rakendada. Loomulikult ei olnud spikerdamine 
lubatud, aga mul olid valemid ikkagi spikrina vihiku tagakaanel. Sa pead 
aru saama probleemist ja vahenditest, mida selle 
lahendamiseks kasutada. See õpetaja vaatas valemite teadmisele läbi sõrmede, sest 
kui probleemist aru ei saa, siis lihtsalt valemid füüsikas ei aita. 

Tõnismäe Reaalkoolis oli legendaarne 
matemaatikaõpetaja Mihhail Vassiljevitš\index[ppl]{Vassiljevitš, Mihhail}, kes õpetab seal
siiani. See inimene on tõeline autoriteet, kohtleb 
õpilasi ühtemoodi! Meie matemaatikaklassis oli kolm-neli tippõpilast, kes 
võitsid kõik riiklikud olümpiaadid ja käisid ka maailmaolümpiaadidel. Loomulikult ta 
tegeles nendega, aga ka kogu ülejäänud rahvaga. Oli neidki, 
kes ei saanud matemaatikast väga aru, aga tema juures nende tase tõusis. Ta oskas 
selgitada ka keerulisi asju nii lihtsalt, et kogu klass 
oli paar taset teistest koolidest üle. Ainuüksi selles  
keskkonnas olemine tõstis taset nii kõvasti.


\question{Jällegi tuleb välja, et matemaatikatunnis õpiti lisaks 
matemaatikale suhtumist, ja just see on sul aastate järel meeles.}

Mina ei saanud olümpiaadidel küll mingeid kohti, aga 
meist aasta vanemas klassis oli selline lugu, et umbes kümme inimest läksid keemia-, kümme 
matemaatika- ja kümme füüsikaolümpiaadile. Põhimõtteliselt terve klass osales Tartus riiklikel
olümpiaadidel, aga erinevatel aladel. Ja kuna nad olid juba seal kohal, siis 
neil oli lubatud ka teiste ainete olümpiaadidest osa võtta. Selle tulemusel said enam-vähem kõik, isegi need, kes algul ei kvalifitseerunud,
kõikidel aladel esikümnesse. Hämmastavalt võimas klass!

\question{Miks sa läksid majandust õppima?}

Sest mu vanemad ütlesid, et meil on peres juba kaks inseneri olemas, ema oli 
soojustehnik. Eks ma mõtlesin ka muid variante, aga kodu juures 
oli palju lihtsam. 

\question{Kas sul oli mingi ettekujutus ka sellest, mida sa tahad pärast oma 
haridusega ette võtta?}

Erilist ettekujutust ega plaani mul ei olnud. Tahtsin lihtsalt näha, mis see majandus 
õigupoolest on. Ühel suvel proovisin töötamist müügiinimesena ja selgus, et 
see ei sobi mulle absoluutselt. Müügitöös ütleb 
üheksakümmend kaheksa protsenti inimestest \enquote{ei}, aga mulle ei 
meeldi feilida ja minu jaoks oli \enquote{ei} tol ajal feil. 
Tegelikult nüüd, kui olen Pipedrive'is\index{Pipedrive} juba seitse 
aastat töötanud, saan aru, et see on osa protsessist, statistika. Feil on see, kui sa ei tee seda üheksakümne üheksandat 
müüki, mis võib õnnestuda. Müük on see, et tead neid statistilisi 
numbreid ja plaanid vastavalt nendele. See ei ole feilimine, kui esimene juhuslik 
inimene ütleb, et tal ei ole seda teenust vaja.

\question{Mida sa müüsid?}

See oli tänavamüük, müüsime erinevaid tooteid, näiteks 
tööriistakaste, mis läksid päris hästi, 
elektroonilisi hambaharju ja nii edasi.

\question{See on ju igavesti raske töö!}

See oli väga raske töö. Tulime igal hommikul lattu ja saime päevakvoodi, näiteks tuli müüa viisteist 
tööriistakasti. Kui täitsid kvoodi kahe nädala jooksul, siis 
said järgmise tiitli ja koos sellega endale õpilasi. Ja kui viis 
õpilast said omakorda kvoodi täidetud, siis said 
nii-öelda enda äri. Mina sain õppetunni, et see töö ei ole kindlasti minu 
jaoks. Teadsin, et kui lähen programmeerijaks, saan oluliselt 
rahulikuma töö eest oluliselt suuremat tasu. See sundiski mind umbes 
pool aastat hiljem ütlema: \enquote{Okei, ma lähen.} Ja nii ma läksingi ülikooli 
teise aasta keskel informaatikagruppi.

\question{Kas sa siis programmeerisid juba tõsisemaid asju ka 
või puutusid nendega ainult loengus kokku?}

Tegin kahte projekti, mis tõid natuke raha sisse.

Tol ajal olid hästi populaarsed 
SAT-TV\sidenote{Kaheksakümnendate lõpus ja üheksakümnendatel oli isiklik satelliidivastuvõtja ületamatult 
kallis, piraatlusele vaadati läbi sõrmede, suuri teenusepakkujaid veel polnud, aga väikestel oli juba 
võimalus tegutseda. Siis pandigi mõne kortermaja katusele satelliiditaldrik, 
hangiti piraatkaart tasuliste kanalite jaoks, veeti üle katuste 
ümberkaudsetesse majadesse kaablid ja asuti teenust müüma} firmad. Mõnes 
väikeses rajoonis oli oma kunn, kes pakkus SAT-TVd kuutasu eest. 
Mul oli üks tuttav, kes palus teha infosüsteemi, kus oleks kirjas, kes on 
liitunud, kes ei ole, kui palju nad maksavad ja mis teenust kasutavad. 
Emal oli tööl arvuti, millega sain teha Accessi\index{Microsoft 
Access} andmebaasi ja selle peale väikese liidese.

Teine projekt oli veel huvitavam. Kui sain teada, kui palju raha ma selle töö eest 
saan, olin väga imestunud. Isa sõbrad tegelesid valvesüsteemidega ja neil oli 
üks vanglaprojekt, valvesüsteemi panemine vanglasse. Neil oli 
tarvis joonistada vangla projekti järgi skeem, 
kus oleks näha, kus on alarmid tööle läinud. See ei olnud otseselt 
programmeerimine, rohkem disain. Mina pidingi selle skeemi
joonistama, sain selle kolme nädalaga tehtud ja tasuks
umbes isa poole aasta palga. Siis sain aru, et 
arvutitega tasub toimetada.

\question{Kust sa infot said? Accessis programmeerimine ei ole 
niisama lihtne, et hakkad muudkui otsast tegema.}

Accessi kohta ma ei mäletagi, eks vist lugesin dokumentatsiooni. 
Programmeerimist õppisin 
raamatutest. Mul oli üks venekeelne Pascali raamat, mis õpetas objektorienteeritud 
programmeerimist. Ka ülikoolis olid mõned ained väga-väga 
kasulikud, näiteks andmebaaside projekteerimine. Tänapäeval paljud 
inimesed ei oska relatsioonilist andmebaasi projekteerida, aga see on üks 
vajalikumaid oskusi, kui tahad kasvõi lihtsat süsteemi kokku 
panna. Tänapäeval lahendatakse selliseid asju tihti jõuga.

\question{Kas sinu reaalainete ja arvutihuvi juurde käis ka 
mõne spetsiifiline, näiteks ulme- või raamatuhuvi? Vene keeles oli ju palju 
rohkem asju kättesaadavad, mina ei olnud suuteline tol ajal Strugatskeid 
originaalis lugema.}

Ei mäleta, et oleks väga olnud. Raamatuid lugeda mulle meeldis, samuti
ulme või fantastika. Aga arvutite suhtes ei tekkinud mul tugevat
tunnet, minu jaoks oli arvuti nii praktiline asi, kui olla saab. Lugesin
Bulõtšovit\sidenote{Kir Bulõtšov (1934--2003), Nõukogude 
ulmekirjanik} ja Strugatskeid\sidenote{Arkadi Strugatski (1925--1991)  ja Boris Strugatski (1933–2012). Nõukogude ulmekirjanikud, kes kirjutasid enamasti koos, seega tuntud kui \begin{russian}братья Стругацкие\end{russian} või lihtsalt Strugatskid}, aga ka 
välismaa asju. Olen ka kõik Barbar Conani\sidenote{Robert E. 
Howardi (1906--1936) 1932. aastal loodud tegelane, kes on sellest ajast tembutanud 
kõikvõimalikes meediumides ajakirjadest ja raamatutest filmide ja 
videomängudeni} ja Tarzani\sidenote{Edgar Rice Burroughsi (1875–1950) 1912. aastal 
loodud tegelane, kes sai Nõukogude Liidus tuntuks kinodes näidatud 
trofeefilmidest (Johnny Weissmulleri kehastatud tegelane erines küll oluliselt 
raamatukangelasest)} raamatud läbi lugenud.

\question{Mis su esimene päris programmeerijatöö oli ja millal?}

Veebruaris 1996 läksin tööle Aeteci
Finantsvara ASi\index{Aeteci Finantsvara AS|see{Profit Software}}, mis nüüdseks 
on Profit Software\index{Profit Software}. Mul olid seal sõbrad ees. Nad tegid soomlastele 
finantskindlustussüsteeme. Oma esimese tööülesandega ma 
feilisin, sest mulle anti mingisuguse
valemi programmeerimine. See pidi Cs\index{C} olema ja sellest pidi 
\emph{library} saama. Ma ei teadnud, kuidas Csi 
kirjutada, ma ei saanud sellest valemist aru (see oli kõrgem matemaatika). 
Ühesõnaga, sellega ma feilisin. 

See-eest olin väga hea Lotus 
Notesi\index{Lotus Notes Domino} tarkvaras, mida kasutati suhtlemiseks omavahel 
ja soomlastega. See oli dokumendiandmebaas, millel oli oma 
skriptimiskeel. Sellega ma kirjutasin reisikindlustuse süsteemi 
kindlustusagentidele, et nad saaksid välja arvutada, palju reisimine maksab, 
ja poliisi teha. Ja see oli internetipõhine aastal 1997. Dominoga 
oli võimalik samu dokumente, mida muidu nägi Lotus Notesis kliendi 
kohta, ka veebiserveri kaudu ehk HTML-dokumentidena näidata.

See kogemus aitas mul saada Hansapanga\index{Hansapank} internetipanga 
tiimi.

\question{Kuidas sa sinna sattusid?}

Hansapanga ITs või üldse pankades ilmselgelt 
oli rohkem raha kui mõnes IT-firmas. Kui olin kaks aastat Aeteci
Finantsvaras töötanud, tundsin, et võiks nii-öelda karjääri teha. Proovisin tegelikult 
kõikidesse pankadesse tööle saada, igal pool oli vabu kohti. 
SEBs\index{SEB|see{Ühispank}} ehk toonases Ühispangas\index{Ühispank} ma ei 
saanud isegi vist jutule, aga Hoiupangas\index{Hoiupank} rääkisin Aleksei 
Bljahhiniga\index[ppl]{Bljahhin, Aleksei}. Hansas oli ka tööintervjuu, läksime sinna koos 
Vilve Vene\index[ppl]{Vene, Vilve} ja Heiki Kübbariga\index[ppl]{Kübbar, 
Heiki}. Ja sain mõlemast pangast tööpakkumise umbes sama summa peale. 
Otsustasin Hansapanga kasuks, sest arvasin, et seal võib olla natuke rohkem 
karjäärivõimalusi. 

Minu esimene tööpäev Hansapangas oli 
19. jaanuaril 1998. Fuajeesse astudes märkasin värsket 
Äripäeva, kus oli kirjas, et Hoiupank ja Hansapank ühinevad. Nii et minu 
esimesel tööpäeval teatati ühinemisest ja see määras kogu mu järgneva karjääri.


\question{See tähendab, et pidid suhteliselt ruttu hakkama 
internetipanga asemel tegelema hoopis Light Telleri\label{sisu:teller} nimelise telleri 
töökohasüsteemiga?}

Sinna läks veel natuke aega. Otsus hakata seda tegema sündis 
umbes viis-kuus kuud peale seda, kui ühinemine pihta hakkas. Alguses  
ei olnud ju veel selge, kumba süsteemi üldse hakatakse kasutama ja kuidas see 
otsus tehakse. Sellel ajal õppisin mina, kuidas internetipanka teha.

\question{See kõik on mulle üllatus. Mina läksin sinna panka 
1999. aasta lõpus. Light Teller oli selleks ajaks olemas ja laua taga oli 
vana kala nimega Sergei, kes oli selle oma käega valmis teinud. Kui nüüd 
näppudel arvutada, siis järelikult tegid sa nullist 
täisfunktsionaalse veebipõhise telleri töökoha umbes kolme kuuga?}

Ega ma seda üksi teinud. Aga astume sammu tagasi. Hansapanga esimene 
internetipank oli üles ehitatud tehnoloogiale, mis oli ajast ees. See 
oli Oracle'i\index{Oracle} veebikomponent või -server, kus 
sai PL/SQLiga\index{PL/SQL} tekitada HTMLi, mida kliendid 
vaatasid. See oli omal ajal hästi lihtne, ilma igasuguse disainita, sest 
disaineritest ei teadnud tol ajal vist keegi, et on olemas selline amet nagu disainer. 
Trükidisainerid kindlasti olid, aga kasutajaliidese disaineritest polnud keegi kuulnud. 

Mina mõtlesin, et oo, milline 
ebavõrdsus, et internetipank on ainult eesti keeles. Ütlesin, et ma võin teha selle 
mitmekeelseks. Seepeale öeldi, et tee. Ja tegingi. Kaks nädalat tegelesin 
sellega, et võtsin kõik tekstid välja ja asendasin \verb|lang|-funktsiooniga, mis 
arvestas ka kasutajaprofiiliga. Samal ajal õppisin veel ülikoolis, olin sel
päeval, kui Madis Ollisaar\index[ppl]{Ollisaar, Madis} asja tootmisse pani, koolis. 
Logisin sisse, et vaadata, kas töötab. Eesti keel töötas, inglise keel 
töötas, vene keel aga näitas küsimärke. Ilmselt inimesed mässavad siiamaani nende 
\emph{encoding}'utega, aga see oli minu esimene kokkupuude sellega, et minu 
arvutis töötab, aga serveris mitte.

Samal ajal hakkas ühinemise tõttu juhtuma mitu asja korraga.  
Aleksei Bljahhin\index[ppl]{Bljahhin, Aleksei} tegeles \emph{data} migraga. 
Tekkis probleem, kuna telleriprogramm oli kirjutatud Oracle Formsis ja igas 
kontoris oli Formsi server. Kõik tellerid kasutasid Formsi klienti, mida 
serveeriti serverist, ja nad võtsid peaserveriga Oracle'i 
andmebaasiühenduse. Oracle'i litsentside eest maksti teatavasti ühenduste arvu 
pealt. Hansapangal oli tol hetkel, no ma ei tea, mingi nelikümmend 
kontorit. Nüüdseks see on juba suur number, aga Hoiupangal oli nelisada 
kontorit!. Paljudes maakohtades ei olnud isegi nii head sidet, et 
hoida pidevat ühendust andmebaasiga. Kui nad arvutasid, kui palju 
Oracle'i litsentsid oleksid kokku maksnud, siis nad ütlesid, et võib-olla anname 
Hoiupanga tagasi. 

Tegelikult tehti väga julge otsus teha interneti telleriprogrammi. Otsustajateks olid ilmselt needsamad Vilve\index[ppl]{Vene, Vilve} ja 
Gibbs\index[ppl]{Gibbs|see{Kübbar, Heiki}}\sidenote{Sergei peab silmas Heiki Kübbarat, kes oli toona paljude Hansapanga innovatiivsete ideede taga.}. Samal ajal müüdi meile internetipanga tegemiseks uus tehnoloogia, BroadVisioni\index{BroadVision} platvorm. 
BroadVisioni müügiargumendiks oli, et saame põhimõtteliselt e-kommertsi 
platvormi, millel sai igale kasutajale näidata personaalselt välja nägevat 
rakendust.
Samas iga kasutaja maksis, mis tähendas, et me ei kasutanudki kunagi seda
võimalust, süsteemi mõttes oli kõik anonüümne. Ühtlasi pakkus BroadVision
\emph{template}'imise võimalust, mis oli väga suur samm edasi võrreldes Oracle'i 
PL/SQLiga, kus tuli oma HTML ise kokku panna. Nii et selle peale me internetipanga ehitasimegi. 
Tagantjärele mõeldes oli see telleri arhitektuur lihtne, aga võimas. See võimaldas 
kiiresti ja suures koguses funktsionaalsust toota.

\question{See arhitektuur oli siis toonaseid vahendeid kasutades täpselt selline, nagu tänased \emph{de facto} 
veebirakendused on. JavaScript\index{JavaScript} jooksis brauseris ja 
tegi \emph{backend}'i poole päringuid. See lahendus oli 20 aastat ajast ees, 
kuidas see sündis?}

Meil tuli arvestada piirangut, et maakontorite ühendus oli väga aeglane. 
Pidime optimeerima, kui palju \emph{data}'t kliendi ja serveri 
vahel liigutada. See sundiski palju tööd juba 
kliendipoolel ära tegema. Kliendiks oli brauser ja JavaScripti versioon oli 
selline, et parimal juhul sai teha valideerimist. Midagi joonistada või dünaamiliseks teha eriti
ei saanud. Samal ajal tuli 
Internet Explorer 4.0\index{Internet Explorer}, kus olid \emph{custom} 
JavaScripti võimalused, mis lasid palju dünaamilisemat 
lehte ehitada. Tol ajal ei olnud ju mingisuguseid JavaScripti \emph{library}'sid, nagu 
Reactid\index{React} ja muud, mis võimaldavad kõike teha. Sa kirjutasid puhast 
JavaScripti, isegi Githubi ega Stack Overflow'd ei olnud. Oli Internet Exploreri 
dokumentatsioon.

Ja kuna kõik Hoiupanga töötajad olid harjunud ilma hiireta
terminaliga (hiire kasutamine aeglustab tööd), siis oli ka 
nõue, et kasutaja pidi saama navigeerida brauserirakenduses ilma hiireta. 

\question{Põhimõtteliselt ju tehtav, aga kasutajaliidese disaini mõttes 
päris keeruline ülesanne.}

Arvestades kõiki neid piiranguid pidin välja tulema mingisuguse kliendipoolse 
raamistikuga ja tulin ka. Seal tekkis päris palju koodi ja tol 
ajal tuli tüüpilises brauserirakenduses vajutada \emph{submit}-nuppu, mispeale 
terve leht laeti uuesti. Meil aga ei olnud kontorite vahel \emph{bandwidth}'i. Näiteks kui viis tellerit, kes istusid 28 K 
modemi\sidenote{Sidet üle telefoniliinide 
reguleerisid Rahvusvahelise Telekommunikatsiooni Liidu V seeria 
soovitused. V.34 kirjeldas sidet kuni 33,6 kbit/s, kuigi levinuim oli 
mainitud 28,8 kbit/s kiirus.} peal, vajutas nuppe, hakkas iga nupuvajutusega 
tulema sadades kilobaitides lehte. Tollal tekkisid 
\emph{frame}'id ja \emph{frameset}'id\sidenote{HTML 4.0, mis avaldati 1997. aastal 
W3C soovitusena, sisaldas eraldi variatsiooni \enquote{raamide} (ingl 
\emph{frame}) toega. Raamid võimaldasid jagada HTML-lehe eri aadressidelt 
laetavateks alamosadeks. HTML 5.0 enam raame ei toeta.}, mille vahel sai andmeid 
vahetada brauseri sees. Nii et oligi üks \enquote{menu} \emph{frame}, kus oli 
enamik JavaScripti loogikat, mida kunagi uuesti ei laetud, ja 
\enquote{main} \emph{frame}, mille sees laeti iga konkreetne tegevus.

\question{Seal tehti veel mõningaid huvitavaid asju, näiteks olid peidetud raamid, 
mis käitusid nagu praegune brauserist algatatud REST päring.}

Eks see arenes. Rakenduses oli \enquote{main} \emph{frame} ja 
kliendiandmete \emph{frame}, sest tavaline \emph{workflow} oli selline, et kui 
klient tuli, siis leidsid tema konto ja said seal teha makseid, 
hoiuseid ja mida iganes. Klienti otsides tuli laadida 
tema andmed eraldi kliendiraami, kus olid nähtavad kliendi nimi, konto nimi ja 
kontonumber, aga seal all olid veel ka brauseripoole peal kliendiandmed. Ja siis 
meil oli \enquote{foori} \emph{frame}, mille kaudu \emph{submit}'isime vormi 
andmeid, sest valideerimine pidi jällegi toimuma kohapeal. Nupp käivitas 
valideerimismeetodi, mille tulemusel saadeti andmed teise vormi 
kaudu serverisse. Ma ei mäleta, miks me nii tegime, ju oli vaja. Aga see 
oli nagu raam, mille sees said kõik pangafunktsioonid tehtud. Selle püstipanekuks ja esimese 
Eesti-sisese maksevormi tegemiseks kulus kuu aega. Kui see sai valmis, siis kõik 
ülejäänud funktsioonid tulid kahe kuuga. Põhimõtteliselt 
\emph{copy-paste}, midagi keerulist ei olnud, ainult pärast pisut vigade 
parandamist ja optimeerimist.

\question{Kui sa nüüd tagasi mõtled, siis mis sulle andis põhja, et selline asi teha? Oli see 
ülikool, lihtsalt häkkerimentaliteet või veel midagi?}

Ei olnud mitte midagi peale probleemi, mida oli vaja 
lahendada. Muidugi oli sealjuures ka muid nõudmisi, millest ei 
saanud üle ega ümber. Näiteks tellerirakenduse puhul oli spetsiifiline nõue, et see ei tohi inimest väsitada, st me
ei tohtinud kasutada erksaid värve, sest selle programmiga tehti päevas kaheksa tundi 
tööd. Sellepärast see saigi hall. Tol ajal me tegime ka hanza.net'i\index{Hansapank!hanza.net} 
ja see oli värviline, disaini mõiste oli juba olemas.

\question{Sellise asja peale tänapäeval sageli isegi ei mõelda, kust 
see nõue tuli?}

Meil oli tubli pangatehnoloogia osakond, kes mõtles, kuidas tellerid saaksid oma tööd teha
hästi efektiivselt. Kordan, et mina olin ainult 
teostaja, asja taga oli terve tiim. Meil oli Toomas Rand\index[ppl]{Rand, 
Toomas}, kes kirjutas kogu pangaloogika; mina tegelesin 
ainult kasutajaliidesega ja andsin talle andmed. Pangasüsteemis 
toimuv oli tema teha ja ta istus täpselt samamoodi kaksteist tundi päevas töölaua taga ja 
tegi. Tänu sellele projektile tekkis pangasüsteemi arhitektuuris 
korrastatus. Oracle Formsiga sai kutsuda suvalisi funktsioone otse vormist, seevastu kui 
meie arhitektuuri ütles, et üks nupuvajutus ja ongi kogu tehing tehtud. Ennekõike tuli kokku leppida liideses 
ja siis said osapooled oma osaga edasi tegeleda. See 
võimaldas testimist, testimise automatiseerimist ja töö paralleliseerimist. 

Kitsendused sunnivad tegelikult tegema õigeid otsuseid. Paljudel inimestel ei ole piiratud  
ressurssidega toimetamise kogemust, eriti välismaalastel. Näiteks tuleb Silicon Valleyst inimene, kes ei saa aru, miks
me ei palka inimesi juurde. Mis mõttes ei saa kõiki oma ideid realiseerida, vaid
peab prioritiseerima? See on tema jaoks probleem, kuna ta ei saa 
aru, mis tähendab, et raha ei ole. Näen, et Eestis 
on palju ära tehtud väga vähese ressursiga täpselt selle pärast, 
et inimesed oskavad teha õigeid valikuid. Prioritiseerima peab, sest ressurssi ei 
ole.

\question{Kui ilusast arhitektuurist edasi minna, siis milline on ilus kood?}

Ilus on kood siis, kus inimene ei pea küsima, mida see teeb. Väga 
paljud, kes oskavad programmeerida, arvavad millegipärast, et mida 
optimeeritum või lakoonilisem kood on, seda parem, kuid see teeb halba. On piir, kust
edasi teine inimene ei saa enam aru, mida kood teeb. Selline kood ei ole hea, isegi kui teeb õiget asja. See on üks asi. Teiseks pean ma 
ütlema sulle suur aitäh selle eest, et tõid omal ajal Eestisse Joshua 
Kerievsky\index[ppl]{Kerievsky, Joshua}\sidenote{Joshua Kerievsky on USA firma 
Industrial Logic asutaja ja üks pikema kogemusega agiilse tarkvaraarenduse 
praktikuid ja koolitajaid maailmas. Tema Eestisse toomise Hansapanga arendajate 
koolitamiseks kas 2000. aasta lõpus või 2001. aasta algul algatas siiski Erik 
Jõgi\index[ppl]{Jõgi, Erik}}. Elus tekivad hetked, kui saad aru, 
et see on nüüd \emph{step function}. Tema koolitus viis
väga paljud asjad oma kohale. Joshua on tegelenud koodi \emph{refactor}'iga ehk kuidas teha kehvast koodist ilusat. Samuti rääkisime temaga \emph{unit}'i 
testimisest \ldots

See aitabki ilusat koodi kirjutada: sa 
pead seda mitu korda ümber kirjutama, enne kui see näeb loogiline välja.

\question{Tagantjärele mõeldes oli kogu see 
agiilse arenduse liikumine ja mõtteviis tol ajal veel väga noor.}

Kui ma tulin Skype'ist\index{Skype} Pipedrive'i, siis siin on meil 
igasugu \emph{agile coach}'e. Ma korraldasin sellise eksperimendi, et rivistasime oma 
\emph{agile coach}'id, arendajad selle järgi, kes on 
\emph{agile} liikumisega kõige kauem tegelenud või sellest vähemalt teadlik olnud. Enamiku puhul oli see aeg 7-8 aastat. Mina olen sellega 20 
aastat tegelenud! \emph{Agile Manifesto}\sidenote{Vt. \url{https://agilemanifesto.org/}} tekkis vist 2001. või 
2002. aastal. Tegelikult me kõik saime seda maitsta enne, kui see popiks muutus.

\question{Mida sa praegu teed?}

Ma isegi ei saa öelda, et juhin \emph{engineering}'u 
organisatsiooni, sest ma juhin ka muid organisatsioone. Olen 
Pipedrive'is\index{Pipedrive} juba seitse aastat olnud. Aastal 
2013 meeskonnaga liitudes oli see väike ja ambitsioonikas firma. Tööintervjuul küsiti minult, kas ma usun, et suudame Salesforce'iga võistelda. Ütlesin, et päris 
Salesforce'iks me ei kasva, aga võib-olla veerand sellest on võimalik. Siis 
oli meil kakskümmend inimest, kümme inseneri. Nüüdseks, kuus aastat hiljem ja natuke peale, on meid kuussada.

Kõik need aastad olen tegelenud skaleerimisega: nii infosüsteemi kui ka 
organisatsiooni skaleerimisega. Selle aja jooksul ei ole kordagi tekkinud mõtet, et 
äkki meil ei õnnestu, äkki me ei kasva. Niipea kui hakkad niimoodi mõtlema, siis 
ei kasvagi. Ma ei ole tegelikult siiamaani kindel, kumb on põhjus ja 
kumb tagajärg: kas see, et oleme skaleerinud \emph{engineering}'ut, 
aitas Pipedrive'il kasvada või see, et ta kasvas, aitas meil skaleerida 
\emph{engineering}'ut.

Kui vaadata teisi osakondi, siis näiteks turundus ei skaleerunud. 
\emph{Product} pidi skaleeruma koos \emph{engineering}uga, muidu inseneridel 
poleks midagi teha. Müük ei skaleerunud, \emph{support} skaleerus 
nii-öelda tagantjärele. Tegelikult \emph{engineering}'u kasvatamine 
kasvatas firmat. Samas, kui ettevõte ei kasvaks, siis ei saaks ju ka 
inimesi juurde palgata. Küsimus on, 
mis tõukas kasvu tagant. Me eriti ei mõelnud sellele, vaid olime 
kindlad, et peame skaleeruma. Minu kõige suurem hirm on olnud jääda pudelikaelaks. See, et \emph{engineering}'u peale hakatakse 
näpuga näitama, et tahaks küll seda või toda teha, aga 
\emph{engineering}'ul ei ole ressurssi või süsteemid hakkavad 
kokku kukkuma, kui kliente on liiga palju. Või et palkame inimesi juurde ja 
nad ei saa oma tööd teha, sest kuskil protsessis on pudelikael. 
Või me ei saagi inimesi palgata, sest nad ei taha meile tööle tulla. Neid 
pudelikaelu, millega korraga tegeleda, on olnud palju. Aga kui kuidagi ei saa, siis kuidagi ikka saab!


\chapter{Arne Ansper}
\index[ppl]{Ansper, Arne}
\question{Nagu ikka alustame sellest, kuidas asjad alguse said. Kuidas nad siis 
said sinu jaoks alguse?}

No minu jaoks need asjad alguse sellest, et kui ma põhikooli lõpetasin siis 
minu matemaatikaõpetaja arvas, et ma peaksin minema Nõkku\index{Koolid!Nõo 
Keskkool} edasi õppima. Ja suutis mu vanemaid ära veenda, et see on suurepärane 
mõte, siis ma sinna läksingi.

\question{Aga kus sa põhikooli lõpetasid?}

Jõõpres\index{Koolid!Jõõpre kool}\index{Jõõpre}, selline pisike koht Pärnu 
lähedal. Sada õpilast oli see põhikool meil vanas  mõisas, mitte mõisamajas 
endas aga koolimaja oli mõisa keskel. Niisugune väga mõnus koht oli. Ja siis 
mul matemaatika nagu sobis ja õpetaja oli väga usin, andis mulle lisaülesandeid 
ja lõpuks saatis olümpiaadile ja seal läks ka suht hästi.


\question{Sa siis tulid puhtalt matemaatika ja mitte arvutite nurga alt sinna 
Nõkku?}

Ei, mul oli null kokkupuudet arvutiga enne. Vanemad seejuures pigem nagu 
tahtsid, et ma läheks. Ma ise olin väga  kahtleval seisukohal, et kas kodust 
nii kaugele minek, et see on äkki kuidagi raske ja paha ja nii edasi. 

\question{Mis aastal see oli?}

1985. 

\question{Sel ajal oli juba logistiliselt ju keeruline Pärnu lähedalt Nõkku 
saada?}

See oli lihtne ja tüütu, selles mõttes, et olid bussid, mis sõitsid neli tundi 
ja olid tavaliselt maast laeni rahvast täis ja siis veel Pärnust koju kus buss 
käis kahe tunni tagant. Seal ikkagi võttis aega, ütleme nii.  

\question{Ja Nõos pandi kohe arvuti ette?}

Ei, Nõos see oli tavaline keskkoolielu selle väikse vahega, et tuli ühikas 
elada. Mina olin viimane aasta, kes elas poiste ühikas, mis on selline 
suhteliselt raju ja legendaarne koht. Ehitatud kuskil tsaariaja lõpus, Eesti 
aja alguses. Talvel oli niimoodi, et tulid  kodust, tõid sihukesed suured 
märjad puunotid, läksid oma tuppa, mis oli  null kraadi lähedal kütsid ta siis 
üles selleks, et magada saaks. Hommikul lõid ikkagi pesukausi pealt jää katki, 
kui hakkasid hambaid pesema, niisugune koht oli. Esimene aasta oli hästi lahe. 

Alguses oli tavaline keskkond ja siis tuli programmeerimise õpetamise lihtsalt 
ühe regulaarse ainena sisse ja hakati õpetama. See oli ikkagi matemaatika ja 
füüsika kallakuga kool aga programmeerimise õpetamine seal oli lihtsalt nagu 
aine nagu mida iganes muugi. Mahud, loomulikult, olid suuremad nii 
matemaatikal, füüsikal kui ka sellel, programmeerimisel, millel mujal oli null, 
et seal oli siis nagu mingi muu number.

\question{Räägi palun Nõo kooli taustast, kuidas sinna üldse sai?}

Tead, ma ei tea. Mina olin tollal niisugune inimene, et emaga koos me sinna 
läksime. Ma arvan, et me käisime direktori juures rääkimas. Et kuna mul oli 
tegelikult olümpiaadilt mingisugune koht ette näidata siis kuidagi ma sinna 
igatahes sisse sain. Kuidas täpselt, kas seal oli mingi konkurss või mingi muu 
süsteem, ei tea. 

\question{Kes Nõo kooli direktor tol ajal oli? See kool tundus kellegi 
entusiasmi peal käivat?}

Enn Liba\index[ppl]{Liba, Enn} oli minu meelest tol ajal direktor\sidenote{Nõo 
kooli arendas selliseks reaalteaduste ja programmeerimise õppe keskuseks, nagu 
me teda praegu tunneme, Kalju Aigro\index[ppl]{Aigro, Kalju}. Ta oli kooli 
direktoriks aastatel 1951---1982, talle järgneski selles ametis Enn Liba.}. Aga 
seda entusiasmi aspekti ja ajalugu, ma pean tunnistama,  ma ei oska 
kommenteerida tollal huvitusin  oluliselt muudest asjadest.

\question{Aga mis asjad need olid, millest sa huvitusid?}

Tegelikult mulle meeldis põhikoolis elektroonika. Aga see oli selline 
platooniline huvi, kuna juppe oli hullult raske kätte saada. Ja mulle meeldisid 
mudellennukid, mis oli ka suhteliselt platooniline. Aga Nõos tuli 
programmeerimine hästi kiiresti peale, kui hakkasime seal õppima. Seal oli suur 
Vene \emph{mainframe} Nairi-3-1\index{Arvutid!Nairi-3-1}\sidenote{1964. aastal 
Jerevanis välja töötatud Nõukogude arvutiperekonna Nairi kõige võimekam liige. 
Kool sai selle arvuti 1977. aastal.}. 
KÕPS\index{Keeled!KÕPS} ja ROPS\index{Keeled!ROPS}\sidenote{\label{sidenote:ROPS}KÕPS ja ROPS on 1980. 
aastate teisel poolel Nõo Keskkooli arvutuskeskuses välja töötatud eestikeelsed 
programmeerimiskeeled, millede loomisel osales ka Arne esimene arutiõpetaja Nõos 
Uuno Puus\index[ppl]{Puus, Uuno}. KÕPS oli sarnane MIT-is välja töötatud keelega 
LOGO, võimaldas vaid graafikat ning tugines LOGO looja Seymour Papert-i ideoloogiale. 
ROPS oli KÕPS-i edasiarendus, mis olla sarnanenud Algolile ja võimaldas lisaks 
graafikale ka arvutusi.}, eesti keeles sai 
programmeerida, need olid  vahvad. Siis olid seal Agatid\index{Arvutid!Agat}, 
mille ligi suht ruttu sai, mis olid teistmoodi vahvad, kus sai mingit 
valmistarkvaraga ka kasutada. Ikkagi mingite mängude mängimine oli oluline ja  
siis ise mingite asjade proovimine. See nagu hakkas väga kiiresti meeldima.

a \question{Oskad sa takkajärgi kuidagi reflekteerida, mis sulle seal meeldima 
hakkas?}

Väga ei oska, ausalt öeldes. Ma üritasin mõelda, et mis ma siis tegin nende 
arvutitega toona. Mul on umbes kaks asja meeles mida ma Agatiga tegin. Esimene 
programm oli umbes see, et oli \verb|for| tsükkel: muutis värvi, trükis mingi 
teksti nagu, ütleme, \enquote{tere}. Kõigis keeltes ja siis veel vilkuva 
taustaga ka. Sellega sai vähemalt üks õhtu kui mitte kauem möllatud ja timmitud 
neid efekte, tekste ja asju. Ja siis teine asi, mis mul on meeles, ma püüdsin 
ühte Nintendo mängu (need pisikesed puldi mängud, mis olid\sidenote{\label{sidenote!gameandwatch}Arne peab 
ilmselt silmas Nintendo Game \& Watch\index{Nintendo Game \& Watch} seeria käes hoitavaid mänge. 
Originaalidest oluliselt rohkem oli liikvel nende Nõukogude kloone, mida müüdi 
Elektronika kaubamärgi all. Tegu polnud siiski alati täpsete koopiatega: 
Nintendo EG-26 kloonis IM-02 püüdis mune Miki Hiire asemel hunt tuntud 
Nõukogude multifilmist \begin{russian}Ну, погоди!\end{russian}}) taasluua, ma 
lõingi. Seal oli, nagu ta on, mingi fikseeritud arv positsioone, mingi tegelane 
liikus, mingid teised tegelased liikusid ja siis olid mingid surmasaamised ja 
mingid boonuste saamised. Probleem oli selles, et ma ei teadnud tollal, mis asi 
on massiiv. Põhimõtteliselt oli niimoodi, et iga objekti jaoks oli mul muutuja, 
mis ütles, et kas objekt on või ei ole. Ja kui seal mingid asjad liikusid, siis 
mul oli lehekülgede kaupa \verb|if| lauseid, et kui see muutuja omab seda 
väärtust, siis järgmisel sammul ta omab teist väärtust. Ja muidugi 
refaktoreerimis-tööriistu ei olnud. Kui ma kuskil vea tegin, siis ma nägin 
päevade kaupa vaeva, et ma nimetasin neid oma muutujaid ja \verb|if| lauseid 
ümber.


\question{Väga huvitav. Tol ajal tundus asjadest mitte rääkimine olevat 
õpetamise metoodika osa. Meile näiteks ei räägitud \texttt{for} tsüklist tükk 
aega}

Ütleme nii, et seda Agati\index{Arvutid!Agat} ei õpetanud meile keegi. Õpetati 
Kõpsi ja Ropsi. Kõik, mis Agati peal sai tehtud, see oli puhas enda välja 
võidetud ja võideldud  arvutiaeg, enda entusiasm. Ma isegi ei mäleta, \emph{by 
example} käis see asi vist, et vaatasid, mida keegi teine oli teinud. Mina küll 
ei mäleta, et oleksin ühtegi, Agati või BASICu\index{Keeled!BASIC}  kohta 
käivat raamatud lugenud kunagi. Kõik see oli lihtsalt nagu folkloor, 
katsetamise ja kõlakate tasemel. Et oleks keegi lekitanud selle info, et 
massiivid on olemas, oleks selle Nintendo mänguga palju rutem valmis saanud. 

\question{See oli suur töö ju, pidi ikka kihu olema?}

No aega oli palju, segavaid faktoreid oli vähe, eks ole. Ja ilmselt siis see 
arvuti alistamine meeldis, nagu välja tuleb. Agatiga\index{Arvutid!Agat} ma 
mäletan seda kindlasti, et ma hankisin endale selle 
assembleri\index{Keeled!Assembler} nii-öelda manuaali. Mis oli põhimõtteliselt 
paar-kolm ruudulist lehte, kuhu ma siis kirjutasin tähtsamad käsud ja registrid 
ja värgid üles ja siis studeerisin seda. Ja ma tean, et ma ikkagi nagu 
tuuseldasin seal Agati assembleri poole peal ringi. Aga mida ma tegin, seda ma 
kindlasti ei mäleta. Mäletan olulisimaid registreid, mida näppides käis piiks 
ja kust sai lugeda mingit vist klaviatuuri sümboleid või midagi sellist, aga 
\emph{that's it}.

\question{Kuidas Nõos tase oli, seal olid kõik sinusugused koos?}

Seal oli  selliseid inimesi, kes olid üle vabariigi kokku tulnud, kellel olid  
mingid huvid ja eeldused  reaalainetega tegelemiseks. Aga seal oli ka noh 
lähikonna inimesi. Et see on nagu päris, selline geto kuskil, see oli ikkagi 
nagu natukene spetsialiseeritud kohalik kool, et seal oli igasuguseid inimesi

\question{Kas sealkandis mingit äri tegemist ka juba käis, keegi raha eest 
programmi ei kirjutanud? Kaheksakümnendate lõpp ikkagi?}

Võib-olla keegi tegi, aga  ma julgeks öelda, et ma isegi ei huvitunud sellest 
ja ma ei tea sellest midagi. 

\question{Tartu vahet ka käisite?}

Jaa. Mingil hetkel, ma ei mäleta enam mis klassis, aga siis ma sain teada, et 
Tartu Ülikooli Raamatukogus\index{Tartu Ülikool!Raamatukogu} on mingisugune 
XTde\index{Arvutid!XT} klass. Kaheksa kuni kümme arvutit oli seal. Kuidagi ma 
sain sinna juurde, ma ei mäleta, mis alustel sinna seda aega sai reserveerida. 
Igatahes ma tean, et ma seal ikkagi jõlkusin päris mitu õhtut nädalas. Sa 
said seal mingisuguse tunni või kahese \emph{slot}i, mul oli umbes kaks flopit, 
millest ühe peal oli Turbo C\index{Keeled!Turbo C} ja teise peal oli siis tüüpi 
opsüsteemi oma asjad ja siis  midagi ma seal programmeerisin. 

\question{Aga kust sa said tolle Turbo C?}

Ma ei kujuta ette, kus ma selle saada võisin. Seal ma käisin päris tükk 
aega aga seal ma põhiliselt tegelesin ka sellega, et mängisin selle Turbo Cga. 
Aga kas mul ka mingi eesmärk oli, seda ma ei mäleta. Aga Turbo C see oli 
igatahes.


\question{On ikka paras hüpe Kõpsust ja Ropsust C ja mälu ja pointeriteni? 
Mille pealt too hüpe tuli sul?}

Jällegi nii kauge aeg, et ma kardan, et meile koolis ma isegi mäletan seda, mis 
meile üheksandas klassis  programmeerimist õpetati, aga ma ei mäleta, mis edasi 
sai, ausalt öeldes. Mida meil seal üldse räägiti. Ilmselt ise liikusin 
kiiremini edasi. Pärast TPIs\index{TPI|see{Tallinna Tehnikaülikool}} 
\index{TPI} ka see asi esimestel kursustel, et need  programmeerimise loengud 
olid  sellised, et sealt ei olnud midagi uut saada. Seal  mingid teised asjad 
olid pigem  need, mis olid uued, aga mitte see programmeerimise pool. 


\question{Kuidas sa sealt Nõost TPIsse\index{TPI} sattusid? Oleks ju loogiline, 
et sa lähed sealt Tartusse matemaatikasse?}

See oli ka suht \emph{random}iga selles mõttes, et ma mõtlesin, et võib sinna 
minna või tänna minna. Need argumendid, miks  Tallinnasse proovida, need olid 
niisugused väga otsitud ja õrnad, et miks ma just sinna Tallinnasse läksin 
proovima,  seda ma tegin. 

\question{Mida õppima?}

LI\index{Tallinna Tehnikaülikool!Automaatikateaduskond!LI}. Ma täpselt ei mäleta, kas oli arvutid ja 
arvutisüsteemid, tõenäoliselt võis olla.

\question{See LI lühend jookseb mitmelt poolt läbi aga keegi ei tundu teadvat, 
mida see tähendas}

Kas ta üldse midagi tähendas? Et \enquote{L} on tõenäoliselt mingi 
automaatikateaduskonna kood, eks ole, ja \enquote{I} on mingi muu asja kood. 
Seal oli LA, mis oli äkki rohkem automaatika teisi tähti ei mäleta, äkki on LS 
ka olemas olnud. LI  oli jah see, kus mina oma aega veetsin.

\question{Sa ütlesid, et programmeerimise õpe sind väga edasi ei aidanud, kas 
seal üldse midagi õpetati, mis sulle midagi juurde andis?}

Tagantjärgi  vaadates tundub, et  seal LI-s räägiti nagu laiuti alates sellest, 
kuidas transistori teha, kuidas transistoridest saaks teha mingeid 
mikrolülitusi, kuidas saaks kõik see, mis sorti registrid meil on, kuidas 
registritest mingit automaatikat ehitada. Kuidas protsessorit teha, kui sul on 
neid registreid hulgi käes. Ja  teisele  poole minnes ka, kõik sellised asjad 
nagu siduteooria. Need asjad andsid, tagasivaates, need teadmised, et kui sa 
vaatad tänapäeval enda ümber, siis maagilisi asju, mille kohta ma ei tea, et 
kuidas seda saaks teha või ma pean uskuma midagi või ma vajaduse korral ei 
saaks sinna lõpuni välja kaevuda, neid on väga vähe. Ja see, ma arvan, on üks 
asi, mis mina olen leidnud, hästi kasulik. Tänapäeval on neid kihte sinna nii 
palju juurde tulnud, et vanasti oli ikkagi väga lihtne. See oli umbes nagu 
renessansiajastul, kui üks tüüp suutis  kõike, mida oli mõtet teada, teada. 
Natukene, kui  mina seal TPIs käisin, see aeg hakkas läbi saama. Ütleme 
niimoodi, et tänapäeval ilmselt ei ole võimalik, et sa tead kõike, mida oleks 
kasulik teada arvutiasjandusest. Ma mõtlen just tänapäeval seda, mis riistvara 
poole peal on juhtunud. Sinna on laotud neid kihte ja neid virtualiseerimise 
tasemeid ja mida iganes veel juurde. Ja siis \emph{soft}i poolel on ka vastu 
tuldud, sinna neli kihti virtualiseerimist vahele laotud ja nii edasi. See on 
nagu see, kus kipub nagu raskeks minema see järje pidamine.

\question{Kas TPIsse minek oli asjade loomulik käik või oli sul mingi plaan ka, 
mida tegema hakata?}

Mul niisugused pikaajalisi plaane ausalt öeldes ei olnud. Mulle meeldis teha, 
mulle meeldis nende arvutitega mässata, kas ma mässan Tallinnas või mässan 
Tartus, vahet pole. Ja siis ma mässasin nendega Tallinnas. Üks huvitav nüanss 
on veel see, et et umbes seal keskkooli lõpus ma sain isikliku arvuti ka. See 
oli midagi teistsugust, see oli Atari 520 STf\index{Arvutid!Atari 520 STf}. Mis 
oli siis Atari Motorola 68000 prosega tükk. 512kB oli tal mälu, selle ma 
\emph{upgradesin}  ühe megani mingil hetkel. Selle peal ma siis elasin ja 
siis selle peal ma püüdsin nagu süvitsi minna kogu sellega, mis seal nagu teada 
oli. 


\question{Kust sa sihukese aparaadi said kaheksakümnendate lõpus?}

Mul olid vanaonud, kes elasid Rootsis. Ema ja isa ükskord käisid seal ja siis 
sealtkaudu ma selle siis sain. 

\question{See pidi Agati kõrval ikka ulmeline aparaat olema}

Tegelikult oli niimoodi, et teised olid PCde peal. Kui ma nüüd vaatan, siis 
need inimesed, kellega me siis igal pool nagu koos ringi käisin, siis noh 
üheksakümnenda aasta paiku umbes, normaalsed inimesed said PCdele ligi ja siis 
toimetasid nendega.  Ja siis minul oli kodus Atari  ja tegelesin sellega 
põhiliselt. 

\question{Ataril on kihte vähem, sai lihtsamini sügavale välja minna}

Jaa, see oli nagu hoomatav täiesti,  mis seal toimus, midagi väga ulmelist 
polnud. Natuke mängisin ka, aga mitte liiga palju. Mul ikka see 
programmeerimine meeldis kõige rohkem selle asja juures. Selle Atari peal ma 
tegin igasuguseid imelikke asju.

Ma üritasin CAD programmi teha, joonistamisprogrammi. See isegi lõpuks selles 
mõttes töötas, et seal sai teha ringe ja jooni, igast värke, salvestada ja 
laadida. Ja siis mul oli, tagasi vaadates jälle hullumeelsus, et mulle nagu 
kohutav tegi muret see, et mälu saab otsa. Et kui sa teed dünaamilist 
mäluhaldust, eks ole, et siis saab mälu otsa. Üritasin seda siis minimeerida. 
Näiteks mulle tundus, et nagu lokaalsed muutujad, mis on \emph{stack}is, on 
kuradi ebaefektiivsed. Ja sisuliselt see CAD programm oli kirjutatud 
sajaprotsendiliselt globaalsete muutujate otsa. See oli täiesti hullumeelsus 
nagu tagasi mõeldes, seal tuli ikka kõvasti refaktoreerida, sest ma ikkagi panin 
täitsa puusse alguses. Seal seda loll ümberkirjutamist oli nii palju, sealt ma 
sain selgeks, et okei, nii ma mitte kunagi rohkem ja mitte ühtegi asja ei tee. 
Väga-väga palju vigu sai igatahes tehtud.

\question{Eks see on ju õppeprotsess, mõnda asja teoreetiliselt selgeks ei saa}

Jah, absoluutselt nõus. Ütleme, et nii võimekaid inimesi, kes kogu aeg teiste 
vigadest õpivad, et neid väga palju ei ole. Ikka enamus kipub oma vigadest 
õppima. 

\question{Kui sa TPIsse\index{TPI} jõudsid, kas sa seal teisi omasuguseid ka 
kohtasid?}

Meil oli hästi lahe kursus. Aga tegelikult oli niimoodi, et seal TPI ja alguses 
ma ikkagi õppisin, eks ole. Mis sest, et seal programmeerimise vallas mul ei 
olnud väga huvitav, aga neid muid ained ma ikka õppisin korralikult. Ma olen 
ikka väga usin õppur olnud. Ja mul juhtus niisugune asi, et mind 
Tarvi\index[ppl]{Martens, Tarvi} kutsus ühel hetkel Ektaco-sse\index{Ektaco}, 
ma arvan, et see oli üheksakümmend üks aasta. Ja see oli siis see 
\emph{community}, kus ma siis hakkasin nagu inimestega koos olema ja oli siis 
ka töise karjääri algus. Ma arvan, et see võis olla, see võis olla 1991, aga  
sada protsenti kindel ei ole. Mingi kolmas kursus äkki umbes.

\question{Kolmas kursus on üsna hilja ju?}

Tegelikult ongi see, et programmeerimise õppimine, üldse arvutiasjanduse 
õppimine võtab ikkagi aega. Ma tagasi vaadates mõtlen, et mis ma siis tookord 
oskasin või kuidas ma mõtlesin või  kuivõrd hästi ma siis programmeerisin.  
Ütleksin, et palju varem ei ole mõistlik seda tööd üritada teha. See võib  
frustratsiooni tekitada. Mis mul oli, ma olin ikka viis aastat nüüd innustunult 
selle asjaga tegelenud. Ma arvan, et kui ma  tööle sain, siis ma olin ka noh, 
enam-vähem miinimumtasemel, kus oleks  mõistlik, et keegi annab sulle 
ülesandeid, millele on ka mingi tähtsus ja tähendus ja sa teed nad  ära.

\question{Kas sul midagi sellist ei olnud, nagu inimesed on rääkinud, et 
lihtsalt arvutiaja saamiseks tekkis mingi arvutiklassi admini koht?}

Ei, mul ei ole ju midagi taolist. Ütleme tõesti mälu võib olla natuke petab, et 
mis aastal mul see Atari sinna täpselt tekkis, aga mul kuidagi oli alati 
mingisugune võimalus olemas, nii palju, kui mul seda tarvis oli ja sellest 
piisas. 

\question{Oskad sa mõnda näidet tuua, mida sa seal Ektacos alguses 
programmeerisid?}

Ektaco oli niisugune  firma, kus tehti riistvara ja tarkvara. Ta tegi 
tööstuskontrollereid, automatiseeris tehaseid, eks ole. Ja olid need 
sardsüsteemid, seal on väiksed mikroprotsessorid neid oli vaja programmeerida 
ja need programmaatorid olid kallid. Ja siis Ektaco hakkaks tegema oma 
programmaatorit. Põhimõtteliselt mingisugune lisaseade PCle, millega sa saad 
neid kivisid kõrvetada. Üks teine tüüp, kes oli nagu riistvara poole peal (ma 
ei tea, aga ma arvan, et ta oli umbes nagu mina, värskelt laekunud staatuses) 
ja mina tegin siis softi. See oli selles mõttes nagu päris huvitav, et meil oli 
PC/AT platvorm, seal oli ISA siin ja selle arvuti me süstemaatiliselt kogu aeg 
ajasime ikka täiesti lukku. Ja selleks, et saaks mingit sotti, siis meil oli 
seal siuke äge asi nagu loogikaanalüsaator. See on niisugune aparaat, et kui 
Ostsilloskoobiga saab visualiseerida mingit analoogsignaali, siis 
loogikaanalüsaatoril on palju-palju pisikesi klemme, mis sa paned kuskile prose 
või mingite digitaalsignaalide külge. Siis sul on teine arvuti mis  
visualiseerib, et kuidas need signaali mustrid on ja siis sa saad panna 
\emph{triggereid}, et kui mul tekib selline muster,  siis salvesta ja 
taasesita. Ehk et kui me ajasime selle selle PC täiesti hulluks, siis me saime 
sealt loogikaanalüsaatori pealt pärast vaadata, et mis siis juhtus, et mis me 
valesti tegime. Ühesõnaga tema siis tegi riista ja kirjutas siis sinna 
kontrolleri peale programmi ja mina kirjutasin PC peale siis põhimõtteliselt 
draiverite programmi vastu, mis omavahel suhtlesid. Ja siis tegin sellele ka 
kasutajaliidest.

Meil olid igasugused Inteli ja IBM-i \emph{manual}id laua peal, neid me siis 
seal sobrasime ja dekodeerisime, et mis me peame nüüd tegema, et siit 
midagigi läbi läheks. 


\question{See kõlab kuidagi hästi süsteemse ja korraldatud ettevõtmisena?}

Ei, see oli hull häkkimine. Nojah, Ektacos seda kraami, mille abil nagu häkkida, 
seda oli ja meil meil oli võimalus seda kasutada. Ja tegelikult ma tõesti selle 
teise tüübi  tausta ei tea, et võib-olla tema oli  kuidagi kogenum, tema tuli 
ju loogikaanalüsaatoriga sinna laua taha. Aga see oli suhteliselt niisugune 
kasulik ja kergesti omandatav seade, et noh kuidas sa seda pruugid. 

\question{Jah, aga võrreldes sellega, kui (nagu on räägitud) inimesed 
vaibanoaga emaplaadi pealt radu maha kratsisid, et modem tööle saada on tegu 
ikka \emph{high-tech} häkkimisega}

No me tegime ikka sinna radu juurde selleks et see kuidagi tööle saada, me ei 
kratsinud midagi maha! Mina ise seda riista-poolt tol ajal ei puutunud. Ehkki 
meil Ektacos programmeerija töövahendite hulgas oli kindlasti tinutus kolb, et 
nii raua lähedal oli seal see enamus sellest elust. 


\question{Kas te saite tööle ka selle kupatuse?}

Ja, loomulikult. Ja siis sellega seoses muidugi, kuna  see oli veel see aeg, et 
 see Borlandi\index{Borland}\sidenote{Borland Software Corporation oli 1983. 
aastal asutatud ja eri nimede all siiani toimetav tarkvaraettevõte, tuntud 
eelkõige arendajate töövahendite poolest. Neist kuulsaimad olid 
\enquote{Turbo-} eesliitega keeled Assembler, BASIC, C, C++, Pascal ning hiljem ka Delphi}
toodang, igasugused Turbo-blaahid, mis neil olid, need olid nagu  standard, eks 
ole. Siis loomulikult sai kirjutatud oma akendussüsteem, mis nägi välja nagu 
see Borlandi Turbo Vision\index{Turbo Vision}\sidenote{Borlandi poolt 
üheksakümnendate alul arendatud tekstipõhine kasutajaliidese raamistik Pascali 
ja C++ jaoks}, aga oli hoopis parem ja teistmoodi tehtud ja seega töötas väga 
kenasti. 

\question{Milles see väljendus, et ta parem oli?}

Ta oli nagu ägedamini struktureeritud. Siis mul hakkas juba 
C++\index{Keeled!C++}  meeldima, ta oli hullult objektorienteeritud. Tal olid 
mingid oma kontseptsioonid, et kuidas sa neid aknaid ja asju  esitad, kuidas sa 
sündmusi käsitled  selles mõttes, et sul on klaviatuur ja hiir. Mingi asi on 
fookuses, kuidas need sündmused jõuavad õige objektini, ja see on  klaviatuuri 
ja hiire puhul väga erinev loogika. Ja kõik see oli selliseks loogiliseks 
kompotiks keeratud, et sinna oli lihtne rakendusi teha. Sellel tükil oligi 
umbes üks programm, mis  seda ägedat raamistiku kasutas, see oli see sama 
programmaatori kasutajaliidese. Aga noh, selles mõttes oli Ektaco väga tore, et 
need tööülesanded ei olnud väga piiravad. Sa võisid ikkagi, ma ei tea, 
kuude või isegi aastate kaupa rahulikult häkkida ja sealt lõpuks tuli mingi asi 
välja. 

\question{Ja teistpidi, ega sul ei olnud neid akende joonistamise asju võtta 
riiulist kümneid?}

Ei, ikka oli. Sedasama Turbo Visionit oleks võinud pruukida ja seal oli 
igasuguseid teeke. Aga kuidagi, mis see siis on, nagu ametiuhkus ei lubanud 
teise mehe akna teeki kasutada. Tuleks ikka enda oma teha, sest et no mis 
mõttes, ma ei oska nüüd parimat akendusteeki teha. 

\question{Sellist suhtumist pannakse tänapäeval pahaks? Või ei panda?}

Seda tehakse teisel tasemel, eks ole. Tasemeid on juurde tulnud, seal 
nokitsetakse hoopis mingisuguste muude asjade juures, aga mina arvan, et see on 
nagu suht paratamatu, et see on hädavajalik, et inimesed heas mõttes 
leiutatakse jalgratast. Teeks asju, mis on juba tehtud, aga teeks teistmoodi, 
teeks paremini. Põhimõtteliselt olid ju opsüsteemid olemas, et mis mõte oli 
seda Linuxit hakata tegema, PC-Unix oli olemas. See oli olemas, et no mis siis 
häda oli sellel SCOl või millel iganes. 


\question{Jah, põhimõtteliselt oleks ju võinud olla, et siiamaani kõik 
kasutaksid sinu aknategijat}

Kindlasti need inimesed, kes on armunud kaheksakümmend korda kakskümmend viis 
teksti ekraanisse, need oleks olnud siiamaani selle andunud kasutajad. 

\question{Mäletan, FoxPro\index{FoxPro} joonistas lausa mingeid varjusid 
akende taha}

Ja, see on loomulik, varjud akendel pidid olema.

\question{Kas seda teie kiibikõrvetajat kasutati väljaspool 
Ektacot\index{Ektaco} ka?}

Need asjaolud muutusid nii kiiresti, et see, mis oli kallis ja kättesaamatu 
kaks aastat tagasi,  kaks aastat hiljem ei olnud enam seda. Ja ma arvan, et 
seda võib-olla tehti mingi üks või kaks eksemplari ja seda pruugiti Ektaco 
siseselt, aga sellest mingit edulugu ei tulnud. Ja see ei olnudki põhitegevus. 
Mina jälle ei tea, eks ole, et miks seda üldse tegema hakati, kas tõesti oli 
siis nii kättesaamatu või lihtsalt oli äge seda teha.  

\question{Jah, kui ma sind kuulan, see ei kõla suurepärase ärina}

Ektaco tegi ju  äri ka. Ja ma pean tunnistama ausalt, et  mind huvitas tollal 
programmeerimine. See, et mida  kolleegid nagu tegid, ma teadsin, aga ma väga 
ei süvenenud sellesse. See oli hästi selline fokusseeritud toimetamine.

\question{Kas tol ajal tekkis mingi kokkupuude arvutisidega ka juba?}

Seal Ektacos oli mul terve hulk toredaid kolleege. Olid 
Tarvi\index[ppl]{Martens, Tarvi}, Heiki Kask\index[ppl]{Kask, Heiki}, Jaak 
Niit\index[ppl]{Niit, Jaak}, Gunnar Valge\index[ppl]{Valge, Gunnar} oli seal 
minuga samas toas, kindlasti oli veel paar-kolm inimest. Ja siis meil oli 
Fido\index{FidoNet} \emph{point}, mis siis tekkis jälle seal Tarvi ja Heiki 
initsiatiivil, minu meelest ennekõike. Me olime alguses Lõvi point. 
Lõvi\index[ppl]{Lõvi|see{Lepp, Andres}}\sidenote{Lõvi, pärisnimega Andres 
Lepp\index[ppl]{Lepp, Andres}, on legendaarne TPI arvuti-mees, paljude meie 
põlvkonna inimeste sõber, teejuht ja eeskuju} oli siis TPI 
Arvutuskeskuses\index{Tallinna Tehnikaülikool!Arvutuskeskus}. Minu jaoks oli ta 
kunn, ma ei tea, mis ta seal tegelikult oli ja siis olime seal Lõvi 
\emph{point}. Jooksutasime seal FrontDoori\index{FrontDoor}\sidenote{FrontDoor 
oli üks populaarsemaid FidoNeti mailereid} ja mida iganes me jooksutasime. 

Ma arvan, et mingil hetkel me \emph{point}i staatusest \emph{upgrade}sime 
ennast \emph{node}ks. 71 oli meie number, julgeks arvata. Ja me helistasime 
kuhugi sisse ka, sest ma mäletan, et ma olen mingisuguse \emph{prompt}i otsas 
rippunud. Ja vaat seda jälle ei tea, et kust ma sain teada, mis käskudega seal 
Unixis\index{Unix} midagi teha. Ja kuidas mingi binaarne fail ära 
\emph{uuencode}da, selleks et ma saaks seda üle terminali endale 
\emph{dump}ida, selle \emph{dump}i salvestada, oma masinast \emph{decode}da ja 
mingit zipi sealt seest kätte saada. Kuidagi ma teadsin seda, kuidagi ma 
mingisuguseid asju imesin. Aga see on jälle niimoodi, et mingid asjad olid nagu 
õhus nagu mingisugused hallitusseene eosed laiali. Nii, kui kusagil pinnase 
sai, kohe läks kasvama. 

\question{Nii mitu sammu selleks, et midagi kätte saada, barjäärid olid jube 
kõrged toona.}

Info ikkagi liikus, see, ma arvan, ei olnud probleem. Küsimus oli ikkagi 
ennekõike riistvaras ja \emph{access}is ja  telefoniliinides ja niisuguses 
kraamis. Modemid olid ju roppkallid asjad, eks ole. Arvutid,kõik oli roppkallis 
välja arvatud aeg. Töö juures õnneks meil mingeid modemid olid, mitte küll 
kõige härjemad. Meil oli mingi 2400 ja MNP5\sidenote{\emph{Microcom Network 
Protocols (MNP)} on perekond (tähistatud numbritega ühest kümneni) 
veaparandusprotokolle, mida sageli kasutati varastes kiiretes (2400 bit/s ja 
rohkem) modemites} oli see meie lagi, millega me seal alguses toimetasime siis. 
Aga kõik olulised asjad liikusid ikka flopide peal, seda ei viitsinud keegi 
ära tõmmata, tõmmati mingeid pisikesi nublakaid. Tollal oli flopiga bussi peale 
minek reaalselt kiirem kui modemiga toimetamine.

\question{Mis sorti materjali te oma nodes hoidsite?}

Point oli meil puhas Fido point. Meil minu meelest küll BBSi ega midagi olnud. 
Meil oli ikkagi sõnumivahetus, \emph{Echomail} ja \emph{Netmail}, ehk siis 
privaatkirjad ja niisugused avalikud foorumid. See oli see, miks me nii-öelda 
suures pildis seda \emph{node}i pidasime. Kui keegi midagi tõmbas, siis ta 
tõmbas enda jaoks ja võib olla jagas  kolleegidega kuidagi midagi aga meil 
mingit sihukest varamut või niisugust ei olnud.

\question{Kellega te neid meile vahetasite, mis uudisgruppe lugesite? Kogukond 
ei olnud ju suur? Lõviga sai ju niisama ka juttu rääkida, ei pidanud kirja 
saatma?}

Mina lugesin põhiliselt \emph{Echomail}i, mul mingisuguseid kirjasõpru, kellega 
mingeid asju seal väga oleks olnud ajada, et tegelikult väga ei ei olnud. Minu 
jaoks oli see lihtsalt nagu foorum, kus sa saad huvitavat ja enamasti ka väga 
humoorikat  sisu. See väljendustase, see, kuidas inimesed, ükskõik mis teemal, 
viitsisid oma mõtteid sõnastada, need iroonia, sarkasm, huumor, kõik need 
tasemed, see oli niivõrd hea tekst valdavas osas, et seda oli  alati lust 
lugeda. Ükskõik mis oli, mingid autofoorumid, mul polnud  sooja ega külma 
nendest autodest. Aga lihtsalt need naljad, need vihjed, see oli lihtsalt hea 
meelelahutus, enamuses. Muidugi seal on ikka programmeerimised ja riistvara ja 
kõik muud teemad ka. See oli kasulik ja naljakas.

\question{No aga skaalal Tolkienist üle autode C++-ni?}

No kõike, absoluutselt. Kogu elu oli seal minu meelest. Seda jaksas tervikuna 
läbi lugeda sest inimesi oli vähe, palju sa ikka seda head kvaliteetset sisu 
suudad toota. Seda  oli vähe tegelikult, mis seal liikus minu meelest.

\question{Ühesõnaga, praeguses mõistes oli võimalik kogu sisuloomel silm peal?}

No sellel, mis Fido \emph{Echomaili} kaudu tuli, jah. Seal kuskil paralleelselt 
hakkasid arenema mingeid \emph{newsgroupid}, ka Eesti omad, millega mina 
alguses eriti ei puutunud  kokku. See oli natukene teine seltskond minu 
meelest, kes seal nii-öelda internetimaailmas hakkas toimetama. 

\question{Need olid kaks eri maailma, nende vahel mingit silda ei olnud?}

Nii ja naa, kontseptsiooni mõttes olid interneti uudisgrupid ja Fido omad 
samad, aga seal olid mingid ebamugavad erisused. Kunagi  hiljem, kui ma 
Ektacost Küberneetika Instituuti läksin\index{Küber|see{Küberneetika 
Instituut}}\index{Küberneetika Instituut|see{Cybernetica}} siis ma tegin oma 
\emph{node} Solarise\index{OS!Solaris} peale. Meil oli seal üks 
SPARC\index{Arvutid!SPARC}\sidenote{\emph{Scalable Processor Architecture 
(SPARC)} on Sun Microsystems'i poolt arendatud RISC-arhitektuur. Sun müüs 
sellele arhitektuurile tuginevaid, siinmail populaarseid, servereid ja 
tööjaamu} server ja siis ma ajasin seal peal käima kogu selle Fido softi. Üks 
venelane oli selle kirjutanud. Ja siis ma tegin \emph{news}i \emph{gateway}, 
mis nagu Fido Echomaili \emph{newsgroup}ideks köitis kahesuunaliselt ja siis 
ühtlasi ka Netmaili siis tavaliseks meiliks köitis. See oli päris popp, ma 
isegi ei mäleta, millal see maha sai võetud. Ma arvan, et seal juhtus see asi, 
et sellele Solarisele oli lõpuks vaja  korralik \emph{upgrade} teha ja siis ma 
ei viitsinud vist enam. Fido oli ära surnud selleks hetkeks ja siis ma tõmbasin 
ta maha. Aga mingil ajal  oli ta hästi popp, mul oli seal, ma arvan, ikkagi 
sadu sadu kliente oli oma personaalse \emph{account}iga seal minu \emph{news}i 
serveri küljes, kellel oli siis nii-öelda kirjutamisõigus Fido gruppidesse. 
Fidos oli see korrapidamine nagu olulisem, seal ei olnud sellist anonüümset 
kasutust, keegi vastutas alati kellegi eest. Keegi  kuskilt kaudu sai 
\emph{access}i ja kui see keegi oli nõme, siis see \emph{access} võeti talt 
ära. Kui ma hakkasin seda asja Newsi \emph{gate}ma, siis ma lubasin sedasama 
teha, eks ole. Ma ei andnud kellelegi Fido gruppidele  kirjutamisõigust, kui ma 
ei teadnud, kes ta on ja ma ei saanud seda \emph{access}i talt ära võtta. 

\question{Aga see on ju, ütleks, autoritaarne?}

See toimis. See oli nagu  endale olulise keskkonna  normaalsena hoidmise 
eeldus. Teistmoodi ei saa. 

\question{Aga mis on \enquote{nõme}?}

No, solvad teisi inimesi, trollid, ütled puhasti, eks ole. See ongi põhiline, 
et kui sa lähed isiklikuks, teed teisele haiget, halba emotsiooni, sihukest 
asja ei ole vaja. See kui sa vaidled, see on okei, seda peab olema, see ongi 
tähtis, eks ole. Aga sa ei tohi  teistele haiget teha. 

\question{Kõlab nagu lihtne, eluterve ja samas fundamentaalne definitsioon. Aga 
kui sa Ektacost ära tulid, kas sa veel õppisid?}

Ei, mu õppimised olid selleks hetkeks õpitud või noh, mitte päris lõpuni 
õpitud, aga ma olin oma inseneridiplomi kätte saanud, vist 1993 või 1992. Sain 
oma kraadiselt kätte. Magistrikraadiks \emph{upgrade}sin ma ta siin natuke 
hiljem. Mina õppisin, viis aastat, sain süsteemiinseneri diplomi, aga pärast 
hakati  kogu seda kraadi värki järjest lahjendama, eks ole. Kui nüüd õppeaastad 
järjest lühenesid, siis \emph{by default} oli mul bakalaureus, aga siis ma 
pidin veel natuke juurde õppima ja tegema magistritöö, ma saaks magistriks. See 
oli kunagi seal 2001. aastal umbes, kui ma selle  ette võtsin. 

\question{Aga tol hetkel sul ei olnud sellist tunnet, nutikas ja usin õppur 
nagu sa olid, et peaks teadusmaailma sukelduma?}

Ega ma seal TPIs ise teadusmaailma suurt kokku ei puutunud. Kuna ma sealt poole 
pealt hakkasin programmeerijana tööd tegema, eks ole, siis see  haaras  
enam-vähem täielikult. Ma arvan, et lõpus läks kas see õppimine natukene 
nigelamaks, sest töö juures oli palju huvitavam ja palju nagu väljakutseid 
pakkuvam. Viimased asjad mis seal Ektacos\index{Ektaco} sai tehtud, oli 
kontrollerite uue sideprotokolli disainimine. Ma olin hullult vaimustatud 
TCP/IPst ja  siis ma trükkisin välja kõik standardid, mis ma sain: TCP, IP, 
Etherneti. Aga kontrollerid on mingi 8051 peal, mis on umbes nagu, nagu väga 
väike. Aga siis ma lugesin need RFCd kõik läbi ja siis ma tegin mingi oma 
sideprotokolli, mis inspireerus siis kõigest: Ethernetist, IPst ja TCPst. Ehkki 
ta ei olnud nagu päris \emph{flow}le orienteeritud, aga pigem  selline 
\emph{datagram}i-põhine protokoll. Sihukesed vanad riistvaraässad Ektacos  olid 
väga nördinud ja solvunud, et mis mõttes ma kirjutan protokolli, mis ei ole 
deterministlik. Mitte \emph{master-slave}, vaid igaüks võib traadi peal 
lobiseda, kui mõte pähe tuleb, ja siis lahendatakse konfliktid ära ja tehakse 
re-transmissioon. Nad olid väga pahased minu katsetuste peale, aga ma arvan, et 
programmeerisin selle lõpuks sinna ära ja ta mingil määral töötas ka. See oli 
päris äge.

\question{Aga mille vahel see protokoll siis käis?}

Põhimõtteliselt oli see, et PC, mis siis juhtis neid tööstusarvuteid. Neil oli 
sihuke karp, mille nimi oli satelliit, mis oli siis tööstuskontroller, millel 
olid igasugused digi- ja analoogsisendid-väljundid, mis kuskil tehases keerasid 
mingit nuppu, et betooni teha või midagi. Ja siis sellel olid mingid
juht-programmid ja neid tuli konfida. Tüüpiline värk, eks ole. Sa pead teadma, mis 
sul tehases toimub. Sa pead käske andma, selleks on mingit võrku vaja. Ja neid 
satelliidikontrollereid võis seal korralikus tehases ikka palju olla. Ja siis 
ta tuli PCsse kokku tõmmata ja ma usun, et keegi kirjutas siis mingit softi 
sinna PC poolele, mis siis neid satelliite siis jälgis ja juhtis.

\question{See kupatus oli päriselt \emph{production}is ja Eesti Vabariigis 
tehti betooni niisuguste seadmetega?}

Jaa. Ma arvan, Palivere Ehitusmaterjalide Tehas\index{Palivere 
Ehitusmaterjalide Tehas} vist oli see, mis oli ära automatiseeritud Ektaco 
poolt ja ma millegipärast arvan, et midagi oli Tallinna 
Veepuhastusjaamas\index{Tallinna Veepuhastusjaam}.Aga seda ma väga kindlalt ei 
tea, aga seal oli neid veel. Neid objekte ikka oli.

Mida mina tegin, see oli järgmine generatsioon, need objektid juba töötasid 
mingisuguse muu protokolli ja mingi muu  tehnika peale, aga kõik see kasvas, 
eks ole. Ja siis algatati uue generatsiooni satelliidi väljatöötamise projekt, 
kus mina siis  protokolli kontributeerisin ja realiseerisin. 

\question{Kui sa võrgundusest juba nii palju teadsid, sind kuhugi interneti 
varasesse maailma ei tõmmatud kaableid vedama või midagi?}

Ei, mulle meeldis programmeerida. Nende muude asjadega ma tegelesin nii palju, 
kui nad olid kasulikud ja vajalikud selleks, et saaks midagi ägedat 
programmeerida. 


\question{Ja Küberis\index{Küber} sai ägedamalt programmeerida?}

Lõpuks jah. Jälle Tarvi\index[ppl]{Martens, Tarvi} kutsus mind sinna. 
Küberneetikasse oli tehtud infotehnoloogia osakond, mis peitis seda infot, et 
tegelikult tegeldi seal infoturbega ja siis oli seal mingi riiklik programm, 
mille eesmärk oli Eesti riigi infoturbe ja krüptograafia vajadusi rahuldada. 

\question{See oli juba enne, kui tekkis AS Cybernetica?}

Jaa, see oli enne seda. Mina läksin sinna 1994, aga see töögrupp tehti 1993, ma 
arvan. Ja siis seal oli terve hulk nutikaid inimesi koos, kes siis  selle 
missiooni elluviimisega tegelesid, et  kompetentsikeskust ehitada.

\question{Kes selle taga oli? Keegi pidi ju selle tellimuse formuleerima, et 
riiklikult on tarvis tegeleda krüpto ja infoturbega?}
Ülo Jaaksoo\index[ppl]{Jaaksoo, Ülo} oli siis Küberneetika 
Instituudi\index{Küberneetika Instituut} direktor. Minu vaates oli tema see, 
kes seda kõike lõi ja korraldas. Kuidas ja  kellega tema läbi rääkis või kust 
see mandaat tuli, seda mina ei oska küll öelda. Aga tema oli jah, kellel see 
visioon  oli, et seda on tarvis. 

\question{Arvestades, kui vähe vajas Eesti riik krüptot ja infoturvet praegu ja 
kui strateegiliselt oluline teema see praegu on, siis sellise visiooni jaoks on 
ju tarvis väga ägedat ettenägemisvõimet?}

No aga kaugemale vaatamine ongi teadlaste ja akadeemikute ülesanne. Kust mujalt 
see tulla saab? 

\question{Visioon visiooniks, mida see töö toona praktiliselt tähendas?}

Esiteks, ise õppida. Teiseks, teisi õpetada. Eestikeelne terminoloogia, 
standardid, profiilid, seminarid, koolitused mida iganes.  Ja  teistpidi 
hakkasid niisugused praktilised asjad tulema. Vaata, tollel ajal maailm oli 
nagu väiksem, ka krüpto ja infoturbemaailm oli väiksem ja mingil hetkel on 
ikkagi veel võimalik hoomata  kõike, mis oli oluline. Mitte küll päris üksi, 
aga sihukese väikese töögrupi sees nagu meil oli. Ma arvan, et mis meil  väga 
hästi läks, oli see, et meil olid inimesed, kes  tegelikult  huvitusid just  
sellest infoturbe süsteemsest poolest. Et mitte see, et mis on nagu see 
tehnika. Aga mis on see organisatsioon, need inimesed, need reeglid, eks ole, 
seadusandlus seal ümber. Ühesõnaga süsteemne lähenemine valdkonnale kui 
tervikule, mis on  väga tähtis ja  mis sellest meie grupist välja kasvas. 

Teiselt poolt oli see, et meil on seal sihukesed \emph{hardcore} häkkerid ja 
\emph{hardcore} krüptograafid, kes nagu olid valmis mida iganes tegema. See 
sümbioos oli minu meelest hästi lahe. Ma arvan, et minu esimene töö 
Küberneetika Instituudis oli see, et ma pidingi riigiasutustele kirjutama 
juhendi, kuidas KA9Q\index{KA9Q} otsas ehitada endale internetti ruuter. 

\question{Mille otsas?}

KA9Q on üks soft. \enquote{KA9Q} on mingi radistide kutsung, mis vastab mingile 
inimesele, kes selle softi kirjutas, on minu arusaamine. Ja see oli DOSi peal 
jooksev \emph{all singing all dancing} asi, mis realiseeris TCP, kõikvõimalikud 
sideprotokollid, võrgukaartide toed, SLIP, PPP, ruuterid, mida iganes. FTP 
deemonid. Täiesti müstilisi asju on tehtud maailmas.  Et kui sul oli üks  
üleliigne PC, modem, võrgukaart ja see soft, siis sa said teha endale ruuteri, 
millega oma organisatsioon kuhugi ära ühendada. Ja siis mina peksin selle käima 
ja kirjutasin eestikeelse lühijuhendi, kuidas seda asja  pruukida, hooldada ja 
nii-öelda käimas hoida. See oli mu esimene nii-öelda, ma ei tea, praktikandi 
töö või mis iganes töö seal Küberis. Aga siis hakkasid igasugused muud asjad
tulema.

Me olime mingis hästi varajases europrojektis, ma mäletan, see võis olla 1995. 
aastal. Ma tean, et ma käisin Darmstadtis\index{Darmstadt}. Sakslased olid 
kirjutanud sellise tarkvara nagu secu-d, mis oli,  ma ei kujuta ette, et ma 
pakun mingi kümme mega haljast C koodi väga halvasti kirjutatud, mis  
realiseeris kogu krüpto, mis tolleks hetkeks oli teada. Kõik sertide töötlus, 
särk-värk. Ja siis me üritasime seda secu-d'd kuidagi rakendada ja kuidagi 
käima peksta. Ütleme niimoodi, et selline \emph{cross-platform} arendus tollal, 
et sul on kood, mida sa kompileerid mingi UNIXi jaoks ja mingi PC jaoks ja siis 
tulid Windowsid, eks ole. Ja teha nii, et see kuidagi enam-vähem  töötab ja 
piisavalt vähe mälu lekib ja piisavalt harva sama mäluplokki kaks korda 
vabastab on  raske ülesanne. Ja siis ma selle secu-d najal ehitasin 
mingisuguseid asju. Turvalist meiliklienti näiteks ja sertifitseerimiskeskust. 
Sertifitseerimiskeskused olid lahedad,  seal mingisugusel  ajaperioodil oli 
see, et me seal Küberneetika Instituudis iga aasta programmeerime vähemalt ühe 
sertifitseerimiskeskus valmis softi mõttes.

\question{Miks?}

See oli mingisugune \emph{blend} sellistest praktilistest vajadustest ja 
teadustöö eesmärkidest. Et üks  sertifitseerimiskeskus, mille me näiteks 
programmeerimine oli näiteks selline. Tollal ei olnud ju mingeid kiipkaarte ja 
riistvaralisi turvamooduleid kätte saada. Ja see oht, et kui sul 
sertifitseerimiskeskuse võti ära 
 kompromiteerub, et siis keegi annab võltssertifikaate välja, see oli suur. Või 
et keegi annab sellele operaatorile altkäemaksu, et annaks võltssertifikaadi 
välja. Sul oleks vaja mitmesilma printsiipi ja sihukest  topeltkaitset. Ja siis 
me realiseerisime selle, et me võitsime selle RSA võtme tükkideks. See on 
seesama, mida praegu SplitKey\index{SplitKey} ja SmartID\index{SmartID} teevad. 
Meil ei olnud küll seda turvalist mitmes osas võtme genereerimist, me lihtsalt 
RSA võtme, jagasime ta osakuteks ja siis meil oli sihuke m-n-ist skeem. 
Selleks, et sertifikaati välja anda, siis viiest operaatorist kolm pidid  
allkirja andma ja siis me kombineeris neist korrektse sertifikaadi kokku. Selle 
nii-öelda initsialiseerimisprotsessi käigus tekitati viis flopit,  millega need 
 operaatorid ringi oleks pidanud käima. Selles mõttes oli ta praktiline, et ta 
töötas,  tegi täitsa korrektseid X.509  sertifikaate ja oli kasutajajuhendiga 
varustatud.  
 
\question{Tundub, et kui sa enne seal ISA siini peal tegelesid väga madala 
taseme asjade katsetamise ja läbi mängimisega, siis nüüd sa tegid sedasama 
krüpto jaoks põhiolemuses olulisi primitiive ja protsesse läbi realiseerides?} 

Jah, et seda võib öelda küll, et mingis mõttes me tegelesime selliste hästi 
\emph{basic} asjadega. Me jõudsime ka rakendusteni välja. Meil oli ka 
hästi-hästi praktilisi asju, aga me kontrollisime tegelikult kogu seda pinu 
ülevalt alla välja. Et sellel ühel hetkel me tegime tulemüüre, mis oli väga 
hästi müüv toode Eesti turul, Barrikaad\index{Barrikaad} oli selle nimi, mul 
siiamaani barrikaadi T-särk alles. Siis me tegime VPN toote, mis oli veel 
ägedam. Selle VPNi teine versioon oli igasugustes Eesti riigiasutustes 
väga-väga pikalt kasutusel ka peale seda, kui selle tugi ametlikult õnnetuseks 
ära lõppes. Ja selle põhieelis oli see, et ta oli projekteeritud hästi 
turvaliseks, keskelt administreeritavaks, eriti töökindlaks. Ehk et see, et sul 
on  harukontorid, kust sa ei taha üldse interneti väljapääsu, vaid tahad läbi 
keskse tulemüüri (mis oli kallis) neid välja juhtida, see oli meil sinna sisse 
ehitatud. Igasugused paralleelsed ruutingud üle erinevate kanalite, eks ole. 
Seal tekivad probleemid, kui sul on VPN tunnel, sul on  sisemised aadressid, 
välimised aadressid, kuidas sa neid majandad niimoodi, et see ruutingu info ka 
seal sisevõrgus korrektselt leviks ja tegelikult ka töötaks. Et kasutajad 
ei peaks  ootama, kuni nende seanss katkisest kanalist tervesse kolib, eks ole, 
et see lihtsalt töötakski. Ja kogu see administreerimine. Meil oli tehtud see 
tükk, mis võimaldas süsteemi konfiguratsiooni muuta, see oli eraldi, see võis 
offlainis olla, see suhtles  muu maailmaga floppide kaudu, see ei olnud võrgus. 
Ja siis oli meil võrgus olev tükk, mis ainult monitooris, kogu sealt infot ja 
täitis neid käske, mis võrgust väljas olev tükk talle  ette pani. Niisugune 
eriti kõrgete turvanõuete jaoks tehtud haldussüsteem. Ja, ja seal me muuhulgas 
siis, kuna tollal ikkagi see PC krüpteerimisvõime oli nõrk, siis me 
realiseerisime ise  šifreid. Tollal just MMXi laiendused tulid prosele välja, 
mis võimalused sul näiteks IDEAt\sidenote{\emph{International Data Encryption 
Algorithm (IDEA)} on esmakordselt 1991. aastal kirjeldatud sümmeetriliste 
võtmetega plokkšiffer} paralleelselt arvutada, mitu plokki korraga. Ja siis 
Helger Lipmaa\index[ppl]{Lipmaa, Helger} oli veel Küberis tööl, kes 
programmeeris siis Linuxi tuuma jaoks MMXi \emph{extension}eid  kasutava 
AESi\sidenote{\emph{Advanced Encryption Standard (AES)} on Belgia 
krüptograafide poolt välja töötatud Rijndael plokkšifri alamhulk. 1997. aastal 
teatas NIST (\emph{National Institute of Standards and Technology of the United 
States (NIST)}) plaanist asendada avaliku protsessi abil tolleks ajaks 
ohtlikult nõrgenenud DES algoritm. Vincent Rijmen ja Joan Daemen esitasid oma 
ettepaneku valikuprotsessi ja see standardiseeriti NISTi poolt 2001. aastal}  
realisatsiooni. Meil seal Linuxi\index{OS!Linux} tuumas olid oma draiverid, mis 
seda VPNi asja haldasid, seal peal olid  oma deemonid võtmete vahetuseks, konfi 
levituseks, kõigeks muuks  ja siis niimoodi hierarhiliselt üles välja.

\question{See, mis sa räägid, et see ei kõla enam nagu programmeerimine, see 
kõlab nagu arhitekti töö. Kas sa liikusid programmeerija rollist arhitekti 
rolli või mõtlesite te neid asju kambakesi välja, kuidas see käis teil?}

Selles mõttes, et välja mõtlesin kogu aeg lihtsalt enamasti oli see teine tüüp, 
kes asju realiseeris,  sellesama peakolu sees. Lihtsalt seal tulid inimesed 
nagu appi. Meil ei olnud  väga selgelt nagu defineeritud rolle, eriti alguses, 
eks ole. Arhitekt, projektijuht, projektijuhid olid üldse väga haruldased 
nähtused, Me ei teadnud isegi, mis projekt on, me lihtsalt programmeerisime 
mingi hetkeni. Meil oli seal, jah, ikkagi terve hulk inimesi, kes arutasid 
intensiivselt praktiliselt kõigil teemadel. Kui asjad olid selged ja siis 
igaüks natukene läks oma  valdkonnas  süvitsi sellega.

\question{Nutikatel inimestel on vahel oma nutikusele vastav ego ka, keegi nina 
püsti ei ajanud ja ennast arhitektiks ei kuulutanud?}

Ei, päris nii ei olnud. Aga ma ise kardan tagantjärgi võib-olla mina ise 
kippusingi see tüüp olema, kes oma  arvamust teistele peale surus. Aga ma tol 
hetkel ei tajunud seda kindlasti niimoodi. 

\question{Ma arvan, et ega teised ka ei tajunud ja soft ju lõpuks ikkagi 
töötas ju}

Absoluutselt. Nii see tulemüür kui ka see VPN, olid meil ikkagi lõpuks ikkagi 
ääretult stabiilsed ja, ma ütleks, kvaliteetset tükid. 

\question{Privador kasvas ka ju sealt välja?}
Jah, Privador\index{Privador} oli siis Küberneetika Aktsiaseltsi spin-off 
firma, mis siis sai need nii-öelda infoturbetooteid, eesmärgiga need laia 
maailma viia, aga see kahjuks ei õnnestunud. Seal oli  kindlasti ports 
probleeme ja üks probleem, mida mina nägin oli see, et tollal hakkasid tekkima 
standardid, et mis asi on VPN, mis asi on standardne VPN. Ja IPSec oli 
enam-vähem ära standardiseeritud, IKE oli ära standardiseeritud ja see oli 
tegelikult see, mida oleks tahetud osta. \emph{Vendor lockin}i juba päris 
mõõdukalt kuni palju kardeti. Ja ehkki meie olime oma asja ehitanud, eriti need 
alumised kihid, need olid  standardite põhjal ehitatud aga mudel,  kuidas me 
nägime seda võrgu tervikut ette ja mida me pidime tegema, selleks, et neid häid 
omadusi saada,  seal tekkisid konfliktid IKE või ütleme, IPSeci, ideoloogiaga 
natukene. Meil  tegelikult oli töölaua peal  versioon kolm VPNist, mis oleks 
siis olnud täiesti standarditega ühilduv, mis loodetavasti selle  firmapärasuse 
probleemi oleks ära kõrvaldanud, aga see kahjuks ei läinud realiseerimisele. 
Selle asemel me tegime digiallkirja tarkvara ja ajatembeldustarkvara ja 
Notariseerimistarkvara ja kõike muud. Me nagu natuke ennustasime valesti, et 
mis on see \emph{killer} rakendus krüptomaailmas järgmise kümne aasta jooksul. 
Olime nagu natuke ajast seest selles mõttes.

\question{See lähenemine, et võtame alumise kihi standardid ja paneme nad 
kuidagi täitsa uut moodi ülemise kihi standarditeks kokku on ju seesama, mis 
sai digiallkirja konteineriga tehtud ja X-Teega ka}

Absoluutselt. Aga vaat seal ongi see, et standardid on ja peavadki olema 
tegelikult geneerilised, eks ole. Nad peavad olema sellised, et nad lahendavad 
paljude inimeste paljusid probleeme, siis nad on elujõulised. Nii. Aga aga kui 
sa võtad ühe konkreetse riigiasutuse, kellel on konkreetsed vajadused, mis ta 
peab ära lahendama efektiivsel viisil, siis sa ei pääse lihtsalt sellega, et sa 
võtad standarditele vastavat tüki ja evitad selle. See ei ole efektiivne. Ja 
see oli siis see, mida meie tegime. Aga seal oligi vaata natukene see, et me 
võib-olla ei tajunud seda, et kui suur see maailm on ja kui võimas ta on ja kui 
suure massiga ja kui kiiresti ta liigub. Me mõtlesime, et me teeme ikka rajult 
ägeda asja. Ja noh, see on nagu \emph{way}  parem ja praktilisem väga suure 
hulga klientide jaoks. Aga see teadmine, et miski asi on hea ja praktiline,  
seda on väga raske efektiivselt ja kiiresti ühest peast teise viia.  

\question{Arvestades, et samast pundist tulid ju ka X-Tee\index{X-Tee} ja 
ID-kaardi kontseptsioon, siis kahest kolm ei ole üldse mitte paha edu protsent}

X-Teega on muidugi see, et X-Tee omab selles meie VPNi tootes väga selgeid 
juuri. Tegelikult, kui me seda X-Teed tegime, see oli 2001.  Mais või juunis 
hakkas asi pihta või isegi natuke hiljem ja detsembris läks tootesse. Eks ole. 
See oli võimalik ainult tänu sellele, et me võtsime oma selle VPN toote kui 
substraadi. Meil oli kõik see olemas, et kuidas me teeme ühe Linuxi purgi 
turvaliseks, kuidas me sellele  Linuxi purgile paneme peale oma tarkvara 
\emph{patch}id, särgid-värgid, kuidas me seda Linuxit konfime, kuidas me hoiame 
konfi niimoodi, et see on efektiivne, kuidas konfi jagamine käib, see kõik oli 
olemas. Me lihtsalt selle asja peale ehitasime ühe natukene teistsuguse 
protokolli vahenduse tüki, eks ole. 

\question{Aga see kõik on natuke hilisem lugu. Kui mina sinuga esimest korda 
kliendina kohtusin, siis sa ikkagi juba juhtisid vägesid. Mina rääkisin oma 
mure ära ja sina tegid nii, et asjad sündisid. Kuidas sul inimeste juhtimine 
rollina esile kerkis ja kas sa üldse mõtestad seda tegevust niimoodi?}

See tekkis Barrikaadi\index{Barrikaad} või VPNi või Privadori\index{Privador} 
programmeerimise käigus, kui meeskond läks suuremaks. Eriti selle VPNi juures, 
ma arvan,  koordineeriv funktsioon oli ikkagi minu peale, et kes nüüd mida 
programmeerib, eks ole, mis ajaks. Ja kes neid asju evitamas käis, ikka meie 
ise, sealt tuli ka see klientidega suhtlus, eks ole. \emph{Helpdesk}, 
projektijuht, arhitekt, programmeerija, testija, tarneinsener, et mu roll oli 
natukene nagu kõik koos. 

\question{Aga ometi kuidagi jäi see koordineeriv roll just sinu peale?}

No ju siis selles pundis see  kõige paremini  mulle sobis, ei oska muud midagi 
arvata. Keegi pidi selle ära tegema, eks ole. Kui see olin mina, siis olin see 
mina, nii see läks.

\question{Ma selle pärast küsin, et ega sul mingisugust kihu ei olnud inimesi 
juhtida?}

Ei. See pigem oligi sedapidi, et, ma nägin seda, mis see asi võiks olla, mida 
me teeme,  päris detailselt päris paljudes aspektides. Ja siis ma nagu tahtsin, 
et see nii läheks, siis ma olin sunnitud  inimestele  ülesandeid või siis 
eesmärke püstitama. See tuli pigem sedapidi, et üksinda ei jaksa kõik ära 
progeda.

\question{Aga see on jällegi arhitekti vaatenurk. Minu peas on olemas täiuslik 
mudel süsteemist ja siis ma teen niimoodi, et see saaks teoks tehtud. Mis sa 
praegu teed?}

Mis sa praegu teed? Väga paljusid erinevaid asju. Ma suhtlen hästi palju 
klientidega ja potentsiaalsete klientidega, et aru saada, mis on  nende  mured 
ja vajadused, kuidas me saame   neid aidata. See on alates müügitööst, projekti 
juhtimiseni. Teistpidi ikkagi see, ütleme, arhitektuurne töö. Kui probleem on  
arusaadav, et mis oleks see lahendus. Ja need probleemid on keerulisemaks ja 
mastaapsemaks läinud. Mõnes mõttes ka vastutusrikkamaks selles mõttes, et me  
ikkagi tegutseme suuresti turvavaldkonnas. Ja see keskkond on nii palju 
vaenulikum ja nii palju keerulisem ja need panused on nii palju suuremad, et sa 
pead lihtsalt palju palju paremaid asju tegema kui me kunagi tegime. Sedasorti 
arhitektuurne  mõtlemine ja siis inimestele nende ideede jagamine. Nõustamine, 
mõnes mõttes ka võiks öelda isegi natukene koolitamise moodi asjad. 

\question{Sa oled kogenud arhitekt ja tead, mida on vaja selleks, et projekt 
välja tuleks. Kuidas sa viid entusiastlikult pihta hakanud meeskonnale kohale 
selle, et sinu arvates projekt ei saa välja tulla? Ja seda nii, et sind pärast 
tuppa tagasi ka lastakse?}

Samm üks on see, et sa pead aru saama. See võtab tegelikult päris kaua aega ja 
see on nagu see koht, kus tihti suhtled väga vähe. Ega seda, et vaatad peale, 
saad kohe aru,  mis valesti on, kuidas peaks olema, seda ei ole. Kõigepealt 
pead probleemist aru saama. Ja võib olla, et  sellepärast see see olukord ongi 
võib-olla keeruline või halb,  et see ongi olemuslikult keeruline probleem. 
Seal on mingisugused mingisugused põhjused, keegi on teinud mingeid otsuseid, 
mingeid probleeme on lahendatud ja selle käigus on tekkinud niisugune asi. Sa 
pead sellest aru saama. Sa ei saa lihtsalt minna, et \emph{sorry}, vanad, et 
siin on jama. Sa pead kõigepealt aru saama, mida on tehtud ja miks on tehtud, 
need probleemid endale selgeks tegema. Ja siis sa tõenäoliselt marineerid nende 
otsas päris kaua ja see ei tule niimoodi, et hops, homme hommikuks on valmis, 
eks ole. Sa mõtled ja kirjutad ja räägid. Ja ehkki tihti on niimoodi, et  sulle 
endale võib tunduda, et lõpuks kui sa mingeid asju hakkad tegema, et selline 
lahendus oli algusest peale selge. Aga kui sa lähed kontrollima fakte, et mida 
sa tegelikult rääkisid, mida sa oled ise kirjutanud, mis sa arvasid, siis 
selgub, et tegelikult see lõplik lahendus on sinu juurde väga suure kaarega 
tulnud. Sa pead selle lihtsalt välja kannatama ja selle ära tegema. Aga, aga 
point on lõpuks see, et kui sa oled jõudnud mingisuguse asjani, millest sa 
näed, et see ongi okei ja lahendab ära  selle probleemi ja selle probleemi ja 
selle probleemi. Võib-olla see lahendus on keeruline ja on kulukas nagu 
realiseerida ja on isegi riskantne aga ta on õige, ta on juba olemuslikult 
õige. Sa saad aru, et mis see probleem olemuslikult on, kuidas seda asja  
tükeldada, kuidas seda keerukust peita, kuidas seda asja üldistada. Ja siis sa 
pead väga kannatlikult väga paljudele inimestele seletama, miks me võiks teha 
just nii. Seda jõuga ei saa teha. Sa pead neid julgustama ja sa pead olema 
valmis nende eest viskuma džotile, juhul kui on vaja. Aga ma ise muidugi usun, 
et ei lähe vaja, või siis sealt džotist ei tule midagi surmavat välja, eks ole.


\chapter{Ahti Heinla}
\index[ppl]{Heinla, Ahti}
\question{Kuidas sa sattusid arvutite juurde?}
Ma tulen sellisest perekonnast, et minu ema ja isa olid mõlemad 
programmeerijad. Nad olid  ülikooli lõpetanud ja  said tööl kokku ka, 
see oli kuskil kuuekümnendate lõpp. See oli  aeg, kus Eestisse tekkisid 
esimesed arvutid, mis sel ajal olid muidugi kapi suurused, aga ikkagi.

\question{Kuuekümnendate lõpus ei saanud neid programmeerijaid ju palju olla?}

Jah, kindlasti neid ei olnud palju, kuigi neid siiski ikkagi täiesti 
mingisugusel määral oli. Minul muide muide oli hiljuti selline asi, et  emal 
oli selline suur juubel, üle seitsmekümne ja niimoodi, ja ta kutsus enda 
kursusekaaslased külla. Ja kes need kursusekaaslased siis on, need on 
rakendusmatemaatikud, praegu siis sellised üle seitsmekümne aastaseid  
inimesed, nii mehed kui naised. Põhimõtteliselt kõik  
professionaalsed programmeerijad olnud, enam-vähem kõik, mehi ja naisi 
võrdselt. Ja, näed, meil on ikkagi asi juba nii kaugel, et meil on juba nagu 
suhteliselt kaugeid põlvkondi, kes on üles kasvanud programmeerijatena. Ja mina 
sündisin kahe sellise inimese järeltulijana.

\question{Kas see on pigem vedamine või vastupidi? Oleks võinud ju ka ära 
hirmutada?}

Mind see kindlasti ära ei hirmutanud, ma kasvasin üles perfolintide vahel. 
Vahetevahel, kuna arvutiaeg oli ju piltlikult öeldes talongidega 
jagatav, arvuti pidi ikka õhtuti töös olema, ema ja isa käisid 
õhtuti tööl, kui nad said arvuti aja kella kaheksaks õhtul. 
Ja siis nad võtsid vahepeal minu ja mu õe 
 kaasa, mina jooksin  arvutikappide vahel ringi ja vaatasin, kuidas seal 
magnetlindid vaikselt käisin nii ja naa ja see oli kindlasti hästi põnev. 
Hoopis teistsugune keskkond ja isegi helid on teistsugused. Vaatad, kuidas 
need masinad seal toimetavad, mingid magnettrumlid vaikselt vihisevad ja 
sahisevad ja kindlasti oli põnev.

\question{Legendid räägivad, et selle põlvkonna rahvas korraldas Ameerikamaal 
lindikappide võidujookse ja muud sellist, tolles sinu arvutiruumis midagi 
sellist ka toimus?}

Mina selliseid asju ei näinud. Ma saan aru, et ka Eestis  tehti selle sel ajal 
sellist  pulli, aga võib-olla  seda tegid natukene nooremad inimesed, kellel 
lapsi ei olnud. Minu isa ja ema olid ikka natuke sihukesed ontlikumad. Nad 
üritasid mingisuguseid konstruktiivseid asju arvutiga teha,  panna neid just 
ühel või teisel moel  käima, aga nad ei olnud sellised, kuidas öelda, häkkerid 
tänapäeva mõistes. Et  mismoodi arvutiga  pead pesta, näiteks, et selliseid 
asju nad ei mõelnud.

\question{Sind ju esialgu ei lastud linte perforeerima? Mis esimene asi oli, 
millele sa ise käed külge said?}

No mu vanemad olid programmeerijad aga mina ei olnud programmeerija, tavaline 
laps nagu ikka. Ma vist olin nagu natuke  matemaatiliselt  andekas, aga 
otseselt arvutitega minu  esimene kokkupuude oli tegelikult ikkagi sellest, kui 
ma olin kümne aastane. Lihtsalt järsku päevapealt  ühel õhtul tuli ema  koju ja 
ütles, et kuule, Ahti,  ma õpetan sulle midagi, istume maha. Istusime maha ja 
ta õpetas mind programmeerima. Kolm õhtut niimoodi õpetas. Ja ma sain selle 
kolme õhtuga tegelikult sellest oast aru, et mismoodi see asi käib. Sealt 
alates  siis hakkasin juba ise edasi mõtlema, proovima, katsetama, lugema, 
natukene lolle küsimusi küsima. Kolm päeva ma olen sellist süstemaatilist 
programmeerimise õpetust saanud.

\question{Mis ta siis rääkis, et see kümneaastasele huvitav oli?}

Eks mind huvitasid sellised asjad kindlasti. Sel ajal polnud ju ka niimoodi,  
aasta oli 1992, et  lapsel on tohutult palju mingisuguseid ahvatlusi 
ümberringi, et Facebookid ja Instagrammid hüppavad siia-sinna ja kõikvõimalikud 
muud asjad käivad. Sel ajal oli ikkagi niimoodi, et ega meil kodus ju telefoni 
näiteks ei olnud. Arvutit ka ei olnud, ma kirjutasin programmi  
alguses ikkagi paberi peale. Need kolm päeva õpet käis paberi 
peal. Ja kui  ema tuli õhtul koju ja sellist asja ütles, siis me ikkagi mitu 
tundi istusime maas, eks ole. Ei olnud niimoodi, et mul oleks kogu aeg telefon 
helisenud ja hüpanud, mingisugune asi, et \enquote{kuule, Ahti, tule nüüd sinna, teeme 
seda}. Selles mõttes võimalik, et ei olnudki nii väga vaja, et see oleks nagu 
hullult kuidagi põnev olnud kümneaastasele lapsele. Mul pigem oligi lõpuks  
põnev see, kui ma sain aru, kuidas see asi töötab.

\question{See peab olema päris korralik ettekujutusvõime, et sa paberi peale 
kirjutades saad aru, kuidas miski asi töötab. Sest paberil ei tööta sul midagi, 
seal on lihtsalt tekst}

Nojah, samas aga eks programmeerimise üks selline  võtmeoskus ongi tegelikult 
ju oskus ette kujutada, et mismoodi  masin  töötab. Lõppkokkuvõttes  
programmeerija ehitab ju masinat. Ja noh, piltlikult öeldes, ikka samasugust 
masinat nagu  mingisugused hammasrattad kuskil käiksid. Üks koodirida on 
piltlikult öeldes üks hammasratas, teine koodirida on mingi kangikene 
kuskil seal, eks ole. Ja kui sa ehitad sihukest füüsilist või mehaanilist 
masinat, siis sa näed, kuidas see töötab, et siin mingi ratas keerab 
ja siis kang liigub ja kuidas siis teine asi kuskilt midagi lükkab ja mingi 
lint või tross kuskilt midagi tõmbab. Sa näed seda kõike füüsiliselt. Ja 
programmeerija peab ka nägema. Aga ta peab nägema seda vaimusilmas, sest seda  
füüsilises maailmas  silmadega ei näe. Ja see vaimusilmas nägemise oskus on 
programmeerijale ülivajalik. Tagantjärele vaadates võib öelda, et eks ema mulle 
seda tegelikult õpetaski see kolm päeva.

\question{Kas \texttt{goto} käib nii- või naapidi või tehete järjekord on 
selline või teine, on teisejärguline.}

Just. Tegelikult oleks põhiliselt vaja teada, et mida \verb|goto| tegelikult teeb 
või et selles masinas, millise hammasratta, millise kujuga asja, see 
\verb|goto| seal teeb.

\question{See kolm päeva tekitas huvi, sa said enam-vähem aru, kuidas arvuti 
töötab, aga mis edasi sai?}

Siis läksime kuskil õhtul emaga  sinna arvuti juurde, ema tööle, ja  
tippisime selle programmi sisse. Kui ma õieti mäletan, siis seda sisse 
tippimist võis juba mitu päeva olla. See oli ikka mitu lehekülge, see minu 
programm ja mõnikord läks midagi valesti ka ja nii edasi. Ema aitas mul siis 
seal mõned vead ära parandada ja  tuli välja, et  tegelikult see programm 
töötas. See lahendas ühte väikest sihukest matemaatika  keerdülesannet, kus  
loogika oli  selles, et kui sul on  näiteks sada ühikut raha ja sa lähed 
raamatupoodi ja sa tahad seda sada ühikut raha ära kulutada. Siis sa pead 
kombineerima, et osta üks raamat, mis maksab viiskümmend seitse ja teine raamat 
nüüd maksab kolmkümmend, selline klassikaline  matemaatika keerdülesanne, 
kuidas kombineerida niimoodi, et kokku saada  summa, mis on võimalikult 
lähedane sajale aga mitte üle selles. Ja sellist ülesannet lahendas see minu 
programm. Ei ole nagu kõige triviaalsem asi, see ei ole nagu päris niimoodi, et 
vajutad nuppu ja programm ütleb lihtsalt \enquote{tere}. Tänapäeval ikkagi 
pigem kõik asjad üritatakse, ka heal põhjusel, ehitada niimoodi, et sul on 
selline nagu hästi kiire rahuldus või et sa nagu näed kaks minutit vaeva ja 
juba midagi hästi väikest nagu töötab ja siis sa näed veel viis minutit vaeva 
ja tuleb veel midagi. Mina pidin kolm päeva vaeva nägema, enne 
kui tulemust oli. Enne seda oli kõik ainult vaimusilmas.  Aga, tõepoolest, kui 
sul kogu aeg Facebook taskus ei hüppa, siis on nagu natuke lihtsam ka seda kolme 
päeva leida. 

\question{Mis tolle arvuti nimi oli?}

Ausalt öelda ma isegi ei mäleta, ei pruukinud isegi nõukogude masin olla,  seal 
oli tegelikult ka lääne aparaate.

Sedasama ühte programmi, mis ma kirjutasin, sai minu meelest 
 isegi  mitmel arvutil käitatud. Et see ei olnudki niimoodi, et 
\enquote{kuule Ahti see on nüüd sinu arvuti, millega nüüd sina  mitu päeva 
tegeled}. See isegi vist nägi niimoodi välja, et ma pool programmi 
tippisin ühel arvutil sisse, mis oli sihuke suur must kapp ja siis järgmisel 
õhtul läksime ühe hoopis  läänelikuma välimusega  nagu nõtkema 
välimusega moodsama asja taha ja tippisin teise osa sisse. Et ma juba sain ka 
natuke kogemusi sellest, et see programm on ikka hoopis midagi muud, see ei ole 
see füüsiline arvuti, millega ma tegelen. Ma võin istuda ühe arvuti taha ja 
siis ma võin minna teisele korrusele teise arvuti taha, mis on terve toa suurune 
ja seesama programm jookseb mõlemas.

\question{Mille peale sa vahepeal kirjutasid selle programmi? Kaartide peale?}

Siis olid ikka juba diskid olemas. Mitte need kolmetollised 
disketid, vaid sihukesed  kaheksa või viie tollised või mingid sellised asjad, 
pigem kaheksatollised ilmselt. Aga kindlasti see esimene programm oli ainult 
selline algus, eks ole, sellest tuli mingisugune  oskus ja huvi asja vastu. 
Edasi hakkasin ise vaatama ja  sattusin kokku juba teiste poistega, 
kes analoogse asja vastu huvi tundsid. Lähemate aastate jooksul hakkasid 
tekkima ka personaalarvutid ja enam ei olnud alati niimoodi, et sa pead õhtul 
tingimata ema töö juurde minema, vaid oli kuskil juba muid kohti ka olemas.

\question{Kust sellised tutvused tekkisid, internetti ju polnud?}

Internetti ei olnud, küll aga  oli olemas näiteks kaheksakümnendatel tekkinud 
selline asi, nagu Raaliklubi\index{Arvutiklubi!Raaliklubi}, mida vedas selline tegelane 
nagu Jaak Loonde\index[ppl]{Loonde, Jaak}. Mina olin ka vahepeal selle klubi liige ja see koondas sihukesi huvilisi poisikesi. Ega mul on raske öelda 
täpselt, ise nagu poisikesena tol ajal süstemaatiliselt ei pannud tähele ja ei 
jätnud meelde ka täpselt, mida täpselt Jaak Loonde tegi ja kas ta üldse midagi 
tegi peale selle, et lihtsalt need poisid kokku tuua. Aga täiesti võimalik, et 
sellest piisabki, et sama huviga poisid kokku tuua ja siis nad 
omavahel juba vahetavad kogemusi, kellel kuskil jälle ema või isa töötab 
kuskil. 

Minul oli näiteks üks niisugune reliikvia, mille ema mulle andis: ta õpetas 
mind kolm päeva ja siis ta andis mulle ühe ingliskeelse raamatu, mis oli 
põhimõtteliselt Pascali programmeerimiskeele õpik. See oli inglise keeles, ehk 
siis ma ei saanud sellest eriti midagi aru. Ma koolis õppisin saksa keelt, 
mitte inglise keelt\sidenote{Tol ajal jagunesid koolid kaheks: kas lisaks vene 
keelele õpetatakse läbivalt inglise või saksa keelt}. Küll aga ma sain aru 
nendest  programmi näidetest, mis seal oli, eks ole, ja seal oli asjad ikkagi 
mingisugused loogilises  järjekorras. Tegelikult, kuigi ma inglise keelt ei 
osanud, ma siiski suutsin sellest raamatust kindlasti midagi õppida ja sealt 
tuli ideid, mida katsetada. Seal oli kuskil mingi, piltlikult öeldes, mingi 
\verb|goto| käsk, oletame. Ma ei teadnud, mis see tähendas, aga ma sain 
selle \verb|goto| käsu  hiljem mingisse arvutisse sisse tippida ja 
vaadata, mis ta teeb ja küsida kelleltki teiselt, et mis see \verb|goto| tähendab. 
See on midagi muud kui lihtsalt öelda, et \enquote{õpeta mulle programmeerimist},  
sul on juba ka konkreetne küsimus. Niimoodi läbi nii-öelda lukuaugu see 
õppimine käis. Internetti ei olnud, aga  inimestel ei olnud ka internetti, siis 
kui nad  lennukeid ja autosid ehitasid, ja sai hakkama.

\question{Seal arvutiklubis sa käisid seepärast, et programmeerimine pakkus 
huvi?}

Jah, mulle pakkus see programmeerimise pool huvi. Mul tegelikult oli niimoodi, 
et isegi enne programmeerimist sattus kätte mingisugune lastele mõeldud 
elektroonika raamat ja ma  natukene nagu harjutasin või mõtlesin selle 
elektroonika peale ka, et kuidas näiteks transistorid töötavad ja muu selline. 
Ja see pakkus ka mulle kindlasti huvi. Aga tagantjärele vaadates  ma 
ütleks, et minu elektroonika tegemine sel ajal oli  ülialgelisel tasemel. Ma 
nii-öelda  kuidagi nagu hästi õrnalt natuke nagu kõditasin elektroonikat ja 
üritasin sellest midagi aru saada, aga samas programmeerimisega ma tegelesin  
tegelikult. Selles mõttes oli seal väga suur vahe ja loomulikult oli ka väga 
suur vahe siis minu  professionaalsuse tasemes, mis  tekkis.

\question{Kas sa oskad öelda, kas see oli pigem eeskuju või midagi muud, mis 
sind pigem programmeerimise poole suunas?}

Üks asi oli kindlasti eeskuju, aga teine asi oli ka kindlalt puhtalt ju see, et 
selle jaoks, et elektroonikaga tegeleda,  on vaja ikkagi mingisuguseid 
teatud füüsilisi asju. Sul on vaja elektroonikakomponente, sul on vaja 
tööriistu ja nii edasi, mida ju ei olnud. Isegi tänapäeval on ju poes  kõik 
olemas, aga sa pead minema ja üldse mitte vähe raha kulutama ja need 
endale ostma. Ma olen natukene ka, hobi korras, elektroonikaga tegelenud, 
tinutan üht-teist ja nii edasi. Ja noh, praegu, kui kõik on nagu 
justkui valla, kõik on olemas ja kahe päevaga tuuakse koju ära, raha ikka kulub 
selle jaoks. Mingi  üks jootekolb ja teine suurendusklaas, takistite 
komplektid, väiksed mikroprotsessorid, igasugused kivid ja sensorid ja andurid 
ja displeid ja nii edasi. Sellega ei ole lihtne alustada. Programmeerimine on 
niimoodi, et sul tegelikult on vaja seda kohta, kus nii-öelda arvutis käia, eks 
ole, paber ja pliiats ja kolm päeva on täitsa piisavad alustamiseks.

Nagu mul sõber ja töökaaslane Jaan Tallinn\index[ppl]{Tallinn, Jaan} on  
öelnud, et programmeerimine on selline naljakas asi, et  enamikes muudes  
valdkondades, kui sa hakkad   õppima, siis sa saad mingisuguse 
algse  taseme kätte ja  pead veel rohkem õppima, et saada järgmisele 
tasemele. Ja sa ei saa iseseisvalt õppida,  
on vaja kedagi, kes õpetab. Kui sa õpid näiteks klaverimängu, siis sul on 
vaja tegelikult, et keegi sulle pidevalt klaverimängu õpetaks, sa ei saa 
ise õppida klaverit mängima. On mingisugused teatud käelised asjad, et 
mismoodi sa seda teed, parimal juhul sa saad või mingist YouTube'i, videost või 
mingist õpikust õppida. Aga sul on vaja seda YouTube'i videod või õpikut. 
Programmeerimine, aga, on tegelikult selline asi, et kui sa oled selle algse oa 
kätte saanud ja sind siis suletakse üksikule saarele aastaks koos arvutiga, siis sa tegelikult suudad ise ilma ühegi õppevahendita õppida 
ennast väga heale tasemele, kui tahad. Puhtalt ise katsetamise,  ise mõtlemise  
teel. Ja eks tegelikult täpselt seda ma tegingi, kui ma teismeline 
olin.

\question{Kas sel ajal hakkas ka juba personaalarvuti moodi arvuteid liikuma?}

Jah, personaalarvuteid hakkas täiesti tulema ja meile koju tekkis ka üks Apple 
II\index{Arvutid!Apple II}. Sellega siis mina hakkasin toimetama, aga see oli 
üsna  kaheksakümnendate lõpus kuskil. Ma ei oska täpselt aastanumbrit öelda, 
aga ju ta võis juba 1988 olla või midagi niimoodi. Ma juba ikkagi  
nagu täiesti oskasin sel ajal programmeerida, ma ei olnud nagu päris enam kümneaastane, olin juba viisteist või kuusteist või midagi niimoodi. Inimestel, 
kes on seitsmeteist ja kaheksateist aasta vanused, on enamasti üsna 
kõvasti nagu meri põlvini  ja peod ja seltsielu ja asjad käivad. Aga minul on 
nagu paar asja teisiti. Esiteks ma olen üldiselt introvertne inimene ja mitte 
üliseltsiv, seltsielu mul kuidagi nii hästi nagu välja ei tulnud. See on 
üks asi. Teine lihtne tõsiasi oli see, et vist kuni viimaste aastateni, umbes 
kolmekümne viienda või neljakümnenda eluaastani oli mul elus selline asi, et kui ma 
joon kaks klaasi veini ära, siis hakkab mul pea valutama. Ma lihtsalt ei pea 
ühel  korralikul peol kaua vastu, ma lähen hiljemalt 
keskööks koju. Ja niimoodi on  kogu aeg olnud ja oli ka siis, kui ma olin 
kaheksateist, eks ole. Aga alates kella kaheteistkümnest ju nagu tegelik 
\emph{action} hakkab pihta, nagu mulle räägitud, ma nagu väga palju ise kogenud 
ei ole. Ja siis ongi, et kui  ülejäänud inimesed avastavad seal seltsielu 
ja pidusid, siis osad avastavad arvutiasju ja avastavad 
seltsielu natukene hiljem lihtsalt.

\question{Mida sa programmeerisid? Sellise jõukohase aga samas huvitava 
ülesande leidmine ei ole ju üldse lihtne?}

No eks poisikesi ikkagi mängud huvitavad üsna palju ja kindlasti ma arvan  
minul ja minu kaasvõitlejatel  kõigil oli ju üks esimesi unistusi, et 
kirjutada oma üks arvutimäng. Sel ajal olid juba  
Yamaha arvutid  tekkinud, eks ole, ja juba ka Apple II peal oli täitsa 
korralikke  mänge olemas. Need kommertsiaalsed mängud olid ikka sellisel 
tasemel, mida üks mingi hobistist poisikene ikkagi nädalaga valmis ei viska. Ja 
ega see tähelepanu ulatus  kolmeteistaastasel või viieteistaastasel ei ole väga 
selline, et suudaks midagi väga palju pikemat ette võtta. Selliseid väga 
lihtsaid mängukesi sai kindlasti ehitatud ja kindlasti ka proovitud üritada 
siis niimoodi häkkerlikult natukene läheneda sellele arvutile. Et mida on 
võimalik arvutit tegema panna, mis hääli on võimalik arvutit tegema panna ja  
igasuguseid lollakaid visuaalseid kujundeid ette manada, ja muu selline. 
See pool kindlasti ka huvitas. 

Aga hiljem, tegelikult teismeeas, sai igasuguseid asju proovitud. Ega täpselt ei 
teadnud, mida võiks  teha, aga valdkond kindlasti huvitas. Järjekindlamalt 
hakkasime mänge programmeerima Jaan Tallinna\index[ppl]{Tallinn, Jaan} ja Priit 
Kasesaluga\index[ppl]{Kasesalu, Priit} kui me olime keskkoolis. Siis oli 
juba niimoodi, et tähelepanu ulatus on inimesel juba nagu natukene 
kasvanud, eks ole, ja võtsime ette ühe mängu kirjutamise projekti. See sai 
natukene pikema vinnaga, et paneme nüüd kõik oma seni õpitud väiksed kogemused 
ja oskused kokku. Ja paneme kohe mitu inimest, mitte niimoodi, et igaüks oma 
nurgas pusib mingit oma mängu, vaid teeme ikka sellise tiimitöö. Jagame 
ülesanded omavahel ära ja kuude viisi töötame selle kallal. 

\question{Kust selline mõtte üldse tuli või arvamus, et selline asi üldse 
võimalik võiks olla?}

Kuskilt iseenesest tuli, ma ei oska täpselt mõelda. Meil isegi ei olnud  
mingit arutelu sel teemal. Lihtsalt sündis, et proovime midagi sellist. 

\question{Mis keskkool see oli?}

Mina õppisin Gustav Adolfi Gümnaasiumis\index{Koolid!Gustav Adolfi Gümnaasium}
\index{Koolid!Gustav Adolfi Gümnaasium|see{Tallinna 1. Keskkool}}
ja Jaan Tallinn\index[ppl]{Tallinn, Jaan} oli minu pinginaaber. Ja Priit 
Kasesalu\index[ppl]{Kasesalu, Priit} oli Jaan Tallinna pinginaaber eelmisest 
koolist, kus Jaan käis. Nii et me olime mõlemad Jaani pinginaabrid olnud. Viimase keskkooliaasta jooksul niimoodi kolmekesi kirjutasimegi ühe mängu, 
millel oli nimeks Kosmonaut\index{Mängud!Kosmonaut}. Mina küll  
kirjutasin seda kui hobiprojekti, aga Jaan ikka ütles, et see asi tuleb teha 
nagu äriks või see asi tuleks maha müüa ja selle eest  raha saada. 

\question{See oli nõukogude aeg ju veel, selle eest võis kinni minna ju?}

Peaaegu. See oli nõukogude aja lõpp küll, sel ajal, kus juba igasuguseid 
metalliärikaid juba juba käis ringi ja niisugune nagu väikene üle piiri  
kaubandus käis ja kooperatiivid ja asjad ja selline värk juba täitsa toimis. Me 
muidugi ei teadnud tuhkagi sellest, kuidas  see  ettevõtluse või selline maailm 
üldse käib. Ja ega tegelikult ei teadnud seda ka need suured inimesed, kes sel 
ajal ettevõtluses olid. Aga mingi tegevus toimus ja metalliäri alal 
mõningase kogemusega või 
sidemetega inimeste abil õnnestus meil tõesti see Kosmonaudi mäng müüa Rootsi. 
See oli selles mõttes muidugi pöördeline sündmus, et me saime selle eest 
lõppkokkuvõttes ikkagi, kui ma õigesti mäletan, siis oli see viis tuhat 
dollarit. See oli täiesti kosmiline number,  aasta oli mingi 1990 ja rubla 
kurss oli seal selline, et vist kui ma õieti mäletan, ühe dollari eest sai 
kolmkümmend rubla juba. Ja kui kokku arvutada, siis see viis tuhat dollarit 
oli ikkagi umbes selline summa, mis meie vanemad olid elu jooksul teeninud või 
midagi umbes sellist. Loomulikult inflatsiooniga võib korrutada ja 
korrigeerida, aga ikkagi oluline number, väga-väga oluline number. Kas 
nüüd mõelda, et kui õigesti me seda summat  kasutasime, kokkuvõttes sai ikkagi 
ka valuutapoes käidud ja Coca-Colat ostetud, selle peale kulus ka ikkagi 
märgatav osa sellest ära. Aga mina ja Jaan ostsime enda endale näiteks kahe 
peale arvuti. Selle peale läks pool minu ja Jaani osast sellest rahast 

Selle raha me saime kätte kuskil, see oli juba üheksakümnendate alguses,  Eesti 
kroon oli just tulnud või tulemas. Sellega on selline lugu, et see oli just 
täpselt see aeg, kus Eesti kroon tuli niimoodi, et minul oli see raha rubladena 
käes. Ja siis oli mingi kooperatiiv või firmakene, kust me siis olime kokku 
lepitud ja välja valitud mingi 386SX protsessoriga arvuti ja me olime seda 
ostmas. Ja ma mäletan, oli see hetk, kus meil oli teada, et järgmine 
päev on rahareform ja minul oli kümnete tuhandete 
kaupa neid arvuti jaoks mõeldud rublasid käes. Meil oli kokku 
lepitud, eks ole, et me anname nii palju rublasid ja saame  
arvuti. See kooperatiivitegelane, kellele helistasin, ütles midagi, et \enquote{too 
homme see raha} või midagi niimoodi. Aga mul ikka nii palju oidu oli, et ma 
ütlesin, et ei, on kokku lepitud, ma toon täna selle raha. Ja ma tõingi täna 
selle raha, ta võttis selle täna vastu ja me saime selle arvuti kätte. 
Nii et jah, ma ei tea, mis oleks 
juhtunud, kui me oleks tegelikult üritanud homme selle raha  maksta. 

\question{Eks ajalugu oleks läinud tonks teistmoodi. Aga see oli juba 386, mis 
oli juba päris korralik aparaat. Sinna vahele jääb ju õige mitu aastat 
puselemist mingite teiste inimeste arvutite juures. Kus te selle mängu 
kirjutasite? Kodus kellegi juures?}

Mängu me kirjutasime suurel määral tegelikult Jaani\index[ppl]{Tallinn, Jaan} 
ja Priidu\index[ppl]{Kasesalu, Priit} töökohas. Sest Jaan ja Priit keskkooli 
kõrvalt töötasid programmeerijatena ühes kooperatiivis. Mina tegelikult ka 
töötasin keskkooli kõrvalt programmeerijana poole kohaga minu vanemate töökohas 
ehk Küberneetika Instituudis\index{Küberneetika Instituut}. Aga  ütleme 
niimoodi, et ma arvan, et  minu vanemate tööandja oli selles mõttes mõistlik. 
Kui ma ise tööandjana mõtlen, et kui mingisugune seitsmeteistaastane poiss 
tahab tööle tulla,  alles õpib programmeerima või niimoodi, et ega esiteks ma 
ei maksaks talle väga palju või ma ei võtaks teda nagunii väga tõsiselt.  
Samuti ma võib-olla ei annaks talle nii palju mingeid võimalusi, ma vast ei 
annaks talle missioonikriitilisi asju. 

Aga Jaan ja Priit olid, olid tööl ühes kooperatiivis, kus nemad olid vaata et 
 peaaegu et juhtprogrammeerijad või midagi niimoodi.  Ja neil oli 
tunduvalt paremad võimalused  käes. Mis on noh, tänapäeval vaadates, ma ütleks, 
ikkagi küllalt ebamõistlik, aga need olidki  ebamõistlikud ajad. See tähendas, et 
nad ei saanud oma arvuteid nii-öelda töölt koju kaasa võtta, aga neil oli 
tegelikult töökoht, kus nad said päeval  olla koolis, aga õhtud-ööd said olla 
arvutis. Ja sel ajal,  kui sa oled kuusteist ja seitseteist, siis võid vastu 
pidada niimoodi, et magad kuus tundi päevas, siis kui vaja.

\question{Kui ma nüüd kokku loen, siis te käisite Gustav Adolfi Gümnaasiumis, 
mis polnud lihtne asi, te töötasite programmeerijatena ja takkapihta 
kirjutasite mängu, mille kannatas pärast maha müüa. Kõike samal ajal?}

Jah, peab ütlema küll,  et vähemalt siis, kui mina  programmeerijana töötasin, 
ma töötajana ei ole uhke tööpanuse üle, mille ma Küberneetika Instituudile 
andsin\index{Küberneetika Instituut}. Tõsi küll,  ma sain ikkagi midagi valmis 
ja mu tööandja oli sellega rahul. Ma ei olnud ka tegelikult ainus, oli natukene 
teisigi selliseid õppijaid ja mõni üliõpilane, kes oli seal niimoodi tööl ja ma 
sain isegi aru, et mu tööandja isegi oli pigem minuga rohkem rahul kui seal 
mõnede teistega. Aga ma arvan, et see ütleb rohkem nende teiste  kui 
minu kohta. Mina ikkagi kulutasin enamiku ajast selle mängu ja koolis käimise 
peale.

\question{Sel ajal hakkasid tekkima esimesed BBS-id ka?}

BBS-id hakkasid tekkima ja minu tutvusringkonnast siis Priit 
Kasesalu\index[ppl]{Kasesalu, Priit} oli see põhiline, kes meie kambas tegeles 
BBS-idega ja ühe ka püsti pani, mille nimi oli \emph{Dark Corner}\index{BBS!Dark 
Corner}, kui ma õigesti mäletan. Ja mille Fido, kuidas seda siis nimetati, 
\emph{node} number või midagi sellist, oli, kui ma õieti mäletan, neliteist. Ja 
teda tõmbas nagu see pool kuidagi rohkem või kuidagi väga palju ja eks 
kindlasti mind ka, sest BBS-iga tekkis järsku  võimalus  ekraani kaudu suhelda 
hästi paljude teiste inimestega, kellega sa võib-olla füüsiliselt ei istu koos. 
Teatud mõttes võiks isegi öelda, et järsku nendele inimestele anti natuke nagu 
Facebook kätte. Mitte taskusse otseselt, aga ikkagi kätte või niimoodi, et 
järsku tekkis hulk sõpru, kellega ma olin suhelnud ainult interneti teel. Fidos vahetati mõtteid  kõikide asjade üle, mitte ainult arvutite üle ja tekkis 
järsku üks mingisugune  täiesti isevärki sotsiaalne seltskond. Tolle aja 
kohta oli see väga isevärki sotsiaalne seltskond. Tänapäeval on niimoodi, et 
sotsiaalne seltskond, kes on mingi Facebookis  grupi, olgu mingi 
MMS-i klubi või ma ei tea mis, liige,  siis nad võivad aeg-ajalt kokku saada. 
Netiajastul on see tegelikult väga-väga tavaline. Aga  selline, kuidas 
öelda elustiil või tutvusringkonna ülesehitus järsku tekkis  meile kätte, kui  
aasta oli umbes 1990 või umbes kuskil sealkandis.

\question{See seltskond pidi siis olema ka teatavas mõttes homogeenne, sest 
Fido külge saamise barjäärid olid kõrged?}

Jah, eks muidugi oli palju ka inimesi, kes nii-öelda jõlkusid kaasas. Olid 
sellised entusiastid nagu näiteks Priit Kasesalu\index[ppl]{Kasesalu, Priit} 
või Tarmo Mamers\index[ppl]{Mamers, Tarmo} näiteks ja no nende muud sõbrad  
aeg-ajalt tekkisid ju ka sinna sisse, kellele siis  Tarmo või Priit võimaldasid 
ligipääsu. Ja see oli kindlasti väga huvitav. Tekkis selline  sotsiaalne 
distants-suhtlus. 

Ma mäletan ühte juhtumit, oli juba tegelikult siis, kui vaikselt Internet juba 
hakkas Eestisse tekkima. Internet kui selline tehniliselt oli ju olemas juba 
 kuskil seitsmekümnendatel või kaheksakümnendatel, aga  Eestisse  ta hakkas umbes sel 
ajal niimoodi natukene  tekkima. Mul oli selline sõber, siiamaani väga hea 
sõber, nimega Sulo Kallas\index[ppl]{Kallas, Sulo}, kellel oli ka BBS ja kes praegu 
töötab minuga koos Starshipis\index{Starship Technologies}. Tema andis 
mulle kasutada ühte oma kontot ühes Unixi arvutis. Ja Unixis oli olemas selline 
programm nagu \verb|talk|, kus omavahel ekraani kaudu said suhelda 
inimesed, kes olid samasse masinasse sisse loginud. Ja ma mäletan, et  minu 
jaoks oli üks ikkagi täiesti selline silmi avav elamus. Mul ei olnud sel ajal 
kodus telefonigi. Midagi ma toimetasin selle Sulo kontoga Sulo nime alt 
selles arvutis ja järsku selle \verb|talk|iga  hakkab minuga keegi 
rääkima.  Ütleb, et minu nimi on Epp. Nii, ja mina siis esimese asjana, kuna ma 
teadsin, et ma kasutan Sulo kontot, eks ole, keegi Epp tahab Suloga rääkida. 
Selgitasin talle, et kuule, mina ei ole Sulo, et mina olen hoopis üks 
teine inimene. Tema ütleb vastu, et  sellest pole midagi, räägime ikka. Ma ei 
saanud täpselt aru, mis värk on nagu, mis mõttes, ta ju tahab Suloga rääkida, 
eks ole. Aga siis ma sain aru, et ta tahab tegelikult lihtsalt kellelegi 
rääkida, et tal  tegelikult on täitsa okei, et ta räägib  minuga. Sihuke 
jutuajamine tekkis sealt, ja ma sain selle jutuajamise käigus teada, et  
tegemist on ühe Eesti tüdrukuga, kelle nimi on Epp ja kes hetkel füüsiliselt 
asub Ameerikas, ta oli ühe Ameerikasse  ülikooli üliõpilane. 
Ja mina istun Eestis, eks ole, ja reaalajas räägin arvuti 
ekraani vahendusel  temaga juttu. Me rääkisime maast ja ilmast 
mingisugune tund aega, see oli  väga-väga kummaline kogemus. Sa  suhtled 
kellegagi reaalajas, kes on  sinust väga-väga kaugel. Ma siiamaani ei tea, 
kes Epp täpselt oli, ta ütles oma perekonnanime ka, ma ei ole seda nime mitte 
kunagi hiljem kuulnud, mitte kunagi hiljem selle inimesega suhelnud. Aga see 
oli ikka väga kummaline kogemus minu jaoks. Ongi naljakas tegelikult, et 
tänapäeval ju selline asi on ju niivõrd tavaline, kõigil mingid Snapchatid ja 
asjad on kuskil taskus, eks ole. Ja tol ajal oli sotsiaalses mõttes see, et sa 
võid suhelda inimestega üle maailma,  oli nendele interneti häkkeritele 
võimalik ja teistele inimestele ei olnud.

\question{Sa ütlesid, et BBS-ides räägiti igasugustel teemadel. Näiteks, 
millest räägiti?}

\label{sisu!inimeseks}Kui ma õieti mäletan, seal oli igasugust, sellist elulist, nagu tänapäeva 
internetifoorumid, eks ole. Kõigest võidakse seal rääkida. Seal oli mingisugune 
filosoofiateemaline  vestlusgrupp, kus  inimesed olid ju enamasti sellised 
kaheksateistaastased, kes alles mõtestavad oma elu. Ongi selline aeg inimeste 
elus, kus kõik mõtlevad, mida  mingisugused asjad tähendavad ja kas ikka inimene 
peaks panustama sellele või tollele. Tänapäeval neljakümneaastasena väga 
võib-olla ei viitsi sel teemal juttu vesta, kõigil on juba oma elu 
tõekspidamised välja kujunenud, aga tol ajal minul kindlasti ei olnud ja enamik 
sellest ülejäänud BBS-i seltskonnast oli ka umbes sama vanad, eks ole. Siis oli 
seal igasuguseid psühholoogiateemalisi, neid vestlusi oli igasuguseid, see 
kindlasti ei olnud sugugi mitte ainult tehnoloogiateemaline. 

\question{See, mis sa ütled, kõlab väga oluliselt. Sest see tähendab, et 
mingisugune ports nutikaid inimesi mitte üksinda ja mitte juhuslike inimestega 
vaid koos sama moodi mõtlevate ja samade oskustega inimestega mõtestasid seda, 
mida tähendab olla inimene kõige laiemas mõttes}

Absoluutselt. See oli tegelikult üks niisugune virtuaalne sõpruskond.  Võib 
olla võib öelda, et see Fido seltskond oli kõige esimene virtuaalne sõpruskond 
Eestis üldse. Tänapäeval on  igaühel virtuaalseid sõpruskondi taskus sada tükki 
aga see võis olla võib olla täiesti esimene.

\question{Kas selle kõige juurde käis ka mingi spetsiifiline raamatu-, muusika- 
või filmihuvi?}

Ahaa, muusikakanaleid, muusikateemalisi  vestlusgruppe,  oli loomulikult ka. 
Aga huvi mõttes, minul ei käinud. Võib-olla natukene. Ma arvan, et  selles ringkonnas pigem olid 
populaarsed sihukesed elektroonilise muusika bändid. Nii, ja naa, ütleme. 
Kraftwerk mulle ei meeldinud ja ei meeldi siiamaani, Jean-Michel Jarre samuti 
mitte nii väga palju aga Tangerine Dream näiteks meeldis mulle väga ja 
siiamaani meeldib, mul on ikka mingi viisteist nende plaati ja nii edasi. Aga 
samas jälle ma olen inimene, kes ei ole kunagi vaadanud Star Warsi, ma ei ole 
kunagi lugenud \emph{Hitchhiker's Guide to The Galaxy}'t. Minu jaoks  on 
esteetiline subkultuur ja arvutid natukene lahus seisnud.

\question{Endal sul BBS-i ei olnud?}

Minul endal BBS-i ei olnud. Ma vist nagu kuidagi ei tahtnud ka, see oli ikka hull 
jahmerdamine, mis selle üleval hoidmiseks vajalik oli. 
Mul oli väga hea meel, et ma sain  Priidu BBS-i kasutada.

\question{Selge. Aga siis te müüsite selle mängu maha, mis edasi sai?}

Noh, kui üheksateistaastasele inimesele anda nii palju raha, nagu tema vanemad 
on kogu elu jooksul teeninud, eks ole, siis tal karjäärivalik on nagu selge 
kohe. Et noh, sellist küsimust nagu ei olnud, et mida ma siis 
tulevikus professionaalselt tegema hakkan. Loomulikult programmeerija.  Ja  mul 
oli ka selline mõtlemine, ma ei tea, kui õigustatud see oli, aga ma arvasin, et 
et noh, eriti üheksakümnendate alguses Eesti ülikoolides eriti midagi väga 
kasulikku sel teemal ei õpetatud. Ma ei tea,  kui õige või vale see on. 
Kindlasti vastas tõele see, et meil keskkoolis oli  arvutiõpetus ka ja üldiselt 
ikkagi meie klassist pigem paljud teadsid rohkem kui meie õpetaja. Ma 
miskipärast oletasin, et ülikooliga oli samamoodi, ei tea, kas see on tõsi või 
mitte. Tänapäeval see kindlasti ei ole tõsi aga  tol ajal  võib-olla pigem oli. 
Igatahes ma tegin selle otsuse, et ma ei lähe ülikooli õppima midagi 
programmeerimise või arvutitega seotut, vaid ma läksin hoopis õppima füüsikat. 
Füüsika oli kindlasti mul  niisugune teine selline huviala,  ma olin 
füüsikaolümpiaadidel käinud  ja mulle see kindlasti kindlasti väga meeldis. 

\question{Mis sulle füüsika juures meeldis?}

No võib-olla natuke sihuke filosoofiline aspekt, et ma sain kuidagi aru, kuidas 
nii-öelda maailm toimib teatud mõttes. See oli põnev. Mingid sihukesed asjad, kui 
 mingid tuumafüüsikad ja mingid planeedid, kuidas liiguvad ja niimoodi, see 
natukene andis võib just sellist filosoofilist mõõdet. Et mis see maailm meie 
ümber on ja kui suured või väikesed meie, inimesed, selles maailmas  oleme.  Ja pigem ikkagi väga väikesed oleme. 

\question{Kuhu sa läksid  füüsikat õppima?}

Ma läksin  füüsikat õppima Tartusse\index{Tartu Ülikool}, koos Jaan 
Tallinnaga\index[ppl]{Tallinn, Jaan}. Pinginaabrid läksid mõlemad õppima 
füüsikat. Sellega läks niimoodi, ma kindlasti  tegelikult ei väärtustanud seda, 
et piltlikult öeldes mul oleks paber taskus, et mul ülikool oleks   
edukalt lõpetanud. Ja kui ma olin ühe või poolteist aastat ülikoolis ära 
olnud, siis mulle hakkas veel rohkem kohale jõudma see, et tegelikult ma ju 
tegelen programmeerimisega kogu aeg, töötan professionaalse programmeerijana. 
Samal ajal tegime järgmist mängu, mille me kavatsesime maha müüa ja nii 
edasi ja nii edasi. Ja ma ei kavatsenud kunagi füüsikuna töötada, ma õppisin hobi 
korras. Kui esimese aasta sai hobi korras  füüsikat õppida, 
siis teisel aastal hakkad aru saama, et tegelikult  õppejõud ikkagi eeldavad, 
et sa tõsiselt tegeled selle asjaga, panustad  füüsika õppimisse enamiku oma ajast. 

Ja siis ma tulin ülikoolist ära. Ma sain aru, et see asi lihtsalt nõuab rohkem 
tööd, kui ma olen nõus sinna sisse parema ja siiamaani ma ülikooli lõpetanud ei 
ole. Jaan Tallinn\index[ppl]{Tallinn, Jaan} käis ülikooli lõpuni ja õppis 
füüsika lõpuni. Tegi oma oma lõputöö, kui ma õieti mäletan, 
relatiivsusteooriast. Sellest, kuidas ruumi painutada selle jaoks, et reisida 
valguse kiirusest suuremate kiirusega ühest kohast teise. Ma küll oletan, et 
tõenäoliselt  ta mingisugust väga suurt teadmist ühiskonnale sellega juurde nii 
väga ei lisanud selle nelja aastaga, mis ta õppis, aga sellise töö ta tegi. Ta 
on rääkinud, et ükskord, kui ta kuskil seltskonnas kirjeldas oma seda tööd, 
mida ta tegi, siis tema vestluskaaslane küsis  vastu, et kas see oli nagu 
rohkem teoreetiline töö või tuli seal ka mingeid praktilisi laboratoorseid 
katseajale.

\question{Selle asja nimi, mida te tol hetkel kampas pidasite, oli juba 
Bluemoon\index{Bluemoon}?}\label{sisu!bluemoon}

Jah. See mängutegijate punt, me hakkasime ennast nimetama nimega Bluemoon 
Software ja Bluemoon Interactive. Inimesed ikka tahavad panna mingisuguseid 
kõlavaid firmanimesid.

\question{Aga miks just Bluemoon?}

Lihtsalt oli üks nimi. Ma arvan, et me ei osanud nimesid üldse välja mõelda ja 
ma olen kogu aeg pidanud ennast väga halvaks nimede väljamõtlejaks ja et ma ei 
valda seda valdkonda üldse ja niimoodi, aga kui Starshipile\index{Starship 
Technologies} nime panin, siis ikkagi osalesin selles kõvasti ja  lõpuks oli 
ikkagi minu pakutud nimi, mis selleks lõpuks sai.

\question{Programmeerimise juures pidi olema täpselt üks raske asi, nimede 
välja mõtlemine}\sidenote{Eksin tsitaadiga. Täpne tsitaat on 
Netscape\index{Netscape} arhitekti Philip Karltoni\index[ppl]{Karlton, Philip} 
poolt ja kõlab nii: \enquote{\emph{There are only two hard things in Computer 
Science: cache invalidation and naming things}}.}

Ma olen täitsa nõus sellega, võib-olla nüüd neljakümneaastasena on juba 
natukene rohkem käppa seda saadud. 

\question{Mis sa praegu teed?}

Praegu ma olen sellises firmas nagu Starship Technologies ja ehitan 
pakiroboteid. Asutasime selle selle firma koos Skype'i\index{Skype} kaasasutaja 
Janus Friisiga\index[ppl]{Friis, Janus}  neli pool aastat 
tagasi\sidenote{Intervjuu Ahtiga toimus jaanuaris 2019}. Ja meil oli selline 
visioon, et asjad võiksid ju maailmas liikuda automaatselt samamoodi, nagu 
elekter tuleb meile stepslisse  ise  ja veevärk on olemas ja 
informatsioon tuleb läbi interneti. Aga asjad liiguvad ikkagi  läbi meie maja 
või korteri ukse, tulevad füüsiliselt kohale ja alati mingisugune inimene toob 
seda, kas sa ise tood või siis sa maksad kellelegi inimesele, kes toob. Ja see 
on hirmus raiskav ja asjad võiksid liikuda automaatselt samamoodi, nagu me 
lennukipileteid broneerime üle interneti nii öelda automaatselt, ilma et me 
läheksime füüsiliselt kohale kuskile reisibüroosse seda lennukipiletit ostma.

\question{Starshipi tegemine on ju juhtimise töö. Kuidas sa jõudsid 
programmeerimise juurest selle töö juurde, mida sa praegu teed ja kui erinevad 
nad sinu jaoks on?}

No need on ikka väga erinevad. Minu jaoks on see areng olnud selline, et ma 
olin programmeerija ja ma olin programmeerija üsna kaua aega, ilma et ma oleks 
üldse midagi kuskil juhtinud. Ja kui me hakkasime startuppe tegema koos Jaanus 
Friisi ja Niklas Zennströmiga\index[ppl]{Zennström, Niklas} siis ma olin 
neis tehnilise arhitekti rollis. Arhitekti roll on juba rohkem 
natukene nagu  juhtimisega seotud, aga sa ei juhi nii väga  inimesi või 
organisatsioone või protsesse, vaid sa juhid just tehnilist arhitektuuri. Et 
milline see masin niisuguses suures plaanis kokku tuleb, mida siis terve suurem 
tiim inimesi ehitab. Nagu maja ehitamisega: osad inimesed ehitavad ja panevad 
kive üksteise peale ja on ka teisi inimesi, kes vaatavad seda projekti 
suuremalt, et kus peaks olema aken ja mitu akent me üldse teeme ja kas 
me teeme rohkem ümmargused aknad või teeme kandilised  ja nii edasi ja nii 
edasi. Ja ma olin Skype'is alguses tehniline peaarhitekt ja 
mitmetest teistes startuppides samuti. Skype'is veel natukene pooleldi juhtisin 
ka ühte väikest tiimi, kus  ma tegelesin sellega, et mõelda umbes viiele 
inimesele välja seda, mida nad tegema peaksid ja koordineerida nende tööd. 
Mõtlesin välja, mis meie eesmärk peaks olema, kuhu poole me peaksime liikuma ja 
nii edasi. Sihukene viieinimeselise tiimi juhtimine oli selline 
nagu väike harjutus või  sissejuhatus, et mingisuguseid kogemusi natukene sain 
või natukene kujutasin ette. Hiljem olen juhtinud siis ka natuke suuremaid 
tiime, umbes kümneinimeselisi ja niimoodi. Aga Starship oli esimene koht, kus 
ma üsna kiiresti, esimese kahe nädalaga, võtsin tööle kümme inimest
ja esimese poole aastaga oli juba umbes kakskümmend inimest meil tööl ja 
nii edasi  läks juba natukene suuremaks see asi. Eks ma niimoodi käigu pealt 
natukene siis  õppisin, et  kuidas juhtimine käib. Ju ma olen kindlasti veel 
üsna  alguses, et me oleme siin Starshipis olnud sihukeses  naljakas 
olukorras, kus nagu juhtimises ikkagi üsna kogenematu juht on olnud sellel 
firmal. Neli aastat ma olin tegevjuht ja  nüüd jõudis pool aastat tagasi siis 
asi nii kaugele, et me palkasime  professionaalse tegevjuhi Lex 
Bayeri\index[ppl]{Bayer, Lex} Californiast. Ja mina olen CTO ehk 
tehnikadirektor, kus ka peab üsna palju juhtima, aga nüüd enam mitte kahtsadat 
inimest, vaid natukene väiksemat hulka inimesi.

\question{See on siis olnud pikk ja just vajadusest ja huvist kantud õppimine?}

Jah, absoluutselt.  Üldiselt ma ütleks niimoodi, et paljud programmeerijad, 
kaasa arvatud ka mina, meile programmeerimine meeldib nii palju, see on niivõrd 
tore  ja niivõrd äge tegevus, et selliseid masinaid ehitada, et tahaks  
muudkui ehitada neid masinaid. Inimeste juhtimine on pigem selline asi, mida 
enamik programmeerijaid väga ei taha teha ja ma ei ole päris kindel ka ise, kui 
palju mina seda tegelikult teha tahan. Aga küll on lihtsalt asi selles, et kui 
sa oled  üksikprogrammeerija ja sul kogemus tekib ja sa oled  arhitekt siis sa 
oskad juba rohkem  arvata, mismoodi me seda tarkvara peaksime ehitama ja mis 
asjad on selle juures olulised ja mis need ei ole. Siis on nagu on kaks võimalust. 
Kas sa  oled vait ja osaled selles protsessis  kellegi teise juhtimisel või siis 
sa üha rohkem nagu vaatad, et ei, ma teen ise, ma teeksin seda paremini 
kui see juht, kes meil on. Ma tahaks ise seda asja juhtida või mul on juba nii 
hea ettekujutus, kuidas seda teha, et ma ei suuda pealt vaadata, kui 
mingisugune teine inimene, kes on võib-olla väiksema kogemusega kui mina,  
kuidagi seda asja juhib ja mitte selles suunas, kus mina olen täiesti 
veendunud, et  õige oleks. Ehk siis see on tulnud justkui nagu vajadusest. Kui 
sa oled üksikprogrammeerija, siis sa aja jooksul ikkagi saad aru, et sa saad 
tegelikult lõppkokkuvõttes rohkem tehtud, kui sa piltlikult öeldes palkad 
endale tiimi ja hakkad juhtima mingisuguseid suuremaid seltskondi. 

Minu jaoks küll see nii-öelda raketiga lendamine nagu Starshipis, kümneinimeselise 
tiimi juhtimisest kuni selleni, et ma tükk aega juhtisin üle kahesaja inimesega firmat, võttis ikkagi pea ringi käima. Et ma kindlasti 
edutasin ennast  oma ebakompetentsuse tasemele. Aga eks kohati öeldaksegi, et 
starupid ongi asjad, mis  väga sageli on  klassikalise sellise juhtimise 
distsipliini ja teooria ja juhtimispraktikate mõttes väga halvasti juhitud 
organisatsioonid. Mis ei ole siiski tihti takistuseks olnud nende edule, 
sellepärast et nad on olnud nii piisavalt värske mõtlemisega, nende toode on 
olnud piisavalt selline värske ja revolutsiooniline, et sellest ei ole olnud 
hullu, et nad on olnud halvasti juhitud. Tegelikult ikkagi need kakssada 
inimest, kes meie Starshipis töötavad,  ma ikkagi vaatan nende peale küll nagu 
niimoodi, et palun vabandust nende ees, et nad on osalenud sellises loomkatses, 
et ma olen neid mitu aastat juhtinud. See ei ole võib-olla olnud aus nende 
suhtes. Aga samas nad ei ole ka sugugi mitte meil siin firmast minema jooksnud 
ja tunduvad olevat rahul, et võib olla väga hullusti siis ei olegi läinud.


\chapter{Madis Kaal}
\label{cptr:mast}
\index[ppl]{Kaal, Madis}
\index[ppl]{Mast|see{Kaal, Madis}}

\question{Kuidas arvuti Saaremaale sai?}

Arvuti ei saanudki Saaremaale. Minu esimene kokkupuude päris arvutitega oli 
Rahvamajanduse Saavutuste Näitusel\sidenote{Tänapäeva mõistes oli tegu 
messikeskusega, kus ajutistel või püsinäitustel demonstreeriti 
kas liiduvabariigi (nagu Tallinnas asunud näituse puhul) või kogu Nõukogude 
Liidu majanduslikku võimekust. NSVLi Rahvamajanduse Saavutuste Näitusest arenes 
välja Eesti Näituste Messikeskus.}, praeguses Pirita näitusehallis. Käisin 
seal koos oma emaga. Ühes nurgas olid üles pandud 
terminalid, mida manageeris kaks imeilusat tüdrukut. Seda siis ajaloolisest perspektiivist, tõenäoliselt oli tegemist üsna keskmiste 
operaatoritega, aga siis tundusid nad imeilusad ja targad. Terminalide peal oli 
nõukogudeaegne venekeelne raamatukogude otsingu andmebaas. Terminalid ise olid ka 
venekeelsed. See oli esimene kord, kui ma reaalselt nägin, et ekraanil olid 
tähed ja klaviatuuril sai kirjutada. 

\question{Mis aastal see oli?}

Arvatavasti 1983. Ja need terminalid jätsid kustumatu mulje. 

\question{Kas pärast seda tekkis sul selge soov terminalide 
juurde pääseda?}

Pärast seda tekkis väga selge mõte, et see asi huvitab mind. 
Seejärel sattusin Tartusse ja ostsin sealt venekeelse 
raamatu \enquote{Programmeerimine keeles PL/I\index{PL/I}} ning lugesin 
seda. Ma ei teadnud arvutitest veel midagi, aga tasapisi hakkas selgeks saama, 
misasi on programmeerimine ja näiteks \verb|for|. See oli mingi imeline struktuurkeel, mitte päris vene, 
vaid kõlas nagu piraatversioon.

Järgmine kord nägin arvuteid Tehnikaülikooli\index{Tallinna 
Tehnikaülikool}, tolleaegse TPI\index{TPI} lahtiste uste päeval, kus me käisime 
pinginaabriga, kellega koos pärast ka kooli sisse astusime. 
Meile tehti ekskursioon automaatikateaduskonna kõigis 
kateedrites\index{Tallinna Tehnikaülikool!Automaatikateaduskond} neljal 
korrusel ja mõnes kohas olid arvutid. Mäletan selgelt, et Indrek 
Saul\index[ppl]{Saul, Indrek}, kes oli minu meelest sel ajal tudeng ja hiljem 
kinnisvaraärimees, näitas meile analoogarvutit. Sellega
sai analoogpingete ja skeemiga diferentsiaalvõrrandeid lahendada.

\question{Vanasti sihiti ju õhutõrjekahureid analoogarvutitega.}

See masin võis täiesti olla sedalaadi projekti osa. Igatahes mul tekkis kindel soov seda valdkonda
õppima minna, aga pinginaaber veenis mind ümber, et lähme parem 
raadiotehnikasse, ikkagi sama maja.

\question{Kas esimest korda arvuti nägemise ja ülikooli sisseastumise vahele jäi veel 
midagi arvutitega tegelemise mõttes?}

Ainult see üks raamat. Otsus arvuteid õppima minna sündis esimesel korral ja raamat tuli 
pärast seda. Ainuke imelik asi oli otsus raadiotehnikasse 
minna, aga selle vea parandasin ruttu ära. Ülikooli teise korruse otsas oli arvutussaal, kus oli kaks või 
kolm SM-4\index{SM EVM!SM-4}. Need olid PDP-11\index{PDP-11} vene 
versioonid. Pärast seda, kui sain aru, kuidas sinna sisse saab, ma enam 
tundidesse ei jõudnud. Ja kuna olin maalt tulnud poiss ja raha ka üldse ei olnud, 
käisin lihtsalt kõik kateedrid läbi ja küsisin iga ukse vahelt, kas neil on tööd anda. Raadiotehnika kateedris\index{Tallinna 
Tehnikaülikool!Automaatikateaduskond!Raadiotehnika kateeder}\label{sisu!mast_raadiotehnikas} oli, 
mind võeti sinna laborandiks tööle ja nii see läks. Kool jäi pooleli, kateedrisse
jäin seitsmeks aastaks paika.

\question{Mitmendal kursusel kool pooleli jäi?}

Esimesel kursusel. Algul olin raadiotehnika kateedris laborant ja pärast 
tehnik. Sattusin tuppa, kus olid väga toredad inimesed: Mart 
Palmas\index[ppl]{Palmas, Mart}, kes õpetas mulle peaaegu kõike, mida ma 
programmeerimisest tean, ja Villem 
Vannas\index[ppl]{Vannas, Villem}, kes praegu töötab Datelis\index{Datel}. Tema 
õpetas mulle enam-vähem kõike, mida ma rauast tean.

\question{Siis ei jäänud ju haridus pooleli.}

Formaalselt siiski jäi. Tol ajal oli
laborant rohkem nagu abitööline. 
Parandasin seda, mida vaja, aitasin seal, kus vaja. Mu esimene töö oli 
kolikamber tühjaks tõsta.
Algusaegadel oli üsna suvalisi projekte, hiljem tekkis
suund kommunikatsiooni poole, mis tundus mulle sel ajal huvitav. 

1990. aasta paiku tekkis Eestis 
mitu huvitavat suunda. Kõigepealt hakkas tulema 
personaalarvuteid. Sinnasamasse, kus oli kunagi SM-4 arvutiklass, tekkisid 
personaalarvutite klassid. Neid oli mitu tükki ja erinevate portsudena 
toodi Austraaliast MicroBeesid\index{MicroBee}\sidenote{1982. aastal 
Austraalias algselt komponentide komplektina müügile tulnud koduarvuti. Tuntud 
huvitava videolahenduse ja patareitoitel mälu poolest, mis 
võimaldas arvutit teisaldada mälu seisu kaotamata.}. Kuskilt tuli terve klassi jagu MSXi 
arvuteid\index{Yamaha MSX} ja siis mõned 
Robotronid\index{Robotron}\sidenote{Robotron (originaalis VEB Kombinat 
Robotron) oli Ida-Saksamaa suurim arvutitootja.}. 
Raadiotehnika kateedris oli juba siis, kui mina sinna sattusin, olemas 
Apple II\index{Apple II} ja mõned aastad hiljem tekkis sinna IBM 
PC\index{IBM PC}. See oli omapärane kogemus. Apple II peal olid 
harjunud, et lülitad sisse ja pilt on ees. IBMi sisse lülitades ei juhtunud midagi. Ühel hommikul tööle tulles vaatasin, et uus 
arvuti, ja lülitasin sisse. Midagi ei juhtunud. Ootasin natuke aega ja lülitasin välja, ise 
tegin näo, et midagi pole toimunud. Hiljem selgus, et masin tegi \emph{self 
test}'i. Seal oli tublisti mälu sees ja testimine võttis palju aega -- ma ei 
suutnud nii kaua oodata. 

\question{Midagi pidi see ju ekraanil senikaua näitama?}

IBMil oli roheline long-fosfor\sidenote{Katoodkiirtel põhinevates monitorides suunati laetud osakeste kiir fosforühendiga kaetud ekraanile. Kasutatud ühendi tüübist sõltus nii elektronkiire mõjul tekkinud värv kui ka see, kui kauaks ekraan peale kiire edasi liikumist helendama jäi. Selle viimase järgi liigitataksegi ekraanides kasutatavaid fosforühendeid  \enquote{pikkadeks} ja \enquote{lühikesteks} (ingl. \emph{long} ja \emph{short}).} monitor, mis läks tükk 
aega käima, ja ma ei jõudnud esimese \emph{boot}'imise ajal ära oodata, millal 
midagi toimuma hakkab.

Üheksakümnendate paiku tekkis meile tuhande kahesajane modem, mis läks 
PC sisse. Sel ajal olid just tulnud esimesed BBSid ja umbes samal ajal otsustas TPI 
automaatikateaduskond\index{Tallinna Tehnikaülikool!Automaatikateaduskond} 
ehitada arvutivõrgu. Toodi kohale viiesajameetrine kaablirull 
kollast sõrmejämedust Etherneti kaablit ja umbes kümmekond komplekti 
kobakaid kaabli peale, mille külge käis teine jäme kaabel, 
mis läks võrgukaardi taha. See oli nagu esimene Etherneti tehnoloogia. 
Mäletan selgelt, et meile toodi ainult kaabel ja kobakad, ei mingeid tööriistu, pistikuid ega terminaatoreid.

Kateedris oli sel ajal eterniiditahvlitest lagi, mille peale me selle Etherneti kaabli tõmbasime. Et kobakad külge saada, tegime
naaskliga kaabli kesta sisse augud, ajasime nõela läbi ja ühendasime
arvutite külge ning tinutasime otsa terminaatorid ja takistid. 

\question{Tarmo Mamers\index[ppl]{Mamers, Tarmo} rääkis, kuidas te PC ja Maci 
vahele traati vedasite. Kas too kaabeldamine oli enne või pärast seda?}

See oli meil kahe PC -- sellesama raadiotehnika kateedri PC ja Tarmo oma -- vahel, Tarmol oli 
veidi vägevam AT arvuti. Ühendasime need 
kaabliga ja tegime väikese 
\emph{chat}'i programmi, et teineteisega suhelda. 

Arvutivõrk tekkis sellest hiljem. Tehnikaülikooli toodi Novell 2.15\index{Novell} 
server, mille ma installisin ja mis oli üks esimesi väheseid asju, millel oli manuaal 
olemas, nii et kõik oli justnagu ametlik. Novelli serveri peal panin käima Pegasuse 
Maili\index{Pegasus Mail}-nimelise asja, kuhu külge kirjutasin \emph{gateway}, 
millega sai UUCP meili, mida toimetati Küberi 
majja Soomest (ma ei mäleta, kas Soomest siiapoole lükates või siit 
üle telefoniliini tõmmates). Tõmbasime selle oma pisikese modemiga 
Tehnikaülikooli majja ja jagasime kasutajate vahel laiali.

\question{Siin tundub jälle suuremat sorti lünk olema selle vahel, kuidas sulle 
hakati programmeerimist õpetama ja kuidas sa naaskliga kaablit torkisid ja 
\emph{gateway}'sid programmeerisid.}

Mõned aastad tuli õppida asjade kirjutamist lihtsalt erinevaid asju tehes ja ehitades, aega katsetamiseks oli palju. 
Olin noor inimene, peret polnud ja praktiliselt elasin raadiotehnika 
kateedris\index{Tallinna Tehnikaülikool!Automaatikateaduskond!Raadiotehnika 
kateeder}. Meil oli seal omamoodi seltskond: arvutussaali 
kamp, Tarmo\index[ppl]{Mamers, Tarmo} kohe kõrval sama 
koridori peal ja mina üleval raadiotehnikas. Vana kooli mees 
Lõvi\index[ppl]{Lõvi} oli kõrvalkorpuses ja käis aeg-ajalt Apple II peal 
oma projekte arendamas.

\question{Kas meetodiks oli siis peamiselt katsetamine, mitte 
manuaalide tudeerimine?}

Manuaale ega dokumentatsiooni ei olnud üldse. Riiklikul 
tasemel tarkvara varastamise programm pakkus küll ägedat tarkvara, aga 
enamasti ilma dokumentatsioonita. See oli nagu infovaakumis 
tegutsemine ja disassembler\sidenote{Programm, mis teeb masinkoodist 
oluliselt loetavamat Assemblerit.} oli justkui sõber.

\question{Keegi pidi sulle ju ometi ütlema, et selline asi nagu disassembler 
on olemas.}

Jaa, seda tegid head vanemad kolleegid, kes hoidsid kätt ja 
juhendasid. Lõviga\index[ppl]{Lõvi} tegutsesime pikalt koos, temal oli kindlasti 
väga suur mõju minu arengule. Aga see lünk, kuidas ma BBSideni 
jõudsin, sai täidetud nii, et mul oli raadiotehnika kateedris\index{Tallinna 
Tehnikaülikool!Automaatikateaduskond!Raadiotehnika kateeder} 
arvuti, mille sees oli modem ja millega sai helistada. Lähim BBS
asus Küberneetika Instituudi otsas, kus tollal asus 
Proekspert\index{Proekspert} ja kus nüüd on Tehnopoli kontor. Andrus Suitsu\index[ppl]{Suitsu, Andrus} 
oli BBSi mees, käisin tema juures oskusteavet ja tarkvara 
hankimas. Panin algul BBSi ja peatselt pärast seda ka Fido, algul vist
\emph{point}'i, ja käitasin seda üsna mitu aastat. 

\question{Miks sa seda tegid?}

Huvist kommunikatsiooni vastu.

\question{Kas sa mõtled kommunikatsiooni masinate või inimeste vahel?}

Mõlemat. See moment, kui täielikust infopuudusest saab järsku täielik 
infovabadus, on väga ergastav. Tänapäeva inimestel, kellel on internet olemas, ei 
kujuta ette, kuidas saab olla ilma, aga ilma oli väga pime.

Üks asi oli tehniline info, aga Fidoneti ja Useneti grupid
(UUCP meiliga koos toodi ka Useneti gruppe) olid ka muidu väga 
huvitavad. Sealsed diskussioonid olid väga 
informatiivsed. Suurem osa 
juttudest olid muidugi tehnilised, sest seal käisid tehnikud ehk need, kes 
said kanalile ligi.

\question{Kas too kollase kaabliga võrk hakkas tööle ka?}

Ikka, see töötas uhkelt. Novelli server käis veel 1992. aastal, kui ma 
sealt ära läksin. Inimesed said omavahel meilida ja ka välismaailmaga 
suhelda. Ainukene probleem oli see, et arvuteid, millel oli see 
Etherneti äge \emph{interface}, oli suhteliselt vähe, paar tükki kateedri 
peale vist suudeti tekitada.

\question{Kas Etherneti kaart oli defitsiit?}

Tol ajal oli kõik defitsiit, siis oli veel rublaaeg. Millise projekti 
raames see toodi, ei tea. Avo Ots\index[ppl]{Ots, Avo} tegi minu meelest 
doktoritöö selle kohta, kuidas ehitada arvutivõrku. See oli 
oluline kogemus, et toimuks järgmine samm. Pärast tehnikaülikooli 
töötasin lühikest aega Microlinkis\index{Microlink}, kus ma olingi 
arvutivõrkude installeerija ja ühtlasi .EXE\index{.EXE} kirjutaja.

\question{Miks sa sinna läksid?}

Ühel päeval astus uksest sisse Margus 
Kliimask\index[ppl]{Kliimask, Margus}, keda ma teadsin Rainer 
Nõlvaku\index[ppl]{Nõlvak, Rainer} kaudu, ja tegi ettepaneku hakata 
tegema ajakirja. Sellest sai .EXE.

\question{Miks ikkagi? Jälle kõlab suure muutusena, et ühel päeval tõmbasid kollast 
kaablit ja järgmisel päeval tegid ajakirja.}

Täpselt nii oligi. Ma arvan, et Rainer tahtis Microlinki promo teha. 
See võis olla suur motivaator, aga seda peab Rainerilt endalt küsima.

\question{Kust sul üldse tuli mõte, et ajakirja tegemine võiks huvi pakkuda?}

Tundus huvitav. Mul ei olnud siis rohkem kõrgeid eesmärke kui see, et elu oleks 
huvitav.

\question{See on tegelikult kõige kõrgem eesmärk, mis üldse saab olla.}

Algul oli jutt, et teeme ajakirja, ja siis selgus, 
et mul oleks ka uut töökohta vaja. Nii sattusingi korraks Microlinki\index{Microlink}. Olin seal aga loetud kuud, sest 
siis hakati tegema Eesti Forekspanka\index{Eesti Forekspank}\sidenote{Eesti 
Forekspank sündis 1992. aastal ja ühines 1995. aastal Raepangaga\index{Raepank} 
1995.}. Pangal olid oma sidevajadused ja mind kutsuti sinna tööle.

\question{Üheksakümnendate algus oli Eesti panganduses ju hull aeg!}

Jah, ja Forekspank oli sel ajal pisikene valuutavahetuskontor, mis opereeris 
rubla-dollari börsi.
See tegutses tolleaegses hulgifirmas Abestok\index{Abestok}. Selle ühes toas olid 
inimesed, kes otsustasid panga teha. Margus 
Kliimask\index[ppl]{Kliimask, Margus} oli nendega seotud, vist IT-poisi 
staatuses. Temaga läksimegi Rävala puiesteele, istusime koos 
ehitusjuhiga ühte tuppa, mille ühes nurgas hoidsid
ehitajad oma tööriistu, ja ehitasime panka.

\question{Kust tekkis mõte, et panga tegemiseks ei piisa kilekottidega 
sularaha edasi-tagasi lohistamisest?}

Need mehed, kes panga tegid, olid piisavalt targad, mõistmaks, et pank käib 
teistmoodi. Kui palju teistmoodi, sai alles siis selgeks, kui 
Inglismaalt osteti pangatarkvara ja konsultandid rääkisid, kuidas panka 
tehakse. Aga see ei olnud kohe esimesel aastal. Esimestel aastatel ehitasime, 
tõmbasime kaablit ja panime laua alla püsti serveri. Ühel ilusal päeval lükkas
Margus Kliimask\index[ppl]{Kliimask, Margus} kogemata varbaga 
toite välja ja pank jäi seisma. Aga mitte kauaks. 

Nii Rein Usin\index[ppl]{Usin, Rein}, Ivar Lukk\index[ppl]{Lukk, Ivar} kui ka Margus Kliimask\index[ppl]{Kliimask, Margus} olid 
visiooniga inimesed. See pidi olema suhteliselt algusaastatel -- BBSid ja 
Fidonet olid siis veel kuum teema --, kui Margus Kliimask ütles, et teeme 
modemipanga. Tal oli kindel mõte, et see peab olema Norton 
Commanderi\index{Norton Commander} F2 menüüs\sidenote{1986. aastal turule 
tulnud ja 1998. aastal viimase versiooni saanud Norton Commander oli 
ülipopulaarne failihaldur MS-DOSi platvormile. Ekraanil oli korraga kaks 
nimekirja faile ja käsurida, allservas nimekiri saadaolevatest 
klahvivajutusega käivitatavatest käskudest. Nii oli kasutajal ilma suurema 
koolituseta kohe selge, mida ja kuidas teha. Ohtralt kasutati F-klahve 
ja neist olulisemate funktsioonid on inimestel siiani peas (F3 -- faili sisu 
vaatamine, F5 -- faili kopeerimine).}. Kõik kasutasid Norton Commanderit 
ja kõigil oli see olemas, aga keegi ei ostnud, sest tol ajal tarkvara ei ostetud. 

\question{Jah, ma mäletan poes karpe, aga ei mäleta, et keegi oleks neid kunagi 
ostnud.}

Hämmastav oli see, kuidas mõtte väljakäimisest 
modemipanga \emph{launch}'ini läks umbes kaks kuud.

\question{Tegite kahe kuuga nullist modemipanga?}

See oli programm, mis oli mingil määral Norton Commanderiga integreeritud: 
läks sealt menüüst käima, nägi välja nagu Norton Commanderi 
osa, võimaldas makseid ette valmistada, kontoväljavõtteid ja panga teateid 
saada ning enda makseid panka saata.

\question{Ja teisel pool võttis mingi asi kõned vastu, suhtles panga 
tuumaga ja tegi arveldused ära?}

Just. Panga tuumaga suhtlemine oli üsna traagelniitidega asi, kuna selleks 
ajaks oli juba toodud Inglismaalt panga tarkvara, millel ei olnud ühtegi head 
liidest peale terminali.

\question{Ja siis tegite terminali emulaatori?}

Mina jah kirjutasin terminali emulaatori ja üks kolleeg kirjutas programmi, mis 
lükkas emulaatorist maksed pangasüsteemi, ning see toimis 
aastaid niimoodi, enne kui tekkisid tehnilised vahendid, et seda 
natukene viisakamalt teha. \emph{Launch} toimus 
tolle aja kohta suure pressikäraga: tehti korralik meediaüritus, imekenad 
Hansapanga\index{Hansapank} tüdrukud istusid ka seal ja tegid märkmeid. Ja läks mööda vaid
mõni kuu, kui Hansapangal tuli välja
Telehansa\index{Telehansa}.

\question{See kamp, kes tollal
BBSides suhtles, võis olla kokku paarsada inimest. Kust tulid
kliendid modemipangale?}

Kliendid jagunesid umbes pooleks. Forekspanga klientuurist arvestatav 
protsent oli Venemaal, sest suur 
raha oligi tol ajal Venemaal, aga ka Eesti klientuur ei olnud sugugi kehv. Pank 
müüs seda suhteliselt suure summa eest ja Eesti firmad 
ostsid. Käisin seda ise Tallinnas installeerimas. Küsimus ei olnud 
selles, et inimesed ei saanud tulla maalt linna pangaasju ajama, vaid nad 
lihtsalt ei tahtnud kontorist välja tulla. Pangas sai mugavalt ära käia 
laua tagant püsti tõusmata.

\question{Ja see kõik tasus ära, et hakata isegi arvutiga 
makseid ette valmistama?}

Sel ajal oli igas firmas raamatupidamiseks arvuti olemas ja raamatupidajate 
arvutites maksed olidki. Ilmselt mugavus ja aja kokkuhoid tõukasid
Eesti firmad sinnapoole.

\question{Kui palju seal telefoniliine küljes oli?}

Alustasime kahega ja lõpus oli vist kuus. Kuna 
sideseanss oli nii lühike, mahtus enamik sideseanssidest paari minuti 
sisse. Kõik pakiti kohapeal kokku ja saadeti ühe portsuna ära -- Fidonetist õpitud tehnoloogia. Alguses tegin mina kliendipoole ja 
Margus Kliimask\index[ppl]{Kliimask, Margus} kirjutas serveripoole. Hiljem kirjutasin serveripoole veidi paremaks, et see oleks paremini eskaleeritav.

\question{Mida see tähendab?}

Ühe masina taha sai panna mitu modemit.

\question{Kas sa oma BBSi hoidsid siis veel püsti?}

Minu meelest oli meil pangas ka BBS veel mõnda aega, 
Microlinkis\index{Microlink} oli kindlasti. Kuna Forekspank asus Rävala puiesteel, siis kohe, kui 
üheksakümnendate alguses tekkis internet, oli selge, et meil on ka 
seda vaja. Tõmbasime koos Andrus Aaslaiuga\index[ppl]{Aaslaid, 
Andrus} oma valgete käekestega mööda majakatuseid Forekspanga kõrvale KBFI\index{KBFI} majja, 
kus sündis Uninet\index{Uninet}, Etherneti kaabli.

\question{Te olite siis otse Unineti küljes?}

Otse Unineti küljes, olime ühed esimesed kliendid, kodukootud 
ruuteri softiga, mis läks flopi pealt käima. Mõlemas otsas oli üks 
arvutikast ja nii me ennast internetti panime. Muide, ükskord 
lõi meil sinna välk sisse.

\question{Mida te internetis tegite?}

Algul õppisime, mis see on. Ja pangas oli hädavajalik meilivahetus, et suhelda. Üks esimesi asju, mis pangas sai 
ehitatud, oli teleksi \emph{gateway} Pegasus Maili\index{Pegasus Mail}. 

\question{Misasi on teleks?}

Teleks oli viiekümneboodine\sidenote{\emph{Baud rate}, eesti keeles lihtsalt \emph{boodid}, 
näitab, mitu korda sekundis signaal liinil muutub andes indikatsiooni side kiirusest.}  telegraafisüsteem. Kahtlustan, et paljud pangad maailmas kasutavad seda endiselt. Suhtlus ei käi telefoniliini 
pidi, vaid selleks on eraldi teleksivõrk, mis toimib mööda telefonitraate 
hoopis teistsuguste signaalidega kui tavaline telefon.

\question{Kas see oli \emph{circuit switched}\sidenote{Ahelkommuteeritud. 
On ju ilus eestikeelne sõna?}, eks? Siis see vajas eraldi keskjaama?}

Jah. Põhimõtteliselt tuli ikkagi kõne teha ja ühendus püsti seada. 
See ehitati veel sel ajal, kui olid teletaibid -- klaviatuur ja 
paberirull.

\question{See \emph{gateway} ei saanud siis ju olla ainult tarkvaraline, vaid
oli ka riistvara vaja?}

Jah. Seal oli üks kast vahel, mis tegi sellest jadapordi. Esimese kasti tegi minu meelest
Küberneetika Instituudi\index{Küberneetika Instituut} majas üks Sass, Aleksander\index[ppl]{Reitsakas, 
Aleksander}.

See oli väga keeruline kast, tegin hiljem sellest peopesasuuruse 
versiooni flopikarpi.

\question{Mind hämmastab see, et sa ehitasid järjest keerulisemaid asju, aga kust sul tulid selleks teadmised, seda ei selgu.}

See on nagu Youtube'i videot vaadates -- tundub, et kõik asjad juhtuvad ise. 
Vahepeale mahtus siiski kuude kaupa õppimist, häkkimist ja katsetamist.

\question{Sul pidi hirmus kihu seda teha olema.}

Kindlasti, peaasi, et oli huvitav. 
Pangas töötades hakkas esimest korda ka kohusetunne vaevama, sest kui pank hommikul ei toiminud, olin ju mina paha.
Töötunde kulus kõvasti, aga üksiku inimesema ei olnud mul eriti muid kohustusi.

\question{Lisaks rääkisid muudkui teistega juttu BBSides.}

Panga ajal enam mitte, siis võttis töö kogu aja ära. Varem toimus jah BBSides suhtlus, aga kui tuli internet, võttis meilindus asja üle. Meiliga tuli kohe ka  \emph{gateway} 
kohe panga serverisse. Pank oli selles mõttes väga hästi kommunikeeruv.

\question{Legend räägib, et sina kirjutasid esimese eestikeelse klaviatuuri draiveri, 
on see tõsi?}

Nii ja naa. Rainer Nõlvak\index[ppl]{Nõlvak, 
Rainer} leidis esimesena, et klaviatuuril võiks eestikeelne \emph{layout} 
olla. Veel enne, kui infotehnoloogid jaole said, tellis Rainer eestikeelse 
klaviatuuri ära.  Nii et pärast, kui kehtestati  uus standard (EVS 8:1993),  
olid olemas klaviatuur ja oli kirja pandud standard. Lisaks klaviatuurile oli aga vaja ka standardile vastavat 
lokalisatsiooni. Eriti hull lugu oli Windowsi fontidega -- sel ajal oli olemas
Windows 3\index{Windows}. Ja siis korraldati konkurss, kus kõik lähenemised 
olid lubatud.

\question{Kes konkursi korraldas?}

Ma ei mäleta organisatsiooni nimesidenote{Tegemist oli Eesti Informaatikafondiga\index{Eesti Informaatikafond}, sellest sai hiljem Eesti Informaatikakeskus\index{Eesti Informaatikakeskus}, Riigi Infosüsteemi Ameti\index{Riigi Infosüsteemi Amet} eelkäia.}, aga see oli riiklik 
konkurss, mille auhind oli tolle aja kohta täitsa korralik, vist kakskümmend 
tuhat krooni. Olime selleks ajaks Raineriga juba natuke sel alal 
koostööd teinud -- Microlink pani enda klaviatuure müües kaasa draiveri, mis seda 
\emph{layout}'i toetas ka, nii et osa tööd oli juba tehtud. Kui konkurss 
välja kuulutati, ütles Margus Kliimask\index[ppl]{Kliimask, Margus}, visiooniga mees,
et teeme nii, nagu Microsoft teeb. Me \emph{reverse 
engineer}'isime kogu selle DOSi lokalisatsiooni ja klaviatuuri draiverid ning 
tegime installeerimisprogrammi, mis paigaldas 
standadkomponendid: \verb|KEYBOARD.SYS|i, \verb|COUNTRY.SYS|i ja muud
sellised asjad. Kuskilt õnnestus hankida soft, mis tegi Windowsi 
fonte, ja ma joonistasin fondid ka. See ei olnud küll kuigi hea soft, 
ei teinud TrueType'i \emph{hint}'ingut; \emph{kerning} vist 
on see teine, mis teeb fondid ilusaks, kui need väikseks muudad. Eesti 
fondid paistsid ekraanil karvased, aga me ei saanud sinna kahjuks midagi parata. Igal juhul
oli meie lähenemine teistega võrreldes nii palju parem, et võitsime konkursi.

\question{Kas pank läks konkursile osalema?}

Ei, ainult meie Margus Kliimaskiga\index[ppl]{Kliimask, Margus}. 
Meil oli pisike OÜ, koos pangaga tehtud ühisfirma Forex Communications modemipanga müümiseks. 
Selle firma alt osalesimegi. 

\question{Ja osalesite seepärast, et tundus huvitav?}

Sinna läksime ilmselt raha pärast ja võibolla ka 
Näitusepaviljonis toimunud joomingu pärast, mille seesama riiklik asutus piduliku sündmuse puhul 
korraldas.

\question{Kas sul sellepärast saigi panga aeg otsa, et pank sai valmis?}

Pigem pean olema tänulik pangajuhtidele, kes andsid meile 
hämmastavalt vabad käed igasugust tehnoloogiat katsetada ja uurida ning mõelda 
uusi asju. Tänu sellele oli Forekspank ka üks esimesi internetipanga tegijaid -- meil oli olemas internetiühendus ja me juba mõistsime, mis toimub. 

\question{Millega tollast internetipanka tehti?}

Forekspanga esimene internetipank oli minu meelest 
IISi\sidenote{1995. aastal turule toodud \emph{Internet Information 
Server (IIS)} oli Microsofti veebiserver, mis üritas (mõnevõrra tulutult) 
konkurentsi pakkuda tol ajal domineerinud Apache'i veebiserverile.} peal ja töötas
Windowsis\index{Windows}. 

\question{Eksootiline valik tolle aja kohta ...}

Oli küll imelik valik. Aga sel ajal olid meil juba arendus- ja 
hooldusmeeskonnad eraldi. Margus Kliimask\index[ppl]{Kliimask, Margus} oli 
arendusmeeskonnas. 

\question{Ehk te olite \emph{DevOpsist}\sidenote{Arendusmetoodika, kus tarkvara 
ehitamine ja selle edasine käitamine korraga nime kaotavad ehk omavahel 
lahutamatult kokku saavad.} astunud sammu tagasi?}

Panga käigushoidmine ongi natuke omapärane tegevus. Margus 
\index[ppl]{Kliimask, Margus} juhtis internetipanga arendust, tema meeskonnas
oli ka Pronto\index[ppl]{Pronto|see{Raja, Tanel}}\index[ppl]{Pronto}\sidenote{Vt
 lk \pageref{sisu:pronto}.} ja veel paar 
hakkajat selli. 

\question{Kas sina olid ka sellega seotud?}

Mina ei olnud internetipangaga peaaegu üldse seotud. Sel ajal oli
modemipank veel põhikanal, kuna internet oli siis vähestel. Forekspank oli juba üsna suureks kasvanud, 
hooldusmeeskonnas oli kümmekond inimest.

\question{See kõlab juba nagu terve organisatsioon, kahe telefoniliiniga ei saanud enam 
hakkama?}

Sel ajal tekkisid teised probleemid. 
Pangale ostetud tarkvara käis kummalise IBMi platvormi peal, mida aeg-ajalt 
tuli \emph{upgrade}'ida. Selle tarkvara jaoks oli COBOL uus keel. 
Tarkvara oli kirjutatud imelikus keeles nimega \emph{Report Generator Language}, mis 
oli pärit System/36\index{System/36}\sidenote{System/36 oli IBMi poolt 
1983. aastal turule toodud väike mitme kasutaja jaoks mõeldud mitmetegumiline 
server, mida programmeeriti peamiselt platvormipõhises RPG II\index{Report Program Generator} (\emph{Report Program Generator - RPG}) keeles.} ajast. Sellest keelest 
kumasid perfokaardid ikka veel kõvasti läbi.

\question{Vähe sellest, et teil oli visioon, aga raha pidi ju ka olema, et brittide juurde 
minna.}

Server maksis sel ajal meeletu raha. Algul ei olnud pangal jaksu õiget masinat osta, hangiti üks karm 
PC ja selle peal käis System/36 emulaator, millel jooksis 
panga tarkvara. Õnneks kasvasime sellest üsna ruttu välja. Pärast oli meil selline unikaalne platvorm nagu
AS/400\index{AS/400}\sidenote{AS/400, hiljem tuntud kui 
\enquote{System i}, oli IBMi keskmise suurusega serveriplatvorm, mis 
toodi turule 1988. aastal.}, mida ka korduvalt uuendati.

Ilmselt sai pank tarkvara ostes 
ka teadmise sellest, kuidas panka teha. See oli võibolla 
rohkem väärt.

\question{Teil oli Margusega juba siis kahe peale pisike OÜ, aga mõni 
veedab terve elu oma huvi üksnes akadeemilistes sfäärides rahuldades. Kust sul tekkis
arusaam ärist?}

Nagu ma mainisin, siis OÜ sündis modemipanka tehes ja pean jällegi kiitma 
tolleaegseid pangajuhte, kellega koos me ühisfirma lõime. Otseselt äritegemist kui sellist ei olnud: meie 
kirjutasime tarkvara ja inimesed maksid selle eest OÜ-le, pärast 
jagasime pangaga raha ära. Klassikalise äri mõistes ei pidanud meie midagi 
müüma, pank müüs. Muidugi tekkis ettekujutus näiteks
raamatupidamisest, aga erilist ärisoont see minus ei arendanud.
OÜ käigushoidmine mingit tähelepanu ei nõudnud, kogu fookus oli tehnoloogial.

\question{Tõnu Samuel\index[ppl]{Samuel, Tõnu}\sidenote{Vt lk \pageref{sisu:tonu}.}  rääkis mulle, et Mastsidenote{Ehk siis käesoleva loo kangelane.} oli see mees, kelle juurde sai minna riskantsete 
asjadega. Kui oli vaja emaplaadi peal vaibanoaga radu lahti kratsida 
ja sinna relee vahele panna, siis Tõnu teadis, mida teha, aga ei 
julgenud. Seevastu Mast julges.}

Ilmselt oli abiks raadiotehnika kateedri kool. Kui saad aru, 
mida teed, siis sa ei karda lõigata.

\question{Nii et sul sellist aukartust masina ees ei olnud?}

See kadus suhteliselt vara ära, kuna raadiotehnika kateedri Apple 
II\index{Apple II}s oli mitu laienduskaarti sees. Kui sellel oli
kaas peal, siis kuumenes üle, aga kaas ei olnud kunagi peal. Seal võis 
vabalt näppupidi sees sobrada ja mitte keegi ei öelnud, et sa ei tohi seda kivi 
välja võtta. Kõik oli pesades, kõike võis välja võtta. Kui katki läks, siis 
tuligi võtta. 

\question{Kas läks katki ka?}

Ikka läks, aga Apple II\index{Apple II} oli 
lihtsa loogika järgi ehitatud, Vene kivid läksid sinna asemele ja taktsagedus oli üks 
megaherts. Seda sai parandada ja see oli väga õpetlik. Ka 
esimese IBM PC\index{IBM PC}ga tulid kaasa (meil olid 
kõik juhendid olemas) BIOSi \emph{listing}'ud ja skeemid. Kõik olid 
standardtükid, kõike sai parandada ja parandatigi. 

\question{Mida sa pärast panka tegid?}

ITd ühele väikesele investeerimiskontorile. Kirjutasin 
Exceli Visual Basicus\index{Visual Basic} väärtpaberite 
kauplemise programmi. Tol ajal tehti paljusid asju Excelis, näiteks arvutati intressi. Tegin suured Exceli makrod, millega sai 
väärtpaberiportfelle hallata ja tehinguid jagada. 

\question{Kas jälle selle pärast, et oli huvitav?}

See oli rohkem vajaduspõhine. Meie enda investeerimiskontoril oli seda 
vaja ja ühe koopia müüsin maha ka. 

\question{Nii et tegelesid siiski ka müügiga?}

Ma ei tegelenud müügiga. Enamasti oli nii, et keegi tuli ja ütles, et tal 
oleks ka vaja. 

\question{Kui on väärt asi, siis lõpuks ikka tullakse.}

Jah, kui hind sobis, siis miks mitte.

\question{Sa oled BBSummeri\index{BBSummer} kuulsa grupipildi peal. Kas käisid tolle seltskonnaga läbi, kuigi töö võttis enamiku ajast ära?}

BBSummerid algasid siis, kui olin alles tehnikaülikoolis, ja neid ei olnud üldse palju. See grupipilt, mida sina vist 
mõtled\sidenote{Memcpy podcast'i kaanepildiks olev foto, kus on peal 
hämmastavalt paljude suurte asjade toonased või hilisemad algatajad.}, ei ole esimesest BBSummerist, vaid teisest või kolmandast, kus käisid ka FidoNeti tublid mehed Soomest. Seal 
pildil on üks habemega mees nimega Ron Dwight\index[ppl]{Dwight, Ron}, kes 
oli FidoNeti kunn Euroopas, regiooni pealik. Ron 
oli väga tore mees, ma olen tal isegi paar korda külas käinud ja tema juures Soomes 
ööbinud, kui piirid lahti läksid. Ja ma ei ole Eesti kambast ainukene, kes tal
külas käis. 
Soomlased, kes FidoNeti Soomes vedasid, olid tol ajal üldiselt väga toetavad. 
Sa oled teistega rääkinud, kuidas te Soome helistasite, ja keegi ei ole 
maininud, et tegelikult algusaegadel helistasid soomlased siia. Ei olnud nii, 
et ainult sealt oleks tõmmatud. Hiljem, kui BBSid ja firmad said siin jalad 
alla, saime "rinnapiima" otsast lahti, aga algusaegadel 
soomlased toetasid meid tublisti. 

\question{Kas puhtalt missioonitundest? Hõimuvelled ja nii?}

Ma ei tea, kui palju hõimuvendlus rolli mängis, pigem arusaam, et tehnoloogiat tuleb huvitatud inimestega jagada. 

Mul on nendest aegadest väga head mälestused ja sellepärast kutsusimegi neid ka BBSummeritele\index{BBSummer}. Ron käis minu meelest kahel. Igatahes oli
soomlasi esimestel BBSummeritel palju ja ma mäletan, kuidas nad olid selle grupipildi aegsel BBSummeril äärmiselt
hämmastunud sellest, et kõik võivad õlut juua ja et teisel päeval ei toimunud mingeid 
kaklusi!

BBSummeri korraldamise juures oli veel tore see, et korraldustasu
tagas söögi ja joogi kõigiks päevadeks. Ja õlut pidi kõigile jätkuma. Ühele BBSummerile toodi küll õlut Fanta tünnides, nii et 
õllel oli kerge Fanta mekk juures.

\question{Tundub, et sul on inimestega vedanud.}

Mul on jah sõpradega vedanud. Kui ma üksi elasin ja 
tehnikaülikoolis\index{Tallinna Tehnikaülikool} 
vabakutseline olin, siis suhtlesin väga paljudega. 
Hiljem võttis perekond nii palju aega ära, et kahjuks ei jõudnud enam kõigiga 
kontakti hoida.

\question{Aga kriitilisel hetkel olid nad olemas?}

Nad on siiamaani olemas. Näiteks 
Lõvi\index[ppl]{Lõvi} kohtasin ma umbes viis aastat tagasi Selveri 
parklas, nüüd käisin tal hiljuti tehnikaülikoolis külas.

\question{Ahti\index[ppl]{Heinla, Ahti}\sidenote{Vt lk \pageref{sisu:ahti}.}  ütles 
väga targasti, et seltskond noori inimesi sai
omavahel suheldes inimeseks koos Eesti riigiga. Kas sul on ka selline 
tunne?}

Jah, me olime kõik suhteliselt üheealised. Täpselt selles 
vanuses, kui oli huvi teha midagi uut ja selleks tekkis võimalus ning ka omavaheline klapp. Oli ka erandeid, näiteks Henn Ruukel\index[ppl]{Ruukel, Henn} 
oli esimesel BBSummeril selgelt alaealine, aga õlletünni juures passi ei 
küsitud.

\question{Mida sa praegu teed?}

Pean pausi. Aitan ülikoolil satelliiti\sidenote{Masti panusega satelliit lendas 
2020. aastal ka edukalt kosmosesse.} ehitada. 

\question{Sellepärast, et on huvitav?}

Sellepärast, et on huvitav. Kosmos on huvitav.

\question{Kosmos on suur ka, seal ei ole karta, et huvitavad asjad 
saavad otsa.}

Praegu käib sebimine enamjaolt Maale väga lähedal. Orbiidid, kuhu 
väikseid satelliite lastakse, on viie- kuni seitsmesaja kilomeetri kaugusel.

\question{Kas sul üldse on kunagi juhtunud, et järgmist huvitavat asja ei ole 
silmapiiril?}

Ei.

\question{Kuidas see sul on õnnestunud?}

Isegi kui päevatööl ei ole huvitav, siis mul 
kodus käib kogu aeg mõni projekt. Kui üks saab valmis või läheb 
sahtlisse (sinna läheb enamik, sest huvi kaob ära), on 
järgmine kohe laual. Sellist asja ei ole, et mul ei ole midagi teha.

\question{Kas sul sahtel juba täis ei saa?}

Saab. Jube täis on. 

\question{Mida sa siis teed?}

Viskan ära. Suur osa neist on ju eksperimendid. Võtan ära tükid, mis lähevad  
järgmise eksperimendi peale, ja ülejäänu on prügi. Teadmised jäävad alles.


\chapter{Kain Kalju}
\index[ppl]{Kalju, Kain}
\question{Kuidas sina arvutite juurde jõudsid?}
See oli umbes aastal 1990--1991, kui sõpradel tekkisid esimesed arvutid, 
olime kaksteist kuni neliteist aastat vanad. Ühel sõbral oli selline 
imelik asi nagu Texas Instruments TI-99\sidenote{Täpsemalt Texas Instruments 
TI-99/4\index{Texas Instruments TI-99/4}. Ärilistel ja arhitektuursetel 
põhjustel lühikese elueaga koduarvutite perekond. Oli koos samal 1979. aastal 
turule tulnud Atari 8bitiste arvutitega esimesi omataolisi, millel oli 
audio- ja videoülesanneteks eraldi protsessorid.}, see oli Commodore'i ja Apple 
II sarnane riistapuu selles mõttes, et see oli 16bitise protsessoriga ja 
\emph{boot}'is otse BASICusse\index{BASIC}. 

See arvuti oli telekaga ühendatud ja seal olid mõned primitiivsed mängud, nagu näiteks 
Space Invaders\index{Space Invaders}, ning 
loomulikult ka BASIC. Kogu programmi kood tuli kassetilindilt nagu tol 
ajal kombeks, flopisid polnud olemas. See oli minu esimene kokkupuude 
arvutiga, millel oli klaviatuur, kuhu sai sisestada programmi koodi ja 
kus katsetasime ka esimest korda BASICus ise programme teha, toksides 
neid ajakirjadest ja mõeldes ka ise välja. 

\question{Mis linnas see oli?}

Ma olen pärit Keilast\index{Keila} ja mul ei ole kunagi olnud 
spetsiaalset ligipääsu arvutitele mõnes teadusasutuses või koolis. Võrreldes mõne teisega oli minu ligipääs arvutitele suhteliselt piiratud.

\question{Kas sul reaalainete vastu oli huvi?}

Koolis käisin reaalkallakuga klassis. Meil oli 
gümnaasiumis\index{Keila Gümnaasium} väga vahva lend, peaaegu kõik poisid olid 
arvutihuvilised ja nii palju, kui olen nende elukäiku jälginud, on 
pea kõik mingitpidi arvutimaailmas tegevad.

\question{Kuidas see juhtus? Kas teil oli koolis nii korralik tase?}

Gümnaasiumi esimestes klassides (olime just 
läinud üle kaheteistkümne klassi süsteemile) olid kooli arvutiklassis 
Jukud\index{Juku}, mis meid loomulikult absoluutselt ei huvitanud, sest
seal oli Pascal\index{Pascal}, aga meil oli siis juba ligipääs PCdele.

\question{Juku oli omal ajal igavesti äge aparaat!}

Jah, aga Juku tuli hilisemas faasis, kui meil oli 
juba PC ligipääs olemas ja mul endalgi kodus PC. Minu 
suur arvutihuvi läkski lahti sellest hetkest, kui vanemad otsustasid mulle 
PC osta. Seda lugu peab natuke tagasi kerima: seesama sõber, 
kellel oli Texas Instrumentsi imepill, sai aasta hiljem monokroomekraaniga
286, mille ta isa tõi Ameerikast. Nad elasid
kortermaja esimesel korrusel ja PC oli raudkapis luku taga. Oli suur hirm, et 
keegi murrab sisse ja varastab arvuti ära. Aeg oli selline.

\question{Arvuti maksis tollal ju rohkem kui korter. Mõni ime, et PC 
kappi pandi!}

Mu vanematele käis see kohutavalt pinda, et ma ei viibinud üldse kodus, vaid olin
kogu aeg hilisööni sõbra juures külas. Millalgi üheksakümnendatel, 
vahetult enne Eesti krooni tulekut oli aeg, kui rubla devalveerus väga
kiiresti. Olen isa käest küsinud, kuidas see täpselt oli, ja ta on meenutanud, et 
tema sai millegipärast palka Ameerika dollarites ja ostis
kuskilt kooperatiivist dollarite eest
ühe 286. Hinnaklass oli umbes tuhat dollarit. See oli 
VGA ekraaniga, täiesti uus ja väga äge, kuigi 286 oli
tolleks ajaks ilmselt natuke \emph{outdated}, kuna siis oli juba 386 
ajastu.

\question{Võrreldes XTdega, mille abil Tartu Ülikoolis 
programmeerimist õpetati, oli see ikkagi väga kõva sõna. Mida sa selle arvutiga tegid?}

Nagu noor poiss ikka, tõenäoliselt mängisin, aga mind huvitas ka kõikvõimalik 
tarkvara.

Meenub tore lugu, kuidas umbes aasta pärast arvuti saamist käisime sama sõbraga 1993. aastal Ameerikas. 
Meil oli kolmene punt, kes me 
elasime üksteise lähedal, ja meil kõigil oli kodus kas isiklik või vanemate 
tööarvuti. Kord rongiga Tallinnast Keilasse sõites
hakkasime rääkima, et jube lahe oleks minna 
Ameerikasse. Ühel sõbral elas tädi seal ja ta oleks meid hea meelega vastu võtnud, ainult et
kuidas sinna saada. Minu isa töötas Muuga 
sadamas ja tuli jutuks, et põhimõtteliselt saaks ka 
laevaga minna. Olime siis
viie-kuueteistaastased. Rääkisin sellest kodus, kuidagi hakkas 
pall veerema ja ühel hetkel taotlesime juba USA saatkonnas viisat. 
Järgmisel hetkel oli isal kokku lepitud, et saame minna kaasreisijateks 
suurele Ameerika kaubalaevale, ning me sõitsimegi Muuga sadamast laevaga üle Atlandi ookeani 
New Orleansi. Seal pani laevakompanii meid lennukile ja 
edasi lendasime JFK lennuväljale New Yorki, kus sõbra tädi meid vastu 
võttis. Kusjuures me saime laeva peal palka, sest laevafirmale oli palju odavam 
vormistada meid töötajateks. Muidu oleks olnud vaja tasuda suuri 
kindlustusmakseid. Selles mõttes täiesti kreisi käik.

\question{Kaua te sõitsite sinna ja kas te midagi kasulikku ka laeva peal tegite või sõitsite lihtsalt kaasa?}

Laevasõit üle ookeani võttis umbes kaks nädalat. Midagi kasulikku me ei teinud, hängisime nii-öelda ohvitseride
alal. Meile küll näidati, kuidas laev töötab, aga me näiteks ei koristanud tekki ega teinud muud kasulikku. 
Võibolla heal juhul saime ülevaatlikku õpet mootoriruumist justnagu muuseumis. Loomulikult ei lastud meil midagi teha, võibolla avaookeanil saime korraks rooli keerata ja nii-öelda
laeva juhtida.

\question{Millega te tagasi tulite?}

Tagasi tulime lennukiga. Aga miks ma sellest üldse räägin, on see, et 
Ameerika pinnale astudes oli meil päris palju raha, kuna saime laevast 
palka, umbes tuhat viissada dollarit, mis oli tolle aja kohta üüratu 
summa. Mina kulutasin raha ära loomulikult arvutipoes -- tõin endale Ameerikast elu ühe kõige tähtsama riistapuu, 
milleks oli modem. Pärast seda läks elu lahti. 

See oli 2400boodine modem, tüüpi ei mäleta. Lisaks 
tõin Sound Blaster 16\index{Sound Blaster} helikaardi, mis oli tollal täiesti 
tipp\sidenote{Sound Blaster oli Singapuri firma Creative 
Technology (tuntud USAs kui Creative Labs) helikaartide perekond. Need kaardid 
olid PC-maailmas \emph{de facto} standardiks, kuni Windows 95 vastavad liidesed 
standardiseeris ja PC audio muutus tarbeesemeks.}. See oli just paar kuud varem välja tulnud. 

Üks asi, mille ma hiljem avastasin ja mis levisid BBSides, olid helimoodulid. 
Mul oli neid päris palju, kogusin neid mõnda aega. Ilmselt toona 
tegid seda paljud. Need on helifailid, mida tollal pandi kokku
Amiga arvutites ja mis koosnesid sämplitest. Oli umbes kaheksa \emph{track}'i, kuhu sai miksida sämpleid niimoodi 
kokku, et tekkis muusika. 

\question{Ja need liikusid BBSides?}

Jah. Loomulikult sai üritatud neid ka ise teha, aga mul erilist muusikatausta ei olnud, nii et sellest ei tulnud midagi välja.

\question{Kas Ameerikast tagasi tulles panid kohe BBSi püsti?}

Ei, hakkasin siis alles avastama BBSi 
maailma. Vanu asju üle vaadates selgus, et üks mu lemmik BBS oli Dark 
Corner\index{Dark Corner}, mida vedas Priit Kasesalu\index[ppl]{Kasesalu, 
Priit}. Esmalt loomulikult üritasin alla laadida kõike, mida sain. 
Kõik oli ju puhas kuld, kogu tarkvara. 
Tol ajal veel
Kadaka turg\index{Kadaka turg}\sidenote{Aastal 1991 avatud ja 2002. aastal 
kaubanduskeskusega asendatud, Mustamäel asunud turg oli küllaltki metsik 
müügikeskkond, kust oli võimalik hankida kõike alates karvamütsidest ja 
Nõukogude aurahadest kuni kõikvõimaliku piraatkaubani. Sisuliselt oli tegu 
endise Nõukogude Liidu territooriumil toiminud varimajanduse väljundiga 
Eestisse. Turg oli turistide seas hinnas, parematel aegadel käisid sinna 
Tallinna sadamast eribussid.}, kus müüdi piraattarkvara. Nii et
väga palju sain ka sealt. Minu mäletamist mööda BBSides 
otseselt piraattarkvara väljas ei olnud, pigem 
häkkimise stiilis tarkvara.

\question{Windowsi sealt vist keegi endale ei laadinud?}

Jah, just, selliseid asju otse faililistides ei olnud, need olid taha 
nurkadesse ära peidetud. Aga seda mäletan küll, et meil oli kodus 
telefoniliin ja minutitasu ei olnud või siis 
oli see väike. Igal juhul oli meie koduliin enam-vähem
ööpäev ringi kinni, sisse ei olnud võimalik helistada, sest minu 
arvuti helistas ja laadis kogu aeg midagi alla.

\question{Kuidas te alguses rea peale saite? Kuidas sa teada said, mis numbri 
peale helistada?}

Võimalik, et see teadmine tuli .EXE ajakirjast\index{.EXE}. Aga kui oled ühte BBSi juba sisse pääsenud, siis avaneb kogu 
maailm. Üks teema, mida BBS levitas, oli teiste BBSide 
aadressidega failid. Ühel hetkel pani Priit Kasesalu\index[ppl]{Kasesalu, Priit} 
kogu oma BBSi viimase versiooni veebi üles. Laadisin selle alla ja 
avastasin selle kettalt igasugu huvitavaid asju. 

\question{Mida seal leidus?}

Kõikvõimalikke häkkimisvahendeid, C-programmide näiteid, raamatuid 
nagu \enquote{Terrorist Handbook}\sidenote[][-4cm]{Ilmselt peab Kain silmas William 
Powelli raamatut \emph{The Anarchist Cookbook}. Vietnami sõja vastaste 
protestide laineharjal 1971. aastal USAs ilmunud (ja mitmel pool keelatud 
olnud) raamat sisaldas kõikvõimalikku vastandkultuuriga seotud sisu 
termiidi ja LSD valmistamise õpetustest kuni telefonisüsteemide 
murdmise juhisteni. Raamat levis tekstifailina laialt ülikoolide serverite ja FidoNeti 
kaudu ning seda täiendati pidevalt; eriti kuulsad on anonüümse 
autori \enquote{\emph{The Jolly Rogeri}} täiendused.}. Igasugune 
selline kraam, mis pakkus noortele inimestele põnevust.

\question{Tuleme veel kord sinu arvutihuvi alguse juurde. Kas sa olid pigem 
seda tüüpi mees, kes mängis arvutiga, võrgutas arvutit või programmeeris 
arvutiga?}

Tagantjärele mõeldes on olnud mitu ajajärku. Koduse 286 ja BBSide ajal üritasin pigem 
sisse krahmata kõike, mida nägin. Leidus ka 
arvutimänge, aga ma ei mäleta, et oleksin väga palju mänginud. 
Kui mul endal veel arvutit ei olnud, siis sõbra juures mängisime loomulikult, 
mitte ei programmeerinud. Hiljem jäi mängimine tagaplaanile ja püüdsin
aru saada, kuidas arvuti töötab. Näiteks üks teema, mis mind 
kohutavalt paelus, olid viirused. Mul oli alati kõige viimane viirusetõrje 
tarkvara. Mul oli selleks hetkeks juba mitu kõvaketast ehk 
võimalus katsetada, mida viirused teevad. BBSides levitati ka nii-öelda 
viirusekollektsioone ja ma uurisin, kuidas viirus 
põhimõtteliselt töötab. 

Järgmine ajastu tuli siis, kui avastasin enda jaoks 
Linuxi\index{Linux}, samal ajal tekkis ka internet. Gümnaasiumi 
kaheteistkümnendas klassis sattusin tööle Riigi Elektriside 
Inspektsiooni\index{Riigi Elektriside Inspektsioon|see{Tehnilise Järelevalve 
Amet}}, mis on täna Tehnilise Järelevalve Amet\index{Tehnilise Järelevalve 
Amet}. Sattusin patsiga poisiks, kuigi patsi pole mul
kunagi olnud. Olin tavaline arvutipoiss, seadistasin arvuteid.

\question{Kuidas sa kooli kõrvalt sinna sattusid?}

Seesama sõber töötas Pennus\index{Pennu} ja temalt kuulsin,
et otsitakse arvutitüüpi, kes oskaks arvutitega midagi teha. Läksin 
kohale ja mind võeti poole kohaga tööle.

\question{Kas teil klassist töötasid mitmed keskkooli ajal?}

Jah, üks klassivend töötas näiteks Keila linnavalitsuses. Ta oli juba siis kõva programmeerija, kinkis mulle 
mu esimese programmeerimisraamatu \enquote{C Programming Language}\index{The C 
Programming Language}, autoriteks Brian Kernighan ja Dennis Ritchie.

\question{See on seesama salapärane väljaanne\sidenote{\phantomsection\label{sisu:richie_vene}Kainil oli raamat 
jutuajamisel kaasas, selles puudus igasugune märge väljaandja ning trükkimisaja ja -koha kohta. Raamat oli 
korralikult köidetud ja kopeeris isegi värvilist kaanekujundust täpselt. 
Paigas olid ka sellised detailid nagu indeksis mõiste \enquote{recursion}, mis viitas (nagu ka mõiste sisu nõuab) 
tagasi mõistele endale. Mart Palmas\index[ppl]{Palmas, Mart} mäletab, et raamatut 
olla trükitud Novosibirskis. }, mis minulgi oli.}

Just, Amazonis on täpselt seesama raamat müügil. 
See oli mu esimene programmeerimisraamat, aga leidis kasutamist ka
aastaid hiljem, kui tegin netit\index{neti.ee} ja mul tekkis praktiline 
vajadus programmeerida suurema jõudlusega otsingusüsteemi.

\question{Kuidas teil ikkagi juhtus olema selline klass, 
kus mitmed töötasid-programmeerisid juba keskkooliajal?}

Võibolla just sel ajal arvutid ilmusidki rohkem koju ja 
kontorisse ning oli tohutu puudus oskusteabest. Vanemad 
inimesed ehk ei julgenud arvuteid veel kasutada, samal ajal kui noored julgesid nendega 
igasuguseid asju teha.

\question{Igas keskkooliklassis ei olnud nii, et neli-viis 
poissi töötasid arvutispetsialistidena. Miks teil oli?}

Ma ei oska seda tagantjärele öelda. Küll aga mäletan sellist huvitavat seika, et 1995. aastal oli meil kaheteistkümnendas klassis
arvutieksam, mis aga ei seisnenud 
programmeerimises, vaid me seadistasime koolile
arvutiklassi. Kool sai Tiigrihüppe kaudu peaaegu 
klassitäie arvuteid ja siis R-klassi\sidenote{Reaalklass.} poiste ülesanne 
oli võrgutada klass 
füüsiliselt Etherneti kaabliga ning installeerida arvutid ja 
võrguserver, milleks oli Linuxi server. Server jäi minu peale, 
kuna ma olin tollel hetkel kõige suurem Linuxi\index{Linux} käpp võrreldes 
teiste poistega.

\question{Linux ei olnud selleks ajaks ju kuigi vana, kuidas sa selle otsa 
komistasid?}

Linuxi otsa komistasin siis, kui töötasin Riigi Elektriside 
Inspektsioonis\index{Riigi Elektriside Inspektsioon}. Kui ma sinna 
läksin, siis seal veel internetti ei olnud, aga tekkis paar kuud hiljem, 1994. aasta lõpus. 
Inspektsioon asus aadressil Ädala 4d, mis on ka 
legendaarne internetihoone. Meie allkorrusel oli 
Valitsusside\index{Valitsusside}, kus toimetas Taavi Talvik\index[ppl]{Talvik, 
Taavi}. Taavi andis Riigi Elektriside Inspektsioonile juhtmeotsa kätte, 
milleks oli kümnemegabitine koaksiaalkaabel, ja ütles, et palun, siin on 
internet. See koaksiaalkaabel sai veetud kõikidesse ruumidesse, ei mingeid 
\emph{hub}'e ega tähttopoloogiat.

Siis avastasingi enda jaoks interneti. Koolis loomulikult poistele rääkisin, et 
FidoNet on vana ja aeglane jama, toimib üle modemi, aga meil on 
üks palju uuem ja huvitavam asi. Kusjuures Valitsussidest edasi olid kanalid 
üsna kiired. Mäletan, et Tartu Ülikooli FTP-serverist sai kahemegabitise 
kiirusega faile alla laadida, see oli meeletu kiirus. Välislink oli 
loomulikult kuskil 64 või 128 kilobitti. 

\question{Mida Tartu Ülikoolist tõmmata oli?}

Seda ma täpselt ei mäleta, aga ju midagi oli, sest mul on väga selgelt 
meeles kadri.ut.ee\index{kadri.ut.ee}\sidenote{Tartu Ülikooli masinad kadri.ut.ee ja madli.ut.ee said Toomas Soome\index[ppl]{Soome, Toomas} andmetel nimed Otto Telleri\index[ppl]{Teller, Otto} tütarde järgi.} FTP-server. 

FidoNet oli selles mõttes tohutu kullaauk, et avas kõik oma 
\emph{echo}-kanalid. Internet aga avas meililistid ja ühes listis ma 
lugesin, et Anto Veldre\index[ppl]{Veldre, Anto} teeb 43. 
Keskkoolis\index{Tallinna 43. Keskkool} 
sissejuhatavaid kursuseid. Toona ilmus ka ajakiri .EXE\index{.EXE}, 
kuhu Anto artikleid kirjutas. Ma ei mäleta, kumb kummale täpselt eelnes, aga 
igatahes ühel hetkel olin 43. keskkoolis, et 
\enquote{siin ma olen ja ma tahan teadmisi saada}. Seal tegutsesid  
sellised legendaarsed koolipoisid nagu Indrek Mandre\index[ppl]{Mandre, 
Indrek} ja Heno Ivanov\index[ppl]{Ivanov, Heno}. Tagasi tulin  
juba kuue Slackware\index{Slackware}'i distributsiooni installeerimisflopiga. Installeerimisprotsess käis flopi kaupa. 

\question{Kas lisaks kõigele muule jäi Anto peale ka Linuxi-pisiku 
levitamine Eestis?}

Tal oli väga suur roll selles, et Linux 
Eestis käima läks. Igal juhul mina sain küll selle pisiku. Kuna olin tolleks 
hetkeks juba mõnda aega Elektriside Inspektsioonis\index{Riigi Elektriside 
Inspektsioon} töötanud ja ka palka saanud, oli mul päris korralik 
\enquote{taskuraha}. Müüsin oma 286 FidoNetis maha 
(FidoNetis käis ka suur riistvaraga hangeldamine) ja ehitasin endale uue arvuti,
486, kusjuures see ei olnud mitte lihtsalt 486, vaid 486DX4 
100 MHz\sidenote{Inteli nomenklatuuris oli DX-tähistusega protsessorite 
kiibil eraldi matemaatika kaasprotsessor, mis andis märgatava
jõudlusvõidu.} -- absoluutne tipp. 

See oli kõige kõvem 486, mis üldse kunagi tehti. Sel ajal oli mul juba \emph{node} registreeritud. Olin varem saanud Dark Corneri 
BBSist\index{Dark Corner} esimese FidoNeti \emph{point}'i, kust 
pääsesin ligi FidoNeti uudisekanalitele, aga ühel hetkel tundus, et 
\emph{node} oleks ägedam. Kirjutasin Tarmo Mamersile\index[ppl]{Mamers, 
Tarmo} (tema oli Eesti regiooni \emph{manager}, kes jagas aadresse), 
kas oleks võimalik registreerida \emph{node} number kuuskümmend kuus, ja 
Tarmo vastas, et \enquote{tehtud}.

Millalgi seadistasin Elektriside Inspektsioonis 
Linuxi\index{Linux} serveri, sest meil on praktiline vajadus kasutada
printerit, faksi ja faile. Linuxi server jagas 
faile üle Samba teenuse ja võttis vastu fakse. Mul õnnestus ka enda 
FidoNeti \emph{node} samasse serverisse sokutada. Kui muidu töötas FidoNeti 
tarkvara MS-DOSi peal, siis oli ka alternatiiv Unixitele 
Ifmaili\index{Ifmail}-nimelise programmi näol.

\question{Miks riigiametis üldse internetti vaja 
oli? Kas see oli puhtalt sinu huvi või tehti seal sellega midagi kasulikku ka?}

Jah, praktiline vajadus interneti järele oli olemas, sest 
inspektsioon\index{Riigi Elektriside Inspektsioon} tegi koostööd 
ITUga\sidenote{\emph{Rahvusvaheline Telekommunikatsiooniliit.}}, kes 
juhib sageduste jaotust, protokolle ja kõike muud sellist. 
Inspektsioonil oli ITUga tihe kirjavahetus, ilmselt meili teel. Ma 
ei suuda meenutada, kuidas meilivahetus enne kaabliga internetti käis, aga 
pärast oli seesama Linuxi masin ka loomulikult meiliserveriks. Meil
tekkis oma domeen rei.ee ja Linuxi server hakkas rei.ee kirju vastu 
võtma ning ka mina sain endale esimese isikliku ülilühikese meiliaadressi, mis 
oli tollal ülikõva -- kain@rei.ee\sidenote{Lühikesed meili- ja muud aadressid 
olid staatusesümbolid, mis näitasid kuulumist kas serveriadministraatorite 
kõrgesse kasti või neile väga lähedasse ringkonda.}.

\question{Nii et sa avastasid ennast suhteliselt õrnas eas Linuxi ruuduna 
riigiasutuses?}

Just. Ja kui külastasin Anto 
Veldre\index[ppl]{Veldre, Anto} arvutiklassi 43. 
keskkoolis\index{Tallinna 43. Keskkool}, jäi mulle sealt üks asi elu 
lõpuni meelde: kuidas kõik need noored tüübid, kes seal 
siil.edu.ee\index{siil.edu.ee}-nimelise SCO\index{SCO UNIX} masina 
taga istusid, olid tohutult kõvad häkkerid. Nad demonstreerisid,
kuidas suudavad \emph{exploit}'ida Tartu 
Ülikoolis olevaid masinaid\phantomsection\label{sisu!ylikooli_root} ja sealseid professoreid jälgida. See 
avaldas mulle nii suurt muljet, et mind hakkas lisaks 
viiruseteemale huvitama arvutiturvalisus.

Me olime kõik ühise Etherneti kaabli peal. Räägin sellest esimest korda avalikult, et ma 
\emph{sniff}'isin loomulikult ka meie võrgus, mida 
Valitsusside\index{Valitsusside} insenerid seal tegid. Sama kaabli otsas oli kaks ametit: Riigi Elektriside 
Inspektsioon\index{Riigi Elektriside Inspektsioon} ja 
\emph{Valitsusside}. Ja kui Valitsusside insenerid käisid oma ruutereid või 
keskjaamu üle Telneti konfigureerimas, siis levis liiklus lahtise 
tekstina võrgus. Nende tegevust oli päris huvitav jälgida. 
Loomulikult ei kasutanud ma seda kunagi pahatahtlikult ära.

\question{Eks see seik iseloomustab suurepäraselt toonast aega. Kui 
praegu kasutaks keegi lahtise traadi peal lahtist kanalit, korraldataks 
poole tunniga mingi jama.}

Jah, ma arvan ka. Siis oli kogu võrguvärk 
niivõrd ebaturvaline, et niipea kui Soomest keegi härrasmees\sidenote{Tatu 
Ylönen, Helsingi Tehnoloogiaülikooli teadlane} tegi 
\emph{secure shell}'i esimese versiooni, hakkasin seda kasutama kohe, kui teada sain. 

Kui nüüd Keila Gümnaasiumi\index{Keila Gümnaasium} juurde tagasi tulla, siis 
pärast kooli lõpetamist jäin ma seal edasi 
administreerima kooli serverit. Nagu tol ajal ikka, pidid 
kõikidel Unixi masinatel olema ilusad nimed. Kodus sellest rääkides pakkus 
isa välja, et Kratt oleks hea nimi. 
Vaatasin hiljuti nimeserverist järele, et Keila Gümnaasiumi serveri 
nimi on siiamaani kratt.keila.edu.ee\index{kratt.keila.edu.ee}.

\question{Hakka siis nime tagantjärele muutma \ldots Loodetavasti riistvara ei ole 
päris seesama?}

Riistvara ei ole kindlasti sama, sest seda koolimaja füüsiliselt enam alles 
ei ole. Keilas on nüüd uus koolimaja, kus mu enda lapsed käivad, sest ma elan 
siiamaani Keilas. Aga serveri aadress on sama.

\question{Seepärast ongi asjade nimetamine oluline, et nimed võivad pikalt 
kesta.}

Just. FidoNeti ajast on veel üks huvitav asjaolu osutunud 
hiljem väga kasulikuks. Nimelt töötasid modemid
AT-käsustikuga\index{AT-käsustik}\sidenote{Hayesi käsustik, tuntud ka kui
AT-käsustik, on käsukeel, mille Dennis Hayes lõi 1981. 
aastal omanimelise ettevõtte 300-boodise modemi Smartmodem juhtimiseks.}, mis 
oli selles mõttes universaalne, et seda kasutati hiljem 
erinevates muudes rakendustes. BBSidesse sissehelistamine toimus loomulikult
lihtsa terminaliga ehk pidid nagu häkker käsustikku teadma. Enne 
helistamist pidi sisestama ATDT, telefoninumbri ja nii edasi, võibolla ka
seadistama protokolli. Tolleaegsed inimesed teavad täpselt, 
missuguse protokolli heli on kuulda memcpy podcast'i avakõllis. 

\question{Kas BBSil oli kliendisoft ka?}

Ei olnud. Helistada tuli terminaliga, ainult FidoNetil oli kliendisoft 
nimega FrontDoor, mis helistas, ja teine soft, mis pakkis kokku FidoNeti 
\emph{echo}'d ja saatis selle paki edasi. BBSil kliendisofit ei olnud, tuli minna Telnetiga külge ja seal 
edasi tegutseda.

\question{Oleks ju olnud loogiline, et 
keegi oleks teinud BBSide ette, näiteks \emph{cache}'i jaoks 
mingi tarkvara.}

Jah, kui vaadata, mis toimus Ameerikas, kus olid 
\emph{online service provider}'id, nagu AOL ja 
CompuServe\index{CompuServe}\sidenote{Internetieelsel ajal domineerisid USA 
turul agressiivsete turunduskampaaniatega (ühel hetkel oli pool \emph{kõigist} 
toodetud CDdest AOLi logoga) teenusepakkujad, kes pakkusid kummalist segu 
BBSi-laadsetest ja internetiteenustest. Neist suurimad olid CompuServe, Prodigy 
ja America Online.}, siis neil oli tarkvara olemas. Mäletan, et 
kui olin USAst modemi ostnud, siis noorte poistena tahtsime seda loomulikult 
proovida. Kujutad ette, me keerasime kruvikeerajaga lahti 
ühe suure soliidse arvuti, vist Computer 2000\sidenote{Computer 2000 
oli küll ka siinmail tegutsenud arvutiäri, kuid ilmselt peab Kain silmas Gateway 
2000 nimelist ettevõtmist, mis tootis sama nime all personaalarvuteid.}, mis oli 
tol ajal väga kõva valge PC bränd. Sõbra tädimehel oli 
väike arhitektuuribüroo ning meil oli julgust 
omavoliliselt kruvikeerajaga lahti keerata üks nende suur \emph{tower} ja 
selle sees proovida seda sisemist modemit. Modemiga oli kaasas kas 
CompuServe'i või mõne muu sarnase teenuse CD-plaat või flopi, ja siis sai helistatud Ameerika BBSi. 

\question{Kui sa BBSides ringi kolasid, kas sulle jäi midagi muud peale tarkvara 
ka silma? Sa mainisid raamatuid ja MODe.}

Raamatud mind siis eriti ei köitnud, BBSidest laadisin ikkagi 
peaasjalikult tarkvara ja muusika MODe. Aga kogu infovoog 
tuli FidoNetist, see oli minu jaoks kullaauk. Nagu varem 
mainisin, ei olnud mul ligipääsu teadusasutustesse ja
ülikoolidesse ega ka mentorit. Meil oli kamp 
poisse, kes omavahel infot vahetasid, ja kõik käis katse-eksituse meetodil.

\question{Hea, et te selle kambaga paha peale ei läinud. Noored 
poisid, tont teab, mida oleks võinud teha.}

Ju siis olime piisavalt mõistlikud. Sellest ajast saadik on mul 
ise õppimise oskus. Võibolla see sai ka saatuslikuks, miks ma ei suutnud
tehnikaülikoolis kaua õppida, ainult ühe aasta nagu
paljud teisedki tollal.

Peale gümnaasiumi läksin tehnikaülikooli informaatikasse\index{Tallinna 
Tehnikaülikool!Informaatika}, aga kuna ma juba ka töötasin, siis tekkis
igasuguseid huvipakkuvaid projekte. Mina eeldasin, et 
saan hakata ülikoolis programmeerimist ja muid huvitavaid asju 
õppima, aga tuli välja, et kõigepealt tuli läbida füüsika ja 
matemaatika. Mul oli matemaatikast natuke kopp ees, kuna 
meil oli gümnaasiumis väga püüdlik matemaatikaõpetaja ja tegelesime 
matemaatikaga põhjalikult, nii et ülikooli sissesaamise probleemi 
ei olnud -- matemaatikaeksamist lihtsalt 
lendasime läbi.

Ja nii see ülikool järgmisel aastal pooleli jäi.

\question{Kuidas sul kaitseväega lood on?}

Seejärel tuligi kaitsevägi\index{Kaitsevägi}. Kui ülikoolis ei õpi, siis varem või 
hiljem leitakse sind üles. Kaitseväkke läksin 1997. aasta suvel ehk 
olin siis juba aasta otsa Netit teinud. 

Ahjaa, et kuidas ma sinna sattusin. Töö Elektriside Inspektsioonis\index{Riigi 
Elektriside Inspektsioon} hakkas pisut ära tüütama, tahtsin 
edasi areneda ja kuhugi huvitavasse kohta tööle 
minna. Mul tekkis soov kindla peale töötada arvutifirmas, et saada arvutitele väga 
lähedale.

Vanu \emph{backup}'e läbi kammides jäi silma Helmes\index{Helmes} ja ma isegi kandideerisin sinna, aga ei saanud. Õnneks, mõtlen ma nüüd tagantjärele. Keskkooli ja ülikooli vahelisel suvel töötasin poolteist kuud 
Tõnu Samueli\index[ppl]{Samuel, Tõnu} IT-firmas nimega Eramees\index{Eramees} 
ja maandusin samale kohale, kust oli just lahkunud Pronto\index[ppl]{Pronto}. 
Tõnu ütles mulle, et Pronto müüs Gravis 
Ultrasoundi\sidenote{Üheksakümnendatel väga populaarsed helikaardid, mis 
esimesena omataoliste hulgas suutsid toimetada pärisinstrumentide 
sämplingutega.} kaarte ja et hakkaksin ise sellega tegelema. Aga ma olin 
noor koolipoiss ega teadnud kaubandusest mitte midagi. Vaevalt
minust seal ettevõttes muud erilist kasu oli, kui et olin nii-öelda patsiga poiss. 

\question{Päris mitmed inimesed on ühel hetkel tegelenud müügitööga ja üldse mitte 
halvasti.}

Eramehest on mul veel üks asi eredalt meeles. Tõnu BBS oli tal 
kontoris, mis asus Eesti Talleksi majas, Mustamäe tee 1, kui ma ei 
eksi\sidenote[][-8mm]{Siiski Mustamäe tee 4.}. Ja see BBS kujutas endast aknalaual laiali laotatud arvutijuppe: seal 
oli USR Courieri\index{US Robotics!Courier} modem\sidenote[][-8mm]{US Roboticsi 
 Courier tooteliin oli oma töökindluse ja suurte kiiruste tõttu 
BBSide ja varaste internetipakkujate lemmik, ka Eestis.}, emaplaat, toiteplokk 
ning hunnik juppe ja juhtmeid. See siis oligi Tõnu BBS või \emph{node}.

Pärast Erameest kandideerisin Estpak Datasse\index{Estpak Data}, sest mulle 
tundus, et ISP on tegelikult veel huvitavam asi, ja nad 
tegelesid internetiga.

\question{Kas Estpak oli tol ajal juba Eesti Telefoni oma või veel eraldi?}

See oli eraldi. Kui ma õigesti mäletan, siis Estpak Data omanik oli 
Eesti Telekom\sidenote[][-2.8cm]{Eesti Telekom ehk pika nimega Riigiettevõte Eesti 
Telekommunikatsioonid oli Teede- ja Sideministeeriumi haldusalas töötav 
\emph{holding}-ettevõte, mis valdas Eesti Telefoni, Eesti Mobiiltelefoni, Eesti 
Kaugotsingu, EsData, Estpak Data ja TeleMedia aktsiaid. Hiljem viidi ettevõte 
börsile ja ainuomanikuks sai Telia.}, mitte Eesti Telefon. See oli Eesti Telefonist täiesti eraldiseisev ettevõte. Huvitaval kombel oli kellelgi 
tulnud idee edendada veebi 
virtuaalhostimist. Keegi oli välja mõelnud neti.ee\index{neti.ee}-nimelise 
domeeni, mille alt üritati müüa traditsioonilist 
veebihostingut. Tollal see veel traditsiooniline ei olnud, aga 
tänapäeva mõistes küll. Estpak Data palkas mu veebihalduriks,
kes pidi hoolitsema veebi hostinguserveri ja -teenuse eest. Muu seas tekkis 
neil mõte, et kuidas veebihostingu äri ikka muudmoodi 
edendada, kui et on vaja kataloogi. Inimesed peavad ju 
need veebilehed, mida kliendid sinna panevad, üles leidma.

\question{Kas tol ajal oli Meediamaa juba olemas?}

Meediamaa\index{Meediamaa} startis umbes samal ajal. Enne seda oli olemas 
Eesti veebisaitide nimekiri, mis oli nlibi ehk 
Rahvusraamatukogu\index{Rahvusraamatukogu} domeenis, kus tegutses Toomas 
Mölder\index[ppl]{Mölder, Toomas}. Ilmselt kolis tema
selle nimekirja Meediamaasse ja sealt www.ee\index{www.ee}-sse. Kuna 
Meediamaa üks tegelane oli Tarvi Martens\index[ppl]{Martens, Tarvi}, siis neil 
õnnestus EENetilt\index{EENet} välja meelitada domeen nimega 
www.ee\sidenote{Alates oma asutamisest 1993. aastal kuni 2013. aastani oli 
EENet .ee domeeni registripidaja ja rakendas mitmeid suhteliselt rangeid 
reegleid. Näiteks oli domeeni registreerimine küll tasuta, kuid ühel 
organisatsioonil tohtis olla vaid üks domeen.}. Ma arvan, et mitte kellelegi 
teisele kui Tarvile ei oleks sellist domeeni elu sees välja antud.

\question{Kas sa kataloogi tegid algul käsitsi?}

Jah, alguses käsitsi, see oligi väga 
algeline ja puine. Asi hakkas lendama siis, kui kutsusin appi Jaanus Vainu\index[ppl]{Vainu, Jaanus}, kellega tutvusime 
Riigi Elektriside Inspektsioonis\index{Riigi Elektriside Inspektsioon}. 
Jaanus on ka omamoodi huvitav tegelane. Inspektsioonis mõtles tema 
välja kogu meie FM 108 sageduse plaani ehk kõik Eesti raadiojaamade 
sagedusnumbrid on tema tehtud. Nõukogude ajal oli meil teistsugune FM 
sagedusala, et takistada raadiost 
välismaiste raadiojaamade kuulamist. Eesti Vabariigi alguses 
koliti lääne sagedustele üle. Jaanus oli üks nendest, kes käis mööda Eestit 
mõõtmas ja tegi sagedusplaani. Tal joonistas väga detailselt Corel Draw's\index{Corel 
Draw} kõik sagedusringid Eesti kaardile. Eesmärk 
oli planeerida sagedused nii, et saatjatel oleksid kogu Eestis sagedused, 
millel on võimalikult vähe häireid naaberriikidega ja omavahel. 

\question{Kas kogu seda teadust tehti Corel Draw abil?}

Jah. Jaanus on tohutu pedant ja suure töövõimega katalogiseerija. 
Tema enda isiklik huvi on \emph{bluegrass}-muusika. Mäletan, et tema oli esimene 
inimene minu tutvusringkonnas, kes välismaalt e-poest asju tellis, näiteks plaate 
CDNow'st\sidenote{CDNow oli 1994. aastal asutatud internetipõhine muusikamüüja, 
kes paraku ei elanud esimest dot.com-mulli üle ja sulges sajandivahetusel uksed.}, ja imestasin, kuidas selline asi üldse 
võimalik on. Ta tellib jumal teab kust CD ja see tulebki pakiga kohale.

\question{Jaa, isegi üheksakümnendate lõpus oli Amazonist raamatute tellimine 
suhteliselt eksootiline tegevus. Aga mis hetkel ja kuidas te 
neti.ee\index{neti.ee} automatiseerisite?}

Meie tandem Jaanusega töötas selles mõttes ülihästi, et mina olin 
programmeerija ja arendasin tarkvara ning Jaanus oli katalogiseerija. Kui ta 
projektiga liitus, siis hakkas see täielikult 
lendama. Meil läks paar kuud aega, kui olime 
Meediamaast\index{Meediamaa} igatpidi kõikide näitajate poolest mööda läinud. 
Olime tollal ajal võibolla isegi natuke liiga ebaviisakad noored mehed. 
Näiteks reklaamisime netit spämmides: tegime 
masspostituse, saates kõikvõimalikele meiliaadressidele teate, et nüüd 
on selline huvitav teenus olemas nagu neti.ee, tulge ja külastage. Kui vaatasin hiljuti enda \emph{backup}'e, siis 
avastasin, et nimetasin oma \emph{crawler}'it ehk otsingurobotit, kes mööda lehti 
ringi kolab, Nuhiks. 

Huvitaval kombel olin Nuhi programmeerimist alustanud juba mitu kuud 
varem ehk miski oleks mind nagu suunanud sellele teele, et seda võib vaja 
minna. Otsingumootoreid olin ka varem pisut teinud. Kui ma pärast 
Erameest ülikooli läksin, siis üks sealt saadud tuttavatest kutsus mind 
tegema üht ärikataloogisarnast teenust Bartanet. See 
asus EsData\index{EsData} Suni serveris Akadeemia tee 21 
teisel korrusel, samas majas, kus me hetkel viibime. Ja selles Suni serveris 
sain teha FTP-serverite otsingut. Panin püsti otsinguteenuse Filerix, mis töötas umbes 
kolm-neli kuud ja võimaldas väga hõlpsasti 
faile üles leida igasugustest kohalikest FTP \emph{mirror}'itest. 
Marek Tiits\index[ppl]{Tiits, Marek} hostis tollal IBSist\index{Institute of Baltic Studies} 
sellist asja nagu TuCows\sidenote{TuCows (The Ultimate Collection 
Of Winsock Software) keskendus oma algusaegadel tasuta tarkvarale. Kuna 
interneti kiirus sõltus veel väga suurel määral geograafiast, hoidis
ettevõte käigus skeemi, kus huvilised võisid jooksutada lehekülje TuCows.com 
lokaalseid peegleid. Ühte sellist Marek pidaski.}. Minu otsingumootor 
võimaldas kergesti failinimede järgi üles leida tarkvara tolleaegsele Windows 
95-le, vanadele Windowsidele ja nii edasi. Nii et tolle pooleaastase projekti kõrvalprojektina tegin failiotsingut.

\question{Suure hulga failide indekseerimine ei ole naljaasi, vaid 
eeldab programmeerimisoskust. Kust sa selle üles korjasid?}

Tol hetkel oskasin ma programmeerida Perli\index{Perl} ja kõike 
seda, mis Unixi \emph{shell}'is oli saada. See oskus tuligi sellest perioodist, 
kui uurisin, mis on nii-öelda Unixil kõhus.

\question{Kas sa korjasid algoritmika ja muu sellise ise üles?}

Jah, aga mis puudutab veebi \emph{crawl}'imist, siis selle peale tuli juba 
mõelda.

\question{Kaua su \emph{crawler}'il aega läks, et 
kogu Eesti veeb üle käia?}

Umbes ööpäev, veeb oli 
tollal väga väike. Kataloogi suurus võis olla paar tuhat linki, mitte rohkem. 
Keskmine koduleht oli ka kolm kuni viis lehekülge. Huvitavamaks läks pärast, kui linkide hulk ulatus juba 
miljoniteni. Ühel hetkel oli käigus selline \emph{crawler}, mis 
töötas paralleelselt paljudes \emph{thread}'ides, aga see oli 
loomulik evolutsioon. 

Estpak Datasse\index{Estpak Data} võeti mind ilmselt tööle seetõttu, et olin ühe sellise kõrvalprojektina 
teinud HTMLi tutvustuse. Pidin seda siis, kui Keila gümnaasiumis
serverit administreerisin, kellelegi õpetama, kuna 
eestikeelset materjali polnud ja tegin ise ühe esimese eestikeelse 
HTMLi tutvustuse, mis võttis läbi kõik üksikud elemendid.

\question{See tuleb tuttav ette, olen sealt ilmselt isegi infot otsinud.}

See HTMLi tutvustus on samal aadressil praegu ka üleval ja ma 
olen üsna kindel, et see on üks vanemaid veebilehti 
Eesti veebiruumis, mis on originaalkujul originaalaadressil. 

\question{Mis aastast see on?}

Aastast 1996. Olen muide ühe projektina teinud veel ka veebipokkeri. Nii et ei saa öelda, nagu mul poleks kunagi huvi olnud 
mänge teha, aga rohkem olen programmeerinud nii-öelda 
veebiasju kui \emph{desktop}'is või masinas töötavaid rakendusi. 

Nende teadmiste baasil mind Estpaki tööle võeti. Tõenäoliselt 
näitasingi neile veebipokkerit ja 
HTMLi tutvustust ning võibolla rääkisin ka seda, et olen 
\emph{crawler}'i teinud. Igal juhul mind võeti tööle.

\question{Kes teil tootepoolt tegi või polnud siis veel niisugust mõistet nagu 
tootejuht?}

Ei olnudki. Piltlikult öeldes pandi mind laua taha istuma, et palun tee. 
Tegelikult oli see mõnes mõttes ikkagi läbi mõeldud. Estpak Data\index{Estpak 
Data} tegi koostööd reklaamiagentuuriga PRC Nord Decor\index{PRC Nord Decor}, mis rentis ruume Kullo majas 
Mustamäe teel. Nii et minu füüsiline töökoht asuski seal. Mul oli arvuti, millel oli püsiühendus 19.2 
kilobitti sekundis. Noore mehena ei huvitanud mind, kuidas raha liigub, vaid ainult 
tehniline pool. Idee seisnes selles, et reklaamiagentuur aitas 
potentsiaalsetel Estpak Data klientidel teha kodulehti ja neile 
reklaami. Üks kolleeg, Tiit Sermann\index[ppl]{Sermann, 
Tiit}, kes Nord Decoris töötas, oli kunagise 
OK jutuka\index{OK jutukas}\sidenote{OK jutukas oli üks esimesi massidesse läinud 
sotsiaalvõrgustikulaadseid rakendusi Eestis. Jututube ehk kohti, kus sai üle 
telneti kaaskodanikega suhelda, oli teisigi, aga 1996. aastal käivitatud OK oli 
esimesi veebipõhiseid jutukaid ja tõenäoliselt omataolistest siin kandis 
suurim. Üheaegselt lobises omavahel kuni 300 inimest ja jutuka esimese 
aastapäeva pidu kajastas isegi toonane Päevaleht.} üks asutajatest. Teine oli
Kaupo Kalda\index[ppl]{Kalda, Kaupo}. Naljakas oli see, et Tiidu alias oli Ott \sidenote{OK tulenes asutajate nimedest Ott ja Kaupo.}, kuigi
pärisnimi oli Tiit. Praegu tundub, et kogu see 
maailm oli tollal nii pisikene, et kui natukenegi seal ringi 
käisid, siis puutusid paratamatult kõikide nende inimestega kokku, 
kes siis toimetasid.

\question{Räägi palun sellest, kuidas te Hoti tegite.}

Kaitseväest tagasi tulles oli Eesti 
Telefon\index{Eesti Telefon} Estpak Data\index{Estpak Data} ära söönud, see 
lakkas olemast. Ma töötasin Eesti Telefoni teleteenuste arenduse
allüksuses, kelle eesmärk oli välja töötada uusi teenuseid, ja neti.ee tegemisega sattusimegi sinna. 

Kontoriruumi jagasin ühe noormehega, kes arendas 
sissehelistamisteenust. Meil vedeles kapi peal üks pisike 
Ascendi sissehelistamiskeskus ja ma küsisin, 
kas võin seda uurida.

\question{Kas selle külge käisid tavalised modemid 
või oli see juba valmislahendus?}

Ei, see oli spetsiaalne sissehelistamiskeskus: see tuli 
installeerida \emph{rack}'i ja panna juhtmed külge, et see hakkaks numbreid 
kuulama ja teenust osutama. Keskust 
uurides avastasin, et see autendib ennast 
vastu sellist autentimisserverit nagu Radius. Edasi uurides sain teada, et Radius on lihtne sõnastikupõhine protokoll, 
ja nii ma programmeerisingi Radiuse serveri, mis suutis 
sissehelistamiskeskust juhtida. Avastasin ka, et sissehelistamiskeskuse 
\emph{firmware} võimaldas teha igasuguseid huvitavaid asju, näiteks sai kohe Radiuse serverist öelda 
sissehelistamiskeskusele, kui kaua konkreetne kasutaja võib ühenduses olla. 

Sellest teadmisest sündis näiteks selline toode nagu Atlas Surf\index{Atlas Surf}, mida 
Eesti Telefon ettemaksulise internetina\sidenote{Sarnane kontseptsioon nagu mobiiltelefoni kõnekaardid.} müüs. Ühesõnaga, see toode sündis 
sellest, et häkkisin väikest sissehelistamiskeskust, mis oli 
mõeldud mobiiliga sissehelistamiseks. Too keskus toetas V.35 protokolli, millest paljud pole ilmselt kunagi 
kuulnud, aga see oli \emph{wideband}-protokoll, mis töötas üle GSMi. 
Kui sul oli GSM-telefon, mida sai arvutiga ühendada, siis see võimaldas V.35 protokolliga
sisse helistada ja kiirus oli veidi suurem kui 
tavalise modemiga üle mobiili vilistades. 

Hüppan korraks veel minevikku. Oli aasta 2000, ilmselt kõik mäletavad 
Y2K\sidenote{Nagu kogenud programmeerijad ütlevad: \enquote{Sinu lapselapsed neavad päeva, mil sa otsustasid oma 
koodi optimeerida}. Kuna pikka aega optimeeriti koodi hoides aastaarvu  kahekohalise 
numbrina, kulutati aastatuhande vahetuse paiku üüratus koguses tööaega ja raha, 
tagamaks, et aasta 2000 ei oleks arvutite arvates võrdne aastaga 1900. Aastal 2038 ootab meid sarnane probleem, kui Unixi aeg oma andmetüübi jaoks liiga suureks kasvab.} probleemi: kardeti, et arvutid 
lähevad katki, sest nende kell lakkab aastatuhande vahetusel töötamast. Ka Eesti Telefonis 
kardeti seda, sest \emph{legacy}-süsteeme oli tohutult palju. Kõik süsteemidega seotud insenerid pidid jääma valvesse. Ma ei mäleta, kuidas 
mul õnnestus sellest kõrvale nihverdada, aga tol hetkel olin sõpradega Soomes 
suusatamas ja lumelauaga mäest alla laskmas. Paar päeva enne 
aastavahetust tuli mulle klienditeenindusest kõne, et enam ei saa sisse 
helistada. Läksin autosse, kus mul oli sülearvuti, panin telefoni arvuti külge, helistasin 
V.35 protokolliga meie privaatkeskusse sisse ja 
hakkasin uurima, miks Hoti kliendid ei saa sisse helistada. 
Tuli välja, et keegi oli viimasel hetkel Y2K hirmus peale laadinud ühe turvapaiga, mis muutis Radiuse serverile 
minevat teadet, mispeale Radius läks katki, kuna sellele tuli tundmatu 
sisuga \emph{dictionary}.

Surfist edasi juhtus nii, et Eesti Telefoni kontsessioonileping 
oli juba lõppenud või lõppemas ja turule tuli Tele2\index{Tele2} Rootsist. 
Tele2 idee oli korrata Eestis täpselt sama, mida Rootsis: nad
soovisid suurelt \emph{telco}'lt palju raha välja imeda. Kuna Eesti 
Telefon üüris ruume ja liine, oli meile teada, et Tele2 paneb oma 
sissehelistamiskeskuseid püsti. Eesti Telefoni juhtkond oli paanikas, 
ma ise ka külastasin laiendatud juhatuse koosolekut, kus 
seda arutati. Tulin sealt üsna mornilt tagasi -- mulle 
tundus, et vanemad kolleegid ei suuda midagi otsustada ega ära teha. Mina 
noore mehena oleksin tahtnud kohe tegutseda. 

Pidasin telefonikõne Priit Pirsoga\index[ppl]{Pirso, Priit}, kes oli selle valdkonna juht 
Eesti Telefonis, ja me otsustasime teha Eesti 
Telefoni osutatavale Atlas Starteri teenusele alternatiivse teenuse, sellepärast 
et Atlas Starter ei sobinud Tele2ga konkureerimiseks. Meil oli vaja 
teenust, kus kasutajate registreerimise protseduur oleks 
automaatne, st kasutaja registreeriks end ise. Kuna kuutasu poleks pärast Tele2 jampsi niikuinii enam olnud, siis 
ainukesed, mis maksid, olid kõneminuti hinnad. Tele2 lootis raha teenida sellest, et
termineerib kõnet ja Eesti Telefon on sunnitud talle 
vahendama kliendi käest küsitud kõneminuti hinda. 

Selle telefonikõne käigus me leppisime kokku, kes, mida ja kuidas teeb, ja et toode saab 
nimeks Hot\index{hot.ee}. Ma olin siis juba arvutist järele vaadanud,
millised huvitavad domeenid olid vabad. Tollal oli veel see 
aeg, kui EENet\index{EENet} ei nõustunud andma ühele ettevõttele mitut 
domeeni, aga ühel mu praegusel kolleegil, Guido 
Kõivul\index[ppl]{Kõiv, Guido}, õnnestus saada EENetist meile
hot.ee domeen, sama skeemiga, nagu 
Tarvi\index[ppl]{Martens, Tarvi} ilmselt kasutas www.ee jaoks. Igatahes kõik käis ruttu ja kaks nädalat hiljem olime \emph{live}'is: 
meil toimus teenuse \emph{launch} ja kasutajaid hakkas registreeruma 
tempoga tuhat tükki päevas.

Sealt saigi hot.ee alguse ja minu teha jäi Radiuse pool. 
Hoti\index{hot.ee} puhul oli meie huvi see, et 
inimesed helistaksid meile sisse. Tollal hakati juba
kasutajatele ka meiliaadresse andma. Kuna varem küsiti meili eest raha, siis 
meile tundus, et lihtsalt niisama meiliaadresse jagada ei tahaks. Siis sai 
tehtud sedasi, et kasutaja sai küll veebipõhiselt konto luua, aga 
meili- ja ka kodulehekonto ei hakanud tööle enne, kui 
registreeritud kontoga oli tehtud vähemalt üks telefonikõne
sissehelistamiskeskusesse. Seda loogikat võimaldas minu \emph{custom} 
Radius, kes kasutajatel järge pidas. 

\question{Ühel hetkel oli hot.ee-s ka veebimeil, eks?}

Veebimeiler oli suhteliselt algusest peale esimese 
kujunduse osa, aga see ei olnud minu programmeeritud, vaid internetist 
leitud vabavara. Me isegi ei \emph{rebrand}'inud enda värvidesse, vaid see oli lihtsalt meie lehelt 
lingitud ja me ise majutaasime teda.

\question{See seletab, miks meil mõned aastad hiljem veebimeileri tegemine 
Hansapangas\index{Hansapank} nurja läks -- meil miskipärast ei tulnud pähe mõtet 
seda lahendust internetist alla laadida.}

Mulle ei tulnud pähe seda ise teha. Küll aga mäletan 
sellist huvitavat protokolli nagu WAP\sidenote{Wireless Application 
Protocol (WAP) oli sajandivahetuse paiku tehtud 
katse luua toona kasinate sidevõimalustega mobiiltelefonide jaoks lihtsamaid 
internetiprotokolle 4. kuni 7. OSI kihini. Muu hulgas sisaldas WAP
erilist \emph{markup}-keelt toonase mobiiltelefoni mõnerealisele ekraanile 
sobivate kasutajaliideste loomiseks.}, mis kujutas endast interneti mobiilivarianti. Selle WAP-meili tegin Hotile küll täiesti 
nullist.

\question{Õnneks see ei olnud väga pika elueaga, sest ka WAP ei kestnud kaua.}

EMT tollane arendusjuht Ando Meentalolt\index[ppl]{Meentalo, 
Ando} kommenteeris minu WAP-meili nii: \enquote{Sa võid ju sinna suahiili keele ka panna, aga 
ilmselt pole sellest väga palju kasu.} Mul sai WAPiga tegelemine 
alguse sellest, et olin saanud endale WAPi-võimelise telefoni (kusjuures see oli vist ainuke telefon, mida ma olen iialgi tööandjalt saanud). See oli suure ekraani ja klapiga
Nokia 7110\sidenote{Tegu oli 1999. aastal uskumatult innovatiivse 
telefoniga: mitut tekstirida näitav ekraan, rullikuga kasutajaliides, T9 
ennustav tekstisisestus sõnumite puhul, vedruga uhkelt lahti hüppav klapp, WAP, 
ebamaiselt küütlev korpus jne. Oma iseäraliku kuju tõttu sai aparaat rahva seas 
hüüdnimeks \enquote{banaan}.}. 

\question{See telefon oli suurepärane põhjus Hansapangale WAPi-põhine 
internetipank teha, sest selle testimiseks pidi ju pank ometigi väljastama ka 
sobiliku seadme.}

Mul juhtus \emph{vice versa}: kõigepealt sain telefoni ja siis 
tuli idee, et äge oleks enda postkasti vaadata sellisel mugaval moel. Ja siis tegingi 
WAP-meili.

\question{Sellega algab juba uus sajand ja sellest räägime võibolla 
mõni teine kord. Lõpetuseks küsin, mida sa praegu teed?}

Praegu teen Bolti\index{Bolt}\sidenote{Endise nimega Taxify ja asutatud kui mTakso.} serveri infrastruktuuri. Minu üks kauaaegseid 
kolleege Eesti Telefonist Tarmo Kople\index[ppl]{Kople, Tarmo} on 
üks nendest inseneridest, kellega alustasime Bolti 
serverimajandust algusest peale. Ja kui algul oli kliente ja sõite tuhandeid kuus, siis nüüd juba miljoneid.


\chapter{Kersti Kaljulaid}
\index[ppl]{Kaljulaid, Kersti}

\question{Alustaks kohe sealt, kust asjad ikka algavad ja kust me oleme kõiki 
neid jutuajamisi alustanud. Kuidas Teie jõudsite arvutite 
juurde?}

See juhtus üsna ammu, Nõukogude Liidu päevil 
Õpilaste Teaduslikus Ühingus\index{Õpilaste Teaduslik Ühing}. Ma arvan, et 
päris paljud minuvanused inimesed, kes on hiljem töötanud Eesti e-riigi või 
\emph{start-up}-kogukonnas, teavad, mida tähendab 
Küber\index{Küber} või Küberi arvutuskeskus. Tartu 
Ülikoolil\index{Tartu Ülikool} olid samuti olemas arvutuskeskused. Nii põhjas 
kui ka lõunas otsustasid täiskasvanud millegipärast, et lasevad lapsi sinna 
mängima. 

\question{Millised täiskasvanud?}

Õpilaste Teadusliku Ühingu eestvedajad. Näiteks Peeter 
Lorents\index[ppl]{Lorents, Peeter}, kes juhtis matemaatikasektsiooni. Ma ei tea, kes Tartus seda eest vedas, küll aga seda, et ka Tartu 
koolinoortel, näiteks Unineti\index{Uninet} Taavi Talvikul\index[ppl]{Talvik, 
Taavi} oli ligipääs Tartu Ülikooli arvutipargile. Toimus instinktiivne 
õpe, mis viis meid Õpilaste Teaduslikus Ühingus aruteludeni, 
kuidas kirjutada sellist asja, mida keskserveril oleks mõnusam analüüsida. 

Tollal oli nii, et üks asi arvutas ja ümberringi olid terminalid, kus me 
oma koodi kirjutasime. Tolleaegsed masinad olid mitteselektiivsed -- 
need ei otsinud, kes meist efektiivsema rea on kirjutanud, et seda siis 
töödelda. Aga meile tundus, et äkki oleks võimalik üksteisega kunagi 
võistelda, kes kirjutab sellise asja, mida keskserveril 
oleks mõnusam analüüsida. Mäletan laadilisi debatte, ükskord toimus
Õpilaste Teadusliku Ühingu suvelaagris\index{Õpilaste Teaduslik 
Ühing!Suvelaager} vist isegi öine matemaatikadebatt. 

Rõhutan, et mina ei kuulunud matemaatikasektsiooni, aga mind võeti kuidagi 
kampa. Olin üheteistkümnendas klassis ühingu 
teaduslik peasekretär, aga ise tegelikult 
ornitoloogiasektsioonist. Mulle meeldis Linné-aegne 
bioloogia,\sidenote{Carl Linnaeus, pärast aadliseisusse tõstmist 1761. aastal 
Carl von Linné (1707--1778), oli Rootsi teadlane, kes formaliseeris organismide 
nimetamise süsteemi ja keda tuntakse moodsa taksonoomia isana.} mis oli 
koolilapsele kättesaadav. Linnud, loomad, taimed -- kõik see viis mind 
laia maailma, küll kuuendikul planeedist, aga olümpiaadidel käies
sai reisida. 

Igatahes Õpilaste Teaduslikus Ühingus tekkis mul kokkupuude Küberi pundi ja 
arvutusvõimsusega.

\question{Nii et isegi ornitolooge viidi arvuti juurde?}

Ei, ornitolooge ei viidud, mul lihtsalt olid matemaatikasektsioonis sõbrad. Aga 
Lorents\index[ppl]{Lorents, Peeter}, Engelbrecht\index[ppl]{Engelbrecht, Jüri} ja teised ei teinud selles mõttes vahet, et me võisime olla 
neliteist-viisteist-kuusteist, aga saime osaleda täiskasvanute 
akadeemilistes mängudes.

\question{Neljateistaastasel noorel, eriti ornitoloogiahuvilisel, 
on miljon muud asja teha. Mis tõmbas just arvuti poole?}

Esiteks oli see põnev maailm. Teiseks, minu ainuke ornitoloogiaalane 
publitseeritud teadustöö, mis tegeles vainurästa pesitsuskommetega, viis mind statistilise tööni ja ilmselt seeläbi arvutini. See ei 
käinud nii, et vaatasin metsas, kus vainurästas elab -- selle olid 
teised inimesed ära teinud. Eestimaa Ornitoloogiaühingus\index{Eesti Ornitoloogiaühing} (või kuidas iganes Vene ajal seda kogukonda ka ei nimetatud\sidenote{Eesti Looduseuurijate Seltsi ornitoloogiasektsioon.}) oli 
kogunenud meeletus koguses pesitsuskaarte, kõik täiesti 
süstematiseerimata materjal. Minu akadeemiline tegevus Õpilaste Teaduslikus 
Ühingus seisneski selles, et otsisin statistiliste meetoditega erinevaid 
korrelatsioone. Näiteks tuli sealt välja, et linnas pesitseb vainurästas 
kõrgemal kui kuskil looduslikus biotoobis. Vainurästast ennast polnud vaja selle töö jaoks isegi mitte metsas ära tunda. Mitte 
et ma ei tunneks, aga arvutamiseks seda vaja ei olnud.

\question{Milles akadeemilised mängud seisnesid? Mis tüüpi ülesandeid te 
arvutiga lahendasite?}

Lihtsaid programmeerimisharjutusi, mida täna teevad paljud lapsed algkoolideski. Mõni täiskasvanu oli meil alati juures või siis tulime ka ise selle peale. Mina kaugemale väga ei jõudnudki, sest ma 
ei olnud matemaatikasektsioonis.

\question{Millised inimesed matemaatikasektsioonis olid? Kas
tüüpilised nohikud?}

Ei, seal oli erinevaid. Näiteks Tarvi Martens\index[ppl]{Martens, Tarvi} ja 
Tarmo Uustalu\index[ppl]{Uustalu, Tarmo} on täitsa erinevate kategooriate inimesed. Oli mitmesuguseid matemaatikahuvilisi noori erinevatest Eesti 
nurkadest. Minu arust ei ole olemas sellist stereotüüpi, mida alati 
otsitakse. Gruppidevahelised erisused on teadupärast väiksemad kui 
grupisisesed.


\question{Kas Õpilaste Teaduslikul Ühingul oli ka võrgustiku loomise 
funktsioon?}

Kindlasti. Olid erinevad sektsioonid: matemaatika, loodusteaduste, 
geograafia ja ajaloo, sealhulgas NSV Liidu ajaloo sektsioon, mis 
pandi ükskord kinni, mille üle mõned punasemad noored olid väheke nördinud. Näiteks Teet 
Jagomägi\index[ppl]{Jagomägi, Teet}, kes on tänaseks selgelt IT-ettevõtja, juhtis 
geograafiasektsiooni. Päris palju praegusest umbes 50aastaste kogukonnast on sealt ühel või teisel viisil läbi käinud. Kõik 
tundsid kõiki nagu ikka tollal.

\question{Mis siis sai, kui teaduslik ühing lõppes?}

Läksin Tartu Ülikooli. Üksiti oli see ka hüvastijätmine 
Linné-aegse bioloogiaga, sest minu juhendaja Raivo Mänd\index[ppl]{Mänd, 
Raivo} ütles, et tuleb õppida uusi asju, mis toovad tulevikus leiva lauale. Sellepärast ongi minu eriala geneetika, täpsemalt plasmiidigeneetika 
või bakterigeneetika. Bioloogias on ju keemia, füüsika ja matemaatika 
kõik koos, nii et stuudiumi jooksul tugevnes kindlasti minu arusaam 
matemaatikast kui kõike kirjeldavast ja kõiges toeks olevast teadusharust. 
Matemaatika on minu jaoks nagu keel. Ma ei ole selles ülearu osav, aga vajaduse 
piires olen suutnud toimetada.

\question{Tartu Ülikooli Arvutuskeskuse\index{Tartu Ülikool!Arvutuskeskus} 
kohta on räägitud, et seal käis koos üsna 
kirju seltskond teoloogidest jumal teab kelleni. Kas Teil oli selle kohaga ka 
kokkupuudet?}

Arvutuskeskuse seltskonna kirjusus tulenes muu seas sellest, et matemaatika 
ja füüsika olid erialad, millele konkurss Tartu Ülikoolis puudus. Seal oli 
alati kohti rohkem kui rahvast ja tihtipeale pugesid nendesse teaduskondadesse 
peitu ka inimesed, kes iga hinna eest tahtsid näiteks vältida Nõukogude 
sõjaväge. Minu aastal ülikooli astunud füüsikutest vist üks lõpetas füüsikuna. 
Küll aga astus sinna sisse näiteks Anzori 
Barkalaja\index[ppl]{Barkalaja, Anzori}, kindlasti teise eriala inimene. 

Seltskond oli kirju, aga arvutiteadus ongi suuresti 
interdistsiplinaarne, mitte spetsiifiline. Olen 
ise kodus märganud, et see on kuidagi pärilik -- minu esimene abikaasa 
ja vanem poeg elavad mõlemad arvutimaailmas, elavad ja hingavad 
bitte ja baite. Kuigi seda on ju võimatu näidata, kuidas pärilikkus saab 
millegi nii tehnogeensega koos käia, tundub mulle, et mingisugune ajutüüp 
peab selleks siiski olema. Ka vanemal tütrel on arvutiinimese aju. 
Tol ajal hakkaski selguma, et osal inimestel on 
arvutiinimeste ajud -- nad tõmbusid arvutuskeskusesse kokku, said 
üksteisest aru ja hakkasid vähehaaval kaotama sidet humanitaarsema poolega 
ühiskonnast.

\question{Mõtlesin, et lause lõpeb sellega, et \enquote{hakkasid kaotama 
sidet reaalsusega}, sest ka seda juhtus seal majas kergesti \ldots}

Ei, seda mitte. Arvan, et osa 
inimesi see maailm ei kõneta ja teisi intuitiivselt kõnetab. 
Olen kogu aeg tundnud, et ma ise arvutimaailmas sees ei ole, aga võibolla suudan kahe 
maailma vahel natuke tõlkida. See tunnetus on olnud päris varasest 
noorusest peale. Olen telekomisektoris töötanud, telefonijaam on ju nagu
arvuti: seda tuli samamoodi konfigureerida ja programmeerida. Töötasin üheksakümnendatel
palju koos inimestega, kes pidasid oma tööks arvutitega töötamist; ma ise seda ei teinud, küll aga püüdsin luua neile töötingimusi mõnes 
ettevõttes. Ma olen olnud nagu piirpinnal kõndija.

\question{See on väga põnev, sest seestpoolt vaadates tunduvad mõned 
asjad ilmselged ja mõned ebaselged ning kõrvalt vaadates võivad 
asjad paremini paista. Kuidas Te 
bakterigeneetikast ühtäkki telekomi sattusite?}

See oli imelihtne. Kui ma lõpetasin ülikooli, siis täpselt samal päeval tuli 
Eesti kroon\sidenote{20. juunil 1992.} ja ülikoolide pakutud tulutase 
(jäin pärast lõpetamist ülikooli tööle) oli nii väike, et sellega ei olnud 
võimalik lasteaiatasusidki katta. Sain aru, et 
uues Eestis läheb kas väga kaua aega, kuni see valdkond hakkab ära tasuma, või 
siis tuleb minna Eestist ära. Paljud kursusekaaslased lahkusidki Eestist 
ja neil kõigil on olnud väga edukas karjäär. Paljud on tulnud ka
tagasi ning on nüüd professorid Tartu Ülikoolis ja Eesti Maaülikoolis. Nad
ehitasid oma karjääri mujal üles ning kui Euroopa Liit asus 
laienema ja ka meie teadustaristut üles ehitama, siis neil oli super võimalus 
oma kolmekümnendate keskpaigas Eestisse naasta ja oma näo järgi 
laboreid ja uurimiskeskusi kujundada. Selles mõttes tõsine võitjate põlvkond.
 
Mina ei tahtnud Eestist ära minna, sest meil olid väiksed lapsed ja ma 
soovisin, et nad oleksid eestlased. Läksin puhtalt 
raha pärast teadusest erasektorisse. Töötasin pisikeses ettevõttes, mis paigaldas Siemensi 
telekommunikatsioonijaamu. Tööle võeti mind selleks, et tõlgiksin materjale eesti keelde, sest korralik inglise keele 
oskus ei olnud tollal nii levinud. Siis aga selgus, et tõenäoliselt kõlban 
päris hästi ka müüma. Ja kuna väikestel ettevõtetel on üks juht, siis 
oledki nii müügidirektor kui ka lihtsalt direktor. 

\question{Mis aastal see oli?}

Läksin sinna 1994. aasta sügisel. 

\question{1994 oli veel suhteliselt hull aeg. 
Kes tol ajal üldse Siemensi jaama endale paigaldas? Isegi analoogtelefon oli 
mõnes kohas haruldane.}

Metsikult pandi. See oli just see aeg, kus saadi aru, et büroohoonetes, 
ülikoolides, raamatukogudes ja igal pool mujal on tuppa telefoni vaja. 
Mul endal on vastupidine mulje, et Siemens, Ericsson ja 
väiksematest tegijatest näiteks Panasonic 
müüsid terve linna täis alates rahvusraamatukogust ja lõpetades 
pankadega. Turgu oli kõvasti! 

Siis tuli Eesti esimene riigihangete seadus ja
pidi hakkama selle järgi pakkumisi koostama. Tundus tohutu põnev, aga ka pisut 
hirmutav, sest varem ei olnud niimoodi käinud, et teed pakkumise ja siis 
loetakse kõik ette. Mul on meeles, kuidas istusime vist Tallinna Vanglas sealse  
telefonikeskjaama hankel ja selline tunne oli, et ei teagi, kas siit välja enam 
saab. Mitte et oleksime midagi valesti teinud, 
aga maailm muutus, süsteemi tekkis selgroogu ja struktuuri 
juurde.

\question{Te jõudsite pärast ka Eesti Telekomi, aga mina mäletan tollest ajast, et 
see oli mingisugune õudne monstrum!}

See polnud siis veel Eesti Telekom, vaid Eesti Telefon\index{Eesti Telefon} ja 
mina töötasin sellises peenes kohas nagu Eesti Telefoni äriklienditalitus. Läksin erasektorist sinna, sest seal tundus 
olevat rohkem karjäärivõimalusi. Olin väikses ettevõttes tipus ja tundsin, et 
tahaksin edasi liikuda, suuremat struktuuri vaadata. 

Eesti Telefonis oli tollal päris keeruline. Ükskord öeldi mulle, et sel kuul ei saa rohkem müüa, sest meil sai 
sisseostuplaan täis. Siis ma olin jube kuri ja tegin üheselt selgeks, et ma ei 
taha mitte kunagi enam sellist väidet kuulda -- kui müüme, siis müüme, ja 
kui te ei taha, siis ma lähen midagi muud tegema. Aga müüsime küll.

\question{Eesti Telefon oli minusuguste nohikute jaoks 
üsna õudne ettevõtte: küll ei suutnud traati pakkuda ja ei müünud mingil hetkel
isegi internetti, väites, et telefoniliini peal ei 
peagi internet töötama, sest liin on helistamiseks.}

Ma ei tea äriklienditalituse loojate kaalutlusi, aga arvan, et asja 
mõte oligi tuua sinna struktuuri üks üksus, mis hakkaks senist organisatsioonikultuuri seestpoolt õõnestama ja muutma. 
Tolle talituse rahvas pidi müüma täitsa tavalistele eraettevõtetele ja 
 neidsamu jaamu, mida Siemens ja Ericsson ka eraldi müüsid. Ma täpselt ei mäleta, kuidas Valdo Kalm\index[ppl]{Kalm, Valdo} selle 
talitusega seotud oli, aga igal juhul oli ja aitas
ettevõtte muutumisele kindlasti väga palju kaasa. 

Algus oli minu jaoks keeruline. Mõni inimene küsis, miks mind kunagi 
oma laua taga ei ole. Minu arust müügijuht ei peagi olema oma laua taga. Aga tasapisi see kultuurimuutus tekkis.

\question{Kultuuri mõttes oligi huvitav aeg, kui sidevaldkonnas kombineerusid
äri, akadeemiline kogukond ja häkkerite-nohikute maailm. Kas see paistis Eesti Telefoni poolelt ka välja?}

Muidugi paistis, sest inimesed olid samad. Võtame või Taavi 
Talviku\index[ppl]{Talvik, Taavi}, minu esimese abikaasa, kes tuli Tartu 
Ülikooli arvutuskeskusest\index{Tartu Ülikool!Arvutuskeskus} läbi 
Valitsusside\index{Valitsusside}. Seejärel tegid nad Andes 
Baumaniga\index[ppl]{Bauman, Andres} oma ettevõtte Uninet, mis hiljem müüdi ära ja millest sai 
Elisa. 

Selles mõttes oligi üks maailm. Mis seal lõppude lõpuks vahet on, kas helistad või saadad muid 
andmeühikuid? Digitaalne tehnoloogia tol ajal just tuli. Tekkisid 
probleemid, kuidas tagada läbilaskvus ja ühenduste laius -- kogu see maailm 
hakkas vaikselt arenema ja kasvama. Minu meelest 
pole see Eestis kunagi eraldi olnud. Kandja pool ei olnud kindlasti eraldi ja sisu 
poole ettevõtteid tollal ju eriti ei olnudki veel. Esimene internetipank tekkis vist aastal 1994? Aastal 1997 oli juba e-maksuamet.

Sisuteenused hakkasid ka üsna kiiresti tulema, aga siis algas kohe ka see 
võistlus, et toru on küll olemas, aga sisu tahab laiemat, ja kui toru saab laiemaks, siis tahab sisu 
veel laiemat. Ma ise olin selgelt toru, mitte sisu poolel.

\question{Rääkisime, et stereotüüpe ei ole. 
Ometigi on Eesti Vabariigis olnud laialt käibel niisugune mõiste nagu 
\enquote{patsiga poiss}. Mis inimene see on? Mis teda iseloomustab?}

Neid on väga erinevaid. Eesti Telekomi aegadest ma ei mäleta peaaegu kedagi 
peale Valdo Kalmu\index[ppl]{Kalm, Valdo}, kes seal minuga veel koos töötasid -- palun vabandust endiste kolleegide ees! Aga näiteks mäletan Uku 
Kuuti\index[ppl]{Kuut, Uku}, kes oli meil süsadmin, patsiga poiss. Ja 
kui tema tuppa läksid, sest midagi oli paigast ära, siis tal muusika alati käis. 
Tundus küll nagu teine maailm võrreldes paljude teistega, aga kindlasti 
oli ka pöetud habemega rahvast.

Need olidki üsna erinevad seltskonnad. Kui müüsid telefonijaama ja tegelesid 
pankadega, siis oli selgelt näha, et pankade tehnikajuhid vastutasid juba 
tollal suhteliselt suure struktuuri püstihoidmise ja edasiarendamise 
eest, olid hästi makstud ja ei erinenud millegi poolest pankade
raamatupidajatest. Neid ei kutsutud siis veel CTOdeks, aga seda nad sisuliselt 
olid ja ei erinenud muude valdkondade eest vastutajatest. 

Ja siis oli iseõppinud vendi, kes olid kuidagipidi (ega seda ülikoolides ei õpetatud) 
ise arvuteid pidi ringi nuhkinud ja saavutanud oskuse hoida asjad töös. 
Nende hulgas oli jah võibolla seda stereotüüpi, et nad
suhtlevad parema meelega masinaga. Samas ei olnud masin tol ajal nii huvitav 
suhtluspartner ja ei olnud võimalust internetti päris ära kaduda. Ma arvan, et see on üle võimendatud, 
kuidas seal kastis saab kogu ööpäeva ära sisustada. 

Ühesõnaga, inimesi oli igasuguseid.

\question{Meil on olnud teistega juttu sellest, et teatava peakujuga 
inimesi tõmbas Tartu Ülikooli arvutuskeskusse -- võibolla see tõmme on 
ühine nimetaja?}

Kindlasti. Need, kes Tallinnas 
Küberis\index{Küber} koos käisid, on kõik selles sektoris tänini leitavad, nad on olnud
püsivamad ja järjepidevamad kui mina. Midagi ju on, mis tõmbab meid
matemaatika või keelte juurde. Need on erinevad 
asjad. Võibolla praegune 
keeleinstituudi\index{Eesti Keele Instituut} direktor Arvi Tavast\index[ppl]{Tavast, Arvi} on mõlemal poolel 
kõndija: ühtpidi IT-tegija ja teistpidi on teda sügavalt huvitanud 
keeled ning need kaks asja saavad tänases maailmas kokku. Enamik inimesi 
kipub siiski olema paremal või vasakul. Ma ei tea, miks.

\question{Kui vaadata kasvõi inimesi, kellega ma selle raamatu raames rääkinud olen, siis 
naisi on vähe. Tol ajal oligi neid selles valdkonnas vähe. Miks?}\phantomsection\label{sisu:tydrukud}

Naised on alalhoidlikumad ja toimetavad valdavalt sektorites, mis on 
sisse töötatud. Paratamatult on nad ka karjääri mõttes 
alalhoidlikumad. Minulgi oli ühel hetkel päris 
palju ideid ja valikuid, mida võiks teha, näiteks kas tekitada oma butiik ja 
hakata seda arendama. Ma ei teinud seda sel lihtsal põhjusel, et pidin ülal pidama kaht alaealist last -- olin tol hetkel üksikema. See oli teadlik valik, sest mul oli tarvis rohkem kindlustunnet ja struktureeritud elu. Ma 
ei saanud endale lubada, et olen võibolla järgmised viis kuud ilma palgata. 

Kuna see oli tollal uus valdkond, siis naised lihtsalt ei võtnud neid riske. Võibolla 
seiklusjanu oli ka esialgu väiksem ja ega keegi ei näinud ju 
sellega ka teadlikult vaeva. Me räägime sügavatest 1990ndatest! Toon ühe
eheda näite. Mina müüsin siin Siemensi Hicom 300,\sidenote{Siemensi paindlik 
telefonijaamade sari.} Soomes tegeles müügiga üks Tiina. Kord läksime 
Siemensi suurte telefonikeskjaamade müügimeeste kokkutulekule, kus olid peale meie ainult 
mehed, kes küsisid uskumatul ilmel: 
\enquote{Kas te tõesti müüte neid suuri telefonijaamu?} Me ei saanud aru, miks 
me ei võiks seda teha. Tol ajal ei olnud see naiste maailm. 

\question{Kui Eestis oli selline avantürism arusaadav, siis muu maailm oli selles mõttes ikkagi teistsugune?!}

Võtame Kesk-Euroopa, näiteks Saksamaa. Seal on
minu põlvkonnas veel päris palju koduperenaisi, just Lääne-Saksamaal. 
Idas ei ole. Statistiliselt on Ida-Saksamaa naiste 
pensionid kõrgemad, samuti lahutuste arv, sest nad saavad 
seda endale lubada erinevalt läänest. Kui me kujutame ette, et 
üheksakümnendatel oli ärikultuuris tohutu võrdõiguslikkus, siis ma kaldun arvama, 
et see pole sinna päriselt jõudnudki ja selle nimel võideldakse. Meil ei maksa luua endale illusioone, 
et seda tööd ei pea enam tegema. Selles mõttes on Taavi Kotka\index[ppl]{Kotka, 
Taavi} Unicorn Squad, Rakett ja teised sarnased ettevõtmised tüdrukute 
toomiseks tehnoloogia ligi ühiskonnale tohutu väärtusega. Ei 
ole mitte ühtegi põhjust, miks tütarlaps ei võiks toimetada tehnoloogiarikastel 
aladel.

\question{Tütarlapse jaoks võib ehk olla vähem loomulik see, et ta 
magab kontorilaua all, sest uni tuli peale?}

Ära hakka seda \enquote{loomulik või vähem loomulik}! Ei ole niimoodi, kuigi paljud võivad sedasi arvata! Mis seal vahet on? Sul 
on 20-aastane vaba inimene, lapsi ega peret ei ole -- ükskõik, kas ta on mees või naine. Kui 
ta seal kontorilaua all magab, siis teksad on nagunii jalas, soengu ja 
seelikuga sinna ju ei lähe. See ongi see alateadlik stereotüüp, isegi kui see ei ole pahatahtlik. Aga see on täiesti olemas, nagu ka sinu 
väites!

\question{Tõsi, ka punkareid oli igasuguseid.}

Jah, samamoodi võib olla insenere ja keda iganes.

\question{Mul lihtsalt tuleb silme ette üks konkreetne habemega punkar, kes 
vedeles niimoodi hommikul laua all. Aga tõepoolest, see on minu stereotüüp ja 
kujutluspilt, et see on habemega punkar -- miks ta ei võiks olla 
teistsugune!}

Mu tütar rääkis ülikooliajast loo, kuidas üks ettevõte otsis tööjõudu. 
Päris paljud käisid ennast pakkumas ja võeti üks poiss, kes oli ülikoolis 
silmnähtavalt teistest laisem ja kehvema õppeedukusega. Mõne aja pärast, kui ta oli kohanenud, küsis poiss
tööandjalt: \enquote{Kuule, meilt kandideerisid veel need ja 
need inimesed, miks nemad ei saanud?}. Mille peale talle öeldi: 
\enquote{Vaata ümberringi, näed sa siin mõnda naist?} Selline tõrjuv kultuur, 
eks ole. See ei ole nüüd side-, telekomi- ega ka IT-ettevõtte, vaid lihtsalt insenerikultuuri näide Eestist. Ja need inimesed, 
kellest ma räägin, on praegu 32--33, mitte 50-aastased.

\question{Mugavam on palgata omasugust. Iseküsimus, kas ka kasulikum.}

Ei ole, sellepärast et statistiliselt tuleb naiste pähe 50 protsenti headest 
ideedest ja meeste pähe 50 protsenti.

\question{Pigem on isegi teistpidi \ldots}

Ma ei lähe sinna kunagi. Me ei peaks ütlema, 
et naised on kuidagi teistmoodi juhid või insenerid. Väidan, et oleme 
ajupotentsiaali mõttes võrdsed. Paraku on ka minu käest 
küsitud: \enquote{Kuidas siis nüüd nii, et naine juhib elektrijaama?} 
Olen siis väga otsekoheselt vastu küsinud, et kuule, räägi nüüd, 
mida sa selle tilliga teed jaama juhtides? 

\question{Kuna meie IT-värk tuleb sellest seltskonnast, mis oli
faktiliselt ühele poole kallutatud, siis kas see on meid kuidagi 
digiühiskonnana tagasi hoidnud või edasi aidanud või üldse mingit mõju avaldanud?}

Tegelikult ei ole. Olen meie digiriigi arengu peale palju mõelnud. Miks me oleme nii 
teistmoodi, kuigi mujal on palju tugevama IT-sektoriga erasektor kui meil 
Eestis? Ühiskonda muudab ikkagi riik, midagi ei ole teha. 
Erasektoris võidakse teha geniaalseid asju, aga \emph{mainstream}'imise 
maailmameistrid on kõik riigid. See on see koht, kus riik peab midagi 
tegema hakates kaasa võtma kõik: vanad, noored, mehed, naised! Ja 
selle muudatuse on Eesti ära teinud.

Priit Alamäe\index[ppl]{Alamäe, 
Priit} vist tõi kasutusele termini \emph{digitally transformed nation}. Siin on 
nüüd see koht, kus me läksime teist teed kui teised maailma riigid. Kui 
riik hakkab tundma huvi mõne sektori võimaluste kasutamise vastu 
riigiteenuste osutamiseks, siis see hakkab päriselt ühiskonda muutma 
ja kujundama. Meil juhtus see, et ühest hetkest olid kaasatud 
mehed, naised, lapsed, vanurid, ja tekkisid ka positiivsed kõrvalefektid. Näiteks koroonapandeemia tekkides meil ju ei olnud häda, et 
70aastased ei saa pangas käidud või 
telekomilepinguid uuendatud, sest nad olid 50aastased, kui ID-kaart tuli. Sellised
positiivsed kõrvalefektid on olnud kogu ühiskonna jaoks hästi suured.

\question{Miks meil juhtus niimoodi?}

Sellepärast, et meil ei olnud midagi. Ajalooliselt oli ju esimene 
\mbox{e-teenus} \mbox{e-maksuamet}. Kas sa kujutad ette, et Eesti inimesed oleksid 
nõustunud seisma tundidepikkustes sabades, et riigile oma maksud ära viia?

\question{Praegu enam ei kujuta.}

Aga siis ka ei kujutanud. Oli tõenäone, et maksulaekumised ei ole ülearu head, 
kui loodad sellisele asjale. Õnneks oli selline aeg, et 
e-pangandus ju oli ja sai teha e-maksuameti. Keegi ei taha ju maksuametnikku näha! 

\question{Ma ei ole kuskil mujal näinud sellist usaldust 
keskmise bürokraadi ja \emph{hardcore} inseneri vahel. Ühest küljest 
insener usaldab, et bürokraat ei keera asja kihva. Teisest küljest 
bürokraat usaldab, et kui tehnik hakkab rääkima XML-sõnumite 
vahetamisest, siis ta päris udu ei aja ja tarnib tulemuse. Kust meil see usaldus
tuli? IT-kogukond isekeskis küll teadis, tundis ja usaldas üksteist, 
aga kuidas laiem ühiskond juurde tuli?}

Ega ei tulnudki. Kui ma läksin 1999. aastal peaminister Laari\index[ppl]{Laar, 
Mart} juurde tööle, siis tema tellimus oli suhteliselt mittespetsiifiline. Ta võttis oma nõunikud kokku ja ütles: \enquote{Nüüd 
on nii, et me kõik saame aru, et Eestis palgad kasvavad ja varsti me ei 
ole enam rikka mehe kuluefektiivsuse lahendus. Vaadake ringi, kuhu edasi minna.} 
Sealt algas see, et kui minu juurde tuli Andres 
Metspalu\index[ppl]{Metspalu, Andres} jutuga, et oleks vaja teha 
Geenivaramu\index{Geenivaramu}, siis sai kõik rattad käima lükatud. Eiki 
Nestor\index[ppl]{Nestor, Eiki} tuli ka appi ja tegime inimgeeniuuringute 
seaduse. 

Teise valdkonna -- ID-kaardi -- pakkus välja Linnar 
Viik\index[ppl]{Viik, Linnar}. Tol ajal 
hakkas juba tekkima ka sisuteenust ning Kaarel Tarand\index[ppl]{Tarand, Kaarel} nägi 
väga hästi seda pilti, kuhu see sisu pool kommunikatsioonis ja mujal
minemas on. ID-kaardi idee vist tekkiski sellest, et 
pangad ei olnud ju pikaajaliselt nõus võtma vastutust, et riik jooksutab oma 
e-teenuseid nende platvormidelt. Seetõttu tuligi teha ID-kaart, aga see 
sündis ühise veenmistöö tulemusel. 

Jube raske oli veenda rahandusministeeriumit, kes tahtis kohe 
teada, kus on ROI.\sidenote{\emph{Return on investment} -- investeeringu 
tasuvuse näitaja.} Täna tundub see naljakas küsimus: \enquote{Mis 
mõttes, vaadake, milline e-riik meil on!} Aga kust Linnar\index[ppl]{Viik, 
Linnar} tollal need arvud oleks võtnud? Põhjendus, 
millega tegelikult ID-kaarti valitsusele müüdi, oli ju absurdne: e-valitsus. See, et valitsuses olid 
arvutid laua peal, oli toonud meile nii palju tasuta artikleid 
välisajakirjanduses, et see süsteem oli ennast kolme kuuga tasa teeninud 
võrreldes sellega, kui oleksime lihtsalt ostnud \emph{Estonia -- Positively 
Transforming}\sidenote[][-1cm]{2002. aastal käivitatud suur ja mitmesugust 
meediatähelepanu pälvinud Brand Estonia kontseptsioon \enquote{Welcome 
to Estonia} põhiline tunnuslause.} lehepinda. Selle argumendiga 
tehti ID-kaart! 

Nii et see arvamus, et kõik tulid Linnar 
Viigi\index[ppl]{Viik, Linnar} ja teistega kohe kaasa, on vale. Töötasid teised argumendid: kulu ei olnud nii suur ja 
Vabariigi Valitsuse ruum oli tõepoolest väga palju välismaist positiivset tähelepanu 
saanud. Seejuures teenimatut, sest nendesamade Saksa ja Soome inimeste erafirmades oli intranet ju täiesti tavaline. 
Eestis samuti -- Eesti esimene intranet hakkas tööle Postimehe\index{Postimees} toimetuses aastal 1991 või 1992. Aga riikide tasandil seda ei tehtud ja see siis oligi sirgelt ID-kaardi müügiargument: 
võime saada palju välismaist tähelepanu.

\question{Ilmselt oli maitse suus, et korra oleme juba saanud, küllap 
nüüd ka!}

Just. Ja asja geniaalsus seisnes selles, et näiteks Saksamaal pead 
siiamaani omale digi-IDd taotlema, aga meie pistsime selle 
lihtsalt kõikidele kaardi peale. Kasutavad või ei kasuta, aga ega see liiga ka ei 
tee. Tegime ainult ühte tüüpi kiibiga ID-kaardi ja see oli geniaalselt 
õige lahendus.

\question{Minu teada ei ole keegi teine seda ka eriti järele teinud.}

Nüüd vist ikka juba on, aga mitukümmend aastat hiljem \ldots

\question{Kui palju oli arusaamist, et 
tegemist on geniaalse lükkega, kui palju pikka visiooni ja kui palju 
praktilist kaalutlust, et \enquote{paneme lihtsalt käima}?}

Ma mäletan seda arutelu. Äkki ei teeks, äkki teeks. Teeks 
kõigile. Maksab nii palju. Kui teeme osadele, kas maksab vähem? Ei 
maksa vähem, vaid tegelikult rohkem, kui on erinevad süsteemid. 

Oli tõesti hulk inimesi, sealhulgas Infotehnoloogiafirmade 
Liit\index{Infotehnoloogiafirmade Liit} ja Linnar 
Viik\index[ppl]{Viik, Linnar} ning ilmselt teisigi, kes ütlesid, et võtab küll aega, kuni teenused peale 
lähevad, aga anname kõigile. Analoogselt olime ühel hetkel 
otsustanud, et mitte keegi ei hakka enam siin riigis sularahas palka saama, vaid kõik said 
omale pangaarved -- kõik pidid tegema ja raha hakkas minema panka. Kuigi ka siis oli algul palgapäeval ATMi 
juures saba, sest inimesed võtsid kogu raha välja. Funktsionaalsus ei läinud kohe käima. 

Seda analoogi tõime 
palju oma aruteludes ka põhjendusena, miks peaks ID-kaardi tegema 
universaalsena. Mart Laar\index[ppl]{Laar, Mart} lükkas seda hoogsalt tagant ja tihtipeale tuli lükata just nimelt neid, kes 
hästi hoolega raha lugesid -- Reformierakonda.

\question{Kas see lükkamine käis tal põhimõtteliselt sellesama arusaama pealt, 
et see on strateegiliselt oluline asi?}

Jah, tema uskus seda, Linnar\index[ppl]{Viik, Linnar} oli suutnud ta seda 
uskuma panna. Mina pidin selles protsessis olema kaasas sellepärast, et 
olin majandusnõunik (muidu oli peaministri büroos geeniseadus nagu rohkem minu laps). Minule ütleski rahandusminister, et unustage ära, sest te ei 
suuda mulle ROId näidata. Siis me mõtlesimegi välja selle, et \enquote{aga 
see asi tasus ennast ju kolme kuuga ära!}.

\question{Järelikult vastab tõele legend sellest, kuidas 
Linnar\index[ppl]{Viik, Linnar} ja Mart\index[ppl]{Laar, Mart} olla istunud laua 
taga ja Mart olla öelnud: \enquote{Linnar, mis me teeme?} ja Linnar olla 
vastanud: \enquote{Teeme interneti.}}

Vastab tõele küll, aga Mardil ei olnud nii kitsas vaade. Ta tahtis lihtsalt 
teada, et öelge kõik midagi, mida me võiksime teha.

\question{Huvitav kombinatsioon praktikast ja visioonist. Tollane 
Eesti Vabariik oli ikka oluliselt teistsugune kui praegu, muresid oli 
miljon!}

Oligi. Jään ka selle juurde, et me ei saa kogu seda au endale võtta. 
1999. aastal saime ju Euroopa Liidu teadus-arendusprogrammi liikmeks. Siis hakkas Euroopa Liit valmistama meid ette 
liitumiseks ja institutsioonide ning igasuguste muude asjade ehitamiseks tulid rahad peale. Väidan, et Euroopa Liidu rolli ei tohi alahinnata -- 
digiriiki on oluliselt lihtsam ehitada, kui keegi teine maksab koolide, teede ja muude asjade 
remondi kinni. Raha hakkas siin riigis liikuma palju rohkem ja tänu sellele oli ka võimalik 
teatud kõõl sellest digiteenuste arengusse suunata.

\question{Marek Tiitsu\index[ppl]{Tiits, Marek} 
IBS\index{Institute of Baltic Studies} ja igasugused muud asjad olid ju kõik 
välisfondide rahaga tehtud.}

Ja kui teised nägid, mida me teeme, siis tekkis üsna kiiresti laboriefekt: 
aitame, toetame ja saame ise ka näpud vahele sellele, mida nad seal teevad. Meist ei oleks
kindlasti saanud sellist e-riiki, kui meil ei oleks avanenud 
võimalust saada suures koguses välisabi. Vähehaaval oma toonasest SKTst (elasime tollal tegelikult ju ikkagi Maailmapanga vaeste riikide 
kriteeriumite järgi) me seda ei oleks teinud. Kuigi, mööngem, et 
täna ei saa sa ilmselt isegi ühte korralikku e-maksuametit selle raha eest, millega me 
esimesed kümme aastat oma e-riiki ehitasime.

\question{Kuidas see muutus operatiivtasandil käis? Minu käest on
näiteks Prantsuse ametnikud küsinud, et arusaadav, tegite e-riigi, aga 
kuidas te selle ametnikele ära seletasite?}

Jaa, seda küsis ka minu käest president 
Macroni esimene peaminister Édouard Philippe, kes oli otsustanud, et nüüd tuleb Prantsusmaal ka teha 
digipööre. Ta uuris, mis avaliku sektori
töökohtadest sai. Ütlesin talle, et vaata, Édouard, 
nüüd on nii, et meil läks maksuametis 60 protsenti töökohti kaotsi, aga meil 
ei olnudki üldse nii korralikku maksuametit nagu teil. Ära selle pärast 
muretse, te ju teete tööturu liberaliseerimise reformid ka ära, eks! Mida 
nad ongi teinud, ehkki muidugi mitte määrani, mida meie peame igal 
juhul normiks. 

Vastab tõele, et Eestis kadus ka töökohti, aga (see on nüüd 
teisest valdkonnast) kui Hoiu- ja Hansapank liitusid,\sidenote{See toimus 
jaanuaris 1998.} said Eesti ettevõtted endale finantsjuhid. Miks? Sest panganduses jäid üle sisulise poole spetsialistid, kes suutsid arvutada, 
ja ettevõtetel oli neid vaja kasvõi 
selleks, et nendesamade pankadega läbi rääkida. Neid polnud aga kuskilt võtta ja kui nüüd
kaks panka ühinesid, siis korraks tekkis selles valdkonnas tööjõu ülejääk 
ja hopsti! Täpselt samamoodi vajab erasektor e-riigi 
ehitamiseks kogu aeg inimesi, kes teavad, kuidas riigis protsessid käivad, ja minu 
arust on nad kogu aeg ise ära söönud sellesama tööjõu, mille nad on avalikus 
sektoris hävitanud.

\question{Hüve hüveks, aga miks ametnikud vastu ei hakanud töötama?}

Sellepärast ei hakanudki, et eestvedajad, kes võtavad 
juhtrolli, ei jää ju kunagi ilma tööta. Kindlasti oli kuskil ka neid, kes 
kannatasid ja kelle töökohad kadusidki. Eestvedajad ei 
muretse selle asja pärast, sa ise õpid selles protsessis nii palju. Ja 
väga paljud hüppasid teise paati koos erasektoriga -- nad osteti üle, et pakkuda riigile seda teenust tagasi.

\question{Meil ei olnud tol ajal see aparaat veel kivistunud, sa ei saanud 
olla olnud 15 aastat ametis, sest vabariiki polnud nii kaua eksisteerinud.}

Igal pool muutusid ju tegelikult noored meie riigi näoks. Ükskord kui olime Laariga Leedus visiidil, ütles üks Leedu erastamisagentuuri juht mulle, et te eestlased teete hästi julgeid asju, sellepärast 
et te olete kõik nii noored ja ei taju üldse, mis hirmsad riskid selle kõigega 
kaasnevad. Selles oli kindlasti oma iva. Sama lugu oli panganduses. Pärast Hoiu-Hansa ühinemist tuli siia üks 
ungarlane ja ma näitasin talle panga
ülemist korrust, mis oli täis oma nappi kolmekümmet eluaastat 
prilliraamide taha varjata püüdvaid tüüpe. Ta küsis mu käest: 
\enquote{Ütle, Kersti, mis te vanemate inimestega tegite?} Aga selle hind on 
see, et meie põlvkond pidaski üleval oma vanemaid ja kasvatas oma lapsi ning
tõenäoliselt jääb meil keskmise eluea osas mingi negatiivne hüpe 
sisse.

\question{Panga seltskond oli tol ajal jah nooruslik. Ja kui vaatame 
Laari, siis ta tänapäeva mõistes ajalooõpetaja hariduse pealt istus maha ja 
tegi maksureformi!}

Mardil oli meeletu usaldus oma nõunike vastu. Ma ei mäleta 1992.--1994. aasta 
perioodi,\sidenote{Mart Laar\index[ppl]{Laar, Mart} oli peaminister aastatel 1992--1994 ja 
1999--2002.} mina siis seal ei töötanud, aga ta lasi Ardo 
Hanssonil\index[ppl]{Hansson, Ardo} ilmselgelt otsustada ja möllata, nii nagu 
ka hiljem Linnar Viigil\index[ppl]{Viik, Linnar}, Kaarel 
Tarandil\index[ppl]{Tarand, Kaarel}, Simmu Tiigil\index[ppl]{Tiik, Simmu} 
või meil teistel. Saime vabad käed ja ta oli meie seljataga. Ütlesime lihtsalt: \enquote{Kuule, me tahaksime nüüd sellise asja ära 
teha.} 

Tol ajal oli igal ministeeriumil oma pank. Täna on meil 
ainult üks EAS ja KredEx -- mu arust on seegi risuks jalus, aga hea küll. 
Ja siseministeeriumis oli umbes kaks fondi, mis andsid raha 
välja. Ütlesin Mardile, et see on jube ebaefektiivne ja 
milleks need pangad üldse välja on mõeldud, kuna see raha 
seal ei kulu just kõige efektiivsemalt. Mart ütles kohe: \enquote{Tee ära, 
koristage need asjad ära.} Pärast hakkas eurorahasid liikuma ja siis oli 
EASi ka vaja. (Muide, minu magistritöö teema oli riigi asutatud 
sihtasutuste juhtimine ja see oli just nimelt seotud koondamise ja muu säärasega.) Mart ütles niisiis, et andke tuld, aga ministreid tuli ikka 
ise veenda, seda ei hakanud ta meie eest ära tegema. Käisime ja veensime. Padarit ei veennud ära ja Maaelu 
Arendamise Sihtasutus jäigi eraldi.\sidenote[][-1.7cm]{Ivari Padar\index[ppl]{Padar, Ivari} oli 
Mart Laari teises valitsuses aastatel 1999--2002 põllumajandusminister. Maaelu Edendamise 
Sihtasutus (MES) kuulub tänaseni maaeluministeeriumi valitsemisalasse.}

\question{See on täpselt selline kombinatsioon, et sul on 
oma strateegiline vaade, aga toimetama lased suhteliselt 
apoliitilise seltskonna, praktilised inimesed, kes saavad aru, 
mida on vaja teha.}

Meil oli eile Latitude'il\sidenote[][]{Konverents Latitude59 toimus Kultuurikatlas 
19.-20. mail 2022.} arutelu. Minu vestluspartneriks oli prantslane, kes küsis 
kogu aeg, mida peab riik tegema selleks, et ka Euroopas oleksid toredad digifirmad nagu ameeriklastel. Lõpuks
ütlesin talle: \enquote{Me oleme siin nüüd pool tundi rääkinud täpselt sellest, 
kuidas Eesti riik ei sekku sellesse, millised majanduslikud valikud 
erasektoris tehakse ja millised sektorid peavad arenema. 
Dirižiste\sidenote{Diri{\v z}ism, prantsuskeelsest sõnast \emph{diriger} (suunama) 
on majanduslik doktriin, milles riik mängib tugevat suunavat rolli 
kapitalistliku majanduse suhtes.} meie hulgas ei leidu. Tulemus: kümme
\emph{unicorn}'i ühe miljoni kohta. Kas see ütleb sulle midagi või ei ütle?} 

Umbes selline peabki minu arvates olema poliitikategemise roll: sa 
võimaldad asju teha. Samamoodi meie maksusüsteem -- mida tähendab ettevõtete tulumaksuvabastus? Igas investeeringus (rõhutan \emph{igas}, mitte valdkondlikult valitud 
eelisarendatavas valdkonnas) on riik ju 20 protsendiga sees ning võtab 
riski nagu ettevõtjagi. Võibolla ei hakka sealt kunagi dividende tulema, mida 
saaks maksustada.

Meil on tegelikult kihvt riik!

\question{Ka mina olen selle raamatu koostamise ajal korduvalt 
läinud mõttes tagasi 1990ndatesse ja iga kord tunnen ennast 
hästi, kui tore riik meil on!}

Oled sa vahel mõelnud, mis oleks saanud, kui oleksime valinud ennast juhtima 
inimesed, kellel riigi ja oma rahakott lähevad segamini? Meil, muide, on ka 
praegu poliitikas selliseid inimesi päris palju, kes jäävad kogu aeg vahele 
sellega, et nad on oma võimupositsiooni kasutanud enda või partei 
hüvanguks. Väga vabalt oleks võinud niimoodi minna. Ja siis 
oleksime täna omadega Ukrainas.

\question{Meil õnnestus valida mingi ime läbi 
mõistlikud inimesed \ldots}

Tegelikult ei õnnestunud. Lennarti\index[ppl]{Meri, Lennart}\sidenote{Lennart 
Meri, Eesti president aastatel 1992--2001.} kõige vastuolulisem tegu 
põhiseaduse kontekstis oli see, et ta lasi Laaril moodustada valitsuse, kui 
Savisaar oli valimised võitnud. Aga kuna kolmikliit\sidenote{Kolme Eesti partei -- Eesti Reformierakonna, Isamaaliidu ja Mõõdukate --
valimisliit.} oli eelmoodustatud, siis ta läks sellele teele. Kusjuures kolmikliit seisis ka nii habrastel alustel, et poleks üldse 
moodustunud, kui Edgar Savisaar\index[ppl]{Savisaar, Edgar} oleks saanud 
võimaluse mõne nendest potentsiaalsetest partneritest ära rääkida. Oleksime võinud 
minna märksa konservatiivsemat majandusarengut, märksa lõdvemat eelarvepoliitikat ja võibolla ka märksa oligarhsemat majandusmudelit pidi, kui 
mõelda, mida Keskerakond tollal või ka täna endast kujutab.

\question{Ja tolles keskkonnas ei oleks ilmselt ka IT-kogukond saanud oma ideid realiseerida. Kui palju on Tarvi Martens 
ID-kaardiga, Küberi seltskond küberturbega ja teised saanud minna riigi 
juurde ja öelda: \enquote{Kuulge, see on mõistlik asi, teeme!} ja neid on 
kuulatud!\nopagebreak[100000]}

Jah, ka Ukrainaga ei oleks pidanud nii halvast minema, nagu läks, sest
oligarhid võtsid majanduse päriselt vangi. Samas näiteks Sloveenia puhul, kes ühines Euroopa Liiduga ELi keskmise 
tulutasemega umbes 70, ma täna küll ei näe, et nad oleksid meist kuidagi 
paremad -- pigem on meie statistika parem. Nendel läks
majandus rohkem ettevõtete juhtide kätte, samas kui meil oli üksikuid 
selliseid ettevõtteid. Nad ei loonud väga palju uusi sidemeid uute turgudega, 
vaid jooksutasid majandust nii, nagu seda ikka oli jooksutatud, ja majandus 
restruktureerus palju aeglasemalt. 

Meil see Mart Laari  
1992.--1994. aasta periood, kui vana asi istuti katki, oli kohutavalt valus suurele 
osale töötajaskonnast. Paljudel inimestel on siiamaani rusikas taskus ja põhjusega, sest me ei osanud 
neid kõrvalefekte hallata. Meil ei olnud selleks ka raha ja 
uskusime, et tõusulaine tõstab kõiki paate.

Muide, seepärast ma tundsingi 2016. aastal, et nüüd on aeg teadvustada 
endale, et tõusulaine kõiki paate ei tõsta ja nõrgematele tuleb padi alla 
panna, kui me tahame heaoluühiskonda. Nii et kui jõudsin 20 aastat 
hiljem ringiga tagasi riigi tegemiste juurde, olles vahepeal igal pool mujal 
olnud, tekkiski võimalus asuda seda viga parandama ja ma loodan, et 
oleme õigel teel.

\question{Lõpetuseks lähme tagasi päris asjade 
alguse juurde. Miks on Eestis IT-kogukond, mis 
siiamaani toimib koos, aga näiteks lätlastel ei ole?}

Minu jaoks kannab ka meie IT -- tänapäeval isegi mitte ainult IT, sest kõik 
\emph{start-up}'id ei ole ju IT-sektori ettevõtted -- tol ajal tekkinud 
kultuuri, et võtame vastutuse riigi eest enda kätte. Kui võrrelda 
seda meie vana majanduse ettevõtetega ja paradigmaga, siis tähtsad vana majanduse 
ettevõtjad tulid peaministri nõuniku juurde ja ütlesid: \enquote{Me maksame nii palju makse, mida te meie jaoks teete?} 
IT-kogukond on aga alati olnud sellise suhtumisega, et riik ei saa seda 
teha, riik on selle jaoks liiga jäik ja paindumatu, \emph{fine}, me teeme ise! 
Teeme Jõhvi koodikooli või mille iganes! See suhtumine on alles jäänud ja ma 
olen hästi rahul. 

Mingisugune juurikas on kindlasti ka selles, et 
üheksakümnendatel lasti hästi palju teha ja tekkis positiivne tagasiside. 
Meie Pavlovi refleks on see, et saab küll. Paljud 
IT-ettevõtjad ütlevad küll, et alati kui nad on natuke aega ametnikega rääkinud, siis tahaks looteasendisse tõmbuda, aga seda vastu nina saamist ei ole ikkagi 
nii palju olnud, et lõplikult alla anda. Inimestel on 
usk, et saame tegelikult asjad tehtud, ja ka poliitikutel 
on lootus, et nende ametnikud ei ole ainult nagu tennisesein -- pall tuleb ja 
läheb kohe tagasi, vaid kuskilt peab see ka läbi minema. Isegi kui 
läbi läheb üks sajast, on see päris hea tulemus.

\question{Sest meil on kogemus, et on ju saanud ja
toiminud!}

Just. Sellepärast on ka meie \emph{start-up}-kogukond sotsiaalselt väga
vastutustundlik. Kui võtta vahelt ära parlament (mõnes
mõttes ongi võetud, sest meie parlament ei ole täna tegelikult mõttekoda, 
mida ta võiks olla) ja lasta neil ilma keskse organiseerimiseta, difuusselt 
seda riiki ajada, siis väga palju hullem see ei saaks.

\question{Siinkohal ongi ehk mõistlik lõpetada tõdemusega, et päris hea 
riik on saanud!}

On. Aga ärme riku seda ära! Mida kõrgemal tulutasemel oled, seda suuremad on 
riskid midagi teha ja muuta. Peame suutma kogu aeg 
uueneda ja edasi minna. Näen täna, et oleme 
proovinud tekitada lubavat seadusruumi uutele tehnoloogiatele, aga tegelikult 
hästi ei õnnestu. Ja see kõlab nüüd õudselt ebapopulaarselt, aga meie parlamendi palgad 
peaksid olema palju paremad selleks, et parlament töötaks mõttekojana, 
mis viiks ka tehnoloogilist poolt edasi. Ta peab uuesti hakkama tõmbama ligi ka 
akadeemilist ja \emph{start-up}-kogukonda, mida ta täna ei 
kõneta. Midagi ei ole teha -- kahjuks on nii, et mida maksad, seda saad.

\chapter{Andres Kütt}
%!TEX TS-program = arara
% arara: myindex

Sündisin 1975. aastal Võrus. Millestki midagi aru saama hakkasin 
kaheksakümnendate teisel poolel. See oli mitmes mõttes üsna kole 
aeg. Noorukile kõige arusaadavam neist koledustest oli lihtlabane praktiline 
puudus. Päris nälga ei olnud, aga midagi vähegi leivast ja piimast edevamat 
saada ei olnud. Kui linnakeses levis kuuldus, et olla toodud kast jäätist, oli 
poes veerand tunniga saba ning poole tunni pärast kõik otsas. Muu hulgas oli 
kaubandusvõrgus saada kahte tüüpi meeste talvejopesid. Mitte kahtekümmet või
kahtesadat, vaid kahte. Ühed olid hallid ja neid said osta lihtsurelikud\sidenote{Huvitaval kombel oli jope põuetasku 5,25 tolli lai, sinna mahtus 
üks flopi täpselt sisse.} ning teised olid punase A-tähega ja neid said osta 
ainult inimesed, kes teadsid kedagi, kes teadis kedagi. 
Hämmastaval kombel käisid ka seda viletsust inimesed Pihkvast bussidega 
uudistamas ja viimastki kaupa ära ostmas. 

Kogu selle halluse keskel suutis Nõukogude Liit meie Võru \linebreak[4]\mbox{Kreutzwaldi} 
Gümnaasiumile\index{Võru Kreutzwaldi Gümnaasium} tarnida 
arvutiklassitäie arvuteid Agat\index{Agat}\sidenote{Agat oli 
Nõukogude Liidus valmistatud arvuti, mis oli küll Apple IIst\index{Apple 
II} inspireeritud, kuid siiski mitte täpne kloon.}. Kust need tulid ja kes 
seda asja ajas, ei tea. Küll aga mäletan, et nende saabumine oli pikalt oodatud 
ja edasi lükatud. Miks ja mida täpselt oodatud sai, ei oska öelda. Tean ainult, et 
kui klass tekkis, läksin sinna sisse ja enam välja ei tulnud. 

Ega tolle purgiga palju teha ei olnud. Olid mõned mängud ja programmeerimiseks 
BASIC\index{BASIC}, milles meid programmeerima õpetatigi. Esimese hooga ei õpetatud 
seejuures mitte kõiki käske, näiteks for-tsükkel oli tükk aega saladus. Kui aga 
nohikud said aru, et nende eest tarkust varjatakse, kadus igasugune respekt ja 
läks lahti suuremaks isepusimiseks. Kõik muutus, kui kooli saabus noor, vist
värskelt ülikoolist tulnud arvutiõpetaja Aivar 
Halapuu\index[ppl]{Halapuu, Aivar}. Temaga tekkis kohe 
poolkamraadlik side, mis sisaldas siiski alati suurt kogust meiepoolset lugupidamist. Tolleks ajaks oli meil väiksem seltskond poisse, kes seal 
klassis toimetas ja end kohe \emph{in corpore} Aivarile sappa haakis. Aivar 
viitsis meiega tegeleda ja kuigi ta meile suurt midagi arvutite mõttes ei 
õpetanud, sai tema käest midagi kultuuritaolist. Ta
üritas meiega bridži mängida, rääkis mänguteooriast ja nii edasi. Ega me väga palju aru saanud, kuid targa inimese viitsimine meiega tegelda tekitas soovi
tolle viitsimise vääriline olla.

Kuna me sisuliselt elasime arvutiklassis (peale kooli kohe sinna, õhtul 
hilja koju, nädalavahetustel käisime samuti Aivari käest võtit palumas), siis 
usaldati arvutiklassi võti üsna pea meie kätte. Aga \emph{kooli} võtit meie kätte keegi ei andnud. Seetõttu 
oli oluline hoida järjepidevust: keegi oli alati klassis olemas ja lasi hõikamise 
või kivikese viske peale tulija sisse. Mõnikord oli meie käes 
siiski ka välisukse võti, aga tihti roniti sisse-välja akna kaudu. 

Ühel hetkel avanesid kuskil kraanid ja hakkas saabuma humanitaarabi. Võrul oli vist seoses 
rahvamuusikaga põnevaid suhteid välismaa asutustega, kes hakkasid meile 
igasugust huvitavat kola saatma. Kord saabus klassitäis rootsikeelsete 
paberite ja tarkvaraga masinaid, millega me ei osanud mitte midagi teha. Käima 
nad läksid, rootsikeelseid veateid väljastasid, aga sellega asi piirdus. Millega tegu oli ja mis neist sai, 
ei tea. Aga tuli ka üks iidne aparaat, mille külge käis neli-viis 
terminali ja kaks kokku külmkapisuurust kettaseadet, mille sisse käisid 
hiigelsuured plastkarbis kettad. Tegu oli industriaalseadmega: kui tuurid sisse 
võttis, siis oli alla tänavale kuulda, et \enquote{arvuti töötab}. Tolle masina 
peal ei osanud ka keegi midagi tarka teha, sest tarkvara polnud. Kuna masinasse asjade saamiseks 
olid ainult nimetet hiigelsuured kettad, ei olnud tarkvara ka kuskilt võtta. Mängisime mänge 
ja oligi kõik. Mäletan siiski, et seal puutusin esimest korda 
kokku Zorki\index{Zork}-nimelise mänguga\sidenote{\enquote{Zork} on üks varasemaid 
tekstipõhiseid arvutimänge. Mängija sisestas teksti ja talle ka vastati tekstiga 
vastavalt sellele, mis mängus parasjagu juhtus. Kuna mängu alguses sattuti 
lagendikule valge maja ette, oli meie puhul ilmselt tegemist \enquote{Zork I-ga}.}.

Lõpuks tulid meile Jukud\index{Juku} ja üheksakümnendate algul ka 
PCd. Jukusid oodati väga, sest Agat oli päris jube aparaat.\sidenote{Ma ei ole 
kunagi hiljem kohanud arvutit, mis suudab flopikettasse füüsilised sooned tõmmata.} 
Jukud olid väga ägedad, ainus nõrk koht oli 
klaviatuur. 

Mis aga palju ei muutunud, oli tarkvara. Võru ei ole Tartu ega 
Tallinn. Meie seltskond ei suhelnud õieti kellegagi, nii et uut tarkvara ja
teadmisi ei tulnud eriti kuskilt peale. Ajakirjast \enquote{Arvutustehnika \& 
Andmetöötlus}\index{Arvutustehnika \& Andmetöötlus}\sidenote{\phantomsection\label{sisu:aa}Alates aastast 
1987 Eesti esimese infotehnoloogiaettevõtte Algoritm\index{Algoritm|see{Tallinna 
Teadus-Tootmiskeskus}} (sellest põnevast asutusest loe lähemalt lk \pageref{sisu:algoritm}) 
algatusel ja rahastusel ilmunud esimene regulaarne IT-ajakiri. A\&A ilmus Eesti Teadus- ja Tehnikainformatsiooni ning Majandusuuringute 
Instituudi\index{Eesti Teadus- ja Tehnikainformatsiooni ning Majandusuuringute Instituut} 
(lühendatult Eesti Informatsiooni Instituut\index{Eesti Informatsiooni 
Instituut|see{Eesti Teadus- ja Tehnikainformatsiooni ning Majandusuuringute Instituut}}) infoseeriana.} 
võis küll lugeda Unicode'i võludest, aga programmeerida tuli ikkagi 
assembleris või BASICus. Seejuures sain alles hiljem teada, et eksisteeris 
ka makroassembler. Tavalises assembleris pidi JMP-käsule andma argumendiks 
suhtelise aadressi (mis muidugi osutus kohe valeks, kui kuskile mõne rea 
vahele panid)\sidenote{See oli probleem vaid minusugustele surelikele. Inimesed, 
nagu klassivend Vallo Trell\index[ppl]{Trell, Vallo}, suutsid ka otse BIOSi 
prompti peal mällu baite kirjutades masinkoodis programmeerida.}, aga 
uuemas assembleris sai silte kasutada. Käisime ka mõnel üritusel Tallinnas (mäletan 
Pedas\index{Tallinna Pedagoogikaülikool} asunud MSXide\index{Yamaha MSX} klassi) 
ja tõime sealt ka tarkvara kaasa, aga üldiselt olime üsna omaette. Isegi 
flopisid käisime ostmas Tallinnas ühest komisjonipoest. Tavaliselt kasutasime ära mõnda klassiga organiseeritud käiku teatrisse, kui jäi 
ka paar tundi linnas kolamise aega. 

Olin ka üks õnnelikest, kellele lõpuks arvuti suveks koju usaldati -- muidu pidime 
suvekuud veetma arvutiklassi akna all kurvalt kiibitsedes. Esmalt lubati
Agat, siis Juku. Kuna ekraanid olid mõlemal nigelad, veetsin kaks-kolm suve 
ettetõmmatud kardinate taga arvutiga toimetades. Juku peal mäletan oma tegemistest kahte 
suuremat projekti. Esimene oli Norton Commanderi moodi failihaldur ja teine 
fondiredaktor. Jukul sai tähekujusid suhteliselt lihtsasti ümber teha, mälus 
olid vist kaheksabaidised bitimaatriksid ning teksti kuvamine käis kiiremini 
kui muu graafika. Mõlemat kirjutasin assembleris ja kumbki päris valmis ei 
saanudki, sest teatud mahust alates muutus kood hoomamatuks. Sel ajal omandasin 
ka hiljem palju vaeva põhjustanud kombe \enquote{tunde järgi} koodi kirjutada: 
teed muutuse, kompileerid, proovid ja muudad pikalt mõtlemata uuesti, kuni 
asi pigem juhuse kui mõistuse tahtel tööle hakkab. Kood oli
nii kole, et seda oli liiga keeruline iga kord uuesti läbi mõelda. Mingid 
\emph{off-by-one} vead olid sagedased, aga üldjuhul sai mõne konstandi ühe võrra 
nihutamise peale koodi käima. Sellest rumalast kombest pole ma paraku siiani lõpuni 
vabanenud. 

Juku peal sai ka andmebaase teha, dBASE\index{dBASE} oli täitsa olemas. 
Tänu sellele õnnestus maik suhu saada kellelegi arvuti abil kasulik olemisest, kui tegin 
koolivend Aini dieediteemalise uurimistöö jaoks andmestiku ja kirjutasin 
ka programmi kassetiümbriste trükkimiseks. Tollal käibis muusika kassettidel, 
mida ohtralt kopeeriti\sidenote{Eksisteeris ka tänapäeval mõeldamatu täiesti 
põrandapealne muusika kopeerimise asutus, näiteks Tartus. Läksid kohale, 
valisid kataloogist albumi välja, jätsid tühja kasseti maha ja mõni päev hiljem 
said sobiva summa vastu muusikaga kasseti tagasi.\phantomsection\label{sisu!kassetid}}. 
Seetõttu kirjutati lugude 
nimesid käsitsi ning see oli tüütu. Minu tarkvara võimaldas aga kiiresti 
kassetiümbriseid trükkida. Selle teenuse eest sai vist ühelt 
klassivennalt isegi raha küsitud.

Linna peal tegin erinevates kohtades ka PCdega tutvust. Kellelgi oli mööblivabrikus 
tutvusi ja seal toimus isegi mõned korrad mingisugune õpe. Istusime ilmselt 
raamatupidamise masinate taga ja meile näidati, kuidas FoxPros\index{FoxPro} 
vorme joonistada ja andmeid hoida. 

Keskkoolis õnnestus käia väga murdelistel aastatel 1990--1993. Võrus möllas 
punkar Saare Ain\index[ppl]{Saar, Ain}\sidenote{Kodanikunimega Ain Saar, asutas 
Vaba Sõltumatu Noortekolonni number 1 ja tegi muid tükke.}, Võru surnuaial 
taastati Vabadussõja mälestussammas ja miilits ajas koertega üritusi laiali. 
Ühe sellise intsidendi järel oli koolis näha kummalistes ülikondades 
seltsimehi, kes pingsalt vanemate klasside õpilaste nägusid jälgisid ilmses 
lootuses tuttavaid kohata. 

Tekkis ka äri. Leidsime sõpradega 
ajalehest kuulutuse, milles otsiti meie jaoks ulmeliste palkadega (umbes 
vanemate aastapalk paarinädalase projekti eest) meelitades C 
programmeerijaid. Kandideerimise tähtaeg oli kaks nädalat ja see 
tundus täiesti mõistlik aeg, millega omale C selgeks teha. Kuskilt sai hangitud 
klassikaline Brian Kernighani ja Dennis Ritchie \enquote{The C Programming 
Language}\index{The C Programming Language}, mida kambaga tudeerisime ja mis tundus 
loogiline. Kuna meil puudus juurdepääs C kompilaatorile, siis päris koodi 
kirjutada ei saanud. See meid ei heidutanud ja saatsime isegi mingid kirjad välja. Vastust muidugi ei tulnud. Hiljem olen mõelnud, kas tegu võis olla 
tollesama legendaarse lehekuulutusega, mis viis kokku Bluemooni\index{Bluemoon} 
poisid ja Stefan Obergi\index[ppl]{Oberg, Stefan}, aga ajastus siiski vist ei klapi. 

Kõik head asjad saavad kord otsa, nii ka keskkool. Tollal sai lõpueksamit valida ning oleks olnud kummaline, kui meie 
seltskond ei oleks valinud arvutieksamit. Aivarist olime arvutiteadmiste poolest juba kaugel 
ees, sest meil ei olnud sõna tõsises mõttes mitte midagi muud teha kui arvutit 
torkida. Laulsin küll ka kooris\sidenote{Kooriga välisreisile minek oli ka põhjus, miks ma ei ole kunagi vabariiklikul 
informaatikaolümpiaadil käinud. Tol ühel kevadel, kui sinna õnnestus välja 
murda, oli ka reis plaanis. Otsustavaks sai, et ma ei tahtnud koori hätta 
jätta, mitte et oleksin seal kandvat rolli mänginud.}, 
aga põhimõtteliselt kogu muu vaba aeg oli arvutite päralt. Isegi õppetöö ei 
seganud, sest põhikoolis tegin endale kõva põhja alla. Kõik see ei 
vähendanud sugugi eksami pidulikkust. Sisenesime ruumi, võtsime pileti, 
lahendasime, vastasime komisjonile -- kõik oli nii, nagu peab. Aivar oleks võinud 
meile kõigile pikalt mõtlemata viied välja kirjutada, aga ometi viidi eksam täie tõsidusega läbi. 
See tundub siiani oluline.

Kuna mul õnnestus kool nibin-nabin kullaga lõpetada, sain Tartu Ülikooli 
matemaatikateaduskonda\index{Tartu Ülikool!Matemaatikateaduskond} eksamiteta 
sisse. Sinna minek tundus loogiline, sest Tallinn oli kaugel ja tundmata ning 
arvutivärki tahtsin kindlasti õppida. Sõjaväega probleeme ei olnud. Esiteks 
olid segased ajad ning Eesti riik polnud veel päriselt välja mõelnud, mismoodi 
oleks mõistlik väeteenistust korraldada.\sidenote{Hiljem on selgunud, et ülikooli 
minek vabastas väeteenistusest ning meie aastakäik hakkas ülikooli lõpetama just 
siis, kui otsustati siiski enne ülikooli väeteenistuse läbimise kasuks.} 
Teiseks oli mu silmanägemine nii paha, 
et kaitseväe tohtrid ütlesid mulle: \enquote{Kui venelane peale tuleb, 
siis paneme su laipu vedama, seniks mine koju.} Nii veetsingi suve Võru ja 
Tartu vahel hääletades, käisin näiteks ka Steni\index[ppl]{Tamkivi, Sten} 
juures\sidenote{Meie suguseltsid sõbrustasid, 
Steni vanaisa elas Võrus ja saime juba üsna õrnas eas tuttavaks.} 
Primexis\index{Primex Data} külas. Kohtasin seal elus esimest korda Photoshopi-nimelist tarkvara, laserprinterit ning morni näoga, kuid
huvitavat tüüpi, kes osutus Tarmo Taliks\index[ppl]{Tali, Tarmo}. Temaga puutusime
hiljem veel korduvalt kokku. Tarmo on üks neid inimesi, kelle puhul olen veendunud,
et olen temalt kohutavalt palju õppinud, suutmata siiski midagi konkreetset sõnastada. 
Olen tänulik. 

Sügisest algas ülikool ja asusin püsivamalt Tartusse. Kuna jäin paberite ajamisega 
töllerdama, siis ei õnnestunud koos teiste matemaatikutega Tiigi ühikasse kohta saada. Ühe või kaks talve olin sugulase juures üüriliseks, ühe talve 
elasime kambaga Tartu Kurtide Ühingus\index{Tartu Kurtide Ühing}, mis üüris
tudengitele tuba välja. Küll aga sai külas käidud klassivendadel, kellest enamik 
läksid majandust õppima ja kelle ühikaks olid Narva maantee 
tornid. Nii õnnestus ühikaelust maik suhu saada selles siiski kõrvuni osalemata. 
Sellest mul ülemäära kahju pole, sest õlu mulle ei maitse. Tiigi ühikas 
tegid kaastudengid kord koridori lõkke ning Narva maantee tornides pudenes regulaarselt 
keegi rõdult alla. Minu jaoks ei ületanud kahtlemata elava seltsielu paleus 
kommunaalhorrori veidi ligast reaalsust. 

Ülikoolis sain piltlikult öeldes kohe ägeda laksu otse ego pihta. Esmalt 
selgus, et erinevalt keskkoolist oli ülikoolis vaja päriselt õppida, aga vastav oskus 
oli juba kadunud (keskkool möödus arvutite seltsis ja põhikooli seljas 
liugu lastes) ning tuli uuesti tekitada. Teiseks selgus, et 
ropust tööst enam heade hinnete saamiseks ei piisanud, vaja oli ka annet, aga 
seda on mul kogu aeg nappinud. Teistel seevastu annet jagus 
ning see tegi egole haiget. Näiteks Meelis Roos\index[ppl]{Roos, Meelis} 
ja Rene Prillop\index[ppl]{Prillop, Rene} seilasid igasugusest matemaatikast läbi 
ilma nähtava pingutuseta ja kirjutasid koodi nagu jumalad. Margus 
Sutt\index[ppl]{Sutt, Margus} teadis arvutitest nähtavasti kõike ja oli tolleks 
ajaks juba tegelenud täiesti müstilisena tunduvate asjadega. Asko 
Seeba\index[ppl]{Seeba, Asko} oli kõike seda \emph{ja} seejuures veel 
seltskondlik, mängis kitarri ning oli tüdrukute hulgas popp. Ei jäänud 
midagi üle, tuli tasapisi hakata inimeseks õppima. 

Lisaks inimeseks saamisele oli vaja saada tööinimeseks, sest ema käest ei 
saanud ju jäädagi raha küsima. Proovisin saada baarmeniks, vast avatud Atlantise ööklubi valgustajaks ja isegi 
arvutigraafikuks, aga asjata. Lõpuks sattusin ettevõttesse Korel 
IN\index{Korel IN} programmeerijaks, esimene tööpäev oli 1993. aasta detsembri alguses. Mind ja kamraad Veljot\index[ppl]{Hagu, Veljo} võeti palgale 
eesmärgiga luua firmale arvetega majandamiseks vajalik tarkvara. Keeleks oli 
Visual Basic\index{BASIC!Visual Basic} ja ei läinud palju aega, kui meil mõned asjad 
juba töötasid. \enquote{Programmeerija} kõlab märkimisväärselt glamuursemalt, 
kui asi tegelikult välja nägi. Tegime kõike alates kauba tassimisest (kontor 
asus viiendal või kuuendal korrusel ja kahekümnetolline CRT monitor on päris 
raske) kuni isegi mõningase müügitööni. Toonasele arvutiärile iseloomulikult 
ei teadnud eales, mis seisus töökoht kontorisse jõudes oli. Mõnikord oli ära 
müüdud mälu, mõnikord võrgukaart või monitor. Mäletan end kirjutamas koodi 
üheksatollise must-valge kassamonitori ees taburetil istudes\phantomsection\label{sisu:jupimyyk}. 

Tartu ei ole suur linn ja nii puutusime Korelis töötades kokku suure osaga 
toonasest arvutiseltskonnast. Tarmo Tali\index[ppl]{Tali, Tarmo} oli meil 
müügimeheks ja aeg-ajalt käis tal külas Asko Oja\index[ppl]{Oja, Asko}, keda kutsuti
hellitavalt \enquote{Tarmo blondiiniks}. Vahel astus Sorose sajalisi 
luhvtitades läbi Marek Tiits\index[ppl]{Tiits, Marek}, kellele õnnestus mingi ime läbi 
isegi üks Suni tööjaam müüa. Kui ütlen, et puutusime, siis tegelikult 
mina ei puutunud eriti kellegagi kokku, sest olin toona ja olen siiani küllaltki 
asotsiaalne. Igasugust toredat rahvast käis poest läbi ja enamasti kuulasin lihtsalt, 
silmad punnis peas, spetsialistide jutte ilma nende nimesidki teadmata. 

Kuidagi tekkis Korelisse aktiivne kodanik nimega Tanel 
Urbanik\index[ppl]{Urbanik, Tanel}. Ta pandi meile alguses ülemuseks, aga üsna 
varsti vedas ta meid Korelist minema, asutades uue ettevõtmise nimega HClub. 
Nimi tuli sellest, et meie tuba kutsuti Koreli päris ärimeeste hulgas veidi põlastavalt 
häkkeriklubiks. Tanel tahtis tarkvaraäri teha, küllap seetõttu tal 
Koreliga teed lahku läksidki. Meie peamiseks leivanumbriks sai kassasüsteemide 
ehitamine ja põhiklientideks erinevad tanklad, näiteks Favora omad. 
Kirjutasin muu hulgas ka Ravimiametile\index{Ravimiamet} nende ühe 
esimestest andmebaasidest. Selguse mõttes olgu öeldud, et toona mingist 
klient-server arhitektuurist juttu ei olnud. Kõik lahendused hoidsid andmeid 
võrguketta peal Microsoft Accessi\index{Microsoft Access} andmebaasis ja selle 
poole pöördumine käis kliendi juurde paigaldatud \enquote{paksu} kliendi abil. 

Tollele ajale tagasi mõeldes tundub hämmastav, et meie tarkvara töötas. Meid 
olid vähe ja testimisest või versioneerimisest ei 
teadnud keegi midagi. Kord pidin Tartust Võrru tanklasse tagasi 
sõitma, sest värsket versiooni flopi peal kohale viies olin midagi valesti 
teinud ja kriitiline toiming läks hilisõhtul katki. Vähemalt minu kood püsis 
kindlasti koos peamiselt nätsu ja teibiga. Veljo\index[ppl]{Hagu, Veljo} oli märkimisväärselt pädevam programmeerija, aga tarkvaratehnikast polnud 
ilmselt palju aimu temalgi. 

See kõik mind lõpuks HClubist ära viiski (päris suure tüliga, tuleb tunnistada). Ma 
ei jaksanud enam selle kokkupunutud ja päris kliente teenindava tarkvara 
eest vastutada. Põlesin läbi ja kõndisin Tanelit valjusti (ja mõneti teenimatult) needes minema. 
Mõnega toonastest seikadest kohtusin veel aastaid halbades unenägudes. Oma rolli mängis 
ilmselt ka see, et just tol ajal läks põhja mu 
unistus saada arvutialane haridus. Nimelt olid matemaatikateaduskonnas 
esimesed paar aastat kõigile ühised, seejärel tuli valida arvutiteaduse, 
statistika või rakendusmatemaatika suundade vahel. Valik käis seejuures õpitulemuste 
alusel. Minu tulemused võimaldasid napilt ennast tulevaseks arvutiteadlaseks pidada ja 
nii esitasin vajaliku avalduse ning asusin järgmisest semestrist hoogsalt 
arvutiteaduse aineid kuulama. Neid loeti enamasti Liivi tänava 
õppehoones\index{Tartu Ülikool!Liivi õppehoone}. 
Dekanaat oma teadetetahvliga asus aga Vanemuise õppehoones\index{Tartu Ülikool!Vanemuise 
tänava õppehoone}. Ja kuna ma ka oma 
ut.ee meiliaadressi ei jälginud, läks minust täiesti mööda dekanaadi mõte, et 
peaks tudengite käest nende suunavaliku kohta veel mingeid pabereid küsima. 
Kui ma ükskord jaole sain, olid 
arvutiteaduse õppekohad täis ja minust sai statistikaüliõpilane. 

See oli päris 
valus hoop. Kuigi arvutiteaduse ained olid minu jaoks rasked (mäletan end kolm 
korda kompileerimismeetodite eksamit tegemas), oli mul siiski mingi lootus 
sealtkaudu kuidagi paremaks programmeerijaks saada ning kamraadidele järele 
jõuda. Toonane ülikooliharidus oli tänasest väga erinev ja ei omanud reaalse eluga 
suurt sidet, aga lootus jäi. Statistikast huvitusin ma vähe ja 
ei näinud mingit võimalust sellest oma töises elus kasu saada. Masinõppe revolutsioonini 
jäi veel paarkümmend aastat. Seetõttu tegin 
edaspidi minimaalse, et kuidagi koolist läbi saada, ja keskendusin tööle. 

Kogu BBSindus läks minust üsna suure kaarega mööda. Võrus ei olnud kohalikku 
BBSi ja kaugekõne ei tulnud ei hinna ega kättesaadavuse mõttes kõne allagi. 
Sten\index[ppl]{Tamkivi, Sten} Primexis\index{Primex Data} küll vist näitas kuhugi 
helistamist, aga tuhka ma aru sain. Korelis oli väline modem ja aeg-ajalt 
sai kuhugi sisse helistatud, aga väga sporaadiliselt. Peamine 
selleteemalise info allikas oli kursavend Mati Muts\index[ppl]{Muts, Mati} ja 
põhiliselt käisin Luciferi \mbox{BBSis}\index{Luciferi BBS}. Küll aga oli 
ülikoolil tol ajal juba täiesti korralik internetiühendus ja palju aega kulus 
Vanemuise õppehoones\index{Tartu Ülikool!Vanemuise tänava õppehoone} terminali 
taga FTPd pidi ringi kolades. Mäletan, et tõmbasin kas ftp.funet.fi või 
ftp.sunet.se serverist tükk aega Metallica albumi kaanepilti ja olin väga 
rahul, kui see ka päriselt kohale jõudis ning ekraanile ilmus.

Selgelt mäletan ka kohtumist HTMLiga. See oli Liivi 
tänaval\index{Tartu Ülikool!Liivi Õppehoone}, kus asus Suni 
klass\sidenote{Need pidid olema Sunid, sest mäletan ruudulist hiirepatja, mis 
muidugi ei olnud mingi padi. Suni optiline hiir sõltus lihtsalt ruudulisest aluspinnast.} 
ning kus ma sukeldusin veebilehtede 
võrratusse maailma. Pärast pikka pusimist suutsin tekitada oma kodulehe, kus 
asju õiges kohas hoidis tabel! Ega sinna kodulehele midagi kirjutada ei olnud, 
aga tabeli ridade ja lahtrite saladuste lahtipusimine oli põnev.

Kõik see osutus kasulikuks, sest HClubi järel võttis 
klassivend Meelis Mäeots\index[ppl]{Mäeots, Meelis} mind enda juurde tehnikuks. Ta tegeles tol ajal 
igasuguste imelike asjadega, kuid muu hulgas asutas ka internetifirma. See koosnes 
alguses peamiselt minust ja temast. Firma tegeles Unineti\index{Uninet} 
\emph{dial-up}-ühenduste edasimüümisega, tegi kodulehekülgi ja pidas isegi 
Infomeistri-nimelist interneti infokataloogi. See viimane oli täiesti hämmastav 
äri. Meelis käis ja rääkis mingitele firmadele augu pähe, mina kirjutasin firma 
andmed kuskil serveris asunud staatilisse (!) HTMLi. Mis kasu sellest kellelegi 
ammu enne otsingumootorite laia levikut tõusta võis, on mulle siiani 
arusaamatu. Ma ei mäleta ka, et keegi seal lehel väga käinud oleks. Ometi maksti 
meile arved ära ja ma väga loodan, et tolle tegevuse käigus antud lubadused said
enam-vähem täidetud. 

Kuna teadsin Steni\index[ppl]{Tamkivi, Sten} juba varasemast ja Meelis vist ka puutus temaga kokku, 
lõpetasime ühel hetkel modemitega jantimise ja infokataloogi pidamise ning 
asusime Steni asutatud Halo\index{Halo Interactive DDB} nime all kodulehekülgi tegema. Kampa 
võeti ka mõned kunstnikud, näiteks väga andekas Oliver 
Reitalu\index[ppl]{Reitalu, Oliver} ja mitte vähem andekas Alar 
Koort\index[ppl]{Koort, Alar}, keda kutsuti ilmselt tema rajude elukommete tõttu 
Helbekeseks. Projektijuhiks oli Priit Sasi\index[ppl]{Sasi, Priit}, keda 
kõik tema joviaalse oleku ja suure habeme tõttu Sasuks kutsusid. Sasu õpetas 
mind briti punki ja Alar kurjemat sorti hiphoppi kuulama ning elu oli päris tore. 
Minu käe alt tuli Eesti esimene kommertsalustel tehtud 
(st ettevõte maksis kellelegi lehe tegemise eest raha) kodulehekülg, mis sai 
tehtud Tartu Raadiole\index{Tartu Raadio}, kui mälu ei peta. Kunstnik joonistas 
pildid valmis ja lõikas tükkideks, mina kirjutasin Notepadiga HTMLi ja nii see 
töö käis. 

Ühel hetkel hakkasime lehekülgede tekitamist automatiseerima, kirjutasime 
Perli skripte. Mõnda aega ei olnud meil ei oma serverit ega üldse kuskil Perli 
jooksutada. Siis sai programmeeritud nii, et skript läks meiliga
Unineti\index{Uninet} süsadminnile, kes kopeeris faili õigesse kohta, meie 
vajutasime brauseris nuppu, saime veateate, admin saatis meiliga konsooli 
veateated, mina parandasin koodi ja saatsin uue versiooni. Admini kannatus 
lõppes enne kui minu oma. 

Lõpuks jõudsime oma tegemistega siiski päris kaugele. Perli skriptid läksid 
järjest pikemaks ja kuna andmebaasi pidamiseks ei olnud meil serverites 
piisavalt õigusi, hoiti andmeid enamasti lihtsalt tekstifailis. Üllataval moel 
kattis see ära päris suure hulga vajadusi. Perlilt liikusime ühel hetkel PHP-le 
ja tekkis ka levinud, kuid seetõttu mitte vähem rumal mõte endale ise 
oma sisuhaldussüsteem kirjutada. See sai vist isegi valmis, aga konkreetsed 
mälestused tollest elukast puuduvad.

Ma ei mäleta, et see äri oleks kuidagi tänapäevases mõistes äri moodi välja 
näinud. Raha oli alati vähe ja seega tuli teha kõike, mille eest maksti. Kuidagi 
müüs Sten\index[ppl]{Tamkivi, Sten} Ühispangale\index{Ühispank} maha mõtte anda nende aastaraamat välja CD-l. Mis muud, kui
õppisime selgeks Macromedia Directori kasutamise ja video redigeerimise ning 
andsime minna. Ainus asi, millega me hakkama ei saanud, oli heli. Õnneks oli Sten 
hea sõber Lauri Liivakuga\index[ppl]{Liivak, Lauri}, kelle Forwards 
Studio\index{Forwards Studio} asus meiega sama koridori peal. Lauri 
tegi kenad kõllid ja plõnnid ning aitas selle kõik visuaaliga ära 
sünkroniseerida. Tulemus sai päris kena. 

Igatahes hakkas meile järjest rohkem Tallinna kliente siginema. Ühtlasi müüs Sten 
suure tüki ettevõttest Brand Sellers DDB-le\index{Brand Sellers DDB}, mis oli 
minusuguse Tartu nohiku jaoks täiesti müstiline kamp inimesi. Intelligentsed, 
säravad, jõukad (nii mulle tundus) ning andekad. Bruno Lill\index[ppl]{Lill, 
Bruno} oma terava ütlemise ja peene olekuga on siiani meeles. Nii tehti 
kampas otsus kolida kogu Halo Tallinna. 

Olin tegelikult ligi aasta üsna kahepaikne, pendeldades Tartu ja 
Tallinna vahel. Ülikoolis olid veel viimased sabad lõpetada ja 
Mari\index[ppl]{Kütt, Maria}, kellega peagi abiellusime, käis samuti veel 
koolis. Lõpuks sain oma lõputöö kaitstud ja kuna selliseks triviaalseks asjaks ei 
hakanud ju keegi Tartusse sõitma, käis Mari mu diplomit dekanaadist ära toomas. 
Prouad nõudsid allkirjastatud volitust, mis sai ukse taga kohe valmis tehtud, 
ning nii omandasingi oma esimese teaduskraadi. Tartu Ülikooli peahoone 
sammaste vahelt ei ole ma kunagi välja astunud ja kuigi toonaseid õppejõude 
hindan siiani kõrgelt, pean oma \emph{alma mater}'iks siiski Massachusettsi 
Tehnoloogiainstituuti. 

Tallinnasse kolimisega sai läbi üks etapp Halo kasvuloost. Senise boheemliku 
mis-ikka-valesti-võib-minna mentaliteedi asemel tuli hakata käibenumbritest 
rääkima. Samuti oli meeskond kasvanud. Veel Tartu päevil olin saanud omale 
elu esimese alluva, olles ühtlasi ka tema esimeseks ülemuseks. Vist veel 
keskkooli lõpetav noor nutikas tüüp aitas mul koodi kirjutada ja hängis niisama 
ringi -- ei mina teadnud, kuidas inimesi juhitakse või mida üks ülemus tegema 
peaks. Nimeks oli tüübil Taavet Hinrikus\index[ppl]{Hinrikus, Taavet}. 

Inimesi 
lisandus veelgi ja ma ei saanud enam aru, miks ja kuidas asju tehakse. Ühel ilusal päeval
leidsin kuulutuse, et Hansapank\index{Hansapank} 
otsib internetipanga meeskonda inimesi. Läksin intervjuule. Mäletan siiani 
seda tunnet, kui Liivalaia tänava pangahoone tolle aja kohta ülišiki lifti 
uksed kaheksandal korrusel avanesid ja minu ees laius hurmav vaade 
vanalinnale. Olin müüdud mees, õnneks arvas Vilve Vene\index[ppl]{Vene, Vilve}, 
kes seal majas tarkvara arendamist vedas, samamoodi. 

Nii sai minust veidi enne sajandivahetust 
hansapankur. Mul vedas kohutavalt, sest pank oli praeguses mõistes ulmeliselt 
dünaamiline asutus. Vägesid juhatas Indrek Neivelt\index[ppl]{Neivelt, Indrek}. 
Vaata Maailma programm oli just käima minemas ja sellega tegeles Tiit 
Pekk\index[ppl]{Pekk, Tiit}. Marketsi tiim eesotsas Erkki 
Raasukesega\index[ppl]{Raasuke, Erkki} pidas ülejäänud panka talumatuteks 
venivillemiteks ja tema jaoks toodeti Erik Jõgi\index[ppl]{Jõgi, Erik} juhtimisel imeilusat 
koodi. Ehitasime panga jaoks mõne aastaga mitu internetipanka ja panime ka e-riigile käed külge. 
Aga see, nagu öeldakse, on juba üks teine jutt.

Maria Klenskaja ütles ühes intervjuus ilusti umbes midagi sellist, et mõni inimene on lavale sündinud
ja mõni teeb kõvasti tööd ja, kui noot ees, hätta ei jää. Mõni on programmeerija ja 
mõni oskab koodi kirjutada. Kuulun kindlasti viimaste hulka. Võib-olla just 
seepärast ma hindan väga inimesi, kes erinevalt minust mitte ei \emph{tee}, vaid \emph{on}, ja mulle väga meeldivad päris asjad.
Samamoodi tunduvad näiteks Villu Tamme ja Freddy Grenzman päris -- minu kogemuse põhjal ei ole nad laval 
kuigi palju teistsugused kui elus. Vahest see seletab ka, miks 
seesinane raamat on sündinud.

\chapter{Jaanus Lillenberg}
\index[ppl]{Lillenberg, Jaanus}
\question{Kuidas sina said arvutite juurde ja arvutid sinu juurde?}

See sai alguse aastal 1983, kui Tartu 
Ülikoolis\index{Tartu Ülikool} tehti Nõukogude Liidu ja Jaapani koostöö 
tulemusena personaalarvutite klass.

\question{Mis arvutid need olid?}

Need olid Yamaha MSXid\index{Yamaha MSX}. Yamaha MSX kuulub samasse põlvkonda, mis Commodore 64, mõned vihasemad Sinclairid ja 
ka Apple II. Äge oli see, et nendes arvutites jooksis 
tegelikult Microsofti tava-kasutajatele mõeldud operatsioonisüsteem.

\question{Kas see oli Microsofti oma?}

MSX nagu Microsoft \emph{Extended} vist\sidenote{Lühendi päritolu kohta liigub mitmeid variante, ka asja juures olnud inimesed ei mäleta enam täpselt.}. Igatahes nägi see äge välja. Arvutiklass paiknes kooli peauksest kümne sammu kaugusel f
keldris, mille aken avanes täpselt kooliukse ette. Ühele üheteistaastasele, kes läks sellest igal hommikul ja õhtul mööda, 
oli see vastupandamatu. Selles mõttes valikuvõimalusi tegelikult 
ei jäetud. 

\question{Sa lihtsalt pidid sealt uksest sisse minema?}

Kõndisin ühel päeval otse aknast sisse, sest 
aken oli tänavaga samal tasapinnal. Küsisin, kas võib tulla vaatama, ja ära mind otseselt ei aetud. Kolmandal päeval 
andis keegi mulle MSX BASICu\index{BASIC!MSX BASIC} 
manuaali koopia. Ma küll ei saanud 
inglise keelest aru, aga mängude tegemine tundus huvitav. 
Arvutiklassis toimetasin kolm-neli aastat ja olin vahepeal ka abiõpetaja. Kirjutasin ise tekstiredaktoreid ja mänge ning loomulikult häkkisin 
lõputus koguses olemasolevaid mänge. Kirjutasin ka oma elu esimese viiruse, mis hävitas flopiketta. 

\question{Mis koolis sa käisid?}

Tartu 10. Keskkoolis\index{Tartu 10. Keskkool}, praegu on see
Tartu Mart Reiniku Kool\index{Tartu Mart Reiniku Kool|see{Tartu 10. 
Keskkool}}. Arvutiklass paiknes Vanemuise tänaval teatri vastas 
oleva õppehoone\index{Tartu Ülikool!Vanemuise tänava õppehoone} keldrikorrusel. 
Seal oli isegi kaks arvutiklassi. Teises olid 
Agatid\index{Agat}, mis olid venelaste pihta pandud 
Commodore'i või Apple II koopiad\sidenote{Agat kasutas küll sama 
6502 protsessorit, mis Commodore 64 ja Apple II, ning oli suuresti viimasest 
inspireeritud, kuid erines disaini poolest mõlemast ja otsese koopiaga tegu 
ei olnud.}. Kusjuures mul läks rohkem kui aasta, enne kui sain aru, et see oli tegelikult 
Apple II koopia. Klassis olid ka Apple 
II\index{Apple II} arvutid ja kuigi protsessori tasandil olid need sarnased, 
oli sisu väga erinev. 

\question{See nõukogude variant oli üsna industriaalse väljanägemisega.}

Jah. Kas sa tead näiteks, et kui inglise keeles on klaviatuuril vasakult paremale lugedes 
QWERTY, siis vene klaviatuuril tuleb sama moodi lugedes kokku \enquote{pidev \emph{lag}}? 

\question{Tol ajal ei teadnud veel keegi \emph{lag}'ist midagi.}

Kui sellest ajast kümmekond aastat edasi hüpata, siis olid Tartu Ülikoolis juba arvuti- ja 
terminaliklassid. Tol ajal olid arvutid nii võimsad, et neil oli 
hunnik terminale, mis moodustasid terminaliklassi. Siis oli juba ka
väga palju võrgutegevust. Arvutiklassi kõrval oli 
IBMi koopia või litsentsi alusel tehtud ES\index{ES 
EVM}\sidenote{ES EVM (\begin{russian}ЕС ЭВМ, единая система электронных 
вычислительных машин\end{russian}) oli sari IBM 
System/360\index{System/360} ja System/370\index{System/370} 
kloone. Nende riistvara põhines küll IBMi omal, kuid oli väheste eranditega 
siiski Nõukogude Liidus välja töötatud. Tarkvara seevastu oli IBMi tarkvara 
lokaliseeritud ja väheste muutustega koopia. Neid masinaid nimetati eesti 
keeles hellitavalt jessukesteks.}, Nõukogude arvuti vene klaviatuuriga ja \enquote{pidev \emph{lag}} nii klaviatuuril kui arvutis oli väga ilmne kontseptsioon.

\question{Kõik üheteistaastased, kes arvutiklassist mööda kõndisid, ometi ei roninud 
aknast sisse. Sul pidi järelikult olema tehnika- või elektroonikahuvi.}

Ei olnud, ma käisin hoopis ratsutamistrennis. Aga mõni 
asi on kohe visuaalselt uus ja lahe ning vastandub 
kõigele muule ümbritsevale. Kujuta ette, et lähed mööda näiteks
lendamistrennist, kus inimest õpetatakse lendama. Sa ei hakka ju arutama, et ma pidin minema malet mängima või telekat vaatama. 
Lendamine on universaalselt väga \emph{cool} asi, kõigist 
teistest asjadest kümme korda kõvem.

\question{Ja siis ei oskagi pärast hästi seletada, miks sulle 
lendamistrenn meeldis ja miks sa ei läinud malet mängima.}

Lihtsalt kaldusid teelt kõrvale. Muide, ma ratsutasin neli aastat, see ei seganud.

Võtsin ühe klassivenna ka arvutiklassi kaasa. Mäletan, kuidas me arutasime omavahel, kuidas mänge tehakse. Kuidas Assembler või 
masinkood näeb nii suvaline välja ja äkki on 
võimalik \emph{random} kombinatsioone katsetades saada 
lahedaid mänge. Mõtlesime küll, et see vist ikka ei ole tõsi, 
aga oleks äge, kui nii saaks! Katsetad kümmet tuhandet 
kombinatsiooni, kõikvõimalikke koodivariante ja vaatad, milline läheb käima 
ja milline mitte. Õnneks nädal hiljem olime juba \emph{Hello 
World}\sidenote{Siiamaani peetakse oluliseks, et uut programmeerimiskeelt katsetades luuakse esmalt programm, mis väljastab kuhugi teksti \enquote{Hello World}. Traditsioon pärineb kuulsast The C Programming Language raamatust\index{The C Programming Language}.} kirjutanud ja asjad läksid natuke selgemaks. MSX 
BASICust\index{BASIC!MSX BASIC} kasvas muide välja Visual 
Basic\index{Visual Basic}, nii et Visual Basicu õppimine oli meie jaoks 
\emph{what else is new}.

\question{Kes seda arvutiklassi vedas? Pidi ju olema keegi, kes sind aknast sisse 
lasi ja kohe välja ei visanud.}

Mind visati sealt mitu korda välja, aga nad olid oma väljaviskamises 
tunduvalt vähem veenvad kui mina sisseronimises. Ma ei osanud väljaviskamise peale
kuidagi solvuda või seda pahaks panna. 
Sain ju aru, et see klass ei ole minu jaoks tehtud. Näiteks ronisin 
mitu korda sisse õpetajate täiendkoolitusele, kus tegelikult ei õpetatud 
arvuti kasutamist, vaid seda, et maailm muutub ja et arvutiõpe on 
hea käegakatsutav asi seda muutust kirjeldama. Selle visiooni taga oli üks 
väga vinge inimene, Anne Villems\index[ppl]{Villems, Anne}.

Anne Villems on tohutult kirglik, tema kirg on maailma 
paremaks teha. Õpetame neid inimesi, kes õpetavad teisi! Näitame neile ja nemad
näitavad väikestele inimestele, milline maailm võiks olla! 
Arvan, et tema täienduskoolitustele jõudnud inimesed 
olid mõnes mõttes juba paremad. Nad suutsid endale 
sõnastada, et peaksid sinna minema, sest äkki maailm muutub sellest paremaks. Need, 
kes koolituse läbi tegid ja seal omavahel suhtlesid, 
olid üks suur rest kive selles vundamendis, mille peale meie IT-riik on 
ehitatud.

\question{Õpetajad õpetasid omakorda õpilasi, kellest said abiõpetajad, ja nii see teadmine levis.}

Bingo! Koolitusel oli näiteks
üks Tartu Kunstikooli\index{Tartu Kunstikool} õpetaja, kes hiljem tõi oma lapsed 
arvutiklassi tundi pidama. Kunstikooli õpilastel oli üks
joonistusprogramm – 64 värvi, maa ja ilm. Ja nad tegid arvutiga päriselt mingeid asju, olgugi et printerit kahjuks ei olnud. 
Igatahes said nad tunnetuse, kuidas kontseptuaalselt täiesti uuel viisil kunsti teha. 

Isegi mina, kes ma mõtlesin primitiivselt, kuidas saaks mänge teha ja 
mängida, jõudsin lõpuks kuhugi välja. Aga nemad võtsid 
graafilise \emph{editor}'i ja tegid sellega võimsaid
asju\ldots Samasugune trikk nagu iPhone'i tulek: me ei teadnud algul, kui kõva asi see oli, aga kindlasti 
sajal ägedal moel. Omal ajal oli sama lugu personaalarvutite ja MSXidega.

\question{Iga uudsus läheb ju lõpuks üle, kas sinu jaoks arvutite puhul ei 
läinud üle?}

Ei läinud, see liikus baastasandilt järgmisele tasandile. Toon ühe näite kaks aastat hilisemast ajast – Tõravere observatooriumi\index{Tõravere observatoorium} 
astrofüüsikud, kes olid maailmaga hoopis teistsuguses kontaktis kui 
koolipoisid. Üks selline oli minu alumine naaber Enn 
Kasak\index[ppl]{Kasak, Enn}. Ühest küljest olid kontaktid 
teadusmaailmaga, aga teisest küljest maailm sulas ja oli 
võimalik bisnist teha. Nad tõid endale Amiga 
500\index{Amiga!Amiga 500}\sidenote{Tuntud ka kui A500, oli Amiga 500 
koduarvuti 1987. aastal Commodore'i poolt turule toodud professionaalse Amiga 
2000 vaste. Tegu oli populaarseima Amiga mudeliga, eriti Euroopas.}, mis olid 
järgmine põlvkond Commodore 64st. Sellel oli kümme 
korda võimsam protsessor ja hoopis teisest klassist graafika. 
Kui MSXi Z80 protsessor võimaldas kolmehäälset muusikat teha, siis Amiga 
suutis pakkuda kuutteist kanalit. Tänapäeva mõistes oli 1986. aastal võimalik täielik MIDI-lahendus kodus püsti panna. Oi, kuidas 
ma seal nende Amigadega muusikat tegin! Täiesti häbitult ja ööde kaupa.

\question{Kas sul muusikahuvi oli enne olemas või tekkis koos Amigadega?}

Igal inimesel on mingisugune arvamus, kas talle meeldib muusika või mitte. Osaliselt on see seotud sellega, 
kas pead viisi või ei pea. Kui mind ei võetud esimeses klassis lastekoori, siis sain aru, et mulle meeldib muusika, sest ma olin väga kurb. 

Kui ma Yamahadega tegelesin, siis see ei olnud ainult mängimine. 
Meil oli täiesti mitteametlik arvutiring: 
kutid vajusid iga päev pärast kooli kohale ja enne ära ei läinud, kui välja 
visati. Klassis tegutsesid tegelikult 
üliõpilased, näiteks Ain Sakk\index[ppl]{Sakk, Ain}, Alar Pandis\index[ppl]{Pandis, 
Alar} ja mõned teised kutid, kes jätkasid pärast ülikooli lõppu vist ka pedagoogidena. Nad olid lastesõbralikud ja toetasid meid. Meil oli võimalik seal käia sellepärast, et
arvuteid oli klassis viisteist tükki, aga 
täiendkoolitustel enamasti seitse kuni kümme inimest, nii et 
alati olid mõned arvutid vabad. Asi toimis põhimõttel „kes ees, see mees“. Kui arvuti said, siis enam seda ära ei 
andnud. Koolituse ajal muud võis teha, aga mängida mitte, nii et ootasime, hambad ristis, mingi \emph{manual} 
kõrval, mille järgi proovisime asju teha. 

MSX BASICuga\index{BASIC!MSX 
BASIC} sai samuti muusikat teha: noote ritta seada, 
rütmi kiiremaks ja aeglasemaks sättida, oktaavi muuta ning vist ka näiteks 
kolm erinevat meloodiat kokku panna. Ühetoonilist muusikat sai 
kindlasti teha – kuulasin midagi ja proovisin 
järele teha. Amiga oli selle kõrval hoopis teine tera.
Erinevus oli sama suur, nagu panna endale papist 
tiivad külge ja mängida lennukit või minna päris lennuki peale. 
Ühel juhul paned teksti-\emph{editor}'is nooditähti paika, mängid selle ette ja kuulad. Teisel juhul on täisgraafiline muusika-\emph{editor} koos nootide ja digiklaveriga, mida saab arvutiklahvide peal mängida ja salvestada nii nagu tänapäeval. Pluss sadu pille, mille seast valida, mis 
kõlasid küll digipiiksudena, aga mis olid nii ära tuunitud, et viiul ja klaver kostsid kõrvale ikkagi erinevalt.

\question{Selleks et suuta kõrva järgi muusikat järele teha, peab kõrva olema. Kas sul oli muusikaline kuulmine olemas?}

Midagi oli jah. Ega noodid ju kõik õiged pruukinud olla, aga rõõm tegemisest oli suur! Iga kord, kui midagi 
natukenegi välja tuli, viskas see puid alla juurde ja leek läks suurema hooga põlema.

Tartus oli selline võimas 
organisatsioon nagu Tartu 
Tähetorn\index{Tartu Tähetorn}, aga infotehnoloogilise ajaloo prismas oli see ainult väike ripats Eesti 
Biokeskuse\index{Eesti Biokeskus}\sidenote{Eesti Biokeskus moodustati 1986. 
aastal Tartu Ülikooli\index{Tartu Ülikool} ja KBFI\index{KBFI} ühisasutusena.} 
küljes, mis oli tähetorni kõrval väikene kuut, aga kus toimusid 
ülisuured asjad. Tähetorni katusele oli hea panna \enquote{satipann}: sealt paistis kaugele, puid ümber ei olnud ja
signaal oli alati hea. Eesti kahest
esimesest internetiühendusest üks oli Tallinnas KBFIs\index{KBFI} ja teine, Tartu oma, paikneski 
tähetornis, õigemini biokeskuses, mille ruumid olid tähetorni 
lähedal. Biokeskuses tegutses Richard Villems\index[ppl]{Villems, 
Richard}\sidenote{Eesti Biokeskuse juht selle asutamisest 
alates.}, kes koos Lippmaadega üldse selle interneti-maailma Eestile avas.

Igatahes Amigad jõudsid tähetorni ja ühendati internetti, sest kõik, kes tee peal olid, istutasid ennast ka selle traadi peale, mis enne biokeskust katuselt 
alla tuli. 

\question{Kui veel ajas tagasi minna, siis sul pidi tublisti distsipliini olema – koolituse ajal 
taganurgas istudes tuli ju vagusi olla.}

Ma mõtlesin välja sellise asja nagu võtmeluba: mulle anti klassi võti. Kuna nädalavahetustel koolitusi ei toimunud ja valvur ärkas kell seitse üles, 
siis selleks hetkeks sai ukse taha mindud. Ukse avas väga unine valvur, kes alguses ei uskunud, et mul on mingi võtmeluba, ja ajas mind minema. 
Aga kui olin juba kell seitse hommikul kohale läinud, siis ega ma sealt ära ei 
läinud. Palusin süüdimatult helistada pühapäeva hommikul mingitele inimestele, et need võtmeloa olemasolu kinnitaksid. Üksikud uued valvurid ei lasknud sisse, aga paari-kolme kuuga 
olid nad kõik välja õpetatud.

\question{Sest jama ei tekkinud, keegi ei läbustanud ja midagi ei 
varastatud.}

Läbustamiseks polnud aega. Ainukene jama oli vaba arvuti saamine: inimesed ootasid arvutiruumi ukse taga koolituse lõppu, et äkki järgmisel koolitusel on auk ja pääseb sisse. 

\question{Miks see luba just sulle anti? Kas paistsid kuidagi silma? 
Olid eriti tubli, korralik, pealetükkiv?}

Kõik see kokku. Samas tegin ma tänapäeva 
mõistes vabatahtlikku tööd. Tahtsin nii väga olla arvutite juures, et olin nõus tegema koolituste ajal abiõpetaja tööd, 
oma vabast ajast ja ilma rahata. 
Seal õpetati ülilihtsaid oskusi, mille laps omandab paari-kolme 
päevaga, nagu arvuti käimapanek ja mis tähendab \emph{press any key}. Mul ei olnud probleemi näidata tädidele, kuhu tuleb 
vajutada. Tädidel oli ka hea meel, et lapsed oskavad seda teha. Ja Anne Villems\index[ppl]{Villems, Anne} ei visanud ka mind välja eriti. 

\question{\enquote{Eriti}...}

Ma ei tea, kui palju oli sealpool seda, et nad ei 
jaksa enam võidelda ja ei ole mõtet välja visata. Meid oli vist kolm, kellel oli võtmeluba. 

\question{Seda on ikkagi vähe.}

Kõik ei jaksanud kogu aeg käia ega mahtunud ka. Eks visamad lõpuks jäid. 

\question{Kas sa käisid seal kuni keskkoolini?}

Jah. Keskkooli läksin teise kooli, 
Treffnerisse\index{Hugo Treffneri Gümnaasium}. Seal olid küll
oma arvutiklassid, aga siis oli juba oluline tähetorn\index{Tartu Tähetorn}. Seal olid Amigad, seal lindistasime esimesed lood, mu naaber töötas seal, käisin astronoomiaringis
õppisin C-d\index{C} kirjutama. Seda õpetas mulle Kaur Virunurm\index[ppl]{Virunurm, Kaur}, 
ainuke tüüp, kes suutis, sest see, mida me seal tegime, on maailma kõige halvem 
õppimismeetod. Kujuta ette, et sinu kõrval on inimene, kes tahab mingit asja 
õudselt osata, aga ta ei oska mitte midagi ega viitsi \emph{manual}'i 
lugeda. Põhimõtteliselt sind muudetakse elavaks \emph{manual}'iks ja iga kahe-kolme minuti tagant küsib õpilane, et \emph{are we there yet}.

Tartus oli teine keskus veel, füüsikahoone\index{Tartu 
Ülikool!Füüsikahoone} Tähe tänava alguses. Seal toimetas Taavi 
Talvik\index[ppl]{Talvik, Taavi}, kes andis mulle ühe C 
\emph{manual}'i, mis oli vist fotoaparaadiga üles pildistatud.

\question{Kas see võis olla Richie kuulus sinise C-ga raamat\index{The C 
Programming Language}\label{sisu:richie}?}

Jah, aga see oli ainult must ja valge, sinist ei olnud seal midagi. Lehitsesin kümme-viisteist aastat hiljem neid 
üksikuid fotokoopiaid ja vaatasin, et päris hea kraam. 
Ega ma tollal väga palju asju C-s\index{C} ei kirjutanud, aga
hiljem küll. Igatahes Kaur\index[ppl]{Virunurm, Kaur} jaksas minu tohutut huvi taluda. 

Tol ajal olid ilmunud juba esimesed XTd\index{XT} ja tolle 
Assembler\index{Assembler} oli hoopiski teisest klassist kui Z80 Assembler – kaheksa-, mitte neljakohaliste koodidega! 

Ühel hetkel lõi aga murdeiga sisse ja arvutid ei võtnud enam sada, vaid seitsekümmend protsenti ajast. 

\question{Tavaliselt tekib inimestel keskkooliajal mingisugune kultuuriline 
kontekst – muusikat sa mainisid, aga raamatud? Veel midagi?}

Väga hea, et sa selle välja tõid. Me peame aru saama, millisele lavale 
idud kasvama läksid. Tartus oli akadeemiline keskkond ja see tähendas 
enamasti kõrget lugemust ning paremat kirjandusega kursis olekut.

Kasakul\index[ppl]{Kasak, Enn} oli tolle aja kohta täitsa hea raamatukogu, 
samuti mu tädil. Autoritest oli Asimov
kindlasti kõva sõna. 

Mu vanaema oli tõlkija, ta tõlkis kuuekümnendatel näiteks teose \enquote{I, 
robot} eesti keelde. Ta vist tõlkis kellegagi koos terve Asimovi 
kogumiku. Teine väga tugev liin oli sõrmused ja nende 
isandad. 
Võibolla mõtlen natuke üle, aga „Sõrmuste isand“ on tegelikult lugu suurest ja palju tugevamast kurjusest, mille 
vastu ei saa. Mõtle, mis aastad need olid, 1987–1989! See lootus! Need 
raamatud sobisid hästi sellesse aega.

\question{„Kääbik“ oli jah eesti keeles olemas, aga mina sain teada, et see on osa 
suuremast loost, alles üheksakümnendate lõpus. Kas sul olid ingliskeelsed 
raamatud?}

Jah. Need olid erilised, keskmise vene papitrüki kõrval läikivad ja ilusad. Kuidas tunda ära 
inimesi, kes olid tollal tegevad? Neil kõikidel on kodus nähtaval kohal kogu Tolkieni looming.

Ulme oli teine liin, näiteks Asimov ja Bradbury\index{Ray Bradbury}. Osa teoseid avaldati Mirabilia sarjas\sidenote[][-2cm]{Mirabilia oli 
kirjastuse Eesti Raamat aastatel 1973–2012 ilmunud raamatusari, mis 
keskendus peamiselt ulme- ja kriminaalromaanidele. Omal ajal oli tegu 
suurepärase võimalusega tutvuda üldjuhul väga hästi 
tõlgitud ulmekirjanduse klassikaga: Simaki, Lemi, Strugatskite, 
Asimovi, Bradbury ja paljude teiste romaanide ja lühilugude kogumikega. Paljuski 
kujundas just see sari terve põlvkonna ulmehuviliste maitse ja lääne klassikute 
hulgas ilmus ka Eesti, Soome ja Läti autorite loomingut.}. 

\question{Aga Strugatskid?}

Loomulikult nemad ka\index{Strugatskid} ja Stanisław Lem\index{Stanisław Herman 
Lem}\sidenote{Stanisław Herman Lem (1921–2006) oli Poola ulmekirjanik, kelle 
teosed olid ühekorraga nii filosoofilised kui ka satiirilised ja 
humoorikad.} ja teised. Oli selline ulmekirjanduse kogumik nagu \enquote{Lilled 
Algernonile}\sidenote{Ilmus aastal 1976 sarjas \enquote{Ajast Aega}.}, see oli ka suhteliselt kohustuslik kirjandus. Kui 
tütarlastel oli võibolla kotis Herman Hesse „Stepihunt“, siis poisid raamatut 
kaasas ei kandnud, aga kuskil oli neil natuke äranäritud nurkadega \enquote{Lilled 
Algernonile}. USA ulmekirjanduse sissevool lõi toimuvale tõepoolest 
kultuurilise tagapõhja.

Muusikaga oli teistmoodi. Suured inimesed kuulasid suurte inimeste 
muusikat, noored noorte muusikat. Tähetornis\index{Tartu Tähetorn} 
kuulati palju muusikat maki pealt. Seal olid koos naljakad 
kooslused: esiteks tähetorn ise, siis 
füüsikatudengid, kes pühendusid astronoomiasuunale (näiteks Kaur 
Virunurm\index[ppl]{Virunurm, Kaur}), ja teisalt tähetorni 
direktori lapsed, kes käisid Miina Härma 
Gümnaasiumis\index{Miina Härma Gümnaasium} ja vedasid sinna 
omi koolivendi ja -õdesid. 
Tartus oli tol ajal kaks kooli, kes defineerisid, mis on äge: 
Treffner\index{Hugo Treffneri Gümnaasium} ja Miina Härma ning mõlemad 
arvasid, et on parem kui teine. Tartu värk. Tähetornis olidki 
ka mõned Treffneri tüübid. Sedasi tekkis segu keskkoolist, 
ülikoolist ja internetist, millest ei saanudki tulla mitte midagi peale plahvatuse. Seal oli tüüpe, kes on tänapäeval Eestis kõik
väga asjalikud.

\question{Keskkoolinoorena astrofüüsikutega sammu pidada ja originaalkeeles 
Tolkieni lugeda ei ole lihtne. Sa pidid ikka nutikas 
inimene olema.}

Mul oli sõnaraamat kõrval, kuni ühel hetkel polnud seda
enam vaja. Treffneris\index{Hugo Treffneri Gümnaasium} 
oli gümnaasiumis bioloogia-keemia õppesuund, kus õpetati ladina keelt, 
tavalist inglise keelt, aga ka teaduslikku inglise keelt, mille 
kõrval tavaline inglise keel oli \emph{walk in the park}. Seda ainet andis 
muide bioloogiaõpetaja\sidenote{Õpetaja Tago Sarapuu\index[ppl]{Sarapuu, Tago} 
õpetas ka Meelis Roosile\index[ppl]{Roos, Meelis} bioloogiat.}, kes oli tugeva akadeemilise taustaga ja 
tegi hiljem pikalt akadeemilist karjääri. Sa 
mainisid sinise kaanega C-õpikut. Kujuta ette, et sul on näiteks geneetika 
kohta samasugune ning sa võtad ja närid ennast sellest lihtsalt läbi. Paberit lendab 
kahele poole, aga sa tõlgid selle kõik ära. See aitas hiljem väga hästi kaasa.

Meil oli saksa keel ka, ma sain saksa keele lõpueksamil kooli parima hinde. 
Aga miks? Sellepärast, et Kaur Virunurmel\index[ppl]{Virunurm, Kaur} oli samal 
ajal ülikoolis saksa keele eksam. Ta tõmbas sõnaraamatu arvutisse, tegi sellest baasi ja siis oli 
võimalik skoorida: õige vastus andis punkti, vale vastus miinuspunkti. Mina 
valmistusin saksa keele eksamiks niimoodi, et viimasel õhtu enne eksamit mängisin punktide peale saksa keele sõnade tõlkimist, ja 
see aitas mind väga hästi. See oli üks esimesi kordasid, kus 
võin kindlalt väita, et infotehnoloogiline tööriist parandas mu sooritust 
hüppeliselt. Lõpuklassis oli mul küll saksa keel vist ühel veerandil ka kaks, aga 
see oli ealine iseärasus, hinded ja teadmised ei ole alati 
omavahel lineaarses seoses.

Kui nüüd muusika juurde tagasi tulla, siis Miina Härma \index{Miina Härma Gümnaasium} kutid tõid tähetorni
The Smithsi, The Cure'i ja 
ka vene muusikat. Ilmusid välja kitarrid ja 
midagi ka lindistati, ilmselt kassettidele. 

Samas oli tähetornis nõukogudehõnguline teaduskultuur, mille juurde 
käis näiteks konjaki ja kohvi joomine. Keskkooli- ja 
üliõpilased ei saanud seda küll endale lubada, aga tubades oli see hõng üleval. Ma ei 
teagi, mis asjaoludel seal joodi, sest läbusid 
ei toimunud, aga see kõik tekitas erilise atmosfääri.

\question{Seal tehti ju teadust ja mitte halba.}

Just. Ma käisin isegi astronoomiaringis ja mind saadeti 
oma tööga kuskile rahvusvahelisele
õpilaskonverentsile esinema. Kõike sai teha.

\question{Kuhu sa pärast keskkooli õppima läksid?}

Tahtsin minna ülikooli ajakirjandust õppima. Ülikooli mitteametlik \emph{statement} oli see, 
et kui tahad tulla ajakirjandust õppima, siis pidi olema ette näidata portfoolio ehk
pidid olema midagi avaldanud. Ajakirjanduse või üldse meedia õpetamine on 
suhteliselt kallim tegevus kui näiteks keeleõpe ja nad tahtsid olla veendunud, et üliõpilane tõesti tahab ajakirjandust õppida, mitte ei astu 
juhuslikult sisse. Mulle tundus, et kultuuriajakirjanik on äge olla. Ilmselt selles vanuses 
arvab iga mees, et kultuuriajakirjanik on äge olla, sest on olemas oma arvamus maailmast, mille masstiražeerimine tundub 
veidral kombel teiste aitamisena.

Ühesõnaga, tegin ettevalmistusi: käisin 
kontsertidel, kirjutasin intervjuusid ja arvustusi. 
Aga samal ajal käisin ka näiteringis. Lõpuklassis olid veebruaris-märtsis 
Viljandis lavaka sisseastumiskatsete 
eelvoorud. Mõtlesin, et äge oleks minna Viljandisse trallima ja 
seiklema, aga sain eelvoorust edasi ja hiljem lavakasse sisse.

Ma ei pabistanud üldse ja eks see aitas. Teadsin, et mul on
ajakirjandusega plaanid ja head soovituskirjad 
toonastelt tuttavatelt Postimehe\index{Postimees} ajakirjanikelt. Õppisin 
lavakas ligi aasta, ent ühel hetkel sain aru, et see ei ole ikkagi see, mida ma teha tahan. 
Sellele otsusele jõudmist mõjutas kõik eelnevalt kirjeldatu 
väga tugevalt. See „lendamistrenn“ paistis kogu aeg aknast. 

Kooli kõrvalt sattusin sellisesse ägedasse kohta nagu Riigikogu 
Kantselei\index{Riigikogu Kantselei}.

\question{Kas sel ajal olid seal juba võrgud ja BBSid?}

Riigikogu Kantseleis oli täiesti adekvaatne kraam juba aastal 1992. 
Ühtlasi üks äge tekstipõhise kasutajaliidesega võrgumäng, 
põhimõtteliselt tolle aja Fortnite, mille nimi oli 
MUME\sidenote{Üks populaarsemaid MUD-tüüpi mänge, mille nimi tuleneb fraasist 
\emph{Multi-Users in Middle-Earth} ja mis põhines 1991. aastal loodud ja siiani 
aktiivselt arendataval DikuMUDi\index{Muda!DikuMUD} mootoril. Vt ka märkust \ref{sidenote!muda} 
lk \pageref{sidenote!muda}.}, tegu oli Tolkieni ainetel loodud 
Mudaga\index{Muda}. Tuletan meelde, et need ringkonnad olid kõik 
tugevalt Tolkieni usku.

\question{Kas seal oli server, kuhu mängijad külge läksid?}

Jah. Telneti pordi kaudu tõmmati sind külge, kõik istusid oma 
\emph{socket}'is\sidenote{St. omasid iseseisvat ühendust serveriga.}, aga nägid, mida teised teevad. Ja kuna see oli 
tekstipõhine, siis võrguühenduse kiirus ei olnud probleem.

\question{Kuidas adekvaatne kraam Riigikogu Kantseleisse\index{Riigikogu Kantselei} 
sai? 1992. aastal ei olnud Eesti Vabariik veel kuigi 
heal järjel ja oli muudki, mida korrastada.}

Väga hea küsimus. Tarvi 
Martens\index[ppl]{Martens, Tarvi} kindlasti teab seda. Samuti
Toomas Mölder\index[ppl]{Mölder, Toomas}, kes oli nlibi, 
tollase Rahvusraamatukogu\index{Rahvusraamatukogu} IT-juht tänapäeva mõistes. Ja eks KBFI\index{KBFI} rahvas aitas ka.

\question{Kas peale MUME'i mängimise tegite seal kasulikke asju ka?}

Ma käisin tol ajal lavakas ja teadsin küll, et nad teevad 
vingeid asju, aga tavaliselt siis, kui mina sinna saabusin, 
lõppes töö ära, sest tuli \emph{orc}'ideks kehastuda ja minna 
\emph{whiteskin}'e tapma. MUME'i tekitatav adrenaliinitase ei jäänud alla 
tänapäevaste arvutimängude omale. Näiteks olid seal haldjas ja sulle tuli 
teade: \enquote{\emph{An orc enters the room}.} Selle peale lükkab ka täna teatud 
seltskonnal vererõhu kakskümmend protsenti ülespoole. MUDe oli veel, aga 
MUME oli üks esimesi selliseid mänge, mis kestis aastaid. Esimesed 
eestlased, kes seal mängisid, tegid oma tegelased aastal 1991 või 1992 ja mängisid kolm-neli aastat järjest. Pronto\index[ppl]{Pronto} ehk Tanel Raja mängis muide 
väga kõvasti MUME'i. 

\question{Mäletan, et 1993. aastal sisenesid mingid inimesed Liivi tänaval 
VAXi klassi\index{Tartu Ülikool!Liivi Õppehoone!Vase klass}\sidenote{\label{sidenote!vaks}Sõna \enquote{vask} mitmesuguste 
variatsioonidega kutsutud ja ilmselt klassi toitnud arvuti tüübinime 
VAX\index{VAX} järgi 
nime saanud klass asus matemaatikateaduskonna Liivi tänava 
õppehoone esimesel korrusel ja koosnes vask.ut.ee\index{vask.ut.ee} 
külge ühendatud terminalidest.} ja kui mina ükskord ülikooli lõpetasin, siis nad 
olid ikka veel seal.}

Jah, \enquote{pidev \emph{lag}} oli muide sealsamas VAXi klassi kõrval olevas 
ES-klassis, mis oli alati tühi, kuna need arvutid olid jamad\sidenote{Ilmselt 
peab Jaanus silmas Raua\index{raud.ut.ee} klassi, mis käis päris IBMi 
riistvara, mitte ESi peal. Raul Tölp\index[ppl]{Tölp, Raul} meenutab: „Sattusin kas 
1996. või 1997. aastal Liivi 2 hoonesse, kus mul paluti IBMi esindajana teha raud.ut.ee 
serverile masina viimane \emph{power off}. Räägiti, et masin küttis 
vesijahutusega tervet maja.}. Liivi tänava 
VAXi klass on omaette peatükk, milleni kohe jõuame.

Kui nüüd õpingute juurde tagasi tulla, siis lahkusin lavakast esimese kursuse viimasel veerandil. Üheksateistaastane laseb end
välisest tugevalt mõjutada. Tollal erines
näitlejaamet sada protsenti sellest, mis see täna on. 
Oli nädalaid, kui jõin vähemalt pool pudelit viina päevas. 
Organism oli tugev, vedas ilusti välja, aga nägin kutte, kes olid 
seda kümme aastat teinud. Ühel hetkel küsisin endalt, kas ma jaksaksin 
ja tahaksin niimoodi elada. \emph{Hell no}! See ja „lendamistrenn“ akna taga aitasid äratuleku otsust teha. 

Siis läksin Tartusse\index{Tartu Ülikool} eesti keelt, täpsemalt 
arvutuslingvistikat õppima.

\question{Kes seda õpetas? See oli tollal mujal maailmaski suhteliselt uus ala.}

Nüüd jään vastuse võlgu. Seal oli üks lahe lühike vanamees, kes 
oli tõeline guru. Hästi viisakas, vaikne ja rahulik sell, nii palju kui 
mina temaga suhtlesin. Aga tema juurde jõudsin alles kolmandal aastal pärast 
spetsialiseerumist. Enne olime lihtsalt ühes toredas teaduskonnas, kus 
põhiliselt õppisid tüdrukud.

Selle taustal oli mul ikkagi tunne, et peaksin ka mingit tööd tegema. Samas olin
ainult „lendamistrennis“ käinud. Mängu tuli seesama Vase klass.

Mängisime seal \emph{StackMUD}i\index{Muda}. Stacken.kth.se\index{stacken.kth.se} oli Rootsi 
Kuningliku Tehnikaülikooli\index{Rootsi Kuninglik Tehnikaülikool} 
VAX\index{VAX}\sidenote{Virtual Address Extension – arvutisari, mille töötas välja DEC
seitsmekümnendate keskel. Siiani üks tuntumaid omalaadseid arhitektuure,
oli see PDP-11\index{PDP-11} edasiarendus, peamiselt mälu virtuaalse
adresseerimise suunas.}, mille peal 
jooksis BSD\index{BSD}, mille peal pandi käima Muda. 
Originaalne DikuMUD on muidu tehtud Taanis.

Mängimise tegi huvitavaks see, et mängijad ei olnud matemaatika üliõpilased, 
nagu oleks võinud arvata, vaid eesti keele ja usuteaduse 
üliõpilased. Näiteks praegune kirjanik ja usuteaduskonna õppejõud Meelis 
Friedenthal\index[ppl]{Friedenthal, Meelis} oli väga originaalne mudamängija. 
Oli teisigi tüüpe, kes käisid tõesti palju mängimas, mina 
sealhulgas. Suhtlus selle seltskonnaga ei piirdunud ainult 
mängimisega, me ka ehitasime seda maailma. Ma olin üks põhilisi ehitajaid.

\question{Kuidas see käis? Kas kirjutasid koodi või skripti?}

See oli väga lihtne: sain koodi koopia ja mul oli andmebaasi struktuur, kus 
täitsin väljad ära. Võtsin andmebaasi \emph{dump}'i 
teksti \emph{editor}'is lahti ja tegin näiteks ridadest sada 
kuni tuhat koopia ning kirjutasin sinna asjad teistmoodi. \emph{Editor} oli loomulikult vi\sidenote{Unixi spetsifikatsiooni osaks 
saanud, 1976. aastal kirjutatud tekstiredaktor, mis on siiani teatud 
ringkondades (ka käesolev tekst sünnib osalt vi abil) väga populaarne. Siin 
kontekstis on oluline, et erinevalt tänapäevastest tekstiredaktoritest ei 
olnud vi ainult tekstipõhine, vaid ka suhtles kasutajaga 
ähmaste käsu- ja klahvikombinatsioonide abil. Näiteks on 
legendaarsed algajate kasutajate tulutud katsed redaktorist väljuda, kuna 
selleks kasutatavad klahvikombinatsioonid \texttt{ZZ}, \texttt{:q!} ja veel kümmekond 
samalaadset ei ole just intuitiivsed.}. Kirjutamine käis tsoonide kaupa: ühes tsoonis oli 
sada ruumi ja iga ruumi kohta kirjutasin ingliskeelse kirjelduse. Mitmesuguseid asju sai 
ruumis olla vist kuni 255, kolle sai ka olla teatud
kogus. Täitsid kõigi ruumide kohta statistika ära, lisasin kirjeldused ja
tegevused ning postitasin. Kutid kompileerisid selle ära ja nii see tuli. 

Me saime Rootsi mängutegijatega üsna hästi läbi ning rääkisime 
vahel ka olmest ja inimlikest teemadest. Näiteks et meil on ainult üks klass ja 
seegi on kogu aeg pooltäis, ja kui mõni ahv FTPga tont teab mida tõmbab, ei 
saa üldse mängida. Nemad ütlesid, 
et neil on üks arvuti üle, kuna said uuema VAXi\index{VAX}, mille peal on 
BSD\index{BSD}. Ma küsisin, kas me vana arvutit kuidagi endale ei 
saaks, ja öeldi, et saate küll. 

\question{See masin oli ju Rootsis.} \label{sisu!jaanus_liivi_tn}

Jah. Anne Villems soovitas mul rääkida Otto Telleriga\index[ppl]{Teller, Otto}, kes oli vist 
arvutiteaduse õppetooli juht\sidenote{Tõenäoliselt toimetas Otto Teller siiski Tartu Ülikooli 
arvutuskeskuses\index{Tartu Ülikool!Arvutuskeskus}, mis oli 
eraldiseisev üksus.}, ja ütlesin, et ilmselt te mind ei usu, aga Rootsis on üks arvuti. Sellega tuleb kaasa 16 terminali, 
me saaksime teha terve klassi ja võin ise seda 
administreerida. (Kui muda sees mängid, siis oled kaelani porine, st 
süsteemide administreerimise oskus tekkis iseenesest.) 
Aga ma olen lihtsalt üliõpilane ega tööta siin, palun aidake. Lõpuks Otto Teller vastas, et see on küll kõik
väga imelik, aga olgu, ja ajas asja korda.

Siis kirjutasin igale poole kirju ja sain vastuseid. Ma ei tea, mida Otto 
Teller tegi, aga ta hüppas pea ees tundmatusse mingi kahekümneaastase 
kuti kätega vehkimise peale. 1993. sügisel sõitsime Rootsi klassi 
üle vaatama ja talvel oli korraga üks furgoon 
Laia tänava ukse taga. Tekkis Laia tänava arvutiklass\index{Tartu Ülikool!Laia tänava 
arvutiklass} ja mina sain selle administraatoriks. See oli minu 
esimene töökoht.

Tollal ei olnud administraator ainult tehniline töötaja, vaid ka 
administratiivne tegelane, kohalik jumal. Oleks mul võimuiha olnud, 
võinuksin seda väga hästi realiseerida, aga ma andsin hunnikule tüüpidele 
võtmed ja palusin, et nad serveriruumi ei läheks. Kuskilt 
veeti eraldi kaablid, Zyxeli\sidenote{1988. aastal 
Taiwanil asutatud Zyxel Communications Corporation tootis 
ülipopulaarseid ja hinnatud modemeid.}\index{Zyxel} modemid said üle püsiliini 
internetiühenduse ja seal me müttasime.

Olgem ausad, tänapäeva mõistes oli see klass totaalne õnnetus. Nii 
madala käideldavusega asja pole ma hiljem näinud. Arvuti oli väga 
vana, läks tihti katki ja ma ei tundnud nii professionaalset raudvara hästi. Mul oli abiks Viljo 
Soo\index[ppl]{Soo, Viljo}, kes oli tänapäeva mõistes \emph{sysadmin} ja kes 
aitas masinat palju kordi käima panna. Läks paar kuud ja 
saime klassi tööle. Klassi nimi oli Cure\index{cure.ut.ee}, 
The Cure'i järgi. 

Kunagi oli selline mäng nagu Nethack\index{Nethack}\label{sisu:nethack} ja sellest oli 
üks naljakas kloon tehtud. Otsustasin selle eesti keelde 
tõlkida. See käis põhimõtteliselt samamoodi: võtsin C koodi lahti ja 
hakkasin esimesest reast lugema.

\question{NetHacki lähtekood on niisamagi hea lugemine, see on üks kahest, 
mida ma olen oma elus lugemise eesmärgil välja trükkinud. Teine on Perl.}

Ühesõnaga ma tõlkisin kõik eesti keelde esimesest reast viimaseni, kaasa arvatud \emph{library}'d ja kõik muu, mis kaasas oli. Põhilise ekraaniteadete osani jõudsin 
kell neli hommikul. Teadupärast tekib väga suure väsimuse korral ühel hetkel veider pooleufooriline meeleolu. Mul saabus see hetk keset tõlget ja mäng kukkus
naljakas välja\sidenote{Mängus tembutanud \enquote{mõõkhambulisi varblasi} meenutatakse siiani hea sõnaga.}. Seda mängiti klassis väga palju, seda enam, et 
võrguühendus alati ei töötanud, aga NetHack oli kohalik. Senikaua, 
kuni keegi Viljo Sood\index[ppl]{Soo, Viljo} otsis, et ta 
modemitele restardi teeks, mängiti NetHacki. Ma ise ei pidanud seda suureks saavutuseks, lihtsalt tegin valmis. Kahjuks läks kood koos Cure'i 
masinaga kaduma.

\question{Nutikal inimesel oli tollal tüüpiliselt kaks suunda, kuhu kiskus: 
kas akadeemilisse maailma teadust tegema või äri suunas. Kas sind ei tõmmanud 
kumbki?}

Mind äri ei tõmmanud, sest olen pärit äärmiselt vaesest perekonnast. Raha ei 
olnud midagi erilist, mul lihtsalt ei olnud seda kunagi. Kuude kaupa elasin saiast ja piimast. Ja kui raha ei ole, siis ei teki sellega ka lähedast 
suhet. Mis puutub akadeemilise maailma, siis olin sel ajal
alles esimesel kursusel.

Cure'i klassi tegemisest mõni kuu hiljem toimus üks tüvikursus. Pildile ilmus taas Anne Villems\index[ppl]{Villems, 
Anne}, kes korraldas 1994. aasta alguses Eesti esimesed \emph{webmaster}'ite kursused. 
Liivi tänaval olid kuulutused üleval.

Mina olin siis oma arust juba kõva käpp. 
Tol ajal kasutati Gopherit\index{Gopher}\sidenote{Gopher oli varajane hüpertekstiprotokoll, WWW 
protokollistiku eellane. Erinevalt suhteliselt lõdvalt struktureeritud veebist 
surus Gopher sisu küllalt rangesse hierarhiasse ja oli navigeeritav 
menüüsüsteemi abil.}, HTML 1.0 standardi eelkäijat, millest oli lihtne aru saada: oli klient, server ja \emph{markup language}, mille 
põhimõttest sai kümne minutiga aru.

Kursusel selgus, et arusaamisega läheb natuke 
rohkem kui kümme minutit. Kursusel käinud seltskond oli 
kirju, sinna sattus igasuguseid karvaseid ja sulelisi erinevatest 
teaduskondadest. Teiste hulgas oli seal näiteks Anto Veldre\index[ppl]{Veldre, Anto}, aga 
ma ei mäleta, kas õpetaja või õpilasena.

\question{Mis seal ikka nii väga vahet on.}

Tollal ei olnud jah vahet. Üks oli asja läbi lugenud ja rääkis teistele 
edasi. Aga Anne Villems\index[ppl]{Villems, Anne} oli kursuse väga hästi ette valmistanud. Päris kohe selle järel otseselt midagi suurt küll veel ei juhtunud, aga osalejate nimekiri jäi alles. 

Kui EENet\index{EENet} tegi endale veebilehte, olid nad kuulnud, et on olemas
\emph{webmaster}'id, kes oskavad veebi teha, tänu millele tekib organisatsioonis 
avaliku infohalduse funktsioon. (Nüüd oskan ma seda niimoodi nimetada, aga 
tollal panime lihtsalt asju internetti, näiteks 
võtsime Eesti kaardi ja sellele vajutades juhtus midagi.) EENet otsustas teha endale täiskohaga \emph{webmaster}'i ametikoha. Võimalik, et seni oli seda tööd teinud mõni ülimalt nutikas 
sekretär või tollal seal töötanud Marek 
Tiits\index[ppl]{Tiits, Marek}, aga 
asi lõppes sellega, et pool aastat pärast kursust kutsus EENet mind
\emph{webmaster}'iks. Palk oli kolm korda suurem kui arvutiklassis, nii et raske oli ei öelda.

Sealt edasi läks elu väga ägedaks. Saime tuttavaks Tarvi 
Martensi\index[ppl]{Martens, Tarvi} ja Toomas 
Mölderiga\index[ppl]{Mölder, Toomas}, tegime EENetile korraliku veebi ning käisime Eestit esindamas. Tollal oli veeb küll korralik, 
aga hiljem tehti veel ägedamaks, kui liitus näiteks Pille 
Pruulmann-Vengerfeldt\index[ppl]{Pruulmann-Vengerfeldt, Pille}, kes on praegu
Rootsis meediaprofessor\sidenote{Intervjuu läbi viimise ajal oli 
Pille Pruulmann-Vengerfeldt Malmö ülikoolis meedia ja kommunikatsiooni professor} ja 
ERRi\index{Eesti Rahvusringhääling} nõukogu 
liige. Seal puutusin Marek Tiitsu\index[ppl]{Tiits, Marek} kaudu esimest 
korda kokku europrojektidega. Tollal küll veel ei olnud euro, vaid eküü\sidenote{Selle valuuta tähiseks oli ECU: 
\emph{European Currency Unit}.}. 

\question{Kas Marek oli see võlur, kes valdas unikaalset teadmist, 
kuidas fondidest raha saab?}

Just. Mina tulin lagedale veidrate ja üsna ebareaalsete
ideedega ja tema kasutas väikest osa neist hullustest, millel oli mingi point, projektides ära.

\question{Tol ajal liikus EENeti ja IBSi\index{IBS|see{Institute of Baltic 
Studies}}\index{IBS} kaudu tohutult palju põnevaid ja kasulikke projekte.}

Seal jooksis igasuguseid naljakaid teenuseid, aga see polnud kõik, millega tegeleti. 
Näiteks suutis Marek hankida mulle tööarvutiks Silicon 
Graphicsi\index{Silicon Graphics}\sidenote{1990. aastal asutatud ja 
2009. aastal pankrotistunud Silicon Graphics oli peamiselt 3D-graafikale 
keskendunud riist- ja tarkvaratootja. Mitmed varased arvutiabi kasutanud 
filmid, näiteks 1993. aastal linastunud „Jurassic Park“, kasutasid just Silicon 
Graphicsi tööriistu. Ettevõttele tegi lõpu odavate laiatarbe x86-arvutite 
võimsuse kiire kasv.} masina. Silicon Graphics oli väga kõva asi, tänapäeval 
võibolla võrreldav Mac Proga. Ulmeline aparaat ja milline disain! 
Korpused olid värvilised, kõvaketas 
käis lahti kangiga. See oli nagu automaailma 
Bugatti või Porsche 911, täiesti \emph{over the 
edge}. Mõni ime, et ma selle töö hea meelega vastu võtsin, olles tulnud 
totaalselt vananenud VAXi klassi administraatori kohalt.

Kõige selle taga oli Richard Villemsi\index[ppl]{Villems, Richard} pikk, aga heledat ja positiivset 
värvi vari. Tema seda kõike püsti hoidis.

\question{Hoidis püsti, aga vist ka natuke nagu joonte sees, et inimesed päris 
hullustega ei tegeleks?}

Jaa. Richard Villems on muide Anne Villemsi\index[ppl]{Villems, Anne} abikaasa. 
Tänu Richardi suurele 
mõjule toimusid EENetis\index{EENet} väga kõvade projektide arutelud, 
mis ei käinud tegelikult üldse selle organisatsiooni põhikirjajärgse tegevuse alla. 
Sealsamas Liivi tänaval oli ka ülikool, arvutiteaduse 
instituut\index{Tartu Ülikool!Matemaatikateaduskond!Arvutiteaduse instituut}, samuti füüsikamaja. Nii et püssirohtu jagus ja 
tekkis igasuguseid initsiatiive.

\question{Kas selles maailmas BBSid ka kuidagi figureerisid?}

BBSid olid kogu aeg taustal ja mõnes olid mul isegi kasutajad, käisin
isegi seal. Aga BBSi tehnoloogiline \emph{carrier} oli modemiga üle
telefoniliini ühendumine serverisse. Mina sain varakult väga kiire 
interneti juurde, 64 kilobitti sekundis on tuntavalt kiire. 

Edasi tulid väga kiiresti IRCd\sidenote{Internet Relay Chat 
(IRC) on kliendi-serveri arhitektuuril põhinev tekstipõhise kommunikatsiooni protokoll. 
Peamiselt disainitud suhtluseks suuremates gruppides, kuid 
võimaldab ka üks ühele suhtlust.}, mis võtsid BBSide funktsiooni üle. 

EENetis\index{EENet} toimus palju lahedaid asju, mõnes mõttes oli see 
ebaproportsionaalselt nähtav organisatsioon. Näiteks aitasid
nad korraldada koolidesse internetti.

\question{Lisaks sellele, et keegi kuskil poliitilisi otsuseid teeb, peab keegi 
suutma ja viitsima neid otsuseid ka ellu viia. Sõita talveööl 
Põlva kanti kooli modemeid installeerima ei ole palga eest tehtav asi.}

Seesama \emph{case} võtab kokku kogu tolle aja mentaliteedi. 
Seda üldse ei arutatudki, kui palju koolide internetti 
ühendamine maksab, kuna paljudes kohtades raha polnudki. 
Arutati ainult ühte asja: tuleb teha ja Marek\index[ppl]{Tiits, 
Marek} otsib, kust raha saab. Marek ei kantinud raha autodeks ega suvilateks, Marek tegi europrojekti ja tuli. Põmm, kümme 
Suni. Põmm, kakskümmend Zyxelit. Paar-kolm kutti viisid asja ellu ja 
keegi ei pahandanud. Aeg-ajalt käis mõni Antsla kooli mees 
küsimas, kuidas läheb ja kas oktoobris tuleb. Ja tuligi, kuigi vahel 
novembris. Keegi ei arutanud, kas teha. Prooviti vaid vaadata, et teenuse laienedes kvaliteet ei kukuks. 

\question{Mis seda kõike edasi vedas?}

Küllap iga agraarühiskonna tung harida maad, kus midagi ei kasva. Mõnes mõttes oli see lihtne: prioriteedi määras 
see, milline kool kõige rohkem ise huvi tundis. Alguses ei olnudki huvi suur, sest ei saadud aru, millest üldse jutt käis.

\question{See on väga õige lähenemine, et kõige suuremad hädalised, kes 
kõige rohkem oskasid internetiga midagi teha, said selle ka esimesena 
kätte.}

Ma ei tea, võibolla mõnesse kohta jõudiski internet alles 2000. aastal, aga 
vahet ei olnud, sest selleks ajaks oli klõps juba 
ära käinud. Nii naljakas kui see ka pole, aga üheksakümmend protsenti tööst oli 
veel tegemata, kui kümme protsenti internetiühendusega koole oli kaalu juba 
nii alla vajutanud, et ülejäänute puhul oli üksnes aja küsimus, millal juhe nendeni 
viiakse. Aga et seda kõike oli vaja, oli Anne Villemsi\index[ppl]{Villems, 
Anne} ja tema pundi sügav veendumus. 

\question{Mida sa praegu teed?}

Püüan avalik-õiguslikus meedias saada ühele poole transformatsiooniga, mille eraõiguslik meedia on kümmekond aastat tagasi ära teinud.

\question{Kindlasti vääriline töö, kus väljakutseid jagub.}

Kui avalik-õiguslik ringhääling\index{Eesti Rahvusringhääling} ise kaua aega ei 
tunnetanud, et peaks internetikasutajakeskse hüppe tegema, 
siis oli ka teistel institutsioonidel raske seda nende eest ära 
tunnetada. Selle tagajärjel tekkisid mitmed fundamentaalsed küsimused: 
kuidas te ütlete, et teil on sellise asja jaoks raha tarvis? Aga kus te siis 
olite, kui teised organisatsioonid sellega tegelesid? Riigi jaoks on olnud keeruline 
mõista, et kui avalik-õiguslik ringhääling niisuguse hüppe ette võtab, on see
ikkagi ookeani ületamine ja parvega seda ära ei tee.

\question{Ja kui terve riik on läinud teisele poole ookeani, siis on pisut
sandisti, kui ERR teisele poole maha jääb.}

Sellal kui suurem osa audiovisuaalsest meediast toimus kinoringvaate vormis (oli 
filmilint, mis ilmutati ja mida projektori abil näidati), oli 
televisioonil kuuskümmend aastat tagasi juba \emph{live}-signaali halduse kontseptsioon, mis töötas ja oli 
piisavalt lollikindel, et sellega Eestis eetris olla. 
Televisiooni tehnoloogia on arenenud oma kinniste protokollide ja 
signaalihaldusmudelitega ning oli internetist kaua aega signaali 
loogika poolest maas. Teleasi maksab muidugi ulmeliselt palju, aga 
see on olnud kogu aeg terviklik kinnine maailm, mis arenes teist 
evolutsioonipuu haru mööda. Mõni aasta tagasi jõudsid Euroopa 
Ringhäälingute Liit ja teised üleilmsed ringhäälinguorganisatsioonid oma standarditega nii kaugele, et on olemas IP-põhine 
signaalihaldusstandard, mis ei ole veel\sidenote{Jutuajamine Jaanusega toimus novembris 2019} valmis, aga millest mõned tükid 
töötavad. Aga nad tulid sellega lagedale aastal 2017. Mõtle, kui kaua aega on olnud 
normaalselt töötav internet.

Praegu saame öelda, et tegelikult ei ole mõttekas mitte-IP-põhist 
tehnoloogiat ehitada, aga tollal läks terve tööstusharu teist rada pidi kaugele 
edasi.

\question{Seega ERRi vaatepunktist mitte ainult ei ületata parvega ookeani, vaid parvel on 
känguru ja hobune, keda üritatakse panna kuidagi järglasi saama.}

Sealjuures on veel tugevad kogemused hundiga, kes puhub puust ja õlgedest 
maja ära. Järelikult on parv tehtud igaks juhuks betoonist. 

Meil on nii äraspidised kogemused, et puust paadi kontseptsioon 
tundub algatuseks lihtsalt ohtlik. Televisioonisignaali haldusloogika seisneb selles, et ehitame 
signaali nii, et see ei saaks katkeda. Kui palju see maksab? Nii palju, kui vaja! Teeme 
nii, et ei katke! IP-põhine paketihaldusloogika ütleb, et lükkame paketid läbi ja, kui vaja, parandame. Need on 
fundamentaalselt erinevad mudelid, aga pikas plaanis on parandamine 
odavam kui kohe hästi tegemine.


\chapter{Kaspar Loit}
\index[ppl]{Loit, Kaspar}
\index[ppl]{B'Knows}
\index[ppl]{B'Knows|see{Loit, Kaspar}}

\question{Kes sa oled?}

Mina olen Kaspar. Ja kunagi, kuna see võtame selle teema, et me peame tagasi kerima mingisugune miljard aastat, siis toona see aka oli B'Knows. 

\question{Aga kust sa said sihukese aka?}

Seda ei mäleta enam keegi. Seal on nagu kaks komponenti. Üks on nagu \enquote{B} ja siis on nagu \enquote{knows}, ehk siis see B peaks  midagi nagu teadma. Pronto\index[ppl]{Pronto} alati kutsus mind Buttknows.

\question{Kuidas sina arvutite juurde said või arvutid said sinu juurde?}

Mul on selge mälupilt, et mu tädi, kes on superkuul ja minust mõnevõrra vanem, töötas Tartus
vist Bioloogia Instituudis. Ja  talle oli kuidagi jäänud mulje, et mind võivad huvitada sellised asjad. Ma arvan, et ma olin mingi, ma ei tea, kaheksa-üheksa-kümme, \emph{something like that}. Ja siis, kui ma tal ükskord Tartus külas käisin, ta viis mind instituuti. Muidugi peale tööd, kõik oli juba pime ja  seal oli mingi kabinet lahti ja seal seisis laua peal mingisugune masin, mille nimi oli Apple II Europlus\index{Arvutid!Apple II}\sidenote{Apple II Europlus oli Apple Euroopa turule kohandatud versioon. Muu hulgas erines toiteblokk aga ka video osa tuli ümber teha, sest Steve Wozniaki trikid NCTS signaali genereerimisel ei toiminud enam keerukama PAL süsteemi puhul}. See oli nagu legendaarne selline nagu fakinossem. Ja, ja, ja seal sai, ta oli seal mingi laborandi käest küsinud, et kuidas seal midagi käima, enne seda üles kirjutades sai mingi plaadiga mingisugune maine, sellel paar mängu, mis olid teksti ekraanil, jooksid ülisuper ägedad. Ja eks see oli vist mingi trein Roveri, ma mäletan, ma olen seda asja, see pilt silme ees. Ja sellest hetkest ma arvan, ma olin, möödus ka, et ma, ma ei oska nagu meenutada, kas, kas ma enne olin kokku puutunud juba seal? Tõenäoliselt mitte, see oli ikkagi liiga liiga vara ja, ja ma arvan, et selliseid, esimene. Ja, ja kuna mul on mulle tohutult meeldisid koolis Nintendo väikseid Keymengu otsmängud. Võib-olla mäletate, kuidas ei, ei mäleta. Ühesõnaga siuksed, ma ei tea, tänapäeva telefonisuurused umbes. Ja, ja see oli LCD ekraan, millesse oli nagu ette programmeeritud ette joonistatud mingisugused tegelased ja siis nuppudega näoga täna kaugele neid noh, Janno paha nii, ja, ja kogu selle aja eest. Aga Nintendo oli see mingi põhiline tootja neile, neil olid siuksed väga-väga lahedad mängud ja siis ma või isegi kunagi mõtlesin, et oh, kui lahe oleks, kui saaks ise niisuguseid teha, aga no ma sain aru, et seal taga on mingi tootmine ja see ei ole nagu reaalne on ju. Ja nüüd järsku saavad aru, et tegelikult selliseid asju on võimalik sinna noh, nagu masina sisse programmeerida, firmad seda minid elektroonikaskeemi tootma või mingisuguseid noh, muide see võib üldse mingit tehast olema, et see oli nagu see oli üks niisugune realism on ju see muidugi noh, tõenäoliselt viis mind kunagi ka sellele, et me vaikselt tegi mänge. Aga niisugune päris toimetamine hakkasin tööle. Kuskil oligi seal Jaak Laande nimi, Eesti Põlev jookseb ta kindlasti nagu ülioluline tegelane, sest et sellistel noh, siis oli selge see, et arvutile ligipääs oli see asi, mis oli nagu oluline ka valutaks. Ja, ja ma mäletan, et ma just siia sõites tuletasin meelde, et ma tegelikult olin kaardistanud omale kõik kohad, kus üldse nagu tõenäoliselt Eesti Vabariigis arvutite ligi pääses. Nõo oli liiga kaugel selgelt. Aga Tallinnas neid kohti, noh, need ikka olid. Aga Jaak Londoni nagu selles mõttes oli nagu lahe tegelane, et minu meelest ema kaudu ma esimest korda sain progeda. Kas te siis nagu andis selle võimaluse või ta õpetas ka või toika nagu õpetas loomulikult, sest et ega alguses oli lihtsalt mingi mingil põhjusel ma ei tea, kuidas ma sattusin sinna selline neljas keskkoolist või, või kolmas kurat, ja see oli see suur klass, seal olid mingisugused masinad, tõenäoliselt ka need olid emmessiksid. Ma arvan. Ja, ja see kari tegelasi paar tükki, ma tundsin nende kaudu vist kuidagi nendest jadad. Ma mäletan, et üks mu koolikaaslane nägi siukest masinat esimest korda ja meil öeldigi, et natuke midagi tegema selle teisikali esseevõistlusel lõksid ainet emmessiks selle sellepärast tuli üldse pull masinaid ta tegelikult vist mõeldi välja tagantjärgi tarkusega võin öelda selleks, et ühtlustada koduarvutite standardit ja peidikuid, milles nad jooks, mida nad jookseksid ja see oli isegi tegelikult algatatud initsiatiiv mingisuguse Jaapani Microsofti executive poolt, okei butis Peisikusse lihtsalt otsa. Saab võistlus katkestati pea niimoodi, et ma eile just ostsin välja omale ühe ühe emulaatorit tahab selle ekraani ette ja sa võid seal kirjutada kümme ja siis kirjutada reha, paks kerkivad teisegi, antud listist näitab, mis sul on, sa kirjutad uuesti kakskümmend, kirjutad selle rea üle ja, ja kirjutada on nii, et, et see tähendab, see on nagu su kohe käitma. Mõnus. Ja siis ma mäletan lihtsalt seda ka, kuidas mul võiks sõber, kes läks sinna, nägi seda asja esimest korda läinud, et nüüd siin saab midagi teada, kirjutas Liis deroomiiesse.
Ja noh, ütleme, et NLP Janar veel, nagu ta on, äkki on seal midagi tunud?
Aga ja siis seal üldse nagu tegelikult paljudes Karel Kannel oli seal kuidagi toimetas ja, ja kõik need väiksed tattnokad õitseksid Kaasani. Aga palju huvitavam oli tegelikult see, et Jaak Loonder oli ka üks masin, nimi oli Mir, kaks mingi nõukogudeaparaat, mingisugune nõukogudeaparaat, mis oli noh, siukse põhimõtteliselt oli ikka pikem kui viis meetrit oligi pikap poisikesest kõrgem Marje ja Jaagu leida võib-olla ninani.
Ja Tein meeletute häält, sest et oli nagu noh, põhiline osa ilmselt oli jahutuseni. Just seda tegigi villa kraažinis masinaid ilge müra ja siis aeg-ajalt, kui inimestel viskas nagu Kopli ette, siis nad lülitasid selle välja selle jahutuse ja siis see oli, tšekk tuli muidugi tohutult rääkis sest, et see kokku ja ilmselt on. Ja, ja see oli, selles mõttes oli ka nagu geniaalne vaim, kus ta seal üldse kätte saanud, see oli nagu nad lahendust, lahendust, masin, tal oli ikkagi võimalik klaviatuurist sisestada käske kus klaviatuur oli elektronkirjutusmasin, mis need põhimõtteliselt oli nagu klaveriprinter ühekorraga. Nii et nagu kaaned kapitaliga sellest samast masinast ja siis tal oli
Mustvalge telekas aga tal ei ole, kas valguspliiats ehk siis sa said nagu ekraanil tabada mingisuguseid punkte ja see masin tundis selle ära, ilmselt ta luges seda kuradi kineskoopkiirt ja selle järgi pani selle kokku. Ja, ja ta suutis ka mingisugust Rudimentaarselt mingit graafikat, Kuvalehti tal ei olnud nagu ainult teksti ekraan, vaid ta suutis ekraanile kuvada mingit punkti, see oli selle arvuti, eks niisugune tohutu ülesanne see punkti püsinud hästi paigaldas, ikkagi õrnalt ujus asjadega hakkama. Ja siis mu esimene programm, ma mäletan, oli, see oli muidugi see programmeerimiskeel oli vene keeles vene tähe, et kõik olid mingid lühendid super lõkseni ja, ja siis ma progesin mingisuguse graafilise neli tipulised tähele, sest ma arvan, et ta koosnes, ütleme siis Megist umbes kuueteistkümnest punktist tulla ja see ikka tõmbas selle arvutiga täiesti koomas ja kõik see ekraan ujus selle, aga väga uhked. Aga meil on veel lisaks veel, meile õpetas, oli näiteks polindi lugemist nad seda masimiseks õi kahte moodi meediat, üks oli paber, perfolint, siuke õhukene, mis lasti läbiteooriasse džäki, kotid, teravamad vennad suutsid torkida nõelaga augurauaga oleks võinud tegelikult seda perfolindi peale seda proge kirjutada, ma arvan.
Mul on selline tunne, et äkki oskasin.
Aga aga siis olid seal veel mingisugused vahvad asjad, mingisugused magnetkaardid, mis olid umbes sellise magnetkaart, jah, ta nagu perfokaarte ma küll tean, aga ma usun, et magnetkaart oli selline jälle niisugune umbes tänapäeva või tänavale telefonist suurem, noh niisugune mingisugune. Ma arvan, et mingi kaheksa senti korda mingi viisteist senti sihukene troon latakas, mis on põhjust meenutada oma materjalid, seda, mis floppy diski sees on tegelikult. Ja siis on põhjust, panid selle mingist latist sisse, tõmbas seda solisti läbi ta luges sealt midagi mingisuguses koguses hästi, läks ja luges keelde. Ja see oli nagu noh, siuke sel juhul müstika seda enam lugeda ei saadud, see. Aga see niisugune noor nagamannid, see peab olema ikkagi päriselt märksa tahtmine, et sellest ju ekraani peale tähejoonistamine, huvid oleks või see nagu mästidesta ja sellest tõesti sa, sa nagu kirjutasid midagi ja see pilt tekkis seda sinna ekraanile. Siis see huvitav oli just see mina andsin käsu, mis ta tegi midagi või? Noh, see olnud oleks olnud noh, täpselt, et kui see oleks nii lihtne, et plii, Stroomi eeskõneleja, siis see ilmselt oleks kaotanud huvi, aga see oli ikkagi kombigeiti värke seri. Selles oli mingi algkeemiline element, selline ülikõva ja seal oli nagu niisugune noh, mis maistele poistele meeldivad igasugused Salagi keeled ja igasugune koodid ja ma ei tea lipukirja Eestis asjad, et noh, selline nagu see oli kõike seda. Ja veel, mida see oli ikkagi super hästi kukuvad ja, ja, ja sisena jällegi, et sealt tekkis, aga noh, see noh, selge oli see, et see oli nagu meeletult piiratud on, et kaua sa seal ikka seda seal ja, ja lisaks sellele, et seda merre, et üks noh, õnneks emmessiks, las see suurem, aga seal oli vist jälle midagi mingisugused piirangud, kuna see oli keska olla onju ja, ja ilmselt sellepärast säpizzewiski. Ma arvan, et see oli kuidagi ega vist säkiga seotud semis roopa tänavale selle võttega, mida siin ka teistest lugudest läbi jooksnud minema. Ja seal oli siis ka terve klass.
Kus siis oli üks niisugune nagu juhtarvuti, millel oli mingisugune draiv? Ma eeldan, et see oli mingi flopidraiv. Ja, ja siis oli terve klassides arvutid, mis said sellest nagu peaarvutist omale alla laadida asju, isa füüsika, eks ole seal lihtsalt kirjutada oma neid programme, aga kuna kuna Trai oli ainult üks, siis kui sa tahtsid salvestada või midagi, et siis sa viid selle sinna, saab vaid siis tavaliselt see asi. Tegevus oli üsna nagu lihtne, et.
Kõik tšekid ja aeglasel väga nagu hängida on ja siis oli seal kogu aeg oli mingi poiste karjed olid talle lisaks bar nutikamalt vend olid siis nad pannud seda vedama. Üks legendaarne tegelane Emmucats ehk linnade edesiis tänasest kuskil Soomes toimetamine. Aga noh, tema oli nagu selgelt minu esimene nagu niisugune guru, keda ma nägin, et ta oli, ta oli, ma arvan, oli umbes minuvanune, aga ta oli omandanud juba täiesti kõik need peenemad alged, ehk siis põhiline, mida ta oskas, oli see, et oskas kahest programmijupist panna kui ühe terviku ja selle nagu paketeerideni eestlase laadideks. Point oli selles, et väga suur osa seda softi levis tavalist magnetofoni kassettidele. Ja vist oli see kuidagi, ma ei tea, kes oli võrguprotokollist kinni või sellest kas endist kinni või enam-vähem niisugune šehhivad mängud olid kolmkümmend kaks, kilobaiti umbes pikalt, noh see oli ka mässimine. Ja, aga selleks, et nad sinna ka sätivad ära mahuks, nad olid tehtud pooleks kuus kas kuusteistpidi vahepeal nagu keerama ja ühesõnaga see kogu see kasseti majandusele keeruline. Aga, aga kui seal oli juba nii kõva asi nagu flopidraiv, et siis sa said sellega seti pealt lugeda selle kuusteist kilo sisse siis tõsta kuskil mälus mujale ja siis lugeda teise kuusteist kilo, saanud kokku panna asju, tekkis see tervik, mida sai nagu Rootsi või kuidagimoodi, ühesõnaga see oli kõik täielik, seal juba seal juba supermaagia siis oli seal tegelikult tekkis sul siis niisugune teadmise põhinenud nagu eeskuju, see oli keegi inimene, kes, kes vaatasid alt ülesse sellepärast et ta teadis rohkem kui sina, oi, ealiseks tegelasi veel, seal oli mingisugused, ometi võiks ainult käivaid Kont, toimetuste eesnimi oli ühtlasi ka natukene tüsedam lendude, tal oli väikene kohvrikene metallkohver mille sees oli kogu emmessikse manual, see oli fotokopeeritud käes täkazzbeeber, siis ta käis sellega renni väga uhked saajat, et see lahti siis seda midagi selle alusel kirjutas see nagu jälle superlux. Eks tegelasi oli veel seal. Jaa. Ja ega mul seal ka nagu selles mõttes ongi, et kui on, sa istusid seal seal nagu otsest läks, sassi sellele draivi ei olnud, võib-olla seal seal mängisid, mingeid tüütas ära, on ju sisse kirjutasite sama teisikut. Ja sellele tuli mingid mingid piirid ette. Loomulikult miljonites graafika pool. Ma üritasin sinna midagi mingisuguseid pilte manada ja kuna emmessiksil oli tegelikult tal ei olnud graafika ekraani ja seda aimuleeriti tekstiekraaniga, ehk siis põhimõtteliselt iga pilt, iga mäng, mis emmessiksin toona jooksis, oli tegelikult otse mälus tahe generaatori ümber programmeerida. Juku ja juku tehti sama lugu, et seal said laadida mingisuguse oma nii-öelda fondi kuskile mällu. Ja see oli ikkagi tähed, olid mingid, panin siin teha asemele pit, näppisin ennast või mida iganes kokku võtta, et põhimõtteliselt sama sama laks, et seal oli noh, ekraan põhimõtteliselt emmessiksil oli otse aadresseeritav, et kui sa teadsid, et nad selle režiimi ega algas aadressil, seal mingisugune eksas mingi jõuate värvata veel on ja siis sa lähed sinna järjest panna iga, iga bait oli üks rida on ja tal oli, kas tal oli läbipaistev taustavärv või esivesivärve, siis veel mingites režiimides said need iga rida-realt neid värve vahetada, selline Välitel väest. Ehk siis põhimõtteliselt niisugune mõiste nagu ma ei tea, kas sa ilmselt oled kokku puutunud, et mängudes on siuksed tegelikult nagu Spraynyden ja need on need, mis liiguvad, eks ole, tausta ees. Et MSI CD-lt spaitega emuleeri sellesama tähe genega ehk siis kohalik tähe Gene Programmeeriti jooksvalt ringi, eks ekraan kirjutati täis ABCD, ehk ei ole ma nädal aega kõik märgid, mis talle pähe tulid, mis olid, on ja, ja siis need kogu aeg adresseeritud ja kirjutajate ring ja selle jälle iga ekraanist oli alati üks mingis kolmeks osaks. Igas osas sa said eri eri nagu tähestiku väänata ja, ja see oligi nagu see minu jaoks oli võlu, kui ma sain selgeks, et on olemas Assembler see Marissa, plaanida ühele mälu aadressile üks Payton ja ja siis ma mõttelist veetsin suure osa oma ärkveloleku ajast millimeetri paberile, joonistades mingisuguseid tegelasi neid tõlkides paineriks või eksakse siis laanide kuskile mällu. Aga kui ma nüüd refereerin tagasi, siis see ega, ega tänapäeval arvutigraafika ei ole ka lihtne. Aga see keerukus tundub olevat nagu teises kohas, et tänapäeval sa pead aru saama, sest kolm t/a, kolm tee geomeetrilist ja sa pead nendest spetsiifilistest kepihoopidest teadma ja nii edasi, et noh, seal on ikkagi leierzazdafan seal, vahel ka sellele, kas sa põhimõtteliselt võiksid ja panin nagu toorendiselt toppisid otse ekraanile selline, aga kui see sinna toored toppimine, et ega see ei olnud nii et sa ütlesid talle nüüd joonista sellesse kohta mingisugune asi, vaid seal pidin ikkagi tegema allikast algoritmilise tööd. Et noh, mis, kuidas ma seda värski maja ees ja nõia Graszek nagu hea, et ta kähku majakaks ja nii edasi. Noh, ütleme, et seal oli igasugused trikid, et, et ta nagu töötaks, aga, aga kuna see, see keelest ja kõik, see asi oli nii lihtne, siis oligi nagu super elegantne, super nagu lihtne ja ma arvan, et ega minu progemis aeg jäigi sinna kaheksaga.
Keskel pärast ma olen võib-olla natuke mingist mingisugust HTML ja võib-olla see SS-i nokkinud, aga tegelikult ega ma sellesse Võru oleks nagu üle kohe, kui tal läks nagu, nagu läksid nagu keerulisemaks. Aga õnneks tuli tasemele igasugused graafikapaketid ja, ja muudes, aga millal see oli juba keskkooli ajal? No seal jah, sealt ütleme olles enda jaoks kaardistanud ära kõik kohad, kus sai midagi arvutitega näppida, jõudsin ma läbi. Väga tähtis on kusjuures see, et tippis oli ka üks klass, kus teeme siis. Aga seal olid juba igal masinal oli Drive tava. See oli ka juba super, et vahest ja, ja seal oli ka muidugi seal guru Guru staatus oli nagu juba järgmisele levelile, seal olid mingid laborandid.
Ühe nimi neli koma ekselgiga. Ja ei tule enam meelde. Aga aare, tali tuleb kuidagi ette ja ma ei ole kindel, kas see on õige nimi õige näo ees. Aga igal juhul olid seal mingisugust juba üliõpilased on juba nagu kõvemad vennad või isegi oma mingisugune maine postkäädavat või mis iganes ja, ja, ja loomulikult see nabade ja jada seal ukse taga neid selgelt tüütas ja siis nad seal tegid omaette mingisuguseid reegleid, olid suured jumalused ja näiteks mingi hetk oli, kui neil juba olid täiesti noh, infoga leviks, eks ole, et järjest rohkem Kunder äkki see on ju sinna värava taha ja ja siis, kui oli vaja reglementeerida, et keskmise, seda saab ligi, siis nad võtsid ühe kõige popima mängu, mis seal parasjagu oli. Kinks, väli, trükise välja kogu selle soos kooli. See oli nagu täkkov seda perforeeritud paberit ilusti nadid. Ja siis nad otseselt meie naha tavaliselt punase pastakaga ringe ja progesid selle mänguringis selguvad täiest seal juba minu jaoks oli see juba nagu Jumal, see tase. Ja, ja nad progress niimoodi ringi, et nad said vaikselt iseehitatud Sostikudega juhtida neid selle mängukolle. Ja, ja siis põhimõtteliselt oli umbes. Reegel oli see, et kui sa said nende jumalate vastu ühe leveli läbi siis sa said selle ühe päeva käia. Ja teha seda jah, jällegi see iseenesest mängufaktor ja see kõik oli põnev, oli see, et nad tõesti nagu nad võtsid selle mängu, mis minu jaoks tundus nagu superkeeruline nagu ja, ja lihtsalt nad kirjutasid, tuleb aina ringiga, et kirjutasin, mitte lihtsalt ei, ei teinud seal mingile tegelasele mütsi pähe, on ju vaid nad lihtsalt nagu tegidki, kõik ringid, käitumine muutus ja tänapäeval muidugi tagasi vaadates tundub, et see tegelikult
Väga lihtne.
Aga noh, see on seesama see kelleltki, kes ja mis praegu nagu üles tõuseb ja mille peale kõik asjad ehitatakse ja ja pluss see töötas veel ka ju sellel tasemel. Et see on just nimelt tõust testis näed veel nagu nende petnokkadel rivisi, mis veel nagu kõrgemale apteegi veel ihaldusväärsem sinna sissesaamiseks absoluutselt. Aga igal juhul lõppkokkuvõttes lõpetasin kuskil Kullos, kus oli ka üks klass, kus olid vist juba natukene kõvemad emmessisid. Seal oli juba nagu mingi graafika reziim ja, ja igasugused muud asjad, ehkki, ehkki ütleme, et kui seal midagi kahtlast, mis kiiresti liigutaks või riike kasutama seda teistega. Ja, ja seal oli selline legend nagu ränimeister, kes seda seal nagu majandas kes on üks tore toona, ta oli selgelt sihuke tore punkar, kes oli tulnud kuskilt Volgaga gaasianalüsaatorite tehasest temast ja seal vits poistega jahmerdada, aga ta vaikselt seal hakkas tegelema Komoderamiigadega mis oli juba sihukene, Superad väest. Raud. Ja, ja, ja kuidagi tali seda kõike ei seksivideo tootmise ja, ja sellise asja kuidagi tähe all. Ja tänu sellele oli ka loomulikult siis on välja arvestanud, et kus, kus niisugune asi veel toimub, on ju, ja Eesti Televisioon oli selgelt üks ja siis oli mingisugust vene metalliärikat. Kuidagi niisugune turundusharu oli tekkinud Eestisse ilmselt keegi vend oli piisavalt palju lobi teinud ja tal ei ole muud teha. Siis ta oli kuskil Kristiines kuskil keldris oli püsti pannud väikse mingisuguse nii-öelda reklaamistuudio kus ta siis tootis märki ja tal oli seal ka, eks saviga. See pidi olema siis üheksakümnendate algus, eks ole, jah, kuskil sealkandis. Aga ühesõnaga kogu sa selle Kullos nikerdamine sama jaoks või ka mingeid mänge tegema ju võimalus seksida ja Marcus klass Mandel toimetas ja, ja, ja. Raul Keller, kelle alkoholi killer? Ja, ja seal me midagi seal noppisime, isegi ta üritas seal mingisuguseid emmessiksi mängimist nagu pubitsee midagi, aga, aga see tundus kuidagi ikka väga niisugune kahtlane ja, ja naiivne tegevus mulle vähemalt toona, aga siis juba räni kuidagi nähes minus potentsiaali meelitas mu Eesti Televisiooni ja siis olin põhimõtteliselt ma olen isegi veel kestva viimases klassis, ma töötasin juba Aktuaalses Kaameras ja uudistetoimetuse kõrval oli siuke väike Kubrick. Kus me siis tegime Aktuaalse Kaamera infonurki, mis selle diktori taga oli nagu seina peal ja kuna abiga oli selline tore masinat sinna see lasta videosignaali sisse, sealt ise sinna välja, et sa said ega teha nagu tiimixi vist juba toona, eksole, kus on see ju, see oli ju jõhkralt kallis riistvaraga piisiide peal, et enam Keilasse see oligi, et Abigail oli, miks, miks need Amingat siinkandis selles vallas levisid, oli just see, et noh, PC jaoks mingisuguse tasemega videokaart seal ikka mingi, kas see oli mingi hollivuudi või, või sellise tasemega asjani ja, ja need masinad olid ka neil oli, eks ole, mingisugune Seega ja, ja neli värvi on ja samas kui amiga oli nagu Full videos on, põhimõtteliselt võisid talle panna selle võrra isegi arvuti monitori see võistkonna teleka taha ja see toimis ja see oligi ilmselt see point, miks tal oli videosignaal, et oli nagu seks, sest nagu kodukodutarbimisest arenenud selliseks nagu ja, ja seal tegime oma ilmakaarte ja panime need videopilte sinna ja põhilise osa ajast muidugi mängisime arvutimänge, sest jälle Amigo super šefimängudeni. Millega te tegite selleks teiega teie nullist ja kirjutanud kogu seda ka. Öelge, kas seal oli olemas täitsa viisakad graafikapaketid, De Lux Pent on nagu šefimaid graafikas ohte, mis oli nagu igasugustest asjadest, igasugustest Photo soppidest ja kurat teab millest ikka. Kümme aastat enne tuli. Ja, ja me siis ainult noh, me olime nagu selliselt nagu India nõu abiga vennad ja me vaatasime ikka kõikide pisi muude mängude peale ikka ülevalt alla, sest et nad ikka ei teadnud, milles nad seal Sarkisideni paraku lihtsalt saviga pisesena oli, oli kehva ja ta lõpuks jooksis, läks nurja ja aga iseenesest see tehnika oli nagu superäge. Ja seal meil tekkis mingi väikese kohaga selline punt tegelastest, kellel oli, kas oli kodusaami ja/või kuidagi tegelesid siis noh, Martin Rinne, kes täna teeb, direktor on ju tema juba tulid tekkis sealt kuidagi sinna telesse ja siis Margus kliimas Marx samamoodi oli seal, tema tegeles Eesti videos Siilatsi kuidagi sellele, noh, kõigil oli nagu mingisugune Äkses ja, ja siis jällegi loomulikult seal ka võlus mind pigem see, et et sa midagi seal nikerdasid ja sa tekitasid mingisuguse elava pildi sees selle pidanud olema minikaamera ja näitlejad ja mingi asi, sa võisid teha sõna, mingeid väikseid, mingid animatsioone või animatsioonide vedas välja. Noh, selles mõttes, et sa said seda teha põhimõtteliselt stop mausseniga. Noh, ütleme, et seda niisugust animatsiooni reaalajas
Bussiga Nõgisema juba ikkagi reaalajas täie, mis aga ei, me tegelikult ikka tegime tele, tegime reaalajas ka, sest keegi viitsinud stock mossega vastane, aga põhimõtteliselt sai seda teha ka Stognoosiniga, aga minu meelest isegi enamus sellised asjad käisid ikkagi reaalajas, et tal olid juba sellest eluks peenraid sisse ehitatud igasugused nutikad asjad, näiteks nagu liikumise aeglustamine või kiirendamine andsid talle põhjust ette, et siin on sul mingisugune see kast see kastab liikuma mingisuguse viiekümne kaadriga siia, siis ta automaatselt täitis need viiskümmend kaadrit ära. Ja vajadusel, kui sa ütlesid, et siin, siis ta tõmbas lõpus hoo maha ja kõik oli väga-väga fain näiteks. Mativeermet sellist naist, kes, kes pärast sellist naist Tallinna linna mingisugune disainer või, ja, ja ma mäletan, kui ma olles läksin sinna teles ja selle järgi võib muidugi mingi aasta paika panna ja tegime öölaulupeole, tegime mingisuguseid valgus, Kippe, et, et see oli juba ikkagi jällegi superväest, mäletan, et mingisugune nagu asi oli nagu televisioon, Jajah, seal seal olid väga ägedad ja, ja see oli ka nagu kogu teletegelikult siukseid asju tegime, sest et alternatiiv oli tiitri masin, eks ole, mis oli mingi räme puit ja mis oli nagu ikka eriti.
Oleme jälginud.
Sihukene pool analoog ja siis meil oli niisugune super, et väest animatsiooni ja värviline ja seal sain teha mida iganes ja siis me tegime seal vaid vaest Tegime mingit haltuurat mingite reklaamide jaoks ja igast lollusi rist puhas ning iseõppimise värk või kuskilt hakkas tulema, mingit informatsiooni ka juba ei, see oli ikka puhas iseõppimise teema, et selles mõttes, et oli noh need, need vahendid olid suhteliselt piiratud ja ega seal midagi nagu väga keerulistel kunagi mingi hiljem tulid ka, mida sellise koldepaketid sellega sai seal, kus sa rõhutad, meil olid noh, jällegi see tasemete vahe, et on ikka hoopis teine, et seal sa pidid ikkagi mingi punkt ajal konstrueerima mingisuguseid kolm teeb hindu ja siis neid seal kuidagi opereerima, et et tänapäeval vaatad, kuidas väänatakse mingit pump, Mäppinguid ja mingisuguseid asju pleierite kaupa ja siis see kõik kuidagi tuleb välja, et see on nagu täiesti müstika. Mis, mis sa siis tegid, kui sa Eesti Televisioonis enam ei olnud, sest ühel hetkel sa enam jootmist? Jah, seal oli kuidagi tundus, et see videograafika oli nagu väga põnev, aga tundus, et kuna siis oli, hakkas tekkima niisugune nagu Business. Et siis niisugused sõbrad, kes kuidagi olin rohkem sattunud trükigraafika peale, kes seal kujundas Eesti Ekspressi ja kes seal tegi midagi, et see tundus nagu kuidagi nagu rohkem piinas. Ja siis ma kuidagi sattusin, sain aru, et ahah, et videote abi, aga, aga selle sellele Business juurde peaks nagu pisside peale ennast kuidagi sebima ja siis sealsamas telemajas kuidagi tekkisid mingid potensiaalsed kliendid ja, ja ma pidin hakkama tootma mingisugust kujundust, mis on nagu trükikõlbulik mujalt kunagi näinud sellist programmi nagu korraldro on, mul oli see vaja nagu ära teha, siis ma istusin mingi öö läbi, tegin endale selgeks sõiduautot, frustreeriv, sest et ta oli täiesti teine maailm. Ja, ja tänapäeval on ikka see, et saab joonistada, siis see pilt on nagu ekraanil, mida see joonistada, aga siis oli niimoodi, et seal midagi konstrueerisid mustvalgelt mingisuguse vektormessi. Sa panid sinna mingid värvid peale ja siis vaatasin Bregioodist teisele joonistasin aeglasem sa oled. Siis sa läksid nagu uuesti selle kallal, kuidas me selle kohe gruppi, mis oli siis, kui mina üheksakümnendate keskpaigast mäletan korral troon, siis see, see rajakas kipun töötama. Ta tegi mingisuguseid asju nagu tennispalli katki ja lihtsalt, kui Sa salvestad selle valesti ja selles mõttes sa arvestasid ju noh, olles kasvanud, arvestada joast, harrastasid nad aeg-ajalt jooksid kokku ja aeg-ajalt nad aga võib-olla jälle seal mängis ka natuke see, et selles poisikesepõlves selles mõttekas õpitud arvutist ikkagi üleolek läbi ühe lihtsa fakti, et emmessiksil oli paremas nurgas oli port, mille sisse käis kas siis kettaseade või mingi mälukaart vits, pisikene, pisikene, üpriski suur sahtel. Ja nüüd selleks, et mitte seal midagi asja tuksi keerata, siis selle kvartalisse sisselükkamise hetkel seal sees oleks väike lüliti, mis tegi masinaga sätti. Ja loomulikult selle lapiti kiirelt ära, et selle asemel, et Poola väljas selleks, et teha mingi kord, kui sa oled midagi tuksi keerab, näiteks olid kirjutanud programmi, mis loopima ainesse, panid kohe nagu näpud sinna auku oli masinalis surnud, onju ehk siis see kontrollmasinale oli sellest, et selle ühesõnaga, et vaatasin, teadsid, et mingi valemiga saad jao palju tegijaid ülemasinast jah, et see, see teadmine on olnud ma ka siiani, et ma alati tean, et kui ma kuskilt heinast ikka lõpuks juhtme kätte saame nüüd siis on ta surnud on ju, võib ühendada, pelgab.
See on hea teadmine. Selle koha peal ma nüüd pean andma järgi kihule ja ja lõpuks küsin, ma arvan, need küsimused, mida ma väga tahan küsida, me jõuame Maiko ringi ja punkte, eks see, kuidas sa sinna jõudsid. See oligi selles mõttes, et kui ma olin juba selle prindiga alustanud, on ja ja, ja siis ma vahepeal kuidagi sattusin mingisse
Niisugusesse maailma, kus, kus nagu oligi nagu print, oli niisugune asi, millega ma tegelesin ja, ja kuna ma olin vargusega varem suhelnud seal televisioonis ja tema omakorda suhtles selliste tegelastega nagu lõvi, kes on muidugi kõige olulisem tegelane üldse, kes, kelle juurest ilmselt algab kogu Eesti arvuti pisest, kui Jaak Loondest algab kogu Eesti arvutiteadust, siis ma arvan, et lõvist algab kogu arvutipises, Alugete ise poleks pisest kunagi käinud. Aga seal Rainer Nõlvak ja kõik, kõik see nagu plekist kokku ja siis ma saan aru, et Rainer oli Margusele teinud ettepanek toimetada siis mingisugust ajakirja, mis siis nagu alguses toimetaja seal ja ta oli, ta oli noh, nagunii-öelda asutaja, toimetaja mis iganes alguses ja tema siis vits mul varrukast kinni ütles, et davai, et need on vaja teha seda ajakirja, onju mina muidugi, pigem oleks mänginud arvutimänge nagu ma olen harjunud teleselline ikkagi üheksakümmend protsenti meie tegevusest oli arvutimängude mängimine. Aga noh, seal oli ka täiesti. Ma sain omale väga korrektse neli, kaheksa, kuue, ma arvan, seal ja seal jooksis Ultima andev roll ja just asjad, et see oli täitsa tore. Ja siis.
Ma muidugi tegelesin sellise klassikalise nagu noh, toimetus tegevusega, mitte tuttava inimesena mõtlesin, et millest peaks alustama, peaks alustama ikkagi ajakirja esikaanest välja. Ja siis ma sellest korralisse, seda esikaant, sellel hiirega joonist sinu ma joonistasin minu arust praegu tagantjärgi mõeldes muidugi tuleb mõista seda, kui aga noh
Siis tõenäoliselt see nii ei olnud. Aga.
Aga siin oleks jah mingi tohutu aur, et õnneks järgmiste numbritega, kuna siin Brontoloogas ka kokku, et neid nii väga palju ei olnud ja nendega läks palju aega. Ja kuna see on nagu otseselt ka nagunii, kui äriline ettevõtmine vaid oligi sihuke nagu promo, siis keegi nagu väga ei survestanud seda ajaaja poolt ka nii et meil ei olnud nagu kohustust, mille tellija heidetele meil iga kuu ilmuma või vähemalt alguses algusest Olysikenena kuskil Võrus istus üks kuradi nohik siis miks ei ole tulnud, eks, et miks ei ole loobunud näeme, ei adunud, et.
Meil on selline efekt.
MP imper kindlasti oli individuaalne oma näitel võin kindlasti öelda see, et see, see punkt exe praegu foto tehtud niimoodi internetis on, et see on ka kindlasti nagu märk sellest, et ta on ikka päris oluline asi, millest ma ka küsib teise selles mõttes, et oli oluline igas plaanis, sest et tegelikult ta tõesti noh, jällegi, et olles selles asjas sees, siis noh, minu jaoks ei olnud nagu küsimus, et kas arvutid tulevad, muud maailma me siin ei mõelnud sellele, et nendega oli lihtsalt hea asja teha ja nad tõenäoliselt olid inimesed ikka täiesti rumalad, kes seda ei teinud, on ju midagi. Ja noh, kõrvalt vaadates ma isegi ei saanud aru, kuivõrd vähe tegelikult arvuteid kasutati toona, sest me istusime, eks ole, MicroLinki peakontoris seal, kes kauem mingisugune, see vilunge enne ju seal ka noh, telemajas igal pool, noh, mul oli Äkses arvutite oli päris hea. Aga ma mäletan, kas see, eks see esimeses numbris või lihtsalt ei marsi. Esimeses oli arhitekt Kalle Rõõmuse büroo.
Niisugune väikene tutvustus läbi selle, et nad hakkasid kasutama arvuteid projekteerimisel ja see oli see midagi täiesti siukest epohhiloovat ja, ja ma isegi toona ei saanud sellest aru, et kuivõrd imelik see on üldse, et keegi teen nagu paberil midagi ja seepärast tundus kuidagi ära noh, naisena väed hakkavad aga pärast noh, jällegi selle sellega on võib-olla nagu isegi tagantjärgi seda artiklit nagu lugedes üks kord, sest malakas siis sinnapaika ja panin pildid külge ja, ja noh, mind see võib olla vägagi huvitav, mis on kirjutanud, veeretanud näiteks ma ise kirjutasin. Aga, aga tegelikult tõesti, et et kui üks arhitekt käis, eks ole, Staseerimas kuskil Kanadas ja seal tegeleti just sellega, et osteti personaalarvutid ja see nagu jällegi muutis selle töö efektiivsust sellest, kui mingisugused arhitektid, konstruktorid päevad läbi joonistasid kaika peale midagi ja siis järsku maid Bach valitsusele arvutisse ja, ja kõik on nagu hästi, onju. See oli ka nagu väga-väga põnev mõte ja, ja tegelikult on huvitav vaadata seda teed, mis täna toimub, on see, et meil on see nii-öelda see pimm modelleerimine ja, ja, ja, ja, ja siis sa kuuled, mis on need nagu väljakutsed. Et tõesti, et, et et mu üks sõber tõotab start, tapmiseks, tegeleb pimm-mudelite konfliktide analüüsi, et, et kuidagi üritada aru saada, et näiteks ventilatsioonitoru ei tohi läbi akna minna näiteks siis ma vaatan, siis mõtlen, et issand jumal, millega need inimesed on tegelenud, et see noh, miks nad seda arvutit pole varem kasutusele võetud.
Kui raisatud aega, onju?
Seda saab lihtsasti teha programmiga jah, olen teinud jah, aga just see, et, et see, eks see, eks see tõesti ta nagu ta tõesti üritas tuua olu jutust jäi mulje, et see on ikka super, on väheste mingi häkkerite Räkani tegelikult ma arvan, et ikkagi inimestele andis pildi, et mis, mis tegelikult toimub. Noh, nagu üldse, et see arvuti seal nurgas ei ole nagu raamatupidaja kalkulaator ainult, või noh, mingeid muid asju ka teha. Aga kuidas sa selle kirjutamiseni jõudsid joonistamise juures? No ma ütlen, et seal oli vaja ju kontenti toota ja ega keegi toona ei olnud arvutiajakirjanik Ain ja, ja kuna mulle meeldis arvuteid, arvutimänge mängida ja ja ma arvan, et noh, kirjutamine on iseenesest tore tegevus. Siis kuidagi nagu kas sul kooli ajal juba oli see nagu kirjutab ise kirjandisoon oli kuidagi olema? Ei, ma olen võimeline kirjutama okeilt ja iganenud jah, mulle joonistada meeldib võib-olla rohkem, sest kirjutamine on sellin, raske asi, et sa pead nagu lause peaks läbi mõtlema ja siis sulle tundub, et nendele headele. Et, et on nagu liiga.
Kreetne.
Ja siis selleks, et teema jätkuks, mul ikkagi veel üks oluline küsimus.
Nüüd mõni aasta tagasi.
Tõnis Kahu seletas mulle pikalt, kui nad, minu arusaam sellest, mis asi on küberpunk on täitsa vale. Nii pikalt ja põhjalikult, ilmselt tuleb ära, härra Kahu selle koha pealt just uskuda, mis ta teab, millest ta räägib. Nüüd aga minu arusaam sellest, mis asi on küberpunk ühest väga konkreetsest exe artiklist, mille kirjutasid sina ja proto. Ma olen laid, oli seal ka kindlasti öelda, minu meelest olid DVD ka nimed olid seal all. Aga see jutt sellest, mis asi on küberpunk ja kuidas ta teab, mis asi on meie kaheksa. Ja tal on V8 mootoriga auto näiteks muuhulgas Eesti pik nimega. Neid asju, mida kübervähk tegid, räägib ka, kuidas see oli, kuidas te niisuguse sisu sisu suutsite provotseerida.
Raske.
Ta ainult äri, aga eks meil oli, eks meil oli mingi ettekujutus sellest, see on ju jälle, ega, ega küberpunk ei ole mingisugune geneetiline, mingisugune organism, mis on välja arenenud ja siis on, pärast on hea klassifitseerida, et vot see on pool on hüljes ja pool on mingisugune või veel paari ähvardamine või mis iganes. Ei noh, huvitav jah, et meeleolu selgelt olime. Jällegi ma eeldan, et me toona juba teadsime, et mis on, ma eeldan, et oli olemas juba Gibsoni nekromancer, onju, ja see oli kaheksakümnendate keskpõhjust ja see oli kui kõik nagu räägivad sellest Itšalker Kailist, mis oli muidugi ja see oli väga oluline teosena. Aga noh, minu jaoks Gibsoni pööning, kroom ja Nekromant selline ligi lasi ka ju täiesti välja, et see oli noh see oli aru saanud, tõsi, ma siiamaani loen seda regulaarselt üle. Ja härra Gibson kirjutas need raamatud kaheksakümnendate keskpaigast trükimasinasse paberi peale ja ja täpselt nii ongi aasta kaks tuhat üheksateist oled sa näinud uuemaid raamatuid ka lugenud? Mõnda need, need lähevad veel hirmuäratavalt nagu tõepäraseks. Ja aja ajahorisont tuleb lähemale, siis ma ei oska nüüd järgmine küsimus, tee pidi olema mingisugune nagu väikesed, ei mõtelnud selle kedagi nagu mõistet nagu välja kahekesi ei mõista seda saiti, see oligi nagu Gibsoni, kui öeldi, et kui tol hetkel juba mingisugused rahvusvahelised mingid Peebeeessid Internetis, kus te väikesed nagu ma arvan, et see oli kõik kuidagi klikkis tõenäoliselt kokku, et tegelikult noh, jällegi, et ma arvan, et ma ei oska prantslasest rääkida, aga noh, pleier on eraon, eks ole, eepiline nagu nurgakivi on ja siit Meier Olin ja see fotoroloog, kes joonistas ilusaid pilte, tsiklil Distopiline noh see kõik kujundas meil välja mingisugused Pästopilisem pildi tehnilisest maailmast, mis kõik on nagu külge ühendatav. Ja, ja, ja ma arvan, et see pättele ei saa liita, mis täitsa juhuslikult praegu hinna jõudis, onju. Et ma arvan, et vähesed inimesed Eestis teavad seda originaallugu. Ja mina olin selle toonane, mainin nagu totaalfänn. Ma käisin aeg-ajalt Helsingis akadeemilises kirja kaupas, siis sellega ma vaatasin, et kas uus osa on tulnud üheksa raamatut mul on kõik olemas ja see on kõik, on kuidas mingid metalltorude või jäävad su silmamuna sisse ja ajust on selle järgi ainult mingisugused džiibid ja natukene pudru, onju, ja noh, selles mõttes, et see, see, see kuidagi koerad, me elasime selle asja sees, ma arvan ja ja ka mingisugune jälle mingis mängumaailmas tõenäoliselt olid paar mängu jälle, mis, mis kuidagi sinna kontriculteerisid, pluss on see, et me muidugi üritasime ka siis teha mänge enne veel, kui me pruumooniga tegime midagi, nagu mängu ideed see sees kuidagi paralleelselt selles amiga maailmas me üritasime ka midagi teha, vaid ta ja Juhan Soonets
Me tegime Rockexi-nimelise mängu, mis pärast ka ploomul keeras pissi peale palju ägedama, võib-olla võib-olla vähem kena ja, ja siis selle intro, ma mäletan, oli väga selgelt kantud kõigest sellest vee kaheksatest ja, ja rakettidest jama ja kõikides väga tähtis oli kindlasti see, et päikseprillid olid õige kujuga ja siiamaani ja, ja muidugi Andrus Aaslaid veel kõrvale rääkis lugusid, kuidas või, või kirjutas või, või teleri, kuidas siis mingi prinki valgusega saab su aju ümber programmeerida. Ja noh, see, see, see kõik nagu absopeerusel tekitas mingisuguse omaette alternatiivse reaalsuse ja ma arvan, et see on see meie, meie arusaam küberpungist, aga enne tõu kohta küsida siis mind hakkas huvitama see, et kui sa räägid, et te kujutate, siis kuidas tootmine maailmas peast on. Ajust on ainult niisugune kodu ja natuke Tšiped. Ometi rohkem silpe, vähem rohkem sitta raamatut. Kõlab nagu väga hea teise maadel. Aga ometi see te nagu astusite pika sammuga toe tuleviku poole, ilma mingit kõhklust loomata, et see on nagu õige suund sinna, sinna tuleb minna. Seda nagu tagasi hoida on nagu, tõenäoliselt on mõttetu, sest et noh, Loviidid ka üritasid midagi aine, aga, aga noh, parem on olla seal enne teisi. Et sa paned juba õiget Tzipi taha ära ja võtad selle pudru osakaalu väiksemaks. See tahab ikkagi siukseid mõtteid mõelda, üheksakümnendatel see tahab ikkagi nagu visiooni saada. Plumbum, kuidas sa sattusid selle ahti ja jaanitule? Jällegi, et ma kuidagi, kuna ma olin kogu selle super häkkerid, noh, kes igalt ostult toona võib, igaüks mõtles, et on super häkker, see, kellel oli, kattis see mees seksi Vano All või ka see, kes oskas neid faile kokku panna, nii et ma olin üks väheseid tegelasi, kes joonistas pilte. Ja jällegi ma ma tegelikult oskan, oskan ka ilma arvutita päris hästi joonistada, et see on lihtsalt seal tundus see kuidagi nagu lahedam, et seda sai salvestada seal see on tuul ja see on nagu põhiline, et kui sa lihtsalt joonistad, siis saanud uut teha on väga raske. Isegi võiks öelda, et võimatu peaaegu ja kuidagi jällegi, et see seltskond ei olnud nagunii suur ja nad kõik nagu kollektisid kuidagi, eks ole, näiteks seesama nagu siin ka teised on rääkinud, et see sihuke välismaailma riitsimine, et, et see kõik oli nagu nii see sisemaa ja välismaale, et see kuidagi ikka klikist kokku, et et sellega ma teen korra kõrvalehüpe, kiire, tuli lihtsalt meelde, kuidas näited, kuidas Peeveeesside ja värkidega suheldi, et et meil oli sealsamas telemajas oli täpselt samasugune ambitsioon, meile lihtsalt oli see, et olid amiga, ta on ju ja mänge ei olnud, siis pidi neid mänge kuskilt jälle piirama on ju ikka. Ja ma loodan, et tagantjärgi mägi kahekümnendat pidamisasjatundjad ei hakka, ei hakka need peale lendama, aga igal juhul üks viis oligi see, et sa pidid noh, täpselt jõudma kuskile mingisugusesse Peeveeessi, onju ja, ja, ja, ja kuidagi sinna sisse pääsema ja ega see oli olnud niimoodi, et tasku sisse, et seal nagu ennegi on kuulnud satavastist mingid vennad, kes monitorist tegevust jälle Eesti tundus eksklusiivne, niisugune veider koht on ja see on sama hea kui eskimo naine. Ja, ja, ja mingeid, meil isegi tekkis mingisugune treeningpassid, et meil juba oli midagi, mida nagu vastu pakkuda, aga tavaliselt me ikka mängisime sellist vaest sugulast ja, ja siis oli ka, et me isegi noh, nagu bluuboksisime ennast sinna sisse ja noh Otto, tundub, et räägime sellest lähemalt, see tähendab siis seda, et sa pidid kuidagi toonase Telekomi keskjaamale midagi kõrva vilistama, mida too kuulda ei tahtnud tingimata. Kusjuures nüüd, kui ma hakkan mõtlema, et igasugused Margus oli, meil põhineb loobuksid spetsialist, aga kas me nagu reaalselt loobuksime sinna ka jõudsime, seda ma nüüd hea küsimus, sest ma mäletan, et punkteksidele küll manuaal, sellega ja, ja aga põhimõtteliselt samad asjad lugeda, see on ju iseenesest üsna lihtne, kuna need vidinad toona olid suhteliselt rumalad, on ju näiteks simkalle võimeta, kellelegi ma annaksin või kellega sa rääkisid, et et kui kiired olid modemine, mina mäletan, seda tõi kolmesaja Heyes. Ma mäletan ka seda, et minu meelest ma ei mäleta, kas oli mast või keegi oli väidetavasti suuteline händseigi ära vilistama sellele kes ta oli nagu piisavalt aeglane. Et kui sa nagu raagu suutsin talle suusõnaliselt selgeks teha, et et see on see legendaarne käpp, Francise vile, mida Ameerika moel jagati, mis kaks tuhat kuussada üldse välja vilistas täpselt ja, ja just-just-just, sest seda saab kasu kui teha, kui vaja. Ja aga jah, et see kõik oli nagu kuidagi jällegi, et võib-olla oli see, et igaühel meist oma fookus on ju, et kõik, kes oli, tahtis rohkem seal mingit Networki, äkki ta on ja kes tahtis rohkem lihtsalt häkkida häkkida, kes tahtis progeda? Jällegi mind huvitas, mida ütlesid selgelt, mind hoidsid mängud, liikuvad pildid, värvilised pildid, kuidas neid ise teha, kolm tee kõik niisugune värk, et ma nagu, nagu olin. Pigem nagu otsis neid võimalusi, eks see viis meid ka tegelikult kokku siis lõpuks pruumoni pundiga, kellele oli lihtsalt ta oli kindel soov, et nad tahavad teha mängu, onju. Ja kuna minu jaoks oli see lihtsalt natukene niisugune nagu nõme ülesanne, kuna neile oli, mul oleks amiga, seal oli kuradi Maidan miljonit värvid on ju, neil oli mingisugune VGA ekraan ei, alguses ühtegi C ja sinna pidi mingi nelja värviga midagi valmis hingeldama väga palju, et noh, teeme ära. Ja, ja, ja seal kõige naljakam on see, et see kosmonaut, mis sellest sündis. Jällegi see, kuidas sa turustasid, mis ma nagu lihtsalt vaatasin ja imestasin. Ja järelaeg mul nõrgurist meeles seekord nagu telgeitid, need vennad olid lihtsalt toona ja on vist siiamaani, eks. Et nad noh, olid nagu tõesti nagu Pakosta teeme seda asja, aga see, see graafiline pool oli, oligi nagu super lihtne, nokkisin valmis. Siis nad tegid oma selle mingi mingi musa, ehitage sinna marakesin ka mingisugused ikoonid, mingit kitarre ja mängi trummid ja väga vaheli. See tähendas seda, ma mäletan küll, et inimesed, kes oskasid arvutiga joonistada, et neid oli nagu vähe ja kas sul tekkis. Ühesõnaga, et kas sul kõigepealt tuli arvutiga joonistamine ei siis joonistamine või oli sul enne ka joonistada? Ei, ma ikka Ene Jaaniste, selles ma noh, jällegi kes ikka ennast kiidab, kui mitte ise, eks ole, ma nagu. No akadeemilist joonistamist ma valdan nagu suhteliselt väga-väga hästi okeid, et selles mõttes mul ei olnud nagu keeruline omandada enamus inimesi, vaatasin toona joonistatud tarvetitega ega mini, vaid oli seesama kuradi munaga hiir, mille, eks ole, muna aeg-ajalt jooksis mingit kuradit pahna täis, siis jälle küünega puhastama. Ja, aga noh, ma ikkagi endale võtsime kiire, mis nagu enam-vähem jooksis, on ju, et selles mõttes minu jaoks ei ole vahet, kas see on pliiats või, või, või Nov tablett või, või mad Evelin ja et see on nagu noh see arvuti oli sinu, see nii-öelda kunsti tegemise ja selle selle ande nii-öelda laiendus. Jah, ta oli lihtsalt ütleme nagu mingi teistsugune tehnika ja tunduvalt andeks andma kui näiteks mingi akvarell või mis iganes. Et selles mõttes oli.
Täna sa vaatad, eks ole, et et iga kõik kunstnikud kasutavad mingit Maidamegidzindiku tabletti või, või noh, neid noh, neil on kõik super ägedat toolid on ja ma siiamaani aeg-ajalt, kui mul on vaja omanikega midagi, kus täna maanikerd on tegelikult temale parim tants päri ja kõik vaatavad, et ma olen peast soe. See on nagu noh, mugav ja käe järgi tuua tegelikult, et kui seda raamandade keerata, keerad kursori viinud kiireks. Ühel hetkel sul tekkis see koht, kus tekkis mõte, et võiks hakata Weeki tegema. Ja see oli ka Pauluse mainib, et siin jah, see oli ka nagu pigem läbi selle, et kuna ma olin aru saanud, et ma ei ole piisavalt järjepidev ja, ja see sealt ütleme, see pragemise osa oli nagu kõik tundus toona liiga kui välja. Et noh, mul olid sõbrad, kes sellega tegelesid ja, ja miks see üldse nagu see tulemus ei olnud seksikas. Aga teeb järsku noh, ta jälle algusest oli ka mingi superpoorid on ju, et seal ei olnud nagu noh, mis seal oli see osaühing või mis esimene oli see sealt pikemasse? Adria seljaga siis kui ma sain aru, et kui sa said juba tabelite keelata ported maha ja sinna mingite üksikute ühe mikstiste tükkidega seal neid hakata mingit leiavadki tegema ja siis ma olin müünud mees, siis ma nagu hääletades ka siis ma tahtsin ühes reklaamibüroos ja midagi sealt katsetasin, nokkisin. Ja, ja siis mainitakse juba olemas. Ja, ja selle asutajad tulid siis otseselt andmed maja veel natuke reklaamimise, sest ka et paneme midagi mingit seljad kokku ja hakkame, hakkame seal nagu vaatama ja eks seal jällegi, et lihtsalt see, et ma sain mitte lihtsalt enam selle pildi oma käe seest ekraanile, vaid ma ise sain selle pildi nagu pauh kõigile nina ette, eks ole, paljudel ekraanidel jah, ja, ja, ja, ja toona oli noh poisikesed, mis ma tegelikult ikka nimega poisikesed olime, siis oli juba aasta oli siis kakskümmend viis või? Üheksakümnendate keskpaik, teine pool. Jah, jah, et üheksakümmend kuus, üheksakümmend seitse oli juba, et see siis oli juba, noh, Siis sa juba Business teha, siis. Panid jälle, tegid lõikest neid piksleid ja, ja mingisugused, ma ei tea, ma mäletan, meil oli klient, oli Reval Hotel Group, et noh, siuksed mingid nagad tulid ja võtsid siis kliendid ja tegid neile mingeid ägedaid asju. Sõber. Ja, ja see jätkuvalt oli see, et et sa said omale selle pildi panna inimestele silma, et eks ole seal sees ju liigutav faktor, ma olen isegi võib-olla see, et tegelikult mul ei oleks isegi vahet, on kas sa nagu inimestel silma ette vaid just nimelt see, et sa tegid mingisuguse, sul oli mingisugune distsipliin siis selle seal mingisugune HTML, onju. Ja, ja, ja, ja sa teadsid, kuidas optimeerida sa oled, seal oli mingisugune mingisugune tuulised ja saab alles seda suhteliselt hästi ja see tekitas sulle nagu rõõmu, et sa said sellega teha mingisuguseid asju, mida võib-olla teised nagu aga saab teha. Ja, ja see Jobs saatis, läksin värk, et tegelikult noh, ega ma kujutan ette, et muru niitmine on ka selles mõttes lahe, sa näed, kuidas on nagu Oru ja taga maha niidetud selle instan suhteliselt instanud prätifykeissimat. Võib ka nii mõelda, et, aga noh, samas kui sa muidugi ei oska, siis ei tule mingit hetke kehvade isenditega täisajaga ja see ongi, et sa tead, et sa oled sinna mingil määral panustanud, on ühe seal on nagu mingisugune Technical, et on ju, et sa saad. Et, et iga mats ei tule, ei tee seda, et sa saad nagu öelda, et Aak maisse, mis sa praegu teed. Praegu.
Ma olen kuidagi lihtsalt distantseerunud sellest disaineri rollist aga samas mitte, et jälle ajab ikka oma arusaamad, tegelikult selle pildi tegemine on, mõnes mõttes võiks nagu niisugune käsitöölise töö, et tegelikult need lõikelauad, mida minu lapsepõlves turul müüdi, kus olid need selle põletiga oli tehtud Nuubavaliibiale, onju ta nagu veits sarnane, et palju on Tšehhimaad tegelikult võtta ja aru saada mingitest äriprotsessidest või mingisugusest inimeste mõttemallidest ja disainida neist midagi ja, ja, ja jällegi, et see progemine minu jaoks on see, et kui see õigesti sõnastada, siis mingisugused vennad teevad selle valmis ja see, see muutub nagu päris, eks ole, et seal seal jällegi sellise protsessi toetav mingisugune asi seal masina sees mis toimetab täpselt nii nagu sa oled talle nagu öelnud, et toimetan ja et et seal olid need nüüd mitte ei ole, tule sinu käe seest sinna, see pilt vaid tuleb sinu pea seest mingi mõte, kuidas see kupatus võiks käia, Swing programmeerib, valavad selle valmis ja siis käibki niimoodi just just et mul oli lapsepõlves oli kuidagi ma mäletan selgelt, et mul oli mõnus mõte, et tehas on tore asi, sest et ta võtab mingisugused toorme ja see pannakse kokku mingite detailide, siis pannakse sellest kokku mingi asi. Ja jällegi, et me jõuame sedasama selle Nintendo Diva esijuurde, et, et füüsilisel kujul seda toota. Jõle tüütu, palju lihtsam oleks teha seesama asi, nii et oleks bittide jada, mis kõik liigrupeeruvad moodustavaid mustreid ja sellest nagu peaaegu nagu võluväel, eks ole, tekivad mingisugused asjad, mis inimestele tegelikult on tänaseks sama reaalselt tööriistad kui haamer ja, ja höövel, eks.
Nii on.
Aitäh ja sain küsida palju huvitavaid küsimusi, palju targemaks. No ma loodan ka, et ma nagu liiga ei läinud rändama, et see on juba kõik, on, kõik on väga hästi.
Aitäh sulle, aitäh Sulle.


\chapter{Tarmo Mamers}
\index[ppl]{Mamers, Tarmo}
\index[ppl]{MomraT|see{Mamers, Tarmo}}


Loodetavasti tekib siia jutt Oktroobrirajooni ÕTK kohta\label{content!OTK}, B'Knows viitab.



\chapter{Tarvi Martens}
\index[ppl]{Martens, Tarvi}

\question{Kuidas sina said arvutite ja arvutid sinu juurde?\sidenote{Kuna Tarviga rääkisime juttu mitmel 
korral, on jutulõng mõnevõrra hüplik. Katkemiskohad on tekstis markeeritud.}}


Ma olen pärit Pärnust ja seal arvuteid minu meelest tollal ei olnud, aga 
ma käisin olümpiaadidel, nii et matemaatika ei olnud minu jaoks 
mingi teema. Viiendas klassis
võitsin kuuenda klassi matemaatika linnaolümpiaadi, mille üle kõik olid suhteliselt 
jahmunud. Ühe riikliku olümpiaadi käigus viidi meid 
ekskursioonile Nõo Keskkooli\index{Koolid!Nõo Keskkool}, kus oli suur arvuti. See oli teistsugune maailm, aga kui mind sinna õppima 
taheti viia, siis ma ei tahtnud väga minna. Mul oli Pärnus oma bänd.

\question{Sul oli oma bänd?}

Jah. Tegime punki nagu ikka sel ajal. Käisin Pärnus muusikakallakuga koolis ja bänditegemine oli 
elementaarne. Kooliteater tegi ka oma esimesi samme. Kadunud Aare Laanemets\index[ppl]{Laanemets, Aare} ja Elmar 
Trink\index[ppl]{Trink, Elmar} tegid esimese kooliteatri, kus ka mina osalesin. 
Kõik see oli nii tore ja ma mõtlesin, et ei viitsi kuhugi kaugele 
kooli minna. Aga matemaatikaõpetaja käis mu vanemate juures, rääkis nad 
pehmeks ja nii see läks. 

\question{Kas sel ajal Nõo legend alles kujunes või oli see juba tuntud paik?}

Jah, oli kindlasti tuntud. Oli teisigi tugevaid koole, 
Tartus-Tallinnas, aga Nõo kool oli üle kõige. Põhiliselt 
sellepärast, et neile oli oma arvutuskeskus ehitatud, nii et sinna tuldi üle 
vabariigi kokku. Samas enamik olid ümberkaudsed maalapsed, kes ei olnud võibolla väga suured geeniused. 

Nõo Keskkoolis oli Nairi 3-1\index{Arvutid!Nairi!Nairi-3-1}, niisugune 
\emph{mainframe}, millele sai perfolinti sisse sööta ja laiprinterist 
tulemuse välja printida. Aga see ei tundunud väga huvitav. Umbes üheksanda klassi poisina 
avastasin Tartu Ülikooli Vanemuise õppehoone\index{Tartu 
Ülikool!Vanemuise tänava õppehoone} keldrikorruselt kabineti, kus oli 
kaks ja pool Apple II\index{Arvutid!Apple II}. Kaks ja pool sellepärast, 
et üks oli kogu aeg katki ja Andres Peiker\index[ppl]{Peiker, Andres}, kes oli 
selle keldri kunn, remontis seda.

Koolipoisina konkureerisin arvutiaja pärast tõeliste 
üliõpilastega nagu Tanel Tammet\index[ppl]{Tammet, Tanel}, Margus 
Liiv\index[ppl]{Liiv, Margus} ja teised. Sain ennast kuidagi 
vahele pista ja enamiku ajast ei käinud enam väga palju 
koolis, vaid olin rohkem Tartus.

\question{Ometigi oli Nõo kool mõeldud sinusuguste harimiseks süvendatult. Kas sul oli vaja veel rohkem süvitsi minna?}

Mis sa seal Nairi juures perfolindiga harid! Saatuse vingerpussina saabus 
aasta hiljem, kümnendas klassis Nõo kooli hunnik 
Agate\index{Arvutid!Agat}, mis olid Apple II kloonid, 
ainult värvilised. Kõige naljakam oli see, et kohalikud arvutiõpetajaid ei 
teadnud nendest midagi ja siis tuli välja, et on üks Tarvi, kes tunneb Agati
protsessorit läbi ja lõhki. Sel olid küll oma operatsioonisüsteem ja 
venekeelsed programmeerimiskeeled, aga sellest polnud midagi. Nii et ühel hetkel 
oli mul arvutuskeskuses oma kabinet ja arvuti. 

\question{Kas selleks piisas Tartus Apple II uurimisest? Kas said sahibide 
vahel noka piisavalt märjaks, et Nõos kunn olla?}

Täpselt nii, pärast õpetasin õpetajaid. 

\question{Kas Agat oli Apple II kloon kuni riistavara disaini ja arhitektuurini 
välja?}

Vähemalt protsessori mõttes oli see kindlasti sama. Ma ei ole väga suur riistvara 
asjatundja, kuigi assembleris\index{Keeled!Assembler} programmeerisin 
vabalt sel ajal. Küllap see oli üsna täpne kloon, aga 
värviline võrreldes Apple IIga. See tähendab, et pilt virvendas kogu aeg 
silme ees. 

\question{Kui sa omale kabineti said, kas siis oli uhke tunne?}

Mis seal ikka erilist oli. Hea oli see, et sain oma asja ajada ega pidanud enam Tartu vahet 
käima.

\question{Kas see õppimist ei hakanud segama?}

Ei hakanud. Mul ei ole sellega kunagi probleeme olnud. Tuleb 
lihtsalt kontrolltööd ja eksamid ära teha ja siis keegi ei õienda.

Nairi peal olid tõsiste inimeste keeled nagu Algol\index{Keeled!Algol}, aga 
lastele õpetati programmeerimiskeeli ROPS\index{Keeled!ROPS} ja 
KÕPS\index{Keeled!KÕPS}\sidenote{Vt ka märkust 3 lk
\pageref{sidenote:ROPS}.}, mis olid eestikeelsed. KÕPSis 
sai programmeerida joonistamist, näiteks kuidas plotter 
liigub: mine üles, mine alla, mine paremale; jäta joon, ära jäta. ROPS oli 
päris programmeerimiskeel. Ma tegin need keeled ka Agati peale ringi, et 
lapsed ei peaks Nairiga tegelema. 

\question{Matemaatika tuli sul lihtsalt, aga kuidas matemaatikahuvi läks üle nii suureks arvutihuviks, et käisid Nõost Tartus arvutis ja portisid programmeerimiskeeli? Mis 
sind selle puhul tõmbas?}

See on hea küsimus, aga mul ei ole head vastust. Arvuti oli selgelt täiesti 
teistmoodi, nagu praktiline matemaatika – rehkendusmasin, mis on kalkulaatorist intelligentsem. Mõtlesin vist
juba siis, et see on paratamatu tulevik ja teistmoodi ei saagi olla. 

\question{Huvitav, et sul on matemaatika ja arvutite seos algusest peale selge 
olnud. Mõnel tekib see seos palju hiljem kui üldse.}

Matemaatiline loogika on olnud kogu aeg üks minu lemmikdistsipliine, arvutid 
ja muusika on väga loogilised asjad. 

Ühel hetkel lõpetasin kooli ära ja läksin TPIsse\index{Tallinna 
Tehnikaülikool}.

\question{Miks sinna? Tartu Ülikool oli ju sulle juba tuttav.}

Mulle tundus, et TPI oli natukene praktilisema hoiakuga, ja aastal 
1987 räägiti Tartu Ülikooli informaatika kohta, 
et seal rohkem ikka joonistatakse tahvli peale. Ja päris matemaatikuks ma kindlasti 
ei tahtnud saada.

Tegelikult olin Tallinna vahet enne käinud. Seal oli Õpilaste 
Teaduslik Ühing\index{Õpilaste Teaduslik Ühing}, kus Peeter 
Lorents\index[ppl]{Lorents, Peeter} tegi matemaatikasektsiooni. Käisin 
Lorentsi juures aeg-ajalt, ta andis mulle kaelamurdvaid 
ülesandeid. Kahekordsete integraalidega 
elu oli huvitav, nii et TPIsse minek tundus loogiline.

\question{Mida sa õppima läksid?}

Automaatikateaduskonda ja eriala oli
LI\index{Tallinna Tehnikaülikool!Automaatikateaduskond!LI} ehk arvutid ja 
arvutitehnika. Seal juhtus kohe mitu asja. 

Kõigepealt ütlesin esimeses programmeerimistunnis, et siia tundi ma rohkem 
ei tule. Õppejõud ei solvunud, sest kirjutasin sissejuhatavas tunnis salaja
ühe programmi valmis ja näitasin seda talle.

Teiseks oli Teaduste Akadeemia Küberneetika Instituudi 
Erikonstrueerimisbüroo\index{EKTA} juhtimissüsteemide osakonnas\index{Teaduste Akadeemia 
Küberneetika Instituut|see{Küberneetika Instituut}}\index{Küberneetika 
Instituut!Juhtimissüsteemide osakond}\sidenote{Esineb ka nimekuju Arvutustehnika Erikonstrueerimisbüroo ja
Arvutustehnika Arendusbüroo, mis paistavad viitavat samale asutusele.} just
leiutatud kooliarvuti Juku\index{Arvutid!Juku}. Nad asusid sealsamas Küberneetika majas, kus olin juba käinud, ja 
septembri esimesel nädalal sadasin sinna sisse. Mul jäi 
õpilaste keskkondade pärast mure, et kui tuleb kooliarvuti, siis võiks olla ka 
õpilastele mõeldud programmeerimiskeeled, ja ROPSi\index{Keeled!ROPS} portimine 
Jukule oli tegemata. Rääkisin Juku tegijatele, et oleks vaja vastavasuunalist 
arendust. Nad lubasid mul enda juures hängida ja nelja kuu pärast 
olin tööle võetud. 

\question{Kas ülikool jäi kõrvale?}

Ei jäänud, käisin korralikult eksameid tegemas. 
Vahepeal, pärast esimest kursust, käisin Vene kroonus ka. Olin viimane 
lend, kes sai kroonusse minna, ja olen selle üle väga õnnelik. Meid viidi Leningradi lähistele, aga kuna
sain puhkpilliorkestrisse ja tegelikult tegin jälle bändi, siis polnud häda midagi. 
Jälle üks kogemus juures. 

Kroonust tulles paljud langevad ülikoolist välja, sest leiavad, et võiks 
midagi praktilist teha ja ennast targaks ajamine ei tasu ära. Mulgi 
oli teise kursuse poole peal kriis, kui mõtlesin, et mul on kohal 
käimata ja et kui eksameid ära ei tee, siis on kõik. Aga tegin 
eksamid ära ja võtsingi selle elustiili, et pühendasin ülikoolile umbes 
kolm nädalat poole aasta kohta. Imesin materjali sisse, tegin eksamid ära ja 
kõik töötas. 

\question{Minu puhul möödus keskkool mängides ja lauldes, sest 
kõik oli lihtne, kuid ülikooli minnes lõppes lihtsus ära. Kas sinul ei 
lõppenud?}

Lihtsus lõppes tõesti. Õigemini olid keerukad esimesed poolteist või kaks 
aastat, kui taoti pähe fundamentaalset kõrgemat füüsikat ja matemaatikat, mis lööb kaane pealt ära. Aga edasi läks erialasemaks 
ja inimlikumaks, õppimine ei olnud enam nii teoreetiliselt tappev. 

\question{Kas ülejäänud aja tegelesid Jukudega?}

Ei, kui kroonust tulin, oli kontorisse toodud juba esimene 286. Oli huvitav aeg, et käisin küll 
tööl, aga tööd oli vähe. Kui 
leidsid endale haltuuraotsi, oli suhtumine väga soosiv. Kõige suurema haltuuraotsa puhul, 
mida mäletan, tuldi koos arvutiga. Sain personaalse arvuti ja 
tööandja eraldas ka kabineti. 

\question{Kes need haltuurapakkujad olid? Kas oskad mõne näite tuua?}

Igasugused. Arvutiga tuli Soome laevaehitaja. 
Pean seda siiamaani kõige vingemaks programmiks, mille ma olen teinud. Ülesanne 
oli selline, et on kümne tekiga sõjalaev, mis vajab 
elektrivarustust; kuskil on jõuallikad ja kuskil tarbijad. Ja nüüd tuleb
hakata nende asjade vahele erineva jämedusega kaableid vedama. Kaablirennid 
on olemas, aga ühel hetkel saab kaablirenn täis. Mis me teeme? Veame 
teistpidi. Aga kes ütleb, et kaablikulu on sealjuures kõige optimaalsem? 

\question{Kas siis oli veel sügav Nõukogude aeg?}

Ei, siis oli juba sula ja hell aeg. See oli pärast kroonut, 1990 või 1991.

\question{Sel ajal ei tohtinud isegi mitte arvuteid 
Nõukogude Liitu tuua, aga sina arvutasid sõjalaevade kaableid.}

Kes seda ikka teadis. 

\question{Kuidas see haltuurapakkuja oskas sinu juurde tulla?}

See õppejõud, kellele esimeses tunnis ütlesin, et 
ma rohkem ei käi sinu juures, leidis mulle otsi. Inimesed 
teadsid mind ja oskasid soovitada. Just ülikooliajal sai väga 
eripalgelisi asju tehtud. Ma olin siis kõva programmeerija, kirjutasin muu hulgas 
oma andmebaasisüsteemi, mis oli FoxProst kordades kiirem. Vanasti oli 
kõvaketta poole pöördumine ränk tegevus, mis võttis 
aega, mitte nagu praegu SSD puhul. Ma kirjutasin andmebaasisüsteemi, millel olid 
fikseeritud pikkusega väljade asemel sujuva pikkusega väljad. See tähendab, et andmeid oli ketta peal täpselt nii palju, kui oli, mitte ei 
olnud eraldatud kindel hulk megabaite. Tõmbasin
keskmise andmebaasi umbes kaheksa korda kokku ja vastavalt sellele suurenes 
töötlemiskiirus.

\question{Kuidas sa kirjeid pakid ja mis saab siis, kui välja pikkus 
muutub? See ei ole ju lihtne.}

Miks see peaks lihtne olema? Mis see geniaalsele programmeerijale ja 
matemaatikule ära ei ole välja rehkendada? Nagu sõjalaevade kaalutud 
graaf, milline on kõige optimaalsem kaablikulu. 

\question{See tegevus läheb otsapidi teadusse, mujal maailmaski ei olnud
andmebaase teab mis palju. Kas sa teadlaseks ei tahtnud saada?}

Ei, mulle meeldis praktiline pool. Lõpuks läksin pika hambaga 
magistrantuuri ja virelesin seal umbes kuus aastat. Siis kui
ainepunktid hakkasid ära kustuma, tegin jõuga lõputöö. Mulle kuiv teooria ei paku 
eriti midagi, mulle meeldib maailma muuta. 

\question{Kas sa olid kuulus ka?}

Ei olnud. Eks ühe või teise tehtud töö tõttu renomee levis ja ka õppejõud
Peeter Lorents\index[ppl]{Lorents, Peeter} levitas sõna, nii et kõik käis
tutvuste ja sidemete kaudu. See ei olnud massiline, tegin umbes kümmekond projekti, aga need olid päris suured.

Tööasju tegin ka loomulikult, aga tööd oli toona vähe 
ja mentaliteet oli selline, et parem olgu inimene olemas ja valmis. Kui tööd 
tuleb, siis saab seda teha. Too kontor, mis on tänase nimega 
Ektaco\index{Ektaco}, oli fantastiline koht. Seal oli umbes 
viiskümmend inimest, tehti riistvara ja tarkvara, \emph{fifty-fifty}. 
Juku oli muidugi nende tehtud. Muu hulgas tegi Elleri-papi 
ehtekarbist valmis esimese hiire maailmas\sidenote[][-1cm]{Arvo 
Eller\index[ppl]{Eller, Arvo} oli Juku loomise eestvedaja (Ants Vill (2010). 
Meenutusi aegadest, kui arvuteid tehti veel käsitsi. Linnaleht (Tallinn), 
46). Kas tema loodud hiir just maailma esimene oli, aga ehtekarbi lugu kordab 
ka viidatud allikas.}.

Pooled inimesed olid \emph{cum laude} TPI lõpetanud, nii et sealne 
ajupotentsiaal oli nauditav. Näiteks kui ülemusel oli sünnipäev, siis
vennad mõtlesid, et teevad kingiks rääkiva papagoi. Tegidki. Seal oli 
briljantseid ja lahedaid tüüpe. 

\question{Mis see töö sisu seal ikkagi oli? Kas ise mõeldi projekte välja?}

Nii ja naa. Üks põhiline valdkond oli 
tööstuskontrollerid: ise mõtlesid välja, ise tegid, ise programmeerisid. Need olid 
\emph{rack}'i-suurused, täna saab samasuguse asja osta Hiinast 
kiibisuurusena. Kontroller koosneb analoogsisenditest ja 
-väljunditest, digitaalsisenditest ja -väljunditest ning nendevahelisest 
loogikast. 
Tollal oli vaene aeg ja Ektaco\index{Ektaco} tehti ühisettevõttena ühe Soome partneriga. Tänase 
päevani teevad nad kassasüsteeme Compucash, mida võib 
baarides aeg-ajalt siiamaani näha. Toona tuli soomlane ja ütles, et tehke mulle 
proovitöö – selline maatriksklaviatuur, et kui baarmen vajutab \enquote{õlu}, on 
kohe olemas. See tuli välja ja koostöö jätkus. Tollal ei olnud lihtne 
tellimusi leida, seetõttu suur osa
inimesi istuski pool aega jõude. 

\question{Ja sina muudkui programmeerisid?}

Mina muudkui programmeerisin. Ektacos\index{Ektaco} olin kokku viis aastat, enam-vähem kogu
ülikooliaja. Aastal 1992 läksin siiski tagasi
nii-öelda peamajja, Küberneetika Instituuti\index{Küberneetika 
Instituut}. Seal tekkis uus rakuke, mis esialgu alustas krüptograafia alusuuringuid. Seltskonnas 
olid mõned teadlase moodi ülikoolipoisid ka, näiteks Ahto Buldas. Ülo Jaaksoo\index[ppl]{Jaaksoo, Ülo} oli 
toonud välismaalt paksu raamatu krüptograafia aluste kohta ja seda me siis koos 
lugesime. Keegi luges peatüki läbi, proovis aru saada ja seletas 
teistele ka. Krüptograafia kui teadus Eestis puudus arusaadavatel põhjustel. Kui Eesti
iseseisvus, oli plats lage ja kuskilt pidi alustama.

\question{Kuidas mujal maailmas oli krüptoga? Mis tolleks hetkeks juba 
olemas oli?}

RSA oli olemas, aastast 1978. Ma täpselt ei tea, sest ei ole ennast 
kunagi krüptoloogiks pidanud. Minu eriala on rohkem nii-öelda 
rakenduskrüptograafia, mitte süvakrüptograafia.

\question{Miks sa sinna läksid? Sul oli Ektacos ju mõnus oma projekte teha.}

Pooled inimesed olid suurepärased insenerid, lõpetanud \emph{cum laude}, aga firmas ei saadud aru, et nende arenguga peaks tegelema. 
Oli väga selge seisukoht, et igaühe areng on tema enda asi. 
Interneti panek firmasse, ajakirjade ostmine või 
inimeste saatmine konverentsile ei tulnud kõne allagi. Pinge 
kogunes ja mingil hetkel, oma sünnipäeval, saatsin kohalikku võrku essee, mis firmas valesti on, mida tsiteeriti
pärast aastaid. Kümme aastat hiljem võeti see välja ja vaadati, et ikka on sama lugu. 

\question{Kuidas see kamp ülejäänud Eesti kogukonnaga kokku käis? Tol ajal pidas osa inimesi juba BBSe.}

Mu hea sõber ja kolleeg Heiki Kask\index[ppl]{Kask, Heiki} pidas ühte 
BBSi ja ma liitusin sellega. Sealtkaudu sattusin lõpuks fidonautide 
sekka ja hakkasin nendega läbi käima. 

\question{Kas see ei olnud sinu jaoks tähtis asi?}

Fidonet ei olnud minu jaoks tähtis, see oli lahe ja andis 
esialgse maigu suhu, aga nii kui tuli Internet, armusin sellesse.

\question{Mis interneti juures nii armastusväärset oli? Meile ja uudiseid 
sai Fidoneti kaudu ka.}

Meil oli esialgu UUCP ja modemiga helistamine mitu aastat, 1991–1993, kui ma 
ei eksi. Sai meili saata, mis oli väga tore, aga mulle jõudis kohale, et kuskil on 
olemas nii-öelda püsiühendusega internet ja suhelda saab reaalajas\sidenote[][-.8cm]{Mõiste \enquote{püsiühendus} oli tol ajal maagilise 
tähendusega: ei unistatud mitte kiirest, vaid pidevalt ühendatud 
internetist. Võimalus kaugete arvutitega vahetult suhelda tundus imeline.}. 
See oli minu jaoks nii võluv, et 
loomulikult tahtsin seda ühel või teisel moel uurida. Nii et UUCP 
aegadel mäletan ennast pühapäeviti kuskil modemi küljes rippumas ja RFCsid\sidenote{\emph{Request For Comments (RFC)} on juba alates 1969. aastast kasutusel olev standardne viis kõiksugu internetiga seotud standardite avaldamiseks ja kokku leppimiseks, RFCd on nummerdatud ja tuntudki oma numbrite järgi. Need sätestavad sõna tõsises mõttes kõike alates Interneti alusprotokollidest kuni tuvide abil side korraldamise (RFC 1149 — A Standard for the Transmission of IP Datagrams on Avian Carriers, D. Waitzman, 4/1/1990, 2 pp.) ja kohvi keetmiseni (RFC 2324 — Hyper Text Coffee Pot Control Protocol (HTCPCP/1.0), L. Masinter, 4/1/1998, 10 pp.).} 
alla laadimas, et need kõik algusest peale läbi lugeda.

\question{Kas see oli tol ajal võimalik?}

Oli küll. RFCde ülemine ots oli kuskil tuhande kandis alles, nii et see ei olnud 
probleem. Osad olid lühikesed ja osad mõttetud, ja oli ilmselge maniakaalsus 
koguda endale hästi palju materjali, et küll ükspäev loen.

\question{Kas seal uues üksuses oli internet sinu jaoks siis infoallikas?}

Eks jah. Sai meili kirjutada, lahe värk. Enne veebi olid 
põhilised FTP-saidid – ei pidanud mõtlema, mis \emph{node}'ist või kust 
mida saad. Mõnikord sai FTPst ka mõne mängu kätte, seal ikka liikus kraami. 
Seal sai ju samamoodi alla ja üles laadida, nagu Fidonetis. 

\question{Kas sa mängisid arvutimänge ka?}

Suur mängumees ma ei olnud, aga noorest peast midagi ikka õhtuti põristasin 
ja täristasin. See oli lõõgastumisviis, mitte huvi. 

\question{Sinu fookus oli matemaatikal.}

Programmeerimisel, mulle meeldis arvutit oma pilli järgi tantsima panna, mitte 
arvuti pilli järgi tantsida. Kui Windows\index{OS!Windows} tuli, 
siis ma kaotasin usu arvutitesse, sest ma ei suutnud enam igat 
bitti kontrollida. Kuni sinnamaani teadsin opsüsteemi, EEPROMi 
tasemel, mis sünnib, aga nii kui Windows tuli, siis kontroll kadus ja mul läks tuju ära.

\question{Kui tekkis Linux\index{OS!Linux}, kas siis tuli tuju tagasi?}

Linux aitas jah Windowsi aja üle elada, aga hulluks 
Linuxi kasutajaks ma ikkagi ei hakanud. Kui läksin 
Ektacost\index{Ektaco} Küberneetikasse\index{Küberneetika 
Instituut}, siis jätsin programmeerimise maha. Viimane asi, mille 
tegin, oli 1996. aastal mail.ee\index{mail.ee}. 

\question{Miks sa selle tegid?}

UNDP\sidenote{\emph{ÜRO Arenguprogramm}. Üheksakümnendatel läks Eesti veel üsna 
selgesti arengumaana kirja ja sai paljudest kanalitest igasugust abi. 
Tänaseks on humanitaarabi mõiste õnneks suuresti ununenud, kuid toona tuli 
seda kõikvõimalikul kujul päris palju ning oli tõesti abiks.} andis selle tegemiseks väikse grandi.
Kõigepealt tekkis hea mõte, et igal soovijal võiks olla meiliaadress. 

Pean alustama sellest, et 1994. aastal sai tehtud 
firma Teleport\index{Teleport} (mitte ajada segi selle sajandi 
Teleportiga!). Meid oli kaheksa tudengit, kellest kuus õppisid välismaal, sest 
neil oli raha. Eesti tudengitel raha ei olnud. Kaheksakesi panime rahad 
kokku, ostsime Soomest portsu modemeid ja tegime sissehelistamiskeskuse, kus 
sai ilma lepinguta 900-numbri\sidenote{Telefoninumbrid 
algusega 900, millele helistamisel kehtis eritariif. Tariifi 
jagati teenusepakkujaga ja see võimaldas tasulisi teenuseid osutada.} kaudu helistada. Saime 
tänu 900-teenusele kohe oma raha kätte. Kommertsiaalse interneti pakkumine oli sel 
ajal vaat et olematu ja laiadele massidele mõeldes täiesti 
puudulik. 

\question{Mis aastal Uninet\index{Uninet} meile tuli?}

Uninet oli juba olemas, aga selleks tuli leping sõlmida. 
EsData\index{EsData} oli ka olemas, me istusime tegelikult nende võrgu peal. 
Kuu hiljem tuli Microlink Online\index{Micolink Online} ja sõi meid massiga 
ära. Teleportist sai mõnesid partnereid kaasates 
Meediamaa\index{Meediamaa} ehk www.ee\index{www.ee}. See oli Eesti 
esimene veebiäri, kus proovisime inimestele rääkida, et kui sind pole 
internetis, pole sind olemas, ja et tulevikus pole sul oma kaubaauto peal muud vaja kui URLi. Nad vaatasid meid nagu idioote, aga nüüd ainult URLiga 
kaubaautosid näebki. 

\question{Miks teie kui programmeerijad firma tegite?}

Pigem olime ikka tudengid. Tarvi selgitas, et niisugust teenust turul ei ole, ja see 
tundus väga lahe, et inimesed saavad juurdepääsu internetile. 

\question{Kas see oli siis puhas missiooniüritus?}

Eks mõttes lootsime raha ka teenida, sest see tundus olematu bisness, 
kus on võimalik kanda kinnitada. Veebiga oli sama lugu. Samas oli see 
paljuski ka missiooni ja eestvedamise asi. Kirjutasin 1996. aastal internetist ka raamatu, mis oli esimene eestikeelne 
selleteemaline originaalteos\sidenote{Tarvi Martens, Vello Hanson. Internet. Ilo, 
1996.}. See oli interneti propageerimine. Samal ajal 
ehitasin riigile andmesidevõrkusid ja TCP/IP 
tehnoloogia laialdane levik tundus mulle sellel kümnendil väga tähtis.

\question{Miks?}

Saavutamaks seda olukorda, kus me täna oleme. 

\question{Kas sul oli peas olemas teadmine, et selline olukord peab ja hakkab olema ning see on hea?}

Ma teadsin, et see on hea. Ma ei teadnud, kui kiiresti ja kui massiliselt see levib, aga 
hüved olid ilmselged. 

\question{Kas su juttu keegi kuulas ka?}

Arvan, et jah. Me oleme näinud, et igasuguse uue tehnoloogia evitamine 
võtab palju aega. Siis on täitsa loomulik, et räägime kahekümne viie aasta tagusest ajast, mille järelmeid võib näha täna. Samamoodi
ei ole ID-kaardi ja e-hääletamise tulemused tulnud 
päeva, kuu või aastaga. Rääkisin kord 
ühele psühholoogile, mida ma teen, ja ta ütles: \enquote{Tarvi, sa oled 
hull. Need asjad, mida sa teed, on inimeste käitumise muutmine. Ühiskondliku 
käitumise muutumine võtab minimaalselt seitse kuni kaheksa aastat aega. Sa ei 
saa oma tibusid lugeda enne, kui jääd vanaks.}

\question{Vähe sellest, tagantjärele on too algne impulss sisuliselt tuvastamatu 
ja seega keegi aitäh ei ütle.}

Ma ei igatsegi seda, see on väga okei. Lihtsalt vaatan 
ringi ja naeratan. 

\question{Sa mainisid, et tegid riigile 
andmesideühendusi.}

Ojaa, see on üks tore lugu. Tegime sel ajal
riigiga palju koostööd standardite ja andmekogude 
vallas, näiteks disainisime Andmekaitse Inspektsioonile\index{Andmekaitse Inspektsioon}. 
Usun, et oli aasta 1993, kui Eesti toll\index{Tolliamet} ja piirivalve\index{Piirivalveamet} tulid Küberneetika Instituuti\index{Küberneetika Instituut} ja ütlesid, et 
oleks vaja piirivalve ja toll üles ehitada. Neil on 
ühised piiripunktid, kus pole mingit sidet, mõnikord isegi mitte 
telefonisidet, ja kas Küberneetika Instituut saaks aidata. 
Joonistasin projekti, klient tuli paari kuu pärast tagasi ja ütles, et mitte keegi ei 
suuda seda projekti ellu viia ja tehke see ise ära. 
Pidimegi hakkama paberimäärimisest tegudele üle minema. 
Koostöös Eesti Telefoniga\index{Eesti Telefon} 
said esimesed ühendused tehtud ja siis hakkas see tegevus mullina 
paisuma. Järgmisena tuli politsei ja riburada teised järel. Me 
tegutsesime Küberneetika Instituudi katuse all, mis oli väga hea ja
amorfne asutus: tahtsid, tegid teadust; tahtsid, tegid äri.

Raha hakkas liikuma, pidime ruutereid 
ostma (kulud jagasime tellijaga – näiteks ostsime piiripunkti ruuteri ja tegime piirivalvega kulud pooleks) ja seega oli vaja moodustada mingi juriidiline keha. Tegime midagi niisugust, mida 
ei tohtinud tegelikult seaduse järgi teha, põhimõtteliselt MTÜ riigiasutustest. 
See MTÜ oli Andmeside Osakond\index{ASO}\index{Andmeside 
Osakond|see{ASO}}, mida juhtis nõukogu, kus oli iga riigiasutuse esindaja.
Raamatupidamistoimkond nurises iga aasta, et sellist asja ei tohi teha, aga 
ülemused ja ministrid ütlesid, et ärme lõhu 
toimivat asja.

\question{See eeldas, et keegi riigi poolel kuulas sind ja 
mõtles kaasa. Kas need olid tippjuhid või IT-juhid?}

Kõigepealt kuulasid IT-juhid, kes rääkisid oma tippjuhtidele. 
Mäletan selgelt, kuidas 31. detsembril istusid toonase 
piirivalve\index{Piirivalveamet} ülema Kõutsi\index[ppl]{Kõuts, Tarmo} 
kabinetis kõik asjaosalised – politsei, piirivalve, toll ja Küberneetika 
Instituut – laua ümber ja kirjutasid lepingule alla. Kõuts veel ütles: \enquote{Ma saan aru, et meil on siin juhikandidaat ka laua taga.} Ma olin siis alles kahekümne viie aastane naga. 

Edasi läks väga huvitavaks, sest meil oli tegelikult olemas selline asutus nagu 
Valitsusside\index{Valitsusside}, kes tegeles erivõrkudega.

\question{Kas nad su peale kurjaks ei saanud?}

Teatav konflikt tekkis jah erinevatel põhjustel, sealhulgas 
koolkondade vastasseis – Jaaksood \emph{versus} Lippmaad. Vanemad inimesed 
teavad seda väga hästi.

Aga juhtus jah, et piirivalvel oli kagupiir täiesti lage, 
seal polnud mingit sidet. Ja selle asemel et minna 
Valitsussidesse, kes pidanuks seda tegema, tulid nad minu juurde ja ütlesid, 
et näed, Tarvi, siin on kümme miljonit\sidenote{Tegu on Eesti 
kroonidega. Arvestades valuutakursse ja  inflatsiooni, on tänases kontekstis 
tegu umbes 1,2 miljoni euroga. Arvutades protsenti riigieelarve (mis oli tänasega võrreldes väga pisike) 
kuludest, maksis too projekt tänases mõistes suurusjärgus 21 miljonit eurot.}, meil on seal lage 
plats, kaheksa piiripunkti on vaja ühendada, tee midagi. Ma ütlesin, et jaa, väga huvitav. Aasta oli 1994 või 1995.

\question{See oli tol ajal suur raha. Kõike oli ju vaja ehitada, kust tekkis 
idee see raha just sidele kulutada?}

Kui oled keset tühja platsi, kus ei ole 
mobiililevi ega mitte midagi, kuidas sa seda piiri pead? Jutt käib elementaarsest telefonisidest ja sõnumivahetusest, mitte suvalisest veebibrausimisest.

Minust sai projektijuht ja me ehitasime tühjale kohale 2,4gigase raadioside
kaheksa mastiga, taldrikud otsa.

\question{See ei ole raadioside disaini mõttes triviaalne ülesanne – kas
õppisid seda kuskilt raamatust?}

Mõtlesin kasutada 
kõrget sagedust ja seega pidi olema otsenähtavus. Aga kuidas seda 
kindlaks teha? Lõuna-Eesti maastik, mäed ja orud. Leidsin 
Maa-ametist\index{Maa-amet} ühe tuttava, kes oli hakanud 
Vene ohvitserikaarte (kõige täpsemaid, mis tollal oli) digiteerima ja oli 
selle kõige huvitavama osa ehk Võrumaa sisse saanud. Ta suutis mulle 
väljastada profiili: andsin talle otspunktid ja tema mulle arvujada. Kirjutasin ise programmi, keerasin maa kumeraks, panin mastid kasvama ja 
vaatasin, kas on otsenähtavus. Selle järgi sai mastide kõrguse 
rehkendada ning mida kõrgem mast, seda kallim oli. 
EMT\index{EMT} ei 
vaadanud mingit profiili, pani 80 meetrit igale poole. Mul aga oli 
52 ja 54 meetrit, mille puhul pidi 
lennutuled lisama ja jälle oli kallim. Sattusin ühe teadjamehe peale 
Eesti Telefonist\index{Eesti Telefon}, kes vaatas tehtut ja ütles: 
\enquote{Kuule, mees, kas sa tegid nädalaga sellise asja? Trassi projekteerimiseks 
läheb poolteist aastat, tuleb jala kõik läbi käia, puud ära kaardistada!} Aga 
mul olid juba mastid tellitud. Ta rääkis, et on olemas Fresneli tsoon – 
saatja ja vastuvõtja vahele ei teki mitte kiir, vaid vorsti moodi 
asi\sidenote{Fresneli tsoon on ellipsoidne tsoon, mida pidi raadiolained 
saatjast vastuvõtjani levivad. Tsooni võivad sattuda ja seega sidet segada 
ka otsenähtavusest väljapoole jäävad objektid.}. See võttis natuke jahedaks 
küll, kuid mastid olid tellitud ja side läks käima. Järgmisel aastal tegin Peipsi 
äärde sama viguri. 

\question{Ühesõnaga sa ei teadnud, et nii ei saa teha?}

Ei teadnud, mõtlesin inseneri mõistusega, kuidas see käib. 

\question{Miks sa üldse kulude optimeerimisega vaeva nägid, kui nii palju raha anti kätte?}

Vabariigi algusajal ei olnud raha palju. Igas vallas pidi olema optimaalne ja tegema parimat, mis teha 
annab. See ei olnud teab mis üleliia suur raha, kulus kõik ära. 

See oli väga tore aeg, kui sai tõesti käegakatsutavalt riigi arengut 
toetada, pealegi minu lemmiktehnoloogia ehk 
interneti osas. 

\question{Kui ma sind kuulan, siis sa olid programmeerija, kuni saabus internet 
ja leidsid, et tuleb hoopis sinna panustada, sest maailm läheb sellest 
paremaks.}

Jah. Programmeerida oskas sel ajal juba üha rohkem inimesi, ma ei olnud enam 
unikaalne ja kaua sa ikka programmeerid.

\question{Mõni programmeerib eluaeg.}

Arusaadav, aga kõrgemad ja üllamad mõtted tundusid 
järjest paremad. Võibolla see on ka isiksuse arenguga seotud. Ausalt öeldes, 
kui olin programmeerija, siis kartsin telefonihelinat, sest ma ei 
tahtnud inimestega suhelda. Ühel hetkel läks see üle. Linna peal teadsid kõik, et kui Martens tuleb jaurama, siis 
proovib kindlasti Küberneetikasse tööle meelitada. 

\question{Kas sa olid Küberneetika Instituudis\index{Küberneetika Instituut} juhtkonnas, et käisid teisi tööle meelitamas?}

Olin ASO\index{ASO} pealik, see sai üle antud 
Informaatikakeskusele\index{Informaatikakeskus}, mis oli RIA\index{Riigi Infosüsteemi Amet} eelkäija.\sidenote{Eesti Informaatikakeskus koos 
Riigihangete Keskusega liideti aastal 2003 Riigi Infosüsteemi Arenduskeskuseks, 
millest 2011. aastal sai Riigi Infosüsteemi Amet ehk RIA.}. 

Aastal 1997 toimus reformatsioon: instituudid kui eraldiseisvad institutsioonid 
kaotati ja pidid liituma ülikoolidega. Küberneetika Instituut jagunes kolmeks: kõige väiksem osa ehk 
andmesideosakond läks informaatikakeskusele, teisest osast sai aktsiaselts ja kolmas liikus
Tallinna Tehnikaülikooli alla. Kuna Küberneetika Instituudis oli 
praktilist tegevust hästi palju, siis kõigest praktilisest moodustati 
Küberneetika Aktsiaselts\index{Küberneetika AS}, mis on siiamaani alles. See 
asutati riigiettevõttena ja nüüd on vist erastatud. 

Küberneetika AS oli väga 
huvitav kombinatsioon. Oli osakond, kus programmeeriti Tolliameti\index{Tolliamet} 
infosüsteeme. Minu osakond oli keskendunud infoturbele nii teoorias, 
praktikas, konsultatsioonides kui ka analüüsides. Ja seal kõrval oli 
meremärgindus ja -navigatsioon ning valgusfooride tegemine. Lisaks
kinnisvarahaldus, aga seda enam pole. 
 
\question{Sinu jutu sisse sigineb tasapisi juhiroll. Mõned inimesed saavad selle maigu suhu ja siis ainult sellega 
tegelevadki. Kas sul ei olnud nii?}

Pidin jõuga maigu suhu saama, sest tegevust oli vaja laiendada ja 
töö tahtis tegemist. Inimesi oli vaja, neid tuli meelitada. 
Küberneetika ASi\index{Küberneetika AS} moodustamisel sai minust selle
arendusdirektor. 

Mõeldi küll, et vaatan laiemat asja ning tegelen ka meremärkide 
ja poidega, aga selle õnge ma ei läinud. Hakkasin arendama infoturbetooteid. 1996. aastal tegime esimese tulemüüri valmis, siis 
VPNi toote ja SSLi \emph{proxy}'sid. 

\question{Kas see oli pärast Meediamaad?}

Jah, see oli hiljem. Infoturbetoodete arendamine läks esialgu väga
hästi. Tegime Linuxi peale veebipõhise liidese 
jubinatele, millest osav insener saab ise tulemüüri teha. Tegime selle veebiliidese kaudu lihtsamaks ja oligi jämedas plaanis 
toode valmis. Eesmärk oli teha keskmisest viis korda odavam toode – keskmine 
tulemüür maksis tollal kolm tuhat dollarit. Ja tuli välja. 

Ilmselt siin oli seos, sest just 
riigiasutused ostsid meeleldi meie tehtud tooteid. \enquote{Tarvi 
tegi võrgud, nüüd müüb neile turva ka peale.}

\question{Enamasti tekib riikides soov teha omale privaatne turvaline
internet. Kas Eestis seda ei mõeldud või üritati teha ja ei tulnud välja?}

Loomulikult üritati, tegime 
VPNi toote, mis oli võrreldes praegustega unikaalne. Kui kast 
oli võrgul ees, siis ei saanud internetti, see lasi ainult teise omasuguse juurde. 
Näiteks igas maakonnas on kontorid, kus paned rohelise kasti võrgule ette ja kamba peale on üks tulemüür ka, näiteks Tallinnas, ja ainult läbi selle tulemüüri saab 
välja. Muidu on täielikult sisevõrk. 

\question{Sa kirjeldad ju X-teed. Arhitektuuri mõttes tundub 
väga sarnane.}

Ei ole, sellel pole andmete semantikaga mingit pistmist. 

\question{Kas see tähendab, et projektide vahel ei toimunud mingit risttolmlemist?}

Ei, see oli privaattorude ehitamine, X-tee on OSI tasemetes 
natuke kõrgemal.

\question{Kas sa tol ajal tegelesid interneti propageerimisega paralleelselt 
edasi või oli see lihtsalt üks faas?}

Siis oli turul juba piisavalt tegijaid ja ma ei tundnud vajadust 
sellega tegeleda. Pigem oli minu jaoks saabunud järgmine faas teha 
internet turvaliseks. Kolmas elementaarne faas 
oli osapooled internetis identifitseerida, et saaks ka
\emph{business}'it teha. 

\question{Kust tuli mõte, et internet peab turvaline olema?}

Hakkasime teoreetiliselt turvalisusega tegelema juba 
1992. aastal. Kontseptsioon, kuidas ja miks seda 
teha, oli mulle tuttav. Meie roheliste kastide puhul oligi 
eesmärk puhas ja turvaline andmeside, muud midagi. Minu sõnum oli see, 
et ärme teeme eraldi X.25 võrku, sest üle avaliku interneti toimetades on palju 
kuluefektiivsem.

\question{Kuidas sul ikkagi tekkis mõte, et interneti turvalisus on 
probleem, mida tuleb hakata lahendama? Kas keegi luuras või häkkerid kiusasid? Kust 
probleem tekkis?}

Probleem on olnud aegade algusest. Ja olles infoturbega algusest 
peale tegelenud, oli selge, et võrkudes on infoturve teemaks. See on 
elementaarne. 

\question{Kui mina oma ajaloo peale mõtlen, siis minu jaoks ei olnud. Ehitasin pikalt oma asju ja võrke, üldse mõtlemata, et need võiksid ka turvalised 
olla.}

Infoturve oli minu eriala, ükskõik mis 
ametis, ja see sai alguse tolle 
ühe raamatu kooslugemisest.

\question{Lisaks on sul matemaatiku, programmeerija ja antenniehitaja 
taust, nii et saad päris süvitsi minna.}

Jah, ma olen kirjutanud Jukule\index{Arvutid!Juku} püsimälu. 
Minu töö puudutas tähtede joonistamist ekraanile, EEPROMi tasemel 
sai ESC-käskudega aknaid teha. 

\bigskip
\noindent\rule{.3\textwidth}{.7pt}
\bigskip

Mõtlesin, mis lugusid veel võiks rääkida, ja mõned tulid meelde.

Ma ei olnud Tallinna poiss ja Jaak Loondet\index[ppl]{Loonde, Jaak}, keda 
mitmed varasemad rääkijad on maininud, ei tundnud. Küll aga kuulsin temast 
Fidoneti inimestelt. 

Juhtus niisugune lugu, et varajastel üheksakümnendatel, kui 
Eestis ei olnud isegi piisavalt leiba, oli talongide peal\sidenote{1980ndate
lõpust kuni umbes 1993. aastani, kui vaba turg hakkas enam-vähem 
toimima, müüdi elementaarseid toidu- ja tööstuskaupu, sealhulgas 
periooditi leiba, üksnes talongide esitamisel.}, otsustas 
Soome Rotary klubi Eesti koolidele natuke arvuteid kinkida. Ilmselt oli PC-aeg peale tulnud ja ühel tehaseinimesel jäi komptuureid üle. 
Need olid kummalised masinad, aga lahe oli see, et need olid võrgus ja emaarvuti ka. Soome Rotary tegi haridusministeeriumile
ettepaneku kinkida need Eesti koolidele. 
Minu mentor Peeter Lorents\index[ppl]{Lorents, Peeter} oli sel ajal 
ministeeriumis mingi tegelinski ja sattus selle peale. 
Läksimegi kolmekesi – autojuht, Peeter ja mina eksperdina – 
kohapeale vaatama, mis arvutid need on ja kuidas töötavad. Tõime need Eestisse ja siis tekkis küsimus, mida me 
nendega peale hakkame. 

\question{Kui palju neid masinaid oli?}

Kuus-seitse tükki, terve klassitäis. Eesti peale ei olnud palju, aga 
Rotary klubi sai endale linnukese kirja: Eestit aidatud, heategevus tehtud. Ja 
siis meenuski mulle Jaak Loonde\index[ppl]{Loonde, Jaak}. Sain temaga kokku ja Jaak 
oli kohe nõus sellega tegelema, silmad peas põlemas nagu ikka. Mõne aasta pärast saime kokku 
ja küsisin, kas masinatel pruukimist ka oli, ja tuli välja, et need olid väga 
hästi vastu võetud ja nendega igasuguseid vigureid tehtud. 

\question{Nii et Jaak toimetas edasi ka pärast seda, kui 
enamik temast rääkinuid olid koolipoisieast välja kasvanud?}

Jaa, ta oli legendaarne, toimetas arvutitega elu lõpuni. Tema põhiline soov oli, et lapsed saaksid näpud arvuti külge.


\bigskip
\noindent\rule{.3\textwidth}{.7pt}
\bigskip

1993. aastal tegin ma esimese 
jututoa, mille nimi oli Anna\index{Jutukad!Anna}. See oli umbes samasugune asi nagu praegu Messenger: hulk inimesi logib sisse ja hakkab omavahel suhtlema. 

\question{Kas see käis sinu enda tehtud tarkvara peal või said selle kuskilt?}

Sain kuskilt tarkvara ja tõlkisin käsud eesti keelde, käsk algas 
punktiga. Olin tollal Göteborgis neli kuud asumisel ja mul polnud 
seal suurt midagi teha, nii et putitasingi seda jututuba. 

Anna jututoas kaitsti isegi üks Tallinna 
Tehnikaülikooli\index{Tallinna Tehnikaülikool} diplomitöö ära – kaitsja asus 
Uus-Meremaal, õppejõud kogunesid jututuppa.

\question{Mis oli jutukate fenomen? Seal käis igasugust rahvast, mitte ainult tehnikud.}

See oli \emph{community building}, umbes samasugune grupp nagu Fidonet. Edasi 
tekkisid OK \index{Jutukad!OK} ja 
Cafe\index{Jutukad!Cafe}\sidenote{Cafe pärisnimi oli \emph{The Roadkill 
Cafe} ja see asus aadressil \texttt{ns.uninet.ee:5555}. Selle pani 23. 
veebruaril 1996 NUTSi (\emph{Neil's Unix Talk Server}) versiooni 2.3 
lähtekoodist püsti Indrek Siitan\index[ppl]{Siitan, Indrek}.} jutukad. Meil oli 
isegi Anna kasutajate kokkutulek Viljandi lähistel, mida 
Jüri Ruut\index[ppl]{Ruut, Jüri} veab siiamaani, nüüd küll ee.kevade nime all.

Jutukates käis suvaline rahvas, seal ei olnud õnneks üksnes tehnofriigid, vaid ka tütarlapsi. 

\question{See pidi siis olema väga vajalik teenus, sest 
mittetehnofriigile pidi see tehnika olema paras barjäär.}

See oli tegelikult lihtne, kui ainult terminalile ligi said. Panid 
\verb|telnet anna.ioc.ee|\index{Masinad!anna.ioc.ee} ja läks. 

\question{Kas sa hoidsid jutukat Küberneetika Instituudis\index{Küberneetika 
Instituut}?}

Pean tunnistama, et jah. Alustasime Küberis Unixi pruukimist aastal 
1992, kui tõime Soomest flopidega Linuxi\index{OS!Linux}. Teistmoodi ei 
saanud seda kätte. 

\question{Kas otse Linuse käest?}

Enam-vähem. Proovisin tollal Unixi kultuuri aretada. Kord ostsime hirmsa
raha eest ühe Suni. Kui küsiti, mis sellele nimeks panna, siis ütlesin suvaliselt 
\enquote{keeks} ja tekkiski igavesti kuulus FTP-server keeks.ioc.ee\index{Masinad!keeks.ioc.ee}. Pärast pidin \enquote{keeksi} lahti mõtestama ja 
arvasin, et see on Küberi Esimene Eestimeelsete Kasutajate Server.

\question{Tuleme korraks jutukate juurde tagasi. Selleks et sotsiaalvõrk 
lendu läheks, peaks olema algne seltskond. Kes need inimesed olid ja 
kuidas sa selle võrgustiku tekitasid?}

Ma täpselt ei mäleta, aga küllap rääkisin sõpradele, nemad oma 
sõpradele ja nii see vaikselt levis. Ühtegi erilist 
aktsiooni ei mäleta, piisas sõprade ringist, aga lõpuks läks 
ring väga laiaks – üle poole või rohkemgi olid 
täiesti tundmatud inimesed. 

Annaga\index{Jutukad!Anna} juhtus nii, et ühel hetkel vaatasin, et 
teised jutukad hakkavad ka tekkima, ning panin selle pidulikult kinni. 
Anna matused olid eraldi sündmus. Asja peab ära lõpetama, mitte laskma 
sel lihtsalt hääbuda. 

Kui Unixi juurde tagasi tulla, siis oli meil 
Eesti Unixi Pruukijate Selts ehk EUPS\sidenote{Selts asutati 1994. aastal ja sellel oli 62 
asutajaliiget. Asutavasse toimkonda kuulusid lisaks Tarvile Andres 
Bauman\index[ppl]{Bauman, Andres}, Margus Liiv\index[ppl]{Liiv, Margus}, Jaanus 
Pöial\index[ppl]{Pöial, Jaanus} ja Anto Veldre\index[ppl]{Veldre, Anto}.}. Teised tahtsid panna \enquote{Kasutajate Selts}, aga EUKS kõlab 
halvasti ja mina ütlesin, et peab ikka pruukima. Meil oli 
Tõraveres isegi kokkutulek.

\question{Miks te Soomest Linuxi\index{OS!Linux} tõite? Kas te ei tahtnud Sunile 
raha anda?}

Ühelt poolt ei tahtnud raha anda ja teiselt poolt oli see uus värske 
tuul, mis oli vaja ära proovida. Linuxi eelis oli see, et see käis 
PC peal. 

\question{Linux on praeguseni hädas oma kõrge sisenemisbarjääriga, inimestel on 
raske sellega liikuma saada. Kuidas toona oli?}

Me rääkisime Linuxist serveri kontekstis, tööjaama-Linux ei olnud teema. 
Tol hetkel pidi raha eest ostma mingi tarkvara, et failiserverit ringi 
ajada. Ma ütlesin, et ärme tee seda! Panen Linuxi püsti, kasutame 
seda. 

\question{Tol ajal taheti igasuguste asjade eest, nagu 
veebiserver, raha saada ja kommertstarkvara oli väga kallis.}

See oli ropult kallis, kuna kirjutajaid oli vähe ja see oli eksklusiivne asi. Kui hakkasime Küberis\index{Küberneetika 
Instituut} 1996. aastal tegema esimesi tulemüüre nimega 
Barrikaad\index{Barrikaad}, siis tol hetkel maksis keskmine tulemüür maailmas 
kolm tuhat dollarit. See on ju absurdne. Me võtsime Linuxi, tegime näo pähe ja 
müüsime viis korda odavamalt.

Seoses kogukondadega ei saa mainimata jätta 
sellist olulist \emph{community}'t nagu WC Fauna\index{WC Fauna}. 
Raske öelda, mis see täpselt oli või kes sinna kuulusid, see oli rohkem 
mõtte- ja eluviis. Selle liikmed tegid igasuguseid asju, pahatihti käisid 
lihtsalt kõrtsides või tegid niisama nalja ja ehitasid lumelinna.

Vanasti olid kompuutrimessid tähtsad\sidenote{Aastatel 1993–1999 
korraldati Eestis igakevadist arvuti-, side- ja bürootehnika messi 
\enquote{Kompuuter}. Tegu oli olulise kogukondliku ja 
müügiüritusega, mida Päevaleht tituleeris lausa infotehnoloogia laulupeoks.}. Ühel messil pakuti meile oma boksi ja pidime selle 
kuidagi sisustama. Boksis oli üks kompuuter, mis luges sekundeid tuleviku 
alguseni, ja WC Fauna leviala kaart, milleks oli punaste läbipaistvate 
vorstinahkadega kaetud Eesti kaart, politseilindiga ümber tõmmatud. 

\question{Tänapäeval läheks selline asi kunstiprojektina kirja.}

Jah, ilmselt küll. Eks see oli häppening, igasuguseid erinevaid asju sai tehtud. Näiteks oli 
IT-inimeste kokkutulek 
OK-fest\index{OK-fest}\sidenote{1994. aastast Eesti Infotehnoloogia- ja 
Telekommunikatsiooniettevõtjate Liidu\index{Eesti Infotehnoloogia- ja 
Telekommunikatsiooniettevõtjate Liit} korraldatud suvine kokkutulek.}, 
kus \emph{community} kokku sai. WC Fauna nimi sai alguse sellest, et ühel 
OK-festil oli vaja jalgpallimeeskond kokku panna. Mõtlesime, et FC Flora juba 
on, paneme siis WC Fauna. Aga see oli ka vist viimane kord, kui jalgpalli 
mängisime. 

\question{Sinu jutust kumab läbi palju 
ühistegevust, aga tavaliselt ei tegeleta arvutitega sellepärast, et 
meeldib teiste inimestega suhelda. Kuidas sul arvutite ja inimeste suhe 
kokku käib?}

Ma olengi imelik loom, kellest pole kunagi aru saadud. Üks tuttav 
psühholoog ütles: \enquote{On olemas insenerid ja on olemas kunstiteadlased, 
aga kumb sina oled, aru ei saa.} 

Inimene areneb vaikselt. Nagu ma mainisin, siis algusaegadel olin 
introvert, kes istus nurgas ja programmeeris ning kartis, kui telefon 
helises. Hiljem hakkasin inimestega suhtlema, seejärel ühiskonda nägema ja sealt tulid ka riigi- ja 
vaat et maailmalaiused asjad. 

\label{sisu:everyday}Üks lugu, milles maksab kindlasti rääkida, on see, kust Skype\index{Skype} tegelikult alguse sai ja kus see kamp 
kogunes. Ilmselt nii mõnigi mäletab, et umbes 1994. või 
1995. aastal oli lehes kuulutus \enquote{otsime programmeerijat, maksame 
viis tuhat krooni päevas}\sidenote{Teiste allikate alusel oli kuulutus lehes 
1999. aastal, mis on loogilisem – muidu jääb Skype'i asutamise ja 
Bluemooni Tele2-seikluse vahele liiga pikk paus.}. Viis tuhat krooni oli kaks kuupalka. 
Kuulutuse tagamaa oli see, et Tele2\index{Tele2}, kes oli juba Eestis olemas, ja 
Bonnier Media\index{Bonnier Media} sepitsesid Rootsis 
nii-öelda uue põlvkonna portaali
everyday.com\index{everyday.com}. Niipea kui nad uudise välja lasid, et 
niisugune portaal tuleb, tõusis nende turuväärtus poolteist miljardit. 
Absurdne, aga nii see oli. Eestisse tuldi jutuga, et meil on tiimid Itaalias, Rootsis ja Taanis ning kõik on juba tükk aega programmeerinud. Kahte 
programmeerijat Eestist on veel vaja, siis saab kõik korda\sidenote{Eestis 
töötas toona Tele2s Stefan Öberg\index[ppl]{Öberg, Stefan}, kes hiljem täitis Skype'is 
mitmeid juhtivaid rolle. Tema juhataski viimase kahe tegija otsijad 
Eestisse.}. 

Mina sattusin seda otsingut nõustama ja lõpuks projektijuhiks, kes 
pidi need inimesed välja valima ja asjad ära tegema. Valisin välja 
Bluemooni\index{Bluemoon} poisid. Sõitsin kõik need Itaalia, 
Taani ja Rootsi kontorid läbi ning sain aru, et peale Rootsi, kus oli tehtud väike 
andmebaasimootor, olid kõik teised tiimid tootnud täielikku kräppi. Nii ei 
jäänudki projekti päästmiseks muud üle, kui kogu värk ise teha. Bluemooni
poistel ei olnud probleem see käsile võtta ja nädala-paariga 
portaal kokku veeretada, kuigi nad PHPd\index{Keeled!PHP} ei tundnud.

Tulevane miljardär\sidenote{Tarvi peab silmas Niklas 
Zennströmi\index[ppl]{Zennström, Niklas}.} oli Tele2s projektijuht ja talle 
hakkasid need poisid meeldima. 

\question{Sina olid portaalis projektijuht. Kui tegid portaali valmis, kas siis 
ei tekkinud mõtet, et peaks suures Rootsi kontsernis kosmilist karjääri tegema?}

Absoluutselt mitte, see oli kõrvaltegevus – aitamisprojekt ja raha 
maksti ka.

\question{Mis su põhitegevus oli?}

Ehitasin riigivõrku ja juhatasin neid 
vägesid. Sinna kõrvale mahtus veel üks kõrvaltegevus, 
mail.ee\index{mail.ee}, mille omanikuks sai ka lõpuks Tele2. 

\question{Kas mail.ee all oli standardne SMTP-server?}

Täpselt nii. Alustuseks oli ilma näota 
meilboks. See tähendas, et igaüks sai endale aadressi luua, aga pidi enda 
meilerit kasutama. Teine arengufaas oli sellele veebi nägu pähe teha, seal oli veebimeiler ka. See sai täitsa ise kirjutatud, all 
oli loomulikult standardne kompott. 

\question{Nii et sa ei läinud ise sinna maailma midagi leiutama, vaid võtsid 
tükid ja ladusid kokku?}

Jaa, see on mul kogu aeg veres olnud. Ühel hetkel sain aru, et 
programmeerimine on üldse kurjast, sest kõik on juba ära tehtud. 
Tegelikult on kunst tükid üles leida ja oskuslikult kokku panna. 
Tänapäeval on tükkide arv muutunud hoomamatuks ja väga raske on neist midagi kokku panna. Ilmselt on 
tekkinud kildkonnad ja voolud. Kunst on muutunud.

\question{Mis üldse on tänapäeval sinu jaoks programmeerimine?}

See kipub olema järjest igavam asi, sest vanasti oli 
see selgelt loometöö. Nii kui hakkasid tulema igasugused 
mudelid ja RUPid\sidenote{\emph{Rational Unified Process (RUP)}. RUP oli 
1990ndatel suurorganisatsioonides levinud tarkvaraarenduse raamistik, 
mis keskendus arendusprotsessi keerukuse vähendamisele läbi standardiseeritud 
rutiinide. Et samal ajal üritati keerukat tarkvara tarnida harva ja suure 
pauguga, võis RUP küll teha projektid paremini kontrollitavaks, kuid ei vähendanud kuigivõrd arendajate frustratsiooni.}, siis hakkas see
järjest rohkem tunduma kraavikaevamisena. Arhitektid joonistavad asja ette ja sina lihtsalt täidad 
funktsiooni. See ei ole eriti keeruline. 

\question{Ometigi ehitatakse igasuguseid hullusi, nagu tekstiterminalis 
video mahamängimine.}

Loomulikult, nalja pärast saab ikka teha. Ma räägin raha 
eest või tööstuslikust programmeerimisest, kus tuleb konkreetset asja teha. 
Vanasti olid mees nagu orkester ja mõtlesid ise välja, kuidas arhitektuur 
võiks välja näha. Tegid oma äranägemise järgi ja keegi ei kobisenud. Nüüd 
on arhitektid. Loovust on 
programmeerijatele jäänud kindlasti vähemaks. 

\question{Kui me juba selle teema juurde jõudsime, siis küsin ka sinu käest, 
milline on ilus kood?}

Ilus kood on loetav kood, siin ei ole kahtepidi mõtlemist. 

\bigskip
\noindent\rule{.3\textwidth}{.7pt}
\bigskip

\question{Kuidas sündis ID-kaart?}

Küberis\index{Küber} tegutsesin ma kahel rindel. Ühelt poolt ehitasin võrke, 
aga olin ka kogu aeg infoturbe ja krüptograafia keskel. Lisaks 
võrguturbele, mis oli sel ajal väga oluline, tundus avaliku võtme 
krüptograafia huvitav ala ja pakkus oma rakenduste poolel pinget. 
Küberis sai jälgitud, kuidas 1995. aastal vist Rootsi Post alustas oma 
ID-kaardi väljalaskmisega ja avaldas ID-kaardi profiili. Päris vara, 
üheksakümnendatel, toodi mulle Ektacosse\index{Ektaco} Schlumbergeri 
kiipkaardid ja paluti vaadata, mis elukad need on. 
Kirjutasin sinna peale programmi nimega \emph{Clevercard}.

\question{Kas see oli Java kaart?}

Javat polnud veel väljagi mõeldud, 
krüptokaarte ka mitte. Mälukaart see ei olnud, protsessor 
oli sees. Sellele kiipkaardile sai käske anda, näiteks „tee fail“. Kõige all oli 
kaardi operatsioonisüsteem. Baidid ajasid sisse, baidid tulid vastu ja ma kirjutasin 
PC-le programmi, millega seda sai mõnusalt teha. 

Aeg läks vaikselt edasi ja see oli umbes 1996.
aastal, kui tegin Äripäeva lahti ja esimesel leheküljel oli pildil Kaja 
Kuivjõgi\index[ppl]{Kuivjõgi, Kaja}, keda ma tundsin ja kes 
oli siis Kodakondsus- ja Migratsiooniameti\index{Kodakondsus- ja 
Migratsiooniamet} asedirektor. Pildi juures oli kirjas, et riik 
planeerib uut dokumenti ja et esimesed passid, mis võeti kasutusele 1992. aastal, saavad 2002. aastal läbi. 
Sinna on viis aastat aega ja KMAs on moodustatud töörühm, kes 
uurib variante millegi uuega välja tulla. 

Võtsin Kajaga ühendust ja ta 
näitas mulle töörühmas arutatud materjale. Kui olin need 
läbi vaadanud, sain aru, et nende tehniline teadmus on üsna allpool 
nulli. Seal räägiti kiibiga varustatud vöötkootidest. 

\question{Mida nad teha tahtsid? Uut ja paremat passi?}

Nad mõtlesid ikkagi kaardi suunas, aga milline see võiks olla – 
kas kiibiga varustatud vöötkood või mis – ei olnud selge. 

\question{Mina olen kogu aeg arvanud, et kaardi pakkusid
välja tehnikud, mitte ametnikud.}

Soov oli tol hetkel väga hägune ja igasugused 
variandid olid laual. Aga oli selge, et kuna tekib suurem 
passivahetus, siis on võimalik inimesi üllatada millegi uuega ning vaadata, mis 
maailmas tehnoloogia vallas toimub. 

Oli päris selge, et KMA\index{Kodakondsus- ja Migratsiooniamet} 
töörühmal ei ole mõtet jätkata. Tehti ettepanek moodustada 
laiem töörühm ja võtta laua taha ka eksperte: pangad, 
telekomid, riigisektori ja Küberi\index{Küber} inimesed.

\question{Kas tänapäeval tundub veider, et riik võtab pangad ja 
telekomid laua taha sellist dokumenti arutama?} 

Absoluutselt mitte. Ei tundu praegu ega tundunud ka tol ajal. 
Laiapõhjaline koostöö riigi- ja erasektori vahel on meile alati edu 
toonud nii ühes kui ka teises. 

\question{See on haruldane asi, mida mujal sageli ei näe.}

Eesti on nii väike riik, et põhimõtteliselt tead kõiki, kes midagi teavad, ja 
ei ole mõtet kedagi kõrvale jätta sellepärast, et ta on parasjagu erasektoris. Me räägime ikkagi eksperditeadmisest ja 
ekspertide kogumist, mitte institutsionaalsest asjast. 

Tuligi töörühm kokku ja arutas asju. Telliti kaks tööd, 
KMA\index{Kodakondsus- ja Migratsiooniamet} maksis. Ühe töö viis läbi 
aktsiaselts Aprote\index{Aprote}, kes uuris, milleks kõigeks 
võiks seda kaarti kasutada. Nad läksid näiteks tanklaketti ja küsisid, 
mida nemad tahaksid. Tulemus oli muidugi väga ulmeline, aga turuootuste uurimine oli 
vajalik tegevus, vaat et kohustuslik samm. 
Teine töö, mida tegime meie Küberis\index{Küber}, oli 
tehnoloogiline ülevaade, milleks kiipkaardid on suutelised, kaasa
arvatud see, mida on Rootsis ja Soomes tehtud. 
Millised on profiilid ja tehnoloogiad, sealhulgas Microsofti 
PC/SC\sidenote{\emph{Personal Computer/Smart Card} – spetsifikatsioon 
tarkade kaartide integratsiooniks arvutustehnikaga.}. 

1996. aastal joonistasin projektiplaani, et neljateist kuuga 
toome kaardi välja, kaasa arvatud pilootprojekt ja muu säärane. Võttis see siiski viis 
aastat, sest see oli väga oluline samm ühiskonnas 
ja vajas pikemat kaalumist. Peale selle tuli seadusi juurde ja ringi teha. 

\question{Kas kõike seda vedas KMA?}\index{Kodakondsus- ja Migratsiooniamet}

Ei, kindlasti mitte. Digiallkirja seadust näiteks vedas 
Majandusministeerium\index{Majandusministeerium}.

\question{Kuidas nii? Asi ju algas 
dokumendi väljastamise vajadusest ja siis äkki tahtis Majandusministeerium digiallkirja 
teha?}

Kindlasti oli suunanäitajaks Saksamaa, kes võttis esimesena vastu 
digiallkirja seaduse, mille pealt Eesti oma on paljuski maha viksitud. Meie 
digiallkirja seadus või vähemalt selle kavand nägi ilmavalgust enne, kui 
oli olemas Euroopa 1999. aasta direktiiv\sidenote{Euroopa Parlamendi ja nõukogu 
direktiiv 1999/93/EÜ.}. Seetõttu oli meie seadus mõnevõrra erinev. Euroopa 
direktiiv lubas igasuguseid lahjasid allkirju ja sellist koledust nagu 
näpuga ekraanile kirjutamist ning ei läinudki tööle. Seepärast tuli ka 
lõpuks eIDAS\sidenote{\emph{electronic IDentification, Authentication and trust 
Services - eIDAS} – Euroopa Parlamendi ja nõukogu määrus 910/2014 e-identimise ja e-tehingute kohta.}, et direktiiv 
oli väga lahja. Kehitati õlgu ja ei kasutatud, tehti lahjasid allkirju ja 
öeldi, et nüüd ongi kõik hästi. 

Meie seadus ütles algusest peale, et ainult 
kvalifitseeritud allkirjad\sidenote{Lihtsalt öeldes on kvalifitseeritud 
elektrooniline allkiri selline allkiri, mida võib pidada võrdväärseks 
omakäelise allkirjaga. Keerulisemalt on öeldud eelviidatud eIDASi direktiivis ja 
selle rakendusaktides.} on aktsepteeritud, ja mingeid lahjasid allkirju ei 
tunnistatud. Neid seadus ei käsitlenudki. 

\question{Kas seaduse väljatöötamises osalesid ka eksperdid või oli 
see Majandusministeeriumi tehtud?}

Eksperdid olid kaasatud. Oli töörühm, kus osalesid
inimesed krüptoloogist kuni juurateadlasteni. Nad tegid seda tööd ligikaudu
kaks aastat, nii et see ei tekkinud niisama, vaid mõeldi väga põhjalikult 
läbi. Kuskilt mujalt kui Saksamaalt ei olnud šnitti võtta. Nii mõnigi 
seadusepunkt oli inspireeritud nii-öelda krüptograafide mõtlemisest. 

\question{Sinu jutust ei kõla läbi kõikehõlmav õilis visioon 
sellest, kuidas ühel päeval sünnib Eesti digiühiskond ja kõik saab e-teenuste 
abil uueks loodud.}

Eks see võibolla kuskil ajusopis oli, aga mis sellest ikka rääkida, asju 
tuleb teha. 

\question{Seda ma peangi silmas, et liikumine toimus samm-sammult ja tegeldi 
konkreetsete asjadega.}

Jah, kasvõi seesama ID-kaardi väljatoomine. Võib ju digiallkirja seaduse vastu 
võtta (mis aastal 2000 ka vastu võeti), aga kui inimestel ei ole vahendit, 
millega digiallkirja anda, siis pole seadusel suuremat mõtet. 
Euroopas valitses ka selle direktiivi tegemise ajal nägemus, et 
kommertsfirmad hakkavad sertifikaati müüma ja seetõttu on vaja neid 
reguleerida. Kuidas see võiks käia? Teed turule putka ja hakkad sertifikaate 
müüma: suured ja väikesed sertifikaadid, punased ja kollased? 

See ettenägemisvõime oli meil küll, et niisugune visioon, et müüme inimestele 
sertifikaate, neid ostetakse ja kuidagi tekib kasutus, on üdini vale. Selles mõttes oli näiteks Soome, kes tegi ID-kaardi 
mittekohustuslikuks ja pani kohe hinnaks nelikümmend eurot. Siis juhtub see, et teenusepakkujad ei hakka ID-kaarti toetama, sest 
nad teavad, et inimestel ei ole seda (viiel protsendil võibolla on). Ja inimestel ei ole kaarti, sest teenuseid ei 
ole. Siis ongi nokk kinni, saba kinni ja mudel, mis ei toimi. 

Kõigepealt peab elektroonilise identiteedi 
taristu looma ja siis võibolla hakkavad asjad juhtuma. Samamoodi nagu ei 
saa proovida kuskil metsa sees müüa kilomeetrit maanteed kohalikule 
metsaelanikule. 

\question{Su jutust kõlab läbi üsna suur usaldus ekspertide vastu. Poliitika eest vastutava inimese ja krüptoloogi 
vahel pidi olema usalduslik vahekord, et viimasel lasti seadusesse punkte kirjutada.}

Skepsis on väga raske tekkima, kui laua taga on Eesti paremad pead – misasja sa ikka kahtled või kõhkled. Mida targemaks inimesed saavad ja 
mida rohkem eksperte on, seda rohkem tekib diskussiooni. 

\question{Kui suur see ekspertide ring oli, kes töörühmades käis ja seda 
ideed kujundas?}

Viis kuni kümme võtmeinimest.

\question{Ja kogu tarkus tugines tollele salapärasele raamatule, mis 
Küberis oli?}

Oo ei, see raamat oli lihtsalt algus. Me ei räägi ainult 
krüptoloogiast või infoturbest. Näiteks ID-kaart ei puuduta ainult 
infoturvet, vaid väga paljusid rakenduslikke ja isegi sotsiaalseid aspekte. Ei saa rääkida, et krüptograafia päästis maailma. 

\question{Sagedasti inimesed arvavad, et kui asjad saaks 
ära krüptida, siis olekski maailm päästetud.}

Ma jään selle juurde, et ei saa kilomeetrit maanteed müüa 
külaelanikule, sest ta küsib: \enquote{Mis ma teen selle maanteega?} – 
\enquote{Hakkad autoga sõitma.} – \enquote{Aga mis see auto on?} See on 
ufo müümine, täiesti mõttetu tegevus. Sa ehitad teed valmis, lased autod müüki, 
paned sõidukoolid püsti ja siis ühel päeval võibolla inimesed avastavad, et 
transpordist on kasu. Aga kui hakkad sellest pihta, et proovid 
igaühele juppi maanteed müüa, siis see ei toimi. 

\question{Küberneetika Instituut\index{Küberneetika 
Instituut} ja selle järelmid on väga pikalt Eestis olulist rolli mänginud. 
Sina oled olnud seal sees ja, mis veel olulisem, ka sellest väljas. Kas sa oskad 
öelda, mis maagiline asi selle asutuse nii võimsaks teeb?}

Nagu ma mainisin, siis 1997. aastal jagunes Küberneetika Instituut kolmeks. Osa 
läks ülikooli alla, osast moodustus aktsiaselts ja andmeside osa läks riigile. 
Võibolla kõige nähtavam osa IT-inimeste jaoks ongi Küberi instituudi või 
aktsiaseltsi ehk infoturbe ja programmeerimise osa. Seal tehakse meremärke 
ka, aga need on merel ja ei paista välja.

Mul on olnud au omal ajal umbes kolmkümmend inimest sinna tööle võtta ja 
Tartu labor\index{Cybernetica!Andmeturbelabor} asutada, mis on nüüd inimeste arvu mõttes nüüd isegi suurem, kui Tallinn. Need olid väga toredad ajad. Aga fenomen seisneb selles, et Küberit peetakse põhimõtteliselt ainukeseks firmaks 
Eestis, kes oskab turvaliselt programmeerida ja teab midagi infoturbest. Seetõttu 
on neile ka sattunud niisugused tegevused ja projektid alates X-teest ja 
lõpetades Smart-IDga\index{SplitKey}, kus turvalisuse ja krüptograafia komponent on omal 
kohal. 

Lisaks on seal tõsiseid inimesi, kes tegelevad puhtalt teadusega ja koodi ei 
kirjuta. Küberis on oma teadusosakond ja teadusdirektor. Ühtlasi käivad 
teadusega tegelevad inimesed Küberi ja ülikoolide vahet. Sellist sümbioosi 
otsitakse nagu spunki mööda Eestit taga ja ka Teaduste 
Akadeemia president ei väsi rääkimast, et Küber on fenomen ja suur erand. 

\question{Miks ei ole näiteks Helmes võtnud endale teadusdirektorit 
tööle ja hakanud sama tegema?}

Asi on selles, kas teed kõigepealt teadust ja siis hakkad seda 
rakendama ühiskonnas või proovid vastupidi teha: kõigepealt oled kõva
programmeerija ja siis mõtled, et teeks teadust ka kuidagi. Päris 
nii see ei käi, need juured on natuke sügavamal.

Juhtusin hiljuti nägema Küberi töökuulutust: otsitakse 
projektijuhti, nõutav CISA\sidenote{\emph{Certified Information Systems Auditor} – sertifitseeritud infosüsteemide audiitor.} sertifikaat. 
Halleluuja!

Omal ajal kõik teadsid, et Martens tuleb jälle jaurama ja Küberisse tööle 
meelitama. Mul oli väga lihtne äriidee: ajan kõige targemad 
inimesed ühte suurde ruumi ja annan neile teema kätte. Nad asuvad 
plaksti tööle, ise panen jalad seina peal. Töötas! Väga hästi töötas! Pärast 
lugesin kuskilt raamatust, et niimoodi tuleb käituda, ja täpselt nii ma 
olengi käitunud.

%--------------------------


\question{Kas sa oled siis asjade käimalükkaja ja visionäär?}

Kui sa nii ütled.

\question{Ma ei ütle, ma küsin. Sa ütlesid, et programmeerija sa enam ei ole. Kes 
sa niisugune oled?}

Ma ei oska ennast sildistada. Mul on see häda küljes, et mõtlen kogu aeg kuidagi 
laiemalt.

\question{Miks see häda on?}

Võiks ju midagi näpu vahel teha, kaltsuvaiba või midagi. On vähemalt füüsiline tükk taga, 
suurtest sõnadest ei jää midagi\ldots

\question{Mida sa praegu teed?}

Olen endiselt elektroonilise hääletuse juht, juba aastast 2003. 
Hiljuti olid meil kümnendad valimised, kohe on algamas üheteistkümnendad, 
Europarlamendi valimised\sidenote{Jutuajamine Tarviga leidis aset 2019. aasta mai algul, Europarlamendi valimised toimusid 26. mail, edukas 
elektrooniline hääletamine 16.–22. mail.}. Aga valimised võtavad võib-olla 
kaks-kolm kuud tähelepanu. Valimistevahelisel ajal ma palju 
suurt ei teegi. Jõudumööda, nii nagu kutsutakse, käin maailmas ringi ja proovin 
inimesi aidata nende arengus erinevates riikides – nii 
elektroonilise identiteedi teemal kui ka IKT rakendamise alal 
nii-öelda valimismajanduses.



\chapter{Peeter Marvet}
\index[ppl]{Marvet, Peeter}
\index[ppl]{Tehnokratt|see{Marvet, Peeter}}

\question{Kuidas ja millal sa jõudsid arvutite juurde?}

See oli umbes täpselt aastal 1985, pidin siis olema 15 aastat 
vana. Eelnevalt olin arvuteid näinud Soome televisioonist, 
seal reklaamiti ilmselt Commodore 64 ja Spectrumi masinaid. 

\question{Sa oled järelikult Tallinna poiss?}

Jah. Ma olen sündinud Tartus, aga pikalt Tallinnas elanud. 

Kui ajas veel tagasi krutin, siis üks kokkupuude arvutitega oli 
veidi varem, papsi laboris. Ta töötas TPI Veekvaliteedi 
Laboris\index{Tallinna Tehnikaülikool!Veekvaliteedi labor}, mis asus selle koha peal, 
kus keset Järvevana teed on praegu Maru Ehituse maja. Seal oli olemas üks 
terminal, mis käis Datasaabi\index{Datasaab}-nimelise 
arvuti\sidenote{Datasaab oli Rootsi lennukitootja Saab arvutustehnoloogia 
eraldi ettevõtteks kasvanud divisjon, kus toodeti nii tsiviil- kui ka
militaarkasutuseks mõeldud arvuteid.} külge, mis asus kusagil Mustamäe teel. 
See masin oli ostetud ühelt rahvamajandussaavutuste näituselt, kus 
vahetevahel käisid ka välismaalased kohal. Kuskilt oli saadud valuutat ja ostetud selle
eest välismaa arvuti, mille külge käisid modemitega oranžid terminalid. 

Datasaab osteti väidetavasti ilma 
operatsioonisüsteemi ja igasuguse rakendustarkvarata, sest rohkem 
rutsi sellel ajal ei olnud. Aga Nõukogude insenerid olid vinged, kirjutasid sinna ise operatsioonisüsteemi peale. Sellest tarkvarast õnnestus veel mingisugune jupp Datasaabile 
tagasi müüa ja saada ilmselt vastu mälu või 
lisakomponente. 

Datasaabi terminalil õnnestus 
lihtsalt, ilma ühenduseta \emph{backspace}'i ja tühikuga 
\enquote{ronge kokku haakida}. See on minu esimene mälestus arvutiga suhestumisest. 

Järgmine mälestus on samuti aastast 1985, kui olin 
ilmselt seitsmendas klassis. Toonases Pedas toimus Tallinna koolide 
füüsikavõistlus ja meid viidi ka arvutisaali, kus oli Minsk\index{Minsk}. Seal sain tuttavaks ühe aasta vanema
koolivennaga, 
kellel oli kaasas isiklik perfolint programmiga. See ei olnud
keegi muu kui Sulo Kallas\index[ppl]{Kallas, Sulo} ja tema perfolindi peal oli 
üks mängulaadne asi, mis arendas mingisuguseid 
organisme\sidenote[][-2cm]{Tõenäoliselt oli lindil Briti 
matemaatiku John Horton Conway välja mõeldud rakuautomaat, mida tuntakse 
nime all Game of Life. Tegu on mängijateta mänguga, mis ainsa sisendina 
vajab algseisu määratlemist. Automaat on ühest küljest levinud 
programmeerimisülesannne ja teisalt põnev uurimisobjekt, seetõttu võis selle 
realiseerimine olla noorele arvutihuvilisele nii huvitav kui ka jõukohane.}. 
Sulo vend oli Raadiomaja 
Arvutuskeskuses\index{Raadiomaja Arvutuskeskus}, nii et selleks ajaks oli Sulo 
juba mõnda aega arvutitega tegelenud. 

Ja see oligi esimene kord, kui sattusin arvutiga kokku ja mõtlesin 
\enquote{oo, vinge!}.

\question{Mis seal vinget oli? Mis konksu külge sa jäid?}

Tol hetkel oli see rohkem \enquote{ahaa, vau, teebki 
mingisugust asja!}. 

\question{Ja Sulo oli kõva mees oma perfolindiga?}

Nojaa, ikkagi kaheksandik, vanem koolivend, kellel on, kujutad 
ette, isiklik perfolint! Vau! Sellised kutid on ümberringi! Siis peab 
ikka ise ka vaatama, mida seal tehakse. Küllap ka Soome televisioonist 
arvutitega seoses nähtu tekitas soovi, et olgu või 
Nõukogude oma ja perfolindiga, aga ikkagi arvuti. 

Mõni kuu hiljem tulid kooli paar djuudi, kes tegid arvutiklubi ja 
kutsusid mind osalema. 

\question{Kas see oli legendaarne arvutiklubi Ahhaa?}

Ei, see oli legendaarne arvutiklubi Juta\index{Juta}, mida vedas 
juudi papi Lev Moišeejevitš Šoroht\index[ppl]{Šoroht, Lev}. Klubi 
tegutses Raua ja Kreutzwaldi tänava nurga peal, ühe maja keldrikorrusel, kus on kaarega 
aknad. Nende akende taga asuski arvutiklubi Juta. Kui 
Ahhaa puhul võiks ette kujutada, kust see nimi tuleb, siis 
Juta nimi tuleneb vene keelest: \begin{russian}Юный Техник 
Автомат\end{russian}. Ma eeldan, et Lev Moišeejevitš Šoroht ei satu seda 
lugema, aga kui kellelegi meenub, et aastal 1985 vedas ta TPI või Peda 
üliõpilasena noori arvutiklubisse, siis ma suurima hea 
meelega saaksin kokku ja teeksin väiksed õlled välja, sest sealt see suurem arvutihuvi alguse sai. 

Klubis õpetati meile programmeerimist PL/I\index{PL/I} keeles.

\question{Kas klubisse käidi kutsumas, mitte ei joostud ust maha, et arvuti ligi 
saada?}

Ma arvan küll. Ma täpselt ei mäleta, aga sõnum jõudis 
meieni vist kooli või õpetajate kaudu. Küllap Sulo oli ka seal, 
sest kui oli võimalik kusagil veel arvutisse saada, siis loomulikult seda võimalust 
kasutati.

\question{Seda ma mõtlengi, et tol ajal ju otsiti tikutulega kohti, kus 
\enquote{arvutisse saada}.}

Seitsmenda klassi lõpupoole tuli arvutuskeskusest ühe minu 
programmi väljatrükk laia aukudega paberi peal, \emph{line}-printeril 
välja lastud. Keegi klassivendadest, kes oli ilmselt meiega seal 
koos käinud, tõi väljatrüki klassi ja siis kõik vaatasid, et 
oo, Soome telekavad. Tollal oligi inimestel kõige 
üldisem seos arvutitega see, et arvutuskeskustes trükiti välja Soome 
telekavasid, kus olid näiteks tabuleeritud kujul eraldi väljavõtted 
seriaalide kohta. Parimatel vendadel olid olemas nädalakavad. Aastal 1985 
keskmine teadlikkus arvutitest umbes selline oligi.

\question{Kust need kavad saadi?}

Need liikusid arvutuskeskuste vahel ja vähemasti millalgi oli üks selline 
koht Postimaja Arvutuskeskus\index{Postimaja Arvutuskeskus}, mis on 
üks väheseid kohti, kus ma ise pole käinud. Seal oli SM-4\index{SM EVM!SM-4}, mille külge oli ehitatud teksti-TV 
vastuvõtja\sidenote{Kavade allikaid oli rohkem kui üks. Mäletatakse, et vastav 
riistvara oli olemas TPI raadiotehnika kateedris\index{Tallinna Tehnikaülikool!Raadiotehnika kateeder} 
Apple II küljes. Räni Meister\index[ppl]{Meister, Räni} olla selleks otstarbeks kasutanud 
ka Eesti Televisiooni\index{Eesti Rahvusringhääling!Eesti Televisioon} Amigat.}, 
ja sealt see kava tuli. SM-4 küljes oli 300boodine modem, millega 
pumbati kavasid mööda linna laiali. Mäletan üht
etappi, kui minu päralt oli üks PC, välja arvatud vist kolmapäeviti, 
kui lõuna paiku saabus Postimajast üle modemi Soome telekava, mis trükiti maatriksprinteril enam-vähem nähtamatuks kulunud lindiga välja. Siis 
pidin endale muud tegevust leidma, aga muudel pärastlõunatel sain seda 
arvutit kasutada.

\question{Ma katkestasin sind seal, kus sa PL/I keeles programmeerisid \ldots}

Meile õpetati natuke programmeerimist ja seal oli palju segaseid ja 
täiesti arusaamatuid asju. Oli programmeerimiskeel mingisuguste muutujatega, 
mis mõnevõrra koitis. Ja ilmselt esimene programm, mida meile õpetati, oli ruutvõrrandi lahendamine. Annad paar muutujat sisse ja siis 
trükitakse tulemus paberil välja. Alguses meid 
päriselt arvuti juurde ei lastudki --- keegi toksis  meie 
programmid sisse ja pärast saime väljundi kätte. 

Hiljem leiti meile võimalus arvutitega tegeleda veel kahes kohas. Üks oli 
Tihnikus, kus asus ETKVLi Arvutuskeskus\index{ETKVLi Arvutuskeskus}. Järgmised 
põlvkonnad teavad seda kohta kui esimest Maksimarketit, seal ühes majas oli üks 
vinge ES\index{ES EVM}. Teine koht oli Endla tänaval, kus asus Maksuameti maja\sidenote{2013. aastani asus 
Maksu- ja Tolliameti teenindussaal aadressil Endla 8.}. Seal kolmandal 
korrusel olid Ehituskomitee\index{Ehituskomitee} ESid\index{ES EVM}. 

Minuga läks edasi umbes niimoodi, nagu õpetatakse tööõpetuses, et on 
oluline anda lastele midagi, mille nad saavad valmis voolida, näiteks puulusika,
et nad saaksid tulla koju ja seda perele näidata. Siis laps saab kiita, tal läheb edaspidi väga 
hästi ja ta teeb paremaid puulusikaid. Kui ma olin teinud 
oma esimese kolmeteistrealise programmi ruutvõrrandi lahendamiseks, laekusin 
selle väljatrükiga koju ja köögis näitasin vanematele, et 
näete, sihukese raha eest tegin sihukese asja. Paps, kes oma 
teadustegevuses tegeles elektrokeemia ja hapnikuanduritega ning teisalt 
oli džässpianist, vaatas mu tööd ja ütles: \enquote{Mul oli 
just üks tudeng, aga ta kadus ära ja temast jäid ainult mingid listingud 
järele. Kas sa saad nendest sotti? Mul oleks vaja teadusandmeid 
töödelda.} Ülesandeks oli anduri toimekõverate kokkuajamine 
matemaatiliste valemitega, et õnnestuks digitaalseid mõõteriistu teha. Ja nii juhtuski, et olles programmeerinud oma 
esimesed kolmteist rida esimeses mulle täiesti tundmatus keeles, 
läksin kohe üle järgmisele.

Nii ongi mu pea nagu puder ja kapsad selles mõttes, et ma suudan kirjutada 
ainult dokumentatsiooni abil, kaasa arvatud keeli, mida ma igapäevaselt 
kirjutan, nagu PHP\index{PHP} ja JavaScript{Keeled!JavaSctipt}. Need 
süntaksid on peas nii segi, et ma ei mäleta kunagi täpselt 
PHPs \verb|for|-tsüklis parameetrite järjekorda. Õnneks on tänapäeval 
olemas kõikvõimalikud IDEd, mis teevad mõningase töö ära ja aitavad \emph{auto 
complete}'ida. 

\question{Nii et su puulusikas mitte ainult ei saanud kiita, vaid pandi kohe 
ka tööle!}

Puulusikas võeti kohe tööle, sealt edasi olin lapstööjõud. Ühel
hetkel hakkas papsil kahju --- laps võiks lisaks 
ekspluateerimisele natuke ka raha saada. Mind pandi ametlikult kirja 
veerand kohaga laborandina. Tänu sellele oli mul ligipääs kõikide papsi sõprade arvutuskeskustele. 
Ja kuna paps tegeles oma hapnikuga TPIs, siis loomulikult olid nende 
sõprade hulgas TPI ja teisi seltskondi, kes olid seotud mingisuguste 
anduritega. Näiteks Pirital Masti tänaval 
arendati sportlaste mõõtmise lahendusi ja selle teine ots asus 
kiirabihaigla arvutuskeskuses\index{Kiirabihaigla arvutuskeskus}. Seal saingi 
pidevalt üht arvutit kasutada, välja arvatud kolmapäeviti. 

Arvutiks oli Sanyo PC ja see oli väga vinge. Seal oli muuseas olemas ka Apple II\index{Apple II}, mille peal sai mängitud Karatekat\index{Karateka}. Ja kui ma 
õigesti mäletan, oli seal ka üks Labtami\sidenote{Austraalia arvutitootja aastatel 1972--1990, kellel 
olid Nõukogude Liiduga head suhted. Aastal 
1984 disainis Novosibirski Riikliku Ülikooli tudengite Kronos Research Group 
neile emaplaadi. URAL-LABTAM OOO tegutseb Venemaal siiani ning nende arvuteid 
leidub lisaks Austraalia arvutimuuseumidele Tartu Ülikooli omas. Labtami 
arvuteid osteti naftadollarite eest ka Küberneetika 
Instituuti\index{Küberneetika Instituut}.}-nimeline \emph{kone}. Lisaks oli seal
suurte trumlitega andmetöötlus-\emph{kone}, millega mina ei suhestunud ja mille  
nime ma ei mäleta. 

\question{Kas andurid ja elektroonika ei pakkunud sulle huvi?}

Mitte eriti. Progemine oli huvitavam. 

TPI santehnika kateedris\index{Tallinna 
Tehnikaülikool!Santehnika kateeder} oli olemas SM-4\index{SM EVM!SM-4}. Ehituse all oli 
selline kateeder, vee kvaliteet ja kõik selline kuulus sinna alla. Mingil 
hetkel tekkis sinnasamasse Järvevana teele, kus asus ka Läänemere 
Instituut\index{Läänemere Instituut}\sidenote{Ei ole selge, mis asutust Peeter 
silmas peab. Eestis on Läänemere Instituut tegutsenud eelmise sajandi 
kolmekümnendatel ja praegu tegutseb sellenimeline asutus Soomes.}, ka 
SM-4\index{SM EVM!SM-4}, mille ma kohale 
minnes lülitasin ise sisse ning pärast töö tegemist jälle viisakalt välja. 

\question{Arvuteid oli siis ikkagi piisavalt?}

Kui sattusid õigesse kohta ja 
oskasid õigel ajal vait olla ning mitte liiga palju täiskasvanuid segada nende 
tähtsas töös, siis üldiselt jagus. 

Tulles korraks veel tagasi alguse ehk Juta\index{Juta} juurde, 
siis selle asutajal oli endal ka paar huvitavat projekti, millega ta 
üritas Nõukogude Liidu tasemel kuulsaks saada. Üks neist võiks olla võrreldav 
Facebookiga: kirjasõbrad kogu Nõukogude Liidust 
saadaksid oma andmed, mis sisestatakse perfokaartidel 
\emph{mainframe}'i ja see teostab \emph{match-making}'u ning 
siis saadetakse kirjad leitud \emph{match}'idele laiali. 

Mina käisin algusest lõpuni Reaalkoolis ja keskkooliajal tekkisid arvutid ka meie kooli. Saime klassitäie Yamaha MSXe\index{Yamaha MSX}. Kuna erinevate koolide vahel oli masinate saamiseks 
konkurents, olla Reaalkool saanud ka ähvarduskõne, mille peale 
vaprad raadiorufi ja füüsikaklassi tagaruumi noored organiseerisid 
öö läbi valve koolimajja. Arvutikastid olid vist direktori kabinetis, me ööbisime koolis ja valvasime neid. Pärast sai
arvutiklass meie teiseks koduks.

\question{Mis seltskond seal raadiorufi ja füüsikaklassi tagaruumis koos käis?}

Seal olime mina, Sulo Kallas\index[ppl]{Kallas, Sulo}, Heiki 
Savitš\index[ppl]{Savitš, Heiki}, Vallo Veinthal\index[ppl]{Veinthal, Vallo} 
ja Reimo Mesipuu\index[ppl]{Mesipuu, Reimo} --- kindlasti jätan kedagi 
ebaviisakalt mainimata. Avo Nappo\index[ppl]{Nappo, Avo} tiirles meie ümber 
rohkem arvutiseltskonna poolest, raadioruumis olime põhiliselt vist mina, Sulo 
ja Reimo.

\question{Mille alusel seltskond moodustus? 
Klassivennad? Tehnikahuvi?}

Otseseid klassivendi oli vähe, jäime paariaastasesse vahemikku. 
Sulo oli kõige vanem, Vallo ja Heiki olid meist aasta nooremad. Tegu oli pigem kooli aktiiviga, keda huvitas tehniline pool. 
Füüsikaklassi juures oli raadioruum, kus me hängisime, sest seal sai 
nuppe keerata. Sellest pundist tekkis hiljem suurem seltskond arvutiklassi ümber. 

\question{Kas programmerimine ja raadioruumitamine koolitööd ei hakanud segama?}

Lõpetasin kooli neljade-viitega, nii et selles mõttes probleemi ei olnud. Medaliga lõpetajat poleks minust nagunii saanud, see ei olnud minu maailmavaates.

\question{Eks see ongi tunnetuse küsimus, kumb oli tol hetkel primaarne.}

Eks arvutipool oli põhiline. Keskkool möödus üleüldse enam-vähem 
niimoodi, et vahepeal sai käidud kohvikus ja vahepeal 
olümpiaadidel. Kui olid olümpiaadid, olid hinded head, sest õpetajad ei 
saanud ometi olümpiaadil esinejatele halbu hindeid panna. Aga kui olümpiaadil 
ei käinud, siis kippusid hinded kehvemaks minema, sest kooliskäimine ununes. Näen siiamaani unenägusid sellest, et eksam on 
tulekul ja ma olen unustanud terve veerandi tunnis käia. 

Sellest tekkis mõtteviis, et ma ei pea olema midagi õppinud. Kui läksin
TPIsse ja mataeksam tehti koos raamatutega, siis jõudsin eksami 
käigus alati ära õppida selle, mida oli eksamiks vaja. Ma ei pidanud 
eelnevalt liiga palju loengus käimisele pühenduma, vaid võisin 
lihtsalt tulla ja eksamid ära teha. Ülejäänud semestri sai arvutitega 
tegeleda. Ma kindlasti ei soovitaks seda noortele, aga minul juhtus
niimoodi. 

Selline lähenemine tekitas mul teistsuguse 
arusaama ümbritsevast tehnikast. Ma ei karda midagi selles 
mõttes, et kui on vaja asi ära teha, siis tuleb võtta \emph{manual} või 
kood ette. Loomulikult võtab see aega --- läksin eksamile 
esimesena sisse ja tulin viimasena välja, aga sain kolme tunniga 
õige asjaga hakkama. Tundus nagu efektiivne lähenemine. Võibolla 
oleksin saanud targemaks, kui oleksin süsteemsemalt õppinud. 

\question{Mida sa TPIsse õppima läksid?}

See oli TI ehk majandusinfo töötlus\index{Tallinna Tehnikaülikool!TI}. Linnar 
Viik\index[ppl]{Viik, Linnar} lõpetas sama ala mõned aastad enne mind. 
Aasta sattus olema 1989, kui päris pol-ök-i ja kompartei 
ajalugu\sidenote[][-2.5cm]{Nõukogude ajal kuulus ülikoolihariduse juurde kohustuslikus korras \enquote{punaste ainete} 
läbimine: kaheksakümnendatel olid nendeks kommunistliku partei ajalugu, dialektiline ja ajalooline materialism, kapitalismi ja sotsialismi poliitökonoomia, filosoofia ajalugu ja teaduslik kommunism. Lisaks veel teaduslik ateism ja marksistlik eetika. Ka kooli lõpetamisel tuli üks riigieksamitest sooritada mõnes nendest ainetest} ei oleks tahtnud õppida, aga nad ei olnud veel välja 
mõelnud, mida nende asemel õpetada. Oli ka muid asju, mille vajalikkusest ma 
päris täpselt ei saanud toona aru. Näiteks
miks ma peaksin tegema transistoritest valmis 8080 protsessori paar käsku, 
eriti kui normaalsed inimesed kasutavad vähemasti Z80-t, mitte 8080-t. 
Teismelise värk --- ei olnud piisavalt \emph{cool}. \enquote{Intel 8008? 
Zilog\sidenote[][-3mm]{Zilogi toodetud 8-bitine Z80 protsessor oli Inteli 
8080 protsessoriga ühilduv, aga märkimisväärselt odavam.} on normaalne!} 
Täpselt sama lugu, nagu täna on hipsteri habe või muud välised 
tundemärgid. 

Pean tagantjärele tunnistama, et kuigi olin algul selle suhtes 
kriitiline, siis hetkel käin koolitusel, kus räägitakse sellest, kuidas 
\emph{fuzz}'imisega\sidenote{\emph{Fuzzing} on tarkvara (turvalisuse) testimise 
meetod, kus programmile söödetakse süsteemselt juhuslikku sisendit.} 
mälukorruptsiooni juhtumeid leida. Kui lektor ütles, et see on maru keeruline, 
räägime hästi aeglaselt ja mitu korda nagu miilitsatele, siis minu arust midagi 
nii rasket seal polnud, \emph{stack} on \emph{stack}. Protsessoril on 
registrid, ma olen neid transistoritest teinud. Kui 
pead protsessori arhitektuuritasemel läbi mõtlema, kuidas käskude 
töötlemine toimib, kuidas pointereid inkremenditakse ja 
kuidas see mäluga on seotud, siis saad aru, kuidas arvuti 
masinkoodi tasemel töötab. Mul on väga tore kuulata, kuidas mu vanem poeg räägib, et
Tartu Ülikoolis sunnitakse neid ka aru saama protsessori siseehitusest. 
Tõsi küll, raamatu tasemel, aga nad programmeerivad ka assemblerit ja see on väga oluline. 

\question{Kas sind akadeemiline maailm ei tõmmanud, kuigi servapidi olid
juba selle sees?}

Ei, sest ma sattusin keskkooliajal sellisesse seltskonda nagu 
vabariiklik õpilasstaap\index{Vabariiklik õpilasstaap}, mis oli 
komsomoli keskkomitee juures tegutsev mittekommunistlik vastupanuliikumine. 
Tiina Tšatšua\index[ppl]{Tšatšua, Tiina} oli näiteks üks selle eestvedajaid. 
Sellest sai vabariigis üks toonaseid orgunnitiime, kes korraldas 
suurüritusi, milleks alustuseks olid komsomoli ja EKP kongressid. Organiseerimise
mõttes on ju savi, kas tegu on EKP kongressi või Eesti Kongressi või Rahvarindega. Inimesed tulevad kohale, neid tuleb 
registreerida ja toita. Kui on dokumentidega üritus (mida 
tänapäeval eriti ei toimu, aga kõik Eesti Kongressi ja Rahvarinde kongressid 
olid sellised), siis on olemas näiteks redaktsioonitoimkond. Meie olime 
arvutitiim, kes organiseeris seda, et registratuur toimiks 
listide alusel, ja samuti toetasime redaktsioonitoimkonda kõikvõimaliku 
tekstitöötluse, väljatrükkimise ja vormistamisega. 

Kui keskkool sai läbi 1989. aastal, siis oli mul suveks üks tööots. Tallinnas 
toimus ÜRO invaekspertide tipptasemel kokkusaamine. Tallinnas lõigati sel puhul esimesed äärekivid faasi ja minu 
arust Jack Lippmaa\index[ppl]{Lippmaa, Jaak}\sidenote{Peeter peab ilmselt 
silmas Jaak Lippmaad} isiklikult ehitas ümber paar 
Ikaruse bussi\sidenote{Ungari tootja Ikarus bussid olid Eestis laialt kasutusel 
liinibussidena.} nii, et neisse kuidagi ratastooliga sisse saaks. Kuidas see 
võimalik oli, ma ei kujuta ette. Meie ehk Reaalkooli tiim toetas ürituse 
redaktsioonitoimkonda, kes vormistas ÜRO-le kõigis põhikeeltes 
dokumente. See tähendas, et oli posu tõlkijaid, aga aastal 1989 ei olnud ilmselt ükski tõlkija näinud arvutit rohkem kui 
võibolla Soome reklaamides. Meid oli piisavalt palju, kümmekond inimest, ja hoidsime tõlkijatel kätt ja jalga. Kui kellelgi tekkis kivistunud pilk, siis 
keegi meist tuli ja \emph{reboot}'is tõlkija arvuti taga või arvuti enda, kumb 
parasjagu oli rohkem kinni jooksnud. 

Minu enda hilisemas eluloos on see episood huvitav sellepärast, et 
olles parasjagu keskkooli lõpetanud, õnnestus mul tolle ürituse jaoks lihtsalt 
omaenda sõna peale linna pealt toatäis PCsid kokku laenata. 
Paar tükki siit, paar tükki sealt ja kokku sain umbes kaheksa arvutit. Kõige kihvtim 
tuli surnukuurist --- üks PC, mille peal oli 
Xerox Ventura Publisher koos Xeroxi graafilise kasutajaliidesega, milleks oli 
GEM ja mis nägi välja nagu MacOS\sidenote{Graphics 
Environment Manager oli üks varastest graafilistest 
kasutajaliidestest, mille liigne sarnasus Apple'i tarkvaraga viis ka kohtuasjani.}. GEM 
sai DOSist üles \emph{boot}'itud, läks ilusti mustvalgeks ja halliks 
kasutajaliideseks ning seal peal jooksis minu esimene küljendusprogramm. Lisaks
saime neilt ühe laserprinteri kasutada, mis ei olnud küll PostScript, aga siiski laserprinter. 

\question{Siis oli ju veel Nõukogude aeg!}

Ilmselt meditsiin oli saanud üht-teist valuuta eest 
osta. Tegelikult Kivilo\index[ppl]{Kivilo, Ago} plaanis kesklinna oma 
diagnostikakeskust\index{Diagnostikakeskus}\sidenote{1988. aastal asutatud Diagnostikakeskus oli omal 
ajal märgilise tähendusega. Ühest küljest pakuti kõrgtehnoloogilisi 
teenuseid, keskuse algusaegadel asus seal Eesti ainus kompuutertomograaf. 
Teisalt oli tegu väga innovatiivse organisatsioonilise 
konstruktsiooniga, mis viis hiljem mitme keskust ümbritsenud 
kõrge profiiliga afäärini.}, meditsiinis olid 
väga kõvad tegijad. Eesti arvutinduse arendusest teatakse rohkem Tartu
seltskonda, kes on seotud geneetikaga, ja võibolla 
Küberit\index{Küber}. Mina sisenesin meditsiiniliini pidi, selles 
valdkonnas tegeldi päris kõvasti teadus- ja arendustegevusega. 

\question{Kas sealt said ka oma küljendamiskonksu?}

Jah. Otse loomulikult sai hunnik flopikettaid Venturaga ära kopeeritud, mis toona oli igati tavapärane, \emph{standard operating 
procedure}: kõigest, mis kätte sattus, tehti koopia. Ja nii juhtuski, 
et aastatel 1989-1990 oli minu jaoks ülikoolis käimisest palju 
huvitavam arvutiga küljendamine ja kujundamine. 

\question{Kas sul on muidu ka joonistamise soon?}

Ei ole. Kahtlustan, et inimesed, kes on pidanud minu küljendatud raamatuid 
tarbima, on kindlasti selle all kannatanud, nii et ma väga vabandan. Näiteks
Avita\index{Avita kirjastus} kirjastuse algalgusaegade raamatutest oli suur hulk 
minu tehtud.

\question{Mis sind küljendamise juures köitis, kui sul muidu ei olnud
visuaalkunsti huvi?}

See oli hoopis teistmoodi arvutiga 
tegelemine kui programmeerimine ja andmetöötlus, mis olid ka toredad. 
Mind tõmbas see, et õnnestus asju ekraanil teha. 

Pärast 1989. aasta suve üritust läksin ma ülikooli. Ja siis 
Mart Siilmann\index[ppl]{Siilman, Mart}, kes oli äsja lõppenud ürituse orgunni 
pealik, ütles, et kuule, järgmisel suvel on ka üks üritus, kus on 
arvutiabi vaja, tule ka. Aastal 1990 toimuski European Nuclear Disarmament Convention ehk suur rahuvõitlejate ja roheliste üritus. 
Sellega seoses tekkis meil ühte kontorisse, mis asus nüüdseks 
lõpetanud NO-teatri ruumides, üks PC, vist Sanyo. Selle küljes oli 1200boodine või -bpsine 
modem. Mingi koha peal lähevad boodid ja bpsid vist lahku\sidenote[][-3.6cm]{Bps (\emph{bits per second}) on sekundis edastatavate bittide hulk. \enquote{Boodid} (\emph{baud rate}) näitavad aga, mitu korda 
sekundis signaal muutub. Kuniks kasutatakse tavalist jadaporti, kus signaalil 
on kaks taset, on väärtused võrdsed. Keerulisemate skeemide korral võib ühe 
signaalimuutusega edastada rohkem kui ühe biti ning kiiruseühikud lahknevad.}. 

Meie ametlik tegevus oli suhtlus orgkomiteega ja selleks 
sai helistatud kaugekõnega Tallinnast Helsingisse. Eestis oli otsevalimine --- 
meil oli selles mõttes väga vinge positsioon, et mujal Baltikumis välismaa 
numbreid otse valida ei saanud. Ka Tallinnas ei olnud seda võimalust igal pool, aga meil oli, sest see 
oli ürituse jaoks oluline. Mart Siilman, endine Fila direktor\sidenote[][-3cm]{Eesti NSV Riiklik Filharmoonia\index{Eesti Riiklik 
Filharmoonia}, mille järeltulija on alates 1989. aastast Sihtasutus Eesti 
Kontsert. Tegu oli mõjuka asutusega, mille korraldada oli kogu Eesti
kontserdielu, sealhulgas levi- ja jazzmuusika ning estraad. Seega 
oli \enquote{Fila endine direktor} äärmiselt mõjukas inimene, kelle jaoks Soome 
otsevalimise korraldamine oli kindlasti võimalik.}, organiseeris, mida vaja. Kuidas, ei tea. 
Igatahes saime helistada Datapakki X.25 võrku, mille kaudu oli võimalik 
suhelda ühe Rootsi serveriga, teine server oli Kanadas. Sealtkaudu 
suhtlesime ürituse orgkomiteega, aga hakkasime ka vaatama, kuhu veel 
õnnestub helistada.

\question{Kuidas te seda \emph{bootstrap}'isite? Mida kliendi poolel vaja 
oli, et võrku saada?}

Tavalist modemit ja tavalise modemiga suhtlevat terminaliproge. 
Modemiga helistasime Datapaki liidestuspunkti, kust edasi läks asi 
pakettvõrguks või X.25ks. Terminali peal oli nagu ikka: lehekülg skrollib ja menüüst tuleb valida 
\enquote{üks, kaks, üksteist}. Lisaks oli meiliboks, kus sai kirju vahetada, ja
jututubade või listide alajaotus. Ja siis, parafraseerides Heinleini: 
\enquote{\emph{Have modem, will find BBSs}}\sidenote{Robert A. 
Heinleini 1958. aasta jutustus \enquote{Have Space Suit --- Will Travel}.}. 
Loomulikult leidsime üles ka selle, et on olemas BBSid. 1989. 
aasta lõpus tekkis Lembit Pirnil\index[ppl]{Pirn, Lembit} esimene 
PirnBoxi\index{PirnBox}-nimeline BBS, mis asus praeguse SEB taga, kus trammid toona kõva kriginaga keerasid, 
Autotranspordi Arvutuskeskuses\sidenote{Eesti NSV 
Autotranspordi Arvutuskeskus (ATAK).}. Nii et 
alguses helistasime kõik sinna Pirni BBSi sisse. 

Peagi tekkis HNS ehk \emph{Hackers Night 
System}\index{HNS}. Kolmas oli Goodwin BBS\index{Goodwin} meil 
Suloga\index[ppl]{Kallas, Sulo}, mis ilmselt jooksis sellesama 
väljahelistamise liini otsas. Öösel jätsime arvuti sisse ja kõik said 
sisse helistada. Kui tahtsid kuhugi sisse helistada, aga liinid olid kogu aeg 
kinni, siis ainuke võimalus olukorda parandada oli panna ise ka üks
\emph{box} püsti. 

Sealt tekkis siis ka Fido pool ja jällegi sissehelistamise küsimus --- 
kui meil oli võimalik e-post ja jututoad omavahel 
kuidagi sünkroniseerida eri masinates, siis polnud ju vahet, kuhu me sisse 
helistame. Masinad käivad päeval ja ööl ning vahetavad omavahel 
sõnumeid. Fido oli selles mõttes korralikult distributeeritud nett. Seesama, 
mille kohta nüüd öeldakse \enquote{veeb kolm}. 

\question{Võrgustike \emph{bootstrap}'imine on keeruline just inimeste 
mõttes. Selleks, et kuhugi külge minna, peaks seal olema huvitav, ning selleks omakorda 
peaksid seal olema inimesed. Mida te näiteks PirnBoxis huvitavat
tegite?}

Ilmselt lämisesime niisama. Pean tunnistama, et ei mäleta, 
aga väga huvitav oli igal juhul. Oletan, et kuskilt
pääses ligi faili kujul \emph{sci-fi} raamatutele ja 
laiematele uudisegruppidele, mis kuskil liidestusid 
Fidonetiga, nii et informatsiooni liikus. Lihtsalt kirjutada
oli ka huvitav, et vau, kõik liigubki traadi kaudu! 
See oli tollal nii \emph{amazing}. Sellest ma sain aru, et arvutiga saab
programmeerida ja midagi kujundada, aga et ka reaalselt suhelda!

\question{Mis tegi ühe BBSi populaarsemaks kui teise? Goodwin ja HNS 
olid pikka aega populaarsed, kuigi PirnBox oli esimene.}

See oli esimene jah, aga jooksis toona vähem 
levinud softi peal. Meil oli vist Maximus. 

Sulo oli omamoodi arvamusliider, kuna tal olid 
kõikvõimalike asjade suhtes väga toredad ja tugevad seisukohad. 
Mina olin niisuguse tutu-lutu taustaga, olles olnud muu hulgas 
Reaalkooli\index{Tallinna 2. Keskkool} viimane komsomolisekretär.
Arvestades et enne mind oli komsomolisekretär Karl Martin 
Sinijärv\index[ppl]{Sinijärv, Karl Martin}, siis me ilmselgelt ei võtnud seda 
asja väga tõsiselt.

Kuidagi me sattusime seda asja vedama, kuna meil oli tänu sellele 
tuumaüritusele ressurssi käes. Ühel hetkel tekkis meil igatahes kaks 
telefoniliini, võibolla aastake hiljem, kui üritus läbi sai ja 
olime juba Eesti Instituudi\index{Eesti Instituut} ruumides, veidi enne 
seda, kui Eesti Instituut osutus tegelikult Eesti välisesinduste ja iseseisvuse 
ettevalmistuslavaks. Näiteks kui kuulutati välja iseseisvus, 
tuli järsku välja, et Jüri Luigel\index[ppl]{Luik, Jüri} ja kõigil teistel, kes 
mööda maailma laiali olid, olid juhuslikult kaasas ka pruunid ümbrikud 
esitamiseks kohalikule võimupealikule küsimusega, kas teie ekstsellents 
lubaks meil asutada suursaatkonda. 

Eesti Instituudis olid meil ka oma arvutid, aga ma ei mäleta, kas meie enda või instituudi omad. Saatsime Suloga\index[ppl]{Kallas, Sulo} öösiti 
fakse. Mitu toredat kolleegi 
oli, vähemalt huumoriga pooleks, sügavalt veendunud, et faks ongi selline seade, et kui sinna 
peale panna paber koos kollase post-itiga, kus on telefoninumber, siis on see 
hommikuks ennast ära saatnud. Tollased liinid toimisid öösiti oluliselt paremini kui päeval. 

Tingituna sellest, et välisühendust oli meil läbi modemi helistades 
suhteliselt piiramatult käes ja liine oli ka mitu, siis oli meil kaks 
modemiühendust. Ühel hetkel hakkas meie ja Fidoneti kaudu väljapoole 
ühenduma Läti.

\question{Ma teadsin, et Vene Fidonet käis läbi meie, aga et ka Läti?}

Venemaa tekkis ka jah millalgi. Läti oli Fidonetis Eesti all, aga leedukad loomulikult ei oleks millegi 
selliseni laskunud, et nad on mingi Eesti regioon kusagil 
võrgustruktuuris. Nemad selle asemel helistasid kord nädalas ja tõid e-posti enam-vähem 
nagu ämbriga, välja arvatud Kaunase 
Ülikool\index{Kaunase Ülikool} ja Leedu parlament\index{Leedu Seim}, kes olid 
Goodwin BBSi pointid. Seal oli hädasti vaja ja uhkus jäeti kõrvale. 

Lätlased käisid meil külas ka. Panid raha kokku ja tõid selle meile 
ühenduse eest. Investeerisime need kakssada dollarit kahte modemisse --- ostsime US Roboticsi\index{US 
Robotics} HSTd, mille kiirus oli vist neliteist kilobaiti. Väga väärt aparaadid, nii et lätlased panustasid Eesti neti 
arengusse.

Samal ajal ametlikku postivahetust internetiga pidas 
Küberi\index{Küberneetika Instituut} seltskond, aga neil oli Mustamäel suhteliselt sant jaam, mis krabises ilmselt rohkem, kui sidet läbi 
lasi. Ühtlasi olid akadeemilised tüübid millegipärast suured
UNIXi sõbrad ja kasutasid PEPi TrailBlazereid\index{Telebit TrailBlazer}, 
mis esiteks olid 9,6 kilobaiti ehk mõttetult aeglased ja teiseks suutsid HSTd 
postsovetlike liinidega paremini sidet vilistada. Olime 
sügavalt veendunud, et need olid ka oma reaalselt võimekuselt pikki seansse ja 
kiirust üleval pidada märksa paremad. 

\question{Kas sa mõtled \enquote{liini} all ikka telefoniliini?}

Jah, need olid tavalised analoogliinid, mille otsa käis kettaga telefon. 
Keskjaamas numbrit valides jooksid releed kontakte mööda ringi. Kui modem valis, oli kuulda klõbinat, kui see
releega katkestusi tekitas. Kõik oli elektriline, sellepärast ma kujutangi 
ette, kuidas andmeside tegelikult toimub. Aga kuidas on võimalik, et mingid 
vennad panevad läbi ADSLi kümme megabaiti? Meie panime enam-vähem samasugust asjast 
läbi neljateistkümne kilobaidi, nii et täiesti arusaamatu. Wifi täpselt samuti. Ma ei 
kujuta ette, kuidas see põhimõtteliselt saab üldse toimida. 

\question{Kas kujundamisega tegelesid kõige selle kõrval?}

See käis jah kõrval. Umbes 
samal ajajärgul sattusin seltskonda, kellel oli arvuti ja printer ning vajadus 
midagi trükkida. See oli poistekoor\index{RAMi poistekoor}, mida juhtis Venno 
Laul\index[ppl]{Laul, Venno}\sidenote{Venno Laul asutas 1971. aastal Riikliku 
Akadeemilise Meeskoori juurde poistekoori ning oli kuni 1990. aastani selle 
kunstiline juht ja peadirigent.}. Neil oli esimene PostScripti printer, mille 
ostus ma osalesin --- kas Tektronix või muu säärane A4-formaadis 
300 dpi laserprinter. Ühendasin Ventura selle külge.

Põhimõtteliselt kõik toonased kujundusprogrammid olid sellised, et arvutis olid 
\emph{bitmap}-fondid, mis saadeti juhet pidi printerisse, ja see kõik võttis kaua 
aega. Aga PostScriptiga sai lehekülje nagu programmi saata 
printerisse, mis siis oma tarkusega joonistas. See oli Adobe ja Apple'i 
vendade poolt väga mõistlik valik, kui nad kord Silicon Valleys kokku said ja 
otsustasid, kes mis osa maailmast vallutama hakkab. Tõepoolest, kontoris ei pruugi 
igal vennal printerit olla ja selleks, et inimesed saaksid printida, 
võiks olla printer ka tark. Väga spetsiifiliselt tark, et suudaks 
joonistada lehekülje endal mälus valmis ja siis välja trükkida. 

Millalgi samal ajal puutusin kokku ka Sirbiga\index{Sirp} (ma ei tea, ka 
see toona oli juba Sirp või veel Sirp ja Vasar), mis oli üks esimesi 
ajakirjandusväljaandeid, mis läks üle digitaalsele töövoole. Alguses oli protsess umbes 
selline, et toimus tinaladu, millega tehti kas siis üks tõmme paberi peale ja 
see vist pildistati üles. Nii et ofsettrükk toimus 
veel läbi tinalao. Ja nüüd, kui oli võimalik minna üle sellele, et arvutist 
saaks välja trükkida, siis see oli megaraju.

\question{Kas PostScripti printerist lasti trükitavad 
asjad kilele?}

Põhimõtteliselt jah, peegelpildis. Üks asi, mille koos 
Suloga\index[ppl]{Kallas, Sulo} Eesti 
Instituudis\index{Eesti Instituut}tegime, oli PostScripti \emph{pre-header}, mille nimi oli vist Preambul. 
PostScripti puhul tuli programmiga kaasa 
koodijupp, mis kirjeldas programmeerimiskeskkonna, defineeris 
täiendavad funktsioonid ja muud oma käsud. Seejärel tuli kood ise ja lõpus 
koristusfunktsioon või midagi sellist, mis välja trükkis. PostScript oli tore selles 
mõttes, et see oli nagu \emph{open source}. Mitte küll vabatarkvara, aga 
nähtava lähtekoodiga. Ehk oligi võimalik võtta sama Ventura ette, 
mis kusagilt laadis selle Preambuli, mis oli tekstifail ja mida oli võimalik 
muuta. Ette sai kirjutada transformatsiooni, mis 
keeras pildi peeglisse. Meil õnnestus see Sulo PostScripti preambul maha müüa kooliraamatute kirjastusele
Avita\index{Avita kirjastus}, mille eesotsas oli Tiit Aunaste\index[ppl]{Aunaste, 
Tiit}. Hiljem, vist 1991. aastal sattusin ise ka Avitasse tööle, asjad liikusid tollal
väga kiiresti.

\question{Jah, sest umbes viis aastat hiljem mäletan mina sind Eesti 
vaieldamatu autoriteedina teemal, kuidas arvutist värviline asi trükki saada.}

Eks ma olin seda piisavalt praktiseerinud. Tegime Ventura peal ka 
haltuuraotsasid. Liikus muudki tarkvara, näiteks Arts \& Letters, millega oli võimalik paigutada tähti ringikujuliselt. 
Toona asutasid kõik aktsiaseltse ja börse ning 
neil oli vaja pitsateid. See oli meeletu innovatsioon, et oli võimalik 
arvutist ühe matsuga pitsat valmis teha ja ei pidanudki 
kujundajatädi fotolao tähti välja lõikama ja kleepima. 

Sirbi toimetus andis välja Jutulehte\sidenote{Ilmus AS Kodamu väljaandel 
aastatel 1990--1992.}, mille \emph{layout}'i tegin mina. 

Nii ma tasapisi sattusingi selle ala peale. Käisin ka Helsingis vaatamas, 
kuidas Helsingin Sanomati\index{Helsingin Sanomat} tehakse. Neil olid 
miniarvutid ja rohelised terminalid ning 
Linotronic\sidenote{Mergenthaler Linotype Company toodetud 
kõrgekvaliteediline printer. Tegu oli kalli seadmega, mis võimaldas trükkida 
resolutsioonis kuni 2540 dpi.}, millega trükiti veerge 
fotopaberile. Fotopaber oli 30 cm lai ja sellele lasti välja üks 
ajaleheveerg. Siis lõigati veerud kääridega välja ja pandi suure 
maketi peale, mis oli vaha või millegi säärasega koos. Rastreeritud fotod 
pandi veergude vahele, leht sai kokku ja tulemus saadeti faksiga 
trükikotta. Faks ei olnud loomulikult tavaline faks, vaid 
\emph{industrial-grade} ajaleheformaadis kõrgeresolutsiooniline masin, mis 
skännis ühelt poolt sisse ja teiselt poolt trükkis välja 
filmi, millega valgustati trükiplaadid. 

\question{Mis selle juures köitis? Kas tehnoloogiline keerukus või 
see, et protsessil oli palju samme, või veel midagi?}

Kõige huvitavam on tegeleda asjadega, millega teised parasjagu ei tegele. Või 
ka asjadega, mis toimivad teistmoodi, kui olen 
siiamaani arvanud. Niisama Pascalis programmeerida ei olnud väga huvitav, seda õpetati 
koolis. Aga kuna mul oli ilmselt olemas arusaam, kuidas asjad töötavad 
ja mis masinates on, siis ma suutsin asju efektiivsemalt tööle panna. Näiteks kirjastuses siduda küljendus- ja 
tekstitöötlusprogramm. Tekstitöötluses oli levinud WordPerfect (Perfect? 
Prefect? Ford Prefect ja Word Perfect!\sidenote{Peeter viitab tõenäoliselt 
Douglas Adamsi loodud tegelaskujule, mitte omaaegsele populaarsele 
automargile.}), meie küll kasutasime rohkem Volkswriterit\sidenote{Volkswriter oli üks esimese PC-platvormi tekstitöötlusprogramme, 
mida arendati 1980ndatel. 
Volkswriter oli saadaval ka eespool mainitud GEM-platvormile, sellest ilmselt 
selle kasutamine kirjastustöös.}. Tekstitoimetaja toimetas 
WordPerfecti faili, kus olid stiilid juba ära märgendatud. Küljendaja luges
selle oma küljendusprogrammi tagasi ja teksti korrektuuri oli võimalik teha ilma kallima arvuti või 
keerulisema programmita. Seda õnnestuski meil väga efektiivselt juurutada. 

Enne rublaaja lõppu, 1992. aasta alguses, tekkis Prisma 
Printi\index{Prisma Print} esimene Linotronic ehk filmiprinter. Seniajani
peeti 600 dpi laserit väga heaks, nüüd tekkis järsku 1200 dpi 
filmiprinter. Prisma alumisel korrusel olid suured Crosfieldi\sidenote{Crosfield Electronics oli Briti firma, mille toodetud 
skännereid peetakse siiani ühtedeks paremateks, mis iial tehtud.} trummelskännerid, millega sai 
teha värvilahutusi, filmi peale, juba rastrisse. Ja kogu montaaž 
toimus endiselt nii, et tekstikile ja pildikile või film 
valgustati või lõigati füüsiliselt kuidagi kokku. 

Mul oli keskmisest parem ettekujutus, kuidas need süsteemid 
töötavad. Kui mina jõudsin oma failidega kohale, nägid enamasti kõik vaeva, kuidas QuarkXPressist midagi välja printida. Minul olid kaasas oma flopid ja hiljem 
magnetoptilised ja ma trügisin vahele, et 
minu omad vahepeal välja lastaks, kuna ma ei viitsinud teiste järel oodata. Ja lastigi, sest minu asjad käisid tõesti kiiresti läbi, kuna ma 
kujundust tehes kujutasin ette, kuidas see PostScriptiks läheb. Nii 
küljendusprogramm kui ka seesama Ventura või graafikaprogrammid, nagu 
Illustrator või Freehand, aitasid mul visuaalselt 
valmistada ette PostScripti. Teadsin, kuidas see koodis 
välja näeb, ning sain võtta faili ette ja näha, kus miski on. Tänu 
sellele teadsin ka, mis on printeri jaoks keerulised asjad, 
oskasin neid lihtsustada ja mitte liiga keerulisi asju kasutada, sest 
see prose, mis seal taga oli, oli suhteliselt vaene. Kui suudad 
tekitada olukorra, kus programmil on tsükkel tsüklis (tänapäeval tuleb 
sinna otsa veel SQLi päring), siis üldiselt on see asi suhteliselt ebapädev. 

Tänu arvutitaustale ja kujundamisele tekkis mul hulk disaineritest sõpru ja teisi sel alal tegutsejaid, keda ma 
Prisma Prindi väljatrükijärjekorrast teadsin. 

Edasi sattusin tööle Uniprinti. Algul töötasin andetu disainerina, aga 
siis leidsin tasapisi võimalusi vähem disaini nõudvaid asju 
teha, kus mängis rolli just see, et ma suutsin võtta ette näiteks Eesti Näituste 
näitusekataloogi andmebaasi (see oli vist Microsoft Accessis) ja 
genereerida väljundiks tekstifaili, mis oli stiilidega märgendatud ja mida 
oli võimalik küljendusse sisse lugeda. Jällegi asi, mida tollal ei olnud 
\emph{desktop publishing}'is tavaks kasutada: valmistasin 
stiilid ette ja tekst kasutas neid, kohapeal midagi tegema 
ei pidanud.

\question{Nii et põhimõtteliselt oli tegu CSSiga?}

Täpselt. Põhipõhimõtteliselt nagu CSS, ainult et tekst ja paber. Veeb töötab siiamaani niimoodi, aga see oli minu meetod, mis võimaldas teha huvitavaid 
töövoogusid. Minul oli andmebaas käes, näitusetüdrukud, kes müüsid bokse ja korraldasid üritusi, tõstsid asju 
ümber, täiendasid firmade andmeid ja parandasid telefoninumbreid. Mina trükkisin \emph{layout}'i välja, viisin neile 
ja nemad tegid andmebaasi korrektuuri, samal ajal kui mina joonistasin logosid 
puhtaks. Niimoodi ma õppisin. Nagu ma ütlesin, siis ma olen andetu 
disainer, aga tehniliste protsesside ja töökorralduse poolt 
teadsin tollal rohkem kui keegi teine. 

\question{See klapib kenasti sellega, mida sa praegu tundud tegevat. 
Millega sa üldse tegeled?}

Jah, seoseid on igasuguseid. Kui ma veel trükialal tegutsesin, oli mul palju vaba aega tänu sellele, et olin suutnud oma 
tööd optimeerida. Ja ka tänu sellele, et Uniprindi pealikud Sirje ja Andrus 
Reinsoo\index[ppl]{Reinsoo, Sirje}\index[ppl]{Reinsoo, Andrus}, kes on just 
mõlemad lahkunud\sidenote[][-2mm]{Intervjuu Peetriga leidis aset märtsis 2019.}, jätsid 
mulle piisavalt vabadust. Käisin ja kolasin Ameerikas 
konverentsidel. Sel ajal oli valdav suhtumine, et mis 
mõttes väljamaa ja konverentsid? Me oleme Eestist ja teame väga hästi. Mina 
käisin \emph{cyber publishing} seminaridel, mis olid 
seotud just trükipoolega, mis mulle huvi pakkus: plaaditrükk ja 
muu selline. Ja kuna ma olin põhimõtteliselt nagu ajakirjanik, siis oli
mul võimalik möllida ennast konverentsidele, mis muidu maksid 
paar tuhat taala (tolle aja mõistes röögatult palju), 
ajakirjaniku passiga sisse. 

Aastast 1994 hakkasin ka kirjutama. Paralleelklassivend Peeter-Eerik 
Ots\index[ppl]{Ots, Peeter-Eerik} oli Äripäevas ajakirjanik ja kirjutas 
tehnoloogiateemalisi lugusid. Mul reaalikana hakkas 
mõnevõrra piinlik, sest kirjutatu ei tundunud olema piisavalt pädev. Post-BBSi ajastu inimesena olin kindlasti ka võrdlemisi 
\emph{opinionated} noor inimene oma kindlate 
eelarvamustega. Kirjutasin teisele Peetrile paar lugu ette, et avalda parem 
neid, vähemasti kirjutatud kellegi poolt, kes enam-vähem saab aru, millega 
tegu. Peeter ütles, et võiks need ikka minu nime 
alt avaldada ja saaksin honorari ka. Nii sattusingi Äripäeva 
kirjutama, hiljem ka mujale, näiteks
Arvutimaailma\index{Arvutimaailm}. 

\question{Tehnoloogia tehnoloogiaks, aga mis sind kirjutamise juurde tõmbas?}

Kirjandite kirjutamisega sain suhteliselt hästi hakkama juba kooliajal. Minu 
esimene avalikustatud töö oli Pikri\index{Pikker} noorte 
huumorivõistluse võidutöö. Olin üht-teist ka lugenud, sõprus 
sõnaga oli olemas, lisaks olin teinud kooli omavalitsust ja muud 
sellist. Ilmselt olin parasjagu jutukas ka. Kirjutamine ei 
olnud keeruline ja võibolla meeldis mulle ka õpetada --- läbi kirjutamise on 
võimalik teisi õpetada ja panna midagi teisiti tegema. Olen Peeter-Eerik Otsale väga tänulik, et 
ta tegi midagi valesti. See on ka interneti puhul tüüpiline: 
\enquote{\emph{Wait, somebody is just wrong on the Internet!}}. Kirjutamine 
ilmselt sai alguse sellest, et \emph{somebody was wrong} ja mul oli vaja 
kaitsta oma seisukohta ning loomulikult ka Reaalkooli au. 

Sedasi see algas ja hiljem palusid ka teised mul kirjutada. 
Ma siiamaani ei oska ei öelda ja eks edevus mängis ka rolli. 
Pealegi oli see valdkond suuresti katmata. 1995. või 1996. aastal kutsus Avo Raup\index[ppl]{Raup, 
Avo} mind kui juba kirjutanud ja tuntud inimest Raadio 2 saatesse \enquote{Võrgutaja} külaliseks. Meil klappis nii hästi, et minust sai resident-saatekülaline. Esimene inimene, keda 
ma sattusin saates üksinda intervjueerima (Avo oli vist haige), oli Kaido 
Saarma\index[ppl]{Saarma, Kaido} Abobase Systemsist\index{Abobase Systems}. 

1999. aastal tuli minu juurde 
Sarvik\index[ppl]{Sarvik|see{Sarv, Henn}}\sidenote{Legendaarne IT-mees Henn 
Sarv\index[ppl]{Sarv, Henn}.} ja ütles, et Kukust kas Lang või Tiido oli öelnud, et on vaja teha arvutisaadet. Istusime sealsamas Uniprindi 
lähedal Pärnu maanteel Westmani poe vastas keldris Hollandi 
õlletoas ja mõtlesime välja saate Tehnokratt. Juba esimesel hooajal sattusime Kukus kokku 
tegelastega, kellel oli mõte ka ETVs\index{Eesti 
Rahvusringhääling!Eesti Televisioon} midagi sellist toota. \emph{Whatever}, toodame! Nii sattusingi 
telesaatesse, kus pidin olema korraga toimetaja ja saatejuht ning panema kokku ka montaažiriba (mis tuli muidugi mõnevõrra üllatusena).

\question{Ja nüüd oled ringiga tagasi\ldots} 

Kas nüüd tagasi või edasi, aga praegu olen Zone'is\index{Zone}, mis on täiesti 
juhuslikult ajaloos esimene kord, kui töötan mingit otsa pidi 
IT-firmas. Olen küll vahepeal olnud reklaamiagentuuris digitiimi juht, 
mis on ka natuke IT, aga ikkagi reklaamindus. Nii et olen töötanud trükinduses, teisi koolitanud ja kõike muud teinud, 
aga see on esimene kord, kui mingid IT-tüübid mõtlesid, et palkaks Marveti siia 
tuututama. Ametlikult on mu müts seotud turunduse ja kommunikatsiooniga, aga 
tegelen ka sellega, et kui keegi ütleb, et midagi ei tööta, ja kõik 
väidavad, et töötab ju, siis kuidas saada aru, mida inimene 
tegelikult tahab. Äkki tal on õigus, et tal ei tööta. Äkki on võimalik, 
et see asi, mida meie oleme nunnutanud ja silunud ja teinud maailma kõige 
paremaks, tema kontekstis ei tööta. Ja täiesti üllatavalt selgub, 
et kui on piisavalt keerulised süsteemid, siis olukordi, kus tuleks 
kõige suurematele ja parematele püüdlustele vaatamata midagi 
teisiti toimima panna, on uskumatult 
palju. 

\question{Küll sa turunduse ka ära optimeerid, nagu sa kõik asjad 
ära oled optimeerinud!}

Jah, ma üritan. Mul see lootus on natuke teistpidine. 

Kunagi tuli Andres Kulli\index[ppl]{Kull, Andres} ja Kroonpressi\index{Kroonpress} 
seltskond küsima, kuidas panna reklaami ajalehte. Mina rääkisin, et on 
olemas PDF. Teeme parem nii, et kõik teeksid korraliku PDFi, leheküljendaja 
tõstab selle küljendussofti sisse ja kõik töötab. Kull, ikkagi
suure trükikoja juht, ütles seepeale: \enquote{Väga hea, nii teemegi. 
Kõik peavad saatma oma asjad PDFina Postimehesse}. Ja üllatus-üllatus, nii läkski. 

Mu enda roll selle kõige juures oli, et olin olnud pikka aega Prisma Prindis ja 
muudes reprodes selline majasõber, kes sageli tolknes seal ja üritas endale 
tegevust leida ning saada aru, kuidas asjad käivad. Näiteks võtsime
Eesti esimese Linotronicu pulkadeks lahti ja jootsime seal midagi, sest masin otsustas töö lõpetada parasjagu, kui oli vaja midagi välja 
lasta. 

Teadsin, millist roppu vaeva kõik mu repropealikest või -tehnikutest sõbrad olid näinud kehvasti ette valmistatud 
originaalidega. Kui PDFindus hakkas meile endale majja tulema, siis 
mõtlesin, et mina küll ei hakka selle ussipurgi avamist enda peale võtma
(tänapäeval räägitakse rohkem \emph{surströmming}'ust kui Pandora laekast). Ainuke 
asi, mida ma saan teha, on õpetada kliendid paremaid originaale saatma, mis 
loomulikult tundus äärmiselt lihtne:
ütlen neile, et seal on vaja mõned linnukesed panna ja siis kõik 
lähebki nii, nagu vaja. Aga tuleb välja, et ei. Olen õppinud, et päris 
kõva pingutus on aru saada, mida teised inimesed teavad, ja 
panna nemadki aru saama millestki, millest mina aru saan, 
seejuures ise liigselt masendumata või 
nende peale kurjaks saamata. Nii sattusingi õpetama Pagemakerit, 
InDesigni, Photoshopi ja muud säärast just töökorralduse poole 
pealt. Hetkel Zone'is näen ma, et kui vaadata kogu veebiga 
seonduvat, siis ilmselt tuleb proovida selle kõigega veel rohkem edasi minna. 

\chapter{Andres Peiker}
\index[ppl]{Peiker, Andres}

\question{Kuidas ja umbes millal sa jõudsid arvutite juurde?}
Äkki äkki äkki?
Tere See siin on Memm copy ammu enne seda, kui Silicon Valley maailmale näitas, kuidas suuri infosüsteeme skaneeritakse oli Eestis seltskond inimesi, kelle hallatav infosüsteem kahekordistas oma ärimahte iga üheksa kuu tagant ja tegi nii kümme aastat järjest. See kõik oli väga suuresti meie tänaseid külalisi. Teine külas on Andres Peiker. Head kuulamist. Memm labi.
Tere.
Sinu nimi on Andres Raid. Raili. Alustame asjade algusest, sealt, kust kõik asjad on pihta hakanud. Kuidas sinu arvutite juurde said?
Ta oli mingi kaheksakümne neljas aastake ja osta, et.
Suht juhuslikult tegelikult selles mõttes, et ma õppisin siis keskkoolis ja ma käisin mingisugustel füüsika loengutel Tartu Ülikoolis ja ja ühe tolle loengu lõpus mees nimega Otto talle asendus auditooriumite ütles, et, et aga, et kes on arvutitest huvitatud, et võivad natukene siia veel jääda. Ja noh, siis mingisugune seltskond jäi, Otto Teller viis meid siis seal tähe neli olevas õppehoones. Seal oli kaks maili arvutit ma ka ja naine kaks vist olid nood viis näitas nõid kes ütles, et noh, et põhimõtteliselt siin nagu mingitel õhtustel aegadel on või on võimalik käia nagu programmeerida proovida asju.
Millest, millest ma kohe järeldad, sa Tartu boss absoluutselt esimesed kakskümmend viis eluaastat vanad ja, ja siis, millest ma järeldan, on, on see, et kui sa keskkooli ajal kuskil Tartu Ülikoolis mingites loengutes kestis, siis kui pidi olema mingi matemaatika, kui reaalainete või niisugune huvi
No ma õppisin Tartu esimesest keskkoolist tuli matemaatika-füüsika eriklass.
Noh, ma käisin olümpiaadidel või ei mäleta täpselt, kui Kustu ülikooli loengute teema üldse tuli, et noh, füüsikas tüli ja füüsika tundus mulle nagu kõige põnevam asi üldse, et et siis siis saigi seal käidud.
Ja seal fäski loengu lõpus. Mind paneb imestama, et keegi üldse nagu ära, eks, et selles mõttes, et kõik sõnavahed ja rahvas tundus nagu arvutite vastu huvi tundnud või siis nii olnud.
Ei, ikka ei olnud selles mõttes, et ma arvan, et ikkagi pooled läksid ära ja ja ta on grupp jäi tegelikult noh, esimesel korral käisime neid arvuteid vaatamas, siis öelda, et noh, et järgmine kord saaks nagu sel päeval tulla, siis tuli juba vähem inimesi ja lõpuks jäi mingisugune, ma arvan, mingi kolm-neli inimest võib-olla alles, kes on nagu rohkem käi käima hakkasid.
No see oli mingisugune ring või lihtsalt.
Ma enam nii täpselt ei mäleta, et, et ma arvan, et Otto Teller ikkagi seal natukene juhendas ka alguses, et, et mis, mis ja kuidas, et kuidagi me tolle AB programmeerimiskeelega, mis on mägede peal, oli kuidagi me tuttavaks saime, et ma arvan, et läbi läbi talle tema vahet
Mingeid raamatuid niisugust kraami.
Ei, seda ma küll ei mäleta, et oleks olnud.
See on huvitav asi. Tänapäeval me pöörame palju tähelepanu selleks, et õpetada inimesi programmeerima ja see on täiesti läbivalt, mitte keegi suuda meenutada, kurjasid, õpsid, programmeerib, kuidagi sündis täiesti lihtsalt tuli. Aga mida te tegite sinna nairidega?
Noh, see on, seal sai ikkagi teha väga lihtsaid mingisuguseid arvutust või samme selles mõttes, et tollel arvutil ju koha pealt tuli elektroonilise kirjutusmasinaga, nii et sa kirjutasid programmi ja ta tuli paberi peale ja ta oli ainukene eksemplar tollest programmist, mida, mida sa pidid siis alles hoidma tuleb, eks needki saab parandada, tahtsid, siis sa pidid vaatama today prinditud paberit. Et et noh, kõige kõvem asi, mille ma seal valmis tegin, olis dollareid, tähtsad asjad, biorütmid, et noid Arvutuskeskuses tehti ja, ja siis ma tegin, talle meeldib ka peale tegin ka biorütmide programmi, et.
Et ma saan auto, osutus seal populaarseks, et tollest minu perfolingvist keegi tegi koopia ja siis lasti toda seal talle Tähe neli töötajatele usinasti välja, ilma et ma midagi tean.
Biorütmide arvutamine ja see oli kuidagi naljakas, sest tõrvist need algoritmid, mis liikusid, olid vist mõeldud käsitsi arutamiseks ja seal olid minu meelest kuidagi arvutati Siimust.
Kui ta skandaali Taligi lihtne siinus selles mõttes, et lihtsalt sa pettur sünni sünniaja ütlema ja siis tolle elatud päevade arvu pealt noh, tosiinusega ka häbilainepikkus oli lihtsalt nendel emotsionaalne ja füüsiline ja seksuaalne, mis neljas oli, ma ei mäletagi, oli, oli lihtsalt erinev mustajoonistest olla neli siinus sisuliselt välja, tegelikult et noh, täiesti triviaalne asi iseenesest taheti, rohkem oligi too, et noh, kuidas paberi peale toda sinust joonistab, et elektroonilise trükimasinaga
Jah huvitav asi, mis oli nagu oluline meil tänaseks täiesti varaga biorütme hakata joonistama inimestele jumalaid.
Tuli siis mingisugune väga popp asi ja ja tundus tolle arvuti jaoks nagu niisugune jõukohane ülesanne pärast, et ma arvan, et see oli mingi, kas neli kilobaiti oli, oli mälu tollele arutelule ja noh, ta oli sama suur kui mul, ma ei tea, kodus köögimööbel.
Et selline suur asi, milleks ta tänu füüsikud kasutasid
Samamoodi kasutamiseks.
Aga mida mingid?
Ei tea sellest sellest nagu ei olnud juttu selles mõttes, et on ai kaali nagu väiksem masin teises toas oli null kaks. Põhiliselt ma saan aru, kasuti Duda meie sinna masinale nagu eriti pidi ei saanud. Et oli nagu rohkem hõivatud ja Toljad vaatasin ka selles mõttes, et lindiseadmed need suured lindikapid, kus siis seda magnetlinti keerutate ja tol ajal ei olnud mitte see tavaline elektrooniline kirjutusmasinaid, tol oli niisugune trummeliga printer. Noh, mis suutis ikkagi toda paberit nagu päris kiiresti välja lasta, et.
Printimise tehnoloogiaid rublale kuidagi väga oluline ka absoluutselt. Kas sealt see pusimine oli lihtsalt puhtalt niisugune nõu põne, arvutiga möllamine või seal mingisugune nagu sügavam asi ka tagavad, sulle tundus, et kuidagi, et seal mingisugune asi, mis seda teha tahan.
Siis tuli puhtalt seotud tegelikult sellega, et et kuna keskma olin matemaatika-füüsika eriklassis, siis meil oli seal Andres Jaeger, ülikoolist andis, andis programmeerimist ka kolm aastat, aga ta programmeerimine oli sisuliselt ainult mingite blokk skeemide joonistamine paberi peale, ühesõnaga noh, me arvuti ligi ei saanud ja, ja noh, siis too Tähe tänaval oli võimalus nagu ise järele proovida, siis seda, mida sa olid tegelikult nagu paberi peal teinud. See, et toda nagu tavakooli õppeprogramm ei võimaldanud kooliprogrammile joonestuspaberi peale, algas.
See paberi peal paks geimid joonistamine, see võis ju huvi ära tappa, aga sul või tapnud seal millegipärast läksid. Pärast seda loengut jäid sinna ja millega see käsitsi Emmy eid.
Ei, ta ei tapnud kindlasti tolle pärast, et, et ka too blokk, skeemide joonistamine, et näed, kui sa talle ülesande said, noh siis siis ta siis ta ütles ka, et umbes, et noh, et kes suudab nagu mingisuguse
Kolme Ifiga teha, et on, on hea kaheffiga on väga hea, et noh, et ühega noh ma ei tea, ja noh, siis siis sul oli nagu eesmärk olemas, et sa pidid ühe väga tegema talle tulemast, et noh, too asi kõnetas mind ja noh, siis sa said õpetaja käest kiita ka murde on tõesti, et noh, et ma mäletasin, võib-olla viis aastat tagasi oli meil ka üks õpilane, kes suutis nagu selle algoritmi nüüd selliselt ära teha, et ja väga hea
Ühesõnaga sinu jaoks ajalise oskab joonistada niisugune nagu ülesanne, kui naguniisugune pusle või.
Jah, absoluutselt selles mõttes, et noh, et ikkagi ütleme, et toob üles loomulikult jõuga, sa suudad tolle mängu lihtsalt ära teha, aga et noh, et nüüd, kuidas ta nagu kõige optimaalsem saaks kõige parem. Et, et noh, tooli huvita.
Okei ma ütlen, et, et selle koolikoolid olla ühtegi nagu arvutitel on Tartu linna peal ju arvuteid oli nagu küll tootjad.
Meie koolis ei olnud, ei olnud, siis ei olnud ikka mingisuguseid arvuteid kuskil selles mõttes, et.
Tähe tänaval oli, olid kunud kaks naid. Loomulikult seal ülikooli arvutuskeskuses oli jeeess. Lahkus, kes üldse nagu ise arvuti ligi saanud operaatorid lõid programmidesse. Ja siis oli füüsika instituudis Riia maantee lõpus, seal oli ka mine niisugune.
Pidipi üksteist äkki.
Ja ma arvan, et umbes tollel ajal kuskil Anne Villems sebis need Apple kahed ka tegelikult siis Vanemuise tänavale uhke, et noh, tooli nagu.
Asi, kuhu ma, kuhu me järgmisena jõudsin peale vaid naistega sõid? Ma arvan, et tuli ka Otto Teller, kes meid sinna viis ja ma isegi tundsin nagu pärast natukene piinlikkust, et, et tema näitas, no täplid. Ja siis ma tegelikult hülgas endale Tähe tänava ja ei käinud enam tema juures, ainult vahtisin seal Apple'it ei oleks käinud sellepärast et ta oli palju nagu ägeda rahast tol ajal oli ikkagi monitor ja, ja tal oli nelikümmend kaheksa kilogrammi sinu elu ja ja ta oli ikkagi nagu ulmeliselt kiire.
Päris päris asju teha jah. Kes mängimine kaaslasi papi peal tuli ülesse teemaks või?
Ja, ja absoluutselt tooli tooli päris hull selles mõttes, et too oli kindlasti minu elu kõige suurem arvutimänguperiood. Et ma oleks peaaegu kahe klassivennaga keemiaeksamile hiljaks jäänud, tavapärast, et noh siis salvestada seisu ei saanud, sa pidid lihtsalt nii kaugele mängima, kus said ja juhuslikult juhtus ta hästi nii minema, et oleks pidanud juba minema, aga tuli järgmine level ja pidid edasi hängima.
Apple'i peal sihuke standartne mäng on nagu Batman mis on mänguautomaati siis igal pool Apple'i peal nimetatud super Pukmeriks. Ja tuli ma siiamaani pean teda kõige lahedamaks mänguks, mida ma olen kunagi mänginud tolle pärast, et toda Pakmenit oli kõigi teiste arvutite peal ka. Aga, aga seal oli nagu mingi katastroof, haalne erinevust olles Algo Hitmis, kuidas tund neli kolli liikusid tolle pärast, et kõigi ülejäänud arvutite peal, nii palju kui mina olen mänginud liikusid rändumiga. Aga Apple'i peal oli neil oma kindel algoritm. Ja tulemuseks oli see, et kui sa ise tegid täpselt ühtemoodi siis tus, situatsioon kordusmänguks mängu ja me tegelikult tööd meil olid välja töötatud esimese kuue leveli jaoks tegelikult sisuliselt algusest lõpuni, et sa teadsid täpselt, kuidas sa terve tolle ekraani puhtaks mängisid ja jäime sellele noh, sealt edasi oli, oli sisuliselt mingisugused paar avangut, mida sai erinevatel serveritel kasutada.
Kuidas toimub põhimõttest Bäckman kui male?
Natukene natukene, absoluutselt selles mõttes ja, ja noh, selle tõttu olnud võimalik teiste peal mängida sellepärast et nad lihtsalt rändama ka liikusid. Et jah, arvutimängud jah, absoluutselt, no seal oli teisi teisi veel, aga super Uponor'i kindlasti.
Ja ja on programmeerimine.
Nojah, muidugi selles mõttes, et seal teisik. Jaa noh, ütleme, et ma alguses ma kirjutasin ikkagi teisikus, aga, aga pärast pärast sai ikkagi valdavalt Assembler kirjutatud tolle pärast, et noh, programm teeb su oluliselt kiiremini kui Assad, et noh, see on tõesti olnud.
No kuidas teisikustessembrisse hüppamine käis, sest Peisikust ma saan aru, et seal see käsud on inglise keeles, eks ole see korrektsele luks lihtsasti üles. Aga selleks sa pead teadma ikkagi väga täpselt, mida sa teed ja miks sa nii töötavad. Harilik protsessor, arhitektuurist ja nii edasi.
Noh, kapeisiku puhul sa pidid ikkagi tollest arvutiarhitektuurist aru saama, et,
Et noh, kust olles neljakümne kaheksas kilobaitides nüüd paiknes ekraan tekstiekraan, kus paiknes graafiline ekraan, kus paiknes see programm, kus oli opsüsteem et, et tegelikult talle arvutiarhitektuuris arusaamine tekkis tulle peesiku kõrvalt ka suhteliselt kiiresti.
Ja aga, aga noh, too Assembler tuli ikkagi tänu sellele, et et osad asjad olid väga aeglased. Et üks asi, mida ma seal tegin, oli orienteerumisneljapäevakute protokollid.
Tollega alustas tegelikult Peep Paabel, kes rakendusmatemaatikat ülikoolis õppida, aga ta lõpetas ülikooli ja siis ta andis mulle tolle kogudule programmi komplekti üle aga tuli minu jaoks liiga aeglane. Andmemaht oli tol ajal neljakümne kaheksa kilomeetri jaoks natuke liiga suur, seal oli ka mitmeflopiga mängimist, et, et need andmebaasid ära mahuksid ja mõtlesin, et teen, ma kirjutan from spetsiifika sõbraks. Ja siis ma kirjutasingi kõik Assembler start oli kõik palju kiirem vahe.
Ja noh, peale toda sai veel siis, siis sai kogu too opsüsteem tegelikult Ribes Insineeritud, tissassembleerituda sellega, et kogu too täna kommenteeritud siis sai imestatud, kus päris mitmes kohas Steve Wozniak oli hämmastavaid trikke teinud talle talle opsüsteemi kirjutamise juures, et nagu noh, tolleks ajaks kui hakkasin Tizzasolveerima, siis siis ise ka juba arvasid nõutud Assembler tean nagu väga hästi. Aga siis ikkagi paar sellist asja, mida avastasid, et Vao, kuidas teha saab, et et noh, nagu niisugune pisut Hipp Trek tegelikult. Et noh, et Assembler siis lihtsalt olid ühe poidilised, kahe maitside, kolme pallised käsud.
Ja see trikk oli see, et et ühte kolme potist käsk oli võimalik kasutada siis selliselt, et kui sul programmi oksis otsele, niisiis ta kolme baitine käske ei teinud midagi, aga sa said tolle kolme baidi viimast kahte potti kasutada selliselt, et saab kuskilt eespoolt hüppasid tolle teise baidi peale, mis oli siis teine Command, et sa sinna kolme baidise käsu viimasesse kahte Balti paigutasid tegelikult teise Assembleri käsu.
Et siis üks elegantsed tekkele tehtud ja siis pärast püüdsid ise ka nagu mõnes kohas mõelda, et kas ma saadada nagu efektiivselt kasutada.
Kust, kust see teadmine tuli, need nii teha saab, et vett saab, lisab, et saab lahti võtta selle noh, mingi teadmine pidi kuskilt jällegi lekkima, eks.
Ei, no kui me tolle koodina kudissassembleerisime, noh siis, siis sa pidid kogu tollest algoritmist arusaamad, mis, mismoodi ühtset töötab. No tegelikult oli, ta on opsust noh, mitte nüüd bioss selles mõttes, vaid vaid opsüsteem, et kettaga suhtluskett nagu suhteliselt aeglane. Ma tahtsin seda kiiremaks saada, mu püütud, siit taim oli seal üks asi, mis, mis välja mängisid lõppes sellega, et ma tegelikult kirjutasin Assembler ise nagu ketaste kopeerimise programmi, mis töötas siis nagu tollest opsüsteemist nagu mingi kümme korda kiiremini. Kümme kurd, noh sa pidid arvestab optimeerima, lihtsalt tuleb pealiikumised, kui sa tahtsid kogu ketta ära kopeerida, siis, siis too sa pidid, ma ei mäleta, kas seestpoolt väljapoole või väljaspoolt sissepoole sõitma, selleks et siis ta siis ta tegi ühe liikumisega kui kirjutamise ära, mitte ei käinud edasi-tagasi. Et muidu standardselt käski edasi-tagasi alatia.
Jällegi, et kust sa üldse niisugune arusaam, et kettaseadmega saab nii üksi trikke teha, et nagu võiks ette võtta siukse asja seda teadmist ja julgust, niisugust pealehakkamist, pluss natuke ekstravagantse ka, et nagu, mis see vastus ikka teab, kuidas ketast kopeerida?
Noh, noh, too opsüsteem on ikkagi universaalne tehtud, et, et selles mõttes see ketta kopeerimist programm sai tehtud nagu Detiteedid siis optimeerituna mingisuguse konkreetse asjaaegselt noh tol ajal oli oluline, et noh, et mingisugune kuskilt kas või Moskvast mingisugune tüüp tuli, tal oli mingeid kettaid, kassi pidid kiiresti suutma kopeelemmitse sai Jokute seal nagu tund aega kopeerida, vaid et sa saad nagu kiirelt endale ära tõmmata näoga Süüria rohkme munade, kust, kust sa neid programme saab Internetti jõudnud, et too liikus ikkagi nagu ma käisin isegi tegelikult koos ühe klassivennaga korra Moskvas puhtalt sellepärast, et et mingisuguseid arvutimänge saada Su.
Moskva suurlinn, kus huviline
Ei, lihtsalt vend käis ise Tartu Ülikoolis ja, ja me saime ta ju kokku, tal oli mingit rõmmyndokupeedisinud näha ja siis me saime temaga kontakti ja siis ta ütles, et noh, et ta, et umbes, et Moskvasse siis alati, et very welcome. Ja, ja siin me nüüd lihtsalt läksimegi.
On ju rongiga?
Kuid mitte, mis too andmeside kiirus siis tuleb, kui arvestada, et sa sõidad rongiga sinna, siis kupeedia flopid ära jäetud ära, siis see tuleb.
Ma ei julge öelda, mis kolmsada kuuskümmend kilobaiti oli üks ketas või? Nii-öelda päris päris soolikaid, ütles, et, aga, aga eks ta oli vast kõige kiirem viis ikkagi.
Et.
Iimil tuli ikkagi mingisugused aastat hiljem ja too käis ikkagi kord päevas, helistasid modemiga sisse ja tõmbasid meilida.
Aga ühel hetkel sai keskkool otsa siis oleks siit õppima midagi.
Tartu Ülikooli saatikat sõjaväkke võtma. Õnnestus ära viilida, noh, selge. Et Tartu Ülikooli rakendusmatemaatikat, aga tollest õppimisest tegelikult palju välja ei tulnud, tuleb jah, ma istusin ikkagi see lätete juures edasi, nii, nii nagu.
Kooli ajal, et.
Jah esimese kursuse ma.
Tegelikult tegin ära kõik matemaatikaeksamid olid viied, aga aga inglise keele ka kukkusin välja. Kuna hõbemedaliga lõpetasid siis siis sisseastumine uuesti väga lihtne pidi matemaatikaeksamitega, mis minule oli triviaalne. Aga, aga noh, siis ma enam ei viitsinud üldse loengutesse minna, sellepärast et noh, kõik matemaatika eksami tehtud, me oleks pidanud Ants inglise keelega seal esimese kursusega.
Ja siis.
Siis ma istusin seal äplitud, aga nüüd ma tegin loomakasvatuse ja veterinaariainstituudile mingisuguse dolla direktor olnud kolonn, tegi, tegi doktoritööd ja ja tal oli terve bussitäie tädisid, kes olid valmis andmeid sisestama, annan talle ei olnud kuhu neid andmeid sisestada, mis valmis arvuti. Ja siis ma tegin talle talle programmi, mis, mis nüüd on meil võimalus sisestada. Noh, seal oli siis oluline, kui uus NTFS teha, selline, et tädid eksida ei saaks kuidagi. Et noh, tooli kõige keerulisem kindlasti teha.
Ja noh, torudsus oli lihtne tegelikult.
See on vist midagi arvates seal mingisuguseid mingi statistikat lihtsalt.
Nad olid mingid piimaproovid kus siis laktoosi, valgu, igasuguseid hulk kahte eestikud ja noh, ma ei mäleta seal mingit korrelatsioonianalüüsi, tuli teha mingisuguseid noh, tema ütles ikkagi olnud algoritmid ette, mida tuleb teha selles mõttes, et oma või siis on matemaatiliselt nõu anda, aga aga üldiselt ta ikkagi teadis ise mida ta tegi. Mis tähendab seda, et siin kuskil palgal siis juba.
Ma olin poole kohaga palgal Tartu Ülikoolis, jah, seal arvutiklassis ma küll insenerina ja, ja piimaga see loomakasvatuse Veterinaaria Instituut. Kuna kuna ta mulle kuidagi nagu ühekordselt maksta ei saanud, siis mind võeti sinna tööle. Aga ma ei käinud seal kunagi lihtsalt selles mõttes, et ma olin seal mingi aasta või, või, või kaks olin tööl lihtsalt selleks, et saada nii-öelda tule programmi eest tasu, siis ma ei viitsinud palka ka minna välja võtma, noh siis pangakontosid eraldi. Ehk siis siis ta direktor tuli mulle tagajärgi ja viis mind sinna sellepärast, et ta ei olnud kassapidaja kisa ära kuulata.
Tuldi autoga järgi riigi raha saama. Täpselt noh, programmeerija magus elu.
Jah ei too oma vastuses, et direktor valdkonnad oli väga-väga nagu lõbus sell, et.
Et nende oma inimestega, ta oli hirmus kuhi. Alati kui me sinna läksime, siis ta kõigepealt sõimas kõigil näo täis, aga aga, aga väga ettevõtlik tüüp selles mõttes, et ma mäletan kunagi ma olin kodus isaga saunas. Siis ema tuli sõnul, et kui on mingi mees, tuli. Ja noh, sama olgu need siis tuli, tal oli midagi kiirelt vaja. Ja mul ema ütles, et ta ka nagu enam-vähem minna tutvust ta Montreali uksest sisse astunud ja läinud kohe elutuppa ja maha istunud. Teie eelarve vaadata, et kui on probleem, et vahet ei ole, kus, kus ma Alementaator.
Et noh, selles mõttes väga sihikindel
Aga see, et sa nagu loengutesse jõudnud, siis mingi asi võltsimisele arvutite juures kinni. See oligi see Assembler ja pusimise huvi või, või mis sa siin-seal võidis?
Nojah, selles mõttes, et ma tegin Assembler, siis ma kirjutasin tekstiredaktori.
Kuhu sai ikka päris ohtralt igasuguseid kitsesid. Tehtud too oli kindlasti kõige-kõige nagu keerulisem masin, mul peaks vist isegi too paberi peal väljatrükituna kood alles olema, tuli.
Kas kas kas viis tuhat või kuus tuhat, keda Assembler sa seda üldse nii palju, et üksjagu eco asembri koodi mõttes on seda palju, aga arvestada, et nagu tekstiredaktor viie tuhande reaga pole paha.
Et jah.
No maitsesin seal ühel tütarlapsel, kes mulle väga meeldis, hoidsin tal ka kursusetöid teha ja tuleks selle tekstiredaktorit nagu vaja. Et läheb ainult muidu, muidu oleks pidanud kirjutusmasinal trükkima. Et noh, et, aga noh, arvutis ühtegi kohalikku tekstiredaktorit ei olnud noh, oleks ka saanud üht või teistviisi teha seal mingisuguseid hädiseid, asjad olid aga aga noh, selleks, et kõik suured-väiksed, tähed, sellised asjad, noh.
Ei olnud lahenduste, siis ma keetsin.
See.
Jaan Tallinn kirjutas ka Prangli endale ühe esimese asjana kirjutatud tekstide tahtma.
Ka seminarist mis tekitab mõte, et kas see tähendab siis seda, et igasugune kuramuse interneedus ja muud niisugused asjad on teinud nagu hoopis karuteene. Et varsti, kui sa tahtsid niux tekstide traktorist ise kirjutama ja nüüd võtavad Internetist täpselt sellise nagu vaja võtta Tii või noh, mis iganes.
Noh, eks ta siis oli ka natukene lihtsalt see, et sul ei olnud neid programme kuskilt saada, eks, eks Ameerikas olid Apple'i jaoks ilmselt kõik programmid olemas. Aga, aga nad ei olnud lihtsalt Eestisse ja eksis.
Töö olnud ja siis tegi teiseneks.
Noh, aega ka oli ja.
Mis sul see tol ajal sinu ettekujutus oli? Et kuhu see kõik nagu viibekas istubki, nagu järgmised kakskümmend aastat nagu Apple'i Apple kahtede juures Vanemuise tänavas või?
Mul mul ei olnud mingisugust, väga konkreetset plaani küll ausalt öeldes, kuhu see viib, selles mõttes, et,
Noh, siis siis tulid, on mingisugune meilinduse käimapanek seal Vanemuise tänavas ja noh, ta oli kuskil üheksakümnes aasta siis tolle pärast, et siis siis Taavi Talvik kutsus mind Postimehe toimetusse.
Sinna ta oli mingisuguses kuu juuniks valmis pannud ja mingisuguse hulga terminale, mille kaudu siis sada ajakirjanikku artikleid sisestasid. Emaks oli, vist on tekstiredaktor saa. Ja, ja eesmärk oli siis teha eesti keele õigekirjakontrolli programm soo sinna peale ja tollega ma siis seal tegelesin, ühesugused läksid sealt loengus, tänavalt, Postimehesse jah. No no ma käisin seal Vanemuise tänaval ta ikkagi tolle pärast, et et no tädikesed, kes arvutiklassi seal nagu haldasid. Tehniliselt liiga võimekad ei olnud ja, ja noh, siis oli kõige kasulikuma sel õhtul läbi käisin ja meelega mingite asjadega nõu andsin, aga aga jah, siis tuli ikkagi Postimees nagu.
Mullu Taimsoo.
Ja see jällegi, et see eesti keeles Belleri või õigekirjakontrolli tegemine ei ole nagu triviaalne asi seal keelest ka ei ole, ei ole triviaalne asi.
Ja ja siis siis ma avastasingi, kui neetult keeruline see eesti keel ainult nagunii ka, et iga teine sõna veel mingi erand olevat. Väga tüütu oli, et ega me teda valmis ei saanud.
Tegelikult et me ei saanud teda valmis. Jah.
Sest umbes üheksakümne kolmandal aastal kuskil hakkas tekkima filosoftide niisugused asjad, et nad tegid vöödilise eesti keele spellerit. Ja see oli ka ikka päris suur tükk pusimist ja seal oli selleks ajaks riismeid sedasi läinud arvutusvõimsus ka, eks ole.
Jajah absoluutselt aga, aga noh, üheksakümmend kolm oli juba see aeg, kui ma tulin.
Härra Tallinnasse Anna kohta.
Sa tulidki otse Postimehest.
Et seal oli mingisugune lühikene periood Postimehe ja Hansapanga vahel ka tegelikult kus ma olin mingisuguse sulgi Birmas
Saan ma ka kirjutasin mingit programmifaile. Aga, aga ta oli nagu vägagi selline kaootiline koht selles mõttes, et too bisnis läks tollel hulgifirmal nagu hirmus hästi ja iga kuunud kolm kutti, kes, kes talle omanikud olid, ostsid, ostsid igaüks endale uue BMW. Tolmutasin Tõndega ümber tolle maja sõita, et et ei olnud liiga motiveeriv keskkond. Tegelikult.
Eks neid kahjuks, mis sa teed? Kuidas, kuidas see kuidas see sinna Postimehe kutsumine ees pidid siis tolle Taaviga kuidagi tuttav olema, kui kus tahes üles leidis.
Ega ma nüüd ei julge öelda, ausalt öeldes peast, kus, kus ma Taaviga tegelikult tuttavaks sain selles mõttes, et.
Sel ajal, kui mina äplite taga istusin, istus Taavi tegelikult sealsamas Tähe neli kus ma esimest korda nairidega kokku puutusin, istus Tähe neli keldris, kus oli mingisugune IBM PC.
Mugi.
Ja kas Taavi tegi midagi äkki Tartu Ülikooli Raamatukogule ja mina olin ka tolle kuidagi seotud ja kas me äkki seal Tartu Ülikooli raamatukogus tissid, aga saime kuidagi kokku?
Jaa.
Ja ja siis noh, vanetsi näplethaavile ja dominets mulle toda PC-d. Ja siis oli sihukene mände King's kvast ja ja siin me Taaviga mängisimetada Kings kosti seal Tähe tänaval
Ja noh, sealt me tuttavaks saime me tolle King's mästyle
Kings, kes on ju King's Quest, on ju metsakas. Just.
Et tollega tollega läks ikka aega, et lõpuni mängida, et me istusime ikka palju.
Niisiis ühel hetkel abi oli see Postimehes ja siis tal oli abi vaja, siis ta kutsus sind täpselt. Aga vaat nüüd seejuures Hansapanka sattusid. See on huvitav lugu.
Hansapanka ma sattusin suutnud aduda vedelikule. Taavi Talvik ütles valitsussides mahvan tol ajal. Ja mitra lingist Rainer Nõlvak. Mahvan oli see, kes küsis Taavi käest, et et Tõnis Sildmäe otsib kedagi, kes juuliks tunneks.
Panka.
Ta ütles, et tema küll ei taha minna. Ja küsis minu käest. Ja mõtlesin, et ah suva, et ma võin ju rääkida ja kuulata, et mis, mis seal siis. Deemon. Ma tulin Tallinnasse Tõnis silmaga rääkima, Sildmäe küll jättis mulje, et tal on terve bussitäis juunitsi mehi ukse taga. Jah, ega kass, keda kõiki integreerub, aga, aga vist tegelikult selle minu ühtegi ei olnudki. Igatahes igatahes ma sain nagu sinna tööle.
Jaa.
Ja siis tus, kool, juuniks oli sinna juba ära installitud ja, ja Tarmo Pajumets püüdis päädistada vaakled sinna skoobiale, inste skoori UNIXi peale installida. Aga ega nad tulles skoori unistust ka midagi ei teadnud, nii et esimese päeva lõunaks nad mõlemad läksid sealt konsoolid ära, täna ja rohkem sinna tagasi ei tulnud, et kui nad vaatasin, et ma, ma vist tean natuke rohkem.
Noore inimese hästi aru, kust nende teadmiste piirid on. Aga tol hetkel oli pank kui selline oli ju juba olemas. Ja, ja muidugi noh, mille peale käis nüüd, mis infosüsteem oli Oracle'i talle seal kaks paigaldama, siin?
Too käis paradoksi peal. Paradoks on siis oli paha, tooks andmebaas, aga aga eks too Oracle'i andmebaasi majja toomine oli nagu üks väga paljudest. Maa on nagu Hansapanga. Edu aluseks olevates strateegilistes otsusest, kuidas tollased juhid suutsid ette näha, vähem tooks, töötas tol hetkel täiesti normaalselt. Ei olnud häda midagi. Aga, aga juba oli Tõnis Sildmäe nagu välja raadio tegelikult me peaksime nagu mingisuguse tõsisema andmebaasi mootori sinna alla panema. Et esialgu tulekski novelli peal toovacle. Ja aga siis me saime tule Skujunitsiverda nii kaugele, et tsime, leidsime, turnisime skoojunud.
Ja vaat räägingi sellest korra lähemalt, et see tahab niukest. Ühesõnaga see tähendab seda, et kellegi teise peas siis ilmselt oli arusaam sellest, et ummik, arhitektuursed, otsuseid, info, arhitektuursed, otsused on kuidagi seotud nagu äriga või nagu äriedu aluseks. Üheksakümnete alguses see ei kõla nagu üldse, nagu triviaalne teadmine. Kusjuures oli.
No ma usun, et et too seltskond, kes seal oli tol ajal, oli ikkagi ka selge arusaamine tolle paradoksi tehnoloogilistest piirangutest ja samal ajal oli oli ka arusaamine, kuhu poole see pank liigub. Ehk siis ma arvan, et tolleks hetkeks, kui mina sinna tulin, äkki oli, oli seal minus kõrvallaua peal juba esimene sularahaautomaat tegelikult oli niisugune suhteliselt pisikene, mis mahtus laua peale IBMi oma kahte eile käinud sisse, vaid tuli magnetiga lihtsalt läbi tõmmata. Et noh, ma arvan, et Aadeeemmide asi oli üks ka, mis, mis nagu tolle paradoksi andmebaasi piirangud välja tõi noh, samamoodi kuna klientide arv kasvas plahvatuslikult noh, ilmselt ka tolle pealt nähti, et et too paradoks ei suuda tegelikult kui selline kasv jätkub, ära teenindada.
Teine asi, mis mind on ikka huvitanud, on see, et, et samal ajal toimetati päris mitmes pangas ka nagu valmis softiga vastati lihtsalt kuskilt Briti maad, mingisugune pangas oht ja tehti panka, tuleb, miks, miks, miks Hansa teistmoodi?
Ei oska öelda selles mõttes, et vabalt võib ka olla see, et et too seda oskavad need öelda, kes päris algusest olid tolle pärast, et etas. Kuid kuidas ta, kas ta.
Kuidas ta spinn Development sinna Hansapanka tuli? Et noh, nimi dispinda hakanud, et ilmselt oli seal siis mingeid Developer häid. Ja ilmselt ta esialgne ülesanne, mida teha tuli, oligi mingisugune väike tükikene ja kui oleks hästi, siis sealt hakkas asi arenema. Ma, ma ei oska öelda.
Udud sprindi lood, need on mingisugune Greriti algus.
Jah, selles mõttes, et ole, kui minagi tööle läksin, siis esimene palga maksta oli tegelikult spinn Development siis Sistus pin Development minu meelest nimetati lihtsalt grebitics ringi. Ja, ja mingi aeg hiljem siis ma saan aru, Londoni kindlustusfirma ütles pangale, et kuulge, et teine, kuid, ja IT-st mitte midagi, et kogu asi on nagu väljas mingisugusest täiesti iseseisvad ettevõtted, et kuidas ta nagu Magnitski ei huvita ennast lõppes sellega, et et Hansapank ostis siis Tõnise käest, need rebiti aktsiad ära ja kõik me tulime siis Hansapanka tööle, et Nabala Kerbitit jäi siis noh, ta juriidiline keha jäi alles ja kuni tänase päevani siis praeguseks swedbank Support OÜ nime all olemas.
Huvitav on see, et see kultuur oli ikka jätkuvalt nagu kleebiti oma, sest kui mina tulin pangast ära aastal kaks tuhat kaks ma pakun, siis mina sain viimase särgi, mis mulle väljastas, oli kleebiti looga. See oli oi kui jube elujõu vesi.
No kindlasti oli selles mõttes, et on tega ega ta noh, mitte ainult mitte ainult Kebit, vaid toob pank ise tervikuna oli tegelikult äärmiselt elujõuline.
Et noh
Ütleme, vähemalt kuni tolle hetkeni, kui kui Hansapank Hoiupangaga liituti et siis toimus ikkagi suundub tundeline muutus, esimene soojem kultuuridanud tuli lihtsalt väga palju teisi inimesi juurde.
No, mis ta ta kultuuri nagu püsti hoidis, kust, kust see tuli?
Ma ma olen Duda mõelnud, et ma ei tea, ilmselt ühelt poolt ilmselt oli kõigile inimestele, kes seal olid, oli ikkagi väga selge saavutusvajadus, et ikka oma asja väga hästi teha. Sellepärast, et seal isegi ei pidanud minu meelest need, kes võib-olla ei performinud piisavalt hästi lahti laskma, vaid nad läksid ise ära. Mil puhul, et, et noh, seesama et näed, kui ma ütlesin, et Pajumets seda oleks installis. Tegelikult oli üks mees seal veel kõrval kes tolles kuu juunis sinna Collins, Tallinn tegelikult kui ma tulin sinna siis ma saan aru, et too mees ise läks paari nädala pärast ära, tegelikult teda ei lasknud keegi lahti, et ta isegi ütles lahkumisavalduse oleks hea, sellepärast et ta sai aru, et, et noh, et noh, temal ei ole midagi teha selleni, et tatud Asko püüdis kuidagi midagi teha, aga kuskil võib-olla. Ja noh, tuli nagu absoluutselt kõigile olid ühesugune kultuur, et sai, sai seal pidanud kaks korda kellelegi ütlema, sa teadsid, et asi on tehtud.
Jaa. Üks asi, mis mind on, kui ma jutu puhul on veel huvitav, et alustasid pihta siis võid kirjutada Selveris PIN koodi, see on ju puhas nagu arendaja. Aga vallas sa läksid kuidagi kohe selle asja nii nagu opereerimise peale. Kuidas see toimus? Ja miks oskad sa?
Ja ega seal mingisugust nagu teadlikku valikut väga ei olnud, selles mõttes, et ta töö tundus huvitav aga Mogakla baasi olnud varem näinud ja selles mõttes, et noh
Ma ma kuidagi ei mõelnud, et ma olen programmeerija selles mõttes, et noh, tegelikult ju ka seal arvutiklassi säplite juures noh, ta tööülesanne oli tegelikult kõigi nende inimeste Assisteerimine, üllad, hoid, et nood arvutid töötaksid, et probleemid nende arvutitega oleksid lahendatud. Programmeerimine oli puhtalt nagu hobi tolle töö kõrval. Kui kuigi noh, kogu asi algas loomulikult programmeerimisdetailide programmeerime Apple'it meeldima. Et aga kuidagi ei mõelnud. Ja ma arvan, et tollal ka ju liiga palju konteksti, et need on arendajad ja need on ülal hoidnud hiljem et, et ma arvan, et, et ta tuli hiljem, et et siis kõik lihtsalt
Sest siis, kui siis kui mina uksest sisse saabusin, siis oli see juba ammu olemas.
Aga ana ja Saldseks muidugi.
Aga kuidas sulle too kuue uniks sai külge juba Postimehes kujunes see testimine ja Oracle'i, kuidas see sulle külge sai?
No see saigi külge sealt Hansapangast, selles mõttes, et toetlejali
Jah, lihtsalt tegema. Ja mitte väga palju aastaid hiljem oli see üks maailma suurimaid oraakli koodibaas, et minu meelest käis mingi praakspetsialisti, mulle meeldis see jutt sellest, kuidas nad seda ei ole kuskil näinud, et kellelgi on sihuke asi tehtud oraakli peal.
Võis olla küll sellepärast, et,
Vilve Vene auto ostja, kes seal kirjutasin, et ettuda Piia Sikk välja, kirjutati seal seal usinasti noh, toda oli tõesti väga võetud, osa oli väga palju noh, kuna too alati fookuses olid ju tehnoloogia vaid ikkagi selles mõttes, et ilmselt ilmselt oli nii teda kõige efektiivsem teha, jääks. No baasi protseduurid kestlik kiiremini kui mingisugune noh, klient-server asi, et.
Ja sul ei tekkinud seal tunnet, et no see, et, et see Oracle'i püstipanekul padjal püsti pandud, no las ta siis nüüd käib Promeeriks parem.
Noh, eks eks programmeerima pidi veel ikka natukene selles mõttes, et,
Skripte tuli kirjutada, mis mis siis kogutud asju üleval, et kas või seesama, et, et kuidas too, kui, kui sa tolle skool juuniksi masina üles poodi, et kuidas tuhat läheb, aeg käima pannakse. Ega tollel Oracle'i installi juures mingeid skripte ei olnud. Ainult et sa kirjutasid ise need skriptid, mis tollebaasi käima panid, vist enese käima panid. Kogu tolle asja tegid kogu päkapike tegemine tolleks pidiskepsid kirjutama, noh, lisaks sellele kõik need Päts protsessid, mis, mis olid tehtud siis kirjutatud kogu nende käivitamine, et oleks tulist skriptid teha. Et noh, nüüd metsa retk sai tegelikult kirjutatud ikkagi päris päris palju.
No see, mis sa kirjeldad, on päris nagu keeruline asi mis peab olema siis nii arenda, noh, arendamise mõttes, et see on mingi peeles kull, mingi kuramuse mindi, siis pätsid ja Oracle'i baasides nende seas see kõik kuidagi nagu terviku moodustama, et see koos püsiks ja oleks nagu tehtud praegu, kuidas te tervik tekkis, kes seda juhtivi?
Kes arhitekt oli?
Ega siis kedagi arhitektiks ei, ei nimetatud, aga.
Noh, ma julgeks siis arvata vast.
Vastu tulles tantsu kontekstis arhitekt oli ikkagi Vilve Vene. No vähemalt ütleme mulle selline mulje jäänud, et tema oli, tema oli siis tänapäeva mõistes arhitekti keegi selliste terminitega.
Ega tänapäevalgi väga-väga lihtne kasutada.
Et see kontseptsioon, kuidas, kui too kõik kokku töötab tarkvaraliselt, et ma arvan, et ta tuli ikkagi ennekõike ilmelt mingisugune ütleme, non saksa heli laiem võtsid need tekitasin mina. Et NATO samas skriptide asi, et noh, et iga too Siibäam, mis pingid Pätsi tegi, et too ei oleks erinev, et on kuidagi ära standardiseerida, siis ma pidin mingisugused, et mitte funktsionaalsed nõuded esitama, et et nad kõik oleksid ühetaolised, saaksin kasutada mingit ühte skripti paljude asjade käivitamisest.
Ja see on see jõud või mitte, funks aasta nõuet juurde sealt järgmine asi, et kuidas noh, Postimehes on ka ikkagi suur ajaleht, aga see ajaleht ilmub ka siis, kui need ajakirjanikud oma trükimasinaga kirjutavad, oma lood valmis. Aga pank trükimasina peal nagu enam ei käi. Mis tähendas seda, et ühel hetkel see naguniisugune pusin ise ja vaatan, kuidas mossel teinud pidi nagu maad andma sellele, et on mingisugune struktuur ja mingisugune niisugune formaalsemisel kuidas, kuidas see sündis, kas see oli mingisugune otsus, et nüüd hakkame nagu korralikuks või see sündis kuidagi sujuvalt?
Noh, tolle tolle igapäevase ülalhoiu kõrvalt sa pidid tegelikult ikkagi paratamatult noh, igaüks kuna kuna ta kasv oli nii kiire, siis igaüks pidi tegelikult vaatama, aga mitte ainult seda, kuidas see asi täna ära rullib, vaid ka, milline see asi nagu aasta pärast välja näeks. Ja, ja noh, kindlasti Tõnis silma ka Fassilteeris seda, et, et tuleksid igasugused erinevad kontaktid kes, kes mingisuguseid uusi lahendusi pakuksid ja, ja nendega sai, sai konsulteeritud ja äkitudeni. Nii need asjad arenesid edasi ka selles mõttes, et noh Cuzco juuniks ju samamoodi tollel tuli tehnilised piirangud ette.
Üheksakümmend kuus, üheksakümmend seitse kuskil sai ju happe uksi vastu välja vahetatud, enne toda sai, oli mul nii HP kui Sammy serblase laua peal ja sai võrreldud siis kumb, kumb nagu kiirem on noh, tolleks ajaks oli oli ta panka piisavalt suured, siis, siis oli selge, et, et me tegelikult ei pane ühte masinat, vaid me paneme klastri. Sain nende Vendoritega nad klastri lahendused läbi räägitud.
Jah.
Et ega, ega sellist, nagu et mingisugusel hetkel oleks mingisugune Mäppe toime, Tennoli on Ahja, siis, siis tehti kõik asjad korda. Et kõik, kõik see arenes tegelikult ikkagi evolutsiooniliselt, neid asju on lahendusi vahetati iga aasta välja, tegelikult tuttav on vastu tolle pärast, et testi ei oleks lihtsalt toda. Kümme aastat kestnud olukorda, kus, kus iga, ma ei tea, üheksa kuud on olnud kahekordsus, klientide arv.
Käive kasum mis iganes. Kõik numbrid kahekordsest kümme.
Jah, ega last kenasti sellist kasvu ei kujuta tänapäeval nagu väga ette enam kui sa just kuskil Skype moodi asja sõidelt.
Nojah, ega, ega nüüd ettevõtted ongi maailmas väga vähe vist, kes, kes nii kiiresti nii pikalt suudavad kasvada, noh, oli, oli mingisugune substes of Estoniat.
Ma mäletan, aastal just sajandi lõpuks oli sinna panka tekkinud mingisugune niisugune üsna ike spetsialiseeritud tiim nagu inimesi, kes opereeris seda kupatust seal. Kuidas te tiim tekkis, kes igalühel oli oma mingisugune valdkond, millega tegeles vahetanud ja kolmekesi moodustasid te niisuguse asja, millest Veeber seisneb kogu maailm püsti. Et kuidas tuli, kuidas ta kolmik tekkis.
Tohutu tekkis ka aja jooksul selles mõttes, et noh, Madis oli, saar oli enne mind olemas ja on olemas, muutmist, ma siis oli, seal on praegu ka olemas. Et.
Ma ei teagi päris täpselt, mis tema roll päris alguses oli, aga aga ikka ikkagi sel ajal, kui kui mina seal tolle Orkla Masingu toimetama hakkasin, siis minu asi oli nagu toit, tehniline pooletu, andmebaasi ilmsin, töötaks ja Madise asi oli siis luua sinna uusi tabeleid ja teha indekseid ja vaadata, et päringut hästi käivad nii-öelda see tagasi. Ja, ja noh, too, too hall jätkus tal edasi. Toomas Soomets tuli.
Ma pidin peaaegu ütlema, et ta tuli koos Hoiupangaga liitumisega ka tegelikult ei tulnud. Tegelikult tuli kaks aastat enne seda, kui ta töötas Hoiupangas, aga ta tuli kaks aastat enne seda, kui juba kärastati. Et, et tolleks ajaks, kui avastati, istus teha õigel pool lauda juba.
Et ja, ja noh, toomas toomas siis oli, oli selles mõttes nagu täijendastuda seltskonda, et kui Madis oli nagu kõige üle meie nii-öelda tahta, leiab mina, teadsin toda köögiandmebaasi Encini osa siis siis Toomas oli see mees, kes, kes nagu met vöögistest olid, siis täiesti jagas. Noh, siis siis too Kukkondiski nagu kogu tuletehnoloogilisest äkki, et põhjest tiim töötab hästi. Noh, mõistus minu arust toas ühes infoväljas kogu aeg alati on võimalik nagu öelda, mis toimub.
Kas sul oli juba tol ajal, ma tean, sul on Vaarmani huvi? No see oli juba tol ajal olemas. Sest ma mäletan, et kapi otsas oli makk ja sealt tuli aeg-ajalt tuli niukest eepilist klassikast muusikat.
Jajah see oli, seda ma ei oska öelda. Jää ei.
To klassic, jah, ausalt öeldes ma isegi päris A aastaarvu jälle julge öelda tolle pärast, et ei olnud veel omas on, et esimesed siin viibima Internetist ostsin tolle firma nimelist siidi, no punkt kommu. Et ja siis sai tunda klassikalist muusikat, seal mängitud? Jah, mitte küll vaagne, et põhiliselt tegelikult ma julgeks arvata Mozartit tol ajal põhiliselt Mozartit. Et jah, ma ostsin ka mingisugused päris alguses, ma ostsin mingisuguse siniHenrikuga Ruusa plaadid. Aga, aga kas ma ostsin Mozartit ja siis me mängisime seal meid neto ka kuidagi ise kanda? Me, eks me tegime erinevaid asju. Mingi periood oli, kus, kus välja tuldi.
Öeldi, et.
Teatud lõhnad on teie toas igapäevaselt tunda et.
Mingi periood oli tõesti, kus, kus meil oli alati konjakipudel kapis ja ja päeva sai alustatud difitsiga konjakiga. Et loomulikult mingit joomist ei olnud, aga, aga noh, eks tollest ühest pitsist juba noh, sul oli ka klaaslaua peal, võib olla kuni lõunani seal, et ega keegi ei joonud, aga lihtsalt natukene. Ja WRC ralli oli ka, mille Toomas siis püsti pandi, esindatud on FC hallit, mängisime ka seal mingisugune periood ikkagi, et noh, jälle, et ta tahtis väga palju network. Aga kuna Toomas selline löögi põhjal siis siis siis ta bänd vist kellelgi oli siis meie toas ennekõike.
Või akustilise klassiku pidule.
Klassiku huvi tuli sealt Enrico kohustus tegelikult, et mul oli vanematel kodus, oli Vituaalse salto hälli raamat Totu välja oli vist kaugelt sugulane ja Mehhikoga osales, ta kirjutas nagu Henrikuga ruudust raamatu noh loomulikult itaallane ja kuna ta oli sugulane veel, noh, siis ta on ülimad ülistav, aga, aga ta oli huvitav lugeda ja jättis väga sügava mulje. Ja siis, kui internet siis oli võimalik tellida, siis ma tellisin huvi pärast tule siit-sealt Sindi nafta, Edgar Ruusa plaate. Ja teine asi, mis häiris, oli ikkagi kaheksakümne viienda aasta Milos farmani Amadeus. Mida ma kindlasti soovitan kõigil vaadata, aga suurepärane film. Et Mozart ja sealt tuli ta Mozarti või. Ja noh, kui midagi sellist oleks, siis ma tellisin mingisuguseid raamatuid Mozarti eluloost mingi neli-viis ja mis mul on kodus üle tuhande lehekülje paks, et jah, sealt edasi. Ma ei tea, mis, nagu Beethoven, Schubert, Schumann, Tšaikovski. Mis projektid?
Töötan G4S-i, turvalise Everesti baasteenuste arendusjuht, noh sisuliselt siis vastutan õlahoiu eest. Et et kõik asjad oleksid püsti ja valvatud. Jah, selles mõttes, et noh see ongi, et mitte mitte siis ainult IT, vaid, vaid ka see tehniline valve, kuhu puutub, siis ka see raadiosidevõrk on meil hästi, on et kõik need signaalid jõuaksid siis siia keskele kokku.
Selge see ka huvitav ameti, nagu nagu Viksu need seiklused alates sellest tassemberist seal äppidel.


\chapter{Jaan Priisalu}
\index[ppl]{Priisalu, Jaan}

\ldots ütles eestlaste kohta väga hästi. Et eestlased otsused saunas 
teevad, eks ole, on \emph{no-brainer}. Aga kuidas seda tehakse? 
Inimesed käivad saunas, on paljad, räägivad midagi. Ja kui ära 
lähevad, kõik nagu teavad, mis otsus on. Seda ei hääldatud mitte ühtegi korda 
välja 
ja ei ole aru saada, kes on liider, kes selle \emph{move}-ga välja tuli. 
Sul on pikka aega olnud mingid võõrad sellid peal, 
kelle eesmärk on mingi muu, kui kohaliku rahva eesmärk. Ja sa tead, et võimu 
peale loota ei saa. Aga kui sa teed otsuseid sellise \emph{mode}-ga, et sa oma 
liidreid välja ei näita, on see tegelikult liidrite kaitsmise süsteem. Mis 
muuhulgas tähendab, et me oleme projektirahvas. Kui sa Vabadussõda vaatad, 
siis see oli ka selgelt projekt.

\ldots Mulle on seda küsimust mitu korda esitatud. Ameeriklased tulevad ja 
küsivad, et kui me ringi vaatame, siis need \emph{challenge}'d, mis te 
välja käite, on nagu pooltel maailma riikidel. Aga miks siin välja tuleb?

\bigskip
\noindent\rule{.3\textwidth}{.7pt}
\bigskip

\question{Kuidas sina sattusid arvutite juurde?}

Arvutite juurde sattusin neljandas või viiendas klassis. Meil oli pioneerisalk, kellega tegime igasuguseid asju, ja mind pandi seda juhtima. Salgas oli ka klassivend Kermo 
Jaaksoo\index[ppl]{Jaaksoo, Kermo}, kes pakkus välja, et võiksime 
minna tema isa töö juurde. Ülo\sidenote{
Kermo isa, akadeemik Ülo Jaaksoo\index[ppl]{Jaaksoo, Ülo}} töötas sel ajal 
Estonia puiesteel ja tal oli seal ES-1010\index{ES EVM!ES-1010}\sidenote{1010 oli ES EVMi esimese alaseeria esimene mudel.}. Ega me seal arvutiga midagi muud teha ei 
osanud kui Kuule maandumise mängu mängida. 

Ma käisin 1. keskkoolis\index{Tallinna 1. Keskkool}. Kui lastelt küsitakse, kelleks nad saada tahavad, siis tavaliselt 
öeldakse, et tuletõrjujaks, politseinikuks, autojuhiks. Mu vanemad 
väidavad, et kui minu käest seda esimest korda küsiti, siis mina tahtsin saada
inseneriks. Seepeale otsustasid
vanemad, et poiss tuleks panna matemaatikat ja füüsikat õppima. Isa leidis, et 1. keskkool on õige koht, see 
oli elukohajärgne kool ka --- me elasime vanglahoovis, Suur-Patarei 29. 
Läksime kooli katsetele, tegin katsed ära ja siis direktor küsis isalt, 
miks nad tahavad mind sinna panna. Isa rääkis inseneriloo ära ja ütles, et poisil on vaja 
matemaatikat ja füüsikat teada. Direktor mõtles natuke ja ütles, et see kõik on ju väga tore ja tõepoolest, poiss tegi katsed 
edukalt, aga meil on see häda, et matemaatika-füüsika klass hakkab üheksandast 
klassist, mis ta seni teeb? Seepeale pandi mind prantsuse keelt õppima.

\question{See on ka ilmselt väga tarviline olnud!}

Prantsuse keel on paarikümne aasta pärast
kõige kõneldum keel maailmas. Kui mõelda, kui levinud see on Aafrikas, siis on sellel seal sama funktsioon, mis Indias inglise keelel. Ja 
kui 
vaadata, missugune rahvastikuplahvatus neil on ja palju neil maad käes on, siis 
Aafrikal ei ole India inimeste tiheduse probleemi, vaid paisumisruumi on
palju.

\question{Kas pioneerirühmas kohtusite esimest korda essukesega\sidenote{ES EVM seeria masinate levinud hellitusnimi.}, maandusite 
Kuule ja mis veel?}

Põnev oli vaadata, kuidas lindid käivad ringi. Sel ajal olid juba 
vahetatavad kõvakettad. Nad näitasid meile ka trummelsalvestit, kuigi see 
ei olnud käigus, vaid oli lahti ühendatud.

\question{Mis asutus see oli Estonia puiesteel?}

Üks Teaduste Akadeemia asutus, ma täpselt ei mäleta. Ilmselt 
praegune Küber\index{Küber}, sest Ülo oli kunagi Küberi 
direktor.

\question{Kas tollest ajast jäidki seal oma rühmaga käima?}

Ei, mitte päris. Järgmine kord oli siis, kui meil olid arvutiõppe tunnid 
ja õpetaja Loonde\index[ppl]{Loonde, Jaak} viis meid Pedasse\index{Tallinna Pedagoogikaülikool}. 
Teine arvuti mu elus oli sealne Nairi-2\index{Nairi!Nairi-2}. Nairi on 
transistorite peal arvuti, perfolint on viierealine. Kui 
matemaatika-füüsika klassi läksime, siis Jako Bergson\index[ppl]{Bergson, Jako} tõi 
Kirovi kalurikolhoosist\index{Kirovi Kalurikolhoos} MIR-2\index{MIR-2} 
ära. Mir-2 on tegelikult maailma esimene personaalarvuti, mis oli tehtud inimese aitamiseks. Selle protsessor kaalus pool 
tonni, aga ideoloogia oli selles, et inimene saaks 
oma rehkendused tehtud. Muu hulgas olid integreerimine ja diferentseerimine 
rauas\sidenote{St. riistvaraliselt.} realiseeritud.

\question{Sest inimesel oli ju vaja integreerida ja diferentseerida, mille 
jaoks talle muidu üldse arvuti!}

See oli Ukrainas tehtud arvuti, seal arvutati gaasiturbiine, 
rakette ja muud säärast. Programmil oli 
Algoliga sarnane keel ja see algas käsuga \verb|RAZR|, mis tähendas 
\begin{russian}разрядность\end{russian} ehk kui pikad arvud on. 

\question{Kui pikaks arve sai keerata?}

Me keerasime kas 300 või 400 peale, kümnendkohtades. Panime programmi piid arvutama ja arve ritta ajama ning see lõppes sellega, et 
kuigi oli talveaeg ja me tegime aknad lahti, kuumenes arvuti ikkagi üle. 
Pooleteist tundi saigi sellega tegutseda.

\question{Kust teil selline mõte, et võiks piid arvutada?}

See lihtsalt tundus lahe.

\question{Ja kust te matemaatika saite?}

Mul oli üks paks venekeelne matemaatika õpik või 
entsüklopeedia. Seal olid igasugused read ja pii rida oli 
üks nendest.

\question{Järelikult oli teil koolis tolleks hetkeks matemaatika juba pihta 
hakanud?}

Jah. Meie koolis oli nii, et kui käisid 
matemaatika-füüsika eriklassis, siis esimese asjana jagas õpetaja 
Uudelepp\index[ppl]{Uudelepp, Helgi}\sidenote{Gustav Adolfi 
Gümnaasiumi legendaarne matemaatikaõpetaja Helgi Uudelepp.} klassi pooleks. Pool klassi hakkas õppima tavalise 
keskkooliprogrammi järgi ja ülejäänud kambale anti olümpiaadi ülesandeid. 
Kamp oli väga kõva (näiteks Mati 
Pentus näiteks\index[ppl]{Pentus, Mati}\sidenote{Mati Pentus on Eesti 
matemaatik, alates 2003. aastast Moskva Riikliku Ülikooli professor.}), probleem oli koolist üldse välja jõuda. Rajoonist sai 
niikuinii vabariiklikule edasi.\sidenote{Toonased olümpiaadid olid organiseeritud kooli-, rajooni- ja 
vabariiklikeks olümpiaadideks. Vabariiklikult olümpiaadilt oli võimalik pääseda ka 
üleliidulisele olümpiaadile.}. 

\question{Kas teil oli koolis arvutitund ka ja kasutasite MIR-2?}

Jah. See asus küll kooli spordihoones. 

Selge see, et 1. keskkool seisis direktor Viikholmi\sidenote{Helmi 
Viikholm\index[ppl]{Viikholm, Helmi} oli kooli direktor aastatel 1962---1982.} najal ja kui 
Viikholm läks pensionile, juhtus nagu ikka organisatsioonidega juhtub, et 
hakatakse rootsi keelt või muud sellist õpetama.

\question{Kas selleks ajaks olid sina sealt koolist juba läinud?}

Ma käisin siis veel koolis, kui Viikholm ära läks. Organisatsioon toimis vana energiaga 
veel natuke aega edasi, tavaliselt võtab lagunemine 
kaks-kolm aastat aega.

\question{Kas keskkooli ajal pusisite Jaaguga MIRi peal või oli juba muid 
võimalusi ka?}

Mina sain oma esimese palga progemise eest aastal 1984, 
üks suvi enne keskkooli lõpetamist. Paldiski maantee 1 asus termo- ja 
elektrofüüsika instituut\sidenote{Eesti NSV Teaduste 
Akadeemia Termofüüsika ja Elektrofüüsika Instituut 
(TEFI).\index{Teaduste Akadeemia!Termofüüsika ja Elektrofüüsika Instituut}}, nende käes oli näiteks Arnold Veimeri nimeline laev.
Mina olin akadeemik Krummi\sidenote{Akadeemik Lembit Krumm\index[ppl]{Krumm, Lembit} (1928---2016).} juures, 
kes arvutas elektrivõrkude staatilisi režiime. Neil olid ka 
arvutid ja mina tegin oma esimese töö arvutil 
Iskra-226\index{Iskra!Iskra-226}, mis on sisu poolest Wang 2200 koopia ja millel on muu hulgas videoprotsessor
8080. Üks vend tegeles sellega, et 
pani selle videoprotsessori peale käima CP/Mi. 

\question{Kust nad su leidsid?}

Ma ei mäleta, aga ilmselt tutvuste 
kaudu või siis lihtsalt läksin ise sinna.

\question{Ja seal arvutati elektrivõrke?}

Esimese asjana pidin tegema rehkenduse ühe
ministeeriumi aruandluse jaoks --- tabelarvutuse Basicus\index{BASIC}. Tegin programmi valmis ja nägin esimest korda 
päris \emph{user}'i probleemi ka. Korraldasime piduliku üleandmise, 
komisjon tuli 
kokku ja naisterahvale, kes pidi seda 
programmi kasutama hakkama, öeldi, et istu arvuti taha ja proovi 
midagi toksida. Naine keeldus. Keegi ei saanud aru, 
mis juhtus --- kas programm ei meeldi või milles asi. \enquote{Ei, ma kardan 
elektrit!} vastas tema. Ma ei mäletagi, kuidas see tsirkus lõppes. 
Siis tuli juba järgmine töö. Neil oli programm, millega nad suuri jakobiaane 
arvutasid, ja see käis essukese\index{ES EVM} peal, mis oli vist 1055. Neid 
oli Küberis kaks tükki: üks oli 360 ja teine 370 koopia\sidenote{ES EVMi 
esimene alaseeria oli IBM System/360 ning teine ja kolmas System/370 koopiad. 
Mudel 1055 kuulus teise ja 1066 kolmandasse seeriasse.}. Nende terminal ei olnud 
mitte VT100 nagu mujal maailmas, vaid IBMi VT52, mis näeb üsna 
hüperteksti moodi välja. Kirjeldad ära sisendi ja väljundi väljad ning terve 
ekraan saadetakse korraga, töötab nagu veebileht. Sinna peale 
ma tegin ühe programmi, mis võimaldas sisendandmeid mõistlikult sisestada ja tulemusi vaadata. Kuna essuke oli \emph{patch}-arvuti, siis pidi 
vahele tegema programmi, mis \emph{patch}'i tulemused 
sisse söödaks või välja võtaks ning 
terminaliga suhtleks. Tolle vahejupi tegi vist Tarmo Mere\index[ppl]{Mere, 
Tarmo}. See kõik oli üsna aeglane, ringitõstmine võttis
aega ja käis läbi ketta.

Teine Küberi essuke oli 1066. Selle peal nad andsid mulle ühe
assembleri makro, et paneksin asja käima, aga protsessor oli juba 32bitine. Olin Inteli assemblerit vaadanud, aga no üldse ei olnud 
sarnane. Kõiki registreid sai kõikides funktsioonides kasutada ja kõige 
hullem, millest ma ei saanud aru, oli see, et kõiki aadresse võeti 
baasregistri suhtes. Sealjuures eeldati, et sa lihtsalt 
tead seda. Aga kuidas teha kindlaks, milline on baasregister ja kust see 
algab? Leidsin baasregistri valimise käsu üles, kuid 
midagi oli ikka puudu. Tuli välja, et seal, kust baasregister hakkas aadresse 
lugema, oli vaja lihtsalt kõikide käsuridade ette panna üks 
tärn.

\question{Programmeerimisoskus oli sul järelikult olemas. Kas õppisid seda Jaagult, 
pusisid ise või kust see tuli?}

Jaak\index[ppl]{Loonde, Jaak} õpetas jah, MIR-2 peal me mingit nalja 
tegime. Kui juba piid arvutad, siis peaksid ka paar rida oskama kirjutada. Järgmisena tuli Basic TEFIs. VT52 puhul ma ei mäleta, 
milles ma programmi tegin, võibolla oli Fortran. 

\question{Kuidas VT52 puhul progemine käis, kui ekraanil olid lihtsalt 
mingisugused väljad?}

Ei mäleta, interaktsiooni kirjeldus on selline, et saadad 
terve 
andmepaketi korraga minema ja saad terve andmepaketi korraga tagasi. Andmete 
töötlus ja esitus on eraldatud. 

\question{Naljakaid aparaate on olemas!}

Mäletan, kuidas vennad presenteerisid modemit, millega me Küberisse 
helistasime. 1200boodine modem oli külmkapisuurune. Kui tuli 2400boodine modem, siis see oli tükk maad väiksem, pool külmkappi.

Ja millega vennikesed veel tegelesid?! 1984. aastal olid olümpiamängud LAs. Mängude 
ajal nad jälgisid elektrivõrgu parameetreid, peamiselt sagedust, ja selle põhjal 
ütlesid, palju tootmist seisab ja mitu inimest vaatab olümpiamänge. 

\question{Kas nad ütlesid seda ka ametlikult kellelegi?}

Ei, nad vaatasid oma lõbuks. Neil oli kihlveokontor --- vedasid kihla, palju järgmisel päeval 
vaatajaid on.

\question{Sa olid matemaatikas ja füüsikas tugev, aga arvutiasja 
pidid suuresti ise pusima. Mis sind selle juures tõmbas?}

Äge on see, kui saab oma kätega midagi teha. Üks liik inimesi armastab teooriaid 
välja mõelda, teised asju kokku 
ja käima panna. Täna ma olen 
mõtlemise ja teooria poole peal.

\question{Eks asjad ole maailmas tasakaalus. Nii et tol ajal meeldis sulle 
vajutada arvutiklahve ja arvuti muudkui tegi?}

Automatiseerimine --- see, et asjad ise juhtusid --- oli väga äge. Näiteks et
auto sõidab ise. Sel ajal oli natukenegi targem 
juhtimisalgoritm haruldus.

\question{Ja see hoidiski sind nii palju arvuti taga, et õppisid programmeerima 
ja õigetesse kohtadesse tärne panema?}

Jah.

\question{Kas selle juurde käis ka mõni laiem valdkondlik huvi? Mõni on 
rääkinud, et raamatud, muusika ja muu selline suunas arvutite poole.}

See oli sügav Vene aeg, mis mõttes \enquote{raamatud ja muusika}? 
Loomulikult lugesin ma Asimovit, robotivärgist räägiti seal päris palju. Lugesin kõiki 
raamatuid, mida kätte sain. Neid, mida ei saanud, lugesin 
juba Prantsusmaal, kui läksin Toulouse'i õppima. Ma panin 
kooliminekuga natuke puusse --- kui ma poole septembri pealt kohale läksin, 
polnud koolis veel kedagi. Nii ma siis 
istusin raamatukogus ja lugesin matemaatikat ning Asimovi jutte, et keeleoskust parandada.

\question{Enne kui Toulouse'i juurde jõuame, küsin, kas 
töölkäimine keskkoolis õppimist ei seganud?}

Ei seganud, see oli eluviisi osa, ma olen alati kooli kõrvalt tööl 
käinud.

\question{Mida sa ülikooli õppima läksid?}

Automaatikat, automaatjuhtimissüsteeme tehnikaülikoolis\index{Tallinna 
Tehnikaülikool!Automaatikateaduskond}.

\question{Kuidas see otsus sündis? Oli see loomulik valik?}

Oli küll loomulik valik. Kuulsin oma sugulaselt Jaan 
Võrgult\index[ppl]{Võrk, Jaan}, mis 
see automaatika üldse on. Meie rühm oli väga äge, klassivendi ja -õdesid
oli umbes kümme. 

Gibbs\index[ppl]{Kübbar, Heiki} ütles mulle hiljem, et seda oli 
vastik vaadata --- ise higistad matemaatika- ja füüsikaloengutes, aga 
need vennad tulevad kuskilt, teevad pulli, lähevad eksamile ja saavad 
kõik viied. 

\question{Kas ta oli su kursavend?}

Me oleme kindlasti koos loengus käinud. Ta on aasta noorem, aga ilmselt läksid meil loengud minu sõjaväest tagasitulekuga kuidagi sünki --- ma istusin oma kaks aastat sõjaväes ära.

\question{Kas sind võeti enne ülikooli kroonusse?}

Ülikooli esimeselt kursuselt. Arvasin, et ma ei pea minema, aga tolle aasta viimane võtmine oli detsembris ning mind pandi rongi peale ja läksin.

\question{Kus sa need kaks aastat veetsid?}

Põhikoht oli Rostov Doni ääres sisevägedes ehk siis vangivalvurid. 
Alguses olin mingisuguses isolaatoris, kus oli kolm varianti, mida saab teha 
veest ja hapukapsastest. Esimene oli hapukapsasupp ehk vesi hapukapsastega. 
Teine oli praad ehk hapukapsad ilma veeta. Kolmas oli kissell ehk 
vesi ilma hapukapsasteta. Ma vaatasin, et suren sinna ära, kui pean seal väga pikalt 
olema, ja munsterdasin ennast \emph{utšebka}'sse\sidenote{Õppeväeosa.}, 
natuke luuletasin 
ka. Seepeale saadeti mind Galatši valveseadmete inseneriks õppima. Galatš on see koht, kus Stalingradi kott kokku murti. Nii et ma olen
Volgogradi Venemaa Ema Mamajevi kurgaani peal päriselt lähedalt näinud. 
Hirmus roostes kolakas oli --- kaugelt vaadates ilus, aga lähedale minnes roostes.

\question{Mina küll Vene kroonusse napilt ei jõudnud aga, nohik nagu ma olin, 
kartsin, kuidas ma seal füüsilise koormuse ja keelega hakkama saaksin. Kas sul seda 
hirmu ei olnud?}

See, et üritad ellu jääda, oli igal juhul. Ja selge see, 
et vene keelt koolis ära ei õppinud. Seal aga ei olnud valikut.

\question{Klassikalise haridusega vene proua ju kolme- ja neljatähelisi sõnu
ei õpeta.}

Jah, need on sõnad, mille abil õpid tõepoolest kõiki asju ära ütlema. Seal 
tehti naftast viina. Läksin kord brigaadi töökotta, et juhendada 
järgmist venda, üht Leedu poolakat, kes teadis elektroonikast 
tegelikult rohkem kui mina. Prapporid\sidenote{Praportšik ehk lepinguline 
allohvitser Nõukogude armees.} tulid oma viinapudeliga sinna ja tahtsid, et me selle 
treipingis ära tsentrifuugiksime. Panin pudelile rätiku ümber, treipinki ja 
pöörded peale. Seesama major, kes mind \emph{utšebka}'sse vajas, vaatas kõrvalt 
ja ütles: \enquote{\begin{russian}ты уважай русский язык, ты хот \ldots\ 
скажи!\end{russian}}. See oli päris hull keel, kindlasti mitte tavaline.

\question{Mõned inimesed on rääkinud, et neil oli kolmetäheliste maailmast
keeruline tagasi teadusmaailma tulla. Kas sul seda probleemi ei olnud?}

Kindlasti oli. Sõjaväest tagasi 
tulnud loobiti kõik eraldi kursusele, neid ei lastud puutumatute 
inimestega kokkugi.

\question{Kas sa läksid Tehnikaülikooli tagasi sama asja õppima?}

Jah, sain isegi sama töökoha tagasi, aga siis ühel hetkel läksin
EKTAsse\index{EKTA}\sidenote{Arvutustehnika Erikonstrueerimisbüroo oli 
Eesti NSV Teaduste Akadeemia Küberneetika Instituudi autonoomne osakond}. 
Ektaco\index{Ektaco} on EKTA \emph{spin-off} ja EKTA direktor oli 
Märtin\sidenote{Kaarel Märtin\index[ppl]{Märtin, Kaarel} oli siiski EKTA 
tarkvaraosakonna pealik, tema alluvuses Jaan ilmselt töötaski. EKTA direktor 
oli Kalju Leppik\index[ppl]{Leppik, Kalju}, Ektaco oma Rein 
Haavel\index[ppl]{Haavel, Rein}.}. Ma hakkasin seal FoxPros andmebaase kirjutama.

Üks huvitav kogemus oli käia putši ajal Moskvas. Tegime sealsele 
juveelitehasele 
väärismetallide arvestusprogrammi, aga sel ajal pidi softile autor kaasa 
minema, sest need asjad ei olnud väga töökindlad. Tehase 
osakonna juhataja oli tiba juudi verd. Kui ta kuulis, et 
erakorraline komitee on võimu üle võtnud, ütles ta mulle kohe, et see kõlab 
halvasti, \enquote{vedur teise otsa ja kohe koju tagasi!}. Mina aga olen eluaeg lollustega 
maha saanud või hirmus otse öelnud. Arvasin vastu: 
\enquote{Mis sa jamad, vaata kui hästi teevad
Levitani\sidenote{Juri Borissovitš Levitan oli 
Nõukogude diktor, kelle kanda oli peamiste oluliste uudiste edastamine Teise 
maailmasõja ajal, tema iseloomulikku häält tunti hästi.} järele, 
nagu oleks sõjaajast pärit.} Läksime tänavatele ja need olidki 
BTRe\sidenote{Nõukogude Liidus valmistatud soomustransportöör.} täis ning madin käis. Seal olid 
suured seitsmerealised tänavad, mis olid kõik autodest tühjad. 
Inimesed korjasid sillutisekive ja ehitasid nendest barrikaade. Vennikesed 
hüüdsid mulle veel uhkelt, et vaata, kui kõvad mehed me oleme, ehitame nii kõrgeid barrikaade. Barrikaadid olid aga põlvekõrgused.
Ütlesin neile, et tank T-72 tehnilises 
spetsifikatsioonis on kirjas, et see sõidab 70 kilomeetrit tunnis, kui maapinna 
ebatasasus ei ületa meetrit. 
Juveelimessil rääkis üks korralik proua, kuidas see kõik on nii 
kohutav, ja küsis, mis mina sellest 
arvan. Ma ütlesin, et kõik on ju hästi. Proua imestas: \enquote{Mis mõttes hästi?} 
Ma siis seletasin: \enquote{Vaadake, seni on venelased tapnud kõiki teisi rahvaid, nüüd 
tapavad venelased venelasi.} Populaarsust ma sellega muidugi ei võitnud.

\question{Kas sul igav ei olnud andmebaase treida, tulles matemaatiliselt keeruliste 
asjade juurest? See on ju rutiinne töö.}

Ei olnud, seal oli tegelikult sisendit ja väljundit palju ning pusimist piisavalt, et erinevatele 
inimestele vaated teha. Ja mul olid väga lahedad 
töökaaslased Jüri Freiberg\index[ppl]{Freiberg, Jüri} ja Ülle 
Heinla\index[ppl]{Heinla, Ülle} --- Ahti\index[ppl]{Heinla, Ahti} ema, kes näitas uhkusega poja
tehtud mängu. 

\question{Kaua sa neid andmebaase tegid?}

Ma ei mäleta. Kui ma läksin Ektacosse\index{Ektaco}, tegin seal 
baase edasi. Olin siis just Prantsusmaalt tagasi tulnud ning tegin juba 
niisuguseid baase, kus olid füüsilised asjad ka taga, nagu lukud ja kassad.

\question{Kuidas sa Prantsusmaale sattusid?}

Nõukogude Liidule oli eraldatud 300 stipendiumit ja kui liit 
lagunes, 
siis kuus stipendiumit tuli Eestisse. Kuna sel ajal oli prantsuse keele 
oskajaid suhteliselt vähe, korraldati 
avalik konkurss. TPIst korjati ka inimesi ja sealne prantsuse keele õpetaja 
pani mu naise (kes oli ka 1. keskkoolist) kirja. Aga dekanaat tõmbas ta 
maha, et naine on rase ja kuidas ta sinna läheb. Naine oli suhteliselt kõva 
iseloomuga, et mis see dekanaadi asi on, kas ta on rase või mitte. Otsustasime
saatkonda minna, aga tee peal jäi tal samm järsku 
aeglasemaks ja ta ütles, et kuule, ma olen tõesti rase, mine sina.

Saatkond teatas, et stipendiumi saamiseks 
peab paar tingimust täitma. Esiteks, võimalikult kõrgel õppima. Kuna mul oli 
neli aastat ülikooli seljataga, siis soovitati kohe magistrisse minna. Teiseks soovitati mitte 
Pariisi minna. Kuna Leo Mõtusel\index[ppl]{Mõtus, Leo} oli Toulouse'is 
üks tehisintellektiga tegelev tuttavtegelane, siis läksin 
sinna.

\question{Üheksakümnendate algus oli huvitav aeg tehisintellektiga tegelda, 
see oli ju enne riistvara läbimurret.}

Selle aja peale oli juba igasuguseid asju tehtud: 
produktsioonisüsteemid\sidenote{Produktsioonisüsteem on tüüpiliselt tehisintellekti pakkumiseks rakendatud arvutiprogramm, 
mis koosneb formaalsetest reeglitest, mehhanismidest nende reeglite järgimiseks 
ning süsteemi olekut säilitavast andmebaasist.}, esimesed teoreemitõestajad, otsustuspuud ja muud asjad. Ma ei mäleta, millal 
Rete algoritm\sidenote{Charles L. Forgy poolt 1974. aastal maailmale tutvustatud 
algoritm efektiivseks formaalsete reeglite rakendamiseks.},
produktsioonisüsteemide indeks tehti. Tolleks ajaks oli 
selliseid põhialuseid laotud juba päris palju. Masinõpet vist väga 
ei tehtud ega osatud, nii palju jõudu käes ei olnud.

\question{Kui kaua sa olid Prantsusmaal?}

Aasta. 

\question{Kas sealt hakkas su võrguhuvi tekkima?}

Jah, see oli esimene koht, kus ma internetti nägin. Naine oli veel Tallinnas, 
tema käis Küberis\index{Küber} interneti küljes. Oli niisugune programm nagu talk: 
ühel pool Unixi masinas kirjutad sina ja teisel pool teine. Kuna sel ajal pidi
kaugekõnesid tellima ja see oli keeruline protseduur, siis
talk võimaldas paremini suhelda.

\question{Kas sul sellist mõtet ei olnud, et hakkaks teadust tegema?}

Oli. Aga naine käis mul Prantsusmaal külas ja siis sündis meil teine 
laps ka ning tulin koju tagasi.

\question{Eestis saab ju ka teadust teha.}

Sel ajal ei saanud. Oli üheksakümnendate algus ja lihtsalt raha ei olnud. Pere jaoks oli vaja raha teenida ja kuidagi korter saada. 
Korterihinnad olid naeruväärselt madalad. Sain 
Prantsusmaal kõrget magistristippi ja pool sellest hoidsin kokku 
ning ostsime korteri.

\question{Mida sa Prantsusmaalt tagasi tulles tegema hakkasid? 
Kas programmeerisid jätkuvalt?}

Jah, ikka. Läksin Ektacosse\index{Ektaco} ja tegin lukkude juhtimist. 
Muu hulgas tuli seda teha ühes pullis kohas, Viimsi 
Talveaias\sidenote{Viimsi Talveaed asub Pringi külas ja valmis 1973. aastal 
Kirovi-nimelisele näidiskalurikolhoosile. See kolhoos (ja 
sealsed kolhoosnikud) olid tolle aja mõistes põhjatult rikkad ning Talveaiast 
kujunes Tallinna 
peenema rahva peokoht. Hulludel üheksakümnendatel oli tegu populaarseima 
paigaga, kus 
kiiresti ja kõikvõimalikel viisidel rikastunud inimesed käisid oma rikkust 
demonstreerimas.}. Seal garderoobis oli püramiid, kuhu korjati numbri vastu 
relvad ära. Avamispeo ajal oli lukkudega mingi jama, need ei töötanud. Läksin sinna ja saunapõrandal oli kiht paljaid purjus 
naisi. Väga imelik koht.
 
\question{Mis lukuprogrammeerimises huvitavat oli?}
 
Seal on pusimist, et kõik asjad paika saada. Mul oli ka see 
probleem, et kui Prantsusmaal õpitu kokku võtta, siis oli põhimõtteliselt tegu diskreetse matemaatikaga. Õppisin, kuidas kompilaatoreid tehakse, 
kategooriate teooriat, eri liiki semantikat (loogiline, denotatsiooniline ja 
operatsiooniline). Aga kus seda vaja 
läheb? Tuled tagasi ja raha saad ikka selle eest, kui kellelgi mõne päris 
probleemi lahendad. See lukuprojekt läks hulluks kätte. Kõigepealt 
pidime tegema kassasüsteemi, mille külge tulid lukud, ja nii see pintsaku 
nööbi ümber õmblemine käis. Tellija tahtis hästi palju muutusi, aga ma suutsin andmemudeli kohe niimoodi paika panna, et ei pidanud seda 
pärast enam muutma, ainult juurde tuli panna. Seepeale sain järsku 
aru, et olen midagi õppinud ka.

\question{See vajab päris head rakendusvõimet, et nii abstraktset 
teemat kohe baasi mudelis kasutada. Kategooriate teooria ju ei ütle sulle, 
millised tabelid olema peavad.}

Jah ja ei. Kategooriate teooria õpetab seda, kuidas maailmas 
asjad on korrastatud. Matemaatika point on selles, et see korrastab 
mõtlemist.

\question{Üheksakümnendate lõpus tegelesid juba infoturbe ja -riskidega, 
kuidas sa lukkude juurest selleni jõudsid?}

Mul hakkas igav. Enn Lakspere\index[ppl]{Lakspere, Enn}
läks Küberisse\index{Küber} tööle. Monika\sidenote{Monika 
Oit\index[ppl]{Oit, Monika}} ja Ülo Jaaksoo\index[ppl]{Jaaksoo, Ülo} olid 
teinud turvaseltskonna enne, kui Eesti Vabariigi iseseisvus paistma 
hakkas, sest nad arvasid, et see on strateegiline oskus, mida on igal riigil 
vaja. Ja neil on selles suhtes õigus.

\question{Kas neil oli selline visioon juba tol ajal?}

Jah, neil oli enne iseseisvumist visioon olemas, et iseseisvus 
ühel hetkel tuleb ja selleks ajaks peab kompetentsi olema. Riigi 
infoturve on riigi jaoks strateegiline asi ja see tuleb korda saada.

\question{Mõnes kohas ei ole sellest siiamaani aru saadud!}

Ilmselt ei saadagi. Need olid kindlasti väga suure visiooniga inimesed. 

\question{Ja sa läksid nende juurde tööle?}

Enn Lakspere viis mind sinna. Kokkulepe oli, et mina teen uurimusi ja tema otsib 
tööd. Minu eriala olid kiipkaardid ja teda kaardid huvitasid, kuna ta tuli 
Ektacost, kus kassasüsteemide külge käisid ka kaardid. Mina pidin kiipkaarte 
uurima. Kirjutasingi raha eest uurimusi, kolm lehekülge puhast teksti 
päevas, seda on päris palju.

Vello Hanson\index[ppl]{Hanson, Vello} õpetas mind kirjutama. Osa tema 
õpetustest olen tänaseks küll ära unustanud, aga Vello Hanson on tõsiselt kõva 
vend. 

Kirjutasin näiteks Pankade Kaardikeskuse\index{Pankade 
Kaardikeskus} arhitektuuri. Keskpank tahtis sellele asjale litsentsi anda ja 
menetleda. Aga ma kirjutasin sinna ühe asja, mis oli 
Sildmäe\index[ppl]{Sildmäe, Tõnis} jaoks uudis. Ütlesin, et ärge \emph{settlement}'i ja 
raha liigutamist üldse sinna 
keskusse pange. Võtke lihtsalt info, kes kellele kui palju võlgu on, ja tehke 
bilateraalne \emph{settlement}. Ta käis üle küsimas, 
kas nii saab. Nad hoidsid sedasi paar aastat 
puhast regulatsiooni kokku. Grupivend Margus Aun\index[ppl]{Aun, 
Margus} 
läks seda värki juhendama. 

Ühel hetkel küsis Ülo Jaaksoo minult, kas
kaartidega mässamisest ühiskonnale ka midagi kasulikku teha saab. Kõrval 
oli Ahto Buldas\index[ppl]{Buldas, Ahto}, kes rääkis mulle asümmeetrilisest 
krüptost ja et sellega saab digiallkirja teha. Lugesin selle kohta veel 
kuskilt juurde ja ühel siseseminaril 1994. aastal pakkusin välja, et 
kiipkaardid võiks inimestele kätte anda ja nendega digiallkirja teha. 
Esimene avalik esinemine sel teemal oli Küberis 1995. aastal. Lõpuks müüs
Tarvi\index[ppl]{Martens, Tarvi} selle idee
riigile maha ja nii see läkski. Tarvi oli tegelikult selle asja juurutaja 
ja innovaator.

\question{Kust tuleb legend, et tegu on soomlaste tehnoloogiaga? Või on tehnoloogia
soomlastelt ja idee teilt?}

See ei ole tõsi, soomlaste tehnoloogia ei ole ka originaalne. Kui 
digiallkirja seadust hakati tegema, tellis Tarvi minu käest profiili, missugune 
see kaart peaks olema. Vaatasin ringi, mis kuskil tehtud on, ja rootslastel oli 
kaardi profiil kirjeldatud, nad tegid kolm võtmepaari. Soomlased kopeerisid 
rootslasi ja panid kaks võtmepaari kokku. Eri võtmepaare on vaja 
seetõttu, et neil on täiesti erinev poliitika. Autentimise ja 
krüpteerimise võtmeid ei tohiks kokku panna, sest krüpteerimise võtmel peaks 
olema taaste, kui tahad seda pikaajaliseks säilitamiseks kasutada. Allkirja 
võtmel aga ei tohi olla taastet. 
Autentimisvõtme jaoks ei ole mingit põhjust taastet tahta. Need on erinevad 
poliitikad, mis tegelikult ei sobi hästi kokku, aga nii on lihtsam inimesi 
õpetada. Turbe põhimõte on see, et ainult lihtsad asjad töötavad. 
Jaapanlased võib-olla saavad keerulise asjaga ka hakkama, aga meie ei saa. Ja 
kuna ükski inimene krüpteerida ei oska, siis talle tuleb anda arvuti, mis teeb
seda tema eest. Kiipkaardist lihtsamat arvutit ei ole olemas.

\question{Ja nii jõudsidki infoturbeni?}

Jah.

\question{Infoturve tundub olevat sinu juttu kuulates ideaalne kombinatsioon: 
sai asju ära teha, oli palju matemaatikat ja arvuteid, kõik kenasti koos.}

Küber oli üsna selge teadusasutus: väga palju 
tehti teooriat ja natuke kirjutati ka programmi. Arne\sidenote{Arne 
Ansper\index[ppl]{Ansper, Arne}} oli juba sel ajal kodeerimises kibe käsi.

Ma läksin sealt ära sellepärast, et tekkis tunne, et kirjutan 
igasuguseid plaane ja siis teised mehed plaanide põhjal ehitavad. 
Ühispank\index{Ühispank} oli viimane pank, kellel ei olnud oma 
kaardiserverit, nii et ma läksin seda 
tegema. Ja kuna turvainimesi ka ei olnud, siis pidin olema turbe ja 
maksekaartide peal. Edasi läksin Hansasse\index{Hansapank}. 

Tore oli pangasüsteemi ringitõstmine, kui tuli ühendada kõik 
maapangad\sidenote{Ühispanga asutasid 15. detsembril 1992 kaheksa 
maapanka, Viljandi kommertspank ja Nordpank}, ning selleks oli vaja süsteemi. 
IT-direktor oli 
Novelli-mees. Ma rehkendasin talle \emph{roundtrip} aegadega, mitu 
transaktsiooni ta tänu lukustamistele jõuab üldse päevas teha, see oli kas 30 
000 või 40 000. See tähendas, et tal Tallinnas oleva 
panga jaoks jätkus, aga terve Eesti peale oli vähe. Siis sai Unix sinna alla 
valitud, et lukustamine käiks ühes masinas ära. Tema valis millegipärast HP, aga
tegelikult oli HP-UX\index{HP-UX} väga äge Unix. Inimesed arvavad, mis 
nad arvavad, aga väga robustne riist. Solaris, millest tavaliselt 
räägitakse, oli tükk maad hellem. 

Toona oli Windowsiga selline õnnetus, et TCP \emph{stack}'i eest 
pidi eraldi raha maksma, Linuxi käimapanek oli kaks korda odavam. Seepärast ühendatigi
kõik maapangad niimoodi ära, et pandi Linux \emph{front}'i ja 
ühe ööga keerati kontor ringi, süsteemi vahetus ja \emph{front}'i vahetus. 

\question{Mida sa praegu teed?}

Praegu uurin kriitilisi sõltuvusi, see teema on pärit 
RIA\index{Riigi Infosüsteemi Amet}\sidenote{Jaan oli aastatel 2011---2015 Riigi 
Infosüsteemi Ameti peadirektor} ajast. 
Kui pead vastutama niisuguse asja eest nagu massiivne küberrünnak, siis 
selle juhtimiseks lükkad kokku staabi. Ja esimene küsimus on muidugi, kas see, mida sa näed, on õige. Niisugust infosüsteemist 
sisse tulevat müra, kus pead süsteemi enda käitumist rünnakuks, on 
päris palju. Teine küsimus on, mis edasi 
juhtub ja kust rünnak peale hakkas. Tavaliselt ei oska inimesed kummalegi 
vastata. Minu algne mõte oli, et äkki nendel sõltuvustel on mingi 
võrestruktuur, võre moodi osaline järjestus. Kui leiad
selles osalises järjestuses miinimumi, siis võib see olla algpõhjus. Ja kui võtad transitiivse 
sulundi, saad kõik tuleviku asjad kätte. Nii tekkis mõte hakata pakkuma 
planeerimisabi. Selleks aga on vaja 
need sõltuvused kuidagi kirja panna. Nüüd olen saanud nii palju targemaks, 
et mu arvamus, et seal ei ole tsükleid, ei ole õige. Tsüklid on ja neid on väga 
palju ning väga lühikesi. Kogu majandus on tegelikult tsükleid ja tagasisidet 
täis ning ma ise olen dünaamilisi süsteeme õppinud ja näinud, mida sellised asjad teevad. Ühesõnaga, nüüd üritan neist asjust aru saada.

\question{Kõlab, nagu oleksid asjad jätkuvalt huvitavad ja see on üks väheseid 
olulisi asju. Või eelistad sa igavaid?}

Ei, seda kindlasti mitte, aga mõnikord võivad need liiga huvitavaks minna. 
Keerukus kipub kasvama ja seda peab jõudma
jõuga maha võtta --- \emph{refactoring} on alati töö.


\chapter{Tanel Raja}
\index[ppl]{Raja, Tanel}
\label{sisu:pronto}

\question{Miks sind Prontoks kutsutakse?}

Jäi lihtsalt külge, ma täpselt ei mäleta, mis asjaoludel. Tol 
ajal pidi olema igaühel oma hüüdnimi ja minu nimi oli lõpuks see.

\question{Kuidas sa arvutite juurde sattusid?}

Arvutite juurde sattusin ma enne, kui minust sai Pronto.

Kas sa oled lugenud sellist raamatut nagu „Professor Lillepooli 
kroonika"\sidenote{Herta Laipaiga ulmelugu, mis ilmus kirjastuse Eesti Raamat väljaandes 1982. 
aastal. Raamatu peategelased kohtuvad muu hulgas arvutiga nimega Kunigunde.}? 
Seal toimus kohutavalt põnev tegevus: tüübid tegid oma Musta Kassi 
ordu ja selle käigus käisid vist TPIs ning tegid mingi skeemi 
valmis. See on tagantjärele mõeldes naeruväärne, aga väikse poisina tundus 
tohutult põnev. Sealt tekkiski arvutihuvi. 

Edasi oli kaks liini. Mu onu oli IT valdkonnas tegev 
tükk maad varem kui mina, ta oli juba sügaval 
nõukaajal arvutite kallal. Käisin Tartus tema juures ja see kõik oli kohutavalt põnev.

\question{Mis selle põnevaks tegi?}

Põnevad olid igasugused nupud ja see, et asjad toimusid. 
See oli väikese poisi jaoks unistus, et mingit masinat saab täielikult kontrollida ja et jõud võiks sellest üle käia.

Teine liin oli Tallinnas, Jaak 
Loonde\index[ppl]{Loonde, Jaak} Luise tänava\index{Tallinna 
Oktoobrirajooni Õppetootmiskombinaat} klass, kus olid Yamaha 
MSXid\index{Arvutid!Yamaha MSX}\sidenote{Seal asus Tallinna Oktoobrirajooni Õppetootmiskombinaat, mida on vahel paigutatud ka Roopa tänavale ja mille kohta öeldakse ka lihtsalt Luise tänava klass.}. Sama tüüpi aparaat, mis mul siin külje all 
seisab.

\question{Mis klass see Jaagul oli? Kas see asus mõne kooli juures?}

See oli pigem huvialamaja, ma täpselt ei mäleta. Igatahes sai seal klassis käia arvuti taga istumas ja nikerdamas. Noored nagad tahtsid 
loomulikult hullupööra mängida, aga kurjajuur Jaak Loonde
ütles, et ei-ei, tuleb hirmsal kombel ikka programmeerida. Nii oligi 
tasakaal nende kahe asja vahel üsna hästi paigas, sest niipea kui 
Jack kõrvale vaatas, olid poisid kohe mingid asjad käima tõmmanud. Kui 
keegi lasi võrku mängu, siis said kõik seda endale laadida. 

\question{Kust mängud tulid? Poest neid osta ju ei saanud.}

Klassis oli õpetaja arvuti ja õpilaste töökohad. Nende vahel oli võrk, mis 
oli naljakal kombel üles ehitatud MIDI kaabli otsa. MIDI on küll rohkem mõeldud muusikariistade juhtimiseks – Yamahad olid tegelikult algselt 
muusikaarvutid ja olid läbi muusikasüsteemi kohapeal võrku pandud.

Mängud liikusid kassettidel ja ketastel. Aeg-ajalt käis keegi välismaal. Mäletan, kuidas TTÜs tuli keegi välismaalt 
mingi teadustööga seotud ettevõtmiselt tagasi ja pani lauale kolm kolmetollist flopit. Kõik seisid kõrval ja ootasid, mis 
nende peal on. See oli kohutavalt harras hetk.

\question{Tol ajal ju osta ega alla laadida kuskilt midagi ei olnud, kõik 
käis käest kätte.}

See oli keeruline jah, sest oli sügav nõukaaeg, 
kaheksakümnendate keskpaik. Servast hakkasid vabaduse kiired juba 
terendama, aga need ei olnud kuskilt otsast veel materialiseerunud. Põhiline 
värk oli see, et inimestel lasti teha asju, mille eest varem 
oleks türmi pistetud.

\question{Kas Luise tänaval käisid keskkooli ajal?}

See oli enne keskkooli, olin umbes 13-14aastane – kael juba kandis, aga mitte väga. 
Täiskasvanuks veel ei peetud, selline ebamäärane aeg, kui veel ei
tea, mis sust saab.

\question{Kas ühel hetkel hakkasid tulema BBSid ja Fidonet?}

Need olid tükk aega hiljem. Me kasvasime suuremaks ja sisi hakkasid ka täiskasvanud meiesuguseid nolke 
tõsisemalt võtma. Enne rääkisime põgusalt, et kui nõukaaeg lõppes 
ja Eesti aeg algas, siis esimene palk oli 300 
krooni, mis praegusel ajal on 20 eurot. See oli kuupalk ja sellest elas kenasti ära. Mitte küll nii, et oleks midagi hullupööra huvitavat selle eest 
saanud osta, aga elas ära. See võtab tegelikult nõukaaja 
elatustaseme üsna hästi kokku: kuna asjad, mis meil siin ringi liikusid, olid 
valmistatud oma Normas, Salvos või Kommunaaris, siis olid nii hinnad kui ka sissetulek väiksed. Asjad 
olid tasakaalus. 

\question{Arvutit 20 euro eest ei osta.}

Arvutit tõesti ei osta, aga need vahendid olid olemas asutustel. 
Teine asi oli see, et Nõukogude Liitu oli keelatud eksportida kõvemat 
arvutustehnikat. Et meile tekkisid siia näiteks MSXid, oli osaliselt 
tingitud sellest, et tegu oli suhteliselt alumise otsa masinatega, rohkem 
mänguasjade kui päris tööriistadega. Näiteks kui 
Soomest veeti Eestisse üks 386, kui need oli just äsja välja tulnud, siis sündis
sellest kohutav rahvusvaheline skandaal. Ühendriigid võtsid soomlastel kõri 
pihku ja ütlesid, et mis mõttes te veate Nõukogude Liitu niisugust 
tehnoloogiat, millega on võimalik rakette arvutada ja mida iganes teha! Meie 
mõistes oli see masin tol hetkel väga kõva sõna. Praegu on see muidugi 
naeruväärne, suvaline kell on ka võimsam.

\question{Ehk et arvutile ligi saada, pidi mõnele asutusele külje 
alla pugema.}

Jah. Asutustel olid arvutid, millega nad üritasid oma asjatoimetusi läbi viia.
Arvuteid oli mitmesuguseid, näiteks klassikalisi nõukaaegseid Uralitel 
põhinevaid süsteeme, kus olid terminalid ja suured kastid.
Ajapikku tekkisid ka muud, välismaise päritoluga masinad, peaasjalikult 
286d ning koos nendega võrgud. Seega oli vaja 
inimesi, kes seda kõike haldaksid. Aga inimestest oli põud, kuna 
keegi ei teadnud, mida nende kastidega peale hakata. Ja kui seal kõrval 
nikerdamas käisid, siis muutusid päris kähku kasulikuks. Noore poisina oli 
aega, ei mingeid perekondlikke kohustusi ega muud säärast, huvi oli suur ning 
saidki seal eksperimenteerida. Läksid õhtul pärast kooli sinna, nemad läksid 
töölt ära ja said seal istuda kuni üheksa-kümneni. Ja kokkulepe oli see, et üritad kuidagimoodi kasulik olla, ning 
lõpuks sai arvuti taga istumiskohast töökoht, kui kool läbi sai.

\question{Mis asutuses sina käisid?}

Minul oli alguses linnavalitsus\index{Tallinna Linnavalitsus} ja hiljem 
riigikantselei\index{Riigikantselei} – kogu selle 
huvitava perioodi, kui Eesti Vabariik välja kuulutati, töötasin ma
riigikogu majas. Seda nimetati 
vist peaministri kantseleiks, Stenbocki maja siis veel ei olnud. Seal olid ka 
Uralid\index{Arvutid!Ural}\sidenote{Nõukogude Liidus Pensas aastatel 1956–1964 toodetud arvutite sari.}, 
üüratud kapid, mille sees olid viiemegabaidised trummelkettad, mis tuli 
hommikul käima lükata.

\question{Kas sul akadeemiline haridus jäi pooleli?}

Mul on jah lõpetamata kõrgharidus. 
Üritan seda seniajani lõpetada ja loodetavasti paari aasta 
jooksul seda ka teen.

Tol ajal oli valida, kas tegeled arvutitega või õpid. Suuresti oli mu valik 
väga selge: ma õppisin arvuti taga oluliselt rohkem.

\question{Millal Fidonet Eestis jalad alla võttis?}

Ma täpset aastaarvu ei oska öelda, aga sellega tegutsesid 
Tõnis Reimo\index[ppl]{Reimo, Tõnis}, Tarmo 
Ausing\index[ppl]{Ausing, Tarmo} ja Virko Püss\index[ppl]{Püss, Virko}. Lisaks
jõlkusime seal mina ja Miko Raud\index[ppl]{Raud, Miko}.

\question{Kus te tegutsesite?}

Erinevates kohtades, näiteks Narva maanteel. Eks grupi tuumik 
teab nüansse paremini kui mina. Vahepeal tekkis seal 
võimalusi peaasjalikult soomlastega asju ajada, kui avastati enda jaoks BBSid ja aduti, et tarkvara peab ka kusagilt tulema. Sel ajal
hakkas lisaks flopidele tekkima võimalus modemiga asju alla laadida ja tulid 9600sed modemid.

Asjad hakkasid jumet võtma. Tol ajal olid tarkvarapaketid maksimaalselt paari 
mega baidised ja olid tõmmatavad umbes päevaga.

\question{Seega helistati Soome BBSidesse sisse?}

Jah. 

\question{Kuidas see käis? Jaan Tallinn\index[ppl]{Tallinn, Jaan} on 
rääkinud läbi inimoperaatori arvuti külge helistamisest.}

Igasuguseid imeasju tehti. Näiteks selgus, et lifti 
telefoniühendusest oli võimalik välismaale helistada, sest keegi polnud taibanud 
seda sealt välja lülitada. See tähendab, et liftist sai helistada ja öelda, et 
appi-appi, olen siia kinni jäänud, aga sellesama ühendusega sai helistada ka Soome. Keegi ei olnud nõukaajal kindel, kes 
selle kinni peab maksma, ja seetõttu jäigi see kuidagi ripakile. 
Loomulikult olid ligipääsud erinevatele keskjaamadele ja raha 
tekkis kusagil süsteemides ning kadus kuhugi, nii et 
tegelikult kasutati seda ühte- või teistpidi kurjasti ära. See oli üks viis Nõukogude süsteemi õõnestada.

Sellega seoses tekkisid kontaktid. Näiteks BBSi sisse logides vaatas
\emph{sysop}, et ohoo, Eestist 
mingid tüübid, ja tahtis paar 
sõna juttu ajada. See oli üsna tavaline, et BBSi operaator rääkis külalistega.

BBS ei olnud väga erinev tänapäeva 
sotsiaalmeediast. BBS pandi püsti kahel põhjusel: esiteks, et kontakte luua ja
\emph{networking}'ut teha, olla nii-öelda elu pulsil. Teiseks millegi
propageerimiseks, näiteks oli BBS mõne firma juures või oli 
mingisuguse demo grupp enda oma püsti pannud. See BBS, kust me esimese 
kontakti saime, oli Poison Door\index{BBS!Poison Door}.

\question{BBS võis ka mingi demo grupi juures olla, soomlaste 
\emph{demoscene}\sidenote{Demo on arvutikunsti teos, mis kujutab endast 
terviklikku, sageli väga väikest arvutiprogrammi, mis esitab 
audiovisuaalset vaatemängu. Demo eesmärk on demonstreerida (nagu nimigi 
ütleb) autorite programmeerimise, visuaalkunsti ja arvutimuusika oskusi. Demode 
ümber tekkis kogukond, \emph{demoscene}, mis sai kokku demopidudeks kutsutud 
festivalidel. Üks kuulsamaid on siiamaani regulaarselt Helsingis 
toimuv Assembly.} oli tol ajal väga kõva.}

Mul endal oli kontakt \emph{Future Crew}\index{Future 
Crew}\sidenote{Soome demogrupp, mis peamiselt tegutses aastatel
1987–1994. Nende tehtud oli tõenäoliselt kõigi aegade mõjukaim demo 
„Second Reality“ (avaldati Assembly demopeol 1993. aastal). See tegi 
tänapäeva mõistes olematu riistvara peal reaalajas asju, mis tundusid täiesti 
võimatud, nägi üliäge välja ja sisaldas muusikat, mis siiani kananahka tekitab. 
1999. aastal hääletasid Slashdoti lugejad selle demo kõigi aegade kümne 
vingeima häki hulka.} tüüpidega. Ma laadisin nende BBSist alla niisuguse 
toreda mängu nagu „Wing Commander“\index{Mängud!Wing Commander}. 
Omadele anti asju, mis tegelikult ei olnud
päris ametlikult väljas. Peaaegu kõikidel BBSidel olid tagatoad, 
kus hoiti nodi, mida kasutati vahetuskaubana. Tarkvara oli sel 
ajal kõva valuuta. Me panime püsti kahepoolse ühenduse: mina 
laadisin üles mingi muu asja, mille olin kusagilt saanud, ja 
sealtpoolt tõmbasin vastu „Wing Commanderit“ ning samal ajal sai rääkida ka. 
See tarkvara võimaldas kahepoolset sidet ja samas ka 
\emph{chat}'ida, mis ei võtnud väga palju ühenduse mahtu.

\question{Iga klahvivajutus oli üks sümbol, fondi või värvide 
informatsioon kaasa ei liikunud.}

Just, see oli tavaline tekst. Kogu mängu allatõmbamine võttis 
aega tunde. Selleks ajaks olin endale ise ühenduse sebinud, Riigikantseleil oli
selline võimalus nõukaaja lõpus. 

Ühesõnaga, tutvuti ja info liikus. Ja oli aja küsimus, millal lõpuks siingi 
oma BBS püsti pandi ja Fidoneti kontakt saadi. Ma just hiljuti uurisin selle 
kohta ja paistab, et Fido on nüüdseks lõplikult hinge heitnud.

\question{Üsna kaua võttis aega!}

Võttis küll, aga võibolla on mõttekas see uuesti üles tõmmata. See eksisteerib endiselt ja 
tänapäeval on retroasjad moes, nii et ehk ärkab see kunagi uuesti ellu.

\question{Mis oli Eesti üks esimesi suuri BBSe, kus rahvas hulgakaupa sees 
käis?}

Esimene tõsiseltvõetav BBS, just nimelt Fidoneti mõistes, oli Hackers Night 
System\index{BBS!HNS}\index{BBS!Hackers Night System|see{HNS}}. Nagu 
nimigi ütleb, oli tegu häkkerite öösüsteemiga. Päeval olid telefoniliinid muuks 
otstarbeks, öösel käis nende peal BBSidesse helistamine. 

\question{Miks sel ingliskeelne nimi oli?}

Et oleks rahvusvaheline ja äge.

\question{Kes HNSi käigus hoidis?}
Seesama kamp: Reimo\index[ppl]{Reimo, Tõnis}, Ausing 
\index[ppl]{Ausing, Tarmo} ja Virk\index[ppl]{Püss, Virko}.
 
\question{BBSi jaoks oli ju mingit riistvara ja modemeid vaja?}

Oli jah, sinna juurde käis paras sebimine. Tol ajal olid vahendid suuresti riigi rahakotis. Selle küljes siis 
istuti ja kui oldi juba kasulikud, siis sai alati ka neid ressursse juhtida õiges suunas. 

\question{Mis aastal see oli?}

Kaheksakümnendate lõpus, mitte 1989, vaid varem. Ma täpselt ei mäleta, 
vanus oli selline, et keegi ei olnud veel täiskasvanu, aga ka mitte enam laps. Aeg omas siis teist tähendust ja nüüd hiljem on
raske mõõtkava peale panna.

\question{Kas tol ajal oli BBSil üks modem ja üks liin?}

Ojaa. Tegelikult oli muid ka, paralleelse side katseid, näiteks 
PirnBox\index{BBS!PirnBox}. Fidoneti mõistes klassikalistest BBSidest oli HNS esimene ja sealt läks 
asi krõbinal laiali.\sidenote{Pronto ise pidas BBSi New Age 
System\index{BBS!New Age System} Fidoneti aadressiga 2:490/12.}

\question{Kui palju neid BBSe tipphetkel 
oli?}

Tipphetkel oli 20–30. Süsteem nägi ette \emph{point}'e, mis olid nii-öelda pool-BBSid. Täpsemalt olid \emph{point}'id ja \emph{full 
node}'id. \emph{Node}'il olid kohustused: meile tõmmata, hoida ja 
jagada. \emph{Point}'iga sai lihtsalt tõmmata. Paljud 
BBSid otsustasid \emph{point}'iks olemise kasuks puhtalt sellepärast, et need 
ei saanud ennast kogu aeg käimas hoida. \emph{Node}'idel olid \emph{point}'id, keda nad varustasid 
informatsiooniga, ja \emph{node}'i käimas hoidmine eeldas ühte- või teistpidi 
võimekust olla teatud hetkedel üleval.

\question{Seega oli Eestis tol ajal 
paarkümmend inimest, kellel oli võimekus sebida liin ja riistvara ning ka 
süsteemi käigus hoida.}

Tipphetkel küll jah. Vahepeal sai nõukaaeg otsa ja tuli Eesti Vabariik ning ühel 
hetkel hakkas asi selles mõttes käest ära minema, et raha hakkas omama 
tähendust. Enam ei saanud lihtsalt kusagil ettevõtte küljes istuda ja oma
asju teha. BBS koos telefonikõnedega tekitas kulusid ja peod
hakkasid vaikselt kinni minema. Inimesed vahetasid töökohti ja uutes 
kohtades ei vaadatud selle peale enam lahke pilguga.

\question{Kuidas sellest ürgsupist Eesti arvutifirmad tekkisid? Kas BBSide 
seltskond läks sujuvalt üle teenuste pakkumisele?}

Osaliselt küll. Need inimesed olid ühte- või teistpidi 
arvutifirmadega seotud, aga tihtipeale ei olnud need päris samad 
inimesed. Teatavasti on sogases vees kõige parem kala püüda, seal on kõige 
suuremad purikad. Sogasel ajal leiti erinevaid viise, kuidas endale 
raha teha. Näiteks Peterburist veeti autoga Tallinnasse igasugust IT-tehnikat. Peterburis olid punktid, kust sai asju 
osta ja Eestisse tuua. Nii see elu vaikselt edenes.

\question{Kas sina olid sel ajal veel Riigikantseleis\index{Riigikantselei}?}

Jah, aga oli näha, kuidas hakkasid tekkima esimesed firmad, mõned edukad, 
mõned vähem edukad. Ühel hetkel läksin Riigikantseleist 
minema, sest ka seal toimusid struktuurimuudatused.

\question{Mida sa tol ajal peamiselt arvutiga tegid? Kas kirjutasid 
koodi?}

Nüüd tundub see ehk naljakas, aga siis oli see nagu eluviis. Ega see väga ei erinenudki praegusest eluviisist, vahe on ainult selles, 
et nüüd ei pea näiteks Facebookile ligipääsu saamiseks kulmulihastel ringi roomama. Tollal ei olnud see kõikidele kättesaadav. 

Arvuti kasutamisel oli siis küllaltki kõrge lävi, mis eeldas teatud ülevaadet tehnikast ja võimalustest. Praegu on internet ise ennast 
sõlme tõmmanud, aga varem pidi täpselt teadma aadresse, 
kuhu minna, sest polnud otsinguid. Siis alles hakkasid tekkima esimesed 
otsingumootorid: WebCrawler, AltaVista ja lõpuks Google. Need tõmbasid läve madalaks. 

\question{Mida BBSiga teha sai?}

Sai faile jagada ja kirju vahetada. Fidonet oli tänapäeva mõistes suuresti
interneti meilisüsteemi sarnane. Olid ka uudisegrupid ja \emph{usenet}'i grupid, mis on asendunud näiteks Facebooki ja Redditiga, kus käib 
info vahetamine.

\question{\emph{Usenet}'i grupid olid tollal hierarhilised, aga praeguseks on see struktuur laiali vajunud.}

Jah, olid hierarhiad ja etiketid, mida võhikul oli väga raske 
aduda. Tihtipeale inimesed tundsid küll üksteist üsna lähedalt, aga 
teinekord mõnd jutuajamist jälgides tekkis täieliku \emph{outsider}'i tunne, kui ei saanud aru, millest jutt käib. Kõikidel oli oma taust.

\question{Kus inimesed tuttavaks said? Kas nendes gruppides?}

Oli kaks varianti. Keegi tutvustas ja aitas ree peale või siis kiibitsesid mõnda 
aega ja ühel hetkel hakkasid aru saama, mis toimub. Kui üldse hakkasid, see ei olnud lihtne.

\question{Kui tihedalt Eesti Fidoneti seltskond omavahel läbi käis?}

Seltskond pidi paratamatult läbi käima, sest Fidoneti tekkides moodustusid ka grupid, kus tuli sisu 
tekitada. Ja kuna esialgu oli inimesi vähe, siis paratamatult ei olnud ka
kommunikatsioon meeletult tihe. Fidonetiga tegeles 
paar-kolmkümmend inimest ja isiklikult tuttavaks saamine ei olnud keeruline.

\question{Kes need inimesed olid?}

Enamasti samasugused IT valdkonna inimesed nagu mina, kellel olid
sarnased huvid – meil oli, millest rääkida. Olid ka 
teemad, mis siis olid parajasti \emph{zeitgeist}. 
Näiteks „King's Quest 
IV“\index{Mängud!King's Quest} mängides ei olnud võimalust minna 
veebi ja otsida \emph{walkthrough}'d. Inimesed üritasid omal jõul 
läbi närida ja aeg-ajalt vahetati kogemusi. Muide, sellest ajast pärineb 
ka Habichti raamat \enquote{Selles mängus ei hüpata}\sidenote{Juhan Habichti novellikogumik, mis ilmus 1993. aastal kirjastuse Katherine väljaandel.}. 
See mäng oli 
„Larry“\index{Mängud!Larry}\sidenote{„Leisure Suit Larry“ oli Al Lowe'i
loodud seiklusmängude sari, mis ilmus aastatel 1987–2009 ning oli 
tuntud omapärase huumori ja alaealistele sobimatu sisu poolest. 
Näiteks katsus mängija riiulil seisvat kopratopist ja Larry teatas: \enquote{\emph{I've always 
liked the feeling of a good beaver}}.}.

Samuti räägiti võimalustest ja nende 
vahetamisest. Ühel oli üks asi, teisel teine ja pandi seljad kokku. Kuna 
inimesi oli vähe ja üksteist teati, siis ei olnud ka väga 
suurt kanakitkumist.

\question{Kas trollimist või muud säärast ka toimus?}

Kui auditooriumi ei ole, siis inimesed jäävad tavalisteks inimesteks. Kui 
annad normaalsele inimesele anonüümsuse ja publiku, siis saab tast igavene 
tõpranahk.

\question{Isegi kui sul oli \emph{handle}, siis sa ju tegelikult ei olnud anonüümne.}

\emph{Handle} oli lihtsalt nimi, tegelikult kõik teadsid, kes on kes. 
Isegi kui olid anonüümne, siis ei 
kasutatud laest võetud nimesid. Kui olid oma nimele
feimi tekitanud, siis sa ju ei tahtnud sellega 
uisapäisa ringi käia.

\question{Sa oled siiamaani Pronto ja teistele tähendab see senini midagi. 
Kui keegi hakkas sigatsema, kas ta visati siis välja?}

Jah, juhe tõmmati seinast ja olid kohemaid \emph{persona 
non grata}. Kuna see oli seotud sinu enda huvidega ning
mineviku, oleviku ja tulevikuga, siis ei saanud seda omale lubada.

\question{Seega käitusid kõik viisakalt?}

Kõik olid seal paadis võrdsed. Kui keegi hakkas paati 
kõigutama, siis ta kõigepealt kõigutas seda enda all ja kui ta seda jätkas, siis ta lihtsalt eemaldati paadist ja pidi ise vaatama, kuidas 
veekogus hakkama saab.

\question{Kas seda juhtus ka?}

Otseselt mitte või kui juhtus, siis juba hilisemal ajal. 
Alguses oli ikkagi tihe seltskond ja kuigi kõik ei saanud omavahel 
ideaalselt läbi, mõisteti, et selles paadis ollakse koos. Seetõttu tüli põhjustada võivaid teemasid
lihtsalt välditi.

\question{Seega saadi aru, et teatud asjadest ei tasu rääkida.}

Jah. Trollimine ju ongi rääkimine asjadest, mis teisele 
inimesele peavalu valmistavad.

\question{Tuleme sinu juurde tagasi. Kui sa 
Riigikantseleist\index{Riigikantselei} ära tulid, mida sa siis tegid?}

Töötasin sellises kohas nagu Marvin-Ekspert, sain seal ostmise ja müümisega käe valgeks. Tegelesin selliste toodetega nagu Gravis Ultrasound ja 
IOMega\sidenote{Gravis Ultrasound oli toona PC-maailmas tipp helikaartide tootja ja IOMega tegeles väga innovatiivsete andmesalvetuslahendustega.}.

See oli selles mõttes huvitav aeg, et Gravis Ultrasound maksis väikse 
varanduse, aga samas oli see tükk maad parem kui mõni teine toode. Müüsin neid umbes sama palju kui kõiki 
ülejäänud asju kokku müüdi, kuigi see oli kallis. Mõnes mõttes oli sellega sama lugu nagu 
Apple'iga: kallimat asja on alati lihtsam müüa, sest kalliduse taga on tavaliselt 
väärtus, toode ei ole kallis niisama.

\question{Mis aastal see oli?}

Ilmselt 1994. või 1992. aasta kanti.

\question{Kas sel ajal hakkas tasapisi Microlink tekkima?}

Microlink tekkis tegelikult üsna aegade alguses. See oli üks nendest firmadest, kes 
alustas sellest, et hakati kotiga Peterburist asju tooma. Esialgu müüdi 
arvuteid firmadele, sest nendel oli raha, kuigi 
omandisuhted polnud veel päris paigas.

\question{Kapitalismi oli veel vähe?}

Kapitalismi oli jah vähe, olid veel nõukaaja jäägid --- keegi oli kuskil käpa peale 
pannud. Oli niisugune aeg, kui ma paljusid asju ei teadnud ja 
paljusid teadsin, aga ei tahtnud teada. Asju, mille kohta võib öelda, et mis juhtus Vegases, las jääb Vegasesse. Sel ajal tehti 
igasuguseid asju, mis praegu võivad näida küsitava 
eetilise ja moraalse taustaga, kuid siis olid 
tegelikult õiged ja vajalikud.

\question{Tol ajal ju kujuneski välja, mis on õige ja mis mitte.}

Jah. Loomulikult tehti sel ajal igasugust erastamist ja ärastamist, aga ka see oli 
hädavajalik puhtalt sellepärast, et tookord tehtud otsused eristavadki meid 
tänapäeva Moldovast, kus omal ajal tehti teistsuguseid otsuseid. Isegi 
need, kes meil siin ärastasid, tegid seda teataval määral 
\enquote{eesmärk pühendab abinõu} kaalutlustel.

\question{Räägi pisut ka ajakirjast .EXE\index{Ajakiri!.EXE}.}

See ajakiri oli osaliselt Microlinki püüd ennast nähtavaks 
teha. Eestis oli tollal kaks arvutiajakirja: 
Arvutimaailm\index{Ajakiri!Arvutimaailm} ja .EXE. Arvutustehnika \& 
Andmetöötlus\index{Ajakiri!Arvutustehnika \& Andmetöötlus} ei olnud klassikalises mõistes ajakiri, vaid rohkem 
vihik. Nõukaaja lõpus ja Eesti aja alguses anti välja vihikuformaadis 
erialaväljaandeid, mis ei olnud mõeldud laiaks tarbeks.

.EXE tekkis umbes samal ajal kui Arvutimaailm, Microlink 
püüdis tekitada endale laiatarbeväljundi.

\question{Selles ei olnud palju laiatarbeasju, 
vaid stiilipuhas \emph{hard core} küberpungijutt!}

.EXE oli selles mõttes \enquote{laiatarbeväljund}, et sel ajal ei olnud inimestel 
raha arvuti soetamiseks. Pidi olema ikkagi väga suur tahtmine ja vastavalt sellele kujunes ka ajakirja sisu. 

Sel ajakirjal oli ajastu hõng juures. Mida inimesed arvutiga parasjagu tegid, 
see sealt ka läbi kumas. 

\question{Kuidas sa selle juurde sattusid? Kas kirjutasid juba enne .EXEt?}

Tol ajal kirjutati näiteks naljaviluks mängudest, sisu toodeti vabatahtlikult. 
Gruppidesse postitati dokke, häkiti ja nii edasi. Kuna ma olin mängudest kirjutamisega silma 
paistnud ja ka kirjaoskus enam-vähem olemas, siis nii ma ajakirja sattusin. 

See oli päris 
naljakas aeg – ajakirja koostamine oli omamoodi 
häkkimine. Tavalist kogunes kolleegium (seltskond, kes sisu kokku pani) 
kokku, lükati ette kaks kasti õlut ja enne toast välja ei 
lastud, kui ajakiri oli kokku pandud. Igaüks võttis endale mingid kohustused 
ja kadus nendega tegelema.

\question{Kaua .EXE üldse ilmus?}

See ilmus umbes poolteist aastat.

\question{Nii vähe?}

Jah, ma ühel hetkel korjasin kõik numbrid kokku\sidenote{Aadressil 
\url{punktexe.ee} on kõik ilmunud numbrid täies mahus olemas.}. Esimene number ilmus aprillis 1993 ja viimane 
1995. aastal. Nii et kaks aastat, vahepeal läks ilmumine eklektiliseks.

\question{Kes neid ilusaid kaanepilte joonistas?}

Kaspar Loit\index[ppl]{Loit, Kaspar} alias BKnows.

\question{Arvestades, milline mõju ajakirjal oli, palju seda loeti ja kuidas fännati, 
siis oli ilmumise lühidusest hoolimata tegu väga mõjuka asjaga.}

Jah, numbreid oli kokku vist kaheksa. Igaüks oli omaette šedööver, kuna see oli südamega tehtud, eriala inimestelt eriala inimestele. .EXEt anti välja selleks, et skenet juurutada, mitte et selle pealt 
üüratut kasumit teenida. 

\question{Millist skenet? Arvutiinimeste oma?}

Jah. Inimesele tänavalt
oli see ajakiri ehk pisut raskevõitu. Tol ajal oli 
arvutiajakirjandus teistsugune kui praegu, mil
igaühel on arvuti ja loetakse, kuidas oma mobiiliga 
ühte, teist või kolmandat teha. Arvuti oli siis suur asi, 
seda polnud kaugeltki mitte kõigil. Praeguses mõistes üks-kaks protsenti inimestest tabas tegelikult reaalselt arvutit ja oskas seda 
igapäevaelus kasutada.

\question{Seevastu inimesi, kes tahtsid kasutada, oli rohkem. Ja nii nad lugesidki hardalt, kuidas Pronto seikleb „Day of the 
Tentacle'is“\index{Mängud!Day of the Tentacle}\sidenote{Legendaarne mäng, mis ilmus 1993. aastal LucasArtsi väljalaskel ja uuendatud graafikaga 
2016. aastal ning mille \emph{walkthrough} avaldati .EXE teises numbris 
novembris 1993, autoriteks BKnows\index[ppl]{BKnows} ja 
Pronto\index[ppl]{Pronto}.}}.

Eks see kõik hakkaski pisitasa tuult tiibadesse võtma. Sel ajal toimus 
jõhker inflatsioon ehk räägitud kahekümnest eurost said päris kiiresti 
sajad eurod. Arvutid muutusid jõukohaseks ka teistele ja Rootsist veeti siia humanitaarabi korras pruugitud tehnikat.

\question{Kas kogu selle ajal jooksul müüsid sina muudkui Gravist?}

Gravist ja Iomega Bernouilli draive\sidenote{1992. aastal turule tulnud, oma aja kohta suure mahutavuse ja 
eemaldatava kettaga salvestussüsteem Bernouilli Box 
oli Iomega esimene laialt kasutust leidnud toode.}, QIC-80 
teipe ja muud säärast.

\question{Huvitav, et mitmed inimesed on teatud faasis tegelnud just 
arvutustehnika müügiga.}

Kuskilt tuleb raha teenida. Kätte jõudis aeg, kui varad said laiali 
jagatud ja sa pidid oma tegevust põhjendama, näiteks miks sul on 
BBS. Ainuke võimalus seda asja edasi edendada oligi müügi 
egiidi all.

\question{Kas koodikirjutamisega ei saanud elatist teenida?}

Tol ajal ei olnud eriti mingeid koode, mida kirjutada. Väikseid asju loomulikult oli, aga valdavalt käis koodikirjutamine 
andmebaaside ümber, näiteks olid FoxBase ja DBase, kus tehti 
ettevõtete raamatupidamist ja inventuuri. 

\question{Kas iga ettevõte pusis endale ise tolle rakenduse kokku?}

Kas ise või osteti firmadelt, aga süsteem koosnes tavaliselt 
mõnest andmebaasilahendusest. Oli ka muid asju, 
näiteks meditsiiniga seotud lahendusi, millel olid juba infosüsteemid, aga 
need olid väga spetsiifilised ja neid arendati enamasti väikses mahus.

\question{Eestlane üldiselt ei ole suurem asi müügiinimene, 
aga IT-asja on meil õnnestunud rahvusvaheliselt päris hästi müüa. Kas ehk
seetõttu, et kriitilisel hulgal inimestel on olnud müügikogemus?}

Kindlasti. Tol ajal oli see paratamatu, sest kui tahtsid 
saada ligipääsu, pidi juba siis ennast müüma. See on üks asi, mis on muutnud 
vana kooli IT-vennad teistsuguseks – sa pidid paratamatult suutma müüa. Kui ei suutnud, siis polnud sul IT valdkonda asja. Kõige 
tähtsam kaup olid sa ise.

\question{Sest muud sul ei olnud?}

Muud ei olnud, isegi mitte kogemusi, sest kogemused tulevad töö käigus. Sa pidid suutma endast teha väga vajaliku tegelase.

\question{Nii et kui enesemüügi oskus on olemas, siis võib igasuguseid 
asju juhtuda.}

Kui tähelepanelikult vaadata, siis IT valdkonna müügis ongi 
läbimurrete taga tihtipeale ühed ja samad inimesed ning just 
vana kooli kaader, kes enamasti on oma läbimurde ehk müügi saavutanud mitte 
tänu avalikkusele, vaid vaatamata sellele. Teatavasti tunneb avalikkus 
kohemaid muret, kui keegi teenib paremini või tunneb ennast kuidagi paremini. 
Hari läheb kohe kadedusest punaseks.

\question{Tihti öeldakse, et meil on vedanud, sest õiged inimesed on sattunud 
õigetesse kohtadesse. Sinu jutust tuleb välja, et tollest seltskonnast tulidki 
inimesed, kes sattusid õigetele kohtadele.}

Täpselt nii. Need inimesed on siiani 
alles, osa neist üle viiekümne, osa alla selle, aga üks või teine on suuremate 
läbimurrete taga.

\question{Oskad sa öelda, mis 
hetkel kaotas see maailm oma süütuse? Kui romantilisest õllekasti abil 
toimetamisest sai raha teenimine.}

Ma ei oska seda niimoodi paika panna, sest tegelikult on see 
ikkagi suuresti väljaspool loodud kuvand. Kui on mingisugune grupp, 
siis paratamatult tekivad autsaiderid, kes tunnevad pahatahtlikku 
kadedust. 

\question{Ja nimetavad inimesi häkkeriteks?}

\enquote{Häkker} hakkas omandama lihtsalt teistsugust tähendust.

\question{Viidates ühele .EXE loole, mis on 
küberpunk\sidenote[][-2cm]{Allkirjastamata, kuid BKnowsi\index[ppl]{BKnows} piltidega 
lugu \enquote{Kes sa selline oled, küberpunk?} ilmus .EXE kolmandas 
numbris 1994. aasta aprillis. Seejuures tuleb tunnustada artikli asjakohasust: nii ilmumise (eba)regulaarsuse ja lühiduse kui ka kultusliku staatuse poolest .EXEga sarnane, kuid suurema levikuga ajakiri MONDO 2000 (aastatel 1984–1998 ilmus USAs 17 numbrit) avaldas oma samateemalise satiirilise artikli \enquote{R.U. A CYBERPUNK?} oma 10. väljaandes 1993. aastal.}?}

Kõik asjad, mis on punk, nagu aurupunk, küberpunk või diiselpunk, on lihtsalt 
žanr, mis läbib mitut asja; valdavalt seda, kuidas siduda teadvus 
tehnikaga. Mõnes mõttes on meie ühiskond praegu nii-öelda küberpungi jaoks 
esimesel tasemel, sest see, kui inimesed istuvad ninapidi telefonis, on 
lihtsalt liidestamise küsimus. Inimesed on ennast tegelikult arvutiga juba väga 
intiimselt liidestanud.

\question{Nagu sa mainisid, siis algas see juba kaheksakümnendate lõpus, kui 
kogu sinu elu oli arvutis. Lihtsalt liides oli kandilisem.}

Liides oli kandilisem ja olemas vähestel inimestel; seetõttu polnud see elu, vaid 
mu \emph{alter ego}. See ongi üks põhjus, millepärast valiti omale sellised
tunnused, nagu mul on Pronto – et teha vahet sellel, mis toimub arvutis ja mis 
niisama. Põhimõtteliselt loodi endale identiteet. 

\question{Just nimelt loodi, mitte ei valitud!}

Ja sellega elati osaliselt tulevikus, aga ka muu elu jäi alles. Pere, sõbrad ja see õlu, mida joodi, jäi kõik teise ellu.

\question{BBSi rahvas käis ju koos ka.}

Käis küll. Kõigepealt olid \emph{sysop}'ide saunad ja muud üritused, kust kasvas välja Fidonet. Hiljem tekkisid
BBSummerid\index{BBSummer}.

\question{Kui palju neid toimus?}

Need said alguse nõukaaja lõpus ja neid toimus üksjagu. Üks 
BBSummeritest, vist teine või kolmas, lükati edasi sellepärast, et tankid sõitsid Eestisse 
sisse.

\question{Olen näinud BBSummeri pilte, mille peal on kõik Micolinki, Skype'i, Unineti ja 
teiste hilisemate suurte asjade alustajad. Kas tol ajal, asja sees 
olles, ei olnud niisugust tunnet, et oi, küll me oleme ägedad?}

Muidugi oli! Me olimegi hullult ägedad! See oli ka üks põhjus, miks me sellega tegelesime.

\question{Tulles meie jutu alguse juurde tagasi, kas selle ägeduse tuum oli 
jätkuvalt see, et sai masina mõne näpuliigutusega oma tahtele allutada?}

Kindlasti. Teiseks ei piirdunud elu enam oma õuega, vaid koos Fidonetiga tekkis ka ülejäänud maailm sinna otsa. See ei 
olnud väga erinev tänapäeva Redditist, Facebookist või Twitterist, kus ei
saa suhelda mitte ainult paari lähema tuttavaga, vaid kogu ülejäänud 
maailmaga. See andis 
näiteks võimaluse keeli omandada ja suhelda erinevates keeltes, mis omakorda aitas edasi.

\question{Nii et see tekitas maailma avardumise tunde?}

Maailm avardus kindlasti. See oli mõneti samasugune tunne nagu 
kosmonaudil, kui ta atmosfäärist väljub. Eriti kui see pind, millelt üles 
tõusti, oli tükk maad madalamal kui enamiku maailma jaoks --- me tegime
otse nõukaajast sammu tulevikku.

\question{Ühel hetkel olid Nõukogude pioneer ja pisut hiljem 
vestlesid California kuttidega keskjaamadest.}

Jah, absoluutselt. Tekkisid võimalused ja kogemused. Näiteks mõnes mõttes 
positiivne nähtus oli see, et Eestis puudusid \emph{legacy} süsteemid, meil 
polnud IT valdkonnas mineviku taaka, vaid asi oli lihtsalt poolik. 
Mineviku taaga puudumine võimaldas Eestil kihutada päris kiiresti 
päris kaugele võrreldes ülejäänud maailmaga, kes pidi oma asju käimas hoidma. 
Me oleme nüüd jõudnud sinnamaani, kus meil on oma taak tekkinud ja peame 
sellega tegelema.

\question{Lõpuks ikka saab inerts otsa, aga seni on see meid päris kaugele 
vedanud.}

Seda sai üsna hästi ära kasutatud just nimelt sellepärast, et õigel hetkel sattusid õiged inimesed pumba juurde ja saagi tõmmati 
käima nii kaua, kui jõuti, enne kui ärimehed jaole jõudsid. 

Kuna oli hulk inimesi, kes tegid midagi, mis oli 
müstiline, keeruline, käsitamatu ja ilmselt ka veidi elitaarne, siis loomulikult hakkasid 
tekkima needki, kes hakkasid kaikaid kodaratesse pilduma. Inimesed, 
kes tahtsid ka löögile pääseda ja tundsid ennast halvasti, et neid ei 
võetud jutule puhtalt sellepärast, et nad ei saanud aru paadi mittekõigutamise 
mentaliteedist. See oligi mõnes mõttes ajastu lõpp, kui igaühel 
tekkis ligipääs, lävi läks palju madalamaks ja ka lühemate pükstega mehed said 
paati astuda.

Tekkisid inimesed, keda keegi ei teadnud, kes olid anonüümsed ja kellel olid 
ambitsioonid, aga puudusid võimekus ja soov panustada. 

BBSummerid hakkasid samuti kasvama ja kihistuma. Ürituste lõppu tähistas see, kui hakkasid toimuma BB-üritused BB-ürituste sees. 

\question{Mida sa praegu teed? Kuhu see tee sind on toonud?}

Praegu olen juba viimased kümme aastat tegelenud veebipoodidega. Minu eriala on 
veebiarendused, täpsemalt veebipoed ehk e-kaubandus. 

Olen nüüd rohkem programmeerimise peal, sest tänapäeval on 
peaaegu kõik ühte- või teistmoodi seotud tarkvaraarendusega. Tol ajal ei 
olnud firmadel internetilehte nagu praegu. Tol ajal ei pakutud teenuseid 
interneti kaudu, aga nüüd pakutakse. Ja seega on tekkinud vajadus tehnilise võimekusega inimeste järele. Üks võimalus on värvata nad 
endale või siis palgata firma, kes sellega tegeleb.

\chapter{Priit Raspel}
\index[ppl]{Raspel, Priit}


Minul on selline Jaapani ajaarvamine, mille järgi ma asju paika panen. Tean
mingeid sündmusi, mille suhtes ma teisi määratlen. Näiteks 
1993. aastal käisin maailmameistrivõistlustel. 

\question{Mille maailmameistrivõistlustel?}

4GL\sidenote{Neljanda põlvkonna programmeerimiskeel.} 
programmeerimise maailmameistrivõistlustel. Käisime seal kolmekesi: mina, Veiko 
Herne\index[ppl]{Herne, Veiko} ja Tiiu Lumberg\index[ppl]{Lumberg, Tiiu}. 
Tulime viiendaks ja saime veel eripreemia kõige elegantsema lahenduse eest. 
Mõtlesime välja Amazoni, mida tol hetkel ei olnud veel olemas. 

Ja selle järgi teangi, et samal ajal hakkasin ära tulema 
Innovatsioonipangast\index{Innovatsioonipank}. 

See jaapani \emph{native} ajaarvamine käib suurte sündmuste vahel, näiteks 
keisri võimule tulemisest kolmas aasta. Ja kui tuli suur maavärin, siis 
see keiser unustati ära. Teati küll, et maavärin oli keisri võimuletulekust nii mitu aastat hiljem, aga edaspidi arvutati aega maavärisemise järgi. 
Sedasi on jube hea, sest muidu ei pane asju enam liini peale. 

\question{Kuidas sina arvutite juurde said?}

Mina olen juhuslik inimene. Keskkoolis olin 
täiesti veendunud, et see on viimane roppus, mida ma õppima lähen. 

\question{Mis aastal see oli?}

Keskkooli lõpetasin 1979. aastal. Ma käisin Kusti-koolis ehk \mbox{1.
keskkoolis}\index{Tallinna 1. Keskkool}, praeguses Gustav Adolfi Gümnaasiumis. Tol 
ajal oli seal matemaatika-füüsika eriklass ning arvutid fakultatiivselt õppekavas sees. 

\question{Kas juba 1970ndate lõpus?!}

Jah. Meile õpetati programmeerimist Fortranis\index{Fortran} ja pagan 
teab milles. Arvutis käisime Teaduste Akadeemias\index{Teaduste Akadeemia} 
Lenini puiesteel (praegu Rävala puiestee), seal nurga peal, kus on ka
raamatukogu. Arvuti oli Minsk-32\index{Minsk!Minsk-32}, mis 
koliti siis just välja ja toodi asemele vist ES-1022\index{ES EVM!ES-1022}. Mäletan, kuidas seda Minsk-32 välja koliti -- terve alumine klaasfuajee oli 
tükke täis. Ühe klassivenna onu, kes oli seal programmeerija, ütles, et 
arvuti visatakse prügimäele. Me käisime sealt plaate välja koukimas, sest 
seal oli P13 transistor, mis oli kõige defkam 
ja igasuguste asjade tegemisel kasulik.

\question{Nii et sul oli juba siis elektroonikahuvi?}

Jaa, esimese raadio panin kokku vist kuueaastaselt. Isa oli ju raadiotehnik. Kodus vedeles jube palju
juppe ja raamaturiiulis oli raamat \enquote{Noor 
raadioamatöör}\sidenote[][-2mm]{Noor raadioamatöör, 
Viktor Borissov, tõlkinud Arnold Isotamm, Tallinn: Eesti Riiklik Kirjastus, 
1953.}, kus oli kirjas, kuidas detektorvastuvõtjat teha. Eks ma siis 
nihverdasin isa sahtlist mõned dioodid, tegin pooli, panin kõrvaklapi külge 
ja sain mingi Majaki\sidenote{1964. aastal käivitatud üleliiduline raadiojaam, mis
tegutseb siiani.} kätte. Majak oli muidugi nii võimsa signaaliga, et see oleks tulnud ka 
pliidiraua pealt. Kui pliit oleks pilti näidanud, oleks pilti ka tulnud. 

Õppimisega oli selline värk, et meile õpetati jube kehvasti. Õpetajaid ei 
olnud saada ja igal poolaastal luges erinev õpetaja mingisugust täiesti erinevat 
asja, mida ta parasjagu ise just oskas. Metoodika puudus, aga vähemalt saime käia 
Teaduste Akadeemia arvutis. Seal olid perfokaardi 
\enquote{kivipurustajad}, mis lõid auke läbi. Saime oma pakid ühte kappi panna ja nii palju ma ikka tegin, et see tundus päris huvitav. 

Olin täitsa kindel, et lähen elektroonikat õppima. Tegin poistele 
raadiosaatjaid ja ükskord isa küsis: \enquote{Poiss, kas see on sinu töö, et
peilingaator sõidab akna all?} Muidugi oli minu töö. \enquote{Näita, mis sa 
tegid? Kurat, sul on sihukesed lõpptransid, me peame võimsuse tagasi tõmbama!} 
Elasime Tõnismäel ja segajad olid sealsamas\sidenote{Nõukogude Liidus oli 
komme välismaiseid raadiojaamu sihipäraselt segada. Üks selleotstarbelistest 
raadiojaamadest (need allusid sideministeeriumile), nr 602, asus Tõnismäe 
ligidal Luha tänaval.}. Vennad püüdsid signaali kinni ja tulid otsima, kus see on. 
Lõpuks rehkendasime 300 meetri peale, kust lähemale ei tohtinud minna ja 
kui läks, siis tuli ruttu ära kaduda. 

Nii et ma tinutasin igasuguseid asju kokku. Siis juhtus selline jama, et mul tuli 
üks üsna raske haigus, mis viskas mul korra aastas 
teadvuse ära. Ravi 
oli keeruline: pool aastat üpris kangeid tablette täpse 
režiimi järgi ja kui ei mõjunud, siis pidi pool aastat pausi pidama. Haiguse äraminek võttis kolm aastat aega, nii et veel keskkooli lõpus olin haige 
ja elektroonika õppimaminek oli vastunäidustatud, kuigi ma tahtsin just seda teha, sest olin lapsest peale aparaate 
ehitanud. 

Olin noor vihane inimene ja mõtlesin, et ei lähe kuskile. Ei taha, 
mul pole vaja, ma lähen tööle elektroonikuks! Kolb püsib käes, skeemist saan 
aru, oskan isegi skeemi koostada, montaažplaate teha, telekat 
käsikaudu parandada (selles mõttes, et ilma igasuguste mõõteriistadeta leian 
vea üles ja parandan ära), raadioga saan hakkama. Aga isa rääkis augu pähe. 

\question{Mida su ema tegi?}

Ema oli Eesti Raadio\index{Eesti Rahvusringhääling!Eesti Raadio} 
majandusjuhataja, tema viis mind sealse tehnokeskuse meestega kokku. Poes 
polnud ju suurt midagi. Koostasin listi kõigest, mida ma poest ei saanud, 
läksin tehnokeskusse ja sealt leiti mulle. Ükshaaval küsisin juppe ja kuskilt 
karbist need mulle ka leiti. Ja kui nad teinekord midagi maha kandsid, siis 
andsid need tükid ka mulle. Mul on praegugi üks karbitäis asju alles.

\question{Kas sa kokkuvõttes leidsid, et pead \emph{midagi} ikkagi õppima minema?}

Jah, isa arvas, et võiksin ikkagi õppima minna. Ütles, et ole nüüd kaval, esimesel 
aastal on üldained -- mine õpi midagi sellist, kus eksamid on samad. 
Läksin ITsse, mis oli siis majandusliku informatsiooni 
mehhaniseeritud töötlemise organiseerimine -- Eesti kõige pikema nimega eriala 
üldse. Kusjuures selles suhtes vahva nimega, et sõjavägi ei saanud 
aru, keda me koolitame. Terve selle eriala jooksul ei läinud mitte ükski poiss 
ohvitserina pärast kooli sõjaväkke, sest nad ei taibanud, et seal koolitatakse
puhtaverelisi programmeerijaid.

\question{Nii et sa pole Vene kroonus käinud?}

Ei ole. 

Meil oli lahe grupp, kus oli ka kuus kooliõde-venda, ja seltsielu läks kohe käima. Kokku oli meid 25, poisse ainult kuus, sest meil oli 
majandusteaduskonna grupp ja sinna teaduskonda tuli palju ilusaid ja tarku tüdrukuid, kes programmeerisid ka päris kõvasti. 

Läks poolteist kuud ja ma olin müüdud mees.

\question{Mille peale see sul juhtus?}

Ma nägin selle maailma ilu -- neid võimalusi, 
kuidas kõik, mis kõrvade vahel olemas, on võimalik ka päriselt. Ega see 
mul lihtsalt ei läinud, sest Gustav Adolfi Gümnaasiumis õpetati 
neid asju väga hüplikult. Mulle on eluaeg meeldinud süsteemne lähenemine, 
aga seal räägitut ei pannud keegi minu jaoks süsteemi ja 
tol ajal ju ei olnud kohta ka, kust lugeda. Internetti polnud ja ka
raamatuid ei olnud eriti võimalik saada. 

Üks sõber, Lembit Sammel\index[ppl]{Sammel, Lembit} vedas mu Leo 
Võhandu\index[ppl]{Võhandu, Leo} jutu peale kolmanda ühika alla, kus olid Nairi-2\index{Nairi!Nairi-2}, AP keel\index{AP keel} ja 
elektriline kirjutusmasin Konsul. Hakkasime APs kirjutama: tegime biorütme ja 
silusime neid ning tegin ka oma esimese mängu, mis oli tiku äravõtmine -- masin mängis vastu selle peale, kes võtab 
viimase tiku, ja sain sellega päris ilusti hakkama. Vaat 
sellesama mängu kirjutamise ajal see asi haaraski mind, sest ma ei 
saanud sellest päris lõpuni aru. Mulle on jäänud see eluks ajaks meelde, kuidas istusin ühel vihmasel oktoobriõhtul laua 
taga, paber ees, lamp põlemas, ja kirjutasin blokkskeemi. 
Pusisin ja pusisin, aga ei tulnud, ja ühel hetkel käis täiesti kuuldav nips ja sain 
aru, mida ma pean tegema. Kirjutasin üsna kiiresti algoritmi valmis ja 
pärast seda ei ole mul algoritmide kirjutamisega mitte mingisugust probleemi 
olnud. 

\question{Kas sa sattusid üsna varsti tööle ka?}

Teisel kursusel, sest esimesel kursusel kammisin ikkagi ühika vahet. Meil oli esmaspäev vaba ja hommikul 
läksin nii vara kohale, kui ühikaalune klass lahti tehti, ning tulin alles siis tulema, kui mind jõuga 
välja visati. Lembit Sammel, hüüdnimega Sass, tegi täpselt samamoodi. 
Panime nagu paaris härjad. Ei olnud päeva, kus me mõnda masinat nässu ei 
keeranud, sest kui neid pidevalt piinata, siis need põlesid läbi. Arvutiklassis sain tuttavaks ka Lindre Reinuga\index[ppl]{Lindre, Rein}, kes oli 
seal inseneriks ja kellega me hiljem tegime koos ühe vahva asja. 

Teisel kursusel läksin tööle EPTsse ehk Eesti 
Põllumajandustehnikasse\index{Eesti Põllumajandustehnika}. Keskkontor asus Salve 
tänaval, kus olid Minsk-32\index{Minsk!Minsk-32} ja 
ES-1022\index{ES EVM!ES-1022}. ESi numbriga ma võin eksida, aga 
Minsk-32 oli küll. Suured saalid olid mürisevaid seadmeid täis. 

Sain seal operaatoritega hästi läbi ega pidanud enam TPI arvutuskeskuses päev otsa perfokaardiga jamama, et see 
järgmisel päeval kuskilt kapist kätte saada. Läksin tüdrukute 
juurde ja ütlesin, et kuulge, laske mu pakk läbi. 

\question{Kas sind võeti sinna mõnd konkreetset asja programmeerima?}

Ei, lihtsalt otsiti inimest, kes oleks noor ja avatud ning keda nad saaksid ise
oma käe järgi välja õpetada. Oleg Kase\index[ppl]{Kase, Oleg} oli selle tiimi 
juht, kus tegutses näiteks väga geniaalne programmeerija Tõnu Toomus\index[ppl]{Toomus, Tõnu}, kes läks kahjuks 
Estoniaga minema. Nad 
hakkasid mind õpetama ja alustasin lihtsamatest asjadest. Esialgu tegin
alltöid, mingeid funktsioone ja värke, mida neil vaja oli. Aga üsna pea 
jõudsin selleni, et tahaksin ise midagi teha. Selle peale öeldi, et vali 
ise. Kuna seal oli parasjagu igasuguste 
kasutajaliideste tegemine, laoarvestuse ja kõige muu viimine suurest ESist suurde 
SMi\index{SM EVM!SM-4}, siis ma tegin vormi 
generaatori: joonistasin ekraani, sidusin andmebaasiga ära ja siis
MUMPSi\sidenote[][]{Massachusetts General Hospital Utility 
Multi-Programming System -- transaktsiooniline võtme-väärtuse andmebaas, 
millega on integreeritud ka programmeerimiskeel. Selle süsteemi puhul oli 
fookus jõudlusel (selle kaudu käib tänini rohkem kui poolte USA patsientide 
terviseinfo), mitte loetavusel; kõiki käske võis lühendada ja reavahetus ei 
olnud oluline. Tulemuseks oli sageli raskesti loetav kood.} süsteemipuuga ära ja valmis.

\question{Ma olen MUMPSist lugusid kuulnud. Kui tänapäeval oled õpetatud 
programmeerija, siis MUMPS on hoopis teisest maailmast!}

Ja, see on endiselt täitsa olemas, jäi mulle ükspäev kogemata 
internetis jalgu. 

MUMPSis ei ole indekseid, need peab ise tegema üle inverteeritud 
immituste. Kuna seal on puu, siis peab puu võtme, \emph{path}'i kirja 
panema ja registreerima, et nüüd on selle kohta võti olemas. Võtmega
võib \emph{path}'i järgi objektile otse peale minna. 

MUMPSis on näiteks niimoodi, et kui tahad mõnd seadet kasutusele võtta, 
siis pead teadma seadme numbrit. Näiteks printer oli 
vist 80. Kirjutad \verb|U:80|, mis tähendab, et nüüd läheb kõik ülejäänud jama, mida 
väljundisse paned, printeri peale. Kui tahtsid kuvarit saada, siis 
igal kuvaril oli oma number. Kuna need olid füüsilised masinad, siis tegid 
andmebaasi kõigepealt loendi olemasolevatest kuvaritest ehk nimetasid need
ära. Seejärel öeldes näiteks \verb|U:1|, sattusid 
esimese kuvari, konsooli peale. 

Lisaks oli võimalik anda \emph{wait}-aegu ehk \enquote{oota nii kaua ja siis mine edasi}, aga ma ei mäleta, mis sümbol 
seal vahel oli. Üldiselt oli selle programmeerimiskeel enam-vähem loogiline. Kuna sel oli 
hierarhiline, puukujuline andmebaas all, siis see pani muidugi omaette 
pitseri, sest \emph{pre-} ja \emph{post-order} ning muu säärane 
pidi hästi käpas olema. 

\question{Kas EPT-l oli igal pool kontoreid?}

Jah, näiteks Sauel, Paides, Tartus, igal pool 
tugevad inimesed eesotsas. Tiimid olid väikesed, neli inimest. Muidugi tehti siis kontoris suitsu 
ja joodi kohvi. Kuvarid olid eraldi ruumis, arvuti teises ruumis, 
igaühel oli ühel pool kuvarit tuhatoos ja teisel pool kohvitass. Kohv oli täiesti 
must ja suhkruta, sest piim läheb ju hapuks, seda ei saa kuskil 
kapis hoida, ja suhkurgi saab otsa, mis sest ikka osta. Nii et seal ma õppisingi musta 
kohvi jooma ja suitsu tegema, mis on mind tükati mitmeid aastaid 
saatnud. Sõltuvust mul ei ole, võin jätta suitsetamise 
maha niimoodi, et panen paki lauaservale, tikutopsi peale ja seal see 
seisab, mind see ei häiri. 

\question{Kuidas kontorite vahel side käis?}

Side käis kahtemoodi: kas ümbrikuga\sidenote{Tõenäoliselt peab Priit silmas tavalist postiteenust.} või läbi teletaibi kanali. Igas EPT 
kontoris oli teletaibi aparaat. Kirjutusmasinaga teletaip oli teksti edastamiseks ja selle küljes oli ka perfolindi lugeja. Meie 
masinast lasti perfolint välja ja söödeti teletaipi ning teisel pool, näiteks 
Paides, lasti lint välja ja söödeti sealsesse masinasse. Aga ega see nii 
lihtsalt ei käinud, seal oli protokoll ka, sest side ei olnud püsiv 
ja kippus ära kukkuma. Siis tegi teletaip piiksu. Selle 
peale tõstis inimene telefoni, helistas teisele osapoolele ja ütles: 
\enquote{Kuule, tõstan kümme kirjet tagasi.} Teisel pool tõstis operaator õla 
üles ja sättis perfolindi tagasi, vastuvõtja tõmbas lindile poole pastakaga 
joone. Seda võis mitu korda juhtuda. Ja kui lint oli lõpuni jõudnud, siis 
vastuvõtja lõikas lindid märgitud kohast katki, võrdles, kus asi kokku langeb, 
liimis otsad kokku (spetsiaalne rakis oli, millega augud läbi torgati, et need 
puhtad oleksid) ja söötis lindi masinasse. 

\question{Ühesõnaga elektrooniline seade muutis andmed kõigepealt 
pabermeediasse, siis elektroonilisse meediasse, siis uuesti pabermeediasse ja 
lõpuks tagasi elektroonilise meediasse. Ja veaparandus oli manuaalne!}

Jah. Tekkis neli koopiat linte: üks, mis lasti siitpoolt välja ja võeti sealpool 
vastu, ning teine, mis lasti seal sisse ja võeti siin vastu. 
Töötas suurepäraselt. Aga siis õpetas Leo Võhandu\index[ppl]{Võhandu, 
Leo} mulle andmeedastust ja protokolle. Läksin meie keskuse peainseneri Rolandi 
juurde ja ütlesin: \enquote{Roland, kas saad mulle 
sellise seadme teha, mis paneb selle arvuti ja selle teletaibi kokku, ja 
teisele poole samasuguse vastuvõtmiseks?} Ma ei teadnud, et see on modem, siis 
ei olnud sellist asja olemas. Tema ütles: \enquote{Oot, ma vaatan, ma just Radio ajakirjas (tolleaegne tehniliste nikerdajate ajakiri) nägin ühte skeemi.} Ja tegigi valmis. Montaažplaat oli 50 x 50 
cm, kuna see oli SMi sisemine plaat, läks nagu riiul 
sisse. Aga skeem oli nurga peal 10 x 10 cm. Ta pani selle 
käima ja mina otsisin vahepeal mööda opsüsteemi, mis võimalused on. Leidsin 
ühe struktureerimata ala -- eraldad lihtsalt mälu ja 
struktureerid ära. Ehitasin sinna peale kataloogisüsteemi ja kirjutasin linte väljastavad
programmid niimoodi ringi, et need kirjutasid 
kataloogisüsteemi, mitte ei saatnud, ja ka seda, kellele saata. 
Esialgu ei olnud mujale saata kui Paidesse, kuigi perfektsionist kirjutab ikkagi adressaadi ka juurde, 
sest mine tea, kellele on veel vaja saata. 
Üks programm vaatas aeg-ajalt, kas on linte tekkinud, ja helistas teise poole välja. Kui side 
kukkus, siis tõstis 10 kirjet tagasi, teine teadis seda ja hakkas uuesti saatma. Probleem oli selles, et sa ei teadnud, millisel 
hetkel side kukkus ehk ei olnud võimalik määrata, mis kirjed olid ära läinud. 
Üks saatis minema, teine võttis vastu ja võrdles, kus hakkasid samasugused kirjed 
tulema, ning loksutas paika. Asjad läksid täitsa ilusti üle. Teinekord kui side ei taastunud, siis võis seanss olla terve päev katki.

\question{Mis tempos see andmeside toimus?}

See võis olla 100--300 bitti sekundis, kiirem küll olla ei saanud. 

Side kiirenes siis, kui läksime elektroonseks. Kõik läks hästi niikaua, 
kuni Tartu ütles, et nemad tahavad ka. Ka nüüd läks kõik esialgu 
hästi, kuni juhtus niimoodi, et Paide helistas mulle peale, side kukkus ja 
siis helistas Tartu peale. Aga ma ei teadnud, et see on Tartu. Üritasin vastu 
võtma hakata, aga sealt ei tulnud midagi tuttavat. Siis mõtlesime välja sessioonivõtme ehk ühe \emph{hash}'i. Enne seansi
tekitamist saatis programm \emph{hash}'i ette ja nüüd oli teada, et 
kui see uuesti tuleb, siis sellesama \emph{hash}'iga, nii et võis ka segamini 
vastu võtta. 

\question{TCP\sidenote{Üks interneti alusprotokolle.} käib põhimõtteliselt samamoodi.}

See on jah vahva, et tunned pidevalt asju ära. 

Ühesõnaga, ma töötasin seal kuni TPI lõpuni, kokku neli aastat ja täitsa huvitav oli.

\question{Kas sa lõpetasid ülikooli töö kõrvalt nominaalajaga? Töö koolis käimist ei seganud?}

Ei! Tööst oli palju kasu, ma olin teistest kogu aeg peajagu üle, sest olin 
saanud kõike seda, mis meile õpetati, elus katsetada. EPT seltskond lubas 
mul väga vabalt toimetada ja ütles, et kasuta kõike, mida saad, peaasi 
et on hea. 

Me tegime Eesti esimese \emph{online}-messi 
arvutitega ja vedasime ise pool kilomeetrit kaablit posti otsas Saue mõisa. Tol 
ajal müüdi hooaja lõpus kõik varuosad maha ja see käis tavaliselt 
niimoodi, et kaubatundjad istusid suurte paberite taga ja tõmbasid maha, mis 
müüdud sai, aga meie vedasime Saue mõisa side ja panime viis kuvarit üles. Peainsenerid ja kolhoosiesimehed said ise 
arvutist valida ja asju selekteerida. Töötas suurepäraselt. Meil oli väga innovatiivne 
kamp.

EPTs oli üks tuntud mees, Tõnu Lume\index[ppl]{Lume, Tõnu}, kes mängis filmis 
Lurichit\sidenote{Tallinnfilmis 1984. aastal valminud film \enquote{Lurich}.} ja 
oli EPT arvutuskeskuse juhataja asetäitja. Juhataja oli Jaak Raja\index[ppl]{Raja, Jaak}, karm mees, aga minusse suhtus hästi. 

\question{Miks sa EPTsse pikemaks ei jäänud?}

Tuli kooli lõpp ja suunamine. Tänapäeval keegi ei teagi, mis on suunamine. 
Tol ajal tehti hinnete põhjal pingejärjekord ja said valida, kuhu lähed. Aga nimekirjas oli kaks kohta, mis 
olid spetsiaalselt mulle -- kui mina neid ei valinud, siis keegi teine neid 
ka valida ei saanud. Üks oli EPT ja teine TTÜ, mis oli sel 
ajal muutunud põnevaks kohaks, kuna sinna hakkas välismaist
tehnikalt tulema, sealhulgas personaalarvuteid. 

Käisime koos Sven 
Jürgensoniga\index[ppl]{Jürgenson, Sven} esimesi personaalarvuteid rongiga Moskvast toomas. Need olid kaheksabitised Yamahad, 
Z80 prosega.

\question{Mis ametikohta sulle pakuti?}

Juhtivinseneri kohta, see oli 
teadusliku uurimise sektori ametikoht infotehnoloogia kateedri 
all\index{Tallinna Tehnikaülikool!Infotehnoloogia kateeder}. Lisaks pakuti head 
palka, kuigi ega EPTs ka kehv palk ei olnud. Tollal oli hea kuupalk 
120 rubla, mina teenisin 155 rubla. Tegelikult aga läks elu 
kehvemaks, sest õppimise ajal elasin ma ikka väga priskelt: sain lisaks
60 rubla kõrgendatud stippi, EPTst poole koha eest 60 rubla ja teadusliku uurimise sektorile\index{Tallinna 
Tehnikaülikool!Teadusliku Uurimise Sektor} tehtud tööde eest 40 rubla. Lisaks maksis EPT 
60, vahel isegi 100 rubla kvartalipreemiat. Kokku mingi kakssada rubla kuus! Aga nüüd oli mu palk 155 
rubla. Mul läks tükk aega, enne kui kõik liinid tööle sain ja teadusliku uurimise 
sektor hakkas mulle lisa maksma. 

\question{Mida see teadusliku uurimise sektor endast kujutas?}

TPIs tehti kõik lepingulised tööd teadusliku uurimise 
sektori alt, kes sõlmis lepinguid ja võttis vahelt oma obroki. 

\question{Mida sealt telliti?}

Igasuguseid asju, näiteks kriminalistika infosüsteeme. 

Ma valisin TTÜ ja läksin Raja Jaagule\index[ppl]{Raja, Jaak} 
lahkumisavaldust viima. Tema ütles: \enquote{Priit, jää ikka
poole kohaga tööle. Sa ei pea kogu aeg käima, astu vahel läbi ja 
ütle, mis arvad.} Nii ma töötasingi kaks-kolm 
aastat veel seal. Käisin ikkagi kohal, sest niisama raha vastu võtta
südametunnistus ei lubanud. 
Täitsa huvitavaid asju sai veel tehtud. Aastaid hiljem, kui ma seal enam ei töötanud, olid SMi 
matused viina ja kartulisalatiga. Masin ise oli maha müüdud, aga 
protsessorikast maeti kuskile Sauele maha. 

\question{Sa oled ainus inimene, keda ma tean, kes on päriselt laulu sisse pandud.\sidenote[][]{Ansambli Folkmill 1996. aasta albumi \enquote{Paksult 
rahul} populaarses avaloos \enquote{Madis Mäekalle valss} on salm: \\
Üks talv oli see, jube libe oli tee,\\
Madis mütaki istuli kukkus.\\
Aga igav oli maas, seltsiks vaid kaevukaas,\\
Madis ohkas ja tudile tukkus.\\
Siis ühmatas Raspeli Priidu,\\
kes kunagi ei kiskund riidu:\\
\enquote{Sa aja end, Madis, nüüd püsti\\
ja tunne end pagana hästi.}} Kuidas sa sinna sattusid?}

Lauri Saatpalu\index[ppl]{Saatpalu, Lauri}\sidenote{Folkmilli 
laulja ja käilakuju.} on minu hea sõber ja tal on niisugune komme, et kui tal 
millestki muust enam laule pole kirjutada, siis ta hakkab sõpradest kirjutama. 

\question{Kust sa teda tead?}

Käisime Lauriga EÜEs\index{Eesti Üliõpilaste 
Ehitusmalev}\sidenote[][]{Tagantjärele vaadates nõukogude aega oma vaimsuse, 
suhtumise ja ärimudeliga hämmastavalt halvasti sobitunud, tudengite jaoks 
organiseeritud suvise töö tegemise vorm. EÜE organisaatoritest, legendaarsetest 
trubaduuridest, sõpruskondadest ja suhetest on hiljem nii mitmeski 
valdkonnas suuri asju võrsunud.} ja oleme koos mitmeid laule teinud. 
Üldiselt tulevad tal sõnad hästi, aga on ka juhtunud, et ei tule, ja siis ma olen 
katalüsaatorina töötanud. Olen ise ka maleva jaoks laulusõnu teinud. 

Lauriga kohtusime esimesel Tiirimetsa suvel, aastat ei mäleta. Hakkasime kohe hästi läbi saama, 
ta on vaimukas inimene. Serbati Tom ja 
Mäekalle\sidenote{Tegelased viidatud laulus.} on kõik reaalsed 
inimesed.

\question{Nii et sa oled muusikamees ka?}

Jah. Ma olen õppinud muusikat päris palju, alustasin kuueaastaselt ja 
õppisin suisa neli aastat muusikakeskkoolis\index{Tallinna Muusikakeskkool}, 
aga siis sain aru, et ma ei peaks seal olema. Seal midagi muud ei õpetatud, aga mul olid muud asjad ka tähtsad. 
Pealegi olen ma natuke rutiinitalumatu nagu infotehnoloogid ikka -- kogu aeg peab \emph{action} käima, sama pala kaheksakümnendat 
korda mängida oli piinav. Tegelikult ma ei tahtnud seda katki jätta, aga seal oli üks solfiõpetaja, kes mind terroriseeris. Ta on 
kõiki terroriseerinud, aga mina olin tal eriline lemmik ja see lõi mu lukku, ma 
ei saanud solfiga hakkama. Ja tulingi ära. Ütlesin emale, et lähen hoopis
laste muusikakooli\index{Tallinna Lastemuusikakool} klarnetit 
õppima. Seal sattusin Aleksander Rjabovi\index[ppl]{Rjabov, Aleksander} 
juurde, kes on Eesti džässi suurkuju ja väga hea õpetaja. Solfiõpetaja
Porrason oli ka kuldne inimene. 
Selgus, et mul on kõik oskused olemas, ainult et need olid lukus. Mul 
on absoluutne kuulmine, mitte küll kõige kõrgemal, aga täiesti arvestataval 
tasemel. See on elus ka veidi piinarikas -- nii kui midagi 
valesti kõlab, siis kohe kratsib. 

Õppisin ka laulmist, nii et poistekoor pluss eraldi ansamblitunnid andsid kõva laulmiskooli. Hiljem õppisin ise saksi ja kitarri 
juurde, klaverit ka natuke. Ma ei ole ammu mänginud, aga klarnet ja saks on 
nii käes, et need tuleb ainult kastist välja võtta. Mul on 
kapis üks siinkandi paremaid klarneteid. Iseenesest on kahju, et see minu käes on, aga kuna see on kingitud pill, 
siis ei saa seda ära anda. Selle kinkis üks 
Eesti välishelilooja sünninimega Elmar Rossman\index[ppl]{Rossman, 
Elmar}, kes on Priit Ardna\index[ppl]{Ardna, 
Priit|see{Rossman, Elmar}} nime all kirjutanud \enquote{Kuldrannakese}. Käisime nädalavahetusel just Ugalas, kus ajaloomuuseumis
on tema ooperi reklaam seina peal. Väärt inimesed on elust läbi 
käinud.

\question{Tuleme korraks tagasi tehnikaülikooli juurde.}
 
Tehnikaülikoolis sattusin 
Toomas Mikli\index[ppl]{Mikli, Toomas} juurde, kellega saime väga hästi 
läbi. Ta oli väga keeruline tüüp, temaga oli raske rääkida. 
Seda suutsid suhteliselt vähesed inimesed, sest ta jättis umbes kolm loogilist 
taset vahele ja alustas neljandalt ning sa pidid ise puuduvad kihid vahele 
ehitama ja mina suutsin seda. 
Tema suutis panna mind andmebaasidest innustuma ja oli 
mu diplomitöö juhendaja. Diplomitöö oli meil muide 300 lehekülge. 

Natuke uhkustan ka, et aastal 1984, kui keegi selle peale veel ei mõelnud, oli mul üks osa
diplomitööst pühendatud konsultatiivinfole ehk \emph{help}-tekstidele. 
Tegelesin tööl metoodilise palgaarvestusega, kus muu 
hulgas õpetasin ja juhendasin kasutajat, mismoodi süsteem töötab. 

\question{Täitsa innovatiivne mõte tol ajal!}

Tollal jah keegi sellest veel suurt ei rääkinud. Korjasin selle teema
Tomiga vestluse käigus üles ja tegin ära. 
Diplomitöö kirjutamise käigus sain veel ühe asjaga hakkama. TTÜs kasutati 
SETORi\sidenote{Varastel kaheksakümnendatel liikvele läinud TOTALi 
andmebaasisüsteemi kloon ESide jaoks.} ehk andmebaasi, mida ülejäänud 
maailm tunneb nimega Total\sidenote{Ka TOTAL. 1968. aastal asutatud Cincom 
Systems Inci andmebaasimootor, mis oli esimene omasuguste seas.}. Arvutitel 
oli mälu vähe, 256 KB, millest 16 kilo jäi puhvrisse, kui kuvarid taga 
olid, ja nüüd ei mahtunud enam kompilaatorid ja linkurid mällu ära. Diplomitöö 
käigus kirjutasin skripti, mis vaatas programmis järele, milliseid teeke 
vaja on, ja linkis ainult need teegi osad sinna külge, mida tõepoolest vaja 
oli. Sedasi oli võimalik 16 KBga hakkama saada. Veel mõtlesin välja puhverdamissüsteemi, kuidas läbi puhvri erinevaid mooduleid siduda, sest 
suur tükk ei mahtunud korraga mällu. 

Tom pani kokku grupi, kuhu kuulusin mina, Mart Roost\index[ppl]{Roost, Mart} 
(praegu tunnustatud õppejõud), Lea 
Elmik\index[ppl]{Elmik, Lea} ja Tiiu Lumberg\index[ppl]{Lumberg, Tiiu}. Meid hakati kutsuma \enquote{Mikli noorteks 
ekstremistideks}. Me kõik kirjutasime oma teadustööd, aga me ei teinud 
kunagi midagi nii nagu teised. 

Moskvas oli üks kaval juut Tjomov, kes istus kuskil instituudis 
Iskra-226\index{Iskra!Iskra-226} peal, mis oli laetava BASICuga arvuti, ja
kirjutas opsüsteemi Skoropis ehk kiirkiri. See oli esimene viitadega keel, 
mida ma nägin. Tal oli \emph{time sharing} ilusti sisse ehitatud. Programmi täitmine 
käis nii, et tõmbasid programmi stringi, panid viida peale ja ütlesid, et 
selle viida järgi hakkad nüüd täitma. Mälu oli jälle vähe, 64 KB, aga meil 
tekkis Lindre Reinuga\index[ppl]{Lindre, Rein}, kellega me 
arvutisaalis tuttavaks saime, mõte panna Iskrale veel kaks kuvarit külge. Mina 
kirjutasin opsüsteemi ringi, tema tegi kaks videokaarti, panime Videotoni kuvarid 
taha ja vaatasime, kas hakkab tööle. Selleks ma tegin 
\emph{overlapping}'u -- kui tundsin ära, et programm on 
juba mälus, siis lisasin teise viida ja panin selle veel kord tööle. Programm visati 
välja alles siis, kui viitasid enam ei olnud. 

\question{Kas mälukaitse või turve ei olnud probleem?}

Ei. Kogu infoturve seisnes selles, et masinat ei saanud käimagi, flopi oli välja 
võetud ja tuba käis lukku. Kuigi oli olemas ka kahemegane ketas, mis nägi välja nagu
suur valge \emph{baraban} -- mul on praegugi kapi otsas kaks tükki, üks 
Iskra ja teine SMi oma.

Mul on seal kapi otsas terve muuseum, näiteks lint ja kolmesajane 
modem (nimega Nightingale, laksutab nagu ööbik) ja üks 
esimesi läpakaid, mis Eestisse tuli ja mis oli Siim Kallase\index[ppl]{Kallas, Siim} oma, 
kui ta oli Eesti Panga president. Lisaks arvelaud, lükati, kaheksa-, viie- ja kolmetollised 
flopid, magnetoptilised kettad -- 
ühesõnaga kõik, mis elus ette on jäänud. Kõige vanem eksemplar, taskukalkulaator, on pärit 
aastast 1936.

\question{Kas Feliks?\sidenote{Nõukogude Liidus aastatel 1920--1970 toodetud 
mehaaniliste kalkulaatorite sari, mille tootmise algatas Nõukogude 
julgeolekuteenistuse asutaja Feliks Edmundovit{\v s} Dzer{\v z}inski, mistõttu laienes 
tema hüüdnimi Raudne Feliks ka kalkulaatoritele.}}

Feliks on ka. Aga see kõige vanem on sakslaste tehtud mehaaniline numbrinäiduga taskukalkulaator, mis liidab ja 
lahutab ning on umbes 3 mm paks ja 6 x 10 cm suur. Vanaisa kinkis selle isale kuuendaks 
sünnipäevaks ja isa kinkis mulle. 

\question{Kas sa tehnikaülikoolis teadust ka tegid?}

Jaa, ma hakkasin tegelema andmeedastusega, aga tulid segased ajad, raha sai 
otsa ja see jäi seisma. Tegelesin sünkronisatsioonimudelitega, millega olen 
elus hiljemgi väga palju tegelenud, ja praegu võiks nendest kirjutada sellise
töö, mida keegi pole kunagi välja mõelnud. Aga nüüd on mul muud huvid tekkinud\ldots 

\question{Mis on sünkronisatsioonimudelid?}

Nende põhimõte seisneb selles, et kui on kaks infosüsteemi, siis millist mudelit 
kasutada, et kõige odavamalt välja tulla, ja mismoodi see automaatselt käima 
saada, et nad süngis oleksid. Tollal ma mõtlesin välja ühe termini, mida on hakatud minu suureks rõõmuks tänapäeval kasutama -- 
\enquote{automaagiline}. Kasutasin seda kunagi ühel konverentsil ja Jaak Tepandi\index[ppl]{Tepandi, Jaak} küsis, mida ma selle all mõtlen. See on asi, mis muutub automaatselt, aga ma ei 
tea täpselt, mismoodi. Ja nüüd kasutatakse seda reklaamideski. 

Meil tekkis 
Reinuga\index[ppl]{Lindre, Rein} oma rühm, sest Žiguli autotehas AutoVAZ soovis meilt süsteemi väikejaamade jaoks -- 
ladu, remont ja muu säärane. Ütlesime, et teeme küll, aga omamoodi, 
meil peab teadus sees olema. Kirjutasimegi 
neljakesi nullist süsteemi, mille loogikat poleks tänagi häbi näidata. Mart\index[ppl]{Roost, Mart} kirjutas 
andmebaasi mootori, Tiiu\index[ppl]{Lumberg, Tiiu} vormi generaatori, Lea\index[ppl]{Elmik, Lea} raporti generaatori ja 
mina süsteemi arhitektuuri kirjelduse ning mõtlesin välja ka
tolle aja mõistes XMLi. Suurem-väiksem märgi asemel olid kandilised sulud ja 
\emph{slash}'i asemel sõna \enquote{END}, aga keel oli sama. 
Tõestus on olemas ühes TPI kogumikus, kuhu ma kirjutasin selle kohta artikli. 

Mul oli väga lihtne tõsta asjad seal keeles ringi ja süsteem hakkaski 
teistmoodi menüüsid ehitama ning igasuguseid küsimusi küsima.

\question{Kas nad võtsid süsteemi kasutusele ka?}

Jah, me kasutasime seda AutoVAZi jaoks ja hiljem mujalgi.

\question{Kas ühel hetkel sukeldusid pangandusse?}

Selleni läks aega, enne toimus see
maailmameistrivõistlus. 

Ühel hetkel sai raha otsa ja palka sai 
TPIst\index{Tallinna Tehnikaülikool} nii palju, et kui auto oli olemas, 
jaksasid autoga tööl käimiseks bensiini osta. Sain tuttavaks niisuguse huvitava mehega nagu Veiko 
Herne\index[ppl]{Herne, Veiko}, kes praegu elab Euroopas nii-öelda kodutuna. Ta tahabki seda ja see 
ei tähenda, et ta halvasti elaks, vaid ta rändab ringi. Tema eluunistus oli olla vaba. Ja nüüd ta kirjutabki mobiiliäppe pargis. Kui kuskil on 
põllumajandusperiood, siis läheb põllumajandusse tööle ja aeg-ajalt paneb 
feissarisse, kus ta käinud on. Välimuselt on ta minu täielik vastand: 
pisike, kõhetu ja ümmarguste prillidega. Geniaalne programmeerija ja orgunnimeister. Ta kutsus mind tarkvara tegema oma loodud firmasse
OÜ Tarkvara ja andis kolmandiku osakuid mulle.

\question{Kui OÜ, siis pidi ajahetk olema 1991.}

Millalgi siis jah. Ühel päeval ütles ta: \enquote{Kuule, hakkame tõsiselt 
tegema -- ma leidsin Microsoft Magazini sabast ühe süsteemi nimega 
Gupta\index{Gupta} SQLBase\sidenote{Tegu oli esimese relatsioonilise kliendi-serveri 
andmebaasiga, mis jooksis PC platvormil, mitte mini{\-}arvutitel.}. 
Nad pakuvad, et hakkaksime nende esindajaks, ma käin korra Inglismaal.} Mina 
olin TPIst selleks ajaks otsad juba lahti võtnud. Üks asi, millega me raha 
teenisime, oli Robotroni nõelprinterite ümberprogrammeerimine eesti 
tähestiku peale. Sellest tekkis natukene 
algkapitali. Inglismaa-sõidu ja lansseerimise 
peale läks ilge raha, 100 000 rubla, aga kuidagi me selle kokku 
kraapisime. Igal juhul Veiks tuli Inglismaalt tagasi ja olimegi esimese kliendi-serveri süsteemi ametlikud esindajad Eestis. Siis ei 
olnud veel Oracle'it, Cybase'i ega kedagi. Korraldasime seminari ja tuli ainult 
vilistada -- terve Küberi amfiteater oli inimesti puupüsti täis. 

Tegime lepingu Põlva Piimaga ja Võrus oli 
eksperimentaalne õmblustootmiskoondis, kellega sõlmisime süsteemide 
arenduslepingud. Uurisime süsteemid välja ja panime andmebaasid käima. Põlva 
Piim oli väga suur projekt, seda me ei hallanud enam kolmekesi ära, nii et võtsime 
Andres Lombi\index[ppl]{Lomp, Andres} ja IE-tarkvara\index{IE-Tarkvara} appi 
programmeerima. 

Mul on õudselt hea nina igasuguste vigade peale. Leidsin SQLBase'ist ühe laheda vea, et kui tingimused
\verb|IN| ja \verb|NOT IN| olid järjest, siis täitusid suvalised 
tingimused. Ja kui panid sinna vahele \verb|1=1 AND|, siis hakkas tööle. 
Vennad ei uskunud seda ja kaks tükki tulid suisa kohale. Korraldasime ruttu seminari ja panime nad esinema. Näitasin neile enda tehtud asju ja kuidas me oleme nende 
süsteemi kasutanud. Saime nendega täitsa \enquote{kuuma liini}. Ükspäev teatas Veiko\index[ppl]{Herne, Veiko}, et 
Gupta\index{Gupta} otsib endale esindajat 
maailmameistrivõistlustele 4GL programmeerimises ja kas lähme. Guptast öeldi, et te 
olete nii kõvad vennad küll, minge, aga ise peate oma arvutitega Rootsi jõudma. 

Oli selline väga tark soome poiss nagu Pauli Visuri\index[ppl]{Visuri, 
Pauli}, kes müüs Olivettisid. Tänu talle tõi
üks Rootsi Olivetti esindaja meile messiboksi tuttuued masinad, meie 
lihtsalt sõitsime lennukiga kohale. Tahtsime ööbida mingis tagasihoidlikus 
kohakeses, aga Gupta ütles, et ei, meie meeskond ööbib ainult Kung 
Carlis, maksame selle teile kinni. 

Läksime sinna ja nägime esimest korda elus 66 MHz Suprema masinaid, millel oli peal Plug-n-Play, Windows 3.11. Hakkasime installima, aga ei õnnestunud, hiir ei läinud külge. Arvasin, 
et seal on Microsofti meeskond, panin käed puusa ja teatasin: 
\enquote{See teie opsüsteem on igavene pask! \emph{Plug and play} küll, aga hiired külge ei 
lähe!} Kaks venda istusid meie masina taha ja hea oli vaadata, kuidas 
ini-failid\sidenote{.INI laiendiga failides hoiti Windowsi platvormil 
tavakohaselt programmide konfiguratsiooni.} lendasid näppude alt 
välja. Lasid-lasid ja üks masin läks käima. Ajasin nad minema, 
kopeerisin ini-failid kõikidesse masinatesse ja oligi korras. Veiko hakkas 
proovima häältuvastust, mis oli just välja tulnud, aga kuna messihalli helifoon oli 
väga kõva, siis ta karjus oma arvuti peale: \enquote{Õupen, õupen, õupen, 
klõus, klõus, klõus, ran!} Järsku kostis teiselt poolt seina: 
\enquote{\emph{Clear all!}} Küllap ta käis kellelegi närvidele. 

\question{Mis ülesannet te lahendasite?}

Ülesanne oli vahva, umbes selline, et kass ärkas, sirutas, hüppas ja sattus klaviatuurile. Arvuti tegi 
piiks, kass tegi näu ja selle peale ärkas üles tema perenaine Celia, kes 
mõtles, et täna on kolmapäev -- mida ma olen tellinud, mis kaubad peaksid täna 
tulema ja mis mul veel tellida oleks vaja? Siis räägiti Peterist, kes istub 
kesklaos ja paneb kaupu liini peale, ning Larryst, kes sõidab 
\emph{lorry}'ga ringi ja veab kaupu laiali. Klassikaline 
veebikaubanduse logistika, mida tollal ajal veel ei olnud. Meile anti 
ette kaart ja GPS-signaal ning pidime programmeerime auto armatuurlaua 
koos GPSi liigutamisega ja tsentrumi. Meie lahendus erines teiste omast, sest ma ei 
viitsinud seda igavat ladu programmeerida ja tegin keskele 
logistikakeskuse, kus laod olid eraldi. Ladusid imiteerisime omaette failidest 
ja Peter võttis lihtsalt tellimusi vastu ja jagas laiali. 
Pärast messi lõpetamisel istusime žüriiga ühes lauas ja nad ütlesid, et kurat, mingid postsotsialistlikud vennad tulevad meile kapitalismi 
õpetama! 

Võistlus kestis 24 tundi: algas ühel päeval kell kolm ja lõppes teisel päeval 
kell kolm, seejärel hakkasid järjest esitlused tulema. Meie saime esitlema 
alles kell kümme õhtul, kui olime 24 tundi üleval olnud. Mina kirjutasin kogu koodi ja 
projekteerisin peas asjad. Veiko, hea suhtleja, süstematiseeris mu küsimused ja tõi 
žürii käest vastused. Tiiu joonistas vorme ja tegeles kogu kasutajaliidesega. 

Alustasin 15:20 ja kell viis öösel läksid näpud krampi -- umbes pool tundi 
ei liikunud, siis läks uuesti lahti. Korraks tekkis psühholoogiline tõrge, aga siis tegime edasi, kaks tundi enne tähtaega saime 
valmis. Hommikul kell kaheksa tekkis uuesti jama tunne, kui üks
Maci meeskond juba lõpetas. Mõtlesin, et olen ikka jube sant mees. Lõpuks 
selgus, et nad katkestasid.

\question{Kes sellist üritust korraldas?}

Täpselt ei mäleta, aga üks rootslaste softiliit. 

Ma teadsin sellest tänu ühele sõbrale, Tartu EPT juhile Kalle Kullmanile\index[ppl]{Kullman, Kalle}, kes oli seal kunagi ülesande tegijana osalenud. Temaga saime 
kokku marksistliku-leninliku kommunismi kandidaadimiinimumi täiendusloengus, kus me istusime kõrvuti. Loengut 
luges Otto Stein\index[ppl]{Stein, Otto}, keda kutsuti Otto 
von Steiniks. Ta oli saadetud Tartust Tallinnasse kommunistlike filosoofide 
kaadri tugevdamiseks, mispeale nii kaader kui seltsimees Stein tugevnesid. Hull vanamees oli.

Mäletan siiamaani, et loeng oli \enquote{Kommunismi on kolm allikat, kolm 
komponenti}. Need on inglise poliitökonoomia, saksa utopism ja \ldots\sidenote[][-.7cm]{Kommunismi kolm allikat 
toonase õppe järgi olid saksa klassikaline filosoofia (peamiselt Georg Wilhelm 
Friedrich Hegeli ja Ludwig Feuerbachi järgi), inglise poliitiline ökonoomia 
(Adam Smith, David Ricardo) ja prantsuse utopistlik sotsialism (Claude Henri de 
Saint-Simon ja Charles Fourier). Neid \enquote{arendasid edasi} 
marksismi-leninismi kolm komponenti: dialektiline ja ajalooline materialism, 
poliitiline ökonoomia ja teaduslik kommunism.}. Stein läks esimese tudengi juurde: 
\enquote{Öelge esimene, nii, õige.} Siis teise juurde: \enquote{Öelge teine.} 
Kolmanda juurde: \enquote{Öelge kolmas.} Ja nii edasi ühe inimese juurest teise juurde: 
\enquote{Teine, esimene, kolmas, teine.} Kui tiir minuni jõudis, tõusin püsti 
ja ütlesin: \enquote{Teate, mina selles tsirkuses ei osale!} ja jalutasin välja. 
Kui Stein püüdis asja leevendada ja ütles Kallele, et no öelge siis 
teie, teatas Kalle: \enquote{Mina ka mitte!} ja tõusis samuti püsti. Läksime 
välja, istusime Tuljaku baari maha, ajasime juttu ja oleme siiamaani suured 
sõbrad.

\question{Kuidas sa ikkagi panka sattusid?}

Olin nii-öelda vabakutseline häkker. 1991. aastal oli 
elutempo selline, et päevarütm oli täiesti sassis võrreldes teistega ja 
tööpäevad olid 72tunnised, pärast mida sõitsin autoga Valka ja Põlvasse asju 
üle andma. Mul oli siis juba kaks last, Anna oli just sündinud, ja selgus, et 
selline töörütm ei klapi enam. Seesama Tiiu\index[ppl]{Lumberg, Tiiu}, 
kellega käisime maailmameistrivõistlustel, rääkis, et 
Innovatsioonipank\index{Innovatsioonipank}\sidenote{18. 
septembril 1989. aastal ENSV Ministrite Nõukogu presiidiumi otsusega number 21 
Eesti NSV riigieelarve \enquote{üle plaani laekunud tulude} arvel asutatud 
pank.} otsib IT-juhti. Seda panka juhtis Peep 
Sillandi\index[ppl]{Sillandi, Peep}, pärastine mikro- ja makroökonoomika 
õppejõud EBSis, kes õpetas tudengeid softi peal mudeleid koostama. 
Peep oli lahe kuju, meil jutt klappis kohe ja lõpuks ta küsis: \enquote{Homme siis tuled 
või? Näe, tool on siin.} Mõtlesin, et olgu peale. 

Ja nii saigi minust IT-juht. Hommikul tuli minna poole üheksaks tööle, harjuda ära
sellega, et ei saa kella kolmeni öösel üleval olla. Ma ei ole sellega siiamaani 
harjunud, lähen üsna tihti praegugi öösel kell kaks magama ja ärkan kell 
seitse. Viis tundi on minu jaoks \emph{enough}. 

\question{Mida kujutas endast 1991. aastal väikepanga IT-juhi töö?}

Igasuguseid asju. Kui esimene päev uksest sisse tulin, istusin maha ja mõtlesin, kuidas mind on võimalik 
vangi panna (ma olen muuseas seda 
pärast igas ettevõttes teinud). 

Hakkasin seda kohta otsima ja leidsingi. Tollal käis keskpangaga 
infovahetus programmiga, mille kirjutas väidetavalt üks
armeenlane, kes oli kõik juuksed peast ajanud, et kammimise peale aega ei 
läheks, ja kuna suhkur on ajutoit, sõi ainult suhkrut -- geniaalne vend! Ta oli 
teinud nii käsuliidese kui ka dialoogiga käiva suhteliselt pisikese programmi, mis natuke 
krüptis ka -- küll väga vähe, aga tolle aja kohta ilmselt kõvasti -- ja saatis maksed 
panka ära. Iga päev tehti pangas kaks faili, üks hommikul ja teine õhtul. 
Õhtuses failis olid hommikuni käibed, mis tuli keskpanka saata, ja teises oli 
teistpidi. Igal pangal oli oma aeg, millal ta pidi failid ära saatma, ja samal 
ajal sai teistest pankadest tulnud asjad vastu. Kogu see asi seisis vabalt. 

Sain kohe aru, et panka saab röövida niimoodi, et ma ei võta kellegi kontolt 
raha ära, vaid tekitan sellesse kanalisse raha juurde. Siis ei hakka keegi 
selle järele igatsema, natuke aega tuleb ainult nostro- ja vostro-kontode sisu 
varjata, et ei oleks näha, et seal on mingi jama tekkinud, aga sellega 
saab hakkama. 

Esimene asjana tegime nii, et sellesse kohta sai faile panna 
ainult üks konkreetne programm, teine sai faile võtta ja kui keegi sellesse 
piirkonda sisse logis, katkestati kõik ära. 

Teine jama oli 
pangakontorite vaheliste ühendustega, näiteks kuidas saada 
Mustamäele kontor püsti nii, et see meie süsteemiga kokku saaks. Ehitasime ja 
testisime mingeid seadmeid. Telefonikanal ju ei 
püsinud.

Juhtus ka triviaalsemaid asju. Ühel päeval tuli teller minu juurde ja ütles: 
\enquote{Kuule, Priit, klient küsib oma käivet sellest ajast, aga seda ei ole.} 
Uurima hakates selgus, et süsteem oli üles ehitatud niimoodi, et kaks aastat vanad 
käibed hävitati ilma küsimata ja pikemat aega panna ei saanudki. 

\question{Mis too panga tuum oli, mis niimoodi tegi?}

See oli Midas Kapiti süsteem Kapiti, mis istus AS/400\index{AS/400} otsas. 
Aga kuna meil AS/400 ei olnud, siis meil käis OS/2 Warp\index{OS/2!OS/2 
Warp}\sidenote{OS/2 oli IBMi arendatud personaalarvutite 
operatsioonisüsteem. OS/2 Warp oli selle kolmas versioon, mis tuli turule 1994. 
aastal.}, mille peal istus AS/400 emulaator ja mille sees käis panga tuum. 

\question{Kas see oli kuskilt ostetud?}

Jaa, Kapiti käest, Midaseks muutus see hiljem. Ja mida 
Priit tegi? Istus maha, poisid hakkasid sortima \emph{backup}'e (neid me 
tegime hoolega) ja kirjutasime sellise softi, mis lappas \emph{backup}'idest SQLBase'i 
peale kokku andmelao. Me siis ei teadnud, et see asi on andmeladu. Kui ma panka läksin, lasin SQLBase'i osta, sest see oli hea 
kliendi-serveri lahendus, lihtsasti kättesaadav ja ei olnud väga kallis võrreldes 
teistega. 

Lisaks olin panga nõukogu sekretär. Peep arvas, et ma oskan 
piisavalt loetavalt kirjutada, pärast puhtaks lüüa ja saan asjast aru 
ka. Ütles veel, et ega sa liige ole, aga arvamuse saad ikka sekka 
öelda. Nii et olin nõukogus hääleõiguseta arvamusliider.

\question{See oli ju IT-juhile väga praktiline koht, said info kätte!}

Just nimelt, see oligi Peebu mõte ja jube hea mõte. Nõukogus 
olid väga vahvad liikmed, kellega ma tuttavaks sain. Näiteks Arvo 
Kallion\index[ppl]{Kallion, Arvo}, omaaegne
parteiboss ja valitsuses keegi, aga väga tark mees. 

Innovatsioonipank oli taskupank. Genin\index[ppl]{Genin, Alex}\sidenote{Alex 
Genin, Innovatsioonipanga nõukogu esimees.} oli niisugune juut Ameerikast, kes 
elas sellest, et tegi erinevates riikides panku, ajas nad riigi süül 
pankrotti ja siis hakkas kahjutasu nõudma. 
Sotsiaalpank\index{Sotsiaalpank} läks pankrotti, sealt pankrotipesast 
ostis ta kõige vingema kontori ning lasi panga põhja. Ma nägin ette, et see läheb nii. Selleks ajaks Peepu enam ei olnud 
sest Genin oli oma Miša (ma ei mäleta Mihhaili perekonnanime) panga etteotsa 
pannud. Tore poiss muidu, aga tegi täpselt, mis Genin ütles. Läksin Miša juurde, 
panin avalduse lauale ja ütlesin, et Miša, ma lähen nüüd ära. 
\enquote{Aga miks sa lähed?}  \enquote{See pank läheb varsti pankrotti.} 
\enquote{Sa eksid!}  \enquote{Ei eksi, Miša, ma olen majandusharidusega ja ma 
näen igal õhtul bilanssi, see pank läheb varsti pankrotti.} 

Mul oli on uus koht olemas, Tööstuspank\index{Tööstuspank}. Üks 
laenujuht läks sinna ja kutsus mind arendustiimi juhiks, kuna neil oli vaja uut 
infosüsteemi. Ma ei saanud aru, mis toimub: sain seal kõik, mida küsisin. Oma kontor ehitati Koplisse koos magamisruumi, köögi ja kõige muuga. Tirisin Innovatsioonipangast Eriku\index[ppl]{Matt, 
Erik}\sidenote{Erik Matt.} ja Raivo\index[ppl]{Tali, Raivo}\sidenote{Raivo 
Tali.} kaasa ja kuskilt tõin ära Ville Remmeri\index[ppl]{Remmer, 
Ville}.

Kuus kuud uurisime ja puurisime, ja kui meil oli kõik valmis, siis öeldi mulle: 
\enquote{Tead, me valetasime sulle. Me ei kutsunudki sind siia uut süsteemi 
tegema. Meil on siin IT-osakond, aga me ei usalda neid.} Seal IT-osakonnas 
oli igasugu tegelasi ja juht oli Poldzadze, kes nägi välja nagu Kirgiisi bai. Aga 
panga juhtkonnal oli vaja inimest, kes teaks, mis panga ITs toimub, sest nad 
hakkasid just Hoiupangaga kokku minema. 

Mulle öeldi: \enquote{Sa võtad nüüd IT juhtimise üle.} Sain endale kõige uhkema ametinimetuse, 
mis mul kunagi olnud on: panga esimehe 
volitatud eriesindaja IT küsimustes. Mul oli õigus käskida, puua ja lasta. 
IT-osakond oli sotsialistlik kamp, kellel oli vaja nimega ülemust. Mul oli õigus 
teha ükskõik mida, peaasi, et pank püsti püsiks ja midagi ära ei läheks. Ja 
ühel päeval läkski kuus ja pool miljonit krooni minema. Täpselt sedasama teed pidi, nagu 
ma arvasin. 

Sellel päeval, kui ma võimu võitsin, ei olnud mul veel võimalik midagi teha. 
Tõnu Liik\index[ppl]{Liik, Tõnu}, kes oli Hoiupanga IT-juht, tuli sinna, Ants 
Leitmäe\index[ppl]{Leitmäe, Ants} kaasas. Ants istus aknalauale ja kuulas 
tuima näoga pealt, tema pidi mind oma tiimiga tehniliselt toetama hakkama, sest ma 
ei teadnud kedagi usaldada. Siis kutsuti kõik kokku ja Tõnu teatas, et 
uus juht on nüüd siin. Ja kõik läks ilusti. 

Esimese asjana sai kõigil 
kasutajaõigused ära võetud, panin oma poisid masinate taha ja hakkasime uuesti 
õigusi õigetesse kohtadesse tagasi andma. Õhtul istusin pearaamatupidaja Irina juures (perekonnanime ei mäleta), kes toodi ka enne liitumist majja, ja ajasime 
juttu. Irina võttis lahti tagasi tulnud hommikuse kontrollsummade faili 
ja tegi istmelt meetrise hüppe üles. Kuus ja pool miljonit oli jagunenud pankade 
vahel teisiti, kui oli hommikul välja läinud -- raha oli läinud Rakvere Maapanka\index{Rakvere Maapank}. Egas midagi, 
joostes minema, kogu IT-osakond puhtaks ja arvutid tuli lahti jätta. Panime oma poisid 
peale ja hakkasime kontonumbri järgi \emph{search}'i tegema. Ja leidsimegi ühest 
masinast, kusjuures avastasime tänu sellele, et võtsin kõigil hommikul 
õigused ära. Tüübil oli kontrollsumma muutus ka tehtud, aga ta ei saanud 
õhtul enam vajalikule kohale ligi. Irina helistas kohe Rakverre, blokkis summa 
ära ja järgmisel päeval saime tagasi, sest õhtul olid pangad juba kella neljast kinni ja pangaautomaati 
ei olnud. Nad olid plaaninud tegutseda järgmisel hommikul. 

Vend võeti kinni ja viidi raudus minema. Masina panime raha transportimise kotti, pitseerisime 
kinni ja lukustasime seifi. Järgmisel päeval tegime manukate juuresolekul lahti, võtsime 
ketta välja ja tegime sellest kolm koopiat. Üks läks TTÜsse analüüsi, teine Hoiupanga 
tiimile ja kolmas minu tiimile. Seal masinas oli peale kontonumbri veel üks 
klimp, mis oli parooliga zipitud. Küsime venna käest: \enquote{Mis \emph{password} on?} -- \enquote{Ma ei 
tea, kui te mulle ütleksite, oleks mul endalgi huvitavaid asju seal sees.} 
Läksime kontorisse, et installida murdmisklaster. Siis läksime 
Raivoga\index[ppl]{Tali, Raivo} suitsu tegema ja tagasi tulles ütles 
Erik\index[ppl]{Matt, Erik}: \enquote{Poisid, vabandust, te oleksite ka 
kindlasti seda näha tahtnud. Ma lasin prooviks klastri käima, leidis, sõnastiku alguses, 
\emph{konjak} väikse tähega\ldots}. Raivo ütles selle peale: \enquote{Jube madal 
profiil, ma oleksin vähemalt Mercedes-Benz pannud.} 

Seal failis oligi kogu värk sees. Süsteem oli Foxis, andmebaas oli DBF, siis 
oli tehtud üks \emph{browse}, mida klikkisid ja mis jäeti meelde, ning lõpuks F10 
vajutades kanti kõik ühe konto peale kokku ja tehti fail valmis. Samuti
kontrollsumma fail, mis oleks õhtul lihtsalt õigesse kohta 
tõstetud. 

\question{Nii et sulle köögi ehitamine tasus kohe esimesel päeval ära?}

Ma ütlen sulle, et siin ei ole midagi oodata. Kõik juhtub kohe.

Hoiupank ostis Eesti Kindlustuse ära ja ma läksin sinna IT-juhiks. Seal oli ITs 
kaks ja pool meest, kellest ühte ma ei näinudki. See oli osakonna juhataja, kellel 
olid mingid probleemid, ja kui ta kuulis, et on uus IT-juht, siis ta ei 
tulnudki enam. Iseenesest geniaalne mees, mitmeid matemaatikaõpikuid 
kirjutanud. Võtsin oma poisid Ville 
Remmeri\index[ppl]{Remmer, Ville}, Erik Matti\index[ppl]{Matt, Erik} ja Raivo 
Tali\index[ppl]{Tali, Raivo} kaasa ja kolistasime sinna. 

Leidsime eest mingi õuduste maa. Katastroofi kuubis. Või okse kolme x-iga -- ma ei tea, kuidas seda kõige paremini
iseloomustada. Näiteks elukindlustus käis niimoodi, et paberi peal korjati dokumendid 
kokku ja need läksid kümne fakiirsisestaja kätte. Neil oli ekraani peal triip, 
kus olid postid vahel, ja triibu vahele sisestasid nad andmed, mille põhjal 
tehti reserviarvutusi ja kõike muud. Neil oli käigus postivõrk -- venelased kutsuvad seda
pastlavõrguks. Kõik raportid pandi posti, 
saadeti Tallinna, siin sisestati ära, trükiti raportid välja, pandi postikotti 
ja saadeti tagasi. Ma panin Ville kiiresti kirjutama lokaalset kasutajaliidest. 
Raivo ja Eerik hakkasid otsima võimalusi, kuidas teha ära side kõikide meie 
kontoritega. Ise kappasin esimestel nädalatel mööda kõiki esindusi, et uurida, mis 
probleemid seal on. Ühe tehnikutest võtsin kaasa, et ta vaataks tehnilist 
poolt. 

See kõik juhtus augustis. Mulle lubati kahe töökoha vahel viis päeva puhkust, 
nii kiire oli asjaga. Uue süsteemi tõmbasime käima jaanuaris. Selleks 
ajaks olid meil kõik ühendused tehtud, uus server olemas, 
soft töötas ja inimesed koolitatud. Iga agent hakkas ise oma asju sisestama.

\question{Mis aastal see oli?}

Umbes 1996. 

Nüüd tuli hakata ka ülejäänud süsteemi uuendama, aga võrk oli kohutav. 
Tellisime võrguehitustööd, ehitasime korraliku serveriruumi, panime seina
tulekindlad materjalid ja väljapoole jahutuse. Serveriruum asus maja keskel kõige paksemate 
müüride vahel, lasin sealt WC ja duširuumid koos torudega välja 
lõhkuda. Ja siis ühel päeval, vist märtsis, juhtus niisugune lugu, et kui 
hakkasime üle viima oma viimast serverit, siis
selgus, et \emph{backup} ei loe ja ketas läks nässu. \emph{Backup} oli meil 
korralikult tehtud, aga majas käis remont ja tolm oli selle ära rikkunud. 
Lindiseadmed olid tol ajal nii lollid, et ei näidanud seda. Ja kes see tol ajal ikka
\emph{backup}'e kontrollis -- kui kahes eksemplaris teha, siis oli ju piisav. Aga nüüd olid mõlemad tuksis. Seal peal olid raamatupidamisandmed ja me olime ju 
börsiettevõte. 

Egas midagi, Priit võttis ketta kaenlasse ja sõitis järgmisel päeval Inglismaale OnTracki, kes taastab kettaid. Väga kihvt firma, kõik maailma 
suured on nende kliendid, kaasa arvatud CIA ja KGB, aga vaevalt, et nad oma 
kettaid taastavad. Jõudsin kohale reedel ja tagasi tulin teisipäeva hommikul 
kahe kassetiga, kus olid kõik andmed peal. Kutid olid selle ajaga kõik 
valmis pannud ja nii kui ma lennukist maha astusin, võeti lindid, taastati 
ära ja asi läks käima.

Siis hakkas Hansapank Hoiupanka ära sööma. Tegelikult alguses oli ühinemine, 
pärast ülevõtmine. Mul ei ole selle kohta paberit, aga seest 
vaadates käis minu arvates asi niimoodi, et kõigepealt kuulutati välja ühinemine ja kui kõik 
oli teada, siis võeti lihtsalt üle. 

Tõnu Liik\index[ppl]{Liik, Tõnu} viis suurema osa ITst 
Hanschmidti\sidenote{Toonane legendaarne Ühispanga juhatuse esimees Ain Hanschmidt.} juurde ehk tollasesse Ühispanka\index{Ühispank}. Me tegime uut vinget 
infosüsteemi ka, aga see ei saanud valmis, sest kindlustus lõpetati ära. 

Elasin Mähel suvilas ja Tõnu helistas mulle ühel laupäeva hommikul. Tundsin Tõnu 
juba Tööstuspanga aegadest, kui tegime koos esimesi kaardiprojekte. 
Sulo Muldia\index[ppl]{Muldia, Sulo} Raepangast\index{Raepank} ja kes meil seal kõik olid, omaaegsed 
karismaatilised kujud erinevatest pankadest. Niisiis, Tõnu helistas mulle: \enquote{Kus sa oled? Ma tulen kohe sinu juurde.} Jõin parajasti aias kohvi ja ei jõudnud hommikumantlitki ära võtta, kui
Tõnu astus juba uksest: \enquote{Nüüd on sihuke värk, et 
ütle jah ja ma lähen kohe ära. Ega ma enne ei lähe ka. Tule, ma annan 
sulle uue kindlustuse, tee see valmis, mis tegemata jäi.} Pakkusin talle kohvi, 
jõime selle ära ja oligi kokku lepitud. Läksin SEBsse\index{SEB} kindlustuse 
arendusjuhiks. Tegime hea mudeliga elukindlustuse, mis oli 
selles mõttes märgiline süsteem, et see on \emph{proof of concept}. 
Mul on nimelt oma andmete modelleerimise teooria ja too süsteem on mudel, mis 
näitab, et see teooria töötab, sest need süsteemi osad, mis me siis tegime, on 
aastast 1999 samad. 

\question{See on kõige parem kvaliteedinäitaja, et asi peab kõigile muutustele 
vastu!}

Tegelikult on üks veel vahvam näide, aga see on varasemast ja ma ei teinud seda 
teadlikult. Aastal 1986 kirjutasin ma ühe palgasüsteemi ja müüsin seda ka 
mõnele, aga siis lõpetasin ära, sest see oli FoxPro ja musta ekraaniga. Ja aastal 1994 
helistati mulle ja öeldi, et \enquote{kuulge, see teie palgasüsteem \ldots} 
\enquote{Mul ei ole ühtegi palgasüsteemi.} -- \enquote{Mäletate, te müüsite selle meile aastal 1986, aga meil nüüd firma nimi muutus ja me ei oska seda ära vahetada. Proovisime
teisi süsteemi ka, aga seal on vead sees, oleme kõik muu suutnud selle järgi 
häälestada.} Sotsialismist tuli süsteem kapitalismi ja elas selle asja üle! 

\question{Mida sa praegu teed?}

SEBs\index{SEB} olin ma 17 aastat. Käisin küll 
vahepeal ära, kui mul oli üks kümnekuune huvitav periood. Tänapäeval nimetatakse 
neid \mbox{startup}'ideks, aga tol ajal me lihtsalt arvasime, et oleks tarvis teha üks 
produkt, mis õnnetuseks sattus IT-mulli lõhkemisega samale ajale ja me ei 
saanud enam riskirahasid peale. Me lõime sellist süsteemi, mis teeb kirjelduste 
pealt suvalisi dialooge, ühesõnaga salvestab andmed andmebaasi. 
Kirjeldused on olemas ja samast kirjeldusest võib teha \emph{voice}'i või 
mobiilirakenduse või ükskõik mida. Oma aja kohta 
oli see kõvasti ajast ees.

Sealt ma tulin tagasi SEBsse ja siis läksin RIAsse.\index{Riigi Infosüsteemi 
Amet}\sidenote{Riigi Infosüsteemi Amet.} RIAs juhtus niisugune lugu, et Katrin 
Reinhold\index[ppl]{Reinhold, Katrin}\sidenote{Priidu kolleeg Riigi 
Infosüsteemi Ametis.} tuli TEHIKusse\sidenote{Tervise ja Heaolu Infosüsteemide Keskus, sotsiaalministeeriumi 
IT-maja.} direktoriks. Katrin hakkas mind aeg-ajalt kutsuma arvamust avaldama 
ja kord, kui me siia alla kööki läksime, küsisin ta käest, et 
Katrin, sa vist lubasid Taimarile\index[ppl]{Peterkop, 
Taimar}\sidenote{Riigi Infosüsteemi Ameti peadirektor ajal, kui Katrin ja Priit 
seal töötasid.}, et sa ei võta kedagi kaasa. \enquote{Jah, ma lubasin.} -- 
\enquote{Aga kui ma ise küsin?} Ja ta vastas nagu Teele: \enquote{Ma 
mõtlesin, et sa ei küsigi!} See oli elu kõige lühem tööle värbamise vestlus. 

Tööle asudes oli meil analüüsi osakond, aga nüüd on selle 
nimi andmekorralduse ja andmeanalüüsi osakond. Selle all on kaks talitust. Üks 
on andmekorralduse talitus, kes teeb HL7\sidenote{Meditsiinis laialt 
kasutatav andmestandard.} standardi peale andmevorminguid, millega kogu 
tervishoiu infovahetus Eestis käib. Meil on väga hea ja detailse mõtlemisega arhitekt Andrus 
Tamboom\index[ppl]{Tamboom, Andrus}, kellele mina 
käin algul mõne asja välja ja tema mõtleb edasi ning joonistab lõpuni. Kahe analüütikuga ehitan üles 
analüüsikeskkonda ja palkan ka parasjagu andmelao talituse juhti, et kogu 
andmelaondus korralikult üles ehitada. 

\question{Järeldan kõige selle põhjal, et sul 
läheb hästi.}

Ei mul pole häda, väga huvitav on! Tegelen 
selliste asjadega, millega ma varem pole otseselt kokku puutunud, aga see valdkond muutub ka 
sellise kiirusega, et teinekord on raske kannul püsida. Kogu aeg 
lappan kohutavat kogust materjale läbi, kas on midagi uut. 

\question{Sul on vist kogu aeg nii olnud?}

Just nimelt, ma ei ole kunagi töötanud sellise ameti peal, kus mulle ei ole 
meeldinud. See on kõige olulisem. 

\chapter{Tõnis Reimo}
\index[ppl]{Reimo, Tõnis}

                 
\ldots ma arvan, et see minu alguse lugu, nagu ma arvan paljudel, oli ammu enne BBSindust. Ta on ikkagi otseselt  seotud sellega, et isa töötas tollal arvutite teemal, vist oli Rahvusraamatukogu\index{Rahvusraamatukogu} (tollal  Kreutzwaldi nimelise raamatukogu) mingi arvutus- või arenduskeskusega, mingi sellise imeasjaga, seotud. Sealt pääsesin arvutite ligi. Tollal tekkisid ka esimesed sellised, noh, arenenumad kirjamasinad. 

\question{Mis vanuses see umbes oli?}

Ma arvan, et see võis olla mingi viies või kuues klass. Tollal mindi aasta hiljem kooli, nii et siis tänapäevase mõistes mingi kuues-seitsmes klass. 

Sealt tekkis selline võimalus arvutitele üldse ligi pääseda ja loomulikult mängima saada, sest see oli tollal nagu ainuke asi, mis huvitas.

\question{Päriselt või?}

Ütleks niimoodi, et progemisest oli asi veel kaugel.

Mänge ju ei olnud. Kas olid malmist põranda külge needitud mänguasjad või siis need esimesed arvutimängud, mis  umbes nagu jooksid pigem trükimasinal kui arvutil. 

\question{Mis arvutid need olid?}

Eks sain natukene näpitud mingeid jessukesi\index{Arvutid!ES EVM} ja sellist Vene toodangut, aga tekkisid mingid esmased personaalarvutid nagu Schneider\sidenote{Schneider oli kunagise arvutitootja Amstradi esindaja Saksamaal, Austrias ja \v{S}veitsis, kelle müügivõrku viimane kasutas ning kes Amstradi arvuteid ka mõnel määral oma turule kohaldas. 1988. aastast alates läksid ettevõtete teed lahku ja Schneider hakkas tootma oma PC arvuteid.}, Lääne-Saksamaal importtoodang.

\question{See oli siis mingi XT analoog?}

Isegi XT eelne. Aga tema peal oli juba võimalik  mingeid primitiivseid mänge mängida ja sealt vaikselt  see huvi arenes. 

\question{Mida nende arvutitega päris tööks tehti? Sina käisid mängimas, aga\ldots}

Milleks täiskasvanud neid rakendasid sellet,  ma alguses nagu aru ei saanud ja väga ei huvitanud ka. Minu jaoks olid täiskasvanud lihtsalt tüliks, sest nad takistasid arvutisse saamist. See arusaamine, mida täiskasvanud arvutiga teevad,  tuli alles  aastaid hiljem siis, kui tulid ka esimesed katsetused progeda.

Sealt kasvas ka selline mingi laiem huvi välja, et sai liitutud  Jaak Loonde\index[ppl]{Loonde, Jaak} veetud arvutiklubiga Ahhaa\index{Arvutiklubi!Ahhaa}. Ahhaa, loodi minu arust kas 1985 või 1986, Jaak Loonde oli siis 3. Keskkooli\index{Koolid!Tallinna 3. Keskkool} legendaarne matemaatikaõpetaja. 

Ta vedas seda arvutiklubi alguses mingi nõukaaegse toodangu peal, see oli mingi ES-i laadne masin, millel meie mõistes ekraani ei olnud, aga oli teletaip. Tekst trükiti paberile ja arvuti oma tulemuse trükkis ka paberile. Oõhimõtteliselt oli trükimasin, kuhu lõid käsud, käsud trükiti sinnasamasse paberile, sinna sinu käskude alla trükkis masin oma vastused. Loomulikult olis, eks ole, mingid lindi pealt sisse lugemise võimalused.

Sealt läksime juba kiiresti üle esimestele Yamaha MSX-idele\index{Arvutid!Yamaha MSX} mis tulid. Sealt edasi tulid juba Jukud ja Iskrad ja kogu see kloonindus.

\question{Kuidas sa sinna arvutiklubisse sattusid? Kas huvitas või mõni sõber kutsus?}

Ma isegi ei mäleta, kuidas. Võib-olla see asi, et ma esimesed kolm aastat oma elust käisin ise 3. Keskkoolis, seetõttu oli nagu mingi seos olemas. Võib-olla kooli kaudu teadsin, kuidas ma sinna sattusin, ma ausalt öeldes enam ei mäleta. 
                 
\question{Konteksti mõttes, aruvti-inimesed on tavaliselt \emph{sci-fi} sõbrad ja muud sellist, kas sul sihukest asja ka oli?}

Ikka, Soome televisioonist sai Battlestar Galactica-t vaadata.

\question{Originaalseeria, eks, kandiliste plekist robotitega!}

Jaa, Cylonid, tuli käis edasi-tagasi. Ja loomulikult kogu nõukaaegne \emph{sci-fi} kirjandus oli läbi töötatud, niipalju kui kätte ulatus.
      
\question{Mingit konkreetset eredamat asja oskad meenutada?}           

No ikka alustatud sai Maailm ja Mõnda\sidenote{Maailm ja Mõnda oli Eestis ilmunud raamatusari, mida algselt andis välja Eesti Riiklik Kirjastus, hiljem Eesti Raamat ja teised. Sari keskendus  peamiselt reisikirjadele ja loodusraamatutele.} sarjast, keerukam ja elegantsem osa kõik tuli kõik hiljem. 

Praegu ma panen puusalt, igasugused \enquote{Purpurpunaste pilvede maad}\sidenote{Seda raamatut meenutab ka Tarmo Mamers leheküljel \pageref{sisu:purpur}.} ja kõik sellised asjad. See oli nõukogude vaimustuses kantud \emph{sci-fi}, jutustas kuidas kommunistliku ühiskonna liikmed kangelaslikult kosmost vallutavad.

\question{Ega neid raamatuid ei olnud Eesti keeles palju saada, seega mingi seltskond luges väga paljus samu raamatuid ja saadi üksteisest paremini aru.}         

Jah, see oli üks asi, aga seltskonnast, kellega mina tollal kokku puutusin oli ikkagi suhteliselt piiratud. Omavanustest olid need ikkagi selliste inimeste lapsed, kes töötasid arvutitega. See tähendas seda, et nende vanemad olid kas KBFI-s, Tallinna Tehnikaülikoolis või mingites asutustes, kus oli arvuteid. Tehniline intelligents. Päris juhuslikku rahvast väga palju seltskonnas ei olnud, enamasti inimesed, kellel olid varem arvutitega kokku puutunud. Mitte niivõrd \emph{sci-fi}-st lahti pääsenud kuivõrd lihtsalt arvutitest.

Nii palju, kui mina nägin, siis enamus, nii 80\%, oli keskendunud arvutitega mängimisele. Sealt edasi tulid sellised praktilised probleemid, et kuidas mänge kopeerida ja kuidas mänge avada ja kuidas need üles otsida. Sealt tegid opsüsteemiga tutvust ja lõpuks siis tekkisid nagu esimesed huvid, et \enquote{aga kuidas neid ise teha?}
                 
\question{Kaua sa seal Ahhaa klubis käisid?}

Ahhaa klubil olid rohkem nagu üritused. Nii palju kui mina mäletan, et mingitel kindlatel päevadel oli sul kusagil ligipääs arvutitele. See \enquote{kusagil} oli mingi Tööjõureservide Õppekeskus, Tehnikaülikool, mingid sellised veidrad kohad.
Ja eks aja jooksul tuli neid kohti juurde, kellelgi jälle vanemad või sõbrad jälle sokutasid ja nii me rändtirtsudena lendasime peale vabale arvutusressursile.

\question{Kas need täiskasvanud inimesed, kes loodetavasti tões ja väes tööd üritasid teha, ei pahandanud?}
                 
Enamasti oli see tegevus ikkagi  töökeskkonnast nagu eraldatud. Ma väljaspool isa töökeskkonda ei mäleta väga palju, et me oleksime otseselt tööruumides olnud. Mingit õppeklassid ja mingid sellised kohad. Loomulikult, hiljem vaadates, siis kindlasti sai kõvasti närvidele käidud isa kolleegidele. Selle moel, et kas nende arvutiterminale hõivatud siis kui nad tahtsid tööd teha. Aga see oli nagu selline põnev mäng, et nad peitsid mängud ära ja siis meil oli jälle põhjust vaeva näha nende üles otsimiseks.

\question{Millest ma järeldan, et täiskasvanutel olid ka mängud kuskil seal masinates!}

Olid olid, ega ma ise neid sinna ei pannud. No mõningatesse panime. Aga  kui tollase Tallinna Linna Täitevkomitee (mis oli tollal üks isa töökohtadest) keldrisse tekkis UNIXi masin, siis sinna ikka ise midagi  ei kopeerinud. Masin oli  varustatud sellega, mis seal oli ja siis tuli seal kähku orienteeruda, \verb|su| käsud selgeks saada ja ruudud.

\question{Huvitav on see, et kui nad Eestisse jõudsid, olid UNIXi purgid pigem suletud ja kõik muud pigem sellised, kuhu sai ise asju sokutada.}

No ega sa Jessukesse ka kindlasti said sokutada, aga lihtsalt  programmeerimine ja programmi sokutamine olid väga tülikad, sest see käis perfokaartide kaudu.
                 
\question{Sa rääkisid, et su ja root said selgeks, kuidas nad selgeks said, kust see info tuli?}

Ma praegu loomulikult ei mäleta, tõenäoliselt ta pidi tekkima seeläbi, et vingusid oma mängu senikaua, kuni keegi sulle midagi ette näitas. Vaata, kui vanalt tänapuaeval nutitelefon sellega saadakse, enam-vähem samal ajal koos rääkimisega, et asi siis nüüd selle \emph{super useri} kasutamine selgeks saada on. Oluliselt vanemana, kusjuures. Seda enam, et tollal  infoturbe teema oli suhteliselt olematu,  kõikide login oli eesnimi ja kõikide parool oli tema perekonnanimi.

\question{See võis konduktiivne küll olla ringi pusimisele. Ehk, kui tol hetkel oleks infoturve olnud paremini paigas, sisi terve põlvkond inimesi oleks sisuliselt ilma arvutita jäänud?}

Ma arvan, et mitte, sest tegelikult saadi masina ligi läbi vanemate ja eks sa ikka naaksud oma isa ja ema kallal senikaua, kuni ta selle mängu käima paneb. Eks siis vahepeal vaatad, jälgid, paned tähele, mida tehti, kuidas sai. Aga ma arvan, see tase oli erinev, sest mõned tundsid programmeerimise vastu nagu võib-olla põhjalikumalt huvi, mina nagu alguses vähem. Ausalt öeldes programmeerimine ei ole läbi elu mu tugevam külg olnud. Ma olen sellise müügi, turunduse, juhtimise, projektijuhtimise, tootejuhtimise kallakuga olnud läbi elu.

\question{Teades, mis tooteid sa oled juhtinud, seda ju ei saa teha saamata väga hästi aru, mis kapoti all toimub?}

Jah. See huvi on ka loomulikult alati olnud, et kuidas asi töötab. Aga selleks ei pea alati ise tegema.
                 
\question{Kui sa said tonks vanemaks, siis mingil hetkel see külakorda käimine pidi ju muutuma?}

Ühiskond läks edasi, eks ole. Alguses tegutsesime  isa loodud Eesti-Rootsi ühisfirma tiiva all, kus olid personaalarvutid. 286, 386, hiljem 486. Ja sealt edasi siis lõime ka oma firma. Aga alguses sai selle ühisema ruumides ja tiiva all tegutsetud ja tegeletud arvutite maale toomisega. Sama eesmärk, et enamus aega läks mängimisele, aga oli ju vaja kusagilt saada seda riistvara, millega mängida. See oligi üks \emph{driver}. Ostad, mängid mõnda aega, müüd maha ja nii see äri käima läks. Hiljem võttis äri mängimise üle, sest et siis ei olnud enam aega mängimisega tegeleda, kogu aeg läks äri peale. 

\question{Mis võib olla nii positiivne, kui negatiivne. See oli pärast keskkooli?}

Jah, see on kusagil pärast tehnikumi. Ma lõpetasin Tallinna Polütehnikumi\index{Tallinna Polütehnikum} aastal 1990, 1991. aastal vist tegime siis oma HNS-i\index{HNS} nimelise firma mis omakorda tulenes juba enne seda loodud BBS-ist, mis kandis \emph{Hackers Night System}-i\index{BBS!Hackers Night System} nime. Mis omakorda ei tekkinud tühja koha peale, vaid tegelikult oli enne seda olemas Lembit Pirni\index[ppl]{Pirn, Lembit} Eesti BBS \#1\index{BBS!Eesti BBS \#1}. Lembit Pirn tegutses täna Tornimäel asuvas sellises madalas valges hoones. See oli tollal mingi transpordi informaatika keskus või midagi sellist. Ja temal oli esimene modemiga töötav BBS püsti pandud, seda sai kõvasti  külastatud. 

\question{Miks ta selle tegi ja miks seal oli vaja käia?}

Miks Lembit seda tegi peab tema käest küsima. Seal oli väljas ühelt poolt mänge, teiselt poolt seal oli mingi suhtluskeskkond, hakkas tekkima juba nagu selline \emph{bulletin board}. Ta läks käima tarkvaravahetuse pealt, mis tollal oli täiesti tavaline, tänapäevases mõistes räigelt illegaalne tegevus. Aga noh, nõukogude ajal ja üleminekuajal see mõiste oli võõras. 

Kuna ta vist oli ka Fidoneti liige, siis sealt sai infot ka teiste BBS-ide kohta maailmas. Sai hakatud Soomes külastama, mis oli Jouni Salo BBS\index{BBS!Jouni Salo} ja Ron Dwightiga\index[ppl]{Dwight, Ron}, kes  oli siis Fidoneti Euroopa \emph{tsooni} pidaja. Eks nemad aitasid-juhendasid edasi ja sealt tekkis loomulikult mõte oma BBS püsti panna. Tänu sellele, et me saime isa firma ruumides tegutseda, oli meil unikaalne võimalus teha otse välismaale kaugekõnesid. See, mis on tänapäeval suhteliselt elementaarne, et sa kusagile otse helistad, ei olnud tollal isegi Eesti piires elementaarne. Kõik kõned tehti läbi keskjaama. Keskjaama oli  sellele kindel number, kuhu sa helistasid, kus võttis keegi naisterahvas vastu,  kellele  teatasid, kuhu sa tahad helistada. Lugesid oma numbri ette. Kui liin vabanes, siis ta  helistas sulle tagasi ja teatas, et nüüd on siis kõne. Aga kuna tegemist oli Eesti-Rootsi ühisfirmaga, siis oli seal unikaalse võimalusena võimalik automaatvalimisega helistada otse maailmas igale poole. Ja see võimaldas ka modemiga enam-vähem üle kogu maailma helistada.

\question{Kust need modemid tulid?}

Esimene modem oli  mingi arvutiga kaasas, mingi 2400 bitti sekundis läbi laskev modem. Hiljem meil õnnestus Ron Dwighti\index[ppl]{Dwight, Ron} ja nende kaudu saada esimene US Robotics\index{US Robotics} mis vist oli kas 9600 või midagi sellist, ehk oluline edasihüpe. Kuna tollal nendes BBS-ides liikus palju erinevat tarkvara (ma nüüd ei ütleks, et  legaalset), siis selles ringi surfamine sobramine andis ühelt poolt vahendid ja teisalt  teadmise ja oskuse kuidas,  erinevad tarkvarad töötasid, mida nendega teha sai ja nii edasi. Sisuliselt kõik on nagu ise õpitud  puhtalt  katsetades eksitusmeetodil. Isegi ma mäletan nõukaaja lõpus, kui veel lennata sai Nõukogude Liidu piires täiesti vabalt, siis Vladivostokist, Moskvast ja Leningradist, oli teisi Fidaneti kasutajaid, kes lendasid külla viietolliste flopidega.
                 
\question{Inimesed tulid Vladivostokist viie tolliste floppidega Fidoneti?!}

Jah,  kuna modemiga imeda võttis  rohkem aega ja raha, kui lihtsalt võtta nagu sadu viietolliseid flopisid kohvriga kaasa ja  lennata Vladivostokist Tallinnasse ja lihtsalt kopeerida neid paar ööd-päeva läbi.

\question{Ahaa, nii et Vene suunal oli ka Fido side olemas?}

Jaa, nad samamoodi helistasid peale Eesti saitidele ja sealt liikus info nende suunas. 

\question{Sa korra mainisid, et Eesti BBS \#1\index{BBS!Eesti BBS \#1} ümber toimus mingi suhtlemine ja tekkis kogukond. Kes need inimesed olid?}
                 
Vot ma seda väga palju ei mäleta. Ma mäletan, et sealt me liikusime kiiresti edasi. Ta vist oli suhteliselt staatiline, vaikne ja rahulik pärast seda loomist, et seal vist väga sellist kommuuni ei tekkinud või siis vähemalt ma ei mäleta sellest.  Kommuun tekkis siis, kui BBS-i pidajaid tuli juurde ja mingil hetkel sai kokku kutsutud siis esimene süsoppide nõupidamine, mis toimus Viru hotelli kas  20. või 22. korrusel asunud väikeses äärepealses toas. Seal olid Lõvi\index[ppl]{Lõvi}, mina Tarmo Ausing\index[ppl]{Ausing, Tarmo}, Tarmo Mamers\index[ppl]{Mamers, Tarmo}, Virko Püss\index[ppl]{Püss, Virko} vist. Tegelikult on Mamersil vist isegi mingid memuaarid kirjas, nad olid mõnda aega veel isegi internetis üleval. 

                 
\question{Kas too kokkusaamine oli rohkem  sotsiaalne üritus või oli seal mingisugust probleemi ka lahendada?}

Ta oli mõlemat. Natukene oli minu arust juttu  sellest, et kuidas Fidonetti  korraldada, organiseerida, kuidas  meililiiklust vist teha. Kuna meil olid tollal väliskõned nii-öelda tasuta käes, siis meie saime olla Eesti  see esimene \emph{node}, kelle kaudu meil  liikus Eestist välja. 

\question{Ja see meie on Hackers Night System?\index{BBS!Hackers Night System}}

Jah. Teised saatsid oma kirjad meile, meie saatsime öösel siis need kirjad üle modemi järgmisele Euroopa \emph{node}-le, kes siis need omakorda laiali jagas.
                 
\question{BBS-i püstipanekuks on ju mingit tarkvara ka vaja?}

See oli nagu suhteliselt sellise paketina saadaval, samamoodi BBS-ist tõmbasid alla. Mingi Maximus BBS või midagi sellist, ja see oli selline vahva platvorm. Tõmbasid \emph{boot}-imisel BAT failiga üles, see jäi modemist ootama sisse tulevat kõnet ja kogu lugu. Seal all olid siis põhimõtteliselt faili kataloogid ja meilisuhtlus ja kasutajate haldus.

\question{BBS-i sisse helistamiseks ei olnud mingit eraldi softi vaja?}

Tavaline terminali soft oli, see  oli vist tollal isegi, ma pakun, enamusele opsüsteemidele olemas. Ma praegu muidugi oletan, aga kuna tollal oli ka suur osa ju \emph{mainframe}-del, nendega suhtlemine käis üle telneti.

\question{Ma miskipärast mäletan Norton Commanderi sees mingit helistamisvõimalust}

Võib olla. Seda mina ei taibanud kasutada. Mis kellelgi käepärast oli. Sellega sai siis juhtida nii kasutatava modemi režiimi, terminali režiime, sealt sai alla tirida tarkvara, see oli meie tegevuste põhiskop lisaks siis mängimisele ja selle tarkvara uurimisele, mis me saime.
                 
\question{Ehk, kogukond tuli põhimõttelielt BBS-i adminide hulgast. Aga tuumiku ümber pidi ju olema ju ka kasutajaid. Oskad sa suurusjärgus hinnata, palju neid võis Eesti peale olla?}

Ma arvan, et alguses, üheksakümnenda aasta paiku, võis olla mingi paarkümmend kasutajat per \emph{node} ja neid \emph{node}-sid oli ka nii viie kuni kümne ringis. Hiljem, interneti tuleku eelsel ajal, läks asi juba päris suureks ja massiliseks aga selleks ajaks olime meie juba sellest kogukonnast eemaldunud. Peamiselt seetõttu, et äri võttis kogu aja üle. 

\question{Arvutame siis. 20 inimeste per \emph{node}, 5-10 \emph{node} teeb\ldots} 

Vast paarsada inimest üle Eesti, neid võiks ka nagu rohkem olla. Tollal oli ka see asi, et igalühel ei olnud oma arvutit ja oma modemit vaid see oli selline \emph{round-robin}, et 10 inimest sama arvuti ja modemiga. 

\question{Nojah, klassitäis kaake helistab sisse, mine võta kinni, palju neid seal on}                 

Ei, vaata, neid \enquote{klassitäis kaake} nagu väga palju ei olnud, sest et ikkagi enamasti olid meievanused tegelased, aga olid ka sellised natuke vanemad tegelased, kes sellega tegelesid. Päris sellist akadeemilist seltskonda, teadlasi ja uurijaid, ma ei mäleta sealt. Neil olid ilmselt omad vahendid ja võimalused. Ka Usenet oli kusagil olemas.

\question{Kas Usenetil olid mingisugused konkreetsed grupid ka, kus palju eestlasi käis?}

Ma Useneti Interneti-eelsest kasutusest palju ei tea. Ma arvan, et seal oli peamiselt akadeemilised kasutajad.

\question{Panite oma HNS-i püsti ja hakkasite Nõukogude Liidu avarustest juppe ja neid Eestis maha müüma?}

Pigem vastupidi. Me hakkasime tooma Saksamaalt arvutijuppe, neid siin kokku panema ja siis Eestis maha müüma. Võib-olla tänu sellele, et, isa kõrvalt tekkis varakult selline impordi-ekspordi kogemus, mis tollal oli üllatavalt haruldane kompetents. Et kuidas väljamaalt midagi osta ja üle tollipiiride Eestisse tuua. See oli (20 aastat enne e-poode) siiski nagu suhteliselt haruldane teadmine. Esiteks, kuidas leida üldse see kontakt, kellele helistada, Kuidas talt hinnapakkumist saada. Sul ju ei ole aimugi, kes müüb. Sa ei tea, kellele helistada, et talt üldse hinnapakkumist küsida. 

\question{Aga kust sina teada said?}

Tänu sellele, et isa oli selline aktiivne tegelane. Kuna tema  inglise keelt väga ei osanud, siis ma pidin jooksvalt ka tema asju ajama ja siis tema eksimustest ja õnnestumistest, eks siis sai õpitud. 

\question{Eks me kõik seisame hiiglaste õlgadel. BBS-induses alguses mingit ärilist aspekti ju ei olnud?}

Ei. Puhas selline fänlus. Paljuski seisis ta kahel jalal. Esiteks suhtlemine, ehk  vestlustoad, inimesed omavahel suhtlesid erinevatel teemadel, selline community värk. Ja teine oli siis softi vahetamine.

\question{Tollel ajal hakkasid ju esimesed jutukad ka tekkima?}                 

Jah, ma arvan, et  jutukad saidki paljuski alguses nendest samadest BBS-ide tubadest.

\question{Mis need esimesed olid? Ma isegi ei mäleta nimesid, ei ole kunagi neis käinud?}

Ma jutukates ei ole väga palju olnud. Kas OK jutukas\index{Jutukad!OK} ei olnud üks ja Cafe\index{Jutukad!Cafe} vist oli teine. Need hakkasid tekkima siis, kui ma nagu enam väga selle sotsiaalse tegevusega ei tegelenud, kogu aeg ja energia läks oma firma arendamisele.
                 
\question{Kuidas see äri tol ajal toimus, sest üheksakümnendate keskel oli suhteliselt palju kogukondlikku toimetamist, kasvõi .EXE tegemine?}

Jah. Ma arvan, et too äri seisis sellesama kogukonna õlgadel, mis Fidonetist  alguse sai. Eks lõpuks pidid ju kõik kusagil tööd tegema. Ja pigem lihtsalt needsamad suhted ja asjad liikusid ärisse edasi. Paljuski needsamad nimed, kes ka seda ajakirja .EXE\index{.EXE} tegid olid ka Fidonetist tuttavad. 

\question{Sina keskendusid siis täitsa ärile?}

90.-te algusest või keskelt nagu läks enam-vähem selline üheteisttunnised tööpäevad ja. 

\question{Sealt kõrvalt vist väga palju enam programmeerimiseks aega ei jäänud?}

Ei. Me tegelesime eelkõige riistvara vahendamisega. Riistvara, võrgud, selline suhteliselt primi tegevus. HNS\index{HNS},  hilisem Zebra Infosüsteemid\index{Zebra Infosüsteemid}, nagu tarkvaraarenduseni ei jõudnudki.

\question{Isegi tavalise võrgu ehitamine oli ju koaksiaalkaablil?}

Koaksiaalkaabli vedamine oli mul väga selge. Mitmete tänaste suurfirmade laudade alt sai roomatudja tolmu pühitud ladudest ja seintelt ja lagedelt.

\question{Jaa. Mäletan Ühispanga kontorit, kus kontoriarvutid olid seljaga kliendi poole ja kaabel jooksis masina juurest masina juurde. Järjekorras seistes oleks võinud lihtsasti termika maha keerata ja terve kontor oleks seisma jäänud.}

Siis, kui me saime kunagi Seesamile\index{Seesam Kindlustus} IBM Token Ring nimelist võrku vedada, mis oli siis
 esimene selline tähtkujuline topoloogia, millega me kokku puutusime, millel olid sellised rusikasuurused stepslid, siis see oli täielik müstika.
 
\question{Kust teil tuli mõte sihukese müstikaga tegelda, see hakkab ju tasapisi üle piiri kasvama, mille endale lihtsalt katsetades selgeks teeb?} 

Selles mõttes, et ega nagu äris ikka. Sa võtad mingeid riske ja ütled, et \enquote{loomulikult me saame selle asjaga hakkama}. Ja mis see siis nüüd ära ei ole. Hakkad tegema ja  selgub, et  kõik töötabki niimoodi, nagu ette nähtud.
Ja ega  nende võrkude ja asjade üles panek ei olnud  progemisega võrreldes nagu mingi eriline tegevused. Lihtsalt konfigureerid süsteemi ära ja jalutad koju.

\question{Tol ajal koolist niisugust teadmist vist üldse saada ei olnud võimalik}

Absoluutselt. Mina lõpetasin polütehnikumis\index{Koolid!Tallinna Polütehnikum} raadioside ja levi eriala. See valdavalt põhines lamp-elektroonikal, kuigi meile õpetati ka pooljuhte ja nii edasi, aga loogilisi skeeme, loogilisi ahelaid, kõike seda me saime vist pool aastat. Ja sedagi niimoodi teoreetiliselt. Enamus  meie  elektroonika õppimisest polütehnikumi ajal oli väga selline analoog elektroonika: lambid, elektromehaanika, samm-valijatel põhinevad telefonijaamad. Noh, mis oli selline paras küberpunk tänapäeva mõistes juba. 

\question{Kui ma kuulan su juttu, siis sealt kumab välja soov asjadest aru saada, mõista, mis on karbi sees?}

Absoluutselt. Et ma läksin polütehnikumi, oli pigem nagu selline perekonna traditsiooni järgimine. Ma olen soodumuselt pigem nagu humanitaar olnud, kuna sinnamaani, ehk tehnikumi minekuni, olid mul neljad-viied kõik humanitaarained: keeled, kirjandus. Kõik reaalained olid kahed-kolmed.

Ilmselt siis tasub õppida seda, mida sa ei tea, ülejäänud tuleb niigi, lihtsalt. 

Aga noh, see on andnud  selle tugeva külje või plussi või aluse, et sa oskad asju, millest sa aru ei saa, üldistada või teisendada sellisteks mustadeks kastideks, millel on mingid defineeritud sisendid ja väljundid ja mille põhjal saad sa mingeid järeldusi  selle musta kasti sisu kohta. Selline paradigma oli seal koolituses pidevalt olemas, ma arvan, et see on siiski selle põhja ja vundamendi andnud.
                 
\question{Nojah, mis seal kasti sees on, ei jõua sulle keegi õpetada, sest homme on seal teine asi. Aga lähenemine on kasulik.}

Vaata, tollal muutused olid palju aeglasemad, et XT-d olid ikkagi nagu aastaid ja 286-t oli ka ikka aastaid, enne kui 386 ja 486 tulid. Lihtsalt nüüd on see muutuste tempo nagu palju radikaliseerunud.
                 
\question{Huvitaval kombel see alati tundub nii, et just praegu on palju kiirem, kui vanasti}

Võib-olla jah, on see, et vanemaks saades elu tempo ise muutub.

\question{Mis hetkel see kogukond hakkas selgelt alla jääma sellele, et igaühel oli laen ja liisin ja pere?}

BBS-i kogukond tegelikult oli ju mitte ainult need \emph{bulletin board}-id vaid toimusid ühisüritused. BBWinterid\index{BBWinter}, BBSummerid\index{BBSummer}. Iseenesest, need said alguse just sellestsamas süsoppide esimesest kokkutulekust, sealt hiljem hakati juba üritusi laiemalt tegema, kuhu olid kutsutud ka BBS-ide kasutajad. Ja seda tegelikult jätkus veel Interneti tulekust veel edasigi, et siin vist äsja veel arutati, et kellel veel BBS töötab. Ma üldse ei imesta, kui veel leitakse Eestist mõni töötav BBS kusagil virtukas tiksumas. 

Ma arvan, et minu jaoks ta loomulikult vajus laiali rohkem sellega, et ise ei jaksanud sinna enam panustada. Aga seal elu ja tegevus toimus veel aastaid-aastaid pärast sedagi. Ta hakkas  laiali vajuma sellest, kui internetipõhised keskkonnad hakkasid  kasutajaid lihtsalt üle võtma, kus suhtluskeskkond muutus  igapäevase töö  ja muude tegemistega seotud keskkonnaga ühtseks keskkonnaks, kus modemiga kuhugi helistamine tundus nagu pisut arusaamatu.

\question{Et kui BBS-is vahetati faile ja aeti juttu, siis Internetis sai tööd ka teha}

Absoluutselt, Internetis olid need Cafe ja OK jutukad  olemas ja ega enne Facebooke ja Rate-sid olid ju ka olemas Geocities ja mis iganes need keskkonnad olid. Ega need ei ole mingid uued asjad. Nad on vahetanud kesta ja vormi ja platvormi ja värvi ja natuke funktsionaalsust, aga inimesed on juba 30-40 aastat liikunud sellesama tegevusega  ühest keskkonnast teise.
                 
\question{Kes BBSummereid ja BBWintereid korraldas ja kes seal käisid?}

Ma mäletan ühte korraldajat, kelle kasutajanimi oli vist Kristrap aga Piret Part\index[ppl]{Part, Piret} oli vist pärisnimi. Tema oli üks kes aitas hiljem neid üritusi korraldada.

\question{See oli puhas kogukonna värk, mingit sponsorit taga ei olnud?}

Hiljem, kui üritused suuremaks läksid, siis me HNS-iga firma poolt loomulikult sponsoreerisime. Aga alguses need üritused olid piisavalt väikesed, et 10 inimest omavahel kokku tuleksid, natuke asja arutaksid ja selle juures mõned õlled teeks ei vaja sponsorit. Hiljem  arvutifirmad toetasid, samad inimesed töötasid ju arvutifirmades paljuski, olid olulised tegelased kui mitte omanikud, ja panustasid.
                 
\question{Ehk, Eesti Fido kogukond läks sujuvalt üle Eesti IT-tööstuseks?}

Jah. Absoluutselt. Ega paljud tuttavad nimed on ju sealtsamast pärit. Tõnu Samuel näiteks on ju samamoodi sealt keskkonnast pärit. Ma arvan, et lihtsam on võtta ja üles otsida tollased nimekirjad Fidonetist, mingi jututubade \emph{printout}-id, mida vist Mamers\index[ppl]{Mamers, Tarmo} peab ja vaadata, kes need nimed on, kes seal eksisteerivad. Ma arvan, et sa leiad nad praeguste it-meeste ja muude asjade hulgast üles.

\question{Kes neil üritustel käisid? Süsopid, ma saan aru, aga lisaks?}
                 
Ega ma ei oska sulle öelda. Süsopid loomulikult, aga eks oli neid, kes ei tahtnud, ei viitsinud või ei saanud ise BBS-i pidada. Aga kellele see oli lihtsalt üks selline asi, mida arvutiga teha. Külastada BBS-e.

\question{Nojah, kuna progemiseks olid barjäärid kõrged ja kogu aeg mängida ei jaksa, siis tehti midagi muud ka.}

Eks mõned tulidki läbi progemise huvi. Minule endale, tuli see läbi mänguhuvi, eks ole. See oli võimalus suhtlemiseks ja tarkvara vahetamiseks,  ka tehniliste teadmiste vahetamiseks.

Tollal kusjuures ju suhtlemine oli oluliselt raskem, et kui kõigil ei olnud isegi mitte kodus telefoni. Tänapäeval tundub  kõik on nagu käeulatuses, et sa pead lihtsalt taskust telefoni võtma, Google'isse otsingu panema. Tollal sa pidid otsima telefoni, mõtlema, kellele helistada, et küsida, kas ta teab kedagi, kes teab midagi. Fidonet andis ju ka  selle võimaluse, et sa said avalikku \emph{board}-i panna mingi küsimuse, et kas keegi teab, kuidas lahendada ühte või teist probleemi.

\question{Tuli vastuseid ka?}

Kindlasti. Kus nad pääsesid. Eks see sõltus  küsimusest. Eks seal oli asjalike jutte, olid oma läbu kohad, kus niisama jaurati, tehnilised vestlusringid.
                 
\question{Jah, aeg paneb asjad teise konteksti. Kui sul isegi kodus telefoni ei ole, siis võimalus rahvusvaheliselt inimestega suhelda on märkimisväärselt teise väärtusega, kui siis, kui sul on taskus mobiiltelefon ja Internetti ühendatud arvuti alati käepärast.}

No just. Mõtle sellele, et Google'it ei olnud tollal.

Google oli sul üle laua su kolleeg, naaber ja sõber ja sul endal pidi olema selline kontaktibaas ja varamu piisavalt suur ja lai, et teaksid kedagi, kes teaks kedagi, kes teab kedagi, kes oskab sulle öelda midagi.
                 
\question{Kas sa oled nõus Pronto ütlemisega, et väga paljus seesama kontaktibaas võimaldas väga loomulikult kogukonnast ärisse üle minna sest ka äris sõltus palju kontaktidest?}

Seda võiks ka nimetada \emph{street reputation}-iks, eks ju. Sul oli \emph{credibility} mingil määral olemas. Sind juba teati ja tunti selles keskkonnas. Kui tänapäeval räägitakse \emph{Estonian Mafia}-st, siis tollal see oli Fidoneti seltskond.

Loomulikult, eks paljud meie kliendid tulid läbi Fidoeneti või vähemalt teadsid meid sealt kaudu.

Paljud klientidest töötasid, eks ole, mingis pangas või firmas hiljem, firmad laienesid, tahtsid saada arvuteid. Kusagil oli neil ju palgatud mingi itimees, kes pidi selle probleemi lahendama ja ega tal ka ju ei olnud Google'it või e-poodi, kust  parimat pakkumist küsima minna. Tal oli endal ka inimesed, keda ta teadis ja usaldas kelle käest siis seda pakkumist minna küsima.
                 
\question{Ei olnud nii, et lähed poodi: \enquote{Palun mulle 16 arvutit}.}

Ütleme niimoodi, et ega see arvutiäri alguses oli ka see, et ega kuna valuutat ju väga palju kellelgi ringelnud, siis laoseisud olid ju olematud. Põhimõtteliselt võeti ettemaks, ettemaks maksti välismaale, selle eest oodati, kuni arvuti kohale laekus, siis pandi see kokku ja tarniti kliendile. Kui hästi läks, sai alla kuu aja kätte. Et ka see eeldas tegelikult ju usaldust, et sa annad kellegile nagu \ldots Ta maksis 1000 raha\ldots arvutikategooria mõttes hinnad ei ole eriti muutunud hea arvuti oli üle 1000 ja tavaline 1000.

See, et sa annad mingile matsile selle raha ära, ta ütleb, et \enquote{ära muretse, kuu aja pärast saad kätte}. See eeldab üksjagu usaldust,  ega valuutat ei olnud, enamasti tehingud alguses tehti rublades. Rubla ei olnud konverteeritav millekski muuks kui rublaks. Ja oli veel see aeg, kui paljudel asjadel oli kaks hinda: ülekande rubla hind ja sularaha rubla hind. Odavam oli sularahas, sellepärast et sa ei saanud alati pangast sularaha kätte. Olid mingid kindlad hetked, millal panka toodi sularaha ja siis sa pidid teadma, kas õigeid inimesi õiget aega, et saada sularaha.
                 

\question{See muidugi seletab kõiki neid legende, kuidas arvutifirmas ja pangas hoiti sularahapakke kuskil kapis ja vetsus ja kus iganes.}

Kui taheti midagi ülekandega osta, siis see oli nüüd meie enda valik, et kas me müüsime midagi valuuta või rublade eest. Kui  ülekande eest otsustasime müüa, siis oli hind kallim lihtsalt puhtalt seetõttu, et pärast selle raha kätte saamine pangast oli nagu oluliselt keerulisem.

\question{Ja sellest ajast sa oledki jäänud niimoodi arvutitega tegelevaid inimesi juhtima?}

Nojah, mul on nagu see taust selline, et mida aeg edasi, seda enam on mind huvitanud rohkem sisulised asjad, kuidas asjad töötavad. Ja vähem huvitanud inimeste juhtimine. Ütleme niimoodi, et inimesed on keerulised, arvutid on lihtsad.

\question{Kuidas sa infoturbe ja selle maailma juurde jõudsid?}

Esimest korda me jõudsime läbi väga praktiliste sammude. Me nimelt häkkisime Täitevkomitee arvuteid. Selleks, et sinna ligi saada ja mängima saada me avastasime, et tollal Õnnepaleeks kutsutud majas on olemas üks modemite peal töötav teenus, mille 
 kaudu sai abielusid, sünde ja surmasid registreerida. Ja muu hulgas sellesama modemi otsas sai eraldada kortereid. Nimelt kortereid ei saanud tollal osta, vaid neid eraldati sulle riigi poolt. 

\question{Ja teie avastasite koha, kuhu sai sisse helistada ja eraldada kortereid?}

Me avastasime, et sinna sai sisse helistada, aga teenus oli parooliga kaitstud ja parool oli jällegi eesnimi ja perekonnanimi. Me alustasime nii-öelda sellest tagumisest otsast, sellel infoturbel.

Mina ise sattusin Privadori\index{Privador}, siis juba  10-15 aastat hiljem aastal 1990, kui Tarvi Martens oli tollasest Küberneetikast\index{Küber} teinud investorite kaasabil sellise asja, mida tol ajal nimetati \emph{spin-off} ja mida tänapäeval kutsutakse \emph{startup}. Privadori  eesmärgiks oli  digitaalselt signeeritud dokumentide pikaajalise tõestusväärtuse loomise süsteem nimega TruSign. Seda muidugi hakati tegema kaks aastat enne ID-kaardi projekti algust, enne kui ühtegi signeeritud dokumenti keegi näinudki polnud.

\question{Väga huvitav. Krüptot enam paljakäsi ei tee, selleks on haridust vaja. Ehk, juba üheksakümnendate lõpus pidid omavahel kokku saama inimesed, kes olid iseõppijad ja minigid teistsugused inimesed, kes kindlasti ei olnud iseõppijad. KUidas see käis, seal mingisugustu hõõrumist ei tekkinud?}
                 
Enamus tollastest kolleegidest olid vanad tuttavad Fido ajast. Mina Privadori liitusin  küll turunduse ja müügi funktsioonis. Alles hiljem hakkasin tootejuhtimise ja nii-öelda juhatajana seal tööle. Aga algne funktsioon oli mul seal pigem müük. 

\question{See Fido seltskond ei olnud väga suur ju}

Ei olnud. Tegelikult kogu see seltskond, kellel üldse oli üheksakümnendatel ligipääs arvutitele, ei olnud väga suur. See oli veel see aeg, kui arvutid olid peamiselt firmades aga mitte kodudes. Üheksakümnendate lõpus alles hakkas tekkima see trend, kus firmad olid enam-vähem arvutitega varustatud, hakati ostma rohkem rakendustarkvara (või noh, ütleme turu trendid liikusid rohkem rakendustarkvara poole) ja eraisikud hakkasid endale koju arvutit ostma.

\question{Nõo kool oli juba pikalt olemas olnud selleks ajaks, sealt tuli igal aastal paarkümmend aruvit-inimest välja?}

Ega tegelikult  sovhoosidest ja kolhoosidest parematel olid omad arvutuskeskused olemas juba nõukogude ajal. Arvuti kui selline Eestis ei tekkinud päris üheksakümnendatest, vaid see ikka oli ammu enne minu sündi olemas. 


\chapter{Meelis Roos}
%!TEX TS-program = arara
% arara: myindex

\index[ppl]{Roos, Meelis}
\textbf{\enquote{Kuidas sina arvutite juurde said?}}

Mina sattusin arvutite juurde isa töö juures kaheksakümnendate lõpus. Füüsikud ostsid omale mõned arvutid elektromeetria laborisse, eksperimendi juhtimiseks. Arvutid olid CAMAC\sidenote{\emph{Computer-Aided Measurement And Control (CAMAC)}. Elektroonikastandard andmete kogumiseks ja seadmete kontrolliks. Kasutusel (osakeste) füüsikas aga ka tööstuses} kontrolleriga vene DVKd\index{Arvutid!DVK}\sidenote{\begin{russian}ДВК, Диалоговый вычислительный комплекс\end{russian}. Nõukogude personaalarvuti, ühilduv DECi PDP-11\index{PDP-11} perekonnaga. Varasemad mudelid on tuntud ka kui  Elektronika MS-0501\index{Arvutid!Elektronika} ja Elektronika MS-0502}.

\textbf{\enquote{Kus see kõik sündis?}}

See juhtus Tartus\index{Tartu}, Tartu Ülikooli\index{Tartu Ülikool} juures. Isa oli Tartu Ülikooli füüsika institituudis\index{Tartu Ülikool!Füüsika instituut} füüsik. Nad tegelesid elektroonika mõõteseadmete välja töötamisega ja näiteks said mingisuguse auhinna elektromeetri eest, mis eriti väikesi laenguid registreeris. Näiteks visati pastaka kuul, millel oli mingi laeng kusagilt läbi ja mõõdeti selle laeng mööda minnes ära. Neil oli lahe töögrupp, elektromeetria sektor, mida vedas üks mees, kes sellesse ilmselt uskus. Noored ülikoolist tulnud mehed tegid koos lahedaid asju, minu arusaamise järgi. 

Et katseid juhtida ja mingeid andmeid töödelda oli spetsiaalne lisablokk, mis käis arvuti külge. Seal sees oli analoog-digitaaal muundur (võibolla vastupidi ka aga igatahes nii pidi neid kasutati). Nad õppisid programmeerida, et suuta oma eksperimendi andmeid reaalajas kätte saada. 

\textbf{\enquote{Aga mis arvuti see selline oli, mis suutis andmeid niimoodi reaalajas kätte saada?}}

DVK-2M. Vene LSI-11\sidenote{DECi PDP-11 perekonna liige, tuntud ka kui PDP-11/03. Masinat tutvustati 1975. aastal ja ta oli oma sarjas esimene, mille CPU oli integreeritud. Mitte küll ühele, vaid neljale Western Digitali poolt toodetud \emph{Large Scale Integraton (LSI)} kiibile). Meelise sõnul: \enquote{PDP-11 oli legendaarne DECi masin iidsel ajal enne meie aega}} analoogid. Peaaegu täpne kloon aga natuke kohapeal ka täiendatud. Programmide poolt ühilduv aga mitte identne. DVK peal jooksis näiteks DECi originaal opsüsteem RT-11\index{RT-11}. RT-11SJ oli igapäevane opsüsteem, see oli \emph{single job} ja RT-11FB sellel oli \emph{foreground} ja \emph{background}, millega sai taustal jooksutada mingisugust teist tegevust. 

\textbf{\enquote{Kui vana sa olid, kui su isa need arvutid omale hankis?}}

Põhikooli teises pooles. Ega mul ei olnud põhjalikku teadmist, mida selle arvutiga teha saab. Minu jaoks sai arvutiga teha kahte asja. Kõigepealt, kui ma tegin isale teksisisestustööd, näiteks sugupuu andmete sisestamiseks, siis sain ma pärast seda kuni õhtuni mängida. Lemmik mäng oli \emph{wall}\index{Mängud!Wall}, seina pommitamine mingi reketiga. Mind pandi kohe tööle, et miskit kasu oleks. Mis ma niisama aega raiskan. Programmeerima õpetati ka, eks nad ise ka õppisid. Basicus\index{Keeled!Basic} ja Fortranis\index{Keeled!Fortran} ja CASICus\index{Keeled!CASIC}. See viimane oli CAMACi kontrollerite programmeerimiseks mõeldud Basicu ja Pascali\index{Keeled!Pascal} vaheline keel\sidenote{Ilmselt peetakse silmas keelt formaalse nimega \emph{ANSI Standard Real-Time BASIC}, mille spetsifitseerib IEEE standard \enquote{726-1982 - IEEE Standard Real-Time BASIC for CAMAC}}. Selles mina ei sattunud programmeerima. Aga ma õppisin Basicus programmeerima. Minu parim programm oli programm, mis ajas inimesega eesti keeles juttu. Ütleb sulle ühe lause, sina ütled lause, tema ütleb lause, ütled lause ja tema valib juhuslikult ühe lause. Aga ta suutis mõnikord teemas püsida. Näiteks ütleb \enquote{Osta elevant ära} ja siis järgmised kaks lauset olid, et \enquote{Kõik ütlevad nii aga osta elevant ära}. Enne ta ei läinud järgmist lauset valima kui ta oli kaks vastust saanud. Ja teiste töötajate lapsed mängisid seda ja neil oli lõbus. See oli lahe emotsioon, et ma tegin midagi, mis teistele lahe oli. 

\textbf{\enquote{Huvitav, et sa kohe hakkasid mängu tegema ja seejures kohe midagi AI-sarnast}}

See tundus kõige lahedam asi mida teha! 

\textbf{\enquote{Need füüsikud pidid ju kähku õppima, sest reaalajas riistvarast andmeid lugeda on ju keeruline}}

Neid oli vähemasti kolm meest, kes õppisid programmeerimist ja neil oli üks natuke noorem pundis, kes oli nende põhiline arvuti-mees ja kes seda vist paremini jagas, kui teised ja kelle juures CAMAC kontroller oli. Arvuteid oli vist vähemasti kolm tükki selle labori peale aga üks oli see põhiline eksperimendi juhtimise oma. See, mida mina kasutasin oli niisama masinakirjutaja toas. Seda sai kasutada siis näiteks programmide sisestamiseks ja muidu andmetöötluse jaoks. Näiteks isa tegi sugupuu üles joonistamist arvutisse. Oli puukujuline puu, puu läks vasakult paremale ja siis sai rull-paberile välja trükkida ja siis oli pärast mitmeid rulle. Kui koolis tulid mingid tudengid ja andsid igaühele paberi, kuhu oli natuke templeiti ette tehtud isa ema ja lapse kohta, et joonistage oma sugupuu üles, siis mina palusin isal ühe koopia välja trükkida. 

\textbf{\enquote{Aga miks sa lasid ennast sellesse suhteliselt igava andme-sisestaja rolli suruda? Lihtalt, et saaks mängida?}}

Algul selleks, et sai mängida aga kui selgus, et ise programmeerida saab ka ja see on täitsa lahe, siis ma pigem mängimise asemel keskendusin sellele rohkem. Ma ikka mängisin ka vahel, ma ei jätnud mängimist päris maha. 

\textbf{\enquote{Mis selle programmeerimise juures lahe oli? Mis sind köitis?}}

Ma tegin ka mingeid asju käsitsi. Näiteks ESC koodidega printerile õigeid asju saates\sidenote{\emph{Epson Standard Code for Printers, ESC/P\index{ESC/P}} on Epsoni poolt maatriksprinterite jaoks välja töötatud (ja termoprinteritel siiani kasutusel olev) keel, mis võimaldab juhtida rastrivõimekuseta printerit. Keel sai oma nime sellest, et tema käsud algasid sümboliga ESC (ASCII 27). Näiteks ESC E lülitas sisse ja ESC F välja rasvase trüki} trükkisin oma õpiku silte, kus oli boldis ja suuremas ja väiksemas kirjas kõik erinevatel ridadel asjad kirjas. Ja siis üks ema tuttav tahtis oma firma logo visiitkaartidele. See firma logo tuli siis teisendada EPSONi printeri graafika ESC-jadadeks. Ma joonistasin selle \emph{bitmap}ina üles aga siis leidsime, et ei tasu vaeva ja seda logo ma ei teinud. Aga programselt oleks selliseid asju lihtsam teha. See oli näiteks koht, kus ma leidsin, et programmist võiks oluliselt kasu olla. 

Oli firma nimega Tensiid, mille logol oli mingi kolmeharjuline neljast aatomist koosnev molekul visualiseeritud. Keskel üks lömmis ja kolm tükki külgede pealt sees. Nad tegelesid keemiliste mingisuguste ühendite sünteesiga. Ülikooli keemiahoonest välja kasvanud firma minu arusaamist mööda. Muu hulgas käisid aegajalt reisidel. Sellest kasvas välja Tensi Reisi, keemia asemel tegeletigi reiside korraldamisega. 

Aga mina üritasin alguses Tensiidi logo arvutis joonistada ja mõtlesin, et programm võiks seda minu eest teha. 

\textbf{\enquote{Kuidas õppimine käis?}}

Ma sain mingisuguseid venekeelseid raamatuid. Osalt raamatukogust isa tõi, osalt oli ehk mõni raamat tal töö juures olemas.  Need olid enamasti kusagilt laenatud ajutiselt. Näiteks mul oli segadus ASCII koodi ja \emph{escape} koodidega. \emph{Escape} koode tuli terminalile saata ja printerile sai saata ja. Siis ma mäletan, et küsisin isalt nõu, et mis neil vahet on et kas see on seesama asi. Ja siis oli raamatuid erinevaid. Näiteks oli üks raamat Basicu kohta. Mingisuguse käsu kohta on mul siia maani meeles kirjeldus, mis minu meelest ei sobinud niisugusse raamatusse \begin{russian}\enquote{эта команда работает хорошо}\end{russian}. See käsk töötab hästi. Minu meelest oli see lati liiga madalale laskmine. Minu meelest peaks kõik hästi töötama. Asjad tuleks nii teha. 

\textbf{\enquote{Sul oli ju siis päris korralik vene keele oskus?}}

Jah, ma olin üheksandas klassis umbes kui ma programmeerimist õppisin ja kannatas vene keelset raamatut lugeda küll. Meil oli põhikoolis selline vene keele õpetaja, kellega pidi õppima niiet mul tõenäoliselt oli üsna normaalne vene keele oskus selle vanuse kohta. Ma käisin Tartu 12. Keskkoolis\index{Koolid!Tartu 12. Keskkool}. Meil oli üks ukrainlanna, vana Zinaida Tovkatš vene keele õpetajaks kelle kohta meie kirusime et ta on väga range ja isegi haige ei ole kunagi, et muudkui peab õppima ja muidu ei pääse. 

\textbf{\enquote{Kas keegi sind õpetas ka või käis ainult raamatu järgi see asi?}}

Isa õpetas mulle neid asju, mida tema teadis ja õpetas blokk skeeme ja see keskis kuni keskkooli ajani välja, et kui mina tegin programmi ja see ei töötanud, siis oli kask viisi debugimiseks. Üks oli see, et ma trükin ta rullpaberil välja ja loen õhtul kodus. Teine võimalus on see, et ma joonistan selle asja blokkskeemiks ja lähen näitan isale. Sealt pealt tema oskas vigu leida küll. Ja blokkskeemiks joonistamisel leidsin ma tihti vead ise ka üles. Et blokkskeemid oli asi, mida isa mulle õpetas sest tema õppis programmeerimist nendega. 

Ja isegi kui ma Pascal keeles kirjutasin, mida isa ei osanud, ma sain ikkagi isalt abi blokkskeemide tasemel. Sest isal oli hea loogiline mõtlemina ja ta seletas mulle minu vead ära küll. Mis veel lõbus oli, ma süütasin kodus katelt. Minu ülesanne oli keskkütte katla alla tuli teha. Ja süütamiseks oli toodud vanapaberit füüsika osakonnast. Ja seal oli teinekord mingeid arvuti väljatrükke, mida ma lugesin. Panin paberi kõrvale ja ajalehed ja muud läksid katla süütamiseks. Näiteks ma leidsin Minsk 32\index{Arvutid!Minsk-32}-e\sidenote{Minsk-32 loodi kuuekümnendatel, nagu nimigi ütleb, Minskis. Tegu oli mitmest mudelist koosneva Minsk suurarvutite sarja kõige võimekama esindajaga. Oli laialdasel kasutusel, kuni asendati seitsmekümnendatel IBM 360 kloonidega} mingisugused crash dumpid või mälu dumpid kolmekümne kahe bitised. Ma olin üllatunud, et vau, minul on 16-bitised PCd (see oli tol hetkel hiljem vist kui ma juba PC taga olin) aga nende oli juba siis 32-bitine arvuti. Ja seal olid Fortran programmid, mida ma huviga lugesin. Isa kõrvalt ütleb, et ah, need ei ole suurt midagi väärt eriti et see mees, kelle programmid need on ei oska veel eriti programmeerida et tema Fortran programmide pealt pole eriti mõtet eeskuju võtta. Aga põnev oli neid lugeda sellegi poolest. Fortranit õppisin keldris katla kütmise juures!

\textbf{\enquote{Miski pani sind tulehakatust lugema, mis see oli?}}

Seal olid uued põnevad asjad!

\textbf{\enquote{Kas selle asja juurde mingi kirjanduse või muusika huvi ka käis?}}

Ulme huvi natuke oli. Mul õnnestus saada venekeelsed Asumi seeria raamatud, mida oli rohkem kui kaks esimest. Asumid mulle meeldisid ja ühe isa sõbra käest laenasime venekeelsed ülejäänud Asumi raamatud. Ja mul õnnestus vene keeles raamatut lugeda ja ma olin selle üle sügavalt üllatunud. Isa algul luges neid ise ja mina lugesin ka vist midagi neist. Niiet ulme huvi oli küll aga see ei olnud niimoodi väga sügavavalt. Seda oli valdavalt nii palju, kui kodus sattus Mirabili ja mida iganes seal ulmekaid olema. Need said kõik läbi loetud aga see ei olnud esialgu kuidagi eriti seotud arvutitega. Arvutid olid asi, mis tuli reaalsest maailmast. Näiteks sõitsin bussiga koju ja ükskord Pärmivabriku peatuse juures mööda sõites parajasti ema seletas mulle arvuti viiruste kohta mida ta oli lugenud kuskilt Horisondist või mõnest niisugusest kohast. Ja väga põnev oli. Parajasti sõitsime Pärmivabriku peatusest mööda, kui ma esimest korda arvutiviirustest kuulsin. Seda ma mäletan. 

\textbf{\enquote{Kas sa olümpiaadidel ka käisid?}}

Jaa, käisin. Neljandast klassist saadik käisin matemaatikaolümpiaadil. Seal nägi natukene tuttavaid. Oli naljakas korrelatsioon: need lapsed kellega ma olin koos käinud ülikooli töötajate lasteaias, neist nii mõnigi oli seal olümpiaadidel. Järgmine laine sellega oli minnes keskkooli. Miks ma läksin vanast koolist ära? Vanas koolis oli nii, et keskkoolis pidi tulema kaks klassi. Reaalkallakuga ja humanitaarkallakuga. Ja humanitaarkallakuga pidi see \enquote{A} ja eliitklass tulema kuhu paremad õpilased lähevad ja ülejäänud võiks minna sinna reaalkallakuga klassi. Ma leidsin, et see on lati alla laskmine, et ma tahaksin ikka paremat. Mind kutsuti Nõkku\index{Koolid!Nõo Keskkool}. Hilisem ülemus Cyberneticast\index{Cybernetica} toonane Nõo kooli direktor Uuno Puus\index[ppl]{Puus, Uuno} saatis laiali kõikidele olümpiaadikutele Nõo kooli kutseid. Sain ka. Kaalusin. Oli kaugel. Raske. Siis selgus, et esimene keskkool Tartus\index{Koolid!Tartu 1. Keskkool} on ka täitsa kõva. Helistasin kooli ja küsisin, et kas teil arvutiklass on. Direktori võttis vastu ja reklaamis, et neil on väga hea arvutiklass. Selle peale ma otsustasin, et ma lähen esimesse keskkooli. Viisin paberid esimesse keskkooli, kui sisse astusin 1990 oli juba Hugo Trefneri Gümnaasium\index{Koolid!Hugo Trefneri Gümnaasium|see{Tartu 1. Keskkool}}. Olid väga head arvutid. Oli arvutiklass ja Juku\index{Arvutid!Juku} klass. 

\textbf{\enquote{Sul oli siis selge arusaam, et sa just sinna kooli tahad minna?}}

Jah, ma läksin nimelt sinna. Selle kohta tegi ajaloo õpetaja meil kunagi pisikese kiire küsitluse üheksanda klassi kevadel. Et paljud teist siia jäävad ja paljud lähevad kuhugi mujale. Ja mina olin see, kes leidis, et ma tahan ise oma tulevikku kujundada, et mulle sobib see asi paremini. Ja siis ta küsis kolme tema nina all oleva tegelase käest. Esimeses pingis sattusin mina istuma ja minu tagant kahe tüdruku käest, kes olid ka kätt tõstnud, et lähevad mujale küsiti, mis nad teevad. Need oli täpselt need kolm, kes läksid esimesse Keskkooli. Niiet kõik, mis ta küsis, sai vastuseks, et lähme ära esimesse keskkooli. Nemad läksid teise paralleeli, sinna bioloogia-keemia harusse. Aga see tundus olevat sinnakanti, et umbes see vanus oli koht, kus mõned hakkasid ise mõtlema oma tulevikule ja planeerima ja mõned lasid isevoolu teed minna. Et mõned olid need, kes planeerisid. 

\textbf{\enquote{See oli see aeg, kui ühiskonnas hakkas juba muutus tulema, eks ole}}

Natuke oli juba varem selles mõttes, et koperatiivid olid varem ja asjadest tohtis rääkida varem. Selle sama üheksanda klassi jooksul ma jõudsin kaks korda kirjutada mingisuguseid referaate millest võib olla aasta varem oleks vanematel pahandus tulnud. Aga siis juba tohtis. Sellesama õpetaja kohta oli teada, et ta on üks paras punane. Aga temale ma need referaadid kirjutasin ja sain kiita, mis oli üllatav. Ma olin üllatunud, ma mõtlesin, et tuleb kaitsta kuidagi oma seisukohti. 

\textbf{\enquote{Kas sind keskkooli ajal tööle ei tõmmatud kuhugi?}}

Ainult natukene käisin. Tiražeerisin isa töö juures elektromeetrite trükkplaate. Joonistasin ahjulakiga ja risti ära lõigatud otsaga süstlaga rajad, söövitasin plaadi ära, tinatasin ära ja jootsin sinna peale kõik elemendid vastavalt skeemile. 

\textbf{\enquote{Aga see tahab ju käelist oskust ja elektroonikahuvi, kust sul see?}}

Seitsmeaastaselt oli mulle vist isa töö juures kolb kätte sattunud esimest korda kui ma suvalisi tükke kokku jootsin. Eks ma oskasin kolbi hoida ja elektroonika huvi mul oli. Aga elektroonikat ma ei osanud, ei ole kunagi ära õppinud analoogelektroonikat. Üldisi põhimõtteid tean aga ise midagi teha ei ole osanud. 

Digielektroonika oli seal kõrval. Kui keskkool hakkas läbi saama ja ülikooli oli vaja minna, siis mina olin neljandast klassist peale kindel olnud, et ma lähen füüsikat ja nimelt elektroonikat õppima. Aga siis mingid arvutid tulnud, kah põnev elektroonika värk. Aga arvuteid sai matemaatika poolt ka õppida. Mul oli kuhugi ilma eksamiteta sisse saamisesd, äkki matemaatikasse ja füüsikasse olümpiaadi tulemuste pärast või midagi ja ma otsustasin matemaatika kasuks sest füüsika osakonnas ma olin kogu aeg kohal ja mulle ei meeldinud see. Tundus nihukene, et kui midagi ära tahta teha siis peab ise muudkui tegema. Oli nihukesi saarekesi, kes tegelesid oma kitsa erialaga aga laiemat kandepinda ma ei märganud seal. Oli töögruppe, kes olid vingel tasemel ja tegelesid oma asjaga. Aga võibolla ma ei sattunud õigete inimestega kokku aga tundus, et pigem nihukene nagu oleks seisev konnatiik et igaüks on seal kinni, kus on ja nii on. 

No seal oli huvitavaid ja põnevaid asju ka. Näiteks olid füüsika päevad, kus rääkis Undo Uus\index[ppl]{Uus, Undo}, keda mu isa käis kuulamas, rääkis materialismi ümber lükkamisest filosoofiliselt. Isa tuli koju, jutustas. Panin kõrva taha. Selliseid asju oli sealt ikka päris mitmeid. Sellist füüsikalist maailmapilti tuli vanemate kõrvalt üksjagu, see oli mul olemas. 

\textbf{\enquote{Kuidas sa siis ikkagi matemaatikat sattusid õppima? Lihtsalt seepärast, et sai eksamiteta sisse?}}

Füüsikasse ma oleks vist ka saanud ilma eksamiteta. Eksamid ei oleks probleem ka olnud, ma arvan. Lihtsalt laisk. Laisad me olime kõik. Keskkoolis klassijuhatajal tuli üheksandas klassis üritada meile ikka auku pähe rääkida, et poisid olge tublid ja võtke tehke need eksamid ikka ära, siis saab medalile pretendeerida, muidu ei saa. Vaja oleks ju medaleid ka. Siis me tegime vist kolm medalit klassi peale või midagi. Mina sain hõbeda. Ma täpselt ei mäletanudki. Kunagi hiljem kooli koduleheküljelt lugesin, et ma hõbemedali sain. Seda ma mäletasin, et medal oli aga mis medal, seda ei mäletanud. Polnud oluline, see tuli iseenesest. 

\textbf{\enquote{Ühesõnaga, matemaatikasse sa läksid seepärast, et füüsika tundus natuke seisev vesi olevat?}}

Jah. Ja ma olin kuu aega enne paberite sisse andmist kindel, et matemaatikasse ma küll ei lähe. Me käisime Moskva\index{Moskva} lahtisel olümpiaadil matemaatikas koolist tiimiga. Seal olid mingid doktorandid, kes tegelesid meiega. Seal ühtlasi toimus \begin{russian}Международная конференция старшикласников "Наука, природа, человек"\end{russian} kus keskkooli õpilased said ise asju esitada, mis nad olid teinud. Keegi oli teinud kiiret vektorgraafikat, et voldime siin kuubikut kiiremini, kui AutoCAD või mis iganes wirefreimis. Ja ägedaid asju oli tehtud. Seal oli mingit Hollandi rahvast, oli rahvusvaheline küll. Seal need doktorandid, kes meiega tegelesid, olid nihukesed parajad uhuud. Näiteks tuleb tegelane hommikul tahvli ette triiksärk on lükatud alukate sisse, alukad ulatuvad kümme sentimeetrit pikkade pükste peale välja ja tuleb niimoodi tahvli ette. Ma leidsin, et vot matemaatikuks mina küll ei lähe. Aga siis ma mõtlesin ikkagi ümber. Matemaatikuks ma ei tahtnudki, ma läksin neid arvuteid õppima Matemaatikateaduskonna\index{Tartu Ülikool!Matemaatikateaduskond} poolt. Mitte eleketroonika poolt aga programmeerimise poolt. 

\textbf{\enquote{Kuidas sulle ülikooli üleminek tundus? Sa ütlesid, et olla laisk olnud aga minu mälu järgi pidi ülikoolis kohe hakkama tööd tegema?}}

Jaa. Keskkoolis ma sain endale lubada laisk olemist isegi seal eliitkoolis, no mingil tasemel vähemasti. Ja ma sain keskkoolis arvutimängude mängimise isu täis mängida. Ostsin omale üheksanda klassi lõpus ZX Spectrum-i\index{Arvutid!ZX Spectrum}\sidenote{ZX Spectrum oli Sinclair Research'i poolt 1982. aastal Ühendkuningriigi turule lastud 8-bitine personaalarvuti, mõeldud peamislt koduseks kasutamiseks. Selle kloone liikus Nõukogude Liidus hulganisti, skeemid olid kogunisti hobiajakirjades avaldatud} Leningradi turu klooni 1500 rubla eest kui rubla juba kukkus. Siis oli suur rahanumber aga ma sain mängida täis oma mängimise isu. Joystick sai peeneks mängitud ja plastmassi paigatud alumiiniumiga. Tuttav treial tegi sinna uue varre, pärast kippusid kontaktid läbi põhja tulema. Aga Spectrum oli nii hea arvuti, sellest sai aru igat pidi! Basicus sai programmeerida, Assembleris\index{Keeled!Assembler} sai programmeerida Z80 peal. Sellest sai täiesti aru saada. Ja elektroonikast võib peaaegu üleni aru saada, välja arvatud videopildi kildi genereerimise osa, see ULA kivi või selle realiseerimine niisama lause-elektroonikana nagu selles vene variandis oli kui seda kivi ei olnud kloonina võtta. Niiet ma sain sõbra Sinclairi diagnoosimisega hakkama, et sul on ROMi see ja see jalg lahti ja ei anna kontakti. Et sellest tulevad tähtedel vertikaalsed kriipsud läbi nagu dollarimärgid. Sest ROMis oli see tähtede tabel ja kui seal bitt oli maas, siis on vertikaal. Seal oli kaks ROMi kivi et sellel ROMi kivil see jalg peab järelikult mitte kontaktis olema. 

\textbf{\enquote{See tähendab, seda, et sa pidid neid asju põhjalikumalt uurima?}}

Skeeme ma ikka kuskilt raamatutest ja niimoodi nägin. Keskkooli lõpus, kui Venemaal käisin, ostsin metroost omale raamatu \begin{russian}Введение в схемотехники IBM PC / AT\end{russian}. Venelased olid 286 skeemid välja ajanud arvuti järgi ja üles joonistanud. Neil oli seal viga minu meelest. Mingi reset signaal, selles oli aktiivne null versus aktiivne üks kusagil vist segamini, mul on nihuke mälestus. See oli lõbus igatahes avastada niisugust asja trükitud raamatus. See oli seesama kord, kui me olümpiaadil käisime ja konverentsil, millest me enne ei teadnud midagi, kui me sinna kohale sattusime. Meil ei olnud mingeid ettekandeid, me kuulasime niisama, mis räägitakse. Ja vaatasime, mihukesed on kenamad tüdrukud. Üks vene Maša oli kõige kenam. 

Olümpiaadil me eriti hiilgavaid tulemusi keegi ei saanud. Mina sain meie pundist kõige parema tulemuse, sest ma ei joonud eelmisel õhtul alkoholi. Aga seda oli seal saada ja seda käis ringi ja siis järgmine hommik oli pohmakas ja siis inimesed ei esinenud oma võimete tasemel. Ja mina olin meie omadest parim kuigi seal oli meil vähemasti üks nendest meestest, kes veel oli kaasas, oli parema peaga, kui mina. Minu jaoks oli õppetund, mida rõõmsalt teistele edasi jagada, et näe, olümpiaadi tulemus sõltus selgelt sellest. 

\textbf{\enquote{Räägi palun ülikoolist. Me sattusime seal 1993. aastal kokku, kuidas sulle see matemaatika tundus, mida me kohe esimese semestri alguses saama hakkasime?}}

See oli üks suur kukkumine. Ma mõtlesin ülikooli tulles, et ma tean, mis on reaalarv näiteks. Siis tuleb Matemaatilise Analüüsi esimene loeng, kus hakatakse neid defineerima. Ja kõike pidi algusest hakkama defineerima, ainult nende definitsioonide otsa ehitati kogu seda kõike. See tahtis palju harjumist ja palju tööd, mina ei olnud harjunud tööd tegema. Mina mõtlesin, et ma oskan programmeerida, kui ma ülikooli tulin. Aga asi, mis mulle näitas, et on veel palju, oli Rein Pranki\index[ppl]{Prank, Rein} mat. loogika õppeprogrammid, kus tehti tõestuspuu layouti ja ma mõtlesin, et vau, puu layouti niimoodi teha ma ei oska. Me õppisime seda küll alles hiljem umbes kolmandal kursusel Varmo Vene\index[ppl]{Vene, Varmo} funktsionaalses programmeerimises, kus me mingi minimaxi ülesande tüübi näite ülesandeks tegime puu layouti. No seda oleks saanud rekursiooniga esimese kursuse järel ka ehk kuidagi tehtud saanud. Aga see oli jah näide sellest, et ei ole kõik ikka triviaalne jõuga peale ja teeme ära. 

\textbf{\enquote{Matemaatiline analüüs, eriti Matemaatiline Analüüs II, võttis meil kursuse peal palju rahvast hõredamaks, see tahtis harjumist saada}}

Ja algebra tahtis ka. Kogu see matemaatiline lähenemine, et me ehitame asju üles mingite definitsioonide ja aksioomide millegi otsa. Kogu see asi tahtis kõvasti tööd. Ja ma kukkusin esimesel kursusel haiglasse. Et sessi ajal ma ei jõudnud mõnesid eksameid tehtudki, tegin neid alles järgmise semestri sees. Käisin dekaanilt küsimas sessi pikendust, sest vanemad õpetasid, et nii tuleb teha. Siis dekaan ütles, et meie ajal enam niisugust asja pole, lihtsalt tehke need eksamid ära, kuidas saate. 

\textbf{\enquote{Mis hetkel oli võimalik minna arvutiteadust õppima?}} 

Selleks oli kas esimese aasta järel spetsialiseerumine. Mingid põhimoodulid oli vaja ära teha ja siis vist sai. Kuna ma sain need vist kokku, siis mina kaldusin üldisest õppekavast kõrvale sellega, et mina läksin võtsin koos aasta vanematega põnevaid arvuti aineid. Käisin aasta vanema rahvaga koos kuulamas asju, mis olid lahedad. Peast ei mäleta, aga igasugu aineid, mis meil seal oli. Ja siis järgmine aasta tuli mul võtta siis need asjad ka, mis õppekavast puudu olid. Minu oma kursus oli need ära teinud, mina tegin neid siis koos aasta noorematega. Mingeid tõenäosusteooriaid ja mingisuguseid matemaatika aineid. 

Juhtus ka seda, et ma kirjutasin maha kodutöö programme teiste pealt. Meil oli mingisugused algebra ja analüüsi numbrilised meetodid, kus me arvutusmeetoditega numbriliselt tegelesime. Ma sain algoritmidest aru, nad ei pakkunud mulle algoritmi tasemel pinget ja ma ei viitsinud neid teha. Kui ma olin aru saanud, mis seal tehakse, siis sellest piisas. Siis oli üks lahke kaastudeng Jane, kelle programme ma kasutasin selleks, et neid esitada. Muutsin vist natuke treppimist ja muutujad nimesid mõnes. Mäletan, ma kirjutasin ühele kommentaaridesse üles \enquote{Viimati modifitseerinud Meelis Roos}. Eks see praksi juhendaja, et neid üksteise pealt üksjagu maha võetakse ja kuna ta lasi endale ette seletada, mida see programm täpselt teeb ja algoritm, siis sellega polnud probleemi, ma sain kõik asjad ilusti tehtud. Kirjutasin programme tüdrukute pealt maha, sest ma ei viitsinud programmeerida. 

\textbf{\enquote{Kas see ülikooli arvutuskeskus seal Liivi tänaval ei neelanud sind kuidagi endasse, nagu ta nii mõnedki neelas?}}\index{Tartu Ülikool!Matemaatikateaduskond!Liivi tänava õppehoone} 

Neelas ka mind aga natuke teistel viisidel. Mina ei kadunud ära Muda\index{Mängud!Muda} mängima. Muda oli küll tore, aga siis kui ma kirjutasin oma telneti klienti, siis sai seda Muda serveri vastu testida näiteks. Selleks oli Muda tore. 

\textbf{\enquote{Miks sa kirjutasid oma telneti kliendi?}} 

Võrguprogrammeerimise harjutamiseks. Tahtsin osata sokliühendusi igasuguseid teha. Ma kirjutasin oma netcati laadset mingit asja, mis telneti handshake'i ei teinud mingisugust ja ei osanud echo offi ja selliseid advanced featuure vaid lihtsalt sokli kuhugi ühendas. Sellise asja kirjutasin endale, et torkida igasuguseid asju. Seal olid mingid mured stiilis kui pikkade pakettidega asju saata ja vastu võtta ja TCP võis selle suvalisel koha pealt ära hakkida. Ei saanud eeldada, et kui teiselt poolt rida sisse kirjutakse, et sa täpselt rea suuruste tükkidena kätte saad. See oli põnev.

Aga mind neelas see arvutuskeskus natuke teistmoodi. Teisel korrusel Ülo Kaasiku\index[ppl]{Kaasik, Ülo} kabineti kõrval oli magistrandide arvutiklass, kus olid värvilised Sunid. See oli ette nähtud magistrandidele aga kellelgi ei olnud eriti probleeme, kui mina ka sinna imbusin. Aegajalt seal ei olnud kohti ja tuli ette, et ma kellelegi kohta pidin loovutama mõnikord aga enamasti töötas. Aasta vanema Raul Tölbiga\index[ppl]{Tölp, Raul} istusime seal koos ja seal sai õpitud ära Unix. 

Ja lõbus oli omakorda see, kuidas ma üldse sinna Unixit kasutama sattusin. Seda ma võin lausa rääkida, kust on pärit minu kasutajanimi mroos. Minu esimene online konto oli masinas vask.ut.ee\index{Masinad!vask.ut.ee}. See oli VAX\index{Arvutid!VAX}\sidenote{Arvutisari, mille töötas DEC välja seitsmekümnendate keskel. Siiani üks kõige tuntumaid omalaadseid arhitektuure, oli ta PDP-11 edasiarendus, peamiselt mälu virtuaalse adresseerimise suunas. \emph{VAX - Virtual Address Extension}} tüüpi arvuti VMS\sidenote{VAX arvutite \enquote{kohalik} operatsioonisüsteem} opsüsteemiga. Nihuke umbes kuupmeetrine kast pluss kettad kõrval. Teine VAX oli rubiin.physic.ut.ee\index{Masinad!rubiin.ut.ee} füüsika majas. See oli MicroVAXm sahtlitumba suurune masin ainult. Vot need olid VMSid. Esimesel kursusel, selle asemel, et sessi ajal õppida, mina olin raamatukogust võtnud omale VAX VMSi raamatu ja õppisin VMSi. Seal oli huvitavaid asju! Näiteks olid struktuursed failid. Sa võisid tekitada tühja faili, millel on ette antud kirje struktuur. Opsüsteemi tasemel oli record management system, RMS, millega mingis keeles kirjeldati struktuur ära ja tekitati selle kirjelduse järgi fail. Fail võis olla ka tühi aga tal oli struktuur olemas. 

Õigusi oli seal jõle palju ja keeruliselt. Kogu see õiguste süsteem op süsteemis oli keeruline tokenite süsteem. Windows NT\index{Windows NT} on selle sisemiselt pärinud või umbes niimoodi. Nii keerukas ei ole minu meelest kui VMSis aga kui ma nägin Windows syscalli create process koos portsu argumentidega, siis tuli tuttav ette sest VMSi sys\$createprocess oli umbes samasuguse listi argumentidega. sys\$ käis syscallide funktsioonide nimede ette lihtsalt. 

Sealt ma käisin näiteks Lynxiga veebis surfamas, tõmbasin mingeid faile FTPga, mida ma sain kuskilt kolmandat teed mööda kuidagi flopi peale. Käisin internetis veel midagi lugemas. Ma eriti ei programmeerinud VMSis. Kui vaja oli kursaõele Pascalis programmeerimist õpetada aga ainult VAXu klass vaba oli, siis ma näitasin talle Pascalis programmeerida VAXu peal. Ta oli väga üllatunud, et saab seda arvutit ka programmeerida. Aga sai. Aga seal oli lahe programm nimega swim, mis lasi ühe terminali peale multipleksida mitu akent. Akende suurusi lausa sai muuta. Ja see oli lahe ja sellega ma kasutasin kolme rakendust korraga. Aga swim kippus ajama terminali hanguma, kõditas vist mingit VMSi terminali draiveri bugi või mida iganes ja siis tuli leida administraator, keda tihti majas ei olnud või siis keegi sõber tudeng logis kuhugi üle võrgu rubiini\index{Masinad!rubiin.physic.ut.ee} ja talk-is Ville Hallikuga\index[ppl]{Hallik, Ville}, kes oli sealne VMSi admin ja kellel oli juurdepääs vaske olema ja kes sai tulla ja terminali päästa sest selle terminali tagant ei saanud keegi enam midagi kasutada, terminal oli hangunud. Tappis swimi ja mingid asjad ära seal niiet terminal sai jälle vabaks. Ja swim oli tülikas. Ja siis keegi rääkis, et arvutiteaduse instituudi SUNides on Unixis programm nimega screen, millega seda sama teha saab. Ja siis tekkis mõte, et kasutaks seda. Ma olin Unixit seni juba korra kasutanud. Math.ut.ee-s\index{Masinad!math.ut.ee}, kui tekkis online võrk, tuli 386BSD. Ja see upgreiditi 93. aasta lõpus mingile uuele tundmatule opsüsteemile. Sinna osteti 486 arvuti asemele, suure kahekigase scsi vindiga ja selle scsi kaardi jaoks ei sobinud enam 386BSD vaid pandi asemele mingi uus tundmatu asi nimega Linux. Versioon 0.99PL midagi. 

\textbf{\enquote{Kust selline asi sattus Tartu linna?}} 

No aga kust 386BSD sai? Internet oli ju olemas. Ja kasutajad koliti 386BSDst Linuxisse siuhti üle ja mul oli mingis Linuxis kasutaja. Jaanuaris umbes apgreiditi see Linux ära versioonile 1.0.2, kerneli versioon. Mul oli siis Linuxis kasutaja. Ma olin natukene nuusutanud Linuxit. Kui ma tahtsin seal Liivi tänaval Unixi screeni, siis math.ut.ee ühendus oli aeglane, lagis päris kõvasti. 9600ne ühendus jagatud paljude kasutajate ja meilide ja muude vahel. Siis ma küsisin sinna cs1, cs2, cs3 olid masinad ühise loginiga. Küsisin omale cs3-e (hilisem romulus.cs.ut.ee) konto ja põhjendasin seda, et tahaksin näppida mõnda mitte-Linux Unixit. Seal oli Solaris. Ja see tundus Toomas Soomele\index[ppl]{Soome, Toomas} piisavalt hea põhjendus. Toomas Soome kasutajanimi oli tsoome, ma mõtlesin, et ahaa, et eks Unixis käib see niimoodi. Küsisin siis omale tema süsteemi sama skeemi järgi kasutajanimeks mroos. Antigi. Seda ma olen sellest ajast edaspidi kasutanud igal pool. Isegi kui mul on kodus testarvuti  siis seal olen ma ka harjumusest mroos. Et tsoome mulle kasutajanime teeks tuli öelda, et ma tahan Solarist kasutada ja kasutajanimi peaks ka samas formaadis olema, et võimalikult vähe küsimusi oleks. 

Mul möödunud aastal oli väga sürr kogemus, kui kevadel võttis minuga ühendust Toomas Soome, kellel oli siiamaani magister tegemata. Ta tahtis, et ma juhendaksin tema magistritööd. Ma mõtlesin, et muna õpetab kana, et mida mina siin teen. Aga tal oli korralik tehniline töö olemas ja mina teadsin, mismoodi üks magistritöö peab enam-vähem välja nägema. Sellest teadmisest oli kasu, niiet see töö sai tal vormistatud magistritööks ja ta tegi selle edukalt ära. Aga algul lihtsalt oli väga sürr reaktsioon. Arvutiteaduste Instituudis\index{Tartu Ülikool!Matemaatikateaduskond!Arvutiteaduste Instituut} oli terve hulk rahvast, kes tegid hiljem magistrit. 

\textbf{\enquote{Kas sind teadust ei tõmmanud tegema?}} 

Ei, vot teadust tegema ei ole mind kunagi eriti tõmmanud ja keegi ei suutnud mulle ka auku pähe rääkida sel teemal. Väga ei proovitud ka. Meelitati erinevate viisidega, mingeid materjale ette söötes. Materjalid olid nii teadusega kui mitte-teadusega seotud. Näiteks Jaanus Pöial\index[ppl]{Pöial, Jaanus} jagas mulle omal algatusel kunagi Java Language Specificationi, et näe üks uus moodne asi. Et selliseid asju ülikoolist ikka sattus. 

Ma mäletan, ma olin rebane. Ma ei olnud veel spetsialiseerunud Arvutiteaduse Instituuti informaatika erialale. Aga mul oli vaja kusagil välja trükkida viietollise flopi pealt mingit tekstifaili. Ma lihtsalt vajusin kohale Liivi tänavale ja käisin mööda uksi koputamas. Äkki oli mingi laupäev ka või midagi või muidu õhtune aeg et seal ei olnud palju rahvast. Sattusin Mati Tombaku\index[ppl]{Tombak, Mati} ukse taha, kes lahkelt lasi trükkida. See oli raamatukogust mingisuguse kataloogi otsingu tulemus välja trükitud mingi raamatute otsimiseks. Äkki oli laupäevasel päeval vaja välja trükkida. Ja siis Mati Tombak oli see, kes lasi mul trükkida. Ja sellest tekkis nihukene tänutunne kogu selle ATI vastu, et siin on lahked inimesed. See oli minul nihukene esimene sedasorti kontakt. 

\textbf{\enquote{Millal sa tööle läksid?}} 	

Minu esimene ametlik töökoht oli Tartu Ülikooli Täpisteaduste Koolis\index{Tartu Ülikool!Täpisteaduste Kool} metoodik. See oli postmasteri töö tegelikult. Aga postmasteri nimelist ametinimetust ei olnud, oli metoodik. Korraldati programmeerimis kursust e-mailitsi koolides. Ja mina olin see, kes pidas arvet selle üle, kellel olid mis ülesanded lahendatud ja saata neile järgmisi. Arvutiõpetajad, kellele vastused saadeti ja kes parandasid saatsid minule seisu ja mina siis selle järgi saatsin edasi. Mina olen laisk inimene. Mina esimesel tööpäeval võtsin nägin pool päeva vaeva ja kirjutasin skripti. Panin kuhugi tekstifaili valmis nimed. Programm võttis sealt järjest nimesid ja saatis neile ära ja pidas arvestust, et kellele on juba saadetud, et kellelegi topelt ei saaks. Ja kui ma selle skripti rubiin.physic.ut.ee tollane Füüsikamaja Unixi server kõristas umbes pool tundi. Pärastpoole ma õppisin nice käsu ka ära. Aga see tähendas, et kogu minu edasine töö pärast selle skripti kirjutamist oli copy-paste meili seest sinna sisendfaili ja skript tööle lükata. Automatiseerisin oma töö lihtsalt ära. 

\textbf{\enquote{Aga kuidas sa sinna sattusid?}}

Ma arvan, et Indrek Jentson\index[ppl]{Jentson, Indrek} Täpisteaduste koolist kutsus mind, kes mat teaduskonnas oli vanem tegelane ja olümpiaadidega tegelenud. Tema kutsus mind. Ma läksin Täpisteaduste Kooli ukse taha, tuli Viire Sepp\index[ppl]{Sepp, Viire} vastu, kes juhataja oli, ütlesin, et tere, tulin töö lepingut tegema. Mis töölepingut? Ma siis seletasin, et Indrek Jentson saatis postmateri töölepingut tegema mind siia. Kuskil 95 või 96 algul, täpselt ei mäleta. 

\textbf{\enquote{See oli üsna vara ju? Tuleb häbiga tunnistada, ma läksin 93. aastal tööle juba}}

Te olite Veljo Haguga\index[ppl]{Hagu, Veljo} Korelis\index{Korel IN}, eks? Ma käisin Veljo töö juures vahel. Seal olid mingid mängud. Dune'i\index{Mängud!Dune} mängis Veljo näiteks õhtul näiteks millalgi kui ma sinna sattusin, vaatasin, kuidas see käib. Mängimisega ei olnud mul erilist suhet. Ma sain keskkooli ajal oma mängimise isu täis mängida Sinclairi peal ja lülituda juba programmeerimisele juba sellega, et ma tean, et see on palju põnevam asi. Ma kirjutasin näiteks oma binary editori näiteks, millega mängudest järgmiste levelite paroole välja nuuskida ja muid nihukesi asju. See oli juba keskkoolis, et sai igasugustel arvutiturva teemadel nuusitud ja huvi tuntud. 

Arvutiturva teema on mul keskkoolist saadik sees tõesti. Meil olid keskkoolis väga põnevad võidujooksud arvutiõpetajaga. Väga harivad. Näiteks oli õpetaja arvuti klaviatuur parooli all. Aegajalt tehti sellega meilivahetust niiet masinal klaviatuur oli lukus aga muidu masin töötas edasi IBM PS/2\index{Arvutid!IBM PS/2}\sidenote{PS/2 oli IBMi kolmas personaalarvutite põlvkond, mida tutvustati 1987. aastal. Paljud tolle masina innovatsioonid nagu näiteks VGA video muutusid \emph{de facto} standardiks pikkadeks aastateks}tedel oli mingi selline keyboard locki feature. Küll ma üritasin leida meetodeid sellest mööda hiilimaks. Kui ma sain mingeid skeeme kuskilt näha, siis mul tekkis idee, kuidas i8042 klaviatuurikontrolleri kaudu teha masinale sobivat warm booti, et sealt mööda hiilida aga klaviatuuri kontroller oli lukus edasi. Kirusin, et IBMi omad on kavalad olnud. See oli algul. 

Lõpuks selle arvuti parool saadi teada lihtsal viisil. Vaadatai üle selle arvutiõpetaja õla, kes aeglasemalt tippis. Kui see oli teada saadud, ega me sellega midagi ei teinud, see ei olnud eesmärk. Aga minul oli edasi põnevam see, kui keskkoolis viimasel aastal oli 386d kohale jõudnud ja nende C ketas, kõvaketas pandi kirjutuskaitse alla nii, et mingi spetsiaalne draiver laaditi config.sys-ist, mis tegi virtuaalse D draivi ja keeras kogu C ready-only-ks. Ja ma avastasin selle niimoodi, et mul oli mingi enda softi katsetamiseks see asi autoexec.bati või config.sysi panna või sealt midagi välja kommenterida, et minu asi ära mahuks või täpselt ei mäleta mis. Igatahes oli mul vaja sinna sekkuda. Kui ma sekkutud sain, siis ma pärast taastasin endise olukorra alati. 

\textbf{\enquote{Ka tol ajal mingit võrgu häkkimist ei toimunud?}}

Anto Veldre\index[ppl]{Veldre, Anto} rääkis jah, kuidas tema poisid ülikooli adminidel ruutusid käest ära võtsid. Tema jagas oma poistele modemeid ja terminale mis tulid kuskilt humanitaarabina. Meil oli üks modem. Õpetaja arvuti modemit ei puutud ja ühel poisil oli oma modem korra koolis kaasas, mida ta näitas aga vot me ei osanud nendega midagi teha ja LANi meil ei olnud. LAN tekkis meile alles 12. klassi kevadel, kui ma enam väga ei tegelenud sellega. OK, ma häkksin selle Lantasticu lahti social engineeringu meetodil. Sügisel pärast minu ära minekut oli kellelgi vaja saada Lantasticule juurdepääsu ja oli server masinas oli nihuke koht nagu network control directory. Seal olid andmebaasid binaarsena. Ja vot minu programm oskas käia ja binaarselt andmebaasi modifitseerida ja tekitada ühe administraatori juurde või panna kellelegi õigusi juurde või midagi. Ehk siis tuli meelitada noorem arvutiõpetaja flopi pealt ühte programmi käivitama seal masinas, viisakalt tänada ja puha. Tema poolt oli ka kõik OK. 

Aga varem oli see C-ketta kirjutuskaitse. Algul me käisime Nortoni Disc Editoriga kuskil seal config.sys algust ära sodimas, et seda ei loetaks. Aga siis oli paremini kaitstud järgmisel softil ja ei saanud sellega ka ligi ja siis oli vaja ikka flopi pealt bootida. Aga BIOS oli parooli all. A ja C, C ja A. Noh, siis järelikult muugime BIOSi paroolid lahti. See on obfuskeeritud kujul kirjutatud kuhugi CMOS mälusse. Selle sai sealt obfuskeeritud kujul välja lugeda. Ja masina ROM oli välja loetav. Ma võtsin ja disassembleerisin selle sourcereri nimelise disassembleriga ja matemaatika tunni ajal kirjutasin omale matemaatika vihikusse kõrval lehe peale programmi, mis seda obfuskeeritud asja lahti võtab. Järgmine tund oli ajaloo tund. Läksin ajaloo tunnist ära arvutiklassi ja realiseerisin selle ära ja muukisin BIOSi paroolid lahti. Mul tuli suur pahandus, sest see oli ajaloo tund, kust väga paljud olid puudunud ja õpetaja oli väga kuri ja keeras käkki. Mul oli pärast vaja järgi teha ja õnnestus ikkagi. Põhjendasime ikka kui väga hea programmi me tegime spetsifitseerimata, mis see oli. Et väga hea idee oli ja tuli lihtsalt minna arvutiklassi ja kohe ära teha. Parool oli obfuskeeritud striimsšifrina või bait haaval võibolla isegi, et otsast proovides järjest täht haaval sai selle ära arvata. Ma kunagi arvutiõpetajalt küsisin, et miks teil nii imelik parool on. Ja siis ta lahendas selle turvaprobleemi niimoodi, et delegeeris osa vastutust arvutiklassi haldamises ja võttis nigu appi arvutiklassi haldama, natuke. Väga hea pedagoogiline meetod, töötas. Ei häkitud enam, ei olnud enam huvi edasi jagada paroole, mida ma kätte saan. 


\textbf{\enquote{Aga kust sul see krüpto huvi?}}

Seda läks sealsamas kandis ka vaja. Näiteks meie õpetaja ässitas Norton Diskreet'i\sidenote{Diskreet oli tarkvarapaketi Norton Utilities 6.0 osa ning sisaldas paljuski kurikuulsat (Kevin Mitnicku\index[ppl]{Mitnick, Kevin} andmetel kasutati väidetud 56 biti asemel 30 bitist võtit, ka teised uurijad on osundanud mitmetele olulistele nõrkustele) DESi implementatsiooni} DESi\sidenote{\emph{DES - Data Encryption Standard} on sümmeetriline algoritm andmete krüpteerimiseks. Algoritm on oma väikese võtmeruumi tõttu tänapäeval kasutamiseks sobimatu (murti avalikult jaanuaris 1999), kuid oli siiski alates 1977. aastast USA föderaalse andmetöötlusstandardi (FIPS) osa.} kallale. DESist ma ei saanud jagu, ma ei saanud DESist arugi tol hetkel. Aga tema suunas. Ta oli üldse sedasorti kaval mees, et kui ta näiteks kunagi kui meil Veljo Haguga\index[ppl]{Hagu, Veljo} oli plaan kirjutada viirus. Me olime mingeid olemasolevaid viirusi disassembleerinud ja vaadanud, kuidas need käivad. 302 oli kõige lühem vist. Veljo oli mu pinginaaber. Õpetaja sattus pealt kuulma, kui me rääkisime viiruse tegemisest ja ütles, et kui teha, siis teha kohe selline stealth-viirus. Me olime väga nõus aga seda me ei viitsinud teha ja jäi tegemata viirus. 

Ta leidis meile muidu ka rakendust. Keskkoolis üldine taustaülesanne oli midagi arvutada. Minu arvutusülesanne oli arvutada arvu $e$ kahe tuhande komakohaga 30 sekundi 10Mhz 286 peal. Üks klassivend arvutas $\pi$-d tuhande komakohaga 60 sekundiga sest see koondus aeglasemalt. Ja kust tulid ajapiirangud? Õpetaja oli vaadanud, kui kiiresti temal vastus tuleb selle arvuti peal. Ma sain 35 sekundilise programmiga juba viie kätte, sest vastus oli õigem, kui õpetajal. Kuna need erinesid, siis ta võttis targa raamatu ja siis selgus, et minul oli õige. Mul oli selleks hetkeks 21 sekundiline proramm, mis käigu pealt suurendas mingi hetk arvutüübi pikkust. Algul tegi lühema tüübiga ja hiljem pikemaga, et kiiremini saaks. Aga see oli veel bugine. See veel ei töötanud õigesti. Ma kontrollisin oma enda programmi vastu. Ma olin minut aega töötava programmiga algul arvutanud tulemuse välja ja faili kirjutanud. Siis oli mul ka näiteks variant programmist, mis küsis, et kui mitme sekundiga oli vaja arvutada ja siis ütles hard-coded vastuse. Aga see ei sobinud õpetajale. Aga 35 sekundiline juba sobis, kui vastus oli õigem tema oma.  Minu 21-sekundiline ei läinud tööle aga õpetaja seepeale võttis ja kirjutas ise asja haljas assembleris ja sai kolme sekundiga. Muidu me kirjutasime Pascalis. 

Teine asi, mida me tegime, millega oli keskkooli ajal hulga nuputamist, oli interferentsi simuleerimine arvuti ekraanil. Kaks punktlaineallikat ringlainetega, kuidas lained liituvad, et tuleb interferentspilt. Seal ma nägin ka vaeva, arvutasin ruutjuurt assembleris Newtoni meetodil. Ma arvutasin iga ekraani punkti kohta pimesi selle faasi välja nii et ühtegi punkti näha ei olnud aga ma sättisin pixelite väärtused nii et palett oli seatud üleni mustaks. Arvutasin kõik väärutsed ära assembleris optimeeritud arvutusvalemiga ja õpetaja õpetas Newtoni meetodit sinna juurde. Oli abiks. Assembleris sai Newtoni meetodit! Oli väga hariv. 

Ja lõpuks ma siis ketrasin VGA paletti. Tehnilise dokumentatsiooni failid liikusid, seal oli kirjas, kuidas VGA paletti muuta ja ma seadsin siis paletti niimoodi, et need värvid, mis mul on liikusid sujuvalt heleduse järgi. Ja siis tulemus oli see, nagu oleks liikunud lained ekraanil. Ja see oli minu meelest tippsaavutus, see oli väga ilus sujuv liikumine selle kümne megahertsi juures, punkte üle arvutada poleks kuidagi jõudnud. Ja siis näidati mulle ühe teise õpetaja tehtud programmi. Tema ütles, et minu ideest see alguse saigi, et interferentsi simuleerida. Tema tegi Juku peal circle käsuga valgeid rõngaid üksteise ümber viie millimeetrise vahega. Need läksid mida edasi seda aeglasemaks ja minu reaalajalise sujuva pildi vastu ei olnud see midagi ja mul oli tükk tegu, et mitte naerma hakata. Aga kiitsin siis takka. Õpetaja suutis anda sellise ülesande, mille peale mul kulus ikka kaua ja sain palju targemaks. Õpetaja oli Tarmo Ainsaar\index[ppl]{Ainsaar, Tarmo}. Seesama, kes suunas meid viiruse kirjutamiselt ära ja kes lahendas selle BIOSi paroolide haldusteema probleemi meiega nii et probleemi ei tulnud. Väga hea õpetaja. Ta suutis meid suunata tegema õigeid asju nii, et me seejuures õpime ja paha peale ei lähe. 

\textbf{\enquote{Kuidas sa Cyberisse sattusid?}}

Ma töötasin HClubis ja mõtlesin, et mida võiks magistriks teha. Seal tegeldi hajusate andmebaasidega. Me saatsime SQL käsk haaval andmebaaside diffe üle võrgu mitmes suunas. See oli põnev, me saime selle tehniliselt lahendatud. Algul käis see mul üle UUCP, hiljem üle PPP ja POP3 ja SMTP. Mina ehitasin internetti sinna alla, oli ka põnev. Ja neid diffe siis saatsime ja tekkis küsimus andmebaaside konsistentsusest, mis tingimustel jääb ja mis tingimustel ei jää konsistentseks. Et kas me saame mingi eventually consistent mudeli sealt või mitte. Ma mõtlesin, et ma hakkan sel teemal magistrit tegema. Aga siis kõrval oli tulnud ... Ma sattusin HClubi tööle seoses sellega, et ma installisin sinna Linuxi serveri, gateway. Ja selle peal käis veeb ja meil ja kõik. 

Jah. Ma kusjuures mõtlesin, et ma võtan sul nööbist kinni ja küsin, miks sa sealt ära läksid. Kas ikka tasub. Aga ma vist ei sattunud sind tol hetkel tabama ja siis ma ei käinud sinu käest küsimas. Või küsisin kunagi tagantjärgi, mul on nagu mälestus olemas. Jaa, küsisin, sain vastuse ka. 

Ja siis HClubis interneti teemal, mis mind huvitas tol hetkel, ei olnud mul eriti kuhugi areneda. Seal ei olnud kellegi teise käest sedasorti asju õppida. Kui siis, ise õppida ja ehitada.  Asju, mida oleks võinud pisi ISP-na veel ehitada sinna ISDN sissehelistamiskeskusi kui oleks leitud raha ja et see rentaabel on. Nihukesi asju oleks ehk saanud aga siis samal ajal ma käisin mõnes koolis abiks Linuxit installimas ja käisin laenamas RedHati install plaati suvel Elmer Joandi\index[ppl]{Joandi, Elmer} käest Tartu lähedalt maalt. Tal oli see plaadina kohe olemas ja ei pidanud flopidega mässama. Ja Elmer ütles, et muide, Tarvil\index[ppl]{Martens, Tarvi} olla plaan Tartusse meiesuguste jaoks pesa teha. Ja siis mina käisin juunikuus umbes Tallinnas Cyberneticas\index{Cybernetica} Helger Lippmaa \index[ppl]{Lipmaa, Helger} juures, et tuleks magistrit tegema hoopis krüpto teemal. Ma mõtlesin, et näiteks pordiks Open SSLi Windowsile, sest mul oli Windowsi all krüptot vaja olnud aga ei olnud. Sellest konkreetsest ideest arvati kehvasti, et näe keegi vist juba on portinud ka midagi. Aga tule meile niisama progema, mitte-krüptot. Ütles Tarvi kõrvalt. Oleks peaaegu Küberisse tulemata jäänud. Aga Helger kutsus mu ikka turva asju tegema. Ja siis kutsuti mind Küberi väljasõiduistungile ehk \enquote{kvartalnajale}. 1997. aasta sügisel Arula motelli. Seal oli kutsutud kogu tulevane Küberi Tartu Andmeturbe Labor. Ja Viljar Tulit\index[ppl]{Tulit, Viljar} oma habemesse diktsiooniga ütles, et seda sa pead ikka ise suutma ära otsustada millegi järgi, kas sa tahad siia või ei taha, kui ma ütlesin, et segane on veel, kas ma tulen või ei tule. Ja siis räägiti ka tehnilistest teemadest ja mina läksin Küberisse tarkade inimeste juurde. Seal olid Arne Ansper\index[ppl]{Ansper, Arne} ja Viljar Tulit, kes oli kogenud süsadmin (kogenum, kui mina). Kui mina tegin näiteks tükk aega FTP otsingumootorit Nuuskur koos teiste tudengitega, siis Arne oli selle stiilis nädala otsa õhtutega ära teinud või niimoodi. Arnel oli ka Vosa nimeline FTP otsingumootor Eesti FTP serverite kohta. Vosa nagu \enquote{Vanaisa Oli Sulle Archie\sidenote{Archie oli üks esimesi interneti otsingumootoreid, mis võimaldas otsingut üle FTP arhiivide}}. Tal oli ainult veebiliides, meil oli muid liideseid ka. Meil oli telneti liides ja archie prospero protokolli\sidenote{Archie kataloogides navigeerimiseks loodud protokoll, mida võib pidada tänapäevase www protokolli eellaseks. Prosperot kasutades võis terve internet välja näha, nagu üks suur ühine kataloogipuu} liides, millega vana archie klient töötaks ja meililiides ja. Meil oli võimas vinge süsteem tehtud kamba peale. Kõike ei teinud mina, teised tegid ka. Ma olin lihtsalt üks vedajaid lõpuks, kes tegi kõige rohkem tükke. Ja sellega selgus, et Arne on tark. Seal oli veel asju, millega see selgus. Näiteks tal oli Fido ja Interneti vaheline gateway. Ma olin selle kaudo Fido lugejda. Ma pole päris Fidonetti kunagi näinudki. Minu jaoks Fido oli just another NNTP server stiilis keeks.ioc.ee. Sinna tuli kasutajanime ja parooliga läheneda ja sai tavalise newsreaderiga lugeda ja kirjutada. Minu jaoks oli Fido teenus üle interneti, mida vahendas Arne tehtud süsteem. 

\textbf{\enquote{Mis sa praegu teed?}}

Praegu ma olen Küberis turvainsener ja praktikas ka tarneinsener, kes pakendab asju ja ehitab mingeid keskkondi automatiseeritult nende otsa. Õpetan ülikoolis, olen ülikoolis hajussüsteemide külalislektor, õpetan operatsioonisüsteeme baaskursusena, andmeturvet baaskursusena ja magistrandidele õpetan turvalist programmeerimist. Kuidas teha nii, et auke poleks koodis. Mõni ikka kuskilt leidub aga eks seda ole aja jooksul endale piisavalt vastu tulnud. Andmeturbe kursus sai tehtud siis, kui ma olin magistrand Helger Lippmaa juhendamisel. Helger ütles, et kuule, et sa võiks teha sellise andmeturbe kursuse ülikooli. Mõeldud tehtud. Tegingi. Kellegagi eriti nõu ei pidanud. Küberi turvaraamatu võtsin vihjete jaoks aluseks. Infosüsteemide Turve Esimene köide, oli vist esimene valdavalt, võibolla esimene ja teine. 


\textbf{\enquote{See tundub olevat nii sinu moodi, et võtad, teed ja saab väga hea}}

Parim kiitus, mis ma aineturbe ainele kuulnud olen oli kunagi kui hakati küberkaitse magistrikava tegema. Oli Tallinnas sel teemal siis koosolek. Ja oli häda, et kui me tahame neile õpetada seda, seda, ja kõiki asju, et see ei mahu meil ainetesse ära. Ja selle peale oli vist Enn Tõugu\index[ppl]{Tõugu, Enn}, kes ütles, et kuida, et Meelis jõuab andmeturbe kursuses neist kõigist asjadest rääkida, et mahutame ikka magistrikursusesse ka ära. Mis sest, et põhjalikumalt aga küll me mahutame. Et see oli hea kompliment kursusele, et Meelis räägib neist kõigist. 




\chapter{Henn Ruukel}
\index[ppl]{Ruukel, Henn}


\question{Kuidas arvutid 
sinu juurde jõudsid? }

Kooli arvutiklassi kaudu.

\question{Mis kool see oli?}

Tallinna 10. Keskkool\index{Tallinna 10. keskkool}, tänane Nõmme 
Gümnaasium\index{Nõmme Gümnaasium|see{Tallinna 10. keskkool}}. Kui ma 
õigesti mäletan, siis pärast üht suvevaheaega tulime kooli ja 
mataklassi oli ehitatud arvutiklass. Laudadel sai plaadi üles tõsta ja seest tulid 
välja pööratava metallkonstruktsiooniga 
Elektronika\index{Elektronika} arvutid. Need olid omavahel võrgus 
niimoodi, et õpetaja laua peal oli flopidraiviga 
Iskra\index{Iskra}. 

Elektronikates jooksis ainult BASIC. 
Lükkasid käima, BASIC\index{BASIC} hakkas kohe jooksma, said koodi 
kirjutada ja olid BASICu käsud, millega sai oma programmi Iskra flopikettale salvestada. Viisime oma flopiketta õpetaja kätte, ta pani selle masinasse, tegi ilmselt midagi Iskras ka ja 
siis saigi salvestada oma proge maha või pärast järgmine päev laadida. Ja nii 
kui see juhtus, siis me hakkasime muidugi kohe mataõpetajat manguma. Õpetaja 
Ramil Izamentinov\index[ppl]{Izamentinov, Ramil} oli äge 
mees -- ma \emph{never} oma elus pole näinud teist sellist inimest (ta oli astmahaige), 
kes nii ägedalt naeraks oma naljade üle. Kui 
ta tegi nalja, siis spetsiifilise kähinaga, et kõik saaksid aru, et see oli nali. 

Ta lasi meil pärast tunde seal klassis olla ja me 
hakkasime kirjutama suvalisi progesid. 

\question{Mitmendas klassis see oli?}

See pidi olema enne keskkooli, võibolla seitsmendas 
klassis ehk 1986. aastal. Käisin kooli arvutiklassis 
umbes ühe õppeaasta. Põhiliselt sai kirjutatud progesid, 
mänge ei olnud seal üldse, nii et kõik mängud, mida 
tahtsime mängida, tuli ise BASICus teha, kirjutada flopile ja sealt pärast laadida. Tegime
lihtsaid mänge, näiteks trips-traps-trulli. Elektronikal oli graafiline liides, millega
sai ekraanile joonistada. Näiteks üks programm oli 
selline, et andsid ruutvõrrandi parameetrid, programm arvutas lahendid välja ja joonistas 
graafiku. 

\question{Miks te pidite manguma, et arvutitele ligi saada?}

See tundus kohe äge asi. Küsisime, kas 
võib käia ja mis õhtutel. Algul oli vist õpetaja ise kohal, 
aga lõpuks oli meil mataklassi võti. Paar huvilist 
oli veel minu klassist ja õppealajuhatajaga tuli teha eraldi \emph{arrangement}, et saada algklasside pikapäevarühma söögi 
peale. Siis me ei pidanud koolist ära minema, kui tunnid lõppesid, vaid saime koos
väikeste lastega sööklas süüa ja olla õhtuni arvutiklassis. 

Arvuti oli ikka nii põnev asi, siis oli ju hoopis teine aeg -- kodus polnud 
videomakkigi, arvutist rääkimata. Või et kellegi töö juures oleks arvuti -- 
sellist asja polnud. 

Seejärel kuulsin mingil peresünnipäeval
ühelt kaugelt sugulaselt, et tema käib TPIs 
arvutiringis\index{TPI arvutiring}. See oli see kuulus 
arvutiring, mida pidasid Julius\index[ppl]{Raimla, Tõnu}\index[ppl]{Julius|see{Railma, Tõnu}}\sidenote{Henn peab ilmselt silmas Tõnu Raimlat toonase hüüdnimega Julius.} ja Aare Tali\index[ppl]{Tali, Aare}. 

See oli selline \emph{need to know 
basis} arvutiring -- pidid teadma, et lähed teisipäeval raadiotehnika 
kateedri\index{Tallinna Tehnikaülikool!Raadiotehnika kateeder} taha otsa ja 
ootad ühe ukse taga. Seal oli jõnglasi nagu murdu ja 
kindlal kellaajal tegi Julius ukse lahti. Klassis olid kolmes reas
Yamahad\index{Yamaha MSX}, MicroBeed\index{MicroBee} ja 
Robotron 1715d\index{Robotron!Robotron 1715}. Yamahade peale käis 
tegelikult tormijooks, kõik tahtsid sinna saada, sest seal olid 
kihvtid mängud. Robotronid olid ka põnevad, samas kui MicroBeed ei huvitanud eriti kedagi. Mina olin põhiliselt Robotronide peal, kus sai 
progeda ja koolireferaate teha. Yamahadesse ei saanud ma
kunagi löögile. 

Yamahad olid 3.5tolliste diskettidega ehk juba järgmine tase. See pidi olema 1988. aasta.

\question{Kas oligi nii, et uks tehti lahti ja kes istuma sai, see sai?} 

Jaa. Kui midagi ära lõhkusid, siis oli selleks päevaks \emph{ban}, aga järgmisel 
päeval võisid tagasi tulla. Mäletan seda sellepärast, et ükskord istusin 
ühe Robotroni taha, kus oli probleem klaviatuuriga, ja ma võtsin teise 
masina klaveri. Sellel oli imelik pistik ja panin selle kuidagi nii kehvasti sisse, et rikkusin pistiku kontaktid kõveraks. Siis tuli minna tagaruumi 
ja öelda, et näed, niisugune jama. Ma ei mäleta, kumb neist tuli, kas Julius 
või Tali, vaatas ja ütles: \enquote{Okei, tänaseks kõik, tule järgmine 
teisipäev tagasi.} 

Tegelikult olid nad kateedris tööl ja ilmselt progesid tagaruumis. Meie 
olime kõik alles 13aastased jõnglased. 
See pidi olema 1988. aasta, sest 1989. aastal läks mu ema tööle 
Diagnostikakeskusse\index{Diagnostikakeskus} ja 
ma liikusin sinna edasi. 

Ürituse lõpp sõltus sellest, millal nemad ei viitsinud enam olla ja tahtsid 
koju minna. Ametlik lõpp oli ka, aga sellest ei pidanud keegi 
kinni. Jõnglased said teisipäeva õhtul umbes kell kuus uksest sisse ja kõigil oli jube 
põnev: kes proges, kes mängis mänge ja kes häkkis mänge, et teha 
sinna oma tegelased sisse. Arvutiring lihtsalt oli, meid ei pandud 
kunagi kuskil kirja ja keegi ei õpetanud midagi, ise tegime. Ühel hetkel 
viskas kuttidel üle, nad tegid tagaruumi ukse lahti ja ütlesid: 
\enquote{Viie mintsa pärast toitekas.} See tähendas, et meil oli viis minutit aega mis 
iganes pooleli oli ära salvestada, sest viie minuti pärast tuli emb-kumb neist, ei hakanud jõnglastega midagi vaidlema, jalutas kilbi juurde ja 
tõmbas laksti peakilbi välja. Klass läks pimedaks ja see oli signaal, et nüüd tuleb koju 
minna.

\question{Võru 1. Keskkoolis oli sama skeem ja kedagi ei huvitanud, mis 
arvutiga juhutub. Minu meelest ei juhtunudki suurt midagi.}

Täiesti \emph{ruthless}. Väga ei juhtunud jah, aga eks ilmselt oli oht, et kui 
kirjutamine jäi pooleli, siis võis kettaga midagi juhtuda. 

See oli tegelikult äge periood. Mäletan uduselt paari tüüpi, sest
koostööd või suhtlust oli vähe. Yamahade juurest oli justkui mingi 
generatsioon, kes tundsid üksteist varasemast ja hoidsid 
kokku, aga ülejäänud nokitsesid igaüks omasoodu. 

See kõik oli ikkagi alles algus. Niisugune \emph{breakthrough}, kus 
mina jõudsin paradiisi ja õndsusesse, toimus tänu Nõukogude Liidu kompartei esimehele Mihhail Gorbatšovile, kes otsustas 
perestroika käigus aastal 1986, et meditsiiniga on Nõukogude 
Liidus suur probleem. Ja kõige suurem probleem 
on see, et pannakse valesid diagnoose või diagnoosi jaoks vajalik info on 
ebatäpne, vale või liiga aeglane. Seega tuleb investeerida infrastruktuuri, 
millega meditsiinipersonal saaks kiiremini teha ära olulised mõõtmised ja 
analüüsid, näiteks südame EKG, vereanalüüsi, 
magnetresonantstomograafia. Asjad, mis on täna põhimõtteliselt 
külahaiglaski olemas. Siis selliseid asju polnud ja vereanalüüsi tegemine võttis mega 
kaua aega ning tulemused tulid tagasi ebatäpsed. 

Gorbatšov otsustas, et kuna tal ei ole piisavalt valuutafondi\sidenote{Nõukogude 
elu valitsesid laias laastus kaks terminit: fond ja limiit. Fond ütles, kui 
palju midagi kellegi jaoks olemas oli, ja limiit seda, kui palju midagi tohtis
toota või tarbida. Näiteks \enquote{pole piisavalt valuutafondi} tähendas, et 
ei olnud eraldatud piisavalt valuutat.}, et teha neid 
võimalusi igasse jumala haiglasse, siis igasse Nõukogude Liidu osariiki, 
näiteks Eestisse, pidi pealinna moodustatama selline asi nagu 
diagnostikakeskus, kuhu investeeritakse valuutafondi, ostetakse välismaalt 
\emph{top notch} tehnika sisse ja terve osariigi analüüsid või uuringud 
tehakse ühes kohas. Näiteks kui sul on kopsuhaigus, siis sind ravib küll
Mustamäe haigla arst, aga Mustamäe haiglasse ei ole ressurssi 
osta kopsuanalüsaatoreid, vaid need ostetakse ühte kohta 
eraldi asutusse.

Eestisse pidi ka tulema selline keskus. Tol ajal kehtis Nõukogude Liidule välismajandusembargo ehk 
liitu ei saanud sisse tuua lääneriikide moodsat tehnoloogiat, välja arvatud
meditsiinitehnoloogiat, see oli okei. PCsid ei olnud meil ka selle pärast, et embargo oli peal. Isegi kui välkarid oleksid tahtnud müüa ja
kellelgi oleks siin valuutat olnud, ei saanud selliseid asju osta. 

Hakatigi siis Tallinnasse moodustama 
Diagnostikakeskust\index{Diagnostikakeskus}. Seda vedas Ago Kivilo\index[ppl]{Kivilo, Ago}, kes on tänaseks siit ilmast kadunud. Keskus pidi tulema 
sinna, kus praegu on vana Tallinna Panga maja Tallinna linnaosavalitsuse 
kõrval. Muidugi läks eht eestlaslikult kemplemiseks selle ümber, kes 
saab keskuse juhiks, millise haigla kõrvale see üldse teha, asukoht ei ole 
ikka hea, haiglatest kaugel ja äkki teha Mustamäe haigla kõrvale. 
Kivilo ütles, et teeb selle ise, eraldi asutusena. Mustamäe haigla 
ilmselt tahtis seda enda allasutuseks. Ma detaile ei tea, 
olin umbes 14, aga natuke tean sellepärast, et 
aitasin Agu Kivilol teha kõiki PowerPointe, mida tal oli vaja. Tol ajal oli
see \emph{skill}. 

\question{Kas PowerPoint oli tollal olemas?} 

Eesti riigis polnud selliseid 
asju nagu laserprinterid, PCd, värvilised kuvarid ja arvutivõrgud siis veel olemas. Okei, arvutivõrke oli, aga need olid pigem 
CP/Mide arvutivõrgud tehnikaülikoolis. Aga kuna 
Diagnostikakeskus oli meditsiiniasutus ja nad ostsid Soomest 
meditsiinitehnikat, siis oli Soomes üks vahendusfirma
Pekka OY (see oli reaalselt omaniku eesnimi), kes vahendas 
Soomest kopsu- ja vereanalüsaatoreid ning muud kama. Küllap juriidiline skeem pidi ka suhteliselt keeruline olema, kuna see kõik liikus 
Moskva valuutafondide kaudu. Kuna see kaup liikus, 
siis selle raames \emph{top notch} PCsid Eestisse tuua polnud mingi probleem, 
sest paberil paistis see kõik nagu meditsiinitehnika. Tänu sellele oligi meil PowerPoint olemas.

\question{See oli vist võrreldes muu tehnikaga ka odav?}

Esimene magnetresonantstomograaf tuli sealtkaudu. Omaette lugu oli see, et Gorbatšov eraldas valuuta, aga seda ei suudetud ära kulutada, sest lõputu aeg 
kulus vaidlusele, kuhu keskus ehitada. Tomograafi jaoks pidi maja vundament olema spetsiifiline, et ei tekiks
värinaid ja vibratsioone. Aeg muudkui kulus ja Nõukogude Liit hakkas juba lagunema, 
aga valuuta oli ikka veel kontol. Tekkis reaalne probleem, kuidas see 
kuhugi ära kulutada, enne kui kaduma läheb. 
Kivilo\index[ppl]{Kivilo, Ago} tegeles sellega. Kokkuvõttes ehitati keskus Magdaleena 
haigla kõrvale. Kivilo sai tomograafi niimoodi kätte, et 
tarnijafirma või mõni vahefirma (detaile ma ei tea) sai 
ettemaksu ja aastaid hiljem tarnis seadme kohale. 

Kõige selle käigus tuli ka arvuteid. Mina 
sattusin sinna niimoodi, et ema läks vereanalüsaatori peale tööle (ta on apteeker) ja küsis Kivilolt, kas tema poeg võib seal pärast kooli
arvutis käia. Kivilo lubaski. Seal sain 
kokku Mart Palmasega\index[ppl]{Palmas, Mart}, kes 
põhimõtteliselt tõigi mind ja Madis Kaalu\index[ppl]{Kaal, 
Madis} Eesti ITsse. Madis Kaalu püüdis ta kuskilt TPI pealt kinni (ta oli sealt 
välja kukkumas), ja ütles, et kuule, tule nüüd, ma panen su arvutite 
peale. Nii et Palmas õpetas mind progema, enne olin omal käel 
harjutanud. 

\question{Mida Mart Diagnostikakeskuses tegi?}

Diagnostikakeskuse arvutite hooldamiseks oli loodud väikeettevõte 
Skriining\index{Skriining}, mille asutas Kalle 
Lotamõis\index[ppl]{Lotamõis, Kalle} ja mis on siiamaani olemas. Mart töötas seal vist programmeerijana.
Skriining hoidis Diagnostika{\-}keskuse arvuteid korras ja mina sain seal ettevõttes
hängida, ma ei olnud nendega kuidagi juriidiliselt seotud. 
Välja arvatud minu üldse esimene töö, kui TPLsid\sidenote{Töö- ja 
puhkelaager -- Nõukogude koolilastele pakutud võimalus suviti organiseeritult 
tööd teha ja elu nautida.} ja rohimist mitte arvestada. Vedasin neile maja peal laiali koaksiaalvõrgu, mille otsa panime käima 
Novell \mbox{Networki}. Arvutitesse tuli paigaldada võrgukaardid, neid häälestada, panna 
õiged IRQd ja ka soft peale. See oli mu 
esimene suvetöö.

\question{Kes sulle sellise ülesande andis ja miks tal oli alust arvata, et sa oskad
seda teha?}

Esiteks olin seal juba kevad läbi hänginud. See oli 
Kalle\index[ppl]{Lotamõis, Kalle}, Skriiningu juhi otsus. 
Küsisin, kas neil oleks mingit suvetööd. Kalle ütles: \enquote{No tule ja pane 
arvuteid kokku -- võta kastist välja, pane üles ja kui kellelgi on 
mõni mure, siis aita.} Me olime Suur-Ameerika 18 majas. 

Mainisin, et meil olid PowerPointid ja laserprinterid. Samal ajal oli terves TPis heal 
juhul kümme 8086 ehk XTd ja ilmselt mustvalge 
\emph{display}'ga. Meil oli terve ruum triiki täis 
avamata arvuteid, mis olid kõik 80286d, kõigil 40 mega IDE vinti ja VGA 
graafika. 

Kui näiteks Mamers\index[ppl]{Mamers, Tarmo} meile tööle tuli, 
siis ta pakkis endale kohe kaks tükki lahti! Ühe peal jooksis BBS, teise 
peal tegi tööd. Kui panna 
tagantjärele konteksti, mida tänapäeval tööks nimetatakse, siis seal tööd 
tegelikult ei olnud. Meditsiinipersonali arvuteid oli 
umbes kümme ja neid hooldas kümme inimest, kes põhiliselt 
lihtsalt kaifisid seda, mis tehnika keskele nad sattusid. Tegime 
muidugi kõik asjad ära, mis teha oli vaja. 

Mina sattusin sinna minnes niivõrd viljastavasse keskkonda. Seal 
olid Mamers\index[ppl]{Mamers, Tarmo} ja Palmas\index[ppl]{Palmas, Mart}, 
Hannu Krosing\index[ppl]{Krosing, Hannu} astus kord nädalas läbi. See oli ka 
koht, kus sain aru, et minust ei saa kunagi progejat. Mäletan nii 
hästi, kuidas ma nädal aega pusisin millegi kallal, ja siis tuli Hannu, näitasin 
talle selle nädalaga kirjutatud koodi ning ta võttis paberilehe ja 
kirjutas kahe reaga sellesama koodi Turbo Cs\index{Turbo C}.

\question{Leidsid ka, kellega ennast võrrelda!}

Ma sain aru, milline on delta. Võibolla see ei olnud ainus põhjus, aga 
ma mõistsin, et talendi ja \emph{skill}'i vahe on ikka 
hüpersuur.

\question{Hannuga võrreldes on kelle iganes talendivahe väga suur!} 

Tõsi-tõsi.

Põhiline projekt, mille kallal ma tol ajal töötasin ja millel oli ka üks 
\emph{user}, oli clabel. Ostsin sellele
ametlikult 25 krooniga Madis Kaalult\index[ppl]{Kaal, Madis} graafilise 
liidese \emph{library}, millega sai menüüd, \emph{pop-up}'e ja muud sellist
teha. Nii et juba toona 
maksin selle eest, et mul oleks ametlik \emph{lifetime}-litsents. 

Niisiis, kirjutasin programmi clabel, mis tegi sellist 
asja, et sul oli muusika (meil olid tol ajal kõigil kopeeritud muusika 
kassetid\sidenote{Vt ka märkus lk \pageref{sisu!kassetid}.}), kassette oli palju ja oli vaja normaalset andmebaasi,
mis lood ja bändid millise kasseti peal on. Lisaks oli vaja need 
välja trükkida, et ilusti ümber kasseti voltida 
ja panna karbikaane alla. Programm trükis sellise paberi välja, et üleval serval 
oli näha, mis on A- ja mis B-poolel, ning suurel küljel kõik
A- ja B-poole laulud. Programmil oli graafiline liides, kus 
sai brausida, edida ja printida. 

\question{Ma teenisin oma esimese arvutiga teenitud raha täpselt samasuguse softi 
abil.}

Mul oli üks väga kasulik \emph{user}, Toivo Annus\index[ppl]{Annus, 
Toivo}, kes reaalselt kasutas seda ja tegi mulle 
kogu aeg bugireporte.

\question{Kuidas sa Toivoga kokku said?}

Fido kaudu. Ilmselt \emph{upload}'isin selle softi kuskile BBSi ja promosin Fidos, et mul on niisugune asi. Toivo hakkas kasutama ja mul külas 
käima ning rääkima, mis ei tööta ja mis töötab. Tema oli siis vist 16, mina 
umbes 14. 

\question{Järelikult sukeldusid sa juba Diagnostikakeskuses Fido maailma?}

Ma ei mäleta, millal Fido tekkis, või kumb oli enne, kas minu Diagnostikakeskusse minek või Fido tulek. Ilmselt 
läksin kõigepealt keskusesse ja siis millalgi tekkis Fido. Aga ühel hetkel olid Fido ja BBSid meil seal 
nagu \emph{bread and butter}. Mardil\index[ppl]{Palmas, Mart} oli oma 
\emph{node}, Mamersil\index[ppl]{Mamers, Tarmo} muidugi oma Mambox\index{MamBox}. 
Siis tuli BBSummer\index{BBSummer}, esimene toimus vist esimesel või 
järgmisel suvel. 

Lisaks kõigele muule oli Diagnostikakeskusel humanitaarabina Rootsist saadud täiesti töökorras Volvo 
põhjale ehitatud kiirabiauto. Kuna keskusel polnud sellega 
midagi teha, sõitis sellega ringi Mart. Autol olid kõik 
operatiivauto load olemas, vilkurid peal, täismäng. Taga oli kanderaam, 
pane või inimene sisse. Käisime sellega koos
Mamersi\index[ppl]{Mamers, Tarmo} ja vist Kaido Kärneriga\index[ppl]{Kärner, 
Kaido} Saku Õlletehasest BBSummeri jaoks õlut toomas. Sellega oli hea vedada. Esimene summer toimus Väänas Tugamanni veskis, mina läksin 
mopeediga kohale. See oli juba ülemineku ajal, 
kui poes polnud midagi saada. Kellegi tutvuste kaudu saadi kuidagi
Sakuga kokkuleppele sealt otse õlut osta. 

\question{Kas see oli seesama kord, kui, nagu Mast\index[ppl]{Kaal, Madis} rääkis, 
õlut sai villitud Fanta tünnidesse ja õllel oli apelsinimekk man?}

Täitsa võimalik. Mina olin selles vanuses, et õlut ei joonud, ja ei oska kommenteerida. Aga mäletan, et üritus oli kihvt.
Esimesel BBSummeril kuulsin esimest korda ka
asjast nimega internet ja asjast nimega e-maili aadress. Tartu füüsika 
instituudis\index{Tartu Ülikool!Füüsika Instituut} oli mingi kamp, lausa 
perekond itikaid, kelle nime ma ei mäleta.\sidenote{Tarmo Mamers\index[ppl]{Mamers, Tarmo} arvab, et tõenäoliselt oli tegu perekond Pruulmannidega ehk muudestki juttudest läbi käiva Jaan Pruulmanni\index[ppl]{Pruulmann, Jaan} ja tema abikaasaga.} Igatahes keegi 
nendest pidas ettekande ja oli tolleks ajaks juba käima pannud interneti ja 
FidoNeti vahelise \emph{gateway} meili jaoks. Nii et põhimõtteliselt kõigil 
Fido inimestel, ja kohe ka mul, sel ajal 
meiliaadress. Ma ei mäleta, mis tagumine ots oli, st domeen, aga 
ülejäänud ots moodustus sellest, mis \emph{node}'i küljes sa olid ja kes sa olid. Nii et teoreetiliselt sai mulle saata meili ja mina sain saata välja ka, 
aga ma ei mäleta, et oleksin seda praktikas kasutanud. Mul oli 
too meiliaadress isegi kuskile märkmikusse üles kirjutatud, aga sellega 
polnud kellelegi saata. 

\question{Esimese faksiomaniku probleem\ldots{ }Järelikult oli BBSummeril ka
hariduslik ja sisuline sisu?} 

Jah, loenguid ja \emph{knowledge-sharing}'ut oli väga palju. Minu, 
14aastase arust oli see väga äge. Mäletan Martiini\index[ppl]{Martiini|see{Rinne, 
Martin}} ehk Martin Rinne\index[ppl]{Rinne, Martin} demost, mida sai teha ja 
mida nad tegid Amigaga. Ta tegi Eesti Tele{\-}visioonis\index{Eesti Rahvusringhääling!Eesti Televisioon} saatetiitreid ja muid asju ning demos seda poolt. Minu jaoks oli see täiesti uus maailm, ma polnud 
seda osa üldse näinud. Mäletan internetiteemalist loengut, aga 
eks seal oli ka väga palju vaba suhtlemist ja jutuajamist. Nägin seal
esimest korda paljusid inimesi, kellega olin juba umbes aasta suhelnud. 

\question{Kas sõnumitega Fidos?}

Jah, olid jututoad teemade kaupa, põhimõtteliselt nagu tänapäeval foorumid. 
See ei olnud kaugeltki \emph{real-time} -- värsked sõnumid tuli alla tõmmata, läbi lugeda, vastused valmis kirjutada ja siis sisse helistada ja valmiskirjutatud asjad \emph{upload}'ida. 

\question{Järelikult pidi sul olema mingi kliendisoft?}

Jah, sellega oli lihtne. Kõige keerulisem oli see, et pidid olema modem ja 
telefoniliin. Neid oli tol ajal raske hankida. 
Niipea kui saime Novelli võrgu püsti, siis meil maja sees liikusid 
sõnumid sealtkaudu. Selleks et mina lugeda ja kirjutada saaksin, ei pidanud ma 
oma masinas modemit omama, sõnumid läksid otse MamBoxi. Täpseid vaheetappe ma ei mäleta, pigem seda, et kui liikusin 
sealt Salva Kindlustusse, kus me Toivoga moodustasime Salva Kindlustuse IT-osakonna, 
siis oli meil lihtsalt ühes masinas \emph{point}. 

\question{Miks te seda kõike tegite? Diagnostikakeskuse jaoks ei olnud seda ju 
vaja. Kas teil oli aega ja tahtmist mängida või andis mõni visiooniga 
inimene teile ülesande võrk ehitada?}

Novelli võrk oli väga praktiline asi, tänapäeval on sama praktiline 
teha igasse kontorisse internetiühendus. See andis ettevõttele väärtuse 
mõttes kaks asja: võrguketta (mina salvestan maha ja sina saad teisest 
arvutist kohe kätte) ja võrguprinteri. See oli ju \emph{pre-Windows} aeg: 
said omale võrguketta külge, said faile jagada ja programmidele ligi ning 
printida. Laserprinterid olid kallid ja see andis hästi suure efekti, kui
said maja peale ühe laseri osta ja ükskõik mis arvutist sinna trükkida -- see oli 
tol ajal \emph{magic}. Kõiki asju tuleb ju konteksti panna ja see oli kindlasti tolle aja kohta eriline. 

\question{Kas Diagnostikakeskuses olid kuni keskkooli lõpuni?}

Diagnostikakeskus\index{Diagnostikakeskus} oli 
ikkagi meditsiiniettevõte, Skriining\index{Skriining} oli selle küljes 
tütarfirma. Algul pidigi väikeettevõtted moodustama mõne
riikliku ettevõtte juurde. Ma ei tea, mis aastal see täpselt oli, aga 
millalgi sai Skriining ennast Diagnostikakeskuse küljest lahti aktsiaseltsiks. 

Kõigepealt hakkasin Skriiningus suvetööl käima, tegin võrguhaldust. Üsna pea 
õnnestus mul ennast sebida kooli kõrvalt palgale, nii et kogu keskkooliaja 
käisin sellisel režiimil, et pärast kooli läksin kohe linna 
Skriiningusse. Siis oli meil juba rohkem kliente: mitte ainult 
Diagnostikakeskus, vaid erinevaid meditsiiniettevõtteid. Põhiäri oli arvutivõrk, kas mõnes haiglas või mujal. Näiteks vedasin
Haigekassasse\index{Haigekassa} arvutivõrgu. See asus praeguse Prantsuse Lütseumi 
algkooli hoones.\sidenote{Hariduse 8, Tallinn.} 

Mäletan seda sellepärast hästi, et 
esiteks ei laseks tänapäeval keegi kaablit lihtsalt pinna 
peale vedada. Tõmbasin kaablit ja lõin klambreid seina peale, mis praegu
tunduks robustne. Teiseks, töövahenditeks olid haamer, klambrid, 
kaabel ja midagi, mis meenutab kaugelt vaadates trelli. Tänapäeval ei kavatseks ilmselt keegi sellega ühtegi auku puurida. Mina pidin sellega kõik augud 
tegema. Trelli otsa käis üks asi, mis meenutas puuri, ja sellega tuli 
suvalisest materjalist nühkimismeetodil läbi minna.

\question{Kas puuri diameeter oli suurem kui kaabli diameeter?}

Jah, vähemalt see oli hea. Põhimõtteliselt ainsad lootustandvad kohad, kust 
õnnestus läbi minna, kuna puuri pikkus ei olnud väga pikk, olid 
uksepiidad. Koaksiaalkaabel peab ju läbi kogu maja 
moodustama pideva \emph{loop}'i. See saab kuskil alata ja 
kuskil lõppeda, aga ei tohi katkeda ja seda ei saa olla mitu. Nii et
pidin mõtlema, et arvutivõrku on vaja kõigisse tubadesse ja
kuidas see ahel teha. Pidin terve toa läbi jalutama ja uuesti kaabliga välja 
minema. Ja kui see kuskil katki läks, oli kogu võrk maas, 
sest see ei olnud selline nagu \emph{twisted pair}'i võrk, kus ruuterist 
või \emph{switch}'ist läheb kaabel seadmeni ja kogu moos. 

Ühesõnaga, Skriiningus käisin ma palgatööl. Ühel hetkel kolisime Suur-Ameerikast 
ära, Skriining sai oma ruumid ja olime pikalt, vist 
kogu mu keskkooliaja, Estonia puiesteel, kus 
praegu on Mati Mobiiliäri.\sidenote{Estonia puiestee 5, Tallinn.} See oli äge aeg. 

Aaslaid\sidenote{Andrus Aaslaid\index[ppl]{Aaslaid, Andrus}.} elas kontoris. Tal oli kuskil korter ka, aga 
ta ei viitsinud seal käia mis iganes põhjusel. Mina tiksusin ka pärast kooli viimase 
bussini seal kas oma programme kirjutades või tööd 
tehes: vedasin võrku laiali, 
hooldasin võrke, paigaldasin servereid. Need muidugi käisid vahepeal maha ja 
tuli joosta kuskile teise linna otsa, et asjad uuesti käima ajada. 

Kõva \emph{bread and butter} Skriiningus oli see, et arvuteid pandi 
komponentidest kokku. Klient ütles, et andke mulle üks arvuti. Lepiti 
mingis enam-vähem konfis kokku, aga ma ei ole kindel, kas
tellija teadis, mis need parameetrid on. Linna pealt otsiti komponendid 
partnerarvutifirmadelt kokku: ühelt mäluplaat, teiselt kõvaketas, 
kolmandalt korpus. Mina keerasin selle kõik kokku, panin ööseks 
testid peale jooksma ja hommikul, kui kõik toimis, läks arvuti karpi, Skriiningu 
kleepekas peale ja kliendi juurde. Siis panin arvuti üles, näitasin inimesele (kelle 
jaoks oli see enamasti elu esimene arvuti), kust sisse 
lülitatakse ja et turbonuppu\sidenote{Vanematel PC tüüpi arvutitel oli küljes nupp 
sildiga \enquote{turbo}. Vastupidiselt ootusele sellele vajutamine vähendas, 
mitte ei suurendanud arvuti töökiirust. Nimelt sõltusid 8088 
protsessoriga arvuti jaoks loodud tarkvara (eriti mängud) mõnikord 
masina 4,77 Mhz taktsagedusest ja seetõttu uuematel, kiirematel 
masinatel korralikult ei käinud. Tagurpidi ühilduvuse jaoks lisati 
riistvaraline võimalus arvutit aeglasemaks teha.} ära vajuta, las see olla kogu aeg sees. Näitasin ka, mida arvutiga teha saab ja kuidas 
võrku logida (põhiliselt oli Novelli võrk). Enamasti oli 
kasutajal vaja saada ligi mõnele raamatupidamisprogrammile, mida ka 
Skriining ise kirjutas, või siis oli tegu Skriiningu enda \emph{custom} 
haiglate infosüsteemidega, näiteks kaardiregistritega.

Sedasi läks mu keskkooliaeg. 

\question{Kas see õppimist ei hakanud segama?} 

Ei hakanud. Pigem ei seganud õppimine tööd. Ega ma viieline ei 
olnud, tol ajal oli mu elu ikkagi väga IT poole kaldu. 
Kool ei huvitanud mind absoluutselt. Mitte et ma poleks
aru saanud, aga ma tegin \emph{bare minimum}'i, et saaks kähku arvuti taha. Tundus, et 
kõik põnev asi toimub seal. 

Ja mitte ainult arvuti taha, vaid, kuna me olime nii tsentraalses asukohas, 
Skriiningu kontor oli nagu läbikäiguhoov. Kogu aeg astus keegi uksest 
sisse, oli see siis Tanel Raja\index[ppl]{Raja, Tanel}, Hannu 
Krosing\index[ppl]{Krosing, Hannu} või keegi teine. Ajas juttu ja sai jälle 
midagi teada, mida tema oli kuskil näinud või kuulnud. Mast\index[ppl]{Kaal, 
Madis} oli Forekspangas kohe üle hoovi, tema käis külas. Nii et elu oli väga 
sotsiaalne. Tänapäeval on kõik \emph{online}'is, tol 
ajal oli suhtlus ka IT-meeste vahel üsna \emph{offline}. 

\question{Aga BBSid?}

BBSid olid olemas, aga minu arust sel ajal hakkas Fido vunk juba hääbuma. 
Sellel oli ilmselt mitu põhjust. Üks oli see, et kapitalism jõudis kohale, 
tööd oli vaja teha. Ei saanud päevad läbi lihtsalt istuda ja häkkida 
arvuteid, et kui palju ma suudan mälu siin efektiivsemaks panna 
või kui ilusti ma oskan oma faile pakkida kettale niimoodi, et neile
võimalikult kiiresti ligi pääseda. Selliste asjadega ei olnud ühel 
hetkel enam aega tegeleda, vaid tuli teha reaalset tööd, kas siis tellimustöid
progeda või arvuteid kokku panna. 
Elu läks tõsisemaks ja suhtlemist vähemaks. 

\question{Kas füüsiliselt ikka üksteisel külas käisite?}

Käisime, aga ka see hakkas ühel hetkel hääbuma. Skriiningu\index{Skriining} ajal see veel oli nii, aga kui me 
Toivoga\index[ppl]{Annus, Toivo} Salva Kindlustuse ITd tegime, siis ma ei 
mäleta, et seal oleks hängimist või kohvitamist olnud või et keegi 
oleks külla tulnud ja oleksime pikalt pläkutanud. 

\question{Kui minna korraks BBSide ja Fido juurde tagasi, siis kas sa oskad tuua mõnda näidet, mis kohtades ja mis teemal tubades sa juttu rääkisid?}

Minu mäletamist mööda olid olemas näiteks sellised ruumid 
nagu EW.NALJAD, kus keegi jagas anekdoote, ja EW.JUTUTUBA, kus olid üldised teemad. Väga kõva 
diskussioon käis igasugustes tehnilistes 
\emph{channel}'ites. Ma kahjuks enam ei mäleta, mis teemade kaupa need olid. 
Kusjuures seal ei olnud ainult Eesti kanalid, vaid sai 
\emph{subscribe}'ida ka globaalsetesse kanalitesse. Näiteks mingil hetkel lugesin globaalseid Novelli halduse ja adminnimise kanaleid, mis olid tänapäeva foorumi laadsed asjad. Need olid 
teemapõhised. 

Päris algul oli Eesti Fido ringkond 40--50 inimest, täitsa 
\emph{manageable size}. Ühel hetkel läks see läbuks kätte ära, kui 
maht kasvas ja lisandus väga erineva taustaga inimesi. Näiteks Salvas, kui meil 
oli \emph{point} niikuinii püsti ja Novellist ligipääsetav, istusid seal sees sekretärid ja kes iganes, kellel oli aega. Varieeruvus läks väga 
suureks, inimeste taust ja huvid väga erinevaks. 

\question{Enam ei olnud nii elitaarne klubi?}

Fido ei olnud minu arust kuidagi suletud või elitaarne klubi, ma ei ole 
kunagi tajunud, et see oli salajane või 
erilise müstilise ligipääsuga, vaid pigem oli algul lihtsalt inimesi vähe. 

Ühesõnaga, Fido jäi minu elus kõrvale ja ka üritused hakkasid hääbuma.

\question{Kas tolleks hetkeks olid sa juba 
leidnud, et kuna maailmas on olemas Hannu Krosing, siis sina enam 
programmeerida ei taha?}

Jah, läksin tööalaselt 
pigem mujale. Kõigepealt oli võrkude adminnimine ja haldus mulle väga jõukohane ja ma sain aru, kus ma väärtust lisan. Sama ajal 
progemises oli nõudlust vähe, vähemalt ses osas, 
kuhu mina oleksin saanud pakkumist esitada. Buum oli ikkagi seal, kus 
oli vaja vett kõrbe viia: arvutid kokku panna, võrku ühendada ja
tööle saada. Paljud inimesed, kes 
oskasid progeda, teenisid oma leiva ikkagi sellist tüüpi asjadega. 
Arvuteid kokku panime ja konfisime me Skriiningus küll kõik. 

Ega see ei olnud nii lihtne nagu tänapäeval. Aitasin just oma poistel mänguarvutid kokku panna. 
Neil olid läpparid, aga nad ütlesid, et need on lahjad ja et nad tahavad \emph{desktop}'e. 
Siis ma huvi pärast tellisin komponendid ja tegin nendega koos. See on 
tänapäeval super lihtne: pistikut ei ole võimalik valesti 
panna, see läheb ainult ühte kohta kogu süsteemis. Tol ajal pidi väga 
täpselt teadma, mis \emph{jumper}'id panna, kuidas ja kuhu panna pistik, ja kui 
tegid seda valesti, siis masin kärssas läbi. Ja ei olnud olemas 
Google'it. Oli dokument või dokumendilaadne asi, mille põhjal leiutada, 
mis \emph{jumper}'itega see kõvaketas töötada võiks, mida sa ilmselt nägid 
esimest ja viimast korda, sest ka juppide tarned Eestisse käisid suhteliselt 
naljakal viisil. 

\question{Kes tõi kohvris kuskilt Singapurist või teab kust.}

Sellega seoses on mul üks hea lugu. Mul olid siis juba autojuhiload, 
pidin üle kaheksateistkümne olema. Ja oli üks habemega 
mees, kes pidas Mustamäel arvutifirmat, mille nimi mul pole enam meeles, 
ja kes läks hiljem Estoniaga põhja. Suur mees nagu karu. 
Ja siis oli Kalle Lotamõis oma Skriininguga\index{Skriining}. Emb-kumb neist 
sai Hiinast faksi (info liikus tollal faksidega), et on soodsalt pakkuda mälu 
SIMe, mälumooduleid. Hind oli röögatult hea. Nad ajasid kahe peale ja ilmselt kuskilt juurde laenates raha kokku ja 
tegid ülekande ära. Kui pappkast jõudis kohale, käisime meie Palmasega sellel tollilaos järel. Kasti lahti tehes avanes kurb 
pilt: sees oli ainult 
vahtplast. Seda raha ei nähtud muidugi enam kunagi ja Skriining lakkus neid haavu päris kaua. Samas, kui seal oleks 
olnud \emph{legit} asi, siis oleks olnud kohe vinge marginaal. 
\emph{Cowboy times}. 

\question{Sa mainisid, et tahtsid rohkem väärtust lisada. Kas sa tõesti mõtlesid juba tol ajal, 
kuidas kasulik olla?} 

Võibolla mõtlesin lihtsalt sellele, mis mul hästi välja kukub. Mulle tundus juba tol ajal, et mul kukub hästi välja tehnoloogia ja inimeste vahel liimiks 
olemine. Lähen inimese juurde, kes pole kunagi ühtegi arvutit 
näinud, pakin selle talle lahti, panen tööle ja näitan, kuidas käib. Ma 
olin kõrvust tõstetud sellest, et sain talle kasulik olla. Tema sai hakata oma tööd 
nüüd hoopis teistmoodi tegema kui varem. 

\question{Sul ei olnud abstraktne klient, vaid konkreetne 
inimene, kellel läks nägu särama!}

Jah. Sealt hakkas tulema ka esimene 24/7 kogemus. Servereid
öösiti enamasti õnneks küll ei kasutatud, aga need olid tegelikult ju 
ärikriitilised ja kui need läksid maha, siis tuli väga kiiresti kohale jõuda ja 
need ruttu tööle saada. Pluss väga vihase kliendiga tegeleda, 
seletada talle, mis juhtus. Kliendi jaoks olid need mingid maagilised kastid 
nurgas. Kuidas siis inimarusaadavas keeles seletada, mis juhtus ja miks sa 
arvad, et seda uuesti ei juhtu?

\question{Ja miks see sinu süü ei ole!} 

Või miks ma arvan, et tõenäosus, et see kohe uuesti juhtuks, on 
väiksem. Eks tihti oli raua probleeme, voolukõikumisi, miljon muud 
asja. 

\question{Tol ajal käis kõik tati ja teibiga kokku. Serveriruume ju polnud.}

Ei-ei, sellist asja polnud olemas. See oli liiga moodne sõna selle aja kohta. 
Millalgi need muidugi tekkisid, aga siis neid veel ei olnud. Valiti lihtsalt mõni 
puhas ruum, kus otseselt vett ei tilguks, et veeavarii tõenäosus 
oleks väiksem. Tihtipeale oli selleks raamatupidaja kabinet või mõni muu ruum, kuhu seda oli kõige 
loogilisem panna. 

\question{Ja ühel hetkel läksid sa Salvasse?}

Jah, Skriiningust sattusin ma Salvasse\index{Salva Kindlustus}. Mast\index[ppl]{Kaal, Madis} oli siis juba 
Foreksis\index{Forekspank}. Toivo\index[ppl]{Annus, Toivo} kutsus mind Salvasse ja Madisega oli ka juttu 
Forekspanka minekust. Ma ei mäletagi, mis põhjusel sai 
Toivo kasuks otsustatud. 

Salvas oli lihtne. Toivo ise käis ülikoolis, tal 
polnud aega sellega \emph{full time} tegeleda ja ta otsis kedagi, kes oleks 
päeval kontoris kohal. Midagi arvutivõrgust oli juba olemas, aga 
ettevõte kasvas kiiresti, nii et esiteks oli vaja tööjaamasid ja võrku 
hooldada ning teiseks inimesi \emph{support}'ida. Ajad läksid kogu aeg kiiresti moodsamaks. Meil ei olnud enam 
Novell 3.11, vaid 3.12. Laserprinterid olid mitmel korrusel. Minu projekt oli vedada maja peal laiali kaablikanalid nii, et kaablid 
polnud enam lihtsalt naelaga seina peale löödud, vaid käisid ilusasti 
plastkanali sees. 

Suur ja äge asi oli internetipanga eellane telefonipank. 
See käis niimoodi, et minu arvutis olid modem ja telefoniliin. Ja kuna me 
olime Novelli võrgus, siis raamatupidaja sai oma arvutis maksed ette 
valmistada, sisetelefoniga mulle helistada, et nüüd on kõik valmis, ja mina 
tegin sessiooni oma modemiga Hansapanka\index{Hansapank}. Ülekanded 
läksid üle ja samal ajal tõmbasime ära panga väljavõtte. 

\question{Hansapangal oli Telehansa siis vist tõesti juba olemas.}

Ma ei mäleta, mis toote täpne nimi oli, aga see käis modemi teel ja mingi 
\emph{fat client} tuli installida, millega sai tõmmata panga väljavõtteid 
ja teha ülekandeid. Arvutiekraanil oli ülekandevorm, mille täitsid ära, 
ja ka \emph{roles and rights} oli juba olemas. Näiteks mina sain teha 
pangasessioone, aga ei saanud teha ülekandeid. 

\question{Tõenäoliselt ei olnud see fail krüptitud, nii et 
mõningase vaevaga oleks saanud raha kanda ka mujale?}

Ma ei mäleta, kui kõva \emph{security} seal oli, aga 
esimesed jäljed \emph{roles and rights}'ist olid juba olemas, selle peale mõeldi. 

Töö või protsessi mõttes oli see suur samm edasi. Vanasti 
ju raamatupidajad tiksusid kogu aeg panga vahet, aga nüüd ei pidanudki 
enam pangas eriti käima. Ainult sularaha oli vaja ära viia. Kontole laekumised, kontojääk ja muud sellised asjad olid
\emph{magically} raamatupidaja arvutis olemas põhimõtteliselt iga kell, kui ta 
tahtis. 

Salva tegi siis ka otsuse, et enam ei maksta palka sularahas, mis oli tol ajal tavaline, vaid Salva Kindlustus avas kõigile 
töötajatele Hansapangas kontod ja deebetkaardid. Need
olid nii kallid asjad, et kui see oleks jäetud töötajate teha, siis ilmselt enamus 
poleks selle projektiga kaasa tulnud. Palju lihtsam oli palk sulas iga 
kuu välja võtta, kõik olid sellega harjunud. Aga siis hakkas palk reaalselt panka tulema 
ja pidi leidma ATMi, kust palk korraga või jupikaupa välja võtta. 

\question{Miks see Salvale kasulik oli?}

Et saaks sularahaga mässamisest lahti ja kõik oleks \emph{clean}. Ega 
kellelegi, eriti mitte raamatupidajale, ei meeldinud sellega tegeleda. 

\question{Mille peal Salva peamine äriprotsess jooksis?}

Hea küsimus. Minu arust oli meil mingi naljakas 
Inglismaalt ostetud programm, nime ei mäleta. Küll aga mäletan, et 
käisime Tõnu Laagiga\index[ppl]{Laak, Tõnu} ühes Lõuna-Soome 
kindlustusseltsis nende infosüsteemi vaatamas plaaniga see osta. Ost 
jäi kokkuvõttes küll katki. Ajastu näitena veel selline infokild, 
et Toivo\index[ppl]{Annus, Toivo} ei saanud sinna kaasa tulla sellepärast, et 
tal ei olnud välispassi, aga minul oli.

\question{Miks sul välispass oli?}

Ma ei mäleta, miks. Olin ehk kuskil spordi pärast 
võistlemas käinud. See võis olla veel 
Vene pass või siis ikkagi Eesti pass ja mul oli viisa, aga Toivol polnud? 
Igatahes Soomes oli neil 
korralik infosüsteemi moodi asi. 

\question{Kas sa siis tegid sporti ka?}

Pigem sel ajal just enam ei teinud. Kooliajal tegin
orienteerumist ja kui tulin ITsse, siis jäi see katki. Nii et sisuliselt ma 
vahetasin spordi IT vastu osalt Nõukogude Liidu lagunemise tõttu ja osalt
arvutite tuleku tõttu. Orienteerumine oli meil väga tugevalt finantseeritud Saue 
sovhoosi\index{Saue sovhoos}\sidenote{V. I. Lenini nimeline 
köögiviljakasvatuse näidissovhoos.} poolt ja kui sovhoos ära kadus, vajus treeningugrupp ka laiali. Samas minu 
jaoks tuli IT varem.

\question{Aga sa teed ju praegu ka sporti?}

Jah, pärast sõjaväge hakkasin uuesti tegema. Sõjaväkke sattusin aastaid 
hiljem, kui olin juba väga vana. Algul olin ülikoolis ja tol ajal 
ülikool vabastas sõjaväest või lükkas seda edasi. Aga mul oli vaja nii palju tööl käia, 
et ülikooli enam ei jõudnud ja kukkusin välja. Siis 
jõudsin veel olla mõnda aega nii, et ei saadetud sõjaväkke, aga lõpuks ikkagi 
läks asi tõsiseks ja tuli ära käia. 

\question{Mida sa ülikoolis õppisid?}

TPIs\index{Tallinna Tehnikaülikool!Informaatika} infosüsteeme. 
Käisin koolis Salva kõrvalt. Keskkooli lõppedes oli mul koolist ja 
töö kõrvalt õppimisest nii suur tüdimus peal, et lubasin endale
kõigepealt aasta aega ainult rahus tööd teha. TPIsse läksingi aasta pärast, aga see oli tegelikult viga, sest siis ma 
olin juba niivõrd töö-\emph{mode}'is, et kooliskäimisest ei tulnud eriti midagi välja. See oli kõige madalama 
prioriteediga asi ja pigem käis kummivenitamine, kuni lõpuks tuli kahe-kolme aastaga eksmatt. 

Pärast sõjaväge tegin EBSis baka ära. EBS tegi 2000ndatel IT-juhtimise 
eriala. Ma olin esimeses lennus 
koos Alek Kozlovi\index[ppl]{Kozlov, Alek} ja muu kambaga. Tegime 
viieaastase baka, nüüd on ju kolmeaastane. See aeg oli 
ka väga kihvt, aga see on juba teine lugu. 

\question{Mida sa pärast Salvat tegid?}

Salvast sattusin kohta, kus nüüd saan uhkustada, et töötasin seal 
koos hilisema Eesti Vabariigi presidendiga\index[ppl]{Kaljulaid, Kersti}. Üks sugulane 
rääkis mulle, et üks firma, kes tegeleb sidesüsteemidega, otsib 
inimest ja kas tahaksin nendega rääkida. Miks mitte. Ma olin Salvas juba mitu aastat olnud. 

Sain kokku Rene 
Maksimovskiga\index[ppl]{Maksimovski, Georgi-Rene} ehk president Kaljulaidi 
abikaasaga, kes oli sellise ettevõtte omanik, mis pani Eestis üles Siemensi 
telefoni keskjaamasid. Sihtgrupp oli suured ettevõtted: pangad, 
riigiasutused. See oli selles mõttes \emph{next phase}, et kui 
personaalarvutid olid raamatupidamisosakondades olemas, siis umbes aastal 1995 
oli sidet vaja: võimalust helistada nii 
majas sees kui ka välismaale nii, et ei krõbiseks ja saaks kaugekõnesid teha. 

See, kuidas me tol ajal Siemensi telefoni keskjaamasid paigaldasime ja hooldasime, 
oli samamoodi vee viimine kõrbesse. Täna pole lauatelefone 
õieti kellelegi tarvis, mobiiltelefonid ajavad asja ära. Aga see oli enne 
mobiiltelefonide aega ja iga inimese lauale oli vaja 
telefoni, millega saaks helistada maja piires ja majast välja. 

\question{Jagada suure hulga inimeste vahel  väikest hulka telefoniliine!}

See oli vinge etapp, aga juba ITst kaugemal, telekomi maailmas. Haberstist\index{Haberst} 
edasi sattusingi Uninetti\index{Uninet}. Seal tegin kaasa selle aja, 
kui Uninet tuli telefonivõrgu turule ja paigaldasime telefonikeskjaama 
Eestisse. Sealt omakorda liikusin Elisasse\index{Elisa} ja Elisast Skype'i\index{Skype}. 

\question{Sinu jutust jääb mulje, et sa oled alati
kaablit tõmmanud, aga nii kaua, kui mina sind tean, on su tegevuse 
subjektiks eestvedaja või juhina pigem inimene kui kaabel. Mis hetkel sul see 
vahetus toimus ja kas sa üldse näed siin vahet?}

Kaablivedamisega sai mu karjäär alguse, 
tegelikult ma ei oska seda üldse hästi. Kuni Haberstini oli mu töö
ikkagi sügavalt arvutite ja arvutivõrkudega seotud ning kogu Habersti ja 
Unineti periood oli pigem keskjaamade progemine ja laiendamine või 
veasituatsioonid -- 24/7 töö. Näiteks kui Jõhvi piirkonna politsei side läheb maha ja seal piirkonnas 112 ei tööta, siis on päris suur 
probleem. 

\question{Aga inimesed?}

See algas Haberstis\index{Haberst}. Organisatsiooni kasvades oli ühel hetkel vaja 
mu roll formaliseerida ja keegi ilmselt tegi mulle ettepaneku hakata teisi insenere juhtima. Ma 
kasvasin tiimist välja tiimi ehk tehnikaosakonna juhiks. See on kindlasti murdepunkt, kui sa ei
võta vastutust mitte ainult enda, vaid ka teiste 
inimeste töö eest. Edasine karjäär on paraku pea kogu aeg 
olnud kas tootejuhtimine, projektijuhtimine, inimeste juhtimine või 
nende segu. 

\question{Miks \enquote{paraku}?} 

Ilus oleks minna tagasi spetsialisti liistude juurde, kus saaksin 
vastutada ainult oma töö eest. 

\question{Miks sa pole läinud?}

Pole vist julgenud. Üks põhjus on kindlasti see, et olen oma kompetentsi 
kõvasti kaotanud. Juhtides 
tehnilisi tiime Skype'is, Elisas või nüüd Fleepis\index{Fleep}, näen, kui andekaid tehnilisi talente on tegelikult olemas. Ja et ma ei ole
tehnilistes oskustes üldse konkurentsivõimeline.

\question{See on väga tuttav tunne.} 

Teine põhjus on illusioon, et olen omandanud inimestevahelise suhtluse ja 
koordineerimise \emph{skill}'id ja 
parem sõidan nende peal.

\question{Kas sul tuli see pärast esimest otsust loomulikult?}

Jah, kuigi Uninetti läksin ma ka ikkagi telefonikeskjaama paigaldama ja haldama. Haberstis olid väiksed ettevõtete keskjaamad. Uninetiga panime üles nii-öelda telekomi keskjaama, 
mille külge käivad väiksed klientide keskjaamad ja mis 
ühendub SS7ga ülejäänud telefonivõrku. Meil olid ühendused nii Soome 
Finneti kui ka kõigi Eesti operaatoritega. See oli ka väga põnev aeg, aga sealgi kasvasin ma 
paraku jälle tehnikatiimi või võrguoperaatori üksuse 
juhiks. Pärast Elisas juhtisin mobiilivõrku, kus oli ka raadio-pool. 
Sisuliselt oli tegu juba inimeste ja tehnoloogia koordineerimisega, aga enamasti on selles 
rollis ikkagi ka tehnoloogiastrateegia pool sees. 

\question{See kõik seletab väga hästi, miks ma tean sind 
praktilise inimesena.}

Ma ei ole oma karjääri kunagi teadlikult planeerinud. Pigem olen sattunud väga huvitavatesse tiimidesse 
või kollektiividesse, mis on mind kuhugi suunas arendanud, 
kogemusi andnud ja järgmisi uksi avanud. Ma ei mõtelnud samme ette, 
kuhu ma tahaksin jõuda. Pigem olen otsinud meeskondi, kus ma 
sisse minnes tajun, et olen kõige rumalam inimene ruumis. 

\question{Tahtsingi just küsida, milline on äge tiim.}

Võimalikult mitmekesine ja üksteist toetav -- 
niisugused asjad on mulle ka tähtsad. Aga ennekõike see, et valdkond arendaks mind. Võibolla kõige parem 
näide oli Salvast Haberstisse minek. Novelli adminnimises ja 
arvutivõrkudes tundsin ennast juba suhteliselt kindlalt, aga mis on telefonikeskjaam, ISDN ja 2megabitised ühendused? Õrna aimugi 
polnud! Veelgi enam, mind saadeti rahvusraamatukokku, kus oli just keskjaam 
üles pandud, et mine tee koolitus. Ma polnud seda telefoni, mille koolitust ma tegema pidin, kunagi elus näinud. Hüppasin vette, ujusin ja õppisin selle käigus, kuidas ujumine käib. 

Väga kihvtid inimesed on olnud ja see on kõige tänuväärsem asi. Lisaks on kihvt, kuidas mingid inimesed käivad ringiga, näiteks kuidas 
jõuan lõpuks Masti\index[ppl]{Kaal, Madis} või Toivoga\index[ppl]{Annus, Toivo} 
Skype'is\index{Skype} kokku tagasi. Mõnes mõttes on see maailm väga suur, aga teistpidi 
jälle tulevad teatud inimesed su juurde ringiga tagasi. Samamoodi hooned, näiteks see, kus me praegu oleme\sidenote{Ajasime juttu Tehnopoli hoones 
Akadeemia teel.}, või Suur-Ameerika 18, kuhu ma sattusin 
Haberstisse minnes uuesti ja töötasin selle ruumi kõrvaltoas, kus ma esimest korda 
286 taga istusin. 

\question{Siinsamas majas oli ju Skype!}

Täpselt selles tiivas -- mina tulin tööle siiasamasse esimesele korrusele. 
Vastas üle koridori oli Paananen\index[ppl]{Paananen, Tiit} oma 
\emph{certification}'i tiimiga. Mul oli mõnes mõttes nagu \emph{coming 
back home}, kui see osa valmis sai ja meile siia ruumi 
pakuti. 

\chapter{Tõnu Samuel}
\index[ppl]{Samuel, Tõnu}
\label{sisu:tonu}

\question{Kuidas jõudsid arvutid sinu juurde ja sina arvutite juurde?}

Arvutite juurde ma sain kolmeteistaastaselt (aastal 1985), kui mul tekkis esimest korda ligipääs arvutile, mis oli tegelikult programmeeritav taskuarvuti. Aga see oli ikka mäekõrguselt rohkem, kui mingisugune MK-51\sidenote{Elektronika MK-51 oli alates 1982. aastast Zelenogradis\index{Zelenograd}, Billuris ja Rodonis toodetud Nõukogude taskuarvuti, mis oli modelleeritud Casio FX-2500 alusel.}, mis oli tollal tavalisel koolijütsil kõige kõrgem unistus-arvuti. 

\question{MK-51 oli nõukogude kalkulaator ju?}

MK-51 oli jah nõukogude kalkulaator, mis oskas trigonomeetriat, aga mina sain ligi välismaa Casio PB-100-le\sidenote{PB-100 oli Casio üks esimesi ja lihtsamaid samme taskuarvuti juurest päris arvuti poole. See toodi nime PB-100 all turule 1982. aastal ning 1983. aastal ka kui TRS-80 PC-4 (Tandy Radio Shack) ja OP-544 (Olympia). Tegu oli QWERTY klaviatuuriga päriselt programmeeritava arvutiga, kuigi üherealine ekraan tegi CASIO BASIC-us\index{BASIC!Casio BASIC} programmeerimise küllalt vaevarikkaks.}. 

\question{See on ju klassika?}

Nojah, mingis mõttes. Ta on taskuarvuti, tal on pool kilobaiti mälu (mis tänapäeval tundub ulmeliselt vähe), oskab ainult BASIC-ut. Aga tollal mul võttis silme eest kirjuks, sellepärast et klaviatuuril oli terve tähestik peal. Tundus ikka, et see on täielik kosmos. 

\question{Kust niisugune asi õndsasse nõukogude vabariiki sattus?}

Kõige esimene ligipääs oli lihtsalt selline, et ma nägin kuskil komisjonipoest sellist asja. Ja kuna sellel kalkulaatoril olid tähed, siis ta jättis minusse nii sügava mulje, et ma käisin rääkisin sellest igale sõbrale. Ja siis paar päeva hiljem tuli üks tuttav, kes juhuslikult oli suht rahakast perest, selle sama asjaga mulle nina alla liputama. Et \enquote{Mäletad, sa rääkisid mulle}. Siis ma sain seda natukene veel näppida. Aga see ei olnud minu oma, ma ei saanud seal suurt midagi teha. Läks mingisugune, ma ei tea, nädal kuu, mingi x aeg, mööda ja juhtus selline ime, mis juhtub ainult filmides. Noh, NSV Liidu ajal oli see täpselt sihukest laadi sündmus. Tuli info, et mul on välismaal rikas tädi. See \enquote{rikas} ma mõtlesin sinna juurde vist ise välja, tollal tundus iga asi rikas, mis oli välismaal. Aga selles mõttes et, \enquote{mul on vanatädi Šveitsis!} Täitsa juhuslikult see vanatädi saatis meile veel kirja ka, et \enquote{aga äkki tahate midagi siit}. Ja mina teadsin täpselt, ma tahan täpselt samasugust asja\sidenote{Tõnu peab oma armastatud Casio-t silmas.} saada. Ma kirjutasin kohe ära talle, et ma tahan täpselt sellist asja saada.

Ma olin kolmteist ja, ma  tagantjärgi just vaatangi, et tollal tundsin ennast väga täiskasvanuna, aga kui ma praegu vaatan kolmeteistaastaseid, siis ma saan aru küll, miks ma nihukese kirja kirjutasin. Väga lakooniliselt, et kui sa juba küsisid, siis üks arvuti oleks mulle just nagu vaja. Ja tuligi. See tuli küll pool aastat hiljem. Tollal käis niimoodi, et kiri käis siit Šveitsi kaks kuud. Siis tuli kaks kuud hiljem vastus, et unustasin tüübi kirjutada. Kusjuures, nii nagu oli paberkandja peal, ega ma ka ei tea, kas ma kirjutasin tüübi või ei kirjutanud. Igatahes pool aastat hiljem tuleb kiri, et \enquote{vabanda, tüüpi olnud kirjas, ma oleks sulle hea meelega saatnud}, ma olin nii solvunud. Siis ma kirjutasin jälle, siis ootasin veel pool aastat, siis ühel hetkel tuli tollist kiri, et  saatke viiskümmend rubla tollimaksuks. See oli tollal nii kosmiliselt suur raha\sidenote{Kooli raamatukoguhoidja kuupalk oli suurusjärgus 100 rubla ja pudel viina maksis 10 rubla, pudel Žiguli õlut aga 33 kopikat.}, et ma kuskilt laenasin selle. 

\question{Tollimaks tol ajal? Loogiline, tegelikult, et kui importida, siis mingi maks rakendub.}

Ma olin tegelikult väga vaesest perest, viiskümmend rubla oli minu jaoks ulmeliselt suur raha. Ja kui ma lõpuks nägin tolli hinnakirja tuli välja, et välismaine kalkulaator on viiskümmend rubla. Aga sellel asjal oli kirjutatud karbi peale \emph{personal computer} ja selle puhul oleks maks olnud vist kolmsada või kolm tuhat või igatahes mingi nihuke number, mida ma kohe kindlasti ei oleks suutnud kuskilt leida. 

Sealt kuskilt hakkas asi minema. Kui ma olin viisteist,  mul õnnestus esimest korda kuhugi Normasse tööle saada. Ja kui ma olin kuusteist, siis selleks ajaks ma olin täiesti veendunud, et ma tahan arvutitega tööd teha. Haridust mul ei olnud, ma teadsin, et keegi mind tööle ei võta. Aga oli selline unistus, et ma lähen kuhugi koristajaks. Ma teadsin tollal, et on olemas arvutuskeskused. Need olid nihukesed kohad, kus oli tohutu suur arvuti, terve maja oli seda täis. Ja ma lootsin, et arvutuskeskuses äkki keegi võtab mind koristajaks. 

\question{Lähme korraks tagasi. Mind hakkas see personaalne kompuuter huvitama. Mis sa tegid sellega?}

Üks asi, ma lahendasin ruutvõrrandit koolis. Kolm numbrit sisse, kaks välja, väga kiire. Teised kõik tagusid oma kalkulaatorit, aga vigaselt ja said tihti valesid vastuseid. Aga, ma ei tea, see seitsmes klass oli selles suhtes imelik, me tegime vist pool aastat ruutvõrrandeid matemaatikas. See viskas kõigil  väga väga ära ja mina olin ainuke, kes kunagi ei eksinud. Sellepärast et mul oli kolm numbrit sisse, kaks välja. Kõige naljakam efekt oli see, et matemaatikaõpetaja lõpetas kodus asjade lahendamise, sest et tunnis käis niimoodi, et \enquote{davai, kodanik, see, tõuse püsti, loe oma vastused ette}. Ja pärast seda õpetaja lihtsalt vaatas küsivalt minu poole: kui ma noogutasin, siis  järelikult oli õige. Ma suutsin enam-vähem nagu reaalajas neid numbreid toksida, sel ajal, kui ta vastuseid luges. 

\question{Sest ega õpetaja tegi ju ka käsitsi, temal ka ei olnud sellist arvutit!}

Ei olnud jah, ta pidi tööd tegema selle nimel. Ja ka eksis. Oligi kuidagi niimoodi, et ma sain ükskord õpetaja vea kätte,  enam-vähem pärast seda ta vist loobuski. 

Teine asi, muidugi, inglise keele õpetajat sai trollitud. Koolis rumal küsimus, küsida inglise keele õpetajalt, kas eksamil arvutit võib kasutada. Nihuke natuke vanemapoolsem daam vaatab tõsiselt hämmastunult, ütleb, et \enquote{kui sul tast kasu on, eks kasuta}. Ja siis käis laua alt  klaviatuur mürtsuga laua peale,  terve sõnastik oli sees. Ja tal oli nagu väga piinlik siis  öelda ikkagi, et enam ei tohi. 

Aga noh, nad õppisid üsna ruttu muidugi.

\question{Mis sind selle aruvuti juures niimoodi paelus. Lihtsalt see klaviatuur, või et teda sai programmeerida või?}

Midagi oli. Ma tagantjärgi ei oska seletada, aga ma olen lapsest saati olnud selline laps, kes absoluutselt kõik asjad ära lõhub. Näiteks kodus oli raadio ja vot minule ei mahtunud pähe, kuidas saab olla inimene selle kasti sees. Selge see, et inimest seal  ei ole, aga sa ju kuuled, et on. Ja vot see konflikt tekitas mul selle, et ma lammutasin kõlari ära ja sain selle eest vist peaaegu peksa kodus. Tollal suur lampraadio, selline kast,  ja  ma uuristasin ennast  seal ees olevast võrgust läbi, et teada saada. Raadio oli tollal asi, mis osteti üks kord elus. Kui sa selle ära lõhud, siis \ldots. Pahandust oli palju. Kõik äratuskellad, lahti oskad ikka võtta, kokku panna enam ei oska. Ega ma selle taskuarvuti ka lõpuks lõhkusin ära. Ma üritasin teda lahti võtta, aga see oli kõik kleepsuga kokku pandud. Ja siis ma niimoodi venitasin teda lahti ja see kleeps muudkui venis ja venis ja ma kasutasin kääre, et nagu natukene kaasa aidata, aga seal oli üks lintkaabel sees, mida ma ei märganud ja suutsin selle ka läbi hammustada. Niimoodi. 

Aga see on meeletu huvi, kuidas asjad töötavad. 

Kuigi mul on üks väga oluline mentor ka elus olnud, naabrimees, kes viitsis mulle seletada. Lapsest saati olen igasugu asju pidanud ehitama, mingeid väikseid raadiosaatjad ja nihukesi vidinaid. Ma ei tea, sellel naabrimehel olid närvid ühelt poolt läbi, aga teiselt poolt aeg-ajalt ta viitsis. 

\question{Sa olid Tallinna poiss?}

Jah, ma olen Tallinnas kogu aeg olnud. 

\question{Siis neid juppe ja pudinaid, mille neid raadiosaateid teha, ikka liikus?}

Väga raske oli tegelikult. Tavaliselt asi läks ikka niimoodi, et said kuskilt mingi skeemi. Skeemid millegi hea tegemiseks olid tavaliselt just sellised, et kui sa jooksid sellega poodi, et nüüd ma hakkan  ehitama, siis selgus, et seda kõige põhilisemat asja ei ole. Näiteks käis tollal raadioajakirjas\sidenote{Tõenäoliselt peab Tõnu silmas ka mujal jutuks olnud ajakirja \begin{russian}Радио\end{russian}.\index{\begin{russian}Радио\end{russian}}} läbi arvuti skeem. Tunduski täpselt nii, et  nüüd ma teen ise arvuti. Hoobilt jooksid, et kust ma need asjad kõik saan ja selgus, et neid mikroskeeme lihtsalt ei ole. Videokontroller ja CPU ja muu oli defitsiit, ma ei omanud ühtegi kanalit, kust midagi saab. Kindlasti, tuttavatel ma tagantjärgi tean, kellel töötasid  vanemad sõjatööstuses, need suutsid kõike hankida. Ühel tuttaval olid näiteks vanemad keemikud,  kui poisikesena tahtsime igasugu plahvatavaid asju teha, siis me käisime nende keemikute juures ükshaaval aineid pinnimas, muidugi varjates nende tegelikku otstarvet. Ma ei tea, kas nad saidki aru. Tagantjärgi saan aru, et see oli väga rumal küsimus, nad pidid tegelikult aru saama, milleks me  näiteks salpeetrit vajame. Või, teine variant, just olid nii targad inimesed, et nad teadsid veel sadat otstarvet ja ei suutnud seal vahel enam ohtu näha. 

\question{Või mõtlesid, et kui poiss oskab juba salpeetrit küsida, siis vast näppe päris küljest ei lase?}

Vot, ei tea, meil keemiaõpetaja läks väga närvi ükskord sellepärast et me hakkasime nitroglütseriini valmistama\sidenote{Ka mina mäletan seda kihu. 1956. aastal ilmus Eesti Riikliku Kirjastuse sarjas \enquote{Seiklusjutte maalt ja merelt} Jules Verne \enquote{Saladuslik saar} ja seal oli juttu nii nitroglütseriini plahvatusjõust kui ka selle valmistamise protsessi detailidest.}. Ma tagantjärgi mäletan seda, kuidas keemiaõpetaja meile innustunult seletas, et see kodustes tingimustes ei õnnestu. Aga tagantjärgi saan aru, et see oligi täpselt see, et ta üritas nagu öelda, et ärge tehke, sest teisiti meid ei veena. 

\question{Nüüd on ju teistpidi. Kui läheb keegi eetrisse ja teatab, et nende süsteemid ei ole häkitavad, siis kohe palju inimesi proovib!}

Tavaliselt ta käib just sellise ülbuse noodiga sealjuures, et meie oleme nüüd paremad kui nemad. Ja kui ükskõik kellest väidad end parem olevat, siis tavaliselt see keegi solvub. Igatahes tavaliselt on mingi motivaator, miks selline väide toob sellise tulemuse. 

\question{Kust sul see arvutuskeskuse mõte tuli, miks just arvutuskeskus?}

Ma ei mäleta, kust ma selle info sain, et nihuke asi üldse olemas on. Meil vist üks tuttav käis kuskil arvutuskeskuses  mingi onu juures, õnnestus korra ennast sinna kaasa nihverdada ja see tundus nii põnev. Tuba oli arvutit täis. Igatahes teadsin, et ma tahan arvutuskeskusesse. Ja kuna ma midagi ei osanud, siis ma, jah, käisin mööda  uksetaguseid. Tagantjärgi saan aru, et ääretult naiivne mõte, et ma lähen sinna tööle. Käia viieteistaastaselt, ilma igasuguse hariduseta, uksi kulutamas. Aga võeti tööle. Kuusteist olin.

\question{Soh. Tegema mida?}

Arvutioperaator. 

\question{Keskkool jäi sul tegemata?}

Üheksanda, tollal kaheksanda, tegin lõpuks tagantjärgi ära. Aga sealt edasi ei ole midagi teinud. 

\question{Mis see arvutioperaatori töö endast kujutas?}

Oi, see oli väga ülbe töö selles mõttes, et ma mul oli lubatud isiklikult arvuti käima panna ja klahve vajutada. See ületas mu ootusi tugevalt, olin sel hetkel tõsiselt iseendaga rahul. Mitte midagi ma sellest ei teadnud, algne ülesanne oli lihtsalt mingit teksti sisse panna. Vene keelt ma oskasin hästi. Arvutuskeskuses üks osa tööst oli  tarkvara tegemine mingitele Venemaa asutustele. Meil oli asutusetäis tädisid, tollal mul oli tunne, et täiesti surmalähedased, vanamammid, sellised neljakümnesed. Kuueteistaastase poisi pilk on selline. Aga nemad siis programmeerisid  midagi, millest me (seal tuli paar poissi veel lõpuks)  nagu väga aru ei saanud. Kui nad kirjutasid kasutaja juhendi, nad kirjutasid selle käsitsi paberi peale ja meie pidime selle arvutisse sisestama ja välja printima. Aga see paratamatult tõi juba õiguse printerit näppida ja\ldots

\question{Mis süsteemi too arvutuskeskus siis kuulus?}

Sideministeeriumi. Ja see oli hästi edev selles mõttes, et asutuse nimi oli Sideministeeriumi Info- ja Arvutuskeskus\index{Sideministeeriumi Info- ja Arvutuskeskus}, mis oli äravahetamiseni sarnane Siseministeeriumiga. Töötõendiga anti kaasa punased kaaned, mis tegid nii mõnegi sellise vajaliku töö ära, kui oli vaja midagi kuskilt läbi suruda. Lõid oma kaaned lauale \ldots. See nägi välja nagu Siseministeeriumis töötaksid, tõsised punased kaaned olid, ma ei mäleta, kas Lenin ka peal oli. 

\question{Küllap oli. Sina sisestasid tekste, aga mida need programmid tegid?}

Ma lõpupoole pidin neid ise tegema. Oli palgaarvestus, mingisugune põhivara arvestus. Mul oli tollal poisikesena väga raske aru saada, et mis asi on põhivara ja mis asi on väikevaras, siis üks mammi seletas, et \enquote{vot sina istud väikevara peal, aga mina istun põhivara peal}. Sest kuskil viiskümmend rubla oli see piir, temal oli  viiekümne viie rublane tool. Mäletan seda, et kirjutasin mingi programmi, milleks oli postitoodangu arvestus. Löödi sisse, et  näiteks kirja kohale tassimine kellelegi koju on null koma null üks kopikat ja postkontor arvutas niimoodi oma toodangu mahtu. 

\question{Põhimõtteliselt jooksis seal arvutuskeskuses kellegi ERP?}

Jah, midagi niisugust. Kuigi tollal olid kõik need sõnad võõrad. See oli muidugi mingis mõttes väga nõme, mis tollal sai tehtud, tänapäeva mõistes väga lihtsad asjad. 

\question{See tekstide toksimine võis üsna nüri tegevus olla, kas see huvi ära ei võtnud? Või sa said millalgi nonde masinatega omi asju ka teha?}

Mul oli huvi nii suur, et ma istusin seal, jah, nii kaua kui  üldse kannatas olla. See venekeelne tekst, ega ma tegelikult sain sellega hästi hakkama. Vene keele klaviatuuri on ju teistsugune, kui ladina klaviatuur. Aga sellega harjub ära ja kui vene keel on käpas\ldots Eesmärk oli saada töö kaelast ära, et teha endale huvitavaid asju. Kui ma sinna majja läksin, siis ma tegelikult ei osanud ma midagi. Ja pool aastat hiljem oli umbes niimoodi, et ma teadsin kõikidest nendest tädidest rohkem. Ja kuna meil tekkis sinna mingisugune paaripoisine seltskond, meid oli  peale minu neli tükki veel, siis me  hakkasime omavahel infot jagama. Tollal internetti ei olnud, mitte midagi ei olnud kuskilt võtta, siis enamus käis kuulujuttude ja ise katsetamise teel. 

Me õppisime näiteks residentseid programme kirjutama. Tänapäeval võiks seda isegi viiruseks nimetada.  See oli tollal nii \emph{high tech}\ldots

\question{Kuidas te sihukese asja välja uurisite? See oli ju nõukogude arvuti?}

Meil olid erinevad. Arvutuskeskuses suured masinad oli ES-1022\index{ES EVM!ES-1022} ja ES-1045\index{Arvutid!ES EVM!ES-1045}, kaks arvutit. Kuus megabaiti mälu oli, terve korrus oli selle jaoks. Ferriitmälu, bitid olid ükshaaval silmaga kõik näha. Aegajalt läks mõni bitt tuksi,  insenerid parandasid neid. 24 inseneri oli kokku, kes seda kõike siis lappima pidid. Ja kuna seda mälu pidi kord nädalas piiritusega puhastama, siis firmapeod möödusid nagu ilma muu alkoholita.  

Aga meil oli teine osakond ja meil olid kõik personaalarvutid. Meil olid kaheksabitised Robotronid, 1715\index{Robotron!Robotron 1715}. Siis tulid Iskra-1030-d\index{Arvutid!Iskra!Iskra-1030}, mis olid PC kloonid, kohutavalt halvad arvutid. Kuskilt tuli mingi DVK-2\index{Arvutid!DVK!DVK-2}, mis oli mingi IBM-i kloon. Igatahes asi muudkui arenes ja näiteks DVK-2 on selles mõttes naljaka arvutina meelde jäänud, et arvutuskeskus oli selline suhteliselt külm maja. Endla 16, tänapäeval Eesti Telefoni maja\sidenote{Selles majas tõesti asus üksvahe Eesti Telekom, aga praeguseks on maja põhjalikult renoveeritud ja seal asub midagi muud.}, mis ei pidanud sooja. Hommikul oli tubades pluss viisteist kraadi, või vähemgi, ja arvuti ei lähe selle külmaga käima. Flopi draivi rihmad olid kuidagi nihukesed jäigad. Mootor käis rihma sees ringi, flopi ringi ei lähe. DVK-l oli nii suur auk seal ees flopi jaoks, et sinna mahtus käsi sisse. Pistsid käe sisse, tõmbasid käega kogu selle asja ringi käima, et sai hoo sisse ja lükkasid floppi ruttu järele ja luugi kinni. Kui seda piisavalt kärmelt teha, sai arvuti käima. Hiljem, päeva peale, ei olnud enam probleemi. 

\question{Ikkagi, kust te infot saite? Manuaalid?}

Manuaalid olid kasutud. Täiesti kasutud. Kui Iskrad\index{Iskra!Iskra-1030} tulid, me  veel eraldi mõnitasime, sest et selle peae oli kirjutatud \begin{russian}электронная персональная вычислительная профессиональная машина\end{russian}. Personaalne professionaalne elektrooniline arvutuslik masin. 

\question{Kuidas te siis oskasite? Ei ole ju nii, et paned aga suvalisi käske ja äkki jääb programm residentseks?}

Üks asi on see, et tollel ajal kui mina arvutuskeskuses olin, seal võrku ei olnud, mitte mingisugust. Ja enamus asju käis nii, et keegi käis teises arvutuskeskuses külas, seal jälle keegi kuskilt oli midagi nuuskinud, kas ta oli välismaalt kuulnud või mujal käinud, aga tavaliselt info liikus koos inimesega. Käis kuskil külas,  tuli tagasi ja  oli lõpuks õppinud, kuidas teha mingit asja, mida me olime juba pool aastat mõelnud. Kusjuures see probleem oli tavaliselt väike. Meil näiteks printeril ei olnud täpitähti. Ja siis keegi tuli väga-väga kavala asjaga, kuidas \emph{map}-ida  klaviatuuril kandilised sulud täpitähtedeks. Tuli selle infoga  näiteks Tervishoiuministeeriumi arvutuskeskusest, sest keegi seal oli jälle mingi häkiga hakkama saanud. Või, teine häda oli see, et klaviatuurid olid aeglased. Meil oli tihti vaja ju mingeid jooni teha niimoodi, et tuli hästi palju miinuseid panna. See miinus, hoiad teda  nagu ma ei tea mida, aga ta ei jookse. Keegi õppis ära seda kiirendama, selleks tuli kuhugi porti kirjutada mingi number. Aga selline info liikus ainult suust suhu. Assemblerit\index{Assembler} ja nihukesi asju me juba oskasime kõik ise teha, aga muu oli nihuke müstika. Mingid pordid ja mida sinna kirjutada oli dokumenteerimata. Kuskilt tuli see info, et kui sa sinna porti kirjutad, läheb see klaviatuurile ja klaver on kiirem. 

\question{Kas te seda folkloori kuidagi üles ka kirjutasite või see lihtsalt jäi inimeste pähe?}

Ei, täiesti suuline. Seda ei olnud nii palju tollal, ei tulnud nagu pähe üles kirjutada. Teine asi oli, et see oli kõik selline info,  mida sa niisama ära ei andnud, sest see tõstis sind teistest kõrgemale. Näiteks ega me nendele tädidele ei rääkinud, kuidas residentseid programme kirjutada. Üks asi, et me pidasime neid madalamaks kassiks, kes nagunii aru ei saaks. Teine asi, et see andis meile võimaluse tehaseal arvutis  mida iganes ilma, et nad aru oleksid saanud. Oleks tollal tahtnud mingit kräkki jooksutada, me oleks seda jooksutanud \emph{background}-is.

\question{Aga nende teiste arvutuskeskuste tüüpidega te oleks seda infot jaganud?}

Tavaliselt see käis, jah, vorst vorsti vastu.  Sul pidi olema mingi mõju inimese üle. Tavaliselt professionaalid hindavad üksteist ja jagavad sellist infot, aga kui tuleb mingi jobu küsima, ega sa ei ütle talle. Sina oled palju vaeva näinud, et see asi ära lahendada. Ringi jooksnud ja küsinud ja mõelnud ja sa ei anna seda infot niisama ära. Vahest antakse, aga see oli osa käitumismustrist. 

Fidonet tõenäoliselt oli üks esimesi arvutivõrke, millest ma kuulsin ja kuskil nägin. Praegu ma arvan, et see võis Tarmo Mamers\index[ppl]{Mamers, Tarmo} olla, aga ma ei ole kindel. Igatahes ta hakkas nihukeseks asjaks muutuma, et ma hakkasin töö juures rääkima, et ma tahan modemit saada, tahaks nagu võrku. Meie arvutuskeskus ei pidanud seda kuidagi vajalikuks ja ma ei osanud  kuhugi õigesse kohta vajutada ka. Mulle korra vist insenerid kuskilt tagatoast pakkusid mingit modemit, mis oli niisugune kingakarbi suurune kast ja mis ei olnud see mudel, mida sa saad telefoniliini otsa panna. Oli mingisugune \emph{leased line} modem või midagi, ta ei osanud helistada, näiteks. Igatahes minu jaoks oli see täiesti tarbetu,  mul oli vaja seda modemit, mis käib telefoniliini külge. Telefoniliin oli ka tol ajal väga suur ressurss. Meil oli asutuse peale piiratud arv telefone, kusjuures me olime Sideministeeriumi arvutuskeskus! Jaam oli meie enda majas, aga meie osakonnas oli kakskümmend inimest ja kaks numbrit. 

\question{Küsin nende inseneride kohta. Kui ma mõtlen, kas lasta noor inimene tarkvara või riistvara juurde, ma julgeks teda pigem riistvara juurde lasta. Miks sind tarkvara juurde lasti?}

Aga riistvara tollal meil, ma ei mäletagi, et midagi erilist oleks olnud.

\question{No oli ju, ma mõtlen, et oleks pandud piiritusega ferriitrõngaid nühkima või midagi?}

Tegelikult see vastus, miks ma sinna tööle sain, on mul tagantjärgi nagu tulnud. Tollal ma uskusin, et ma jätsin nii tõsise mulje, et ma olin oma taskuarvutiga mingeid programme teinud. Tagantjärgi ma saan aru, et sellel ei olnud mingit asja. Tegelikult oli probleem, tollal ei olnud IT eriala üldse populaarne. Keegi ei tahtnud seal töötada, palgad vist olid madalad. Tõsine mees tegi kuskil haltuurat. Näiteks kui sa töötasid kuskil viinapoest, sa said sealt müüa midagi või teha. Arvutuskeskus oli koht, kus sa said oma ametliku palga ja midagi varastada ei saanud. Mingi reputatsiooni häda oli, sinna läksid kõik sellised matemaatikaharidusega naisterahvad, kõik nihukesed teravamad tüübid läksid kuhugi traktoristiks, seal sai kütust varastada või midagi. Igatahes seal oli mingi häda, miks sinna tööle ei mindud. 

\question{Aga siis tuli üks, kes tahtis!}

Jah. Ja mul on mulje, et, vaata, osakonnas oli kakskümmend naist ja osakonna juhataja oli mees. Tagantjärgi ma olen enda jaoks selle dekodeerinud niimoodi, et ta oli andnud kaadriosakonnale käsu, et ükskõik mis meesterahvas tuleb, tuleb lihtsalt kinni võtta ja temale anda. Sest nende paari aasta jooksul, mis ma seal olin, ma nägin, millised kismad seal käisid, naised ikka kraapisid üksteise silmad verele. Midagi füüsilist ei olnud, aga  nihukest ussitamist oli korralikult. Ja, mida ma olen hiljem näinud, et kollektiivis peavad mehed-naised mingis tasakaalus olema, vastasel juhul, mõlemas suunas, tekib jama. Ja mul on tunne, et mina olin esimene meesterahvas, kes talle ukse taha tuli, teda ei huvitanud ükski muu asi peale selle, et ma olen mees.  

\question{BBS-i aeg sattus juba sinna arvutuskeskuses olemise aja tagumisse otsa, sa rääkisid, et tahtsid sinna modemit saada?}

Hakkasin tahtma aga ei saanudki. Tollal, 1989 või niimoodi, hakkasid kooperatiivid. Täpsemalt ei mäletagi, aga  kooperatiivinduse tüüpidel oli raha paksult käes. Kui sa  midagi väga tahtsid, siis nad sulle ostsid. Ja mul õnnestus saada modemi ligi niimoodi, et see oli ainult minu kasutada, see arvuti ja see modem, ja ma suutsin ennast Fidonetti ajada. Ja vot Fidonet oli kullaauk, see oli täpselt nii nagu tänapäeval Internet. Kus seal info jooksis! Kui kellelgi midagi oli, siis sellest räägiti ja see seltskond tundus nii kohutavalt suur võrreldes selle paari arvutuskeskuse inimesega, kellega ma tavaliselt lävisin ja keda ma nagu  vääristasin endaga võrdseks. Nüüd oli kümnete viisi inimesi. Tänapäeva Internetis ei kujuta ette, et kuidas \enquote{kümnete viisi},  aga Fidonet ise oli Eestis, ma ei tea, mingi viiskümmend aktiivset inimest või kuni sada, mitte rohkem. Ja need inimesed olid targad. Sa jooksid hommikul arvuti juurde, et panna see käima, tõmmata kõik viimased kirjad ära ja vaadata, mida nad räägivad. Sest oli terve hulk selliseid legendaarseid mehi, kelle iga kiri oli kulda täis. Sa ainult lugesid seda, mida nad räägivad. 

\question{Näiteks?}

Mulle kuidagi, Sulo Kallas\index[ppl]{Kallas, Sulo} oli nihuke inimene, kes jättis mulle mulje. Tollal CD plaat oli asi, mida me teadsime, et selline asi on olemas, aga oma silmaga polnud näinud. Seda reklaamiti, et  nüüd on lõpuks ometi puhas heli. Sulo Kallas oli mingi audio-friik, kes sai endale Sony CD mängija sel ajal, kui teised igatsesid endale vene oma. Ja  ta tunnistas heli kvaliteedi ebakõlblikuks, pildus selle kasti sisust tühjaks, tegi sinna uue elektroonika. Minule jättis see kustumatu mulje, ma mäletan seda mitukümmend aastat hiljem. Ja nüüd, kus ma olen selle härraga  ise koostööd teinud, ma endiselt austan teda samamoodi. 

\question{Kui sa alguses olid Fidos klient, siis ühel hetkel hakkasid sa oma \emph{node}-i ka pidama. Millal see tuli?}

See tuli üsna ruttu. Ma olin kellelegi \emph{point} alguses, aga ma olin mingil hetkel  \emph{node} numbriga 25. Ja minu jaoks Tarmo Mamers\index[ppl]{Mamers, Tarmo} on  alati olnud see vaimne isa, kelle käest ma olen väga palju vastuseid saanud, väga palju abi ja tõenäoliselt ma tõmbasin tema juurest alguses ka kogu oma meili. Hiljem muutusin ise nii suureks, et ma näiteks vahendasin kogu Venemaa meili Eesti vahel. 

\question{Kas Venemaal BBS-id ja Fido või mõlemad olid kuidagi vähem levinud?}

Millegipärast jah, ja ma ei teagi, miks. Oli kuidagi niimoodi, et Eesti päris algusaegadel, mina olen sellest otsast ilma jäänud, käis kõik Soome küljes. Eestit ei tunnustatud välismaal üldse, meil puudus oma aadressruum, kõik olid Soome \emph{node}-d. Millalgi Eestis hakkas see asi kasvama nagu seen pärmi peal ja siis Vene omad olid kõik Eesti \emph{node}-d. Venemaa võrk oli minu meelest umbes sama suur, kui Eesti oma. 

\question{Ma olen kuulnud legende sellest, kuidas kuskilt kaugelt Venemaa avarustest käidi lennukiga Eestisse Fidosse. Kohver flopisid kaasas ja muudkui kopeeriti öö läbi.}

Aga tollal oli see vist isegi kiirem, sest üle telefoni läksid asjad nii aeglaselt, et oli odavam kohale lennata,   lendamine oli odav. Minu meelest Moskva lend maksis kas üksteist rubla või mingi niisugust, mis oli üsna väike raha.

\question{Kuidas Eesti oma tsooni saamine käies? Oli ju Nõukogude Liit veel, või ei olnud enam?}

Vaat ei oska öelda. Mina jäin sellest otsast ilma. Äkki Sulo Kallas\index[ppl]{Kallas, Sulo} või keegi teine juur-guru vanematest aegadest oskab sellest rääkida. 

\question{Fido \emph{node} sa panid püsti, kas sul BBS ka olnud on?}

Mul ei ole kunagi otseselt BBS-i olnud. Kindlasti ma olen midagi mänginud ja ma olen paar tükki  äri-inimestele püsti pannud  kommertsasjade jaoks. Neid tegelikult \emph{impress}-is kohutavalt see, et  mingid tüübid olid neilt paar aastat raha küsinud, et midagi programmeerida ja nad ei olnud kunagi näinud, mis sellest programmeerimisest tulu sai.  Tuli Tõnu, küsis mingisuguse mõttetu raha ja kaks tundi hiljem asi töötas. Ma mäletan, et tollal see jättis kustumatu mulje, et \enquote{lõpuks ometi keegi, kes aitas}. Aga ma sealt edasi ei ole midagi teinud. 

\question{Oot, kuskil olid mingid inimesed, kellel oli äriline põhjus BBS püsti panna?}

Tollal oli see lootus, et äkki nüüd hakkab äri minema, sest tulid inimesed, kes ütlesid, et nüüd läheb kõik internetti. Kogu äri, kõik. Ma toon samast ajast võrdluseks sellise pisiasja, et  tol ajal kõige populaarsemal tarkvaral, Maximusel, mida kõik BBS-id kasutasid, oli \emph{user}-i  ID  ühebaidine. Ma tahan öelda, et tänapäeval  ei kujuta ette, et sa teeks nii mingit tarkvara. Keegi hoidis ruumi kokku ja ta tegi \emph{user ID} ühebaidse. Enamust BBS-e ei kasutanud nii palju kasutajaid, et nad oleks sellest välja jooksnud. 

\question{Järelikult tehti disainis õige otsus!}

Eestis oli üks või kaks BBS-i, keda see ühel hetkel tõsiselt häirima hakkas. Ma tahan öelda, et see asi ei olnud üldse nii suur. Mina tunnetan seda siiamaani, kui väga elitaarset seltskonda, sellepärast et tegelikult kõik, kes sinna suutsid tulla, olid targad inimesed. 

Näiteks Fidonetti saamiseks oli vaja kolm erinevat tarkvara koostöös tööle panna, et sa suudaksid üldse sinna ligi saada, see ei olnud üldse nii lihtne. 

\question{Sul hakkas siis Fidos tekkima mingi seltskond tuttavaid, kellega juttu rääkida?}

Jah. Ja huvitav on see, et need suhted kestavad nagu üle mitmekümne aasta. Näiteks Venemaale me hakkasime äri tegema, nendesamade Fido \emph{node}-dega, kes sealpool olid. Hüperinflatsioon oli selline kummaline asi, et kuna Nõukogude Liit oli suur, siis ühes otsas hinnad liikusid kiiremini kui teises, Eesti oli Moskvast kuni nädal aega hindadega maas. 

\question{Ahaa, ja sul oli infot ja sa said seda vahendada!}

Enamus inimesi ikkagi toimetas kuskil ajalehtede peal kuulutuste läbi või niimoodi. Aga mina  rääkisin tuttavaga Peterburis või Moskvas, et \enquote{kuule,  ma tahan printereid saada}. Tema ütleb, et hind on selline. Mina ostsin seesama õhtu pileti, hommikuks olin juba seal, ladusin kõik asjad peale, ülejärgmise öö ma tulin Eestisse, müüsin nad kõik Kinexisse\index{Kinex}\sidenote{Üks varaseid Eesti arvutifirmasid, hiljem tegeles äritarkvara ja sellega seotud konsultatsioonidega.} maha. Ja Kinex oli nii õnnelik, nad ostsid mult kõik kakskümmend printerit korraga ära. Raha oli päris palju tollal. Mis sealt vahelt sai, sellega panid jälle Venemaale, tõid järgmise kuhja.

\question{Ja tulidki arvutuskeskusest ära ja hakkasid sihukest äri tegema?}

No enam ei olnud mõtet. Ma mäletan, kuidas ma üldse eraärisse sattusin. No see on täitsa omaette lugu. Arvutuskeskuses ma sain 110 rubla miinus maksud. Eraäris sain juba päris alguses kaheksasada rubla päevas. Neil lihtsalt ei olnud mitte millegagi mind enam motiveerida. Nad motiveerisid mind ainult sellega, et \enquote{Tõnu, sinu programmid näevad paremad välja kui minu omad}, ma kirjutasin neile sihukesi jubinaid, mis käivitasid nende programme. Noh, et nad tegid oma moodulid igaüks eraldi binaarina ja minul oli mingisugune akendega asi, mis neid käivitas. Akendel olid varjud taga, ega seda ka igaüks teha ei osanud. 

\question{See oli siis tekstipõhine värviline terminal?}

Puhas tekstivärk. Meie arvutuskeskus oli sellepärast, muideks,  teistest  nagu halvem, et kõigil teistel oli mingid graafilised asjad. Meile oli neid kas vähe või ei olnud üldse. Ma tundsin ennast maru halvasti, sest teised mängisid värvilisi mänge ja mina ei saanud. 

\question{Mis mänge sa mängisid?}

Digger\index{Digger}\sidenote{Digger on 1983. aastast pärit arvutimäng, suhteliselt lihtsa graafikaga ja omal ajal väga levinud.} näiteks, mis tollal oli täisvärviline. Aga meil olid Iskrad\index{Iskra}, millel olid rohelised ekraanid. Ja nad olid kõik, muideks, veel null ja üks, polnud  pooltoonegi. Me häkkisime  jootekolviga, et saaks  vähemalt pooltoonid. 

\question{Järelikult see jootekolvi ja inseneri-huvi oli tol ajal olemas?}

Oli, aga tegelikult oskused olid madalad.  See info tuli  jälle mingist teises arvutuskeskusest, et vot sinna kohta tuleb panna kaks takistit. Praegusel ajal ma suudaksin absoluutselt ise selle esimese viie minuti jooksul nagu välja mõelda, et teeks sellise \emph{fix}-i sinna monitori. Aga tollal me kõigepealt pool aastat istusime nende monokroomsete täiesti üks-null monitoride taga. Alles siis keegi tuli ülimalt hea infoga, et kui nüüd monitor lahti teha, seal kaks juhet lahti võtta, takistid vahele panna, tekivad pooltoonid. Meil käed värisesid, kui me seda tegime. 

\question{Muidugi, monitor oli ju kallis!}

Sellel ei olnudki hinda. Lihtsalt, kui sa ta katki tegid, siis rohkem ei saanud.

\question{Aga oli piisavalt julgust, et kaas maha võtta?}

Kuidagi oli. Kui viis poissi koos on, küll see julgus tekib. Ei ole vist ilus öelda, aga me aeg-ajalt seal ikka jõime ka koos. Tollal tekkisid mingid arusaamad, et kui nagu natuke peale võtta, siis programmeerimine edeneb kiiremini. 

Kusjuures mingi huvitav nähtus oli see, et kui hommikul iseenda kirjutatud tarkvara vaatasid, oli tunne, et mingi väga tark inimene on kirjutanud. Aru ei saa, töötab, aga nagu puutuma läksid, läks katki. See oli nagu  kõrvalefekt. 

\question{Sa ütlesid, et Fidost hakkas kohe infot tulema. Mis infot? Manuaale?}

Manuaale kui selliseid tollal ei olnudki, sest kõik liikus prinditud info peal ja ei olnud ju OCR-imiseks mingeid lahendusi, mitte midagi. Üks osa oli see, et kui keegi midagi teada sai, siis ta seda levitas. Teine osa oli igasugune ostan-müün-vahetan asi. Nõukogude Liidus oli kõigest puudu. Oli väga oluline infot, et keegi midagi müüb. Keegi müüs oma vana tooli näiteks, üks jalg oli alt ära, või mida iganes, aga see oli NSV Liidus väga väärt info. Ostsin oma esimese auto  Indrek Sauli\index[ppl]{Saul, Indrek} käest, kes oli tollal aktiivne Fidokas. Tal oli  Žiguli eksportvariant. Eksportvariant oli tavaliselt 1500-se mootoriga, aga temal oli 1600! No enam nagu kõvemat autot ei andnud ette kujutada ja ma ostsin selle ära. Ja kuna ma olin seal võrgus, ta reklaamis seda seal, kus ainult paarkümmend inimest nägi, siis oli mul eelis.

\question{Kas mingeid mänge, muusikat, graafikat, sellist kraami ka liikus?}

BBS-ides väga palju. Aga tollal oli arvutivõrk niivõrd aeglane, et poleks tulnud isegi mõte mingite binaaride saatmisest mailiga. Tollal  sa vaatasid, et \enquote{oi, siin on kümme kilobaiti suur asi} ja  panid modemi ööseks tõmbama. Ma ei oska head võrdlust tuua, sest poolel rahval olid mingi 1200-boodised modemid. Peter Marvet\index[ppl]{Marvet, Peeter} kirjutas  suhteliselt ülbe tooniga meili, et  9600 bps modem on see, millega sa tunned, et tegemist on \emph{communication}-iga. 9600 bps oli siis see asi, mille üle sa võisid uhke olla, eks.  Ja oligi! Mina ei jõudnud seda osta, aga temal oli selline kuskilt saadud, ta oligi minust kõrgemal. 

Peamine oli see, et  kui sa midagi väga otsisid, siis keegi teadis, et \enquote{vot seal BBS-is ma nägin seda}, sest enamus ikkagi taandus piraattarkvarale. Muusika tuli veel hiljem. Kui MP3 tuli, ma mäletan siiamaani, et MP3 korralikuks mängimiseks oli vaja 100Mhz 486-te, mis oli täpselt selle piiri peal, et kui sa hiirt liigutasid,  oli muusika kinni. Ja kui tulid 120Mhz 486-d, siis sa võisid hiirt ka liigutada.

\question{Demoskene asjad ju liikusid?}

Oi, demod liikusid, see oli ilus! Ma siiamaani igatsen neid vidinaid, mis olid imeväikesed, aga kui käima tõmbasid, siis oli tuba muusikat täis, ekraan oli  graafikat täis. Nagu kolmemõõtmelised mingisugused\ldots See tundus mulle tollal kosmiliselt ilus. CGA graafika\sidenote{Võimaldas 320x200 ekraanilahutusega kuvada nelja ja 640x200 lahutusega kahte värvi.}, mida tänapäeval keegi ei vaata. Ma mäletan seda hetke, kui ma Diggerit\index{Digger} esimest korda nägin. Meil polnud arvutuskeskuses ühtegi helikaarti ja ma käisin Tervishoiuministeeriumi Arvutuskeskuses\index{Tervishoiuministeeriumi Arvutuskeskus}, kuskil  kellelegi selja tagant korra mingeid asju ajamas ja keegi mängis Diggerit Olivetti arvuti peal. See heli ja see kõik tundus nii võimas ja need värvid ja! 

Muideks see Tervishoiuministeeriumi Arvutuskeskus oli seal kõrval, kus on surnukuur. Minu meelest selle tänava nimi on Tervise tänav, mis surnukuuri viib. 

\question{Mõnikord käib arvuti-huviga ka ulme-huvi kaasas, kas sinul ka?}

Ma ei mäleta. Ma lugesin raamatuid lapsena palju väiksemana. Mul on kuidagi niimoodi, et ma olen igasugu seiklusjutte vee alt ja kuu pealt sarjad kõik läbi lugenud sest ma loen väga varajasest ajast ja väga  ulmeliselt palju. Aga selleks ajaks mul oli rohkem huvi mitte selle nagu väljamõeldud maailma vastu vaid mind meeletult  huvitas reaalne maailm. Sest see väljamõeldud maailm on, on mis on, ta on alati välja mõeldud. Ma olen täiskasvanueas kogu aeg fakte otsinud. Mind meeletult huvitavad faktipõhised asjad, näiteks ajalugu. 

\question{Ajaloo kohta mõni inimene ütleb, et see ei ole ju fakt. Puhas arvamus.}

Vot see on NSV Liidust tulnud inimese üks hea omadus, et sa suudad päris palju filtreerida. Ma selles mõttes kuulan Vene propagandat hea meelega selle nurga alt, et ma saan üsna hästi aru, mida nad üritavad näidata ja mis osa siis  tegelik on. Mind, see ei häiri niivõrd kohutavalt, sest aeg-ajalt just see, kuhu nad täna propagandat suunavad,  annab ise juba vastuseid. 

\question{Pärast toda arvutuskeskust sa toimetasid iseseisvalt või oli sul miski kooperatiiv?}

Ma sattusin eraärisse niimoodi, et  hakkasin tooma Venemaalt arvuteid ja teenisin selle eest tolle aja kohta meeletut raha. Kuigi see oli imelik aeg, et see meeletu raha oli ikkagi väiksem, kui kellegi teise meeletu raha. Aga kuna mul  tollal praktiliselt kõiki Eesti igasugu arvutipoed ja kõik olid kliendid, siis  üks, kellele ma kogu aeg asju vedasin, ütles, et \enquote{kuule,  hakkame parem koos tegema, et mis sa siin jahmerdad}. Igatahes tal oli kuidagi see idee, et ta müüb kogu mu kolu maha, aga mina toon ainult talle. Mind see kuidagi huvitas, sest muidu ma käisin mööda linna ja otsisin, kellele ma oma kolu lükkan. 

\question{See tähendab, et sul pidi olema päris korralik suhtevõrgustik nii Venemaal kui Eestis}

Kuigi see arv oli väike, aga võrgustik oli korralik ja, ütleme, suhted kestavad siiamaani. Näiteks Venemaal ma olen püüdnud igasugu asju ajada ja olen alati lõpuks petta saanud. Ja  on üks inimene, keda ma usaldan seal siiamaani siiralt. 

\question{Ja see võrgustik tuli puhtalt ainult Fido pealt?}

Jah. Need vanad võrgustikud ongi erakordselt usaldusväärsed. Inimesed, kes kolmkümmend aastat on üksteist tundnud, üksteisele mingit jama kokku ei keera. Sest see kommuun on selline, et kui oled seal nagu pleki endale külge saanud, ma ei kujuta ette, kuhu siis enam  taganeda. 

\question{Ma ei kujuta hästi ette, et Fido seltskond oleks ideaalsetest inimestest koosnenud. Kindlasti mingil hetkel visati keegi välja ka?}

Mind  visati ka välja. Ma ei mäleta, mida ma halvasti ütlesin, aga Tarmo Mamers\index[ppl]{Mamers, Tarmo} viskas mu väga kiirelt välja ja see oli väga hea õppetund, et Fidos ei ole demokraatiat. Fidos on igaühel oma kuningriik ja sa oled alati kellegi kuningriigis. Sa pead tema reeglite järgi mängima ja kõik. Kui sa tahad, sa võid oma kuningriigi luua, sinna rahvast hakata meelitama, aga tavaliselt sa istud üksi seal. 

\question{Aga see seab selle esimese adminide sauna-õhtu ju hoopis teise valgusse!}

Neil inimesed oli reaalne võim sind informatsioonist ära lõigata, aga seda tegelikult ei kuritarvitatud. Kes seal asja eest kinga said, need said. Ja mina sain ka asja eest. Aga oligi täpselt see, et väga kiirelt said aru, et kus need piirid on. Kakskümmend neli tundi hiljem olin ma tagasi, sest ma olin kenasti õige inimese juures vabandamas läinud ja rohkem ei teinud. 

Kloune oli seal üht ja teistsuguseid ja võimuga inimesi oli erinevaid. Näiteks  oli teada, et üks mees oskab karated, kui on vaja kellelegi peksa anda, siis oli teada, et ennemini räägiks temaga. Samas kui elektroonikat on vaja teha, tead tolle teise inimesega juttu rääkida. Näiteks Madis Kaal\index[ppl]{Kaal, Madis}, kes hiljem Skype'is töötas. Tema oli see vend, kes julges seda eriti kallist asja, nagu arvuti, ennast häkkida. Me enamus ei julgenud, arvuti oli sul üks elu jooksul. Aga tema siis kraapis seal vaibanoaga mingid rajad lahti, panin relee vahele, et modemit lahti ühendada, kui see lolliks läks. See tundus nii riskantne tegevus, et isegi teades, mida teha, ma ei julgenud seda teha. Sa teadsid, et kui sul on riistvara probleem, siis sa lähed temaga rääkima. 

Siis seal mingid tegelased kogu aeg midagi müüsid ja oli teada, kelle käest mida saab. Kõik need tollased Microlinkid ja asjad,  seal oli igaühel  üks inimene, kes seal töötas. Et kui sa teadsid, et sa tahad allahindlusega asja saada,  siis tuli temaga suhelda. Fidokatel omavahel tavaliselt tehti mingeid allahindlusi. 

\question{Sest kõik ju said aru, et homme on mul vaja, võrgustik oli tihe. Mis selle võrgustiku nii tihedaks tegi? Kas vastastikune respekt kõrge barjääri tõttu või midagi veel?}

Respekt kindlasti, sest sul ei olnud alternatiivi. Nagu su oma tsunft, et kas sa oled seal või sa ei ole seal, kaks varianti. Need olid targad inimesed ja ma ei tea, kas see  ka nagu oluline on, aga tänapäeva internetiga võrreldes on üks tohutu vahe. Kuskil aastast 2000 edasi, on internetti tulnud lollid. Ma ei mõtle kõiki. Aga varem oli see, et kui sa lugesid midagi võrgust, siis see oli kuld. Siis nii oli. Ja kuskil seal üheksakümnendate lõpus, kui Eestisse pandi püsti Delfi ja \ldots See ei olnud  selle portaali häda, aga igaüks sai endale neti koju. Kõik need horoskoobid ja muu jama hakkasid nagu nagunii hullusti levima, et nüüd enam ei tea, mis internetis on tõsi. 

\question{Targad inimesed oleksid ju võinud oma kogukonna kolida ju teise, kõrgema, barjääri taha?}

Fido on eksisteerinud  tükk aega, ta vist eksisteerib mingil kujul siiani. Seesama barjäär on ka väga tarkadel inimestel lihtsalt jalus, ega nad ei viitsi seda teha. Ja tollased piirangud ikkagi \ldots Näiteks see, et need asjad ei käinud reaalajas, sa pidid kuhugi helistama, saatma oma kirjad ära panema toru hargile. Keegi teine pidi helistama sinna sama numbri peale, kus sina olid toru hargile pannud (muidu ta ei saanud helistada), tõmbama kirjad ära. Aga tema ei teadnud, et ta peab täna helistama, et sina oled sinna saatnud. Kui ta otsustas sulle vastata, siis tavaliselt see käis kuskil kahekümne nelja tunnise tsükliga. Ma kirjutasin oma mure ära ja sain sama päeva sees kuidagi oma vastused kätte. 

\question{Mina tean sind ikka rohkem infoturbe-inimesena. Mis hetkel sa liikusid arvutite toomise juurest arvutitega seotud probleemide lahendamise juurde?}

Ma ei tea. Ma just mõtlengi, et kust see tekkis. Üks asi on see, et mul on alati olnud sügav huvi asjade vastu ja millegipärast mõte töötab ka alati tagurpidi, et mida veel selle asjaga teha saab. Mul ei ole sellist kindlat vastust, lihtsalt uitmõte. Sel ajal ma müüsin automaatvastajaid hästi palju, Eestis enamus Panasonicu brändi automaatvastajatest olid minu käest tulnud. Sama moodi Citizeni ja Casio kalkulaatorid olid kõik minu müüdud. Eesti Pank\index{Eesti Pank} kasutas valuutakursside teatamiseks Panasonicu automaatvastajat, see oli ainuke ametlik kanal, kust valuutakursse teada sai, seda muudeti vist kord päevas või kord nädalas. Igatahes, ma tahan öelda, et see oli väga-väga tõsine asi. Helistasid numbrile peale ja ta luges sulle maha, et \enquote{Ameerika dollar nii palju}. See oli väga tõsine infokanal. Ja  neil oli Panasonicu automaatvastaja vaikeparool ära muutmata, seal oli mingi kolmekohaline number. Vist oli 555, sõltus mudelist natukene. Igatahes kui sa valisid rääkimise ajal selle 555, siis  automaatvastaja tegid piiks, ja pärast seda, kui sa vajutasid 7, võisid sinna uue teate peale lugeda. Ma nägin kohe, et põhimõtteliselt ma saaks nii teha. 

\question{Aga kas sa tegid?}

Ei. Lihtsalt ütlen, et selline mõtteviis oli. 

\question{Aga kas sa ütlesid neile, et vahetage oma kood ära?}

Tollal ei olnud nagu kanalit selleks. Ma arvan, et  kindlasti ma ei öelnud, just sellepärast, et tollal kuidagi maailm töötas teistmoodi. Sul polnud isegi endal telefoni, sa pidi minema telefoniputkasse ja otsima raamatust ja\ldots  Ma ei suuda nagu ajaliselt ära määrata, aga see maailm oli teistsugune, kõik asjad ei käinud nii nagu praegu. Sa tahtsid sõbrale helistada, siis mõtlesid, et \enquote{homme helistan talle, et siis ma lähen inimese juurde, kellel on telefon}, kõik töötas teist moodi. 

\question{Aga ometi ei saanud sinust riistvaraärimeest, millalgi sul see asjade toimimise huvi sai nii tugevaks, et hakkasid hoopis sellega tegelema?}

Ma vist olen kogu aeg jooksnud mingi huvi ja raha kombinatsiooni järgi. Näiteks ma sattusin haltuura tegemisest ja spekuleerimisest  Kinexi\index{Kinex} direktoriks, see oli tollal Eesti tuntuim arvutifirma ja küllaltki tõsiseltvõetav. See tegeles kõigega ja see tarkvara osa oli päris oluline. 

\question{Jumal hoidku, sa pidid siis ju inimesi juhtima hakkama!}

Jah, aga ma olin seda  tegelikult kogu aeg teinud, näiteks see meie enda erafirma, mida me kahekesi tegime. Alguses seisime kordamööda letis: kui tema jooksis kauba järgi seisin mina letis  ja kui mina olin Venemaal, siis seisis tema letis. Mingi hetk oli raha nii palju, et mõtlesime, et mida me siin seisame, võtame kellegi tööle. Võtsime kellegi tööle. Siis istusime kodus, vaatasime, kuidas keegi müüb. Ja mina kirjutasin poele  nullist tarkvara. See seltskond läks päris suureks, meil oli tegelikult palju poode Tallinnas ja päris arvestatav hulgiäri. Eestis tegelikult kontoritehnika oli enamus meie oma. 

\question{Inimeste juhtimine tuli siis kuidagi loomulikult?}

Jah, ja see oli sõpruskond. Mina ei ole seal kunagi mingites konfliktides olnud. Ma olen seda täheldanud, et kui ma ära lähen, vaat siis on aeg-ajalt olnud see, et hakkavad üksteisele jalga  taha panema või midagi ja need aeg-ajalt eskaleeruvad ikka päris käest ära. Alati on nagu olnud  see, et me usaldame teineteist, et meil ei ole mingeid suuremaid probleeme olnud. 

\question{Mis sa praegu teed?}

Praegu mul on nihuke firma nagu Tochimo Lab\index{Tochimo Lab}. See on nii uus firma, et sellest ei ole keegi veel kuulnudki, aga selle mõte on tegelikult Planet Way Corporation-i\index{Planet Way} all teha nihukene Skunkworks-i moodi moodustis, kus me teeme uusi projekte. Sest Planet Way ise muutunud praeguseks selliseks, et meil on väga tõsised kliendid, kus tootmine peab igapäevaselt jooksma, seal ei tohi mitte midagi katsetada,  kõik peab  käima nagu kellavärk. 

\question{Aga katsetada sulle meeldib!}

Jah. Aga mul on vaja teha just asju, mida veel ei ole olemas. 

\question{Miks?}

Vot ei tea. Uute asjade tegemine on ääretult põnev ja ma  viimasel ajal olen kuidagi  aru saanud, et mida võimatum ülesanne, seda rohkem ta mulle meeldib. Ja see on osaliselt muutunud mu tugevuseks. Seesama pen testide\sidenote{Ingl. \emph{penetration test}, penetratsioonitest, on autoriseeritud küberrünnak, mille käigus testija üritab süsteemi siseneda nii, nagu päris häkker seda teeks.} tegemine on andnud sellise mõttemaailma, et sa lammutad süsteeme, mis on ehitatud kindlaks. Pen teste  ma lihtsalt ei taha enam teha, sest see on tõsiselt depressiivne töö. Aga ta on nagu andnud sellise lihtsa asja, et kui sa oled kaks nädalat olnud niimoodi, et sul ei tule ühtegi ideed ja sul on totaalne depressioon, siis aeg-ajalt tuleb läbimurre pärast seda. Inseneriteadustes üldiselt on see hea omadus, et kui sa paned piisavalt ressursse alla, siis hakkab iga asi juhtuma. Kuidas öeldakse, et kana läheb sinna pilve või mesipuusse.

\question{Ja nüüd sa tegeled uute asjade leiutamisega?}

On mingid asjad, millel on ärilised vajadused olemas (päris udu ei tee),  aga millel ei ole päris selgeid vastuseid. 

Pooltel inseneridel on see häda, et kui sa annad talle mingi väga uduse ülesande, siis nad ei suuda seda teha.

\question{No jaa, sest kuidas ma arvutan midagi, millest ma ei tea, mida ta tegema hakkab!}

See ongi \emph{skunkworks}-i asi. Vaata, kui nad seda SR-71-te\sidenote{Lockheed SR-71 \enquote{Blackbird} on pikamaa strateegiline luurelennuk, mille arendas välja Lockheedi \enquote{Skunk Works} osakond. Lennuki loomisest Clarence \enquote{Kelly} Johnsoni käe all on tema toonane alluv Ben R. Rich kirjutanud inseneride hulgas populaarse raamatu, mida loetakse nii innovatsiooniõpikuna kui inspiratsiooni saamiseks. Ka minul on nende ridade kirjutamise ajal SR-71 mudel laual just selle raamatu tõttu.} tegid, siis nende eesmärk ei olnud teha mitte kolme-machine-lennuk\sidenote{Ehk lennuk, mis suudaks  kolmekordselt helikiiruse ületada.}, nagu ta lõpuks välja kukkus. Nende eesmärk oli Venemaa kohal kuidagimoodi ära luurata. Ja keegi ei ütle sulle, mida sa selleks tegema pead. Ja tollal, ükskõik mis sa teed ei tundunud reaalne, sest  kõik, mis lendab, saab raketiga alla lasta. 

Peab olema sellise pea kujuga inimene, kes nii kaua mõtleb, kuni ta saab selle vastuse. Tehti nii kiire lennuk, et  selleks ajaks, kui rakett õhku tõuseb, on lennuk lihtsalt taevast kadunud. Ega see vastus tundub praegu elementaarne. Aga tollal  ei olnud see üldse elementaarne,  oli võimatu. Selleks peavad olema  inimesed, kes ei mõtle esimese asjana, et \enquote{ma ei võta ette, see ei ole tehtav} vaid sa suudad kuidagimoodi kuu aega tarbida raha ja kõike muud ja tulla välja kõige hullumeelsemate mõtetega ja siis panna asjale hind külge. Siis on juba kliendi asi, kas ta tahab seda või ei taha. Näiteks SR-71 ehitamisel oli see hind, et sinna läks vaja suuremas koguses titaani, kui terve läänemaailm seda tootis. See tähendas ärioperatsiooni kuskil Venemaa kõhus, sest titaani sai sealt osta. Mis tähendas, et CIA tegi erioperatsiooni puhtalt selleks, et varjata, milleks ostetavat titaani vaja läheb. Kusjuures titaan on nii haruldane materjal, et kuidas sa tema otstarvet varjad: teda ei olegi tollases aja kontekstis millekski vaja. 

\question{Nojah, sa ei tee ju sellises koguses  titaanist kelli!}

Väga-väga tõsine  ressursi probleem. Ja olla see hull, kes ütleb, et \enquote{kuulge, teeme ülikalli asja. Aga me suudame teha}! Kusjuures need vennad, kes lennukit ehitasid ei teadnud, kas nad suudavad. 

\question{Aga nad ütlesid ja neil oli usk.}

Jah, ja kui sa loed neid raamatuid, mida need inimesed on kirjutanud, siis\ldots Lennuk oli pooleldi ehitatud ja siis selgusid mingid hädad. Näiteks üks häda oli see, et lennuk venib kolmkümmend sentimeetrit, sel ajal, kui ta kuumeneb. Aga selgus, et  needsamad kütusevoolikud, mis mootorisse jooksevad, peavad ka venima. Ja kuna see asi läheb seal mingisuguse kuuesaja kraadini, siis voolikuid ei saa ühestki mitte-metallist teha aga metall ei veni. Tekkisid asjad, mida ei olnud võimalik lahendada. Nad tegidki niimoodi, et torud on üksteise sees ja kui see lennuk on maa peal, ta lekib kohutavalt. Lennuk tangitakse minimaalselt täis,  ta lendab üles, teeb paar ülehelikiiruselist sellist tõmmet, kuumeneb mõnisada kraadi ülespoole ja vaat siis pannakse tankurlennuki pealt paagid täis. 

\question{Ja sinul on see usk olemas, et sa mõtled välja ja ongi võimalik?}

No vot, see on see koht, et selleks peab täiesti hull olema, et  mitte tagasi põrkuda. Pen testide tegemine on andnud selle, et ma ei pelga väga hullusti enam meid probleeme. 

\question{Kas on nii ka olnud et ei tule välja?}

Oi kindlasti. Kindlasti on. Ma ei oska nii mõelda enam, aga see oli üks põhjuseid, miks ma pen testimise maha jätsin: iial ei tea, millal ja kas sa  tulemuseni jõuad. Ja see on meeletult depressiivne. Pen testidega veel teine häda on see, et igal juhul saad nagu peksa. Kui sa seda ära ei lõhu, mida sulle ette antakse, siis sa oled nõrk ja kui sa  ära lõhud, siis on kõik su peale solvunud. Tavaliselt on see veel nii lihtne, kuidas sa ta  lõhkusid ja öeldakse, et \enquote{nojah, nii me oleks isegi osanud}. Kuidagi, ma ei tea, ta on tõsiselt ebameeldiv töö, ma ei soovita kellelegi. 

Aga uute asjade ehitamine on selles mõttes lahe, et kui sa sinna nagu aega investeerinud, siis ma olen tavaliselt sealt ise midagi saanud ja tavaliselt ka see kommuun on midagi nagu saanud. Vot see on see koht, kus mulle meeldib inseneride hulgas näidata, mida ma tegin. 

\question{Ja sa ju ei ehita triviaalseid asju! Kuidas sa oskad? Lihtsalt kogemus?}

Vist jah, ma isegi ei tea, mida sa silmas pead. 

\question{No näiteks see sõrmus, mille kallal sa töötasid.\sidenote{Tõnu on töötanud sõrmuse kallal, mis toimib žestikontrolleri, võtme, NFC maksevahendi ja märguandjana olemata palju suurem tavalisest gümnaasiumi lõpusõrmusest.}}

Sõrmusega oli selles mõttes lihtne. Sõrmus on, eksju \emph{bluetooth}-i saatja, patarei, mingi väike mikroprotsessor ja mingid andurid. Sellist asja suudab igaüks ehitada. Ainuke asi, et ei tea, kui suurt. Esimese asjana ma mõtlesin enam-vähem välja, et kui ma tahaks midagi sellist ehitada, kui suur ta umbes tuleks. Vaadates, mida poes müüakse, Arduino näiteks, siis tegelikult igaüks suudab selle väga lihtsalt kokku ehitada. Nüüd on see, et kas ma suudan selle väiksemaks teha. Ja esimene reegel, mille ma võtsin, et kuna ma elan Jaapanis, siis peksame igast suurfirmast välja lahenduse, mis on väiksem kui see, mida turult saab. Ükskõik kui palju väiksem. Kui ma tean, et mingi asi on näiteks kümme millimeetrit suur, ma lähen nende ukse taha ja ütlen, et ma ennem ära ei lähe, kui ma saan üheksa millimeetri suuruse. Ma tahtsin konkurentsieelist. Ja tuli lihtsalt ehitama hakata, siis tulid avastused. Esimene avastus oli näiteks see, et teatud asjad, mida ma pidasin oluliseks, polnud üldse olulised. Näiteks see, palju \emph{bluetooth} voolu tarbib. Selgus, et üldse ei loe. Luges hoopis see, et palju ta magades voolu sööb, sest \emph{bluetooth}-i saatmise hetk on niivõrd lühike, et ta võib tarbida, palju tahab, mind see ei sega. Aga kui ta paari päevaga magades tühjaks jookseb, see mind häirib. Meie suutsime näiteks sõrmuse ajada nii kaugele, et suudab viis aastat karbis voolu sees hoida. Et kui nüüd klient saab karbi kätte ja teeb lahti, siis sõrmus ärkab ellu,  kui see on toodetud viimase viie aasta sees. 

\question{Kui ma sind niimoodi kuulan, siis tundub, et sul on mingisugused sellised põhimõttelised printsiibid, millest lähtudes on võimalik ehitada mida iganes?}

Tavaliselt on mingi väga lihtne läbiv idee küll jah. Minu arust kuidagi tuleb endale teha mingi väga lihtne rusikareegel, et mis asi see on. Et näiteks auto on mootor, rool, pidurid ja  sa pead hakkama teda sealt kuidagi tükkideks tegema, minimaalse asja valmis tegema. Siis tekib arusaam, milline on su probleem, mida sa tegelikult lahendad. 

\question{Praktiline käega katsutav lahendus?}

Jah, sest tegelikult ma tunnistan, et ma tegelikult vist ei saa üldse keerulistest asjadest aru. Minu esimene samm on  asi maha lihtsustada. 

\question{Ega keegi ei saa keerulistest asjadest aru, seepärast nad ongi keerulised!}

Jah, aga mul on tunne, et see vist on see tee, kuidas ma nagu asju teen. 

Toon mingi täiesti teise näite. Mind millalgi hakkas huvitama  arvutiga nägemine: kuidas arvuti näha saab? Võtsin raamatu, hakkasin otsast lugema, et mis asi on see OpenCV teek, mis on \emph{computer vision}-iks, arvutiga nägemiseks, kõige levinum teek. Kui ma olin kuskil poole raamatu peal või isegi vähem, mul juba näpud sügelesid nii kohutavalt, sellepärast et ma olin kõik ideed kätte saanud, mida ta tegelikult teeb ja need printsiibid olid lihtsad. 

Kõik teavad, et arvutiga saab  nägusid otsida, seda tänapäeval teeb juba iga telefon. Aga mind hakkas huvitama, et kas ma  suudan kokku panna, et kui inimene on pooleks lõigatud, et milline on alumine, milline ülemine ots. Tõsiselt, siiras huvi, et nägusid me leiame, aga kas me leiame jalgu või midagi muud? Kas seesama printsiip on rakendatav? Üritasin midagi kokku käkerdada ja sain tulemuseks, et sa võid neid frankensteine ehitada nii palju, kui tahad, arvuti leiab inimesele täiesti sobiva alakeha. See näeb maru naljakas välja, aga ta näeb välja ülimalt loogiline, tegelikult selline inimene võiks isegi olemas olla. Ainult et ta ei ole õige. 

Vot see on see, et sa pead proovima. Tollal ma näiteks jõudsin selle projektiga nii kaugele, et ma sain aru, et tegelikult on taust palju olulisem kui inimene. Ma ei tea, kas sa tead sellest projektist, kus ma tõmbasin terve rate.ee\index{rate.ee}\sidenote{rate.ee oli esimene Eestis tõeliselt populaarseks saanud sotsiaalvõrgu laadne teenus, mille sisu seisnes peamiselt üksteise piltidele hinnangute andmises.} alla. Ta oli hästi lihtsasti kopeeritav ja  seal oli terve elanikkond praktiliselt sees. Ja igast ühest oli terve hulk pilte. Rate.ee sai alla tõmmatud ja sai avastatud, et on Eestis nihuke sait nagu sexinestonia.com. Väike kommuun, umbes tuhat kasutajat. Et kui sul on tuhat kasutajat Eestist, kes kasutab pornosaiti  iseenda reklaamiks, siis, noh, üks on su õpetaja, üks on su kolleeg, üks on su ülemus. Väga  sensitiivne asi. Ma hakkasin siis üritama leidma  paralleele, millised profiilid kattuvad rate.ee omadega. 

\question{See on ju ohtlik või ebameeldiv küsimus, mida küsida?}

Mind huvitas tehniliselt, et kas ma olen võimeline neid kokku viima. Ja avastasin, et neid ei ole väga palju, kes kasutab mõlemas teenuses sama telefoninumbrit. Selle järgi kokku viia oleks nagu elementaarne. Aga seal on muid asju. Ma sain \emph{computer vision}-iga tehtud selle, et ma vaatan mustreid taustal. Tuli välja, et tapeedi muster on üsna unikaalne asi. Ja kui nüüd seal pildis on kaks erinevat mustrit, näiteks tapeedi muster, vaiba muster ja kolmas on näiteks  mööbli muster ja kui nende kombinatsioon on unikaalne, siis see ongi unikaalne ruum. Ja kui sa juba ruumil samasuse leiad, siis tavaliselt oli profiil ka kohe nagu arusaadav. Saad täpselt aru, et see, kes rate.ee-s  on nagu selline tore väikeste lastega on sama, kes seal sexinestonia peal kõiki neid muid asju teeb. 

\question{Need on ju küsimused, millele inimesed ei taha reeglina vastust saada, miks sina tahad?}

Mind huvitas, kas see on võimalik ja see oli võimalik. See oli minu jaoks väga vapustav avastus. Et nii saabki. Ja see tuli ka IT-inimestele tihti üllatuseks, isegi turva-ala inimestele. Kui ma siin käisin  pankuritele seda mingil hetkel näitamas, siis neile tuli see just selles mõttes ebameeldiva üllatusena, et pangas on tuhandeid tellereid. Ma ei tea, kui palju neid on, kohutav arv. Ja nüüd on niimoodi, et kui tema hoiab endast mingisugust alasti pilti kuskil, ta muutub santažeeritavaks. Ja panga jaoks on see probleem, kui mingisugune häkker seda teab, aga sina pangana ei tea. Ja sa ei saagi  teada. Tegelikult sellised tõsised mured. 

Tekkis selline mõtlemine, et tuleb ise-enda igasugu käitumist korrigeerida. Tuleb välja, et keegi teab sust alati rohkem, kui sa ise arvad. Aga lihtsalt tavaliselt on see nihukene vandenõuteooria, aga see on see koht, kus saad ise tunda. Et mina tegingi selle süsteemi, mina tean. 

Ma kasutasin seda ju millalgi spämmerite püüdmisel ju ära. Suutsin sealt nende kohta nii mõnegi pildi leida.

\question{Oleme küll BBS-ide juurest kaugele eemale läinud, aga sellest ei ole lugu, sest on jube huvitav! Aitäh!}

Sa küsid küsimusi, mida ma ei ole iseendalt korralikult küsinud. Et miks ma teen või kuidas ma teen. Ma ei tea. 

\question{Ega ei peagi teadma!}

Ma üritan olla ise ja iseenda vastu aus, et see on nagu põhiline reegel. 


\chapter{Asko Seeba}
\label{sisu:asko}
\index[ppl]{Seeba, Asko}

\question{Kuidas jõudsid arvutid sinu juurde ja sina arvutite juurde?}

Mina elasin oma teadliku lapsepõlve 
Viljandimaal, Viljandist Riia maanteed pidi linnapiirist viis-kuus kilomeetrit välja sõita, ja käisin linnaservas Carl 
Robert Jakobsoni nimelises Viljandi 1. Keskkoolis\index{Viljandi 
1. Keskkool}. Hiljem oli see Jakobsoni 
gümnaasium ja nüüd peale riigigümnaasiumite tegemist gümnaasiumi enam ei ole, 
Jakobsoni kooli nime all on ainult põhikool. 

Arvutiteni jõudsin sealsamas koolis. Meil oli tore arvutiõpetaja 
Heiki Pettai\index[ppl]{Pettai, Heiki}, kes tegutseb vist praegugi 
IT-vallas, küll enam ammu mitte õpetajana, aga rohkem 
spetsialistina. Käisin neljandas või viiendas 
klassis, kui sain teada, et naabripoiss Toomas Aas\index[ppl]{Aas, Toomas} (praegu üks kõvemaid tarkvaraarendajaid) käib arvutiringis. Ta oli sellest 
rääkinud, aga minu teadmised arvutist olid hästi lapselikud. Olin arvutit näinud telekast lastesaates ja sellega
tehti midagi naljakat, aga ma ei osanud sellest tol hetkel 
midagi arvata. 

See hetk, kui klõps käis, oli hästi ootamatu ja 
lühike, enam-vähem sekundiga. Sattusin ükskord nägema koolikoridori 
peal, kuidas seesama naabripoiss läks arvutiklassi pisikesest uksest sisse ja kui uks paotus, paistsid sealt arvutid!
Seal oli küll ainult kolm arvutit, Vene DVK-2d\index{DVK!DVK-2}, aga see oli minu jaoks
maagiline moment, et märkasin midagi, mida olin telekast näinud ja
väga kaugeks pidanud, ning nüüd oli see järsku kahe-kolme meetri 
kaugusel. Klõps käis ära ja ma küsisin täiesti spontaanselt, kas 
tohin ka sisse tulla. Arvutiõpetaja Heiki Pettai 
lubas ja ma olin hetkega müüdud ning tahtsin seal käima 
hakata. Sellest hetkest peale teadsin, et mu ülejäänud elu peab olema 
arvutitega seotud. 

\question{Oskad sa öelda, mis täpselt see maagiline asi oli?}

Mis see lapsel võis olla? Et näed mingisugust lahedat asja, millest aru 
ei saa. Tolleaegsed arvutid olid rohkem sellised -- inglise keeles 
on tore väljend \emph{exposed} -- füüsiliselt avatud, 
igasuguseid keerulisi asju sai silmaga näha: juhtmeid, trükiplaate ja muud värki ja möllu. Tolleaegse 
poisina köitsid mind mehhanismid, tehnika ja asjad oma 
koledas ilus. 

\question{Mis aastal see oli? Kaheksakümnendate keskel?}

1982. aastal läksin esimesse klassi, 
sealt loeme neli-viis aastat edasi, nii et 1986 või 1987. 

\question{Võrus selliseid arvuteid ei olnud, need tekkisid märksa hiljem. 
Teil pidi olema millegi poolest eriline kool, et suudeti arvutid 
välja rääkida. Või oli õpetaja eriline?}

Heiki Pettai\index[ppl]{Pettai, Heiki} oli suhteliselt noor õpetaja, kui ta 
meie kooli tuli, vist otse Tartu Ülikoolist. Ma täpselt ei tea, lihtsalt 
spekuleerin, et tal olid jätkuvalt aktiivsed suhted ülikooliaegse 
kontaktvõrgustikuga, näiteks Viljo Sooga\index[ppl]{Soo, Viljo}. Sealtkaudu võis infot liikuda ja ta 
võis olla õigel ajal õiges kohas, et sai Viljandi kooli midagi hankida. 

\question{Kas arvuteid kasutati ka õppetööks või käis seal ainult 
arvutiring? Kuidas nende kolme masina peal õpetada sai?}

Tagantjärele mõeldes nägi see hästi improviseeritud värk välja küll. 
Ta üritas ka keskkooliõpilastega väikestes rühmades tunde
läbi viia, sest ruumi ei mahtunud 
palju inimesi. DVK-2\index{DVK!DVK-2} masinatel oli olemas graafikakaardi \emph{slot}, 
aga graafikakaarti ühelgi sees ei olnud, jooksis ainult 
tekstipõhine režiim. Lapsed harjutasid, kuidas 
ASCII graafikas tekstiredaktoriga pilti joonistada. Nooleklahvidega ringi
sõites ja sümboleid vajutades sai joonistada 
ning õpetaja pani selle eest hindeid. Lisaks
lihtsamate asjadega tegelejatele oli seal
paar ägedamat, häkkerimat last. Kas Ivar Smolin\index[ppl]{Smolin, 
Ivar} oli juba seal olemas või tuli ta hiljem, kui see klass kolis
kutsekasse? Sealt kerkis järgnevate aastate jooksul teisigi tänapäeval tuntud inimesi, 
nagu Janek Hiis\index[ppl]{Hiis, Janek} ja Kaido 
Kärner\index[ppl]{Kärner, Kaido}.

\question{Kas sina joonistasid ka pilte või tegid midagi muud?} 

Alguses ei osanud ma muud teha, lihtsalt põnev oli arvutit 
katsuda. Jube äge oli klaviatuuri klõbistada ja vaadata, kuidas ekraanil toimub selle 
peale midagi. See oli püha emotsioon, mille nimel 
tasus istuda ja kannatlikult järjekorras oodata. Aeg-ajalt, kui rahvast oli vähem ja kas õpetaja ise või mõni 
edumeelsem õpilane teadis, millise flopiketta peal mängud asusid, 
sai Rottide\index{Rotid}-nimelist mängu mängida, mis oli sisuliselt 
Pacmani\index{Pacman} imitatsioon. Ka Snake\index{Snake} jooksis 
kusagilt flopi pealt. Vahepeal sai niisiis mängida, aga õpetaja üritas 
mängimise fooni loomulikult natuke alla suruda -- see oli rohkem nagu 
preemia, kui midagi asjalikku ära tegid. 

\question{Mis oli \enquote{asjalik}? Kas koodi kirjutasite?}

Mõned ägedamad vennad juba kirjutasid koodi ka. Ma ise DVK-2\index{DVK!DVK-2} peal veel koodi 
kirjutamiseni ei jõudnud. Selleks oli vaja 
rohkem vaba aega, kui lapsed parasjagu ei rüselenud 
liiga palju ja sai süveneda. Need, kes käisid 
lähemalt kooli ja said seal hilisematel õhtutundidel istuda, olid 
eelistatud seisus, sest mina elasin linnast väljas 
ja pidin bussigraafikuga arvestama. Niisama lihtsalt seal hilja õhtuni hängida 
ei õnnestunud. 

Koodimiseni jõudsin aastake või paar hiljem, kui see klass 
liikus suuremasse ruumi ja tulid pisikesed 
BKd\index{Elektronika!BK}\sidenote{Nõukogude kuueteistbitiste 
koduarvutite sari, mida huvitaval kombel (sest Viljandisse sattusid need teises järjekorras) 
peetakse varem mainitud DVKde 
eelkäijaks.}. Nagu tol ajal ikka, tekkisid need kusagilt Vene 
arvutitööstusest\sidenote{Selle arvuti töötas 1983. aastal välja Zelenogradis 
asunud asutus \begin{russian}НПО \enquote{Научный Центр}\end{russian}, 
toonase Nõukogude Liidu juhtiv mikroprotsessorite disaini 
keskus.}, neil olid hästi väikesed monitorid ja väike 
must kandiline aju või plokk, mis oskas kas makilindi pealt 
või siis võrgukaabliga ühendatult emaarvutist andmeid lugeda. Emaarvutiks oli pandud DVK-2\index{DVK!DVK-2}, mille 
ketaste pealt BKd said lugeda mingisuguse protokolliga, 
millest ma ei teadnud tol ajal ega ka tagantjärele midagi. Teadsin ainult, 
mis käske tuleb sisestada, et asjad toimiksid.

BK-l oli kuhugi püsimälusse sisse keevitatud 
BASICu\index{BASIC} interpretaator. Kui selle sisse lülitasid, siis 
esimese asjana tuli ette \emph{line number 10} -- hakka kirjutama. 
Nii et sisuliselt sai interaktiivselt BASICu käske kirjutada ja siis
hakkasingi esimest korda koodi kirjutama. 

\question{Mida need esimesed programmid tegid?}

Mis see teismelise koolipoisi kõige esimene programm ikka põnevat teeb? 
Kõigepealt oli BASICus käsk number 10, mis printis ekraanile midagi toredat, näiteks 
\enquote{loll}. Järgmine rida 
oli käsk number 20, mille peale oli \verb|goto 10|. Selline tore lõpmatu 
tsükli harjutus, aga sellele järgnesid kiiresti igasugused muud näpuharjutused. 
Seal oli juba graafika olemas, sai ekraanile jooni kuvada, ja siis esimene 
tsükli harjutus oli see, et sai joon kuidagi liikuma pandud ekraani ühest 
servast teise. 

\question{Ega ei saanud ju lihtsalt joont liigutada, eelmine positsioon tuli 
mustaga üle joonistada \ldots}

Jah, just, sealt hakkas vaikselt algoritmika, mis
sundis lapse aju algoritmiliselt mõtlema. Kõik vead paistsid kohe 
välja, kui olid midagi valesti mõelnud. 

\question{Kust see programmeerimisõpetus tuli? Kas õpetajalt või raamatutest?}

See tuli pigem kellegi käest, me lastena ei viitsinud väga
manuaale lugeda. Aeg-ajalt näidati küll, et näe, loe sealt. Need tekstid olid üldjuhul venekeelsed. Vaatad natukene tuima näoga nagu ahv kirjutusmasinat ja siis 
küsid ikka naabripoisi käest, et kuule, kuidas sa seda tegid. 
Mõningaid asju näitas õpetaja, teisi asju mõni teine targem laps -- niimoodi killuke siit ja sealt muudkui korjasid. 

\question{Ja esimene tunne ei läinud üle?}

Ei, üle see ei läinud. Psühholoogiline sõltuvus või 
vajadus arvuti taha istuda ja seal midagi teha oli kogu aeg olemas. 
Eks lastel mängib rolli mängudega jändamise võimalus. Mul oli motivatsioon kohale minna, et äkki saab mängida. Aga 
kuna vaikselt tekkis ka programmeerimise kihk, siis oli see 
piisavalt põnev, et kutsus sinna asju tegema. 

Põhiline oli see, et vahel lubas õpetaja lastel, keda ta rohkem tundis või 
usaldas, pisikest BKd\index{Elektronika!BK} kas koolivaheajaks võiks suveks koju viia. See
oli piisavalt väike, mahtus kotti. Aga sellega oli üks probleem: 
kuna salvestusseadet ei olnud, siis oli 
kaks varianti. Kas tõmbasid makilindilt programmi sisse ja selleks pidid olema
kõik vajalikud kaablid, juhtmed ja oskused, et sellega õigesti 
ümber käia. Või teine variant, et sul oli programmi \emph{printout} ja iga 
kord, kui tahtsid mängida, pidid kõigepealt kogu mängukoodi 
vigadeta sisse toksima. Ühesõnaga, tund-poolteist nägid vaeva ja 
järgmised tund-poolteist said mängida -- see oli päris huvitav kogemus. 
Tagantjärele mõeldes pidid need mängukoodid hästi ökonoomsed olema, et neid sai lühikese ajaga 
sisse toksida. 

\question{Kas teil 
häkkimist ei esinenud? Siin on räägitud\sidenote{Vt lk 
\pageref{sisu!ylikooli_root}.}, kuidas inimesed veel enne keskkooli ülikooli 
adminnide käest root-õigused ära võtsid.}

Mul otsest spetsialiseerumist või 
spinni ei tekkinud, et oleks kursi mõne konkreetse 
asja peale võtnud. Olen eluaeg olnud tarkvaraarendaja, ma ei ole 
läinud kuhugi riistvara häkkima ega muud säärast tegema. 

\question{Kas see ei ole huvi pakkunud?} 

Eks vahel on olnud uudishimusähvatusi, aga minu jaoks on kogu aeg 
olnud piisavalt atraktiivne tegeleda mõne uue laheda 
tarkvarakeskkonnaga. 

Ma olin matemaatika-füüsika
süvaklassis ja meil oli keskkoolis eraldi arvutitund. Tegime Jukudes\index{Juku}
dBase'is programmeerimisülesandeid, kus oli vaja programmeerida
andmebaasi- või tabelarvutuselaadseid asju. dBase on FoxPro 
sugulane ja päris äge. Oluline oli see, et 
kogu aeg oli midagi uut avastada, ja midagi muud pole mul motivatsiooniks vaja olnud. Pidevalt on uue 
asja avastamise rõõm. Andi Hektor\index[ppl]{Hektor, Andi} oli mu 
klassivend -- paljud teavad teda kui Eesti üht tuntumat füüsikut. 
Arvutitunnis olime sisuliselt kaks ärksamat pead, istusime ja 
õpetasime vastastikku üksteist ning tegime keerulisemaid asju. 

\question{Kas arvuteid muude ainete, näiteks matemaatikaga ka seoti?}

Mainisin just Andi Hektorit\index[ppl]{Hektor, Andi}, tema 
jaoks olid ilmselt füüsika ja keemia väga köitvad ained. Ta oli 
väga terav ja käis olümpiaadidel, pani neid järjest kinni. Tartu Ülikooli 
füüsikasse sai ta ilma eksamiteta sisse tänu sellele, et oli vahetult enne 
keskkooli lõppu vabariikliku füüsikaolümpiaadi võitnud. Tema puhul 
kindlasti see pool toimis. Minul kukkusid füüsika ja matemaatika kuidagi
loomulikult välja, sain vajalikud hinded kätte ja osalesin isegi 
Tartu Ülikooli matemaatikakoolis, mida keskkooli õpilastele 
kirja teel korraldati. Üritasin valmistuda Tartu 
Ülikooli sissesaamiseks, aga ikkagi rohkem informaatika motiiviga. Mul ei olnud
otsest huvi füüsikavalemitesse kaevuda, tugev emotsioon oli ikkagi 
arvutite vastu. 

\question{Kas sul muusika- või kirjandushuvi oli ka? Sa muusikamees oled ju olnud?}

Hobi korras mängisin jah kitarri. Nokkisin selle üles 
teismelisena isa kõrvalt, aga klassikalist muusikakooliharidust mul ei 
ole. Kõik, mida ma muusikast tean, olen ise üles korjanud. 

\question{Kas sellist mõtet, et üritaks Jukuga midagi lindistada või muusikat 
teha, ei tulnud?}

Nii kaugele ma tollal ei jõudnud. Mingisugused tüübid tulid ükskord 
arvutiklassi ja lasid päris äratuntava 
kvaliteediga Roxette'i muusikat läbi Juku. Aga see oli ka ainus selline 
moment. 

Üks moment meenub veel, olin siis keskkoolis. 
Lõpetasin keskkooli 1993. aastal, nii et see võis olla 1990ndate algul. Millal 
Bluemooni\index{Bluemoon} tüübid SoundClubi\index{SoundClub}\sidenote{SoundClubi hakkasid Ahti\index[ppl]{Heinla, Ahti} ja 
Jaan\index[ppl]{Tallinn, Jaan} kirjutama 1991. aastal Tartus füüsikat õppides 
ja see avaldati \emph{shareware}-litsentsiga 1993. aastal. Samas võisid 
selle versioonid ka varem ringelda.} arendasid? Keskkooli lõpuklassides 
tulid meil 286d -- ma ei mäleta, kas me 386 nägime. 
Igal juhul oli värviline graafiline keskkond juba mingil määral olemas ja 
SoundClub meile sinna kooliarvutitesse jõudis. Sellega sai küll mingit 
tehnomuusikat kokku tõstetud. Mitte et oleksin midagi hullult programmeerinud, oli lihtsalt lahe ja arusaadav kasutajaliides, kus sai
rütmiriffe ja asju kokku pandud, nii palju kui tol hetkel 
muusikalist arusaamist oli.\sidenote{Asko ei olnud ainus. Vennaskonna 
omaaegne hittlugu \enquote{Disko} on loodud SoundClubii abil ja, nagu tegijad on meenutanud, 
umbes samal meetodil.}

\question{Kas arvutiklassis hängiv seltskond oli muus mõttes ka sõpruskond 
või puutusite kokku ainult seal?}

Nii ja naa. Kujunes küll jah välja tuumik, kes sai omavahel ka väga 
hästi läbi. Keskkooli lõpus olid seal peale 
minu ja Andi Hektori\index[ppl]{Hektor, Andi} meist paar aastat 
nooremad Janek Hiis\index[ppl]{Hiis, Janek} ja Janek 
Palõnski\index[ppl]{Palõnski, Janek}. Keegi Kristjan, kelle teine 
nimi praegu ei meenu, oli ka arvutite peal päris kõva tegija. Samuti Raivo 
Kotov\index[ppl]{Kotov, Raivo}, kel on praegu Andrus 
Kõresaarega\index[ppl]{Kõresaar, Andrus} arhitektuuri- ja 
disainibüroo. Andrus oli ka mu klassivend.

\question{Kas sind keskkooliajal tööle ei võetud?}

Kahjuks või õnneks ei tulnud ette. 

\question{Kas Viljandis oleks tollal mõni koht olnud, kes 
oleks võinud programmeerija tööle võtta?}

Ma ei tea, et seal oleks tollal otseselt programmeerija väljavaateid 
olnud. Samas mingid arvutispetsialistid hakkasid küll juba ringi 
toimetama, sest järjest rohkem ettevõtteid võtsid arvuteid kasutusele. Tolleaegse nimega 
Eesti Telefon ja postiasutus panid juba IT-võrke püsti. Valdavalt vist Heiki 
Pettai koordineeriski neid asju. Võib-olla kusagil andmeid sisestada oleks 
heal juhul saanud, mis oleks olnud keskkooliõpilase jaoks okei. 

\question{Kas pärast keskkooli lõppu läksid otsejoones Tartu Ülikooli matemaatikat 
õppima?\index{Tartu Ülikool!Matemaatikateaduskond}}

Jah, kuna olin kooliajal õppeedukuse
poolest suhteliselt lohh, siis sain nibin-nabin ülikooli sisse. 
Õnneks oli siis üks madalama konkurentsiga aastaid. Olin esialgses pingereas joone peal täpselt viimane, kes sisse sai. 
Siis olid ju veel sisseastumiseksamid, kombinatsioon 
eksamihinnetest ja lõpuhinnetest. Sain sisse, tekkis jess!-emotsioon ja edasi hakkas ülikoolielu. 

\question{Seal vahepeal oli kummaline periood, kui Nõukogude sõjaväeteenistus 
läks Eesti kaitseväeteenistuseks üle -- kas sa sõjaväes ei käinud?}

Jah, see aken oli hästi lühike ja ma sattusin täpselt sellesse aknasse: 
nõukaaegne armeekord oli 
lagunenud ja sinna ei võetud juba mitu aastat, aga Eestis kehtestati üldine 
sõjaväekohustus 1993. aasta sügisel. Ma olin suvel ülikooli sisse 
saanud ja kõik enne sügist sissesaanud olid justkui vabad, eeldusel et nad ülikooli ära lõpetavad. 

\question{Naljakas aken oli jah! Mina pääsesin tervisega, aga meie kursuselt 
ei käinud keegi sõjaväes.}

Kusjuures mul oli endal selline suhtumine, et 
oleks täitsa okei olnud minna. Tegelesin sel ajal maskuliinsemate spordialadega, näiteks harrastasin karated, ja 
arvasin, et mis see sõjavägi siis ära ei ole -- kui vaja, siis 
teen ära. Aga ära see minu jaoks jäi ja hiljem olin selle üle õnnelik. 
Tollal oli sõjaväesüsteem 
väga lapsekingades ja ei oleks tõenäoliselt midagi 
väga meenutamisväärset olnud. Minu 
klassivendadest kaotas ajateenistuse tõttu oma elu kaks inimest. See 
näitab tollast taset, õnnetusi ja korralagedust oli veel päris palju. 

\question{Kui ma su juttu kuulan, koorub välja suhteliselt haruldane 
kombinatsioon: teeks sporti ja olümpiaadidel pigem ei käiks, aga samas 
programmeeriks isuga.}

Neid asju, millega ma paralleelselt tegelesin, oli tegelikult mitu. Võib-olla oli seetõttu raske otsustada, mille juurde jääda. 
Laulmise ja muusika mõttes mul kuulmist oli, aga määravaks sai see, et 
kuna mul muusikakooli haridust ei olnud, siis olin juba 
rongist maas. Kaalusin isegi 
kultuurikolledžisse minekut, aga olin ikkagi suhteliselt lahja 
vend. Spordiga sai tegeldud ja käisin ka kunstiringis, muuhulgas koos 
Kotovi\index[ppl]{Kotov, Raivo} ja Kõresaarega\index[ppl]{Kõresaar, Andrus}. Seda vedas Jakobsoni gümnaasiumis Grünbach\sidenote{Asko peab ilmselt silmas õpetaja 
Rein Grünbachi\index[ppl]{Grünbach, Rein}.}. Aga
lõpuks jäin ikkagi arvutite juurde, kuna tundus, et sellega läheb kõige paremini. Motivatsiooni mõttes need teised asjad 
ilmselt ei kinnistunud nii tugevalt kui arvutid. 

\question{Ja nii astusimegi sinuga koos 1993. aastal matemaatikateaduskonda. 
Esimesed kaks aastat tambiti meile haljast matemaatikat, kuidas see tundus?}

Ülikool oli minu jaoks omaette saaga. Tüütu oli see, et kõik pidid 
esimesed kaks aastat sama programmi õppima. Alguses ei saanud veel 
otsustada, kas minna informaatika, matemaatika või statistika suuna 
peale. Valikuvõimalus anti alles teise aasta keskel ja siis vaadati ka õunte 
pealt, kui hea sa oled ühes või teises asjas. Astusin Tartu Ülikooli 
matemaatikateaduskonda selle tõe pähe, et informaatikasuund on seal 
olemas, aga kas sinna saab, seda alguses ei teadnud. Selles 
mõttes oli ülikool minu jaoks paras \emph{challenge}, sest olin lohh edasi, vähemasti esimestel aastatel. Mul ei läinud
matemaatikaained just kõige paremini, kuna need ei olnud minu jaoks motivatsiooni 
põhipõhjus. Alguses oli päris palju keerulisi asju, näiteks matemaatiline analüüs I ja II.

\question{Matemaatiline analüüs I võttis ju lausa kolmandiku kursusest!}

Jah, see niitis rahvast korralikult, aga see ei olnud veel kõige hullem. Kõige 
hullem oli võibolla isegi algebra Mati Kilbi\index[ppl]{Kilp, 
Mati} väga karmi käe all, kõik asjad tuli korrektselt selgeks saada. Minu jaoks oli
algebra oma abstraktsuses kõige raskemini omandatav. 
Matemaatiline loogika seevastu, mida õpetas tollal levinud folkloori järgi 
üks karmimaid õppejõude Rein Prank\index[ppl]{Prank, Rein}, tuli 
lihtsasti, kuna oma mõttemudeli poolest haakus see programmeerimisega palju 
paremini. 

\question{See oli naljakas aine jah, otseselt keeruline ei olnud, aga ometigi
peeti raskeks.}

Ilmselt mingite inimeste jaoks oli see keeruline, aga meie, programmeerijate 
jaoks tuli see kuidagi loomulikult. 

\question{Vanemuise tänava õppehoones olid laiad aknalauad, mille peal istuti, 
sest kuskil mujal ei olnud istuda. Ja seetõttu värviti neid
regulaarselt üle. Sellele vaatamata oli alati kuskile 
sisse kratsitud \enquote{Prank on loll}.}

Meil oli rebaste vandes, kui sa mäletad, palju lauseid, mida me kõike 
tõotasime, ja üks neist oli \enquote{tõotan Prangile kõik eksamid ära teha 
hiljemalt seitsmendal katsel}. 

\question{Paljudel ilmselt nii läkski. Kas sinul oli 
programmeerimisunistus nii tugevalt silme ees, et ronisid ikkagi matemaatikast läbi?}

Lohistasin ennast läbi, aga kriisimoment oli olemas 
küll, olin tegelikult matemaatika tõttu väljakukkumise äärel. Kuna puhas matemaatika mind väga ei motiveerinud, siis veetsin suurema osa 
ülikooliajast 
arvutuskeskuses\index{Tartu Ülikool!Arvutuskeskus}, 
nõndanimetatud Väksu klassis\index{Tartu Ülikool!Liivi Õppehoone!Vase klass}\label{sisu:vase_klass}. Tolleaegse 
vask.ut.ee\index{vask.ut.ee} serveri VT100-terminalid olid ühendatud ühe 
VAX VMSi süsteemi taha. Sealt sain oma esimesed suuremad 
programmeerimise tuleristsed. 

Avastasin seal enda jaoks 
nii-öelda \emph{on the dark side} maailma ehk
mudamängud\index{Muda}. Need olid üheksakümnendate 
esimese poole internetipõhised arvutimängud, virtuaalsed maailmad, kus ei olnud 
midagi graafilist, kõik oli tekstipõhine. Kusagil jooksis server, 
kuhu võeti telnetiga ühendust, ja seal maailmas käis möll ja 
tagaajamine ja \emph{quest}'ide lahendamine. Sattusin üsna kiiresti ise ühe sellise mänguserveri 
programmeerimise meeskonda. Mäng on oma 
olemuselt päris keeruline elukas. Mängumootor peab 
seesmiselt maailma mudeldama, seal on tuhandete viisi 
ruume, liste ja muid asju. Algoritmikat, kuidas seda kõike 
struktureerida, on üksjagu ja see oli minu jaoks üks esimesi tõsisemaid 
C\index{C} programmeerimise kogemusi. 

\question{Kas sa selleks hetkeks juba oskasid Cd?}

Loengutes õpetati programmeerimist Pascali\index{Pascal} baasil, 
nagu sa mäletad. Selle korjasin suhteliselt kergesti üles, kuna see oli 
hea tüpiseeritud keel ja õpetati ka enam-vähem okeilt. Aga kusagil kripeldas, et mingid vennad panevad Cd. Mind
häiris, et ise ei saa. Ostsin eestikeelse C-õpiku, mis oli 
täiesti arusaamatu, kuna oli nii halvasti koostatud. Pealkirja ei mäleta, aga kaanekujundus oli kollane-punane. Selle asemel et näidata esimeses peatükis, kuidas Hello 
Worldi teha, hakati kohe baidi \emph{alignment}'i 
arutama. Kamoon, mis otsast te pihta hakkate! 

Ühel hetkel aga 
jõudis minuni info, et Kernighani ja Richie \enquote{The C Programming 
Language}\index{The C Programming Language}\sidenote{Dennis M. Ritchie, Brian 
W. Kernighan ja Michael E. Lesk. The C programming language. Englewood 
Cliffs: Prentice Hall, 1988. Samast raamatust on juttu ka
lk. \pageref{sisu:richie}.} on hea raamat, otse C-keele autoritelt. 
Igatahes kuskilt ma selle endale hankisin. 

\question{Neid liikus ilmselt Venemaal piraadituna ka.\sidenote{Vt lk.
\pageref{sisu:richie_vene}.}}

Mina ostsin täiesti legitiimse raamatu, mitte 
piraaditud väljatrüki. Mul on see vist siiani riiulis alles, kuigi kapsaks 
muutunud ja natuke teibitud. See oli metoodiliselt hea raamat: 
hakkas lihtsatest asjadest pihta ja läks lõpuks \emph{hard core}'ini 
välja. Neelasin selle läbi, tegin kõik harjutused ära ja 
sain C-keele selgeks. Kui üritada näppu peale panna 
raamatule, mis on mu karjääri kõige rohkem mõjutanud, siis see on 
see. 

Ühesõnaga, olin selle teose läbi protsessinud, enne kui jõudsin mängu 
progemiseni.

\question{Mutta vajus meil kursa pealt ka mitu inimest, kes enam 
Vaxu klassist ei väljunudki.}

Mõned jah, rohkem vajus sinna aasta vanemaid. Muda\index{Muda} 
oli tulnud aasta enne seda, kui meie sisse astusime, internetiga umbes 
ühel ajal. Internet tuligi koos paari 
pahega ja see oli üks peamisi. Nii et osa inimesi sattus mängu haardesse.

\question{Kas sina ei sattunud?}

Mul kestis see periood ainult paar kuud, see konverteerus suhteliselt ruttu 
programmeerimise entusiasmiks. 

\question{Kas toda serverit kirjutanud tiim oli Eestis või välismaal?}

Mudast\index{Muda} oli arendatud palju erinevaid versioone, kõik 
olid tollaste vabavaraliste litsentsidega, avatud koodiga, mida sai 
FTP-saitidelt tõmmata. Nendest arenes siin ja seal erinevate tiimide 
käes igasuguseid \emph{fork}'e\sidenote{\emph{Fork} on tarkvara-maailmas koopia tarkvara
lähtekoodist, mida arendatakse edasi sõltumatult algsest versioonist}. 
Kui \emph{fork}'ide hierarhiat 
joonistada, siis üks kuulsamaid ja levinumaid juur-fork'e oli 
DikuMUD\index{Muda!DikuMUD}, millest oli tehtud haru Merc, 
millest omakorda oli tehtud \emph{fork} nimega ROM. Raul Tölp\index[ppl]{Tölp, Raul} 
oli see, kes võttis ühe ROMi-põhise versiooni ja hakkas sellest Eesti 
oma Estonia-nimelist asja arendama. Tolle versiooniga mina liitusingi. Sai seda maailma edasi arendatud ja kohendatud, vigu parandatud ja muid asju tehtud. 

\question{Too oli ju huvitav kogemus, sest erinevalt ülikoolis õpetatavast 
tavalisest programmeerimispraktikast oli tegu meeskonnatööga!}

Jah, spontaanne meeskonnaelement tuli sisse. Laiem meeskond kirjeldas
mängumaailma, maailmafaile oli hästi palju: 
ruumid, kollid, mobprogid\sidenote{Lühikesed programmijupid, mis käivituvad mängija teatud tegevuste peale ja
võimaldavad kollidel neile reageerida. Samuti on olemas oprogid, mis võimaldavad 
sedasama objektide jaoks.}, propsid ja mis seal sees kõik elasid. 
Ägedamad koodivennad läksid järjest rohkem koodi 
sisse. Ühel hetkel üritasime mingi bandega hakata täiesti nullist, 
hoopis uutel alustel MUDi tegema. 

ROMi-põhine oli C-keele baasil ja võrguga suhtlus 
käis \verb|select| \emph{loop}'iga, deskriptorid olid \verb|select| 
listis, kus on oma piirangud -- maksimum tuhatkond 
\emph{connection}'it saab korraga püsti olla ja kõike protsessiti ühes 
\emph{single-threaded} tsüklis. Hakkasime Toomas 
Soomega\index[ppl]{Soome, Toomas}, kes oli tollal arvutuskeskuse süsadmin, 
ja Peeter Lauaga\index[ppl]{Laud, Peeter}, kes oli minust aasta hiljem ülikooliga 
liitunud, tegema täiesti uut arhitektuuri, mis oli C++\index{C++}-põhine 
ja \emph{multi-threaded}, et saada paralleelsus paremaks. 

Üritasime 
\verb|select| deskriptorite listist ja piirangutest lahti saada ning
tegime igasugust ägedat \emph{hardcore} värki, kus asendasime tekstiga 
maailmafailide sisse parsimise mingi kiirema ja efektiivsem asjaga. Kuna 
maailm koosnes tuhandetest ruumidest, siis serveri \emph{boot} võttis 
mitu minutit aega ja kui server \emph{chrash}'is, närisid mängijad küüsi, et 
kaua läheb ja kas saab uuesti sisse tagasi. Püüdsime selle asendada 
mingisuguse asjaga, kus maailmafailid olid eelkompileeritud 
mälu \emph{dump}'ideks, ja need \verb|mmap|'iga faili sisse 
lugeda, et saaks kohe hoobilt, murdosa sekundiga kõik püsti. 
Moproge hakkasime kirjutama dünaamiliselt lingitud libradeks kompileeritud 
failidena, mida sai jooksev server \verb|dlsym|'iga käigu pealt sisse linkida ja 
käivitada. Ühesõnaga, ajasime kontseptsiooni päris keeruliseks ning omandasime Unixi keskkonnas
korraliku \emph{hardcore} C ja C++ häkkimise oskuse.

\question{Kuidas see kõik sulle külge jäi? Lihtsalt õhust?} 

Korjasime vastastikku järjest üles ja toetasime üksteist. Toomas 
Soome\index[ppl]{Soome, Toomas} tegi otsa lahti ja
viskas palju ideid lauale just arhitektuuri osas. Meie 
Peeter Lauaga\index[ppl]{Laud, Peeter} korjasime ideed suhteliselt kiiresti 
üles ja hakkasime üksteist täiendama. 

\question{Teil pidi siis palju vaba aega olema, kas sa tööl ei käinud?}

Sellepärast mul oligi 
ülikoolis püsimisega raskusi. Veetsin suurema osa ajast 
arvutuskeskuses, vahel varajaste hommikutundideni välja, ja tihtipeale 
loengutesse ei jõudnud. Vahepeal oli mahajäämus vajalikes ainepunktides nii 
suur, et oleksingi võibolla kolmanda aasta keskel välja kukkunud, kui ma 
ei oleks ennast kätte võtnud. Mul käis mingisugune klõps, lapsest sai täiskasvanu. 

Ülikoolistress läks hästi tugevaks ja ühel hetkel jätsin 
mängud ja programmeerimise kõrvale ning hakkasin 
järjest laduma ülikooliõpinguid. Panin ühe 
semestriga 40 ainepunkti\sidenote[][-9mm]{See oli topelt 
tavapärasest semestri õpikoormusest ja nõudis ilmselt tõesti suurt tööd, sest 
kolmandal aastal väga palju lihtsaid aineid enam järel ei olnud.} jutti, 
et saada ree peale. Kui ma varem ei suutnud 
ennast kokku võtta, siis peale seda olen praktiliselt kogu aeg 
suutnud. Baka sai 1998. aastal lõpetatud. Hinded ei olnud head, sest käis sõna otseses mõttes toores tootmine, et kõik ainepunktid 
kätte saada. 

Kui magistrisse läksin (pidin minema tasulise kohale, sest 
hinnete tõttu ma riigieelarvelisele kohale ei pääsenud), siis seal tegin
hindelised ained maksimumi peale. 

\question{Miks sa magistrisse läksid? See ei olnud tol ajal 
vaikimisi valik.\sidenote{Toonane bakalaureusekraad kestis nominaalselt neli 
aastat ja võrdsustati hiljem haridustaseme mõttes praeguse magistrikraadiga.}}

Ma ei mäleta, kuidagi tekkis tahtmine. Magistris sain keskenduda 
teemadele, mis mind ennast huvitasid. Kui bakatasemel oli hästi palju 
sunniviisilist programmi, siis magistris valisin ise, mida teha. Tööle 
läksin 1995. aastal ja esimese palga häkkimise eest sain Tartu 
Ülikooli arvutuskeskusest\index{Tartu Ülikool!Arvutuskeskus}. 

Viljo Soo\index[ppl]{Soo, Viljo} andis operatsioonisüsteemide loengut\sidenote{Viljo 
loetud operatsioonisüsteemide ainel oli väga hea maine, seda peeti keeruliseks 
ning huvitavaks ja seal silma paista oli väga kõva sõna.} ja kord 
ühe loengu vahepausi ajal tuli minu juurde ja küsis, kas tahaksin natukene tööd ka teha. Ta oli ilmselt 
tähele pannud, et korjasin ehk
mingeid asju ladnamalt üles. Arvutuskeskuses oli süsadminnimise 
kõrval ja käigus vaja aeg-ajalt erinevaid asju arendada ja nii mind võetigi 
sinna programmeerijaks. 

\question{Tol hetkel oli võimalik juba ka arvutitega äri teha, kas see ei tõmmanud 
sind?}

Jah, neid tegijaid ümberringi toimetas, aga 
õnneks või kahjuks suutsin niisugustest ahvatlustest kõrvale hoida. Näiteks Alo Toom\index[ppl]{Toom, Alo} ja Ülo Säre\index[ppl]{Säre, 
Ülo} tegid alguses A ja Ü\index{A ja Ü|see{PC Expert}}, millest arenes välja PC 
Expert\index{PC Expert}. Ja mõnda aega eksisteeris niisugune tarkvarafirma 
nagu Codewiser\index{Codewiser}, mis on ka samast ettevõtete perest. Minu lapsepõlvesõber Alo üritas mind A ja Üsse tööle ahvatleda, 
aga tollal oli see rohkem tehnika \emph{support}'i ja konsultatsiooni 
firma, nad aitasid näiteks klientidel katkiläinud kõvakettaid taastada. Mind see eriti ei köitnud, mul oli emotsioon rohkem 
programmikoodi suunas. 

Samuti üritati mind meelitada meditsiinivallas tegutsevasse 
AtFuti\index{AtFut}, mille asutas Jaan 
Pruulmann\index[ppl]{Pruulmann, Jaan}. Teda kutsutakse ka
papa Pruulmaniks, sest Pruulmannide dünastia on suur ja 
lai, neil on seitsmevennaline perekond, kellest suurem osa on täna 
teada-tuntud IT-tegijad. Üks tuttav nimega Elvar Vask\index[ppl]{Vask, Elvar} \emph{alias} 
Cuprum\index[ppl]{Cuprum|see{Vask, Elvar}} sebis mind sinna 
töövestlusele ja ma vist küsisin liiga kõrget 
palka ning sel hetkel lõppes meie diskussioon ära. Muidu tundus, et klikk oli. 

\question{Kas sind akadeemiasse teadust tegema ei tõmmanud?}

Seda, et oleksin hullult tahtnud teadlaseks saada, ei olnud. Pigem tõmbas mind ikka
praktiline programmeerimine ja häkkimine. 

\question{Sa pidid kiusatusele kõvasti vastu seisma, nii mõnigi libastus ja jäi koolist kõrvale. 
Aga sina suutsid keskenduda?}

Selle mänguga seoses tekkis aeg-ajalt ikka unistusi. Arutasime tiimiga, et võib-olla õnnestub sellest midagi kommertsiaalsemat 
välja arendada, 
aga ilmselt ei saanud me tervikpilti piisavalt hästi 
kokku, et see tegevus kuhugi välja 
viiks. 

Oma karjääri alustasin ikkagi palgatöölisena. Kõigepealt töötasin Tartu Ülikooli arvutuskeskuses ja
vahetult enne baka lõppu, 1998. aasta talvel läksin
ProMedi\index{ProMed}, mis just samal kevadel liitus Magnum 
Medicaliga\index{Magnum Medical}, millest hiljem sai Magnum ProMed. Seega
olen kaudselt olnud seotud ka viimasel ajal palju kõneainet pakkuva 
Margus Linnamäe\index[ppl]{Linnamäe, Margus} tegutsemise algusaegadega. 

Selle ühinemise käigus kogu ProMedi seltskond koondati kohe. 
Sain jube mugava paketi: tegin paar kuud sisulist tööd ja siis tuli 
koondamisteade, mis tolle aja reeglite järgi tähendas seda, et kui koondatakse 
rohkem kui 20 inimest, siis makstakse neile nelja kuu raha. Sain ilusti 
oma baka ära lõpetada niimoodi, et ei olnud vaja 
muretseda lauale leiva saamise pärast. Suvel läksin tööle
Medisofti\index{Medisoft}, mis tegeleb tänaseni raviasutuste 
infosüsteemide arendusega ja kus tol hetkel oli valdavaks arenduskeskkonnaks 
Borlandi Delphi\index{Borland Delphi}. See oli ülikooli algusaegadest 
tuttava Pascali keele põhjal, aga täiesti uus graafiline keskkond, kus oli hästi mõnus 
\emph{desktop}-rakendusi teha. Seal tegelesin niisiis
\emph{desktop}-rakenduste ja andmebaaside kokkupanekuga. 

Mul oli veel paar alternatiivi, kuhu tööle 
minna. Tagantjärele mõtlen, et tegin huvitava otsuse -- 
teistpidi otsustades oleks mu elu võibolla praegu teistsugune. Üks variant oli Medisoft ja teine see punt, kus toimetasid kursusekaaslased Rene Prillop\index[ppl]{Prillop, Rene} 
ja Mati Muts\index[ppl]{Muts, Mati}; seesama 
tuumik, kes hiljem Eesti-poolset PlayTechi\index{PlayTech} asutasid ja 
tegid. Nad pakkusid kõrgemat palka, aga ma kaldusin tollal
alalhoidlikkusele ja Medisofti pakkumine tundus parem, kuna seal olid kindlama meelega
vanemad tegijad ja stabiilsemad asjad. Läksin 
selle peale välja. Alguses oli hästi huvitav, sest õppisin
iga päev midagi uut -- mul on kogu aeg millegi uue saamise 
motivatsioon. Olin Medisoftis 1998. aasta suvest 1999. aasta suveni, 
kui mind kutsuti tööle Küberisse\index{Küber}. 

Küberis hakkasid juba naljakamad ja põnevamad 
lood, mis võibolla haakuvad tänapäevaga rohkem. Sattusin 
sinna täpselt sellel kuumal momendil, kui põhituumik -- Tarvi 
Martens\index[ppl]{Martens, Tarvi}, Aarne Ansper\index[ppl]{Ansper, Arne}, 
Viljar Tulit\index[ppl]{Tulit, Viljar} ja Monika Oit\index[ppl]{Oit, Monika} 
--- oli koos ja parasjagu visioneeriti DigiDoci ja muud põnevat. Krüptoteadlaste kambast olid seal Ahto 
Buldas\index[ppl]{Buldas, Ahto}, Helger Lipmaa\index[ppl]{Lipmaa, Helger} ja Jan 
Villemson\index[ppl]{Villemson, Jan}, samuti
Meelis Roos\index[ppl]{Roos, Meelis}, Peeter Laud\index[ppl]{Laud, 
Peeter} ja Ville Hallik\index[ppl]{Hallik, Ville}. Niisugune ajutrust, kõik tuntud korüfeed! 
Esimese hooga sattusin kohe C++-s\index{C++} programmeerima, 
Visual Studios DigiDoci kliendi prototüüpi. 

Teadlased leiutasid ajatemplite linkimise skeeme 
räsiahelate otsas ja samuti arendati digitaalse notari kontseptsiooni. Tollal me veel ei teadnud, et kümme aastat hiljem 
hakatakse seda nimetama \emph{blockchain}'iks. Iga natukese aja tagant tuleb keegi 
välja järjekordse teooriaga, kes on Satoshi Nakamoto\sidenote[][-3mm]{Bitcoini leiutaja või leiutajate poolt kasutatav pseudonüüm.}. Ka \emph{blockchain}'i puhul otsitakse ikka veel selle leiutajat. Paar aastat 
tagasi\sidenote{Jutuajamine Askoga toimus 2020. aasta jaanuaris.} tuli üks USA jurist 
lagedale teooriaga, et see oli Helger Lippmaa\index[ppl]{Lipmaa, 
Helger}\sidenote{Keegi Justin Sobaje 2018. aastal sellise teooriaga, 
mida Helger visalt eitas, tõesti välja tuli.}. Nii et meil on oma 
Satoshi Nakamoto olemas. Ahto Buldaselt\index[ppl]{Buldas, Ahto} võeti ka 
selle jutu peale intervjuu ja tema muigas ning ütles, et põhimõtteliselt oleksime me kõik 
võinud Satoshid olla. 

\question{Lähme korraks üheksakümnendate juurde tagasi. Kuidas sul arvuti ja muu 
maailma tasakaalustamine käis? Matemaatikateaduskonna ja arvutuskeskuse 
ümber käis ju äge seltsielu.}

Arvutuskeskuses istus ööde viisi seltskond koos 
ja toimetas oma asju. Üks püsiv ja stabiilne kaader olid 
Muda\index{Muda} mängijad. Osa inimesi \emph{chat}'is 
tolleaegsetes varajastes telnetipõhistes jutukates -- 
tänapäeval on meil nende asemel Facebook ja muud kohad. Ja mingi seltskond tegi asjalikumaid asju, näiteks õpingutega seotult. 

\question{Tehnikaülikooli arvutuskeskuse kohta on 
öeldud, et see oli nagu pooleldi klubi, kus käisid ka need, kes enam ammu ülikoolis 
ei õppinud.}

Jah, ja Tartu Ülikooli arvutuskeskuse kohta võis üheksakümnendatel 
enam-vähem sama öelda, sest seal ei olnud ainult üliõpilased. Sealt 
käis läbi igasugust rahvast ka väljastpoolt ülikooli ning oli tekkinud 
kamp, kus kõik tundsid enam-vähem kõiki, kes seal stabiilselt käisid. 

\question{Tartus oli vähemalt üks selline kogunemiskoht ka Tähetorn.}

Neid kohti oli isegi rohkem ja tuumikud puutusid omavahel 
kokku ka, aga kokkuvõttes oli mitu gravitatsioonikeskust. Tähetorn oli 
võibolla isegi üks suletumaid ja väiksemaid, ma ise sinna
eriti ei sattunud. Aga lisaks Tartu Ülikooli 
arvutuskeskusele\index{Tartu Ülikool!Arvutuskeskus} oli 
füüsikahoone\index{Tartu Ülikool!Füüsikahoone}, kus Ville 
Hallik\index[ppl]{Hallik, Ville} ja Otto Teller\index[ppl]{Teller, Otto} 
vägesid juhatasid ja asju kontrolli all hoidsid. Seal käisid
füüsikud ja ka sotsiaalteaduskonna 
tudengid. Veel üks
gravitatsioonikeskus oli ülikooli peahoone kõrval kunagine Marksu 
maja\index{Tartu Ülikool!Marksu maja}, mille all oli üks punkt, kus ma ka
ise mingil perioodil üpris tihti viibisin. Keemiahoone\index{Tartu 
Ülikool!Keemiahoone} all oli ka midagi, aga see toimis vist lühemat aega ja
oli natuke ebamäärasem. 

\question{Kui paljud arvutuskeskuse seltskonnast oli matemaatikud? Kui Anne Villems\index[ppl]{Villems, Anne} oma esimesed 
veebmasterite kursused tegi, käis seal kuuldavasti igasugust rahvast psühholoogidest 
usuteadlasteni.}

Teistest teaduskondadest tean ma üksikuid inimesi, nii et ma
statistilist pilti anda ei oska. Kuna arvutuskeskuse 
palgaline kaader oli valdavalt ikkagi 
matemaatika-informaatikateaduskonnaga tihedalt seotud, siis paratamatult oli 
selle teaduskonna tudengkond seal ka kõige rohkem esindatud. 
Praegu tuntud nimedest oli näiteks Jaanus Lillenberg\index[ppl]{Lillenberg, 
Jaanus}, kes praegu juhib ERRis IT-vägesid, mingil perioodil seal hästi 
aktiivne külaline\sidenote{Jaanuse seiklustest Liivi tänaval loe lähemalt 
lk\pageref{sisu!jaanus_liivi_tn}.}. Tema teaduskondlik taust oli vist
midagi muud, ta ei olnud IT-vallast. 

\question{Mida sa praegu teed?}
Praegu olen ettevõtja. 

\question{Vist juba üle kümne aasta?}

Selgelt ettevõtjaks sain ma pärast Skype'i\index{Skype}. Enne 
seda olid aeg-ajalt mingisugused unistused ja visioonid ning
projektilaadsed eksperimendid, aga esimese OÜ, Mooncascade'i\index{Mooncascade} asutasin oma väriseva 
käega paar kuud enne Skype'ist lahkumist, 2007. aasta septembris. 
Sellega olengi kõige rohkem seotud olnud. 

Mooncascade'i visioon tekkis mul juba Skypest lahkudes, aga 
eestlastest Skype'i asutajate punt, Ambient Sound 
Investmenti seltskond, kutsus mind kohe jutule, sest neil oli 
käsil üks inkubaatorilaadne eksperiment, kus jooksis neli 
erinevat projekti. Nad kutsusid mind ühte nendest projektidest vedama 
ja kuna pundis oli ka Ahti Heinla\index[ppl]{Heinla, Ahti}, kes 
täna veab Starshipi, siis minu jaoks oli piisavalt motiveeriv panna Mooncascade seniks riiulile. Nii me tegimegi paar aastat 
ühte suhteliselt jõhkrat andmekaevelaadset projekti, mis äriliselt 
lõpuks ikkagi lendu ei läinud ja sai ära konserveeritud. Seejärel sai Mooncascade 
reaktiveeritud. Mooncascade alustas aktiivset tegutsemist 2009. aasta lõpus või 2010. aasta 
alguses ja tegutseb tänaseni. 


\chapter{Margus Sutt}
\index[ppl]{Sutt, Margus}

\question{Kust sa pärit oled?}
Ma olen Tallinnast.
                 
\question{Aga kuidas sul seal Tallinnas nende arvutitega oli, kuidas arvutid 
sinu juurde said ja sina arvutite juurde?}

Mul niimoodi pool juhuslikult juhtus nii, et kui mul oli aeg kooli minna, siis 
meie pere parajasti kolis. Ja kolis sellise kooli nagu Tallinna 3. 
Keskkool\index{Koolid!Tallinna 3. Keskkool} piirkonda. See on tänapäeval 
Lilleküla gümnaasium\index{Koolid!Lilleküla gümnaasium|see{Tallinna 3. 
Keskkool}}. Ehk, nagu siin juba mitu korda on mainitud, selles koolis oli 
selline tüüp nagu Jaak Loonde\index[ppl]{Loonde, Jaak}.
                 
\question{Jälle Jaak!}

Jah, ma sattusin sinna kooli, esimesed kaheksa aastat põhikooli nagu ei 
juhtunud veel midagi. Ja ka siis, kui oli keskkooli minek, tulid mingisugused 
katsed ja konkursid ja meil tehti sellest aastast alates, kui ma keskkooli 
läksin, juurde kolmas klass (muidu oli kaks keskkooliklassi), mis oli siis 
nii-öelda informaatika-matemaatika eriklass. Aasta oli siis 1985 või 1986.

\question{Lausa informaatika ja matemaatika? Seda vedas siis 
Jack\index[ppl]{Jack}?}

Jah, et see komplekt sinna juurde tuli, organiseeris Jack.

\question{Kas ta oli siis klassijuhataja ka?}

Ei, klassijuhataja ta ei olnud, ajas asju. Kuskilt haridusministeeriumist või 
mis iganes asutused tollal olid, oli ikkagi vene aeg veel, ta igatahes sai 
selle klassi. Ega see klass nüüd ajaliselt vist kaua ei püsinud,  kaks 
komplekti oli seda kindlasti aga kolmandas ma enam väga kindel ei ole, kuna ma 
tulin sealt juba ära. Jack ei pidanud seal koolis enam väga kaua vastu ja peale 
teda see asi hääbus. 

Aga, jah, Jackil oli seal koolis selline tore külmkastide rivi nagu 
MIR-2\index{Arvutid!MIR-2}, kus olid perfolindid ja perfokaardid ja\ldots

\question{See oli tal koolis lausa?}

See oli tema klassiruumis, ruume oli tal muidugi mitu. Tal oli lisaks 
klassiruumile ka mingi raadioruum, mis pärast läkski nagu päris raadioruumiks. 
Meie klassi tegelased ehitasid. Mitte mina, aga põhiaktivist oli selline 
tegelane nagu Andrus Tamboom\index[ppl]{Tamboom, Andrus}. Ta ehitas kooli 
raadiovõrgu, või nii-öelda taastas, sest mingisugused juhtmeid olid vanast 
ajast  seinas, aga ta kohendas seda ja ajas kuidagi käima.

\question{Kuidas perfolindi tingimustes arvutiõpe praktiliselt välja nägi?}

Ega seda õpet seal nüüd väga palju ei olnud. Mingi pool aastat võib-olla oli, 
kui me selle MIR-2-ga\index{Arvutid!MIR-2} tegelesime, edasi läks juba natuke 
advaantsimate asjade juurde. Aga mingisugust parabooli sai ekraani peal 
joonistatud. Perfolindiga oli see, et  oli lugeja,  sa pistsid lindi sinna 
sisse ja kuskilt tuli siis mingi käsk vajutada, ja ta hakkas seda linti hästi 
kiiresti läbi vedama.

\question{Ja kooliõpilased perforeerisid linti või see usaldati kellegi teise 
kätte?}

Arvuti perforeeris ise ka, tal olid mõlemad, nii sisend kui väljund, olemas.

\question{Trükimasin, toksid sisse, tema teeb perfolindi?}

Ja pärast on võimeline seda lugema. 

\question{Kust sul see mõte sündis, et võiks just sinna sellesse paralleeli 
minna, kus informaatikat õpetati?}

Ma olen selle peale mõelnud, aga ega  ei mäleta, miks. Võib-olla vanemad 
torkisid, võib-olla oli endal mingil määral huvi. Matemaatikaga mul probleeme 
ei olnud, iseenesest. Käisin olümpiaadidel, keskkoolis kindlasti juba, ma ei 
tea, kas ka põhikoolis juba, mata nagu väga ei häirinud.

\question{Olümpiaadidel käimise puhul on  \enquote{mittehäirimine} vist natuke 
pehmelt öeldud. Tavaliselt on arvutiõppes kaks osa: see, mis tunnis räägitakse 
ja nii-öelda programmiväline tegevus, mille käigus ise pusitakse. Kuidas teil 
selle vahekorraga oli?}

Selles mõttes tuleb jälle Jack'i\index[ppl]{Jack} tänada. Tema õpetamismeetod 
oli ikka kardinaalselt erinev, kui nüüd nii-öelda klassikaline. Matemaatikat 
tema raamatust ei õpetanud, tal olid mata jaoks tehtud oma töölehed, kus oli 
muu hulgas üritatud tekitada mingeid siirdeid ka teistesse ainetesse. Füüsika, 
eks ole, kõige lihtsam. Keemiat ma ei mäletagi, aga eesti keelt oli näiteks 
kuskil mingites kohtades mainitud. 

\question{Ta õpetas matemaatikat ka?}

Jah, meie klassile ta õpetas matemaatikat ka, järgmisele aastale enam ei 
õpetanud.  

\question{Töölehed ei sobinud?}

No seal oli igasuguseid probleeme. Järgmist aastakäiku ma mainin sellepärast et 
seal õppisid Priit Kasesalu\index[ppl]{Kasesalu, Priit}, Mikk 
Orglaan\index[ppl]{Orglaan, Mikk} (kes on ka juba teada-tuntud) ja Janno 
Ossaar\index[ppl]{Ossaar, Janno}. 

Aga selle õpetamise juurde veel tagasi tulles, siis Jaagu\index[ppl]{Loonde, 
Jaak}  õpetamismeetod oli, jah, suures osas sihukene vette viskamine. Vähemalt 
arvuti poolelt. Et, näe, siin on arvuti, tegelege. Muuhulgas õnnestus tal 
näiteks koos klassijuhataja Tiiu Neemega\index[ppl]{Neeme, Tiiu} viimases 
klassis  minule ja  veel paarile välja rääkida selline asi, et me osades 
tundides ei pidanudki käima. Tegime mingisuguseid semestrite või trimeistrite 
(mis iganes nad seal parajasti olid), lõpus töid ja arvestusi. Selline 
eriprogramm.

\question{Et sellist eriprogrammi võimaldada, pidi nii-öelda akadeemiline 
jõudlus tasemel olema?}

Ilmselt ei olnud, jah sellega probleemi. 

Aga muu arvutitegevusega oli nii. Kõigepealt tulid MSX-id\index{Arvutid!Yamaha 
MSX}, seal Luise tänava ÕTK-s\index{Tallinna Oktoobrirajooni 
Õppetootmiskombinaat}\sidenote{Tallinna Oktoobrirajooni Õppetootmiskombinaat.}. 
Seal sai mingiaeg klassiga käidud, ilmselt hakkasin pärast seal ka ise käima. 
MSX-idega oli tore näiteks see, et Jackil\index{Jack} õnnestus need 
vaheaegadeks (suvevaheajaks eriti, ma teisi vaheaegu ma ei mäleta, kas oli) 
kolmandasse kooli nii-öelda laenata.
          
\question{Soh. Sai nad siis kindlasse kohta ära lohistatud!}

Jah, ligipääs oli hulga parem, sest kolmandas koolis\index{Koolid!Tallinna 3. 
Keskkool} mingi aeg mul oli  lisaks Jacki ruumi võtmele ka kooli välisvõti. Ehk 
keskkooli ajal tekkis võimalus suvevaheajal ööpäev ringi arvutis olla.

\question{Mis te tegite nende arvutitega? Jutte kuulates tundub olema kahte 
liiki inimesi: need, keda huvitas rohkem programmeerimine ja need, keda tõmbas 
mängude poole. Kumb sul rohkem domineeris?}

Esialgu kindlasti mängimine. Väga täpselt ei mäleta, aga 
Yamahal\index{Arvutid!Yamaha MSX} olid head ilusad mängud, eks need meelitasid 
oma graafika ja sellega, et võrreldes MIR-iga oli ta ikka hoopis teine maailm. 
Isegi võrreldes Agatiga\index{Arvutid!Agat}. Ma ei mäleta, kas  keskkooli 
esimesel või teisel aastal tekkis meile üks oma kooli\index{Koolid!Tallinna 3. 
Keskkool} Agat, mis kogu aeg koolis oli. 

Progemise poole pealt oli mul Agatist kindlasti rohkem kasu, sest seal käis 
selline tegelane nagu Tarmo Mamers\index[ppl]{Mamers, Tarmo} vahest  asja 
uurimas ja üle tema õla kiibitsedes õnnestus mul ka üht-teist-kolmandat 
omandada.

Agati peal oli Basic\index{Keeled!BASIC}, aga sai muuhulgas ka juba assemblerit 
vaadatud.

\question{Räägi korra sellest Agatist palun. Ta oli Apple II kloon?}

Ta oli Apple II\index{Arvutid!Apple II} kloon, mille sees  oli originaal 
Apple'i \emph{chip}, kust oli info maha kraabitud. Vähemalt see konkreetne oli 
selline.

\question{Sest mina mäletan Agati pealt mingeid veidrusi, mida ma ei usu, et 
Apple'l oleks olnud. Tundus, et see OS oli seal Nõukogude oma?}

Vot ei mäleta nii täpselt, sest ega ma päris Apple'it ei olegi puutunud, mul ei 
ole võrdlust. Ma tean, et Tarmol\index[ppl]{Mamers, Tarmo} olid Apple pealt 
mingid väljatrükid BIOS-i ja BIOS-i \emph{call}-ide kohta ja ta üritas seda 
Agati peal rakendada.
                 
\question{See on juba oluline teadmine, et Apple-i \emph{call}-id võiksd 
töötada!}

Jah, muu hulgas oli mul ka näiteks selline nii-öelda huviprojekt, et sai 
üritatud Apple peale Norton Commander-it\index{Norton Commander} kirjutada. 
Asmis, loomulikult.
                 
\question{Huvitav, minul oli samasugune projekt Juku peal\ldots}

See tee vist on, jah, aastate jooksul kõigil läbi käidud. 

\question{See toobki järgmise küsimuse juurde, et kust teile  ülesanded tulid. 
Arvutiga on ju see, et kui sa valid lahendamiseks liiga lihtsa ülesande, siis 
see ei ole huvitav. Aga kui sa valid liiga keerulise ülesande, siis sa algajana 
jooksed ennast enne sodiks,  kui üldse mingit  lootusekiirt paistma hakkab. 
Kuidas teil see oli, kas Jack juhendas või ise mõtlesid või kuidas see käis?}
                 
Ei, Jack\index[ppl]{Jack} selles mõttes küll väga ei juhendanud. Tema antud 
programmid olid pigem  matemaatikateemalised. Ja ega Jack ilmselt ei 
küündinudki sellise juhendamiseni. Võib-olla oleks küündinud, aga ei suvatsenud 
või ei pidanud vajalikuks. Tuli ise vaadata, et mida sa teed. 

Mängudega seoses, eks ma mingeid spraite või mingeid selliseid asju ikka 
püüdsin liigutada. Ja mida ma mäletan,  et ma ka ehitasin, oli see, et võtsin 
sinusoidi ja nii-öelda nihutasin ta ruumi. Kui sa tekitad mõlema koordinaadi 
suhtes  väikse nihke, eks ole, siis tekib sihuke ruumiline efekt. Ja kui sa 
paned programmi tagant kustutama ja eest uuesti joonistama, on tulemuseks 
sihuke nagu tänapäeva \emph{screensaver}. Aga tol ajal oli see huvitav.
                 
\question{Programmeerimise mõttes tore ülesanne. Kas arvuti-huvi juurde käis ka 
mingisugune spetsiifiline muusika- või kirjandushuvi? Päris mitmed on rääkinud, 
kuidas nad on Asimovi ja Gibsoni peal kasvanud selle koha pealt.}

Ma olen lugenud küll fantastikat ja \emph{sci-fi}-id ja fantaasiat aga ma ei 
tea, kui palju see nüüd arvutiga seotud on. Pigem on nad selles mõttes nagu 
konfliktis, sest  võitlevad mõlemad aja pärast. Enne arvuti-aega ma lugesin 
väga palju ja väga kiiresti, kõik need seiklusjutud maalt ja merelt. Mis  kätte 
sai, kõik sai läbi loetud. Aga kui arvuti tuli,  ma enam vahepeal nii palju ei 
lugenud. Kuigi nüüd siin, umbes 10 aastat tagasi, ma võtsin kätte ja lugesin 
läbi enamuse  McCaffrey-st\sidenote{Margus peab ilmselt silmas Ameerika-Iiri 
ulmekirjanikku Anne McCaffrey-d. Juba ainuüksi tema Loheratsurite sarjas on 23 
romaani ja see on vaid üks paljudest selle kirjaniku poolt loodud 
maailmadest.}. Alustades lohelugudest, neist kaks on isegi eesti keeles olemas.

\question{Mõned on rääkinud, et juba keskkooli ajal kippus töö tegemiseks 
minema. Sul ei olnud nii?}
                 
Ei, töö tegemiseks ei läinud, kuigi  sidemed esimese töökohaga juba tekkisid.

\question{Mis see esimene töökoht oli?}

Töökohta õige paremini iseloomustab Raivo Rebase\index[ppl]{Rebane, Raivo} 
nimi. Ma ei tea, kuidas tema Jackiga seotud oli, aga igatahes kuidagi ta oli. 
Mingi hetk oli ta Jacki juures kohal ja põhimõtteliselt otsis jüngreid. 
Nii-öelda \emph{head-hunting}.

\question{Vaata, kus! Tänapäeval keskkoolis vist päris ei käida otsimas!}

Ma ei tea, kuidas see niimoodi. Eks ta vist plaanis  oma arvutifirmat teha, tol 
hetkel tal seda veel ei olnud. Ta tegutses Küberis\index{Küber} sellise 
härrasmehe nagu Raul-Roman Tavasti\index[ppl]{Tavast, Raul-roman}  juures. Neil 
oli ka vist mingi firma seal Küberi kõrval. Ma mäletan seda nii palju selle 
pärast, et seal ma sain tõenäoliselt esimest korda PC-d katsuda. Peale seda, 
kui see värbamine aset leidis, saime me paari-kolmekesi koolist hakata  käima 
kuskil Küberi majas (ma ei mäleta, mis hoones), kus olid mingid PC-d. Vist 
lausa 386-d, kus oli neli mega mälu. Ehk siis, kui ma tegin ilusti selle 
\emph{boot floppy}, kus oli mäludrive peal, siis ma sain ikka väga palju 
kõvakettaruumi, kus jooksutada Turbo C \emph{editor}-i.
                 
\question{Ohoh, räägi lähemalt, Turbo C on ikka juba meeste vahend, kuidas sa 
selleni jõudsid?}
                 
Kusjuures jälle ma  mõtlesin, et kuidas ma selle C\index{Keeled!C} juurde 
jõudsin, sest Pascalit ma ei ole kunagi õppinud, aga kuidagi mul tekkis see 
Turbo C. Agati peal teda ei saanud olla, ilmselt seal Rebase kaudu ta kuidagi 
tuli. 

Kas Rebane või\ldots Rebane kindlasti, aga ma ei tea, millal, pistis mulle 
mingisuguse kopeeritud Kernighan-Ritchie\sidenote{Kuulus valgete kaante ja 
sinise C-ga raamat, vt. \pageref{sisu:richie}.} pihku. Selle ma neelasin siis 
mõne nädalaga ilmselt läbi, väga kaua sinna ei läinud.  

\question{Kas MSX-i ja Agati pealt C peale üle minek keeruline ei olnud? 
Pointerid ja värk ja igasuguse turvavõrgu puudumine?}

Ma ei mäleta, kuidas see see täpselt oli. Seal Rebase juures ma käisin ja mingi 
hetk ta lõi  Küberist lahku. Järgmine koht oli Liivalaia tänaval selles majas, 
kus praegu on Swedbank\sidenote{Liivalaia 8, Tallinn}.  Seal 12. korrusel, kus 
praegu on nii-öelda ülemuste korrus, oli arvutuskeskus, ETK või ETKVL või 
mingisugune\sidenote{Margus peab ilmselt silmas sel aadressil asunud EKE 
Projekti nimelist asutust.}. Seda vaadet ma nautisin üheksakümnendatel, ilus 
vaade oli. Pärast õnnestus peaaegu samasse kohta tagasi kolida, üheksandele 
korrusele. Sinna  sai Rebane ruumid. Arvutuskeskuses oli ka mingisugune suur 
nii-öelda kast-arvuti, millega meie õnneks kokku ei puutunud, meil olid  oma 
PC-d kus, kus muuhulgas sai hakatud näiteks ka Unixiga\index{OS!Unix} tegelema.

\question{Kuidas te PC peal Unixit tegite?}

SCO\index{OS!SCO UNIX} ja BSD\index{OS!BSD} olid sel ajal olemas.

\question{Aga kuidas need tol ajal Eesti Vabariiki jõudsid?}

Ma ei tea, kuidas nad jõudsid, see ei olnud minu teha. Kuidagi oli SCO, kuidagi 
oli BSD. Äkki nad olid isegi kuskilt ostetud, SCO-l olid  ikka nagu originaal 
kirjade plaadid mingist hetkest, ma mäletan. Flopid ikka, flopid pidid olema, 
plaadid tulid hiljem. 

\question{Aga millega te seal tegelesite? Firmal pidi ju äri olema?}

Selle äriga oli  alguses kehvasti. Ega ma näiteks ei mäleta, millal ma  palka 
hakkasin saama. Alguses ei olnud väga äri, aga mingi hetk nagu läks käima. Ja 
käima läks  huvitav äri, radarid. Radaritega meil oli koostöö Vene firmadega, 
ühe seltskonnaga Peterburist. Nemad tegid  radarile riistvara, tegid kaardi, 
mis läks PC sisse, ja meie ehitasime sinna peale softi.
           
\question{Radarid on ju sõjaväe ja saladus ja puha?}      

Ei, sõjaväge me väga ei puutunud, me tegime nagu tsiviilsuuna peale. Pulkovo 
radarijaamas sai näiteks korduvalt käidud.

\question{Teie soft siis võttis mis iganes signaali ta radari käest sai ja 
joonistas kaardi peale mummud?}

Jah. See käis alguses BSD peal. Ei mäleta, kui kaugele me SCO-ga jõudsime, 
võib-olla ta alguses käis isegi SCO peal. Aga BSD oli kindlasti vahepeal ja  
lõpuks läksime Linuxi\index{OS!Linux} peale, kui see tekkis ja oli juba niivõrd 
kobe, et sai kasutada.
                 
\question{Alles äsja joonistasid ekraani peale siinust ja nüüd loed radari 
pealt signaale, siin tundub mingisugune lünk olema? Ülesande keerukus on ju 
palju suurem?}

Ega seda  kõike mina ei teinud, eks ole. Seal oli ikkagi grupp inimesi taga.

\question{Kui suur see grupp oli? Kümmekond? Kakskümmend?}

Alla kümne,  ei olnud väga suur. See meie firma ei läinud kunagi väga suureks. 
Seal olime mina,  Raivo Rebane\index[ppl]{Rebane, Raivo}, Mart 
Rüütel\index[ppl]{Rüütel, Mart} (kes on vist praegu ka veel seal). Firma nimi  
on praegu R-Süsteemid\index{R-Süsteemid}, mingi aeg oli ta Virumaa 
Tiivad\index{Virumaa Tiivad|see{R-Süsteemid}}, sest ta tegeles natuke ka 
lennundusega. Mingi aeg oli meil tööl üks lennundusfänn ja  muu hulgas sai 
korraldatud mingisugune väikelennukite ülelend või üritus,  mille raames mul 
õnnestus näiteks sõita Kuressaare lennujaamast Viljandi lennujaama mingi Piperi 
peal. Muidugi, kuna need, kelle lennukiga ma lendasin, olid Saksamaalt, siis 
ega nemad ka ei teadnud, kus lennujaam seal täpselt on ja siis selle Viljandi 
lennujaama otsimisega oli natukene tegemist. Et \enquote{kuhu me nüüd siis 
maandume?}
                 
\question{Sihuke värk! Kes see klient oli? Needsamad venelased?}

See soft oli vist mingi aeg Tallinnas ka kasutusel. Softis olid mõlemad, nii 
primaar- kui sekundaarradar. Primaar on see nii-öelda \enquote{loll} radar, 
tema signaal põrkab lihtsalt kuskilt tagasi või siis ei põrka. Ja 
sekundaarradar on siis see info, mille lennuk  ise välja saadab oma  
transponderiga.
                 
\question{Radari juures on tihtipeale see oluline asi, et ta peab töötama. 
Kuidas te seda tegite, et teie soft robustne oli ja töötas?}

Eks seal oli mitu kihti nagu loogikat peal jah. Aga radaritele sai isegi 
mingisuguseid vahvaid maatriks-algoritme kasutatud. Üks tegelane, kelle nime ma 
paraku ei mäleta (ta eesnimi võis äkki Kaido olla?) tegi nendest lausa TPI 
lõputöö. Ja nagu nende lõputöödega ikka vahest on, et kuskile pannakse mingi 
\emph{catch} sisse lootuses, et niikuinii keegi ei loe. Ja nende algoritmidega 
sai täiesti juhuslikult niimoodi, et see koht koodis, kus seda kõike pidi nagu 
välja kutsutama, oli pikka aega välja kommenteeritud. Pärast tema järeltulijate 
poolt avastati mingi hetk, et oleks hulga parem, kui seda funktsionaalsust 
kasutataks ka.
                 
\question{Radari riistvaraga suhtlemine pidi ju keeruline olema?}

Riistvara suhtlemisel oli ajakriitilisus mängus. Tänu sellele tuli paratamatult 
 kerneli alasse ronida, kuna signaal liigub kiiresti ja sul on vaja täpselt 
teada, millal see signaal sisse tuli. Ja sealt omatehtud kaardi pealt tuli 
puhvrid (mis ei olnud küll suure tol ajal)  kähku ära lugeda.

\question{See oli SCO ajal, eks?}

See algas SCO\index{OS!SCO UNIX} ajal,  selle jaoks minu meelest olid meil 
ametlikud raamatud ja ametlik \emph{dev kit} ostetud.
                 
\question{Neid inimesi, kes seda oskasid, ei saanud Eestis palju olla?}

Jah, ega väga küsida kellegi käest ei olnud. Esimene inimene, kellele ka asi 
huvi pakkus ja kes midagi teemast teadis ja suutis teemas kaasa rääkida, oli 
aastaid hiljem (kui ma 1993. aastal Tartusse jõudsin) Meelis 
Rools\index[ppl]{Roos, Meelis}. Aga ta on minust muidugi kõvasti ette jõudnud, 
sest ta on praeguseks päris palju kerneli \emph{patch}-e \emph{post}-inud . 

\question{SCO osas olid teil raamatud ja asjad aga muu info? Räägime BBS-idest? 
Tarmo Mamers juba jooksis läbi\ldots}

Jah, eks Tarmo körval kogemust kogudes ja vaadates, mis ta seal teeb ja koos 
pizzat süües. Peetri pizzat\sidenote{Umbes sel ajal Tartus toimetades oli ka 
seal Peetri pizza üks igati kuum koht. Tänapäevases mõttes pizzaga polnud tol 
rasvasel jahutootel suurt midagi pistmist aga, veidral kombel, ei tekkinud 
neile ka ühtegi konkurenti.}, sest tol ajal muidugi Tallinnas väga valikut ei 
olnud, see oli enam-vähem ainus pizza koht. 

\question{See toimus Skriiningu kontoris?}

Jah, Skriiningus\index{Skriining} ma olin külaline. Ja teine BBS, kus ma tihti 
külas käisin, oli Dark Corner\index{BBS!Dark Corner}.
                 
\question{Tol ajal oli arvutifirma vist natuke nagu klubi moodi asi. Sõltub 
firmast muidugi, aga kogu aeg käisid inimesed läbi, jõid kohvi ja vahetasid 
uudiseid?}

Ta tundus vist olevat küll, jah. Vähemalt osades kohtades, ka Skriiningus, oli 
küll niimoodi.

\question{Mis need \emph{hot-spot}-id veel olid peale Skriiningu?}

Mina väga palju mujal ei käinud. Microlink vist oli ka millalgi selline koht. 
Dark Corner BBS, Priit Kasesalu\index{Kasesalu, Priit} ja Ahti 
Heinla\index[ppl]{Heinla, Ahti} ja kes seal teised olid, nende juures sai 
ainult õhtuti ja öösiti käidud, sest neil vist oli seal päeval töö ka.
Tarmo juures ka, tegelikult ikkagi see sotsiaalne osa oli vist pigem ikka 
õhtupoolikul, päevasel ajal võib-olla ei olnud nii palju seda sotsiaalset 
läbikäimist.
                 
\question{See oli see aeg, hakkas juba olema vaja tööd teha ja raha teenida. 
Ometi ühel hetkel sa panid omale BBS-i püsti. Sa olid MamBoxi point mingi hetk 
ja siis tegid enda oma?}
                 
Jah, MamBox\index{BBS!MamBox}-i juures ma olin point mõnda aega ja siis sai 
firma abiga enda oma tekitatud. Isiklik ta ei olnud, ta oli ikkagi firma 
riistvara peal, firma kontoris. Kiirust väga ei olnud, alguses oli vist isegi 
1200, mingi hetk sain 2400-se. 

\question{Ooo! Mis su purgi nimi oli?}

Boksi nimi oli Flying Discs BBS. 

\question{Väga lennukas! Aga miks sa seda tegid?}

Tundus põnev. Eks sealt sai mänge. Muusikat ma tõmbasin ka, minu empeekolmendus 
sai sel ajal alguse.
                 
\question{1200-se modemiga MP3-e alla tõmmata võtab ju hirmsa aja!}

Selles mõttes oligi parem nagu külla minna. Võtta pizza kaasa ja minna külla 
oli efektiivsem, kui helistada. Mingi hetk me saime sellise modemi nagu 
Zyxel\index{Zyxel}, see suutis juba natuke rohkem välja vilistada aga vist 
ainult teise Zyxeliga. Siis tuli jälle otsida lähikonnast Skandinaaviast 
selliseid kohti, kuhu nii helistada sai.

\question{Ah juba kaugekõnet sai teha?}

Kaugekõnet sai teha jah, mina ei elanud üle seda aega, kus pidi tädile 
telefonis kõigepealt ütlema, et ühendage mind sinna ja tänna. Minul seal EKE 
projekti arvutuskeskuses oli  välisliin ikka algusest peale olemas. 

\question{Kuidas sa, lilleke, siis 1993. aastal Tartusse sattusid?}

Sellega oli nii ja naa. Mõnes mõttes, eks ta sihukesest mugavustsoonist välja 
minek oli, muidugi. Kui ma 1989. aastal  keskkooli lõpetasin, käisin aasta ikka 
TPI-s ka. LI\index{Tallinna Tehnikaülikool!LI} oli äkki? Arvutid ja 
arvutivõrgud? Seal ma pidasin vastu aasta, sest see ei andnud mulle väga mitte 
midagi. Või noh, ütleme, ei andnud kohe üldse mitte midagi. Arvutiaine eksam 
või arvestus tuli teha Pascalis. Kuigi Pascalit\index{Keeled!Pascal} ma polnud 
kunagi õppinud, tegin ma selle töö esimese kuu lõpuks ära, andsin ära, ja 
rohkem kohal ei käinud. Ehk ühesõnaga täiesti mõttetu minu jaoks. Partei 
ajalugu enam pidanud küll õppima, aga see-eest oli  mingi füüsika, kus olid 
Rusalepad\sidenote[][-5cm]{Ilmselt peab Margus silmas Ervin ja Maret 
Rusaleppa\index[ppl]{Rusalep, Maret}\index[ppl]{Rusalep, Ervin}.} kahekesi 
vastas, sellest ma vist kukkusingi läbi.

\question{Misjärel sa jõudsid Tartusse?}

Ma olin, jah, kolm aastat nii-öelda  tööl, tegelesin igasuguste põnevate  või 
vähem põnevate asjadega või mängisin arvutiga. UNIXi peal ma avastasin enda 
jaoks sellise mängu, millest ma ei ole siiamaani lahti saanud, 
Rogue\index{Mängud!Rogue}\index{Mängud!Nethack}\sidenote[][-6.2cm]{Rogue oli 
esimene mäng, kus tekstipõhisel ekraanil protseduurselt genereeritud 
koobastikest koosnevas fantaasimaailmas tuli seiklusi otsida. Peategelase surm 
oli seejuures permanentne ning mängija võis komistada ka oma varasemate 
tegelaste laipadele. Hiljem nimetatigi sedalaadi mänge (polulaarsemad Hack, 
Nethack, Moria, Angband) ühise nimetajaga \enquote{roguelike} ehk 
\enquote{rogue-sarnased}.} või Nethack\sidenote[][-2cm]{Vt. ka lehekülg 
\pageref{sisu:nethack}.}.

\question{Ooo! Mina jõudsin Nethackini hiljem kuid mängin samuti aega-ajalt 
siiamaani. Mis versiooni\sidenote[][-2.6cm]{Nethack on pigem kultuuriline 
fenomen kui arvutimäng, ka tema lähtekood ja andmefailid on mõnuga loetavad 
sisaldades viiteid algmüütidele, tsitaate, luulet jne. Ühest küljest tähendab 
see pidevat arengut kuid teisalt ka seda, et mängijad peavad vaid üht 
konkreetset versiooni selleks \enquote{õigeks} täpselt nii, nagu suhtutakse 
vahel skepsisega uuematesse Star Wars filmidesse.} sa mängid?}
       
Pean tunnistama, et ma mängin seda ka viimasel ajal päris palju. Võtsin omale 
eesmärgiks ta kõikide rollidega lõpuni mängida. Üks roll on veel jäänud, 
kellega ma ei ole lõpuni jõudnud, see on 
\emph{priest}\sidenote[][-.2cm]{Nethackis on 13 rolli, neist igaühel unikaalsed 
võimed ja vaenlased. Juba ühe rolliga mängu läbimine on keeruline ettevõtmine, 
teha mäng läbi kõigi rollidega peale preestri (mis tähendab, et Margus on 
kõikvõimalikele ohtudele vastu astunud ka näiteks turisti rollis, kelle peamine 
võimekus on vastupanu mürkidele) on Nethacki austajate hulgas üsna eepilistes 
mõõtmetes saavutus.}.

Tegelikult on temaga ju lihtne. Vaata, \emph{priest} näeb kohe ära, et kas asja 
saab selga panna või mitte, et kas see on \emph{blessed} või \emph{cursed}. 
Alguses nagu tundub lihtne, aga võib-olla see lihtsus  maksabki kätte. Mul on 
temaga olnud ka väga pikki mänge, aga ei ole veel lõpuni jõudnud. Mängin 
viimast versiooni, ma küll kõiki neid igasuguseid kahe käe relvi ja muid 
uuendusi ei kasuta, aga mängin viimast versiooni.

\question{Nüüd tuleb kiiresti Tartu juurde tagasi tulla! Miks Tartu ja 
matemaatika?}

No seal me olime koos sinuga ja koos igasugu huvitavate tegelastega. Meelis 
Roos\index[ppl]{Roos, Meelis}\sidenote{Meelise lugu algab leheküljelt 
\pageref{sisu:mroos}.} sai mainitud. Asko Seeba\index[ppl]{Seeba, 
Asko}\sidenote{Asko lugu algab leheküljelt \pageref{sisu:asko}.}, eks ole, meie 
gasell\sidenote{Margus viitab Asko juhitava firma Mooncascade saavutustele 
Äripäeva koostatavas gasellettevõtete pingereas.}. Ülo 
Kaasik\index[ppl]{Kaasik, Ülo} on ka tuntud nimi tänapäeval, mitte küll 
arvutimaailmas.

\question{Põnev seltskond oli tõesti. Aga ikkagi. Miks just Tartu ja miks just 
matemaatika?}

Mul üks koolivend keskkoolist, kellega me hästi klappisime ja tänapäevalgi läbi 
käime, oli kohe pärast kooli läinud Tartusse rakmatti õppima. Kui ta oleks 
kaasa kutsunud, võib olla oleksin läinud. Pärast me oleme sellest nagu 
rääkinud, vaaginud, et võib-olla oleks parem olnud, kui oleksin kohe läinud. 
Aga selliseid asju ei tea ette. Informaatikasse, kuna mul see arvuti-asi on  
südamelähedane, siis mõtlesin, et äkki sealt koolist saab midagi rohkem. 

\question{Kas sai ka?}

Eks aeg oli ka edasi läinud, sealt ikka üht-teist juba sai. Kuigi ega selle 
kooliga ma ka lõpuni ei jõudnud. See oli pikk protsess. Esimesed kolm aastat 
läksid ilusti, olin kõigi asjadega  graafikus, aga siis sai raha otsa. Võtsin 
aasta akadeemilist ja tegin tööd, millest ma päris märgatava osa ajast olin 
Venemaal, Peterburis. Tegime sealsete inimestega koostööd, seesama radari 
teema. R süsteemidel oli teine suund merelokaatorid. Need on sarnased selles 
mõttes, et kas sul on nüüd see kajalokatsioon on õhus või vees, et ega seal 
palju vahet ei ole. Kuigi merepõhja läheb üldiselt keerulisem signaal, kanaleid 
on rohkem.

\question{Kas sa Tartus pidasid BBS-i edasi või oli see pausi peal?}
                 
BBS-i asi jäi Tartusse minnes katki küll. See Tartusse minek oli ikka väga 
mugavustsoonist välja minek. Helistamine jäi nagu ära, kuigi kui ma järele 
mõtlen, siis internetiühendus oli meil ka alguses ikkagi niimoodi, et tuli 
helistada. Pikka aega oli ka seesama asi, et üks telefon on kogu aeg modemi 
taga kinni. Alguses siis BBS-i taga ja pärast internetis. Püsiühendused tulid 
kunagi hiljem.
                 
\question{Kas Tartu Ülikool\index{Tartu Ülikool!Matemaatikateaduskond} sind 
akadeemilisse maailma ei tõmmanud? Neid näiteid oli ka meil ka kursa pealt?}

Ei, väga ei tõmmanud. Kursatöö juhendajaga ma käisin küll korra ühel 
välisreisil kaasas, eks ta üritas mind ilmselt sellega meelitada. Norras, 
Bergenis, oli mingisugune konverents, kus ta pidi oma pabereid esitama. 
Akadeemilises maailmas on see, et sa pead esitama mingeid asju, et saada 
mingeid punkte, \emph{exp}-i ja \emph{level}-it. Ta võttis mu kaasa, sain selle 
eest ilmselt ka mõned \emph{exp}-i punktid, et tema rääkis ja mina vajutasin 
arvutiklahve. Sest tol ajal ei olnud igasuguseid vahvaid pulte ja asju. 

\question{Aga see sind ei tõmmanud?}

Ei, see mind väga ei tõmmanud. Oli küll jah, igasuguseid vahvaid riistvarasid 
nagu SUN-id ja\ldots 

\question{Tartu Ülikool oli vist päris hästi varustatud tollal?}
                 
Ja. Ühikas\sidenote{Tartu Ülikooli Tiigi tänava ühiselamu\index{Tartu 
Ülikool!Tiigi ühikas}} me ju alustasime ise internetiga. Asko 
Tiiduma\index[ppl]{Tiidumaa, Asko} oli meil nii-öelda ühika sysop, sest tema 
toas see sisendpurk. Katusel oli antenn, ma arvan, ja tema  elas seal antenni 
all. Aga jah, kaableid vedada ja ühikat lõhkuda\ldots Ega see ühikas kaabli 
vedamine ei ole lihtne! Kui vales kohas puurid, tuleb terve telliskivi välja!

\question{Lõpetuseks, kuhu see tee sind tänaseks toonud on?}

Tänaseks, või ütleme viimaseks peaagu kahekümneks aastaks, olen ma maandunud  
pangandusse,  IT-valdkonda ikka muidugi. Kunagi oli selle firma nimi Hansapank, 
nüüd on ta Swedbank. Hansasse ma tulin 2000 või oli see 2001, kuskil seal 
kandis ja nüüd ma olen siin olnud.

\question{Üritades nüüd norida, et panganduses on ju ülesanded hoopis teist 
masti, kui kuskilt radari signaali lugemine. Millest selline muutus?}

Oli muutus küll ja ega see muutus ei olnud kerge tulema. Ühelt poolt olid  
muidugi majanduslikud probleemid, sest R süsteemide seltskond oli tore 
seltskond, meil oli näiteks suveti kombeks, et võeti arvutid ja mindi kuskile 
mere äärde. Pandi võrk kuskil mujal püsti, kohaliku kooli juurest  või kust sai 
võeti mingi internet ja siis käisid vahepeal rannas ja siis tegid tööd ja 
selles mõttes oli tore. Aga palgaga oli nagu kehvasti. Töö oli üldiselt 
huvitav, aga kui on väga väike firma, siis kogu aeg ei olnud tellimusi. Ma ei 
ole ise selline müügiinimene, et ma läheks, otsiks tänavalt tööd. Ja kuna seal 
firmast see pool kippus natuke nagu lonkama, siis ma mõnes mõttes läksin 
lihtsama vastupanu teed. 

Et kuhu ma tänapäevaks olen jõudnud. Ma olen vahepeal päris palju baasi 
kirjutanud. Kui ma panka tööle tulin, siis ma mäletan, et vestlusel rääkisin, 
et baasi ma ei ole kirjutanud ja et ma väga ei taha ka. Aga vahepeal hakkas 
baas mulle  täitsa meeldima, aga nüüd ma vaatan, et tuuled on jälle sinna poole 
läinud, et \emph{micro-service} maailmas on baas jälle väga \emph{evil}. 

\question{See vist on pikalt arvutitega toimetamise hüve, et sa näed neid 
tsükleid ja ringe.}

Jah, eks vahepeal ikka kaldutakse äärmustesse ja eks see tõde on seal kuskil 
keskel.


\chapter{Jaan Tallinn}
\index[ppl]{Tallinn, Jaan}

\question{Kuidas ja umbes millal sa jõudsid arvutite juurde?}

Ma mäletan  seda aega, kuskohas mul isa hakkas kaheksakümnendatel Soome vahet 
käima, seal mingisuguseid
filmi ja videorežiitöid tegemas. Ja ta tõi mulle erinevaid ajakirju, neid oli 
hea, odav ja võib-olla isegi tasuta tuua. Neist päris mitmed olid 
arvutiajakirjad ja see tundus kohe olema väga põnev. Armumine esimesest 
pilgust. Algkooli kas  viimases või eelviimases klassis juhtus selline asi, et 
üks kooli lapsevanem valis mind ja mõningaid mu klassivendi (sealhulgas näiteks 
Priit Kasesalu\index[ppl]{Kasesalu, Priit}) eksperimentaal-katsejänesteks, et 
viia  meid õhtuti  kuskil Kopli servas asuvasse Sideministeeriumi 
Arvutuskeskusesse\index{Sideministeeriumi Info- ja Arvutuskeskus} ja lasta seal 
suurte \emph{mainframe}-de peal lahti ja vaadata, mis juhtub. Nii et inimkatse 
tulemus. 

\question{Mis kool see oli?}

See oli Lasnamäel kuuekümnes keskkool\index{Koolid!Tallinna 60. Keskkool}.

\question{Kust selline mõte tuli, et peaks inimestega niimoodi tegema?}

Seda ma ei tea, aga sa võid ta enda käest küsida. Ta nimi on Jüri 
Malsub\index[ppl]{Malsub, Jüri}, talle meeldib sellest väga pikalt rääkida. 
Seal seltskonnas olin mina, Priit Kasesalu ja veel kaks klassivenda, kelles 
ühest (Mikk Orglaan\index[ppl]{Orglaan, Mikk}) sai ka arvuti-ettevõtja. Neljas 
oli Martin Kruusvall\index[ppl]{Kruusvall, Martin}, kellele sai selgeks, et 
numbrid teda väga ei paelu, et ta on rohkem nagu luuletaja tüüp.

Keskkoolis liitus selle seltskonnaga Ahti Heinla\index[ppl]{Heinla, Ahti}. Siis 
ma olin juba läinud Tallinnas Gustav Adolfi Gümnaasiumi, toona esimesse 
keskkooli\index{Koolid!Tallinna 1. Keskkool} ja hakanud tõsiselt tegema 
olümpiaadidega. Meie füüsikaõpetaja, kadunud Vilma Kukrus\index[ppl]{Kukrus, 
Vilma} ühel hetkel (peale seda, kui Ahti oli vabariikliku füüsika olümpiaadi 
kinni pannud) rääkis Ahti pehmeks, et mis sa seal Õismäel passid, tule parem 
Gustav Adolfisse. Nii, et ta tuli meile mitte esimese keskkooli klassi, vaid  
teise  ja me saime suhteliselt kiiresti headeks sõpradeks. Ilmselt mina, kes 
see muu võis olla, kutsusin teda sellesse seltskonna, kellega me olime juba  
mõned aastad seal Kopli piiril tegutsenud. 

\question{Ja kogu selle aja te käisite \emph{mainframe}-i näppimas? Mis te 
tegite nendega?}

No \emph{mainframe}-d said muidugi kiire lõpu, kuna arvutustehnika arenes. 
Esimene mitte-\emph{mainframe} platvorm, kuhu me kolisime, oli sealsamas 
keskuses õhtuti meisterdatud riistvaraplatvorm nimega 
Entel\index{Arvutid!Entel}, mis oli selline CP/M masin. Ta kasutas mingisugust 
Intel 8088 protsessorit, või mingit Vene klooni sellest kuulsast 
kaheksabitilisest protsessorist. CP/M tarkvara oli, aga midagi sellist 
spetsiifilist tema jaoks kirjutatud ei olnud ja  siis oligi nagu koht, kus sai 
hakata mitte-\emph{mainframe}-de peal kätt proovima. 

\question{Aga mis te nende arvutitega siis tegite? Noorel inimesel on ju see 
probleem, et kui valid liiga raske ülesande, ei saa hakkama ja on halb ja kui 
liiga kerge, siis on igav ja ka halb?}

See on üks väga relevantne küsimus, sellepärast et mõnes mõttes meie 
generatsioonil on arvutitega vedanud. Sel hetkel, kui arvutite juurde 
sattusime, olid nad sellised, et nagu midagi väga huvitavat ei toimunud. 
Arvutite peamine köitlus oli potentsiaal, mis neis selgelt sees tuksus. Versus 
see, et sul on Youtube ja Minecraft ühe kliki kaugusel. Ükskõik, kui palju sa  
pingutataksid, midagi  ligilähedastki sa võimeline tegema ei ole. Ja teine asi, 
et arvutid olid toona aeglased,  umbes  miljon korda aeglasemad kui praegu. 
Mistõttu, kui tahtsin midagi ägedat teha, siis pidin kohe kiiresti selle 
hingeelu endale põhjalikult selgeks tegema, et pigistada välja viimane 
efektiivsusepiisk.

\question{Sa jooksid kohe mingitesse riistvara piirangutesse sisse ja isegi 
mingi lihtsa asja ekraanil liigutamiseks pidi hoolega mõtlema, et kuidas see 
ikka täpselt käib!}

Täpselt. Mistõttu suhteliselt kiiresti läksime 
assembleri\index{Keeled!Assembler} peale. Kõigepealt siis kodukootud 
Entel-arvutite peal ja siis aasta-paari pärast tekkisid Eestis esimesed IBM PC 
kloonid. 

\question{Assembleri peale kolimine eeldab siiski, et programmeerimisest on 
mingi aimdus olemas. Kust see tekkis?}

See tekkiski nende \emph{mainframe}-de peal. Robotron või mis ta oli.

\question{Aga kuidas? Lugesite raamatuid või\ldots?}

Lugesin läbi, mis selle nimi oligi, Programmeerimine 
Pascalis\sidenote{Tõenäoliselt R. Jürgenson Programmeerimine Pascal-keeles.} 
või midagi sellist. Mul on seal siiamaani mingisugused esimeste programmide 
väljatrükid  vahel, raamat on raamaturiiulis. Kirjutan 
Basicus\index{Keeled!BASIC} programmi ja kirusin, et keel on ikkagi erinev kui 
Pascal. Basicut ma ei osanud aga Pascalit natuke siis teoreetiliselt oskasin ja 
kahe peale siis hakkasin avastama. Esimene programm vist oli ruutvõrrandi 
lahendaja.

\question{See on klassika, ilmselt seetõttu, et teda on praktiliselt vaja. Aga 
ikkagi, sealt Assemblerisse minna on pikk samm, juba arusaam, et kuskil on 
Assembler, on küsimus. Kust te infot saite? Keegi õpetas? Raamatud? Ajakirjad?}

Jah, seal \emph{mainframe}-de peal ma isegi jäin Basic-usse, tegin seal isegi 
oma esimese mängu. Ja kui me kolisime  \emph{mainframe}-de pealt ära nende 
kodukootud kaheksabitiste arvutite peale, oli näha, et seal on lihtsam  
riistvarale ligi saada, eks. Ja üks asi, mis kohe ahvatlema ja paistma hakkas 
oli C programmeerimiskeel\index{Keeled!C}.  Mäletan, et samas grupis aeg-ajalt 
näitas oma nägu selline sell nagu Hannu Krosing\index[ppl]{Krosing, Hannu}, 
endine Skype kolleeg, kes  otseselt samas seltskonnas ei olnud. Ja tema oli 
selleks hetkeks  kirjutanud Assembleri õpiku, oli mingi selline pisikene 
brošüür põhimõtteliselt.  Ja ta kas pistis selle mulle pihku või, ma ei tea, 
igal juhul ma lihtsalt lugesin selle läbi, et \enquote{ohoo, mingi päris 
huvitav asi}. 

\question{Oot, mis aastal see võis olla?}

See võis olla 1987 äkki? 1988?

\question{87. aastaks oli Hannu kirjutanud Assembleri õpiku!?}

Jah, mingi sellise brošüüri vormis, 
samizdat\sidenote{\begin{russian}Cамиздат\end{russian}, tõlkes umbes 
\enquote{iseavaldamine} oli Nõukogude Liidus levinud keelatud või põrandaaluse 
kirjanduse levitamise viis. Teksti trükiti läbi mitmete kopeerpaberite 
õhukesele paberile ümber, tulemused levisid käest kätte ning neid paljundati 
omakorda. Mäletan, et ka minu vanaema tegeles sellise toksimisega, ning lapsena 
ei mõistnud, miks sellest väga rääkida ei tohi. Kuna kõik klahvidega asjad mind 
väga huvitasid, nuiasin välja võimaluse ka ise tekste ümber lüüa, miskipärast 
olen veendunud, et olen aidanud paljundada mingit budistlikku teksti.} umbes. 
Oli ikka Hannu õpik? Temaga ma mäletan, ma sellel teemal arutasin,  üsna kindel 
et tema oli selle autor.

\question{Ja mis te tegite selle Assembleriga?}

Mäletan, üks nagu selliseid korralikumaid projekte, võib-olla tegin mingeid 
väiksemaid asju ka, oli tekstiredaktor\label{sisu!jaani_tekstiredaktor}. 
Sattusime Priit Kasesaluga\index[ppl]{Kasesalu, Priit} sellisesse 
võistlusrežiimi. Mõtlesime, et mida oleks hea sellele uuele kodukootud 
platvormile kirjutada ja leidsime, et siin ei ole korralikku tekstiredaktorit. 
Hakkasime mõlemad tegema, kõigepealt Basicus\index{Keeled!Basic}. Ja üritasime 
üksteist üle trumbata, et kellel tuleb parem. Mäletan, et 
Hannuga\index[ppl]{Krosing, Hannu} rääkisime  mingist tehnikast, kuidas 
tekstiredaktoris teha  mingisugust \emph{split buffer} arhitektuuri, et 
liikumine ja \emph{insert}-imine kiired oleksid.

Tuli koolivaheaeg ja meil jäi võistlus pooleli. Aga mina panin nagu edasi, suvi 
otsa kirjutasin paberi peal tekstiredaktorit, assembleris. Ja kui tagasi tulin, 
polnud  Priit muidugi suvega midagi viitsinud teha ja sellega oli võistlus 
läbi. Siis kirjutasin selle Assembleri paberi pealt arvutisse.

\question{Töötas ka?}

Esimene kord muidugi ei töötanud, eks ole. Aga tööle ma ta igal juhul sain, asi 
toimimis ja  vaatasin, et \enquote{oo, see on ikka päris äge}. Kiire, mugav ja 
palju parem, kui ükskõik milline tekstiredaktor sellel arvutil. See andis mulle 
väga positiivse tagasiside. Peale seda hakkasin mänge kirjutama.

\question{Reflekteerides siit tundub, et sul pidi olema oskus päris suuri ja 
keerulisi abstraktseid struktuure peas ette kujutada, et sa suudaksid selle 
koodi kõik paberil asmi valada. Kust see oskus tuli või on see sul kogu aeg 
olnud või oskad sa natuke selle juuri avada?}

Ma ei tea, mulle tundus see suhteliselt loomulik. Sellised instruktsioonid 
lihtsalt, nagu sammude kirjeldus. Et mida sa tahad, et arvuti teeks, eks. On 
vaja täpselt üles kirjutada, mida sa tahad. Alguses Basicus sain esimese 
tagasiside, et kuidas tsükkel käib, ja ühel hetkel assembleris nägin, et see on 
lihtsalt natuke tülikam, aga teisalt jälle rohkem positiivset tagasisidet 
pakkuv, kui sa ta käima saad. Käib nagu väga muljetavaldavalt võrreldes 
Basicuga. 

\question{Kas see tekstiredaktor kuskile jõudis ka või sai lihtsalt oma lõbuks 
tehtud?}

See seltskond, kes siis seda Entel\index{Arvutid!Entel} arvutit tegi, 
vormistasid niipea, kui eraettevõtlus seaduslikuks muutus, kooperatiivi ja 
hakkasid  neid arvuteid tootma ja müüma. Muu hulgas käis selle arvuti juurde 
tema jaoks toodetud tarkvara: CP/M ja  lisaks see minu tekstiredaktor.

\question{Ehk esimene suurem projekt, mis sa kirjutasid, läks kohe kliendile 
müüki?}

Jah, ma ei tea, palju seda kasutati, aga kui sa endale kaheksakümnendate lõpus 
selle Eestis toodetud arvuti ostsid, siis oli seal minu tekstiredaktor kaasas. 
Selle tõttu me saime esimest palka ja tekkisid esimesed mingisugused 
sissetulekud

\question{See ju tahab tarkvaraarenduse mõttes küpsust, et sa mõtled kõik 
nurgatagused juhtumid läbi ja võib-olla kirjutad abiteksti?}

Mõnes mõttes mitte väga optimaalne, aga ma olen märganud, et mul on selline 
OCD, \emph{obsessive compulsive disorder}. Et kui midagi alustan, tahan selle 
kindlasti lõpule viia, panna i-dele punktid peale. Seetõttu paljudele 
projektidele, kus ei ole ajasurvet taga, läheb kole palju aega. Alates sellest 
tekstiredaktorist. Ma tahtsin, et kõik oleks väga ilus, kõik funktsionaalsus 
oleks olemas ja pusin senikaua, kuni oli. 

\question{Ehk siis kombinatsioon täiuslikkuse soovist ja võimekusest see ka ära 
täita. Mul võib olla soov täiuslik teemant lihvida aga ma lihtsalt ei oska seda 
teha.}

Ja, jällegi, millega mul vedas, oli see, ma astusin arvutite juurde sellisel 
hetkel kuskohas kogu tarkvara, mis seal arvutites juba oli, oli väga lihtne. 
Mistõttu see ei olnud selline nii-öelda hingemattev kogemus, et ma olen nii 
pisikene selle tarkvara kõrval vaid \enquote{ahah, okei, ma saan enam-vähem 
aru, kas tehtud on, ma teeksin paremini}.

\question{Seda on mitmed öelnud, et oma esimest arvutit nad tundsid põhjani.}

Ka auto-entusiastidel, uunikumide austajatel, on samasugune lugu, neil on 
ikkagi väga lihtsad riistapuud.

\question{Aga see annab sulle kontrolli tunde, eksole.}

No mul täna läks Tesla katki. Ja midagi ei ole teha, tuleb Soome saata.

\question{Ma korraks võtaks kinni noist alguses räägitud arvutiajakirjadest. 
Oskad sa takkajärgi öelda, oled sa mõelnud, mis sind nende juures paelus?}

Asjad, mis seal kohe väga prominentselt silma paistsid, olid mingisuguste 
arvutimängude reklaamid, mingid Atari reklaamid ja sellised asjad. Noh, nagu 
ikka, reklaamidel muidugi joonistati natuke ilusamaks, kui päris maailm, aga 
nad andsid vaate mingisugusesse oma seaduste järgi toimivasse fantaasiamaailma, 
mis tohutult paelus. Mingisugused ekraanitõmmised, kus mingid tegelased on peal 
ja ma vaatasin, et \enquote{ahaa, see on vist väga äge asi!}.

\question{Seda on ka räägitud, et see oli nagu täitsa teine maailm, kuhu sai 
sisse minna.}

Ja veelgi enam, sa said neid maailmu ise luua, see teadmine tekkis mõne aja 
pärast. Et sa ei ole passiivne tarbija vaid aktiivne looja.

\question{Ja selle aktiivsusega sa kirjutasid selle tekstiredaktori valmis ja 
sind võeti palgale?}

Ma ei mäleta, kuidas see järjekord täpselt oli, võimalik, et meid võeti palgale 
seal alguses, kui ma seal niisama katsetasime. Aga võimalik, et see tõesti oli 
pärast seda, kui me esimesed asjad ära tegime.

\question{See oli keskkooli ajal veel?}

Jah, see oli vist keskkooli alguses, ma arvan. Kaheksakümmend seitse oli 
keskkooli algus. Kaheksakümmend kuus võis olla see aasta, kus ma üldse sinna 
sattusin ja siis 87-88 oli see, kus palka hakkasin saama. 

\question{Kas arvutis käimine olümpiaade ei hakanud segama või käis see 
õppimisega kuidagi lihtsasti kokku?}

Üldse ei seganud. Arvuti-värk on mul ikka suhteliselt kogu aeg põhiline asi 
elus olnud, ülikooli lõpetasin ka nii-öelda kõrvalhobina ära. Aga juba siis 
olid kool ja olümpiaadid arvuti taustal.

\question{Kas juba siis hakkas moodustuma seltskond, mis pärast sai 
Bluemooniks\index{Bluemoon}?}

Just. Bluemooni süda tegelikult oligi see seltskond, mõned klassikaaslased. 

\question{Kas tollest arvutikooperatiivist eraldusite kohe eraldi ettevõtteks 
või oli seal vahepeal mingi faas veel?}

See oli niimoodi, et ühel hetkel meil Ahtiga\index[ppl]{Heinla, Ahti} tekkis 
mõte, et teeks ühe arvutimängu. Korraliku mängu, mis jookseb PC peale, mitte 
ainult seal kodukootud arvutite peal. Meil oli mingi eeskuju ka, mille järgi  
mängu teha, mis oli Yamaha MSX-ide\index{Arvutid!Yamaha MSX} peal, mis oli 
palju vähempopulaarsem platvorm kui, PC. Oli näha, et PC-d hakkavad juba jõudma 
sinnamaani, kuskohas saab juba midagi huvitavat teha. Ja väga sellise sügava 
mulje jätsid toona Ahti matemaatikuvõimed. Kuidas ta jagas ära, et 
\enquote{siin tuleb tangensit kasutada, et  perspektiivi luua}. Mingisuguseid 
esimesi eksperimente tegime tema vanemate juures Küberis\index{Küber}, kus tal 
oli arvutitele ligipääs. Ahti hakkas palju varem programmeerima, kui mina. 

Üks tõuge selle mängu tegemiseks oli see, et meil keskkooli viimases klassis 
(oli vist ikka viimane klass?) tekkis võimalus minna klassiga Rootsi. Esimene 
välisreis üldse, aastal 1989 suhteliselt unikaalne võimalus. Läksime sinna läbi 
Leningradi, siis oli vaja teha mingisuguseid imelikke trikke väljamaale 
saamiseks. Seal onutütre mees ütles, et \enquote{Väljamaal nad õudselt tahavad 
softi, kirjutage mingisugune lahe soft. Lähed sinna, müüd maha ja mingit 
probleemi pole!}. Mõtlesin, et \enquote{aga teeme} ja hakkasime tegema. 
Leidsime, et teeme mängu, teeme korraliku mängu. Hakkasime tegema ja muidugi ei 
jõudnud valmis. Softiga, nagu ma nüüd hiljem tean, tuleb kõik ennustused  umbes 
piiga läbim korrutada, kulub umbes kolm korda rohkem aega kui alguses arvad. 

Muidugi me seda valmis ei saanud, aga samas oli juba piisavalt suur hoog sees. 
Et ühel hetkel võtsime sinna juurde korraliku kunstniku, Kaspar 
Loit\index[ppl]{Loit, Kaspar} ehk BKnows. Ja muusika ning heli-inimese Ott 
Aloe\index[ppl]{Aaloe, Ott}. Ja tegime mitte ainult ühe mängu vaid täitsa 
sellise mängude seeria. Millega meil vedas, oli see, et see esimene mäng 
õnnestus Rootsi müüa hoolimata sellest, et meil Rootsis käisime ajal seda mängu 
kuhugi kellelegi pakkuda ei olnud isegi, kui ta oleks valmis olnud.

\question{Kuidas? Meil oli ju veel Nõukogude Eesti?}

Selle Sideministeeriumi arvutuskeskuse\index{Sideministeeriumi Info- ja 
Arvutuskeskus} juhataja Jüri Malsubi\index[ppl]{Malsub, Jüri}  tuttav oli üks 
sell nimega Tiit Vasli\index[ppl]{Vasli, Tiit}, kellel oli väljamaal suhteid, 
ta vahendas metalli, mingeid sihukesi asju. Ma isegi ei teadnud, mida ta 
vahendas. Ta oli selline mees, keda oli kaugelt näha, sellepärast et tal oli 
üks Eesti esimesi mobiiltelefone, mille antenn oli mingi kolm meetrit kõrge. 
Oli kaugelt näha, et tema läheb seal kuskil tänaval. Tal oli Rootsis sidemeid 
ja ta mõtles, et \enquote{noh, ma vaatan, räägin} ja müüski. Ta äripartnerid 
olid huvitatud sellisest eksootilisest asjast nagu raudse eesriide taga 
toodetud mäng. 

Selle mängumüümise tulemusena me teenisime rohkem, kui mu vanemad kunagi oma 
elu jooksul teeninud olid. Mis oli vist mingi viis tuhat dollarit. Arvestades 
muidugi inflatsiooni, mitte reaalväärtuses, vaid nominaalväärtuses. 

Kui see mäng nii-öelda müüki läks, tekkis meil tõsine küsimus, see oli siis 
keskkooli lõpp, ülikooli algus, et kuidas me nüüd seda administratiivselt 
korraldame. Oleme selles kooperatiivis  ametlikult tööl, eks, aga on tegelikult 
näha, et meie plaanid võivad suuremaks kujuneda, kui see kooperatiiv. Rääkisime 
läbi. Mäletan pingelist läbirääkimist Jüri Malsubiga\index[ppl]{Malsub, Jüri} 
kuidas seda mängu tulu jagada. Nemad on ühelt poolt panustanud ja meie oleme 
teiselt poolt panustanud, tahaks nagu oma asja teha. Lõpuks saime meie poolt 
vaadates väga mõistliku kokkuleppe ja leidsime, et nüüd on aeg vormistada asi 
mingiks oma ettevõtteks. Mida me siis ka tegime, aastal 1990, ma arvan. 

\question{Kas te mõtlesite nullist välja, et teil on vaja kunstnikku ja 
muusikut ja kuidas nende töö programmeerimisega siduda või oli teil eeskujusid 
ka?}

No me olime teisi mänge näinud ja nägime, et nad näevad paremad välja kui see 
meie katsetus ilma kunstniketa. Ma ei mäleta, kes meid  
BKnowsiga\index[ppl]{Loit, Kaspar} tutvustas, see võis olla isegi Tanel 
Hiir\index[ppl]{Hiir, Tanel}, ei mäleta. Kaspari kunstniku-võimed toona jätsid 
mulle väga sügava mulje. Teda oli raske tööle saada, ma mäletan, tihtipeale 
pidi selja taga istuma, et \enquote{tee nüüd}, aga kui ta tööle sai, oli väga 
äge. 

\question{Ma just mõtlengi seda, et kindlad viisid graafikat kasutada, 
töödelda, laadida on ju tänaseks välja kujunenud, kas teie mõtlesite need ise 
välja?}

Üsna, jah, sest, jällegi, need platvormid olid miljon korda aeglasemad, kui 
praegu. Mistõttu tööriistad olid Turbo Pascal\index{Keeled!Turbo Pascal} ja 
Borland C\index{Keeled!Turbo C}. Kaspar tegi asju Amigal, seal olid tal oma 
tööriistad.

\question{Mind on painanud see küsimus, et te ju tegite muusikaprogrammi. 
Kuidas te sinna valdkonda sattusite, te pole ükski muusikainimene ju, nii 
palju, kui ma tean?}

Ükskord ülikoolis oli sihuke lahe hetk, kus olin arvutiklassis ja mingid tüübid 
istusid arvuti taga ning komponeerisid  SoundClubis\index{SoundClub} muusikat. 
Kiibitsesin natuke ja ütlesin, et see on minu programm. Nad ei uskunud. 

Ma ei mäleta, kuidas see algtõuge sattus. Tänu sellele, et me olime juba mänge 
teinud, oli meil kindlasti kokkupuude sellega, kuidas teha taustamuusikat. Ja 
toona, üheksakümnendate alguses, oli väga suur trend trackerid, ehk mingite 
sämplite baasil muusika kirjutamise väga sellised platoonilised riistapuud. Ja 
sealt tuli mõte, et  heli on väga hea,  aga kasutajakogemus tundus  vähemalt 
harjumatule silmale väga-väga ebamugav. Mõtlesime, et kuidas kasutada sedasama 
tehnilist võimekust, aga teha  kasutajaliides, mis oleks äge eriti inimestele, 
kes ei ole pidevalt muusika kirjutamise juures.

Teema hakkas järjest huvitama, kuna seal on väga mitmeid nüansse, nagu UI 
disain, muusika pool asjas (kuigi ükski nendest autoritest ei olnud muusikud), 
kuidas tehniliselt teha aeglastel arvutitel head heli. Seal ma puutusin esimest 
korda kokku mingisuguste matemaatiliste teoreemidega, mida ma siis 
Ahti\index[ppl]{Heinla, Ahti} abil üritasin lahendada. Üks huvitav asi oli see, 
et kuna me need instrumendid korjasime endale kuskilt BBS-idest kokku, olid 
need õudse kvaliteediga. Mäletan, et Ahti kirjutas mingisuguse tarkvara, kus ta 
tegi Fourier analüüsi, et nad häälde viia. Ükskord Tartus istusin ja 
häälestasin pille niimoodi, et endal väga suurt muusikaharidust ei olnud, 
natuke olin pilli õppinud. Aga Fourier analüüsiga sai ikka väga hea häälestuse. 

\question{Selle tarkvaraga on ju tehtud igasugu asju Vennaskona Diskost alates 
ja ta käib ka siin mõnest loost läbi. Küll aga ma ei mäleta, et keegi oleks 
rääkinud selle tarkvara ostmisest?}

Jaa!  

Siiamaani võib-olla vähem kui kord nädalas, aga kord kuus vähemalt saame mingi 
fännikirja, et näete ma olen on SkyRoads-i\index{Mängud!SkyRoads} peal üles kasvanud. On isegi mõned 
kloonid tehtud,  teda saab tänapäeval veebis mängida. Ja SoundClub oli teine 
suurem projekt. Meil oli siis juba Bluemoon firmana ja meil oli kaks toodet 
SkyRoads (mis tegelikult oli järg tollele esimesele Rootsi müüdud mängule, 
mille nimi oli Kosmonaut\index{Mängud!Kosmonaut}) ja SoundClub. 

Ja nüüd oli küsimus, kuidas neid müüa. Mäletan, et see oli mingisuguste 
telefonide ja faktidega ja tšekkidega jamamine. Mõlemad olid \emph{shareware}, 
osalt saadeti lihtsalt ümbrikus sularaha aga tavaliselt saadeti tšekke, mida ma 
käisin Eesti Maapangas või Rahvapangas lunastamas, selline kogemus. 

Teine asi, mis oli tegelikult väga äge kogemus, oli läbirääkimiste pidamine 
olukorras, kuskohas teisel poolel ei ole mingit juriidilist motivatsiooni 
lepinguid järgida. Mistõttu tuli tihtilugu tekitada selline olukord, kuskohas 
sa nagu lood sellise helge tuleviku, et partnerlusel oleks jumet. Mõnes mõttes 
selline \emph{iterated prisoner's dilemma}\sidenote{Mänguteoreetiline 
konstruktsioon, mille abil uuritakse osapoolte koostööstrateegiaid. Selle 
valdkonna üks teadustulemusi on, et (eriti mängu iteratiivses, korduvalt 
mängitavas ja eelmisi tulemusi \enquote{mäletavas} versioonis) indiviidile 
annavad pikas perspektiivis parema tulemuse altruistlikud, mitte egoistlikud 
strateegiad.}, sa pead looma olukorra, kus teisel poolel, hoolimata sellest, et 
mingit sundmehhanismi ei ole, on lihtsalt huvi olla osa sinu tulevikust ja 
seeläbi lepinguid järgida.

Alguses oli meil \emph{shareware} aga inimesed hakkasid kirjutama, et tahaks 
seda mingisugusesse ajakirja panna või tahaks seda kuskil levitada. Mingi väga 
lahe sell tekiks meil Saksamaale, kes hakkas mitmeid meie asju levitama, aastal 
1996, käisin tal lõpuks isegi külas. Samuti üks väga lahe omaette kogemus oli 
müük Taiwani telefoni ja faksi abil, kuskil Tartu Estiko\index{Estiko} 
kontoris. 

\question{Kuidas sa Tartusse sattusid?}

Ülikooli läksin.

\question{Mida sa õppima läksid?}

Füüsikat. Nii mina kui Ahti läksime füüsikat õppima aga Ahti kukkus sealt juba 
teisel aastal välja. Mina punnitasin lõpuni. 

\question{Miks just füüsika? Ahti rääkis, et see ala tundus talle mõnes mõttes 
kõige puhtam?}

Ma arvan, et ta on vähem puhtam kui arvutiteadus või matemaatika, eks. Kaks 
põhjust oli füüsika valikuks, Ahti põhjused olid ilmselt korelleeritud. Üks oli 
see, et ma tundsin, et  arvutites ja matemaatikas olen ma juba piisavalt sees, 
et füüsika oleks nagu silmaringi laiendav. Ja teine oli see, et meie füüsika 
õpetaja, Vilma Kukrus\index[ppl]{Kukrus, Vilma}, oli ikka väga väga äge õpetaja 
ja tekitas füüsika vastu sügava huvi. Või vähemalt süvaga austuse. Mul on väga 
hea meel, et ma füüsika lõpetasin.

\question{See ilmselt mõjutas päris olulisel määral noore inimese maailmapilti 
ka?}

Absoluutselt. Füüsika on selles mõttes optimaalne teadus, et sa suhtestud 
reaalse maailmaga niimoodi, et kui matemaatikud võivad minna niivõrd 
abstraktseks, et nad kaugenevad reaalse maailma piirangutest, siis füüsikas 
reaalne maailm tõmbab su alati maa peale tagasi. Sõna otseses mõttes, 
tihtilugu. Ja seetõttu sul tekib intuitiivne arusaam sellest, misasi on teadus. 

\question{Ahtiga\index[ppl]{Heinla, Ahti} oli ka nii, sinu puhul on samasugune 
muster, seepärast küsin. See, mis ma kuulen ei kõla nagu keskmine 
\emph{teenager}. See kõlab nagu üsna küpse inimese jutt?}

No praegu ma enam \emph{teenager} ei ole!

\question{Nüüd jah, aga need otsused ja see viis, kuidas toona asju aeti on 
üsna kaine, arutlev lähenemine. Kust see pärit on?}

Üks oluline asi oli ikkagi, ma arvan, et Ahtile ma võlgnen väga palju tänu. 
Meil oli super hea koostöö. Priit ka, eks, aga praktiliselt kõiki selliseid 
probleeme lahendasime tiimiga. Minu  ja Ahti vahel tekkis väga tihti selline 
asi, et Ahti on nupukas ja ta mõtleb väga erinevalt, kui mina. Mistõttu 
koostöös temaga sündinud otsused olid just nimelt ägedad, kuna nendes oli kaks 
väga erinevat vaatepunkti, mida see otsus pidi rahuldama.

\question{Ma olen alati tahtnud küsida. Võib olla ruttame natuke ette, aga kui 
me vaatame kasvõi Bluemooni kodulehekülge, on seal loetletud üksjagu edukaid 
asju aga ka päris mitu asja, mis ei ole ühel või teisel põhjusel välja tulnud. 
Inimesed ei suuda mõnikord isegi läbi suure edu tiimina toimima jääda aga teie 
olete koos läbi nii suure edu kui mitmete ebaõnnestumiste. Kuidas te seda 
teete?}

Vahemärkusena, mäletan, mõni aasta  tagasi sain Sean Parkeriga\sidenote{Sean 
Parker\index[ppl]{Parker, Sean} on Napsteri kaasasutaja ja, muu hulgas, 
Facebooki esimene president. Ta esineb ka tegelaskujuna Facebookist rääkivas 
filmis, kus kujutatud intriigidest ja tülist tõukub ka eelnev küsimus.} kokku 
ja meenutasime  Napsteri ja Kazaa aegu, tema tegi Napsterit. Selline lahe 
kogemus.

Ma arvan ikkagi, et sellised kohatised eduelamused olid piisavad, et  läbi 
suruda ka sellistest mitte õnnestunud projektidest. Ja mõned hetked olid ikkagi 
jube rasked. Konkreetselt mäletan  sellist hetke, kus kogu mänguarendus läks 
üles-suunas ja siis ühel hetkel lõpetas meie kirjastaja Ameerikas, Interactive 
Magic, \emph{milestone}-de maksmise. Raha jaoks oli meil Exceli tabel, kus 
\emph{runway} oli kogu aeg kirjas, mitmeks kuuks  meil raha on põhimõtteliselt. 
See \emph{runway} hakkas siis kahanema ja ühel hetkel oli selge, et nad on 
pankrotis, sealt enam midagi ei tule. Oli tõsine küsimus, et mis nüüd edasi 
saab. Ja Ahti\index[ppl]{Heinla, Ahti} oli just see, kes ütles, et \enquote{ah, 
küll me välja ujume!}. Ujusimegi.

\question{Ühel hetkel te läksite mängu kirjutamise juurest ära ja kirjutasite 
Everyday. Kas see legend, et see käis kuidagi lehekuulutuse kaudu vastab tõele?}

Vastab tõesti, jah. See oligi just see raske hetk, kuskohas ma mõtlesin, et mis 
me nüüd teeme.

\question{Mis aastal see oli?}

See oli aastal 1999. Ühel reede hommikul vaatasin, ise olin Tartus, oli 
lehekuulutus, et pakutakse inimestele mingisugust ulmelist palka\sidenote{Teist 
perspektiivi sellele loole vaata Tarvi jutust leheküljelt 
\pageref{sisu:everyday}.}. Everyday portaal oli arendushädas ja Tele2 oli 
börsile lubanud, et kohe tuleb sihuke asi välja. Nad olid juba mingi aasta või 
paar arendanud ja olid välja tulekust kaugel. Kuulutuses oli  pikk nimekiri  
nõuetest, mida arendajad peaksid olema osanud. Pikk nimekiri asjadest, millest 
ma elu sees kuulnud ei olnud. IMAP ja POP3 ja PHP ja SQL ja mingid niisugused 
asjad. Tähtaeg oli esmaspäev, oli reede, mina olin Tartus ja teised olid 
Tallinnas. Mäletan, et helistasin Ahtile\index[ppl]{Heinla, Ahti},  rääkisime 
läbi, mõtlesime, et proovime, vaatame, mis juhtub. Tegime kohe  nädalavahetuse 
plaani ja põhimõtteliselt esmaspäevaks oli valmis prototüüp sellest portaalist, 
mida nad tahtsid. Kui me esmaspäeval intervjuule läksime, oli meil dilemma, et 
kas me ütleme, et see oli ühe nädalavahetusega kirjutatud või mitte. Ta nägi 
väga hea välja, kuna meil oli palgal arvutimängudega karastunud kunstnik, 
näiteks. Ma arvan, et prototüüp nägi parem välja, kui lõpptoode. Ja toimis 
täitsa, võisid  sisse logida, erinevaid paneele ringi lohistada, võisid emaili 
kirjutada, uudiseid lugeda, ilmateateid, mida iganes. 

\question{See tähendab ju, et tolle nädalavahetusega pidi sigima päris hea 
arusaam sellest, kuidas HTML ja brauseri renderdus ja muu selline töötab?}

Andmebaasid. Mäletan, et Priidule\index[ppl]{Kasesalu, Priit} jäi 
andmebaasidega tegelemine. Ta sattus hätta, ei saanud  loogikast aru. Ja ma 
mäletan, et ta võttis telefoni, helistas mingile andmebaasieksperdile, kahjuks 
ei mäleta, kes see võis olla. Oli laupäeva hommik. Kuulsin seda kõnet kõrvalt, 
et \enquote{kuule, mul on üks niisugune kogemus, on sul nagu hetk aega? Aa 
okei, okei.} ja pani toru ära. Ei olnud aega. Hea küll, tagasi uurima, kuidas 
SQL  käib uuris välja. Sai tehtud.

\question{Kõlab üsna ulmelisena, seal peab ju olema mingi meetod taga, kuidas 
seda teadmist omandada?}

See oli väga äge kogemus, jah. Ega tegelikult tehnoloogiad toona ei olnud super 
keerulised, nad olid meile lihtsalt võõrad. Ja meil oli tiim tõesti äge tooma 
ning saime tööjaotuse tehtud: igaüks pidi mingi kindla aspekti välja uurima. 
Magasime natuke, mitte eriti. 48 tundi tundi tööd.

\question{See tähendab, et te pidite kuidagimoodi oma tööd ka koordineerima, 
kes seda kampa teil juhtis?}

No mina olin nii-öelda ametlik juht. Samas see tiim töötas ise ka päris hästi. 
Välja arvatud, jah, võib-olla kunsti pool, mis põhjustas võib olla kõige rohkem 
meelehärmi, et kuidas Kasparilt\index[ppl]{Loit, Kaspar}  lubatud asjad kätte 
saada. Kunstniku asi, rohkem boheemlane, kui teised.

\question{Arvutades leiame, et kui kuulutus oli 1999 ja keskkool 
üheksakümnendate alguses, siis te kogu kümnendi kirjutasite mänge?}

Jah, päris mitmeid mänge tegime, kutsusime ennast Eesti mängutööstuseks. 

\question{Kui suur see tiim oli?}

1999. aastaks ega ta väga palju suuremaks ei läinud. \emph{Core} tiim oli siis 
mina, Priit\index[ppl]{Kasesalu, Priit}, Ahti\index[ppl]{Heinla, Ahti}. Artur 
Vill\index[ppl]{Vill, Artur}, kes oli 3D-kunstnik ja kes muide on teinud 
sellise filmi nagu Happy Feet mingid \emph{landscape}-d ja maastikud. Ta kolis 
pärast Bluemooni kokku kukkumist Austraaliasse ja seal tõusis tähelennuna, väga 
kõva vend 3D modelleerimises ja kunstis.

Ja kõrvalt Kaspar\index[ppl]{Loit, Kaspar} tegi kunsti, Ott\index[ppl]{Aaloe, 
Ott} ja Glen Pilvre\index[ppl]{Pilvre, Glen} tegid muusikat. Juhan 
Soomets\index[ppl]{Soomets, Juhan} tegi ka nagu poole kohaga 3D-graafikat ja 
vist oligi kõik. Kui bluemoon.ee lehele minna, siis see tiim on seal siiamaani 
üleval.

\question{Isegi toonase tehnoloogia lihtsuse juures pidi teil siis ju selle 
väikse tiimi peale tööd palju olema?}

Tööd oli päris kõvasti jah. Põnev oli ka muidugi.

\question{Mis see põnevus oli? Kui ühe mängu valmis olite teinud, kas siis 
igavaks ei läinud?}

Mängude tegemine ongi selles mõttes äge, et see on  niivõrd palju rahuldust 
pakkuv, kui mingi asi tööle läheb. Kirjutad mingisugust andmeanalüüsi. Kui asi 
tööle läheb, tuleb ekraanile õige number. Aga kui mängus asi tööle läheb, tuleb 
vägev plahvatus, näiteks. Või tulevad mingid väga, sellist, rahuldust pakkuvad 
stseenid,  efektid või lood või midagi sellist. 

\question{Nojah, vaade mingisse teise maailma, millest sa oled nüüd järgmise 
tüki loonud.}

Just, jah. Ja nüüd sa saad seal testimise käigus  ringi käia ja mingisugused 
kohati väga vapustavaid vaateid, sündmuseid, mis on toimunud\ldots

\question{Ühte asja ma tahtsin veel küsida. Kui sa rääkisid, et sa tegid üksi 
tekstiredaktori ja seal pidi kõik asjad ilusti reas olema, sellest ma saan aru. 
Aga kui meeskonnana softi kirjutada, siis see vajab ju \emph{software 
engineering}-u protsesse ka, kust teil need tulid?}

Üheksakümnendatel olid lihtsalt zip-failid ja \emph{backup directory}-d. 
Versioneerimist või selliseid  asju me üldse ei teinud. 

\question{Aga kuidas te siis tagasite, et see kupatus teil kokku ei kukkunud?}

Me olime väga ettevaatlikud! Üks põnts, mis meil juhtus, oli see, et meil murti 
kontorisse sisse ja varastati arvutid ära. Sealt läkski mingisuguse SkyRoadsi\index{Mängud!SkyRoads} 
või mingi asja mingi versioon. Meil olid diskide peal \emph{backup}-id ja 
midagi jäi alles, aga mingisugused asjad Bluemooni ajaloost läksid lõplikult 
kaduma. 

\question{Siiski, kas te oma töökorralduse mõtlesite lihtsalt jooksu pealt 
välja?}

Istusime telefoni otsas, põhimõtteliselt\sidenote{Huvitav, kust see Skype idee 
küll sündida võis?}. Mina olin Tartus, Ahti\index[ppl]{Heinla, Ahti} kolis ühel 
hetkel Tallinna  tagasi. Istusime telefoni otsas, koordineerimine käis ka meili 
teel. Eks meil tekkis spetsialiseerumine ka. Mina manageerisin tiimi, 
kunstnikke, kirjutasin mingisuguseid tarkvaralõike. Priit\index[ppl]{Kasesalu, 
Priit} spetsialiseerus operatsioonisüsteemi asjadele, graafikale,  Windowsi API 
ja sihukesed asjad. Ahti\index[ppl]{Heinla, Ahti} tegi sellist rohkem 
teadusmahukat asja, kuskohas oli vaja midagi AI-laadset või siis mingisugust 
matemaatikat.  Kui vaja, tal oli võtta. Et \enquote{ahaa, ma tean, selle jaoks 
on siin sellel leheküljel Knuthi Art of Computer Programming-us\sidenote{Knuth, 
Donald E. Art of computer programming (TAOCP). Tuntud ka kui Knuthi Piibel. 
Tegu on monumentaalse teosega, mille seitse köidet pidid katma kogu teadaoleva 
arvutiteaduse. Praeguseks on ilmunud kolm köidet ja esimene osa neljandast. 
Kuna teise köite küljenduse kvaliteet lugupeetud autorit ei rahuldanud, lõi ta 
oma raamatu ilusaks tegemiseks süsteemi TeX, mille derivaatide abil kirjutab 
praegu oma artikleid kogu teadusmaailm ja mille abil on kujundatud ka käesolev 
tekst.} on õige algoritm, teeme selle!}. 

\question{Tundub siis, et kuna meeskond töötas tiimina hästi, siis see lahendas 
ka üksiti ära \emph{software engineering}-u probleemid. Ei tekkinud mingeid 
merge konflikte ega probleeme, sest te töötasite inimlikult nii hästi koos.} 

Tiim oli väike ka, räägime kolmest programmeerijast. 

\question{Kui sa vaatad tagasi enda kui programmeerija peale üheksakümnendatel, 
oskad sa kuidagi kirjeldada enda arengut?}

Põhiasi, mis, ma arvan, domineeris seda arengut, oli see, et arvutid läksid iga 
kahe aasta tagant kaks korda kiiremaks. Mistõttu oli vaja kogu aeg hoolitseda 
selle eest, et sa ajale jalgu ei jää. Lõpuks me ikkagi jäime, mängutööstuse aeg 
sai läbi. Kümme aastat, arvutid läksid selle aja jooksul mingi, mis see siis 
on,  kolmkümmend korda kiiremaks. Ja  võimalused: heli läks rikkalikumaks,  
mälu läks suuremaks, graafika ägedamaks, võrgundus tuli juurde. Kogu aeg tuli 
ennast hoida aja tasemel. 

\question{Aga programmeerimise kunsti mõttes? Mitte see, et kas ma tean üht või 
teist API-t vaid kas ma olen programmeerijana täna parem, kui eile?}

See on huvitav küsimus, ma väga palju ei ole selle peale mõelnud. Kindlasti  
kogemus õpetas. Ma ei oska praegu  tagantjärgi seda kuidagi kompresseerida. Ma 
tean, et programmeerijana arengud on mul pigem nagu hiljem olnud, võib-olla ma 
aga mäletan hilisemaid arenguid paremini. See, kus ma kolisin rohkem 
funktsionaalse programmeerimise peale,  juhtus peale Skype'i. Kuni Skype'i 
lõpuni ma ikka kirjutasin oma vanade tööriistadega.

\question{Pärast Everyday intervjuud, kui kiiresti te tolle \emph{production} 
versiooni välja lasite?}

Ma hästi ei mäleta, aga see võis olla nii, et  suvi otsa kirjutasime ja kuskil 
sügisel või umbes nii tuli välja. Mingi esimese versiooni jaoks võis kolm-neli 
kuud minna.

\question{Sellesama väikse tiimiga?}

Põhimõtteliselt küll. Kuigi nüüd oli nii, et seal oli juures mingeid rootslaste 
tehtud asju ja see väike tiim oli osa  palju suuremast organisatsioonist. 
Mistõttu läks asi ka oluliselt aeglasemaks. Mingeid asju oli vaja rootsi keele 
tõlkida. Ma mäletan ükskord öösel sain Niklaselt\index[ppl]{Zennström, 
Niklas}\sidenote{Niklas Zennström, hilisem Skype asutaja.}, meili 
rootsikeelsete vastetega inglisekeelsetele fraasidele ja all oli 
\enquote{midnight translation services by Niklas Zennström}.

\question{Sellised teenused siis. Mis Niklas tegi seal projektis?}

Niklas oli everyday.com-i CEO. 

\question{Niklas oli siis see mees, kes ei suutnud kogu oma rootslaste tiimiga 
tarnida?}

Seda ma ei tea täpselt, kuidas see atributsioon seal täpselt  oli, aga Niklas 
oli põhimõttelist see, kelle lõplik otsus oli see, et Eestist arendaja otsida. 
Linnar Viik\index[ppl]{Viik, Linnar} vist oli pakkunud, et võtaks Eestis 
programmeerijaid ja Niklas oli see, kes otsuse langetas, et Bluemooni 
seltskonda kaasata.

\question{Kuidas tiimi skaleerumine tundus? Kui te olite kõik see aeg 
kirjutanud kompaktses kõgproffide tiimis keerulist softi, siis veebiarenduses 
on rõhk ju mujal?}

Ma hästi ei mäleta, et seal mingisuguseid olulisi probleeme oleks olnud peale 
selle, et kohe tuli  kommunikatsiooni ülesanne juurde. Nagu ikka, kui on kaks 
programmeerijate tiimi, siis esimene reaktsioon kõigil on, et \enquote{see on 
teise tiimi bugi}. Neid asju  tekkis kohe kõvasti. Aga ma ei mäleta, et oleks 
mingi tohutu külma vee kaela saamine olnud. Saime hakkama küll. 

\question{See kõik toob meid 1999. aastasse ja seega ka otsapidi väljapoole 
meie ajahorisonti, milleks on kaheksa- ja üheksakümnendad. Mitte, et pärast ei 
oleks igasugu põnevaid asju veel juhtunud.}

Enamus asju juhtus hiljem!

\question{Aga inimeseks said sa ju varem. Mis sa praegu teed?}

Peamine ja kõige olulisem tegevus on hoolitsemine selle eest, et juhul, kui 
inimajastu peaks mõne aastakümne (loodetavasti mitte mõne aasta) jooksul 
lõppema,  inimesed siia planeedile alles jääksid.

\question{Ära pane pahaks, aga ma hästi ei näe mõtteliini ekraani peale 
plahvatuse joonistamisest selle teemani. Palun selgita!}

Sinna mahub üks kuni kaks aastakümmet veel, ehk see, millest me ei ole 
rääkinud. 

\question{Sa oled lihtsalt jõudnud selleni, et see on sinu jaoks oluline 
probleem?}

Jah, selleni jõudsin ma aastal 2008 või midagi sihukest, kui Skype'is juba hoog 
hakkas raugema, sealt enam väga väljundit ei olnud. Sattusin rääkima 
inimestega, kellega me viimased kümme aastat olen ehitanud sellist 
\emph{community}-t, kes üritavad teha ära AI-uurijate kodutöö. Ehk siis teha 
asju, mida on vaja selle jaoks, et AI-ga tulevik oleks inimestele soodne, aga 
millega AI-arendajad ise ei ole näidanud mingit motivatsiooni tegeleda peale 
selle abstraktse motivatsiooni, et nad on ka inimesed.

Üks võimalus probleemi kirjeldada on see, et meil on fundamentaalne 
\emph{trade-off}, ma isegi tea, kuidas seda eesti keeles öelda. Et sa ei saa 
nagu kahte asja korraga. Super kompetentset süsteemi ja sellist süsteem, mille 
üle sul on täielik kontroll. See ei ole isegi arvutite spetsiifiline probleem, 
inim-juhtidega on sama probleem: mida rohkem ta delegeerib, seda 
kompetentsemaks muutub süsteem või suuremaks kasvab organisatsiooni võimekus. 
Aga tema isiklik kontroll selle üle, mis toimub, väheneb. See on fundamentaalne 
printsiip. Ja mida inimkond praegu teeb, iga päevaga järjest rohkem, ta 
delegeerib oma otsuseid masinatele. Mistõttu sellise delegatsiooniga tegelikult 
väheneb inimeste kontroll tuleviku üle. Võib juba öelda, et praegu on inimeste 
kontroll tuleviku üle väiksem, kui see oli näiteks viiskümmend aastat tagasi. 
Ja see tendents tõenäoliselt jätkub. Nüüd on küsimus see, et kuidas me siiski 
säilitaksime kontrolli mingisuguste oluliste aspektide üle. Näiteks atmosfääri 
koostis, mis on meile oluline. Temperatuur, mis tundub juba praegu keeruline. 
Räägime siin veebruarikuus, väljas on kolm kraadi sooja, sajab vihma. Juba 
inimestel on raske keskkonna üle kontrolli säilitada. Lisame siia entusiastliku 
delegatsiooni arvutitele, kellel on keskkonnast täiesti ükskõik! Sellepärast me 
saadamegi roboteid radioaktiivsetesse aladesse või kosmosesse, et neid keskkond 
ei huvita. Probleem on selles, et AI arendajatel on motivatsioon aretada just 
nimelt delegatsiooni poolt, et delegatsioon oleks võimalikult lai ja tulemus 
oleks mingi meetrika järgi võimalikult kompetentne. Ja palju vähem on 
motivatsiooni selle jaoks, et mõelda selle peale, et kuidas kogemata mitte 
delegeerida selliseid asju, mida meie elus olekuks on vaja.

\question{Ma südamest loodan, et sul tuleb välja, sest muidu on pahasti!}

Ma tihtilugu ütlen inimestele, et \enquote{wish me luck, you are going to need 
it!}


\chapter{Taavi Talvik}
\index[ppl]{Talvik, Taavi}

\question{Kuidas ja umbes millal sa jõudsid arvutite juurde?}


Tere, see siin on Memkopi ja meie teine hooaeg nagu ka eelmisel hooajal on plaanis viisteist episoodi, kuid erinevalt eelmisest on selle hooaja alguseks kuus episoodi juba purgis, seetõttu loodetavasti vähem viimase minuti rabelemist. Samuti olen natukene parandanud ristitarkvara olukorda ja audio kvaliteediprobleeme, rohkem ei tohiks esineda. Aga külas on meil täna Taavi Talvik mees, kes nagu ka meie teised külalised üldiselt tutvustamist ei vaja. Kuid kelle käsi on olnud mängus eesti internetiseerimise juures tegevuse tagajärgi ühel või teisel viisil kõik tajunud olema, sest tema tegi niisuguse asja nagu uninet. Ja see on suur asi Eesti internetis. Head kuulamist.
Tere, tere. Sinu nimi on minu nimi, on Taavi Talvik.
Oleme kogunenud siia suurepärasesse kontorisse, kus me oleme juba korra varem käinud rääkimaks sellest, millest ei ole võib-olla ammu hästi räägitud. Ehk siis sellest, kuidas asjad alguse said. Aga hakkame sellest pihta, kuidas asjad sinu jaoks alguse said. Kuidas arvutite juurde jõudsid?
Arvutite juurde jõudmine iseenesest on väga lihtne, et kodus sattusid olema paar põnev põnevat draamat, Totteta ise isegi enam ei mäleta, mis see täpselt oli, kas see oli mingi Ustus, Aguri abakusestraalini või või, või mingisugune Norbert Wiener'i mingi küberneetika või? Igal juhul mingi niisugune raamat oli, tundus jube põnev, et ja noh, siis kuna esivanemad olid tööl Tartu ülikooli juures keemikutena ja, ja.
Olid kuulujutud Ülikoolis ikka mõni arvuti on, et siis ma kohe hakkasin neile pinda käima, et kuulge, et ma tahaks panna ja näha, missugune see arvuti päriselt välja näeb.
Aga kas ühesõnaga saad siit kohe järeldama, et sa oled Tartust?
Jah, ma olen Tartust, et selles mõttes tort oli jumala okei, niisugune väike väike linnakene Elva lähedal ja lapsepõlves mulle seal väga meeldis väike puust linn. Kui vana sa olid, kui sa vanematega hakkasin, ma arvan, et see oli kuskil ütleme niimoodi, üheksas kümnes klass pigem üheksas ja, ja tõepoolest neil seal ülikoolis arvutid oli, siukseid, väljamaa omi isegi. See oli aasta umbes kaheksakümmend viis ja, ja siis välja oma arvuti oli suht niisugune haruldus, aga, aga kuna nad tegid mingisuguseid imelik elliptiliste kilede mõõtmisi, siis selle kilede mõõtmismasinaga oli kogemata ta koos ostetud mingisugune arvuti, mille nimi oli Hewlett-Packard kaheksakümmend viis. Oli selline lauaarvuti, ei olnud, see oli lauaarvuti, kus oli sees pisikene. Ma ei tea, viietolline ekraan, klaviatuur, kassetid jaa, jaa, jaa. Termoprinter ning taga oli hunnik juhtmeid, mis ühendasid teda siis selle mõõtmismõõtmisseadmetega, niisugune mudel ei ole küll läbi kellelegi jutust, et kõlab täitsa nagu eksootilisi, see on iseenesest väga eksootiline mudel. Ja tal oli mingisugune oma Hewlett Packardiprotsessor, mis omas omas ajas, oli isegi täitsa innovaatiline ja tore. Kuigi protsessor, protsessor, eks millega see välja paistis, oli tavaline peesik ja see, et ekraani peal sai jutte joonistada ja, ja ütleme kui ise jutte joonistada ei osanud, siis sai mingisugust pingpong pingpongi või kosmonautide maandumist mängida, aga siin tõesti kohe siis sinna juurde. Näe, poiss, võta läbi. Jah. Teie täiesti niimoodi ja noh, eks nad, eks nad sealt, et vanemad inimesed, kui ma õieti mäletan, siis tema nimi oli Zirk õpetasid ka, mis, et kaua sa siin mängid, et proovi kokku liitorve ühest kümneni või midagi sihukest, noh sealt need asjad pihl pihta hakkasid. Okei, koolis ei olnud mingisugust nihukest, just kuulsin Tartu kümnes keskkool, mis on tänapäeval siis Mart Reiniku gümnaasium. Koolis ei olnud see aeg veel mitte midagi, täitsa täitsa täitsa tühi maa. Tõenäoliselt samal ajal noos midagi oli, aga, aga nõo Tartust nii kaugel ja noh, selleks peab ikkagi nõos tutvusi olema. Tseme seid, nagu keegi kutsuks.
Kui ma mõtlen siukse üheksanda ja kümnenda klassini peale, siis seal kipuvad igasugused muud põnevad hobid olema, selle asemel, et lugeda härra viineri küberneetikat või ka agulis
Teost, miks sa lugesid seda? Tore oli, et huvitav oli ja, ja noh, võib-olla vanemad sokutasid ka midagi, et loe, poiss, et, et järsku saad targemaks või midagi nihukest. Noh, eks see tagantjärgi tarkusena sekrilt enam ei mäleta, mis see täpselt see vajanud oli.
Seepärast küsin, et kas sul oli, oligi populaarteaduslike asjade huvi.
Kui oli hea, võiks selles mõttes hooli populaarteaduslike ostjate huvi oli ulmehuvi ja kuna nagu nupp selles reaalteadustes jagas alates igasugustest nendest olümpioodidesti asjadest, siis see tundus nagu naturaalne
Oleks nagu loogiline ja sealt edasi. Kuidas sul see reaalteaduste jagamine nagu, nagu esile kerkis, kas sa kohe nagu esimesest klassist alates tundsin ennast selles osas mugavalt, tegeles keegi su arendamisega spetsiifiliselt? Ma arvan, et põhikool või midagi.
Tundsin suht mugavalt tänu sellele, et isa-ema olid ülikoolis õppejõud ja aeg-ajalt nad ikka keemikutega midagi midagi rääkisid ja, ja ja alati nende käest on ju alati võimalik küsida, et kui ma kuskil füüsikas, keemias, matemaatikas etajain ja noh, kui sa hädast üle saad, keegi sind hädast üle aitab, siis siis endal tekib mugav tunne ka, et ja, ja ei saa nagu vastane väsis.
See on hea turvaline toimetele täpselt. Aga kõlab ju, mis sa siis keemiku teed ei läinud?
Arvutid olid põneva. Peale seda, kui üks etapp oli antud, siis siis see põnevus järjest järjest lihtsalt kasvas.
Seal nagu noh, selle ühe HP selle kaheksakümne viiega või mis ta oli, et ega see ei saanud ju nagu väga kauaks põnevaks jääda.
Eks seal oli, olid omad asjad, et natukene mingid trips traps, trulli laadseid mänge kirjutada ja sai selle esimese hea edukogemuse kätte ja see sealt siis edasi sai kuhugile järgmistesse kohtadesse kus olid siis veidi ägedamad ja võimsamad arvutid.
Et keemiahoone kõrval oli Tartus olemas just kõrval, aga, aga lisaks oli ka füüsikahoone, kus, kus oli see selline inimene nagu Alo raidaru, kellel oli päriselt kuskilt saadud PC-laadsed arvutid sama tähe tänaval läenäosutaja neli kuskil seal keldrikorrusel Tauli ja pissilaadsete arvutitega sai juba teha väga palju rohkemat, kui selle väikse õnnetu happe kaheksakümne viiega. Ja noh, lõpptulemusena siis umbes kümnendast klassist alates füüsikahoones võeti mind nii-öelda laborandina tööle ise endisi, nii isegi niimoodi ja siis siis tööülesandeks oli üht või teist või kolmandat või neljandat programmeerida. Kümnenda klassi poisina.
Oh, kus sul see programmeerimisoskust
Tuli siis no ma ei tea, tuli järjest kõik, kõik, kõik aitasid kõrvalt ja õpetasid ja noh, kuidagi sisse naturaalselt kasvas.
Okei see kõlab päris nagu kiire normaalniukene, normaalne kasvamine, et tühja koha pealt nagu laborandiks.
Aga ei noh, seal selles mõttes ma loodan, et ma tegin seal isegi midagi kasulikku ja, ja, ja ma sain sellest palka ja sain palka ikka täitsa kõvasti laborandi palk oli mingisugune viiskümmend rubla kuus, mis kümnenda klassi poisile oli sihuke sihuke Kresuse tunde tekitas, et see oli väga palju. Ja, ja eks see noh, laborandi pank natuke toetas kõiki neid huvisid ja värke. Tulemus oli see, et enamasti peale koolitundide lõppu oli mitte koju minek, vaid sinna füüsikahoonesse minek.
See on ju, see on ju arusaadav? Ikkagi mind, mind ei jäta rahule see, et, et kui sa niimoodi ise arened, siis peab olema mingisugune niukene huvi, mis siit viib selle asja juurde.
Absoluutselt noh, mis tegi põnevaks. Põnev põnevaks tegi ikkagi see, et kui sa arvutile mingisuguse programmilaadse asja selgeks teed siis ta teebki midagi, mida sa arvasid, et ta võiks teha. Tihtipeale ei tee, aga väga tihti ta tegi ka ja see oli jube kihvt, kui, kui midagi juhtus. Mingi asi ollus sinu kurjusel, mingi asi allus minu korraldusele, noh, täiesti niisugune unikaalne situatsioon maailmas.
Jah, eriti Tiina üles ja kontrolli puudumine võib olla päris absoluutselt suur probleem. Ja oskad sa mõnda näidet tuua, et mis sa seal laborandina prognoosisid?
Laborandina progesin. No üks asi, mis kindlasti meelde tuleb, on antiviiruse antiviiruse kaheksakümmend millegagi kaheksakümmend kuus umbes selles mõttes, et siis sihuke
Avastushetk, et maailm hakkas vaikselt lahti minema ja vaikselt liikuv liikusid, ilmusid viirused Eestis sega ja, ja siis tekki, tekkis see jama, et viirus jõudis sinna meie juurde ka ja kui sellest oli koja kuidagi lahti saada, kuna arvutid hakkasid imelikult käituma. Viiruse nimi oli, kui ma õigesti mäletan, jänki Tuudel mis tegi siis piikse ja ekraani peal vist hakkasite?
Kukkuma jah, tähele hakkasid kukkuma ja siis ta mängis jänki Tuudeid ka, kui ta riistvara võimalik
No ja ja niisugune viir viirus oli ja siis sai uuritud, kuidas see käitub ja siis tehtud pisike programmikene, et sellest lahti saada oli täitsa võimalik. Aga noh, see on niisugune lihtsalt tore mälestus. Põhiline, mida seal Aloiduru laboris tehti, oli tehti elektroonikat füüsikutele ehk mingisuguseid lisasid nendele mõõteseadmetele, katseeksperimentidega ja nii edasi ja seoses sellega nad tegid ise trükkplaate ja trükk plaatide tegemiseks olid esimesed need Smart käädi või laadsed programmid, millega õnnestus elektroonikaskeemi joonistada, trükkplaat joonistada ja taga siis kas välja printida või siis Alo ehit, kuidas arvuti külge freespingijuhtimise interfeissimis free siis selle trükkplaadi välja. Ja, ja aga noh, lisaks välja freesitud trükk, padi radadele oli vaja seda, et Need läbiviigu augud ka puuritakse ja see näiteks läbiviigu aukude puurimise programm oli see, mis usaldati mulle.
Aga nüüd, kui ma nüüd siis moodsasse terminoloogiasse kohe niimoodi tõlgin, siis sa tegelesid kohe esimese hooga iiluateega.
Jah, see, seda võib tänapäeva Liioteega viot teeks nimetada, aga noh, tegelikult oli see trükkplaatide aukude puurimine ugri nimetame Rubootikaks. Ikkagi nimetama õige asja nimedega puurpingi puuri, õigele kohaleviimise ka siis käsuandmised, vajumine olla ja tule üles tagasi.
Ja see pidi siis lahendusena olema, kuna seal tõenäoliselt mingisuguseid valmisid, teeki või valmisliideseid ei olnud, see oli siis nagu sinu programmist kuni restorani välja.
Põhimõtteliselt küll, et sellest samast käädi käädi programmis sai aukude koordinaadid. Jaa jaa. Nende koordinaatide peale lihtsalt tuli see augud puurida sena seal vahepeale sai mingi puuride liigutav puuri liigutamise keel, et vastu sada sammu siiapoole, mine olla vastu, sadas hamba sinnapoole tule üles. Ei. Kohe tuleb üles tulla ikkagi. Ja nii edasi.
Ja selle keele mõtlesid välja sina.
Ross, eks need vanemad inimesed kõrvalt ikka aitasid, et et nii kõige kinni jooksid, õige tuli keegi ikkagi appi.
See on hästi turvale jällegi absoluutselt hästi turvaline. Kasvukeskkond kasvatab, teised on rääkinud, et, et nad üsna varakult saidingi teiste omasugustega kaar ninapidi kokku vahetasin infot ja tekkis mingisugune niukene kogukonna moodi asi.
Kooliajal ei olnud aga pärast kooli ülikooli astudes see tekkis üsna üsna kohe tekki tekkis kogukonnatunne ülikooli esimesel kursusel.
Aga ärme veel ülikond ülikooli lähme, et see keskkoolis õppimist segama ei hakka.
Ei otseselt ei hakanud.
Kuna nupp natukene lõikas, siis võis mõne mõne koha pealt üle nurga lasta, et ei ole. Et noh, selles mõttes koolis jah, lõputult pingutada ei ole vaja, võib-olla seal eesti keele kontrolltööd läksid kehvemaks, aga aga noh, ütleme üldtase sinna nelja juurde jäi. Täitsa okei, ütleme kooli lõpetamisel oli, oli tunnistus umbes selline, et kõik olid neljad, välja arvatud üks, viis, üks, kolm sest keskmise küll ütleksid neljaks mis see kull enam ei mäleta, lihtsalt ei mäleta, või ma võimalik, et oli vene keel, aga ma hetkel enam ei mäleta.
Vaat nüüd me jõuame ülikooli juurde, et see, mis koolis läksime
Tartu Ülikooli füüsika füüsikateaduskond.
Ja seepärast, sest seal sa juba olid laborandina.
Laborandina oli käsi sees ja, ja noh, tegelikult füüsikaga nähtus huvitas ka ja, ja selles mõttes tegelikult füüsikaga nähtuse huvitas oluliselt rohkem kui matemaatika. Et füüsikas oli, oli nagu see
Keerukusaste väiksem selles mõttes, et matemaatika, matemaatikud, need läksid mingid teise tuletise või seitsmenda tulebki sinna välja ja ja, ja samas kui füüsikud ütles, et, et teine tuletis, see on nii ebaoluline juba, et seda efekti sellel kursusel ei aruta, see jääb kolmanda kursuse materjaliks ja see mulle jumalast sobiks.
Muidugi väga huvitav, sest mina just mõtlesin vastupidi, et, et füüsikas on ikka päris maailm ja see on nagu messi ja keeruline ja seal on mingisugune.
Ei, vastupidi vastupidi, et seal on mingisugused, suhteliselt lihtsad rusikareeglit, kui kui nendest suhteliselt lihtsatest rusikareeglitest aru saada, siis need noh, peenhäälestamine, see tuleb peale ja nagu ma ütlesin, see tuleb nagu järgmise kursuse materjalist esialgu kõrvale jätta.
Okei, nii võttes küll jah. See, aga nagu huvitav, olidki seal mingi spetsialiseerumine ka seal tekkis.
Vot spetsialiseerumisega läks natukene natukene sandisti, sellepärast et kohe peale tuli Vene sõjavägi ja, ja, ja peale Vene sõjaväkke ma küll täitsa jätkasin füüsikas kaks pool aastat, aga, aga noh, siis tuli ka muu elu kõrvale ja siis nii-öelda õppimisvaimustus vaikselt. Ütleme kuidas seda viisakalt öelda, siis laius mõjus väga viisakalt. Kus sa teenisid Valgevenes selline koht nagu Borissov kolmteist on niisugune super koht.
Ei ütle mitte midagi, ilmselt mitte kellelegi.
Tõenäoliselt mitte kellelegi alla kõigele, kes oldi. Aga Valgevene ja, ja eks ta oli sihuke sihuke mõnes mõttes ajaraiskamine, teisalt siuke, et sa nägid maailma, et kui palju erinevaid inimesi tegelikult olemas on.
No see arvutiinimesed on üsna niukene nagu silmi avav.
Ilmselt jah, on selles mõttes, et kui palju on palju, palju meist ütleme kaheksakümne kaheksandal, kui ma sõjaväe läheksin, palju meist reisinud, tegelikult olime võib-olla Nõukogude liidu piires siin-seal kuskil käinud, aga see, ega see reisimine ei olnud niisugune nagu teema, mida kõik on teinud ja see, see uute inimeste nägemine tegelikult oli sele sõjav seda päris kasulik kogemus tagantjärgi.
Nojah, sest seda ma just pidasin seda silmas, et arvuti inimesel on tõenäoliselt nagu eri tüüpi arvutite eri programmeerimiskeelte niisuguste asjadega nagu suurusjärk, rohkem kogemust kui eri tüüpi inimeste ja seda kindlasti kultuuride siukse asjaga. Vot siis sa tõid tagasi aastani.
Tagasi tulin, aasta oli kaheksakümmend üheksa, mul õnnestus Vene sõjaväest pääseda ühe aastaga, kuna Gorbatšov ütles, et üliõpilased nemad on meie sotsialistliku riigi tulevik ja minge õppige ülikoolis parem edasi, mitte ei jookske püssiga, ärge jookske püssiga ringi.
Ja siis tulevik tuli Tartu ülikooli edasi.
Tulevik tuli Tartu Ülikoolis edasi füüsikat õppima ja noh, siis ma proovisin ka spetsialiseeruda astronoomia peale. Joosta, surume jälle niisugune niisugune juhus, et,
Tõenäoliselt sa tead sihukest ulmekirjanikku nagu Vaisaka sõimub, olen kuulnud ja suure tõenäosusega oled kuulnud, et lisaks sellele, et ta oli ulme, kirjalik, ta oli jube hea teaduse popule populariseerija ja, ja jälle kodusest raamaturiiulist oli mingisugune raamat, mille, mis, mille pealkiri oli siis universum. See oli küll tõsi küll, venekeelne Psylemm ja aga ta kirjutas niivõrd fantastiliselt seda, kuidas toimib. Et see nagu jäi kuklas kripeldama, et noh, et järsku peaks edasi uurima seda teemat ja, ja see tundub nii kihvt olema ja siis ma proovisin sinna spetsialiseeruda.
Mis, mis see tähendab, proovisid?
Kas see välja ei, otseselt otseselt ei tulnud välja, kuna sellest koolieelsest tööelust kasvas välja järgmine tööelu, mis hakkas natuke õppimist segama. Et seesama Palo, Raido sokutas mind tööle ajalehte edasi.
Ahaa et siis ei olnud ju vaja.
Edasi ei olnud auke puurida vaja, aga aga ka kaheksakümne üheksas aasta oli juba see aasta, kus reaalselt tekkisid ka ettevõtetesse esimesed arvutid. Jaa jaa.
Tekkis niiöelda esimene desktop abishing. Ja kuna üks selle selle olo hobidest oli ülikooliteatmiku väljaandmine, siis tema käest küsiti nõu, et kuule, et meie saime Edasis ühe arvuti, et kas seda saaks kasutada kuidagi ajalehe väljaandmise abiks. Nii, ja siis mind, mind sokutati sinna, et kuule, Taavi, sina mine aita neid, see neid, neid edasi tegelasi siis Maitasingi näiteks neli-viis aastat.
Oh, see, see aitamine pidi siis olema puhtalt ainult ju selle Paulisingu programmi käimaajamine.
Ei, mitte ainult pobishingu programmi käimaajamine, vaid seal on võimalik. Reaalsuses on mingisugused töö rutiinid, et kui need lähevad lihtsamaks kevadel, kiiremini. Sulle nüüd omakorda küsimused, mis sa pakud, et mis oli esimene asi Edasis, mida arvutiga automatiseerita.
Aastal kaheksakümmend üheksa eesti keele spellerit ei ole ju veel.
Eesti keele speller oli ka olemas juba, aga see selleks, mis võiks olla see teema, mida automentiseeriti ei tea, ei oska. Väga lihtne see, see oli see teema, kust raha tuli ajalehte, surmakuulutused surma kuulasid. Et noh, tänavu tänaselgi päeval on Postimehes populaarne paar viimast lehekülge, kus on Need surma surmakuulutused, nende nendele on see hea omadus, et nad on suhteliselt standardses formaadis, seal on mingisugune noh, neli-viis, võib-olla kümmekond erinevat kujundust meile ja siis, kui need kümmekond erinevat kujundust kuidagi mallidena ära implanteerida, siis see surmakuulutuse nii-öelda Väljaandmise publitseerimise aeg kukkus drastiliselt. Nojah, ja, ja kuna see oli sisuliselt ainuke allikas, kust Lisat tellimusele raha tuli, siis selle vast oli lehe juhtkonnas ikka päris normaalne.
Okei aga mis see, mis see väljund?
Väljund oli kile peale trükitud. Selline.
Lehekülg, mis läks siis ofsettrükki?
Ühesõnaga sinu tarkvara, mis automantiseeris seal asja otsekilele.
Jah, otse kylal, laserprinteriga lased paberi abil läbi kile ja siis see asi, mis sealt välja tuleb, see on enam-vähem see, mida saab trükkalitele kätte anda, et kleepige siis.
Nendesse õigesse kohta südamesse produtseeris Postgüti.
See on juba päris keeruline.
No eks seal oli, seda pabistasin tarkvara. Kui ma õigesti mäletan, siis Ventura publis mis tegi põhitöö ära ja seda otseselt puskrüpti tasemele, oli väga mõne mõned üksikud asjad.
Aga see saab jälle kõik teadmist saadet, kus sa seda teadmist juurde hankisid.
No.
Istud ja nokid ja nii ta on, kuid leida tuleb lõpuks küll ta tuleb. Kus ta pääseb?
Kus ta pääseb? Sa ütlesid enne, et sul tekkisid seal juba mingisugused kogukonna moodi asjad.
Ja, ja kogu kogukonna moodi asjad, et üliõpilastena sa käid ikkagi seltskonnaga ringi. Proovid Ühes arvutiklassis proovid teises arvutis, D-klassis, see hetk olid juba tekkinud Tartu Ülikooli matemaatikateaduskonda ka arvutiklassid ja seal sealsete inimestega suheldes mingi kogukond vaikselt tekkis ja samamoodi tekkis kogukond ta nendest, kes olid mul kursusekaaslased. Füüsikas.
Luke tol ajal mingeid arvutiside
Tol tol ajal veel arvutisidet ei olnud. Aga eks see arvutiside tuli ka suhteliselt kiiresti, et ühtedel meest, et meestel oli ühte laadi arvuti, teistel teistlaadi arvuti ja flopikettaheide vahel ei lugenud, siis pandi kaks või kolm traati omavahel kokku ja prooviti neid jälle kuskilt saadud programme teisele mehele ka üle kanda.
Okei.
Ma küsin korra selle edasi kohta veel, et kuidas see töövoog välja nägi. Ikka tahad küsida, et see ajakirjanik kuidagi kirjutas, millega?
Siis edasi aegadel ajakirjanik kirjutas ikkagi kirjutusmasinaga ja siis oli tinaladu. Aga kui tekkis nagu rohkem arvuteid, siis siis läks dinoloolt üle ka ütlema selle kile peale trükitud väljundini. Seal oli terve hunnik etappe veel vahel, et.
Et ajakirjanikele arvutit saada, arvuti oli tol ajal suhteliselt kallis asi ja, ja terminalid olid natukene odavamad. Siis sai Postimehe see, kes oli siis juba erastatud ja postimeheks muutumas ja sai pandud üles üks juuniksi server, kus oli küljes kuusteist terminali mis siis ajakirjanikele maja peale laiali sai veetud terminalid ühenduseks. Need vajalikud kaardid sai Tõraverest, seal oli mingi uraani nimeline firma või millest kasvas välja Astordaata ja, ja.
Teksti sisestamiseks, ajakirjanikul oli terminal piisavalt lihtne, et, et seal ei olnud vaja tal seda ajada, ilusaks peasid, tekst olemas on. Ja seesama tekst siis võeti ja pandi mingi pabistamine, tarkvara ja jälle kile peale välja. Kleeplindiga kleebiti küljeks kokku, mis läks siis öösel trükikotta kunagi ja mis juuliks seal serveris juhtus. Seal serveris jooksis BSD juuniks mis sai täiesti ausalt ostetud soos koodiga kõigega kõige värgiga. Ja.
Nii ta oli tol ajal ju tol tol ajal oli ju lausa mingi embargo asi oli.
Embargot ja asjad olid ka, aga need kolm kaheksa kuuelaadsed arvuti demorga olla ei kukkunud. Ülemine ots, nagu pyydipi oli embargo all.
Ja sisse UNIXi purk käis ka sinna, mitte embargo segas mitte embargo alla. Siis ikkagi ta ka mingit erinevi jaksas vedada.
Jaak rahulikult jaksas need kuusteist terminali välja vedada ja, ja noh, eks arvuta hankimine tol ajal oli nagu sihuke keerukas tegevus, et a la kui telli Postimees oma tellimise rublad kätte, siis, siis veeti need kohvriga. Minge oskuslik ärimeeste juurde, kes siis kuskilt oskuslikult Moskvast said mingi arvutit. Et noh, see oli, too aeg oli nagu niisugune vorsti kaubaaeg. Ja, aga lõpuks oli need vajalikud arvutid võimalik välja ajada, vihikud on, me oleme sellest rääkinud just ja, ja mõned kohvrid jõudsid tema juurde ka ja tema siis oli.
Põhiline Postimehel arvuti tanki.
Jällegi mind pani imestama, kui sujuvalt läheb, nagu skoop läheb nagu laiemaks. Et kui ma sellest programmeerimise asjast veel saad aru siis sellest, et sul see huvi läks nagu laiemaks, nagu mingite uniksite serverite püstipanek, Pablishing ja nii edasi. Et mis sind hoidis nagu laiendamas seda, et oleks olnud lihtne nagu keskendutaks programmeerimisele või.
See ma arvan, et siin oli just siin oli just see, et see seltskond ümberringi
Postimehe ajakirjanikud või edasi ajakirjanikud, neil läks ka silm särama, et näed, niisugune võimalus on seda oma oma tööd paremini teha ja see tekitas nagu surve eelkõige selleks, et neid kuidagi aidata. Et kui, kui inimesel silm särab midagi tehes, kui sa teda aitad, läheb sinul endal ka silm särama ja noh, põhimõtteliselt nii lihtne see ongi.
Aga siis see eeldab ikkagi mingisugust elementaarset asja üldse sedastada mingit elementaarset huvi inimeste vastu ka arvutiinimeste hulgas. Arvutid.
No sellel absoluutselt, aga kui, kui sa igapäevaselt kellelegi kõrval istud, tahes-tahtmata huvi tekib ju, et ei ole võimalik, et ei tekiks ja, ja kui sa istud veel intrigeerivate inimeste kõrval, kes nii-öelda hoiavad kätt elu pulsil, räägivad sulle, oh, seal Tallinnas Toompeal räägiti seda ja toda ja okas paneme selle lehte või parema, hakkab kõrv liikuma küll ja, ja tahad selle, selle selle melu sees olla?
See isegi aega olla küll, sest ilmusid Nelli teatajale hakkasid ilmuma Nelly hiljem, see oli bossistoli siis.
Tartlasena ma neid kõiki Tallinna asju ei tea, hakkas ilmuma see Eesti Ekspress, Liivimaa kuller Kalle Mülleri ja Väino Koorberg vedamisel siis kroonika alustuseks Kalle Mülleri, Sindi, Ingrid Veidembergi vedamisel. Kõikide nende juures olid mingisugused momendid, mis olid nii-öelda mega kihvtid huvitavad, et alguses oli mustvalge, siis tekkis värviline logo siis ja nii edasi ja nii edasi.
Nii suur asi, kooli värviline logo, see ja kõigi nende juures oli mingisugune niisugune Pablisshingu või trükivõi niukene inimene, nagu ametis kas või Peeter Marvet juba kuskil tõmbutes tõenäosust.
Et tõenäoliselt Tallinnas tembutas, aga noh, Tallinn ja Tartu on siuksed erinevad asjad täiesti.
Aga seepärast ma tahtsin küsida, et kas nendel Paulistusega inimestel mingit oma üks kuu on tekkinud ikka vahetasin programmiga asju.
Kindlasti oli, aga vot see on see teema, mida enam ei mäleta, lihtsalt enam ei mäleta, et pärast oli neid asju nii palju peale, et.
Okei, aga ühel hetkel sai edasi asi otsatmise eesti
No edasi asi enne kui ta otsa sai, on kindlasti üks selline huvitav moment, millest ma tahaksin rääkida. Ta edasi ka seoses. Ma sain umbes aastal kaheksakümmend üheksa või üheksakümmend nüüd.
Ei vastuta kummagi numbriõigusest, sain emaili endale. Saidi emaili, jah, siukse emaili meil mille aadress selle umbes niimoodi, et Taavi jäät EVS-i.
E-tabeli punkt SO suunagu Soviet Soviet Union Eesti vabariigis ja niisugune domeen on olemas, jah, niisugune domeen oli olemas ja kusjuures domeen nagu SO on endiselt olemas.
Ja Siimile sõidumale. Jälle mõnes mõttes tänu ülikoolile, kuna ülikool oli ülikooli psühholoogia teaduskonda aasta Tiit mugavam oli käima ajanud Tallinnas.
Küberisse või KBFIsse seda enam ei mäleta, modemega uudsebee meilisideme ja, ja sealt siis tulise meiliaadress. Ja, ja siis sellest ajast ma veel mäletan, esi esimees niisugune suuremamahulist internetioste, interneti või meili umbeajamise intsidenti. Et noh, siis siis olid siuksed asjad nagu lingvistid, mida sai tellida ja, ja noh, siis ma kogemata tellisin mingi siukse aktiivsema kirjavahetusega meelingvisti ja siis kui meile muudkui tuli ja tuli, modem ei pannud toru hargile ja ei pannud ja pannud, pannud läheb tund ja teine ja kolmas täitsa pekkis on kogu see maailm umbe läheb katki ja siis ma võtsin jalad selga üle Toome, kõndisin sinna Tiidu juurde, kuule, aita mul see meilivoo kärakat Kestova. Et noh, selles mõttes siis oli umbes moodeme helistamine minu juures tema juurde ja tema juurest kuhugi Tallinnasse ja Tallinnast võlgu Soome ja mis iganes.
Kusjuures see, see on ju tänapäeval täitsa unustatud, kui oluline asi oli nii meili, mis kõik asjad käisid üle emaili, oli olemas niisugune asi nagu FTP üle emaili ja mingisugused failid, keerati õige sobiva pikkusega tükkideks ja lasti PS kuuekümne neljaga kokku.
Sa oled täpselt, eks ole, et siis oleks see sellise, sellisel kujul oli võimalik kuskilt list serveritest või arhiiviserveritest endale nii-öelda tellida vaba tarkvara lähtekoodi.
Ja isegi minu meelest isegi mingisuguseid pilte levitati.
Pilte levitati, aga seal see kõige huvitavam tol ajal oli. Meie jaoks oli see just see vaba tarkvara lähtetekstide kättesaamine üks huvitav selles mõttes, et siis sa said jälle mingi uut uut võimsamat asja teha selle sammu edasi juures sellesama edasi juures. Noh, eks seal järjest neid automatiseerimis asju tuli peale, et pinge tehti oma kojukanne. Et ma kojukande jaoks oli jubedalt abiks, et postiljoni tele, Need pakid, jagataks siht sihtrajoonide järgi, õiged kleepsud oleks peal õigesse hunnikusse, õige õige kogus ajalehti saadaks, oleksid neil oleks nimekirjad, mille mille järgi viia ja, ja sihuke kujukond infosüsteem sai näiteks tehtud ja siis on jälle jälle, mida tänapäeval ei, täna see kindlasti on olemas Eesti Eesti postis või kommunimas on kojukandeinfosüsteem raudselt olemas.
Just, aga see on asi, mida tänapäeval sagedasti ei juhtu, et sa astud uksest sisse, hakkad nullist programmeerima mingisugust kujuga impostimine, nagu tüüpilist võetakse mingi asja alla.
No see oli see aeg, kus ei olnud võimalik võtta olla lihtsalt ei olnud aga, aga, aga äriliselt Postimehel oli ainuke võimalus teha ise kujukondades süsteem, kuna see riiklik kujuga neid toiminud ta ei saanu. Ta viis ajalehe kätte mingiks lõunaks. Postimees tahtis, et hommikuse kohvijoomise ajaks oleks olemas ja ehitas nullist üles oma kohukojukandesüsteemi, mis pärast rist liitus siis Express Posti omaga. Ja praegusel hetkel vist alternatiivsena kojukandesüsteemina mingi mingis ulatuses isegi toimiv
Sest ma mäletan seda küll sellepärast et see Postimees, ta oli hommikul vara, oli postkastis, oli mingeid täiesti nagu läänes, see oli nagu päris. See oli väga kõva sõna. Et aga missa peale listide lugemises emailiga veel tegid.
Üks üks asi, mis selle emaili teemal on ere õigupoolest kaks asja, mis on eredalt meeles, on see, et üheksakümne esimesel aastal oli seal Moskvas sihuke putšilaadne asi, kust vahetati valitsusi ja, ja tankitankid olid siin igal pool nende teletornide ees ja mis iganes ja siis oli jube infoauk oli, et mis toimub, kus toimub, et siis oli, sai sobivate Moskva, venelaste arvutihäkkerite ka kokku lepitud, et teeme mingi otseliini, et paneme mingisugused info listid käima ja, ja see oli jube põnev, et sa said selle toimuva info ja ajakirjanikele kasuliku info emaili teel kätte natuke ennem kui ta. Kuulge kurat teab, kust tekkis selline kontaktid, olid olemas. Need kontaktid olid olemas kuskilt mingist konverentsil käimisest või midagi sihukest. Ta sai, juuniks ei kasuta, et konverentsil käidud, Moskvad, aga Vladimiri siis mingis siukses linnas. Kusjuures, kus oli kohal esinejana peeti ülikoolist. Mäletan Keit postiku ei usu nii. Ühesõnaga, kaks pööki, juuniksi loojat või guru ja selles mõttes ei tasu naerda, et venelased suutsid need sinu enda juurde meelitada, vedada, tõenäoliselt neil oligi huvitav ja see konverents oli megakihvt.
No muidugi seda ma seda ma imestan, et see võis olla väga kõva sõna.
Ja oli ja noh, tagantjärgi,
See tekitas jälle jälle tunda, et see arvuti teema on nagu hea teema, et sa hoiad nagu näppu pulsil, et oled suhteliselt lähedal sellega, mis maailmas toimub. Ägedat.
Ja see äge ägeda toimumise juurest. Sa ikkagi ühel hetkel liik.
Müüsid ära sealt edasist oota veel natuke natuke selles mõttes, seal ülikooli juures tekkis gaase, esimene moment, kus emelist nähti, et see on päris kihvt asi lisaks email on olemas ka mingi sihuke värk nagu püsiühendused ja asjad ja ja üheksakümne teisel aastal oli see moment, kus Jaak Lippmaa pani Tallinnas püsti taldriku rootsi teleyksiga ja tartus teletorni siis püsiühenduse, mis oli kuuskümmend neli kilobitti ja selleks, et see sealt tähetornist toomel kuhugile mujale Calebi leviks, siis Postimehe eestvedamisel sai üle katuste veetud.
Tähetornist toomel kõigepealt keemiahoonesse, sealt sealt ülikooli peahoonesse ja sealt Postimehe majja, siis siis tort, esimene pühi püsiühendus, mis oli jälle vene sõjaväelaste käest viiesaja rubla eest ostetud. Ülejäänud koobi rullist ehitatud.
Ja ja noh, siis tekkis reaalselt see nii-öelda internetti kommuun sinna ümber. Sellepärast et need, kes sinna vahele jäid, said ju ka endale interlinnu absoluutselt. Ja, ja see oli onlain, see oli täiesti Online, et aga Lott prints pidi kõva protsent, latents oli kuskil kuussada millisekundit ja nii hull ei olnudki ja, ja noh, ega see kuuskümmend neli kilobitti täna tänapäeva mõistes ei ole nagu mingisugune superkiirus, aga noh, tollel ajal, kui internet oli veel tühi igasugustest kassipiltidest, siis oli see täitsa okei, kiirus.
Sai sai asju ajada küll. Kes see kogukond seal siis oli?
Kogukond oli, oli Eeennetti inimesed, Enok Sein on Villems, Richard Villems, Tiit Mokuma, Marek Tiits, pall teeuuringute instituudist. Ja tõenäoliselt neid inimesi on veel kelle, keda praegu mälu mälu meelde ei tule kindlasti sellele reageerinud juba olemas, siis ei veel veel ei olnud neid, palun, aga inimesed, seda videol see tuumik, inimesed, millest teen, et tekkis, olid needsamad kes, nagu selle kogukonna moodustasid. Kes nägid vaeva selle nimel, et see interneti püsiühendus oleks olemas ja ja noh, siis kogunes sinna mudeliga sisse helistajaid ja vaikselt hakkas kasvama.
Mõtlesin loogiliselt järeldan, siis käis Tartu ja Tallinna sidesatelliidi.
Jah alguses käis üle satelliiti, mingi aastapäevad, iili tuli ka maapealne püsiühendus. Meil eestvedajaks oli siis küberja, noh, see on see koht, kus ilmselt Eesti nii-öelda selles internetimaailmas
Tekkis kaks nii öelda rististe suuskadega kommuuni üks oli sele Küberi kommuun ja teine oli siis nagu KBFI kommuunides, mitte siis Aktsiaselt küberneetika instituut, Küberneetika Instituut ja aktsiaselts tekkis ilmselt sellele kunagi hiljem veel. Ants Võrk.
Miks need suusad riskid risti läksid? Vaat seda mina ei tea selles mõttes, et tortu inimesena siukseid tollina siukseid probleemidest aru lihtsalt ei saanud. Tõenäoline tõenäoliselt tõenäoliselt oli kui ressursiga kitsas, mida ta Eesti vabariigi alguses oli siis võimalik, et seal olid lihtsalt mingisugust rahade jagamise või teaduste või finantseerimise mured, et kes oli, kes said oskuslikumalt finantseerimisele ligi või või noh, see oli olelusvõistlus, tegelikult tundus mõlemal pool, kuhu tulevad inimesed. Aga no ma arvan, et see olelusvõitlus, mis oli Eesti vabariigi alguses ja jättis kõigile jäise suli
Aga mis edasi sai? Edasi sai see postime, periood sai läbi, siis mingisuguse aastakese töötasin Tartu Ülikooli raamatukogus kus kuu sai rootsi kunni abiga muretsetud esimene serveri asija otsast hakatud kirjutama raamatu infosüsteemi.
Ma olen niisugune pisikene projekt, sinu ajalisi sattusid vanasti, Tartu ülikoolil olid legendaarsed serveri nimed, kalanimedega, Serbia ei tööta. Kas sinu Eestis?
Enam ei mäleta, lihtsalt enam ei mäleta kilu, LIVE juust. Kilu, n-i p oli mingisugune. Ma arvan, et Spark Station kaks.
Aga kus, kus, kusjuures selle põhiline funktsioon oli ikkagi see, et sinna tekkis esimene elektroolil ülikooli raamatukataloog, mida kohapealsed inimesed ise programmeerisid.
Ja selle jaoks on isegi terminali, seda sai kuidagi ülikooli raamatukogudest kasutada. See oli tolle aja kohta, oli väga niukene, funktsionaalne, tore asi.
Ta ei olnud küll see oli jube töö ja, ja, ja, ja noh, eks andmesisestamine kõik nullist tehes on ikka väga raske
Vaja, sest seal on meeletu kogu selle taga. Ja mitte ainult raamatud, vaid vaid muusika ja mingid käsikirjad ja.
Aga kahjuks kahjuks või õnneks tagantjärgi ei oska öelda, see raamatukoguperiood jäi suhteliselt lühikeseks kuna, kuna Jaak Lippmaa kutsus mind Tallinnasse ja see tundus veel põnevam ju kutsustasid, mida tegema. Tegelikult ta kutsus mind sellises riigiasutusse nagu valitsusside. Et umbes sellise mõttega, et kuule, Taavi, sina oled nüüd selle internetiga side ka natuke kokku puutunud, oskad ühest arvutist teise sümboleid saata, et kuule, et et siin mingisugune KGB sidekeskus tuleb Eesti riigil üle üle võtta ja tule aita. Ja noh, siis tulingi.
See kõlab hästi küll. Siis võtsite valitsusside selle üle.
Põhimõtteliselt küll ja noh, ütleme niimoodi, et,
KGB-st võib rääkida mida iganes, aga selle tehnoloogilise poole pealt nägi asi välja ikka suhteliselt õnnetu, et hädala tänava majas olid suured saalid täis mittetöötavaid, telefonijaamu ja selle ütleme ma võin nüüd natukene valetada, aga suurusjärgus kakssada töötavat telefoni oli ja mis oli siis ettenähtud nende riigi riigi kui nähtuse ja jaoks ja nende igasuguste organite jaoks, aga noh, ütleme kakssada töötavatele uni on ikkagi suhteliselt madin number kogu valitsuse peale kogu valitsuse omavalitsusaparaadiga. Ja, ja järgmine projekt seal Siimetsi telefoni jaamatega reaalselt. Mis siis on?
Toompeal Kadriorus, välisministeeriumis, mis, kes nad seal kõik pikal tänaval olid? Triaalse telesisetelefonijaamade võrgu käima panemine kokku mingi paar tuhat numbrit mis tuli õnneks, tuli täitsa edukalt välja ja, ja oli reaalselt ka mingisugune kasu olemas, see oli oroloogioon veel, eks ole, ei ole, see, see on digijaam, simestop, komandigi jaam võrgustatav ja kõik ja puha. See oli päris, see oli päris sidevõrk.
Äge, aga kust see visioon tuli, et siukest asja ehitada või see pidi kallis ka olema. Et oleks odavam olnud, võta lihtsalt analoogjaamad üritada kuidagi hakkama saada, aga tehti investeerinud.
Tehti investeering, kus investeeringu lükke tuli kasse tuli jaagu isiklikust initsiatiivist. Mida ta siis tehnoloogilise poole pealt konsulteeris, Paavo Picofiga, kes oli Tallinna telefonivõrgus seal tõenäoliselt sinna taha, tulid tänu jaagu ja Endel Lippmaa tutvustele ka riigi riigi funktsioonid. Et ainult seda on tõepoolest vaja ja, ja, ja tisse, visioon osteti ära ja finantseeriti ära. Bioriaalse oli reaalselt reaalselt töötavatel sisetelefonivõrk, aga, aga noh, eks ta mingi aja pärast jäi ajale jalgu.
Roll on kindlasti füüsiline kaabeldus oli Eesti Telefon.
Füüsi füüsi füüsiline kaabeldus oli, osa oli Eesti telefoni käest ja osa oli seda nii-öelda vana KGB kaablivõrku, mida oli tegelikult Tallinnas mõnisada kilomeetrit täitsa.
Aasias, nojah, keskjaame võis, võis katki olla, eks ole, aga kaabelkaardil kaablil.
Olid olid olemas ja olid siuksed korraliku tinakestaga kaablid, mis oli noh, umbes nagu tuumasõja üleelamiseks ettenähtud ja meenutasid rohkem tanki kui kaablit.
Tulidki tuumasõja üleelamiseks ette nähtud ka praegu sihukese projektina, mille käigus kohtumine huvitavate inimestega
Absoluutselt et telefonijaama installeerimine ja käima panemine Kadrioru lossis, kus Lennart Meri vaatab sult üle õla ja õpetab, kuidas telefoni ühe pistikuid ühendada, noh see võib tagantjärgi tarkusena muigama panna, aga noh, see oli niimoodi seal Lennarti moodi absoluutselt, et ta oskas kõike kõikides asjades nõu anda. Et vaadake, poisid, et te teete nii või naa.
Ja kaua sa neid kauneid vedasid seal valitsusse?
Valitsussides ma vedasin kaableid kukuks kolm aastat ja, ja ütleme peale seda, seda esimest edukogemust telefonijaamadega noh, loomulikult me tahtsime seal järgmisi siukseid edukogemusi veel, et see interneti-nimeline asi kogu aeg ronis uksest ja aknast sisse. Kirjutasime järgmiseid pabereid, et nüüd oleks mõttekas selle jaoks kuidagi investeerida. Siia-sinna ma ei võtnud väga vedu. Kuigi, kuigi see hetk juba ükskõik kelle käest, kes natukenegi internetiga tegeles nuutiat kuule, mina tahan ka inimesi tuli uksest ja aknast sisse, et kuulge, tehke midagi, noh, mul läks internetti vaja. Aga, aga.
Paberi- kirjutamin, paberite kirjutamine ei olnud edukas ja, ja siis.
Hakkasin kõikide oma tuttavate juurest rääkige läbi sõitma, et kuulge, niimoodi asi ei toimi, et interneti kõik tahavad, võiks ju teha, aga välja ei tule, et kuulge võtaks pundi kokku, hakkaks nagu normaalselt ise tegema.
Harva, et ma käisin, kõik inimesed, kes vähegi internetiga tegelesid läbi, et kuule, et võtaks pundi kokku, teeks mingisugune ettevõte, mis teeks, teeks midagi ja no ütleme, reaalsusena tartlased vedu ei võtnud, noh, Tartus on nii mugav olla ülikooli juures, et Pirogovi ja kõik kõik Pirogovi ja kõik asjad. Kuigi puht praktilised manectiits aitas päris palju ideed formuleerida ja nii edasi. Ja reaalsusena võttis kõige rohkem vedu KBFIs tundmis Bauman. Ja noh, see siis Andresega koos me tegimegi siukse internetiettevõte käisin poes, ostsin riiulifirma, mille nimi oli Nesper. Mis on, mis on praegu siis praeguseks siis Elisa seesamasugune riiulifirmad peavad vastu ostetud riiulifirma registreerimiskoodi, Elise registreerimiskood on samad.
Jätkame juriidilist järjepidevust, täiesti juriidiline järjepidevus olemas.
Mis aastal see oli? See oli aastal üheksakümmend viis, üheksakümmend neli, üheksakümmend viis algus.
See oli ikkagi üsna niukene, karm aeg, sul ei olnud üldse mitte midagi saada ja kõik asjad tuli nagu nullist ehitada.
No oli küll, aga, aga noh, teisalt inimesed olid ka siis leidlikud, seal tato Telekom, Neeme takis eestvedamisel ehita siis modemeid verest neli, kaks kohe elektrilise ühenduse peal. Kui ei olnud, siis tead ise.
Kes kohandas mingeid asju? Jaa, jaa. Midagi oli võimalik igal juhul teha.
Just nimelt, et ma sind kuulan, siis oligi nagu võimalik, nagu ise teha just peamiselt midagi saada ei ole, sa pead ise tegema.
No ütle, ütleme mingisugused,
Lisaboonused olid ka, et kui tänapäeval inimesed arvavad, et mis ta siis tuleb, et mingi kümme eurot kuus peaks tulema nii ja me intervjuu, kui maailmas üldse olemas on, ehk mis ta siis kolm kohvitassi kuus siis too aeg oli internet, nii nagu kui vastu öelda seksikas või mis iganes, või uus või edev, et siis oldi veel interneti eest nõus maksma. Halo sellised asjad, et ettevõte ostab ise seadmat välja, maksab kinni paigalduse ja siis hakkab kuutasuga veel maksma. Niisugune see oli sihuke helge aeg ja ütleme tänu tänu sellele oli võimalik need esimesed seadme investeeringut teha. Sest et samasuguse mudeliga nagu täna on, et kõik on kõik on kümme või viis kohvitassi kuus. Sellise mudeli ka interneti ei oleks Eestisse jõudnud mitte kunagi.
Ei saa hakkama, siis kui ma õieti mäletan, siis mina hakkasin toimetama teie edasimüüjana avastan mingi üheksakümmend kuus võib olla. Mis tõenäoliselt üsna varakult, teil oli ikkagi üsna niuke laivõrk, mingid juba nagu teenuseid ja mingid all.
Olla põhi põhipõhiline, mis siis oli õli sissehelistamine, ma arvan küll. Et ei usu, et seal. Ma ei mäleta, mis, mis süsteemiga SAIS füüsile, modemi puul telefoninumbreid ja, ja, ja sellesse tegelikult pihta hakkas. Tõeline, tõenäoliselt telefoninumbritega aitas. Kuna need olid Kadevko, aitas Jaak Pippa isa Endel Lippmaa, et, et noh, saaks kuidagi seal mingi sobiva järjekorra mingisuguse sobivasse punkti. On lõpuks need said.
Ja sealt sealt hakkas internetiäri vaikselt kasvama, kuna kuna inimesed tahtsid huvi, huvi, kerg, kes siis üheksakümne seitsmendal aastal või üheksa, ma arvan, et üheksakümne seitsmenda aasta alguses oli see koht, kus kus nii-öelda üldkasutatav interneti välisühendus oli igasuguste puukide poolt, nagu oli meie puukfirma seal köiest nii täis aetud, et teen, et tegi otsuse, et nii nüüd ärikasutajat, muretsege endale oma välisühendus.
Seegi hilja.
Selles mõttes, et üsna kaua saite toimetada nagu akadeemilise kraadiga.
Pea topelt täpselt samamoodi. Noh, ta ei olnud veel päris akadeemilisest maailmast välja pääsenud, see internet ja eks me, eks seal kõik toimetasid akadeemilise traadi peal. Jaa, jaa ja senimaani kolmi Eeennet jalga maha ei pannud, senise osas oli see täiesti täiesti loomulik nähtus. Uuesti siis sõitis Soome kuhugi Soome kuhugile. Mul oli mingist ajast jäänud kontakt Helsingi telefonivõrguga ja Helsingi telefonivõrgu inimeste ka koos sai siis nii-öelda kanal tellitud ühendus hangitud seadmed hangitud ja, ja meil oli juba siis sedavõrd palju käivet, et me enam-vähem suutsime isegi selle väliskanali kuutasu kinni maksta. Mis oli siuke soliidsed pea sada tuhat krooni kuus. Oioi oi Juhk summa tulla, see oli väga jõhker summa ja, ja noh, selles mõttes noh, kogu see internetitehnoloogia või ruuteri, see kogu see värk maksis ikka suhteliselt hingehinda, et niuke Kuue Wordiga Cisco ruuter marki enam ei mäleta, see maksis mingi mingi kakssada viiskümmend tuhat ehk ehk siukse noh.
Ma ei tead S-klassi Mersu hinna õudu, aga sellest ma järeldan, et mingi füüsiline kaabel oli olemas Eesti ja Eesti füüsi füüsi. Füüsiline kaabel oli olemas. Peale seda, kui see Telekom sai kontsessioonilepingu Eesti riigilt suhteliselt kiiresti, vedas Eesti Telekom ka kaks valguskaablit. Eesti-Soome vahele. Ma ei mäleta, mis aastal see oli, aga ma arvan, et see oli ka kuskil üheksakümmend seitse, üheksakümmend q. Päris Voro. Selles mõttes, et tellijal oli see kogemus olemas ja investeerimisvajadus kogu kogu selle analoogvõrgu kõige väljavahetamiseks oli väga selgelt olemas. Lihtsalt investeeriti ja Telial tegelikult investeerimisjõudu vali. Te tegite mingisugust hostingut, kao mingeid servereid ja sai servereid pidada, et ei peaks seda traadi raha maksma. Mis, mis oli, ma arvan, et täiesti normaalne teema.
Kuidas see käis selles mõttes, et kui ma praegu ostan mingisuguse virtuka endale, siis tol ajal ma sain lihtsalt minge kaundia parooli kuskile.
Siis said juuniks Jacoundi endale ja oligi kõik.
Ja mäletan, kuidas sai kuidas veebiprogrammeerimine käis niimoodi, et mul oli Windowsi masin siis ma tegin Berli skripti naalsusele FTP-ga ülesse siis ei töötanud siis ma kellelegi teie süstlad, Minni käest sain väljundi, mis logidesse kirjutati, see saadeti mulle kuidagi. Siis ma vaatasin, ahah, näe, Perli skript ei tööta.
Parandasin ära, saatsin uue versiooni. Vaat tunnistan ausalt seda, seda, seda poolt arendusprotsessist ma ei näinud, kuna, kuna ta olles pumba juures sa ise vaadata koev kas töötab või ei tööta.
Ja ilmselt ilmselt oled selle võrra ka mõnevõrra ajurakkusid säilitama. Et see läks üsna ruttu, enne, kui te teie admin mind välja viskas, selle sellest protsessist tuli mind või seal peal käima ajada. See selleks, et see on, see oli üks legendaarne ettevõtmine, legendaarne aega, aga mis sa praegu teed?
Praegu olen sellises huvitavas ettevõttes nagu rebel rõõm ja meie tegeleme transpordiettevõtetele wifi teenuste pakkumisega ja nende optimeerimiseks. Selles mõttes kliendibaasiks on mööda Euroopat, Ameerikat Inglismaalt sõitvad bussid kus siis on bussis wifi, mida saavad bussireisijad kasutada, või jõelaevad kus siis kruiisilaevad, kus pensionärid ostavad mingi noh, kolme nelja tuhande dollarise nädalase reisipaketi ja kui nad Pariisis Eiffeli torni pildistavad, siis nad peavad õhtul saat saama saata oma selle pildi lastelastele, et muidu on see reisikogemus natukene nõrk, puudulikuks. Ja siis teie teete nii, et Ifi turni Pilsood, Viiust.
Et selles selles mõttes, see võib tunduda imelikuna eestlasele, et meil on kogu aeg internet igal pool vabalt kättesaadav, aga näidetena ei, Prantsusmaal, Ameerikas, Inglismaal ei ole, see ei neligee viiske nii levinud, et see kvaliteet oleks kõigile piisav ja seal on see, ütleme teenuse optimeerimise vajadus täitsa olemas.
Mis tähendab, et need kõik need inimesed, kes Ameerika kiirteel kreondi bussi järel sõidavad, et saaks vihvikus.
Siis nad tegelikult kosutavat teed jokk Krihhound on meie klient. Ma ei tea ka, kas kõikides bussiliinides või, või oli, oli seal rohkem selle lääneranniku pool ma lisada lihtsalt ei tea. See laagri koondan, tuttav nimi küll, et on kliendi läbi jooksnud ka tehnomeeste poole pealt.
Ega sellest sellest ei tahagi rääkida, ma vaatan, et et see ring on jällegi kuidagi nagu tavapäraselt tundub nagu täis saanud. Et kui sa alguses seal edasi ütlesid, et inimestel nagu silm särab, siis siis praegu kaaslased, sa tõid nagu esimese näite selle pensionäri oma Eiffeli torniga, eks. Et lõpuks on ikka inimeste rõõmsaks tegemine absoluutselt läbiv joon, eks.
Ja, ja noh kui sa teda rõõmsaks ei tee ja, ja sinu teenust nagu viha koostadega ta raha sulle ka ei maksa, kui teed rõõmsaks, siis saad tõenäoliselt mingisugune raha enda, panga palga kontoleb sobivatel päevadel ja on ja on rõõm endal ka midagi paremaks teha, uut ja, ja noh, nii ta on koodi veel kirjutatud. Võrku konfilise võrku ei konfise maa, ma arvan, et ma korralikult enam ei oska seda teha, et see võrk on tänapäeval ikkagi niivõrd Äraviltaliseeritud juba. Ja et enam ei oskaks seda konfigureerida.
Mina küll märkasin Facebookis, kuidas keegi küsis ja siis sa mitte ei vastanud, et kuidas saab teha, vaid vastasid if konfigi käsureaga, mis võtmest töötas.
Et ei, noh, selles mõttes seal oma arvuti konfimine mitte ei ole selle võrgu konf võrgu konfrimise all, ma mõtlesin ikkagi seda, et päris võrku, et kus on sul mingisugused jämedad ruuterit, kus vilkuvad sinised LEDid? Jaa, jaa.
Pool Eesti interneti trehvikust, noh see oleks nagu võrgu konfimine, silki.
Lisa annab, tõmbab päris hästi meie jutule joone alla, ma arvan, et,
Selline mastaap on, on päris päris värskendav. Ega ma siin selle koha peal ei oskagi muud soovida, kui et need leed ikka vilguks.
Inimesed naerataks mõne, mõlemad teevad vähemalt miinimum minul tuju heaks, kui LEDid vilguvad nii, nagu ma tahan ja inimeste inimesed on rahul, siis on jumala super ju väga hästi, aitäh sulle.


\chapter{Sten Tamkivi}
\index[ppl]{Tamkivi, Sten}

\question{Kuidas sa arvutite juurde jõudsid?}
                 
Kõneleja 3:
Tere, see siin on memm, kopi. Täna on meil külas mees, keda ehk kõige paremini iseloomustab tema võime teha nii. Tehnoloogia abil hakkab inimeste elu paremaks minema. Selles on osa ettevõtlikkust, inimeste ja organisatsioonide juhtimist. Ja lisaks kõigele muule on tegu inimesega, keda ma tean juba üle 30 aasta. Külas on Sten Tamkivi head kuulamist.
                 
Kõneleja 1:
Tervist minu nimi on Sten Tamkivi.
                 
Kõneleja 2:
Oleme kuulnud siia suurepärasesse kontorisse hall sügisesel päeval rääkimaks sellest, kuidas asjad alguses said ehk siis olulistest asjadest. Kuidas teha, et autopoisiks olla, mina olendatus Tartus sündinud ja kasvanud kuni tegelikult selle saate mõttes kogu selle aja, kuhu sa oled natuke nagu noorema põlvkonna inimene kui ülejäänud rahvas, kellele ma rääkinud olen. Sündinud 78 ja loojub jah, sellepärast kuvas saateid kuule toe need siis minu jaoks enamik neid inimesi.
Et kui üheksakümnendatel, kui mina sattusin või kaheksakümnete lõpul sattusin niimoodi arvutite interneti juurde, siis need olid nagu sihukesed legendaarsed juba establisnimed, et kellega ma võib-olla veel mõnede elu jooksul hiljem tuttavaks saanud ja avastad, et oh, et ma ei tea, Madis kaalu nagu päriselt ka olemas ja ja ei olegi nii palju vanem kui maailmas just nimelt takkajärgi need vahed lähevad nagu kokku paar-kolm aastat, vahet ei ole enam nii suur, aga tol ajal tundusid tõesti kuludena. Kuidas räägi oma sellest kujunemisloost, et siin mõned on rääkinud, et nad mingit hirmsat olümpiaadi hundi toonud ja teised on rääkinud, et neil huvitasin kuu raamat, kuidas sina selle värgi juurde jõudsid? Mina olin seal paati nagu erinevat suunda, et üks asi on see, et ma olen pärit teadlast suguvõsast või mu isa ja vanaisa füüsikut mõlemad ja ma kasvasin üles lapsepõlves Tartus FIE rajoon, mis on füüsika instituudi jõgi mis tähendab seda, et sul kõik lastekaaslased, kõik naabripoisid, kellega õues mängida, et kõik on kuidagi seotud tõenäoliselt füüsikainstituudiga. Ja, ja noh, ma ei tea, kui seal 80 lõpus 90 alguses Füüsika Instituudi elektroonik paneb majadesse Piaat kaabeltelevisiooni ja füüsika instituudist, saad esimeste arvutite ligi ja kõik seda, kui asi asi oli sedapidi seotud. Ja teine ma õppisin mina ema gümnaasiumis või tol hetkel, kui ma läksin, 85, oli Tartu teine keskkool mis on mõnes mõttes nagu humanitaarsihukesed Tartu, nagu tolle aja siukestest esikoolidest, et Navaid esimene keskkool ja minu ema teine keskkool Treffner oli palju selgemalt reaalainete kallakuga, aga siiski mina ilmas ka oli selline reaalainete suund täiesti olemas ja.
Et mina olen, see oli ka suund olemas. Sedapidi ma käisin olümpiaadil ka varasemaid pilte, mis mul on kuskil on Maidan mingis vanemate või, või vanavanemate albumis olemas on selline mustvalge foto, kuidas keegi tõi sinna Miina härma või teise keskkooli algklassilaste näha kooli arvuti Juku. Kuidas selline? Ka mustvalge pildi pealt on näha, et silmad juba läigivad, kui lähedalt näed.
Fiisoris esimene arvutikogemus. Ma arvan küll selline, kus nagu sa oled nagu päris oma aega siia sina ja arvuti, ehk siis et nagu isa pani, pani kuhugi mingil õhtul kellegi kabinetis kuskil asi üldse välja kui arvutiaega ja et seal istuda neid kohti oli veel, et, et selles suhtes mul näiteks pinginaabri isa töötas Tartu Ülikooli raamatukogu sotsiaalMe, käisime arvuti taga ja seal üheksakümnete alguses ta tema hakkas isegi tegelikult pidama vuti kaupu müüma või kooperatiiv poodi pidama, et siis siis oli Amisjoni kodus kaevuti ja siis oli üks koht, ma mäletan, oli esimene koht, kus ma Amingat nägin, oli aasta või kaks noorem koolivend Lemmik Kaplinski, kellel oli isa ilmselt siis kirjanikule kuskilt maailma pealt seda oomika abiga kätte saanud ja siis olid kuulsad tähetornis oli veel mingisugused eksis Noorte tehnikute maja, kus olid jamad. Ehk siis ma ütleksin, kui niimoodi vaidlema hakata, siis just võib-olla iseloomustabki see, et kellelgi ei olnud nagu püsivat kohta, vaid otsiti seda aega, kus sa saad ja, ja mida tehti Tartu-Tal oli ikkagi toimima. Siis noh, see on see koht, kus ta ikkagi nagu on Eesti nagu Cambridge või, või, või projekti, et kui sul nagu väike suhteliselt väikeses asulas on nii domineeriv ülikool, siis see tähendab seda, et noh, et ja kogu see interneti algus ja kõik see, et et sa saad akadeemilistes võrkudes, hakkas peale. Et mäletan Tatu püsiühendused, internetiühendused, mis tekkisid, olid ju ka palju satelliidiga lootsi tähetonnist satelliidiühendus ehk siis enne kui tekkis Tallinn-Tartu liin, tekkis nagu Tatu sõimlikku otsi ülikooli, mis iganes Stocktonis isegi selle üle kosmose just. Aga sinna pole veel sinuga veel jõuame. Kui sa ütled, et sa oled nagu sinu arvutiaega, et.
Kui palju ja kuidas sa sõitsid õpetust sealt nagu mis teed siis lihtsalt vajutasin nuppe sees. Ma arvan, et see muster ikkagi enamikel inimestel täpselt samas alguses tahab mängida ja siis sa tahad aru saada, kuidas neid tehakse, siis hakkab natuke programmeerime. Mul oli nii, et ma üheksandas klassis läksin, pärast kooli hakkasin programmeerijana tööl käima üheksandas klassis ja ma olin 15, ma arvan. Siis ma tolleks hetkeks olin, ma arvan, kuskil kaks-kolm aastat niimoodi omal käel pagenud. Ja see oli jällegi, see oli mu isa oli Tartu teaduspargi asutaja ja siis siis seal teaduspargis tegutseb nagu mitmete firmasid ja siis ta küsis, et kas ilmselt ma eeldan, et ta küsis, et kas keegi seal poisile mingit kasulikku tegevust leiaks. Ja siis oli sihuke hulljulge mees nagu valettindabamov, et üheksakümnete alguses teatavasti toimus Eestis ohjeldamatu metalli jäi, siis oli Tartus sihuke metallikonglomeraat nagu põimeks ja peksid, oli, tütaksime imekslaata kus tehti igasuguseid asju, põhiliselt pandi mingisuguseid pissi kloone kokku. Siis olid seal mõned inimesed, kes pagesid, mingit projektijuhtimistark, mõned inimesed, kes pagesid, mingit raamatupidamistarkvara, näiteks Tarmo Tali ja, ja siis ta palk, Tinn või vàlja palkas mind nii-öelda Promee jaoks, aga tegelikult oli see sellise koolipoisi nagu päästa midagi, ma just käisin ka. Ma ei usu, et sealt midagi üldse. Ma tahaks juhtida. Aga, aga külmal seal haldasin kohaliku ajuti kui aitasime arvutit kokku panna või, või hakkasin BBC pidama ja seal oli selliseid tee, mida tahad või maalimas kõlab nagu mõnusa maailm. Aga selleks ajaks on ikkagi olema piisavalt testikulaarset portituudi, väitas programmeerija raamatute järgi õppisite internetti VPS-i olnud, eks ole, ja, ja seal oli noh, ega neid raamatuid ei olnud ka ju seal alguses kätte saada ja selles suhtes ma kuskil nagu mingites see vist see, mis oli seal noorte tehnikute majas, see oli vist ajuti rinny nime all, aga kuna ma käisin seal nii hooti või see ei olnud nagu niisugune suhteliselt korraliku nagu progrimise alg, alghariduse ma mäletan, et ma olen kirjutanud ka paberi peal koodi, et kui sul parajasti see periood, kus sul ei ole ligipääsu ühelegi arvutile, aga vaata, mis te A ja arvutustehnika ja andmetöötlus, andmetöötlus või näiteks mingi ajakirja, mingid mingid sellised paberimaterjali, siis sealt nagu üritad midagi nagu tuletada või, või teha, teha midagi mõttes valmis, enne kui arvuti taha saadeta. See oli vot see on nüüd läbiv joon kõikidelt, et kui me praegu räägime sellest, kuidas õpetada lapsi programmeerima siis nendest juttudest tuleb järjest välja, nagu keegi ei oska, kuidas nad õppisid programmi, kuidagi imbus läbi naha või läbi õhust või kuidagi tuli. Kuidas see nii on.
Üks asi, mida ma olen näiteks mõeldud, on see, et et eriti, mis puudutab neid nii-öelda kooli arvuteid, nõukogude aegseid aga ette ja, ja Yamahhalsid ja jukusid, et seal oli ikkagi arenduskeskkond, oli esimene asi, kuhu see sisse ennast Puttide alguses need suhteliselt raske oli seda arvutit kasutada niimoodi, et sa komistaks nagu arendusvahendite otsa. Et kui seda ära võtta taifuuni, siis pead kurja vaeva nägema, et saada üleskeskkond, millega sa saaksid siin midagi teha. Et, et see on kindel niisugune muutus. Ja ma mäletan, et tegelikult just, et see kooliarvutite ajastu oli nagu, nii palju põnev
Et kui mul oli onu, ostis endale kunagi mingisuguse läptopi, mis oli selline Tossiga ilmselt mingisugune, ma ei tea kompaq, et kui ma tal külas käia, nagu seda kasutasin nagu ikka tahad ikka arvutiaega ja kuna seal ei olnud ühtegi arendusvahendid, siis no mida sa teed seal kaua tossi, tossi, direktori uus nagu ongi nagu et see väga huvitav ei ole tekstiga traktoreid, niuke, äri, arvuti on ju, et siis ma hakkasin nagu just ükspäev mõtlema, et see, tegelikult see hetk, kus ma esimest korda sattusin, kasutame arvuteid, mis kus ei olnud, mis ei olnud nagu eelkõige arendamiseks mõeldud. Mäletan, sinuga seoses oli ka üks, kui me kunagi tuttavaks saime, oma võrus, sattusin sulle külla, kus said laenanud koolist suveks kooli auti klassist üha kati koju, nii. Ja sellise ekstreemne juhtum, kus, kus selleks, et üldse midagi teha, sa pidid kõigepealt eksis sisestama, ma ei mäleta, kes sisestas teisiku koodi selleks, et saaks punkti, kuhu saaks hakata kirjutama nagu ilm loetud koodi müüb. Hullus aga et noh, kui koolipoiss oma suveajal istub ja kuueteistkümnendsüsteemis koodi sisse toksib arvutisse, selleks et siin on midagi mõistlikku saaks teha, siis selgelt su suhe selle arvutiga on teistsugune kui lihtsalt meediatarbimine. Seda küll jah, see lõksu suhe ilmselt teistsugune, et see on hästi oluline asi, et tulla suhe arvutisse või teine. Ja siis kõik muu tulenes sellest, eks ole. Ja üks asi on veel, et, et sellel arvutite lihtsusel või ütleme piiratusel see, et kui sul on 25 rida korda 80 tähemäki, on su nagu visuaalne mängumaa või hiljem mingisugune AEGA või VGA graafika ehk siis see teeb selle kõik kättesaadavaks, tegelikult nagu ka laps suudab kogeda midagi, et see on nagu nii palju rohkem jäetakse nagu fantaasia jaoks, et kui keegi nagu tekstirežiimis mängu siis see ongi nagu nii-öelda selle arvuti tipptase. Kui täna keegi võtab mingisuguse koduse mängu PC, teeb seal midagi tekstirežiimis, siis noh ühesõnaga kõik, mis ei ole nagu tohutult videokaardi võimalusi kasutav kolmdee, Vendedus reaalajas on, tundub nagu naeruväärne, aga tol hetkel see kõik, mida sa ise suutsid oma kätega teha, ei olnud naerda. Vot täpselt. Ja selle juurde käis mingisugune raamatuhuvi mingisugune, sul pidi olema siis selles seltskonnas, kus liikusid seal liikumi ingliskeelset kirjandust.
Kus ikka, aga ma no jällegi Miina härma oli koolil ja selles suhtes nagu äge äge, et et enamik asju, mis seal toimusid, olidki nagu see nagu sihukeste nagu asjad, mis, mis nagu jäljendavad ma pigem nagu teiste õpilastega. Kooli bände oli kõvasti, ma ise üheski bändis ei olnud küll. Aga ma ei tea jällegi 90.-te lõpus piss näiteks kellega ma väga palju hängisime, kes usatega siiamaani nagu läbi käidud, siis see oli nagu mina ilma kooli bändist välja kasvanud asi, aga mis oli ka nagu otsapidi väga elektrooniline ja väga niisugune ma ei tea, avas ma ei tea, arvuti ja muusikaseose maailma minu jaoks näiteks on ju raamatut mõttesse, mida loeti. Jah, aga see ei olnud nagu ma ütleks, et ma ütlen emps üks küberpungi Šveits Fiction juurde jõudsin pigem juba 90 teises pooles ja siis, kui ma Ameerikasse sattusin, et enne seda võib-olla ma lugesin pigem kääbikute tult, Kiili kui kuiv Priveks, siis programmeerijad. Erinevus on selles ja see oli 93 kus müüakse ülikooli, siin oleks seal pimexis. Eks sellest korra veel, et miks seda softi selle prohveti, kas see oli nagu enda tarbeks sellest äri teha? Minu arust Eesti IT-tööstuse ajalugu on ju sellise lainena, et selline 80 90 alguses oli see, et kui sa alustad täiesti tühjalt lehelt kõigile arvuteid vaja, siis kõik teid, juppe, võidalveteid kokku, et oli püheks saata samal ajal seal mingisugune kõrval majanduse või kuskil seal Hoioodi. Vastan Macro link oli Tallinnas Asto data ja kõik need kõik tegid sama asja. Siis liiguti tasapisi tarka kihti, aga see pigem oli selline. Ma ei tea, riigil raha ei olnud pankadel raha. Kolijaga võib olla huvi olnud hästi palju süsteeme, ainult ise tekkisid need firmad, kes nagu ajendasid
Nagu teenusena ja kogu see, ma ei tea helmeste ja meediat, see laine on nagu selle kõige tugevamad näiteid võib-olla. Ja siis lõpuks tekkisid nagu esimesed nagu või meistri, miks on täna muutunud nagu toodete ehitamine? Et võimeks oli naljakas hübriid, sellest võib üheksakümnendates oli, nii et ühest küljest oli seal see arvuti, mis oli see, mida kõik tegid ja kus nagu tuli, põhiline käive. Aga teisest küljest hakati tegema nagu ikkagi tootena, need olid nagu mingid asjad, mida Nad lootsid ilmselt klapi peal nagu müüa ja et sa ostad. Ma ei tea, kes need Eesti Melita teevosoftid ja kõik need raamatud, mida just hakkab, mingisugune torgib, seal tekkis ja osad neist ju siiamaani. Et, et seal ja see projektijuhtimise tark, kuidas ma mäletan kaks kaklejat nimega uimas ja Jürgen, kes siis minu arust tegid sellest samal ajal Tartu Ülikoolis oma magistritööd vist, noh, et projektijuhtimine, kant kanti graafikud ja selle kohta eesti keeletark, et noh, ilmselt see oli nagu sihuke akadeemiline asi, mida nad lootsid müüjaga. Jällegi ma ei mäleta, et sellest nagu mingit suurt äri oleks nagu tekkinud ja see hübriidid ja pidamine suhteliselt jabur, et see oli nagu mäletan. Ma ei mäleta, kas oli minu arvutiga Tarmo Tali ajutiga või mõlemaga istusime kõrvuti, et siis oli see, et seda hommikul tööle või pärastlõunal ja võtad, nüüd siis hakkaks pagema ja selgub, et keegi soovitust mälu ära müünud näiteks. Et mälu, lihtne võrk, korter on sellistes üles täpselt ja siis sisse nagu tegelikult nagu alustatakse sellega, et okei, et enne oli kaheksa megabaiti mälu, aga äkki laost nelja megabaidi see kivi kuskilt leiab. Ja siis noh, jällegi siukest tarkvara maailma ja. Sest et läbipõimumist oli mättad, istud seal ja nokitseb midagi teha ja siis tuleb siis Jaan, Tallinn, kellest ma olin kuulnud kosmonaut ja, ja noh, legendaarne mängu tiimi siis jaan tuleb monitori ostma, sest et niisuguse põhikooli või keskkoolipoisi käed värisevad, et ma nagu vist vara müügiga tegelenud, aga siis jube põnev, kuid võitsid ligidal. Tundus olevat seda porist koostatud võrrandi läbi käinud, tol ajal oli arvuti arvutimüük, oli nii ägedama originaaliga asi et sealt selle marginaali sisse mahtus igasuguseid naljakaid asju sai teha, mingi tudengi, pidada endal projekti softi kirjutamise ajakirju välja anda. Et selles mõttes oli, oli ilmselt siis jõuavad siis ka siuke vaata, mis juhtub ja no miks mitte, eks täpselt see noh ja ma ei tea, minule on õpetanud see, et näiteks ma arvan, et ma kindlasti täna võtan oluliselt parema meelega endale praktikante intern, töövarjusid, et see, kui palju mind see mõjutas, et see võimalus, mis nagu valja andis sellega, et sa saad lihtsalt.
Ma hakkasin mul isegi palka ju selgelt nagu selgelt nagu olukord, kus noh, eks poleks mingit vahetut, kui nad ei oleks mulle palka maksnud just tuppa sisse lasknud, siis täiesti puudu. Ja seal tegelikult sealt edasi, mis oli, huvitav, oli see seal ma sattusin esimest korda võrkude juurde, et enne seda oli ikkagi arvuti, oli nagu iseseisev eraldi olev asi. Siis kui see oli 93, et siis oligi täpselt umbes sel ajal, kui Tartusse ilmus internet minust, ma, ma pakun, üheksandate võis olla esimene eraettevõte vähemasti teadus pakkimistele. Esimene niisugune Elo ettevõttete koht, kus, nagu mitte ülikooliasjad sattusid võrku. Ja see nägi välja nii, et laseb nagu kuskilt tuleb.
See jadaühenduses nagu võrgud olid, onju, et kus oli otsast terminaator, muidu selges eaka vanemate arvuti. Kuskilt läbi seina tuleb see nagu, nagu ots oli ju, sa ei tea, mis masinad veel selles jadas on, kõik on nagu ühes võrgus maja peal laiali ja siis siis meil oli 93. aastal oli meil nagu püsiühendus internetti lünksi põhine nagu veeb, enne kui need Skype välja või mosaiik nagu välja ilmus. Ühest otsast ma pidasin piibi essi P9 ta nime all ja teistpidi oli meil olemas püsiühendus, kus oli võimalik Maida tuumi demo saada, Wulfelistani, mingid asjad juba FTP ka kätte ja siis panevad BBC üles, mis, kujuta ette, et paljude teiste piibitekstide jaoks oli see kõik nõuske modemiga sihuke loksutamine, mingi faili leviks, aga meil oli nagu niisugune, me olime nii-öelda pumba juures. Ja, ja teine asi, mis ma mäletan, sest sest nagu kuskilt see arvutifirma ja asjad, mida nagu hobina tehes oleks hoopis teistsugused, välja tõid. Meid oli sisevõrk, sisevõrgus oli novelliserver, millel oli 300 megabaidi kõvaketas, millest noh, tööasjadeks oli kulutatud umbes paarkümmend megavatti kood, mida seal kirjutati ja paned nagu hinnagi ja Vöödi fail onju. Ja siis siis ülejäänud oli mingisugused hollandi tüübid, laadisid selle alati öösiti mingit akva täis, sellepärast et kui sa teed ukse lahti kuskil Ida-Euroopas, kus ei ole ilmselt kaid, elektuaalavate viskus Euroopas ei olnud neid reegleid nagunii rangeid, ise mitte nagu ligilähedaselt, rääkimata siis veel üht Eesti Vabariigist ja siis siis tuuled selle läbi ja vaata, et mis on need asjad, mida piibeeessis nagu ülejäänud Eestiga jagada. Aga siis see tähendas seda, et sa pidid nii Interneti otsas kui BBS-i osas ikkagi päris nii ei ole, ovat avad FTP pordi, siis sina hakatakse kohe mingeid faile toppima, et see pidi ikka mingi võrk tekkima. Kuidas see tekkis? Mul on võrgustik selles mõttes.
Ma arvan, et seal oli Youznet. Ta oli ja just need mingeid uudisgrupid oli võib-olla esimene selline kogukondade teema kooma, nagu rahvusvaheliselt sattusin. Eesti Fidanetiga õpid ka, eks ma arvan, et seal oli see õuelapp ilmselt oli täitsa olemas, et kus nagu ka eesti filoneti gruppides arutati juba, kus nagu internetis käia ja kus keegi istub, et et kus, nagu ta akva ligi pääseb, kõik see, et see
Minu arust on minu jaoks igasugused sellised reaalajas, Chatrumid tulid nagu hiljem, et mingisuguseid vändum ja need jututoad, et see Aias siin mingitest kanalitest ma ka istusin, aga ma selle kohta isegi see on, ma ei mäleta, et oleks näiteks mingisugused tohutu nagu side või mingi sõpruskond tekkinud, et see oli nagu rohkem sihuke ajada. Ma ei usu, et oli see peamine võrgustiku ehitamise ja kust sul tuli sihuke mõte hakata öösiti tema valitsuse tööle.
Vot, see on hea küsimus. Ma mäletan, ma, see idee müüsin küll maha sellega, et siin jube kasulik võimekuse ta nagu tarkus või kus nähtaval. Aga, aga pagan teab, mis võib-olla see oli äkki see, kuna laos oli modem, et siis mida teha saab või no mingi tõesti mul kodus ei olnud arvutit 90. lõpuni, et siis ei olnud nagu asi, millega ma mainin piibesside kasutaja, nüüd tekkis võimalus siis ise püsti panna, et see oli kuidagi. Ju ta ikka sedapidi oli, et laos oli modem. Sellega sai helistada sisse teistesse piibeeessidesse, mis juba tallis olid üks, kusjuures enne kui sa külla tulid, siis ma hakkasin mõtlema, et ma mäletan Fidoleti nõudi numbrit peast jätkuvalt, mis oli kaks, neli, üheksa, ükskord üks, kaks punkt kaks, üks oli vist Jaan Kuulmann, kaks sellist Tatu tsoon oli kaks suhteliselt õe veel ja kõik need Tallinnas just ühe ühe all. Ja, ja et siis noh, et oligi, et kui me BBC püssi panin, siis selgus, et on võimalik olla teine pingestatus. Üsna mängu alguses siis pimesi TPS n utsitada.
Tuleb siis. Mis tema number oli, ei tea, aga ma mõtlen, veiko tundes võis olla 66 meetrit vett. Nii palju, et võib-olla see puhtlida. Võib-olla siis see politseile püsti ja, ja sul üks liine, üks vorme.
Hiljem olin uue selle kohta, kui ma olen küsinud, et mida inimesed seal oma selles Peebeeessis võitsid, siis reeglina inimesed ütlesid, et seal üleval nuku asjad, mis nagu endale nagu kasulikud nende huvitavat tundus. Inimesed tõmbasid kuskilt alles, tegid teistele kättesaadavaks, mis sorti seal oli?
Ma arvan, et seal oli, oli suhteliselt kaootiline sörk, et puhtaks selle failist oleks kuskil Audes ses suhtes seal oli ilmselt mingisugune noh, mingi mängude teema siis mingisugune jutt, lülitite teema oli kõigil on ju, et sul alati oli noh, jällegi ressursi kitsikuses on mingid asjad, mida, mis alati oli nagu huvitav või vajalik tehnoloogia, näiteks pakkimisalgoritmid, onju, et alguses on, sul on, ma ei tea. Chip ja ette siis tuli midagi uuemat sihukesi Jutilitis nagu asjad, pluss nendega nendega oli see rõõm, et meil olid väiksed, et see kiiresti liigutada ja siis kui midagi uut, noh, isegi kui see nagu osa sellest masinasse kuidagi sõid, siis olete juba sinna mingisugune pakkija. Mäletan mingi asi, mis tossiss, mälu, võimalus kasutada, mis nõu piirangutest mööda niuksed kohe käima panna, tööstuse Surfraktist loosungiks.
Ja jaa, jaa, Fidoneti gruppide mõttes oli ka ju ikkagi noh, tegelikult huum oli ju ikkagi kallis ja see modemi aeg, et siis, kui sa vaatad, et sul on mingisugune öine meili sünkimine on läinud jälle mingiks tunni ajas, eks, et siis siis jälle vaatad, et võib-olla kõiki neid gruppe pole vaja, mida ise ei loe. See pool oli ka, et see nagu reaalselt käib üle. Seda vaatad vahepeal, palju sa palju sellest aimu, kes küljes käivad, kes lugemuski lihtsalt mingid numbrid eristuvad? Jah, minimaalselt mäletan sellest või, või ma ei tea isegi, kui palju sellest teadsid, et see pigem ma mäletan seda tunnet, et oli nagu, kui sa ise juhtumisi seal olid, noh jällegi, kuna ma öösiti käisin kodus magamas on ilmselt inimesed, kes kuskil kodust välja helistasid, siis siis tegid seda hiljem aga et kui, nagu sel ajal, kui ma arvuti taga olin ja siis see modem nagu kõne vastu võtab, et siis nagu põnevusega vaatavad, mida see inimene teeb seal ja seda sai vaadata? Ma ei tea, kas see oli juba tol hetkel, oli äkki Windows või oli see OS kaks, millega ma katsetes, aga mingil hetkel ma sain nagu sihukese vist OS 200 jooksutada jõudossile tolmu esinenud või mingisuguseid lossiprogramme kuskil aknas, et mul oli multitaskis asi taustad, et isegi kui ma tööd tegin. Ja teine teine asi, mis meil oli, oli välja helistada, seal alguses oli. Minu arust istus Tartu Teaduspark kõige Tatu kõige vanema telefonijaamad, aga sellist legendi õigemini 50. Ma ei tea, kui palju sellest, tõsi on, aga mingisugune viiekümnendatel püsti pandud kanal ka. Ja siis me üritasime ikkagi seda nagu lahti häkkida ka nagu ühte ja teistpidi üks asi, mis minu arust veel kaevad, töötas, võib-olla niisugune linnalegend, et et meil, kui sa võtad kettaga telefoni ja teed nagu üks ühest kuni üheksani on nagu klõpsuda ööbin ja 10 klõpsu null. Ja siis me lugesime kuskilt, et kui, kui sa teda 11 klõpsu, siis sa saad kaugemale trankigetu. Ja siis me tegime 11 klõpsu, saime teise tooniga, Eli, helistasime mingisse Hongkongi, BBC ilusti tulnud kunagisele Pedalvet. Eks selliseid võib-olla mälestus, aga see lõbu kestis jube vähe, et see oli kuule nii tõesti nii muldvana jaam ja ilmselt palju kasutajaid taga, et siis see mingil hetkel vahetati esimeste seas digikeske vastu välja. Aga kuivaks oli siuke. No ma ei tea, jällegi 90 lõpus, kui ma mingisuguste äkki reagi esimest korda nägin, mingi 2600 ja et siis neid lugema hakkad ja siukest nagu USA 90.-te 80 lõppu 90.-te found Froikingu laine, et me saime nagu koiva seda korraks oli Eestis teine, teine asi, mis töötas kindlalt, oli see pooleaastane periood, kus Eesti Telefoni ootus Chipiga kaadiga telefoniautomaadid oli, kus see kinni teipida ühe klemmi ja tasuta helistada. Et see oli ka korralik? Tooge vist ei saagi huviliste toetaja. Ustasin teoseid, kus kuradi kuuesadesse sattusid? Ma arvan, et osa neist hakkasid jällegi, et kui sa pidanetis nagu saad esimese ringi peale Eesti gruppide lisaks tellida kui mõne mõne rahvusvaheliselt USA grupi või siis kuskil Youznetist siukese häkkerigrupid, et seal hakkas osa neid asju nagu skii tekstina, lihtsalt nagu paberkoopiaga paberkoopiat ma nägin siis kui Moussoleksin. Ja siis ma olin tõesti šokeeriv, ütleme, see oli nagu jälle niisugune, nagu ma ei tea. Ilmselt mingitel usklikel on see, et kui sa satud mingisuguse piibellik teksti originaali juurde, et siis, kui sa raamatupoes näed, et see asi on paberid, füüsisite värvilisena täiest või siis sa sattusid Ameerikasse, vahetusüliõpilased keskkooli ja, ja et mitte üliõpilased, vaid lihtsalt ma tegin avalduseoto klubivahetusprogrammi, millel on selline nagu noh, ankeet oli hästi selle keskne, et et mis sa teha või noh, lugu, et aru saada, et kes nagu kooliõpilane on ja foto või vahetus on selline nagu kahesuunaline, nagu üks klubi saadab kuhugi kellelegi kuskile välja ja siis samal ajal võtabki mujalt vahetusõpilasi vastu. Aga see kujuneb välja, saatsid juhtumisi ma ei tea, kas sellepärast, et muuseas ankeet nii nagu ajuti asjade kestmine, aga ma sattusin oma nagu 11. klassi, siis Eesti mõttes sattusin Silicon Valley keskele. Ehk siis sihukeses õrnas eas elasin aasta aega kuppetinos. Mis on see linn, kus on Apple'i päeva kohta ja käisin Monta vista faili, mille siis keskkoolis, mis oli jällegi nagu paljude juhuste kokkulangevuse USAs, oli, just oli seal mingi Ifromission, suprahai või hullus puhkenud ja hallkool oli eelmisel aastal enne minu sinna kohale jõudmist oli USA-s 35 kooli nagu internetipilootkoolideks ja see oli üks nendest. Et see oli põhimõtteliselt Apple'i Apple'i peakorterist mingisuguse paari kilomeetri kaugusel ja sulle, Meil oli, kui ma õieti mäletan, koolis oli 1400 õpilast ja kaheksa arvutit tekkida ja selgus, et arvutilaboris assistent olemise eest saab nagu ainepunkte või et sa võid nagu ühe tunni asemel iga päev istuda arvutilaboris oli, kus šveid nagu Mäkid ja sarni silikon, krahviksi, mingid asjad ja ja, ja jahud, kasutasime aadressil jahutad, siesta stan, fortatiidi ju oli, et sest sest see olgu söömaks muutunud sattusid tundmist paradiisi. Taimesid küll jah. Et see, see kindlasti on ka nagu tohutult mõjutanud seda. Mis, mis edasi. Aga kuidas toime sellega sa ütlesid, et jumala armastatud jalust nõrgaks, kui sa tuled nagu nõukogude liiduvabariigist sisuliselt satud üksuse kohta.
Hea küsimus, see oli mu esimene lend Eestist välja, oli nagu uss, onju, et üksi. Ja, ja ma arvan, et vanemad nagu raha laenanud lennukipiletit saaks lubada ja see kõik nagu ilmselt nende poolt ka nagu üsna hullumeelne. Aga, aga see oli, ütleb ütleme nii, et ega see ette see on nagu vanemate asi on muretseda, ega sa ise ise sellises vanuses nagu lihtsalt teed ja lähed ja oled ja, ja oled nagu käsl ja võtad kõike seda sisse, mis tuleb. Koolikogemuse mõttes oli see, et kui sa eestist vööri, kui sa oled nagu siukest matemaatika füüsika huviga ja nagu vähegi nagu olümpiaadil käinud nagu natuke mingit sihukest niisugust elu elanud, et siis ei ma võtsin, olles seal 11. klassis, ma otsin kõik reaalained olid mul 12. klassi panerist tasemeained ja ma olin kõik selle Eestis juba läbinud, eksis, koolisüsteemid, on nii palju erinevad. See, mis oli teistmoodi seitse esimest korda elus, peab minema mitte lahendama ära üksiküritaja oma testi tulemused vaid vaid sa pead moodustama grupi kolme inimesega, kellega sa oled koos töötanud, midagi koos välja mõtlema, minema, klassi ees ette kandma, mis sa tegid, eks siis nagu õppeviis või on need kõik need asjad, mis tegelikult jällegi noh, et, et Eestis nagu siukest reaalainete tugevus, vöö, mis need asjad olid, kõik, mida me seal piinaga me esimest korda pigem seal oli selline eba ebamäärasem see värk. Ja ühiskondade kontekstis oli küll, et nagu, et ise tuled mõtlen, kui ise mainin enne seda juba nagu tööl käinud ja siis sinna jõudes ka väga heasoovlikult klassivennad vahest küsivad, et kas tal Eestis telekaid on meil see nende arusaam sellest, mis, mis seal raudse eesriide taga nagu toimus, üsna hägune, mis inimesed mõlemat pidi. Aga ma saan aru, käisite tööl? Või vähemalt ju. Ma ei tohtinud seal tööl käia, see oli nagu see vahetusõpilase tegevus kuskil just, et siis ma ütlesin, ma ei tohtinud seal tööl käia, siis selgus alguses see kurvastas mind väga, aga siis ma sain aru, et seda defineeritakse läbi palga. Eks ma käisin mingisuguses arvutipoes nagu pärast kooli seal abiks, ma ei tohtinud palka saada ja siis pärast, kui ma Eestisse kolisin, output kinkis mulle arvuti. Mõlemapoolselt kasulik ööst, et see, see, see oli see lahendus, mis ma leidsin selle. Aga see oli siuke ütleme.
Noh, kuidagi seal see asi oli nagu noh, tõesti pigem käsitleti nagu koolipoissi, kellel lubati arvutit kokku panna, et see seal nagu see idee oli ju palju Klementeeritum kui see, mis nagu Eestis samal ajal toimus juba töö siin, nii kvalifitseerumise lihtsalt. Me ja teine asi, mis mulle väga meeldib, mida puudutanud, mis me enne alustasime, oli see, et meil oli, tekkis sihuke rühmitus nimega intellekti vastu eluks. Mis juhtus ka, ma arvan, seal üheksandas kümnendas klassis olin mina Mark Def Kristjan Jansen ja alavi Aho natuke hiljem. Kes Mac ja kikka olid Treffneris, mina alavi olime minemas ja Mark on ilmselt Ta ütleb, et mis, mis mind nagu, miks ma ei ole, tänapäeva meedia oli see, et mul õnnestus väga noores eas õrnas eas näha lähedalt inimesi, kes tegelikult oskavad programmeerida. Ja Mark oli nagu juba tol hetkel, kui koolipoisina oli, oli inimene, kes hommikul aks tekstifailides Hendrik õhtu, palju seal käima siis töötas, oli, kirjutas näiteks mingisuguse graafikamootor või midagi. Ja, ja see sümbioos, mis meil tekkis, seal oli selles Mac, pages kikka disainis. Mina korraldasin asju, mis võis tähendada mida iganes, alates sellest, et Femeksist laenata sound solklasteri kaarti, et maksaks ka sellele audiodraiveri kirjutada. Ja Alari Aho tegi muusikat ja see toode, mida me ehitasime ajuti mängime trancard.
Ja ja siis minu see.
Isegi ei suutnud meiega välja mõeldud, mis ma, mis ma seal seltskonnas valmistu orga, naise ametlikes paberites, aga et siis mina, mina hakkasin seda maha müüma. Et siis ma olin tihedas kirjavahetuses epic megageim siia äpolsi mingite selliste ettevõtetega, kes nagu kõik nagu olid valmis meiega rääkima ja siis, kui ma usse läksin, siis sel hetkel nagu tekkis nagu veider olukord, et ma sain nagu USA postiaadressid saata flopiga demosid ja välja näha, nagu meil oleks nagu mingi pärisfirmad. Et see kõik, nagu me panime paari aastaga, esiteks kooliga pealt me ei teinud seda asja lõpuni valmis, et nemad olid olemas. Aga, aga me panime paari aastaga selles suhtes nagu mööda, et, et me kirjutasime kaks t platvormikat mis ilmselt nagu 91 oleks nagu lennanud, nagu see taseme juures, mis matjagi ikka nagu, nagu kokku töötasid oleks olnud ilmselt täiesti vabalt nagu müüdav, täpselt nii nagu bluumuni kutid, venda, mängija, maailmaviisid. Aga aga Me komistasime täpselt sel hetkel, kui meie saatsime demod, siis vist oligi vist kas ID soft või Wulf Einstein oli juba väljas tuum tulemas või midagi sellist, ehk siis nagu ütleme see, mis me enne rääkisime ka, et seal graafikatase läks sinna, kus kolm D. Just Tatu koolipoiste platvormikas ei paistnud nagu, nagu säravalt silma, aga põhimõtteliselt asi isegi nagu töötas.
Ise mäletan, tol ajal tundus ettevõtmine sellepärast, et sul on mingisugune muusika, mälutandigid asjades, mitu Seidi käid, muusika, mingid taustad, seda oi, seal oli noh, teine asi oli see, et vaata kõiki ka mängugi, et isegi tänapäeval, aga tol hetkel iga mängukirjutaja, sa alustad sellest, et sa pageda endale töövahendid. Ega sa ei saa, sul ei ole tööstuseks levelit disainida. Mingisugused sul täkkeid täkkeid selliseid nelja või kaheksa kanaliga taustamuusika tegemiseks olid olemas, mida jälgi bluubel tootis veel. Aga, aga mai tea leveli disainimiseks või, või, või isegi mingisuguste. Kui kasutada mingi pildi egi diktoritehas bait valmis, aga kus aspationi animeeritud siis selle jaoks nagu sa pead jälle mingi omava tööandjad tegema, et meil oli kogu sees täku olemas.
Mis teid siis tundus, vägev, oli mingisugune motivatsioon rikkaks saada? See tundus ikka äge, lihtsalt ma mak make giga tegustasid, enne jällegi peaks küsima, kuidas me kokku saime.
Kikka mingi paar aastat tagasi, ma ei tea, ta vist võttis maha, aga paar aastat tagasi oli kokku kogunud kõik meie tolle hetke kirjavahetus ja palju seal avaliku interneti kõik nagu pildifailid ja mis me siis siis mul oli seal juba hea mitu aastat tagasi, et siis oli koju sihuke nostalgiarännak. Aga, aga neil oli Pokk sees, seal mängu taustaloos oli ka mingeid. Ma mäletan c trankad, kes pidi korjama, et oli ta veres, alkoholitase ei langeks. Siis selleks pidi koju pudeleid, siis ta sai tühjade pudelitega loopida ja olid mingid olukorrad, kus ta pidi natuke ajutiselt sama Kainimaks, siis neil oli seal Mackiegi ikka üks klassivend Treffneris olid, kui ta koolipeol liiga palju õlut ja siis ta hakkas kükke tegema Kainimaks, siis trancadiga see ka noolt hoiad all, siis ta tegi kükke alkoholitase veres, langes.
Aga seal on mingisugune uus element, oli sees, sest jooksmine kindlasti ka pold. Meil mingi idee, täpselt meil mingi idee oli, et, et me tahaks teha mängu, kus ei käi nagu tulistamine ja tapmine ja mis oleks nagu selgelt nagu teistsugune.
Näiteks kui teadlikum seda mõtlesime, et see nagu täitmatu uudis, et, et nii-öelda täiskasvanute mäng, mida muidugi ala alaealiste Ida
Arusaam sellest, mis asi on mõnevõrra teistsugune kui tohtidele seega selgelt vastas 90.-te tatuda
Ja olid ka.
Seal Californias midagi huvitavat.
Seal ma ka ikkagi pagesin, ma seal hakkasin. Seal juhtus selline asi, ma ei mäleta, kas ma seal natuke pidasin, viib s, aga noh, kui see eraisikuna pead nagu koduliini BBC ei ole nagu see ja pluss, et nagu sa satud nagu USA telefoninumbrit ruumi ju, tegelikult noh, pigem mai siis nagu BBC kasutaja ikkagi seal. Ja, ja siis hakkas ka internet noh, ütleme graafiline veebibrauseri nagu ilmus orbiidile ja see kõik pilt hakkas muutuma. Ma kirjutasin seal ka ühte BBC softi. Sihukese hobiprojektina tundus, et see võiks olla noh, jällegi nagu sageli teed, et nagu, et kui sa mängu ei pagenud, mis on see asi, mida sul endal vaja on, siis ma hakkasin selle käigus vist juhtus niimoodi, et ma ma uurides, mis seal veel ja kui tuju peal on, leidsin ühe ägeda BBC softi, mis mulle meeldis mis oli šev ja kinni keeratud. Ja siis siis ma mõtlesin selle lahti, et nagu sellise noore häkkeri nagu, et noh, midagi nagu siukest komplitseeritud, aga sa jooksutad seda asjade, pagari, pagari, siis vaatad, et kui kui imelike ootamatute kohtade peal hüpatakse mingisuguseid mälus mingi aadressi peale, tehakse seal müks mingi väga lihtne tehe ja siis lihtsalt nagu nagu masi koodi tasemel. Muudad ära, et sinna ei ame, hüpatakse siis ootamatus, õigused oligi koopiakaitse maas. Siis ma ropoteesin seal autojuht, et muide, selle saab niimoodi maha võtta dist andis mulle eluaegse tasuta litsentsi ja see võttis mul oluliselt poti maha, et oma BBC softi kirjutada, sest jumal oli selle koledad. Aga, aga, ja teine asi, ma mäletan, see oli õudselt hea õppetund jällegi siukest enda tasemest pagejana, et et ma kohutavat abstraheeris siin selle asja üle. Et kui sa nagu Siiblus plussis piibi essi, kus ma katsusin hoida väga puhtalt eri kihtidena näiteks seda, et kuidas käib modemi händlimine, kuidas käib terminali jändamine, mingid asjad nagu olles valmis igasuguseks tulevikuks, et sul neid asju, millega liidestada mitmeid ühesõnaga, ja siis ma ei saanud nagu jube jube kaugele oli kogu aeg sellest, et see asi nagu minimaalses skoobis töötaks sihukeses hilismaaeluprojektideks.
Täpselt kunagi valmisid, aga võib-olla see, et see, et ma täna rohkem start-upidega tegeleda MVP nagu kuidagi Alpso, mõtlesin, pigem näeks nagu vähemaga ja kujundus ette ja siis ta ütles, et Ameerikast tagasi kodus arvuti ja raamatud. Jah, oligi umbes sihukeselt pae kastioidki, mis siis, ma läksin tööle, siis ma läksin keskkooli edasi 12. klassi, aga, aga siis mul oli juba natuke hoog sees ja siis ma läksin tööle tööle sellisesse firmasse nagu triip. Ja mis oli tatus, selline algselt trükikoda disainipeo, aga kuna disaini osa ja mul on eluaeg kuidagi nagu isegi siis, kui ma nagu pagesin, mulle on alati meeldinud nagu see, kus on nagu tehniline osa ja visuaalne osa või kuidagi nagu kokku saavad, et ma olen eluaeg kõik asjad, mida ma olen teinud, ma olen alati töötanud gospagejate ja disaineritega, hiljem on ju, et et olete ka ma ise USAs üks asi, mis me tegime, oli see, et et mättad olid siukesed, kunstirühmitused kes tegid Askyjaatia hiljem VGA mingi taati ja siis ma mingite varjunimede all koos ühe koolivennaga isegi komistasime paari sinna sisse, et minu, minu aski ansi aeti, on olnud mingisugustes distitutsiooni pakkides isegi. Aga et siis selle tausta pealt ja USA-s üks kooliaine, mis seal koolis oli, oli ka laat, oli selline tootedisaini ja pakendi ja mingi selline asi, et siis ma nende näidistega ilmusin sinna triipu välja, ütles, et ma tahaks nagu pääst kooli natuke arvuti taga istuda mis antud juhul tähendas disainimist ja siis sattusingi seal tööle ja pluss noh, see kõik oli nagu üsna kaootiline, teine hästi-hästi mõju, kas inimene või hästi palju mõjuttoni tol hetkel, Marek Tiits kes pidas seda IPS-i või balti õpingute instituuti, mille alt ta testis edukalt mingisuguseid eurorahasid ja tegi sellega ägedaid projekte. Eesti.
Seaduste otsingumootori Kiwomehide oli veel selle ja Marek ka kunagi kuidagi nagu andis mulle, kui lihtsalt siukese ringi hängivad koolipoisile võimalused, tule, tee midagi aeg-ajalt ja, ja see tähendas jällegi ligipääsu arvutitele. Tähe toimis. See oli ka väga naljakas koht, seal oli näiteks üks silikon krahviks arvuti veebikaamera selline 90 keskel ja seal oli ka oluline funktsioon, et sinna oli võimalik sisse logida ja vaadata veebikaamerast, kus kohvimasina täis jooksnud, et siis ei pidanud alumiselt korruselt teisele korrusele, tulevad kas tühja tassiga. Ja seal oli üks mingi erakordselt oluline projekt, miski ei mäleta, miks seda vaja oli. Aga siis ma kirjutasin Berlis, seal oli üks tsükseli modem, millel oli ka faksi funktsionaalsus ja kuna seal majas oli ka internet, siis ma kirjutasin Bellis veebipõhise faktide saatmise vastuvõtmise aplikatsiooni Pirlis veebipõhise foksid, et kui keegi keegi saadab faksi sellele numbrile, mis võttis maden vastu, võttis need failid, kirjutas maha mingisse sunni sööbess, siis oli võimalik üle veebiteid faktsia teha. Et.
Ma ei ole kindel, kas see oli asi, mis, mis ma ise mõtlesin, et võiks teha ja muidugi, või oli see mingi asi, mis, miks projekti jaoks oli vaja, aga, aga sihuke asi seal tekstis ja veel aastaid hiljem vöös teatavas suures ettevõttes, mis kindlasti ei olnud telekommunikatsiooniettevõte, räägiti sellest, et oleks äge Foxy saatori, internet. Minu arust oli kaks aastat tagasi Viljo, tegi aprillinaljana mingeid pagejategid Twilio faksi appi ja nüüd on mingi oluliselt kasvõi välisuutel. Sellepärast et sa pead miljoneid inimesi, kes tahavad kogu aeg faksi saata. Äge jätk, mõtle, mis kõik oleks võinud olla triibupealik, kes su sinna, kuidas seal oli juss, peedimaa Juhan peedimaa oli jälle mu otsapidi kontakt seal ja Eva, kes nüüd on ka peedimaa või sellest ajast ja siis Priit Jagomägi, kes oli Jago mägede, pere kuulus Regio ja kartograafia taga ja Priit oli siis selline räbal vend, kes tegi oma firmat, mitte mitte ei tööta tegijast. Ja, ja see oli siuke k kuidagi 90.-te alguse. Põhimõtteliselt päris mitmed asjad, ma olen Eesti suhteliselt sellepärast, et sul on nagu täiesti tühi maa, on ju, et ma mäletan seda, seda hetke, kui ma keskkooli lõpetasin, Eestis oli umbes 40 panka ja keskmine panga mingis jõude, vanus oli mingisugune 28 umbes et siis tekib nagu tunne mingist rongist maha jäänud, et kui sa oled 18. Ta on joad just ja, ja, ja see tehnoloogiaettevõtlusinternetiga seotud, et see ettevõtlussee oli tegelikult see laine, mida me tol hetkel ei teadnud, kuhu me maandume. Aga mis, mis selgelt oli see, et ei olnud rong veel läinud, et me saime omale ongi ehitada ja see triip oli täpselt selline, kus ta alustas sellest, et oli eesti tühid täiesti tühi maa, iga päev tehakse kümneid firmasid igale firmale vaja logo ja visiitkaarti, siis nad hakkaksid neid tegema, siis ostsid sellest või nii palju rahapealise disaini pealt ostsin oma trükikoja siis ostsid maine veel mingisuguseid asju kokku ja siis selle sisse komistasime selle internetiasjaga jälle. Et kui ma seal noh, et, et seal võib olla, kujundati mingisuguseid trükiseid, reklaame, aga siis mina hakkasin seal tegema, aga veebilehti nendesamade klientidele. Ja siis sealt kasvaski välja, et kui ma keskkooli lõpetasin, siis ma läksin rõivust ära ja tegin oma esimese ettevõte, mis siis tegelikult nagu Voltaire halo, onju? Jah, oli küll. Ja siis saime meie tuttavaks jälle siis või mitte tuttavaks, vaid nagu töiselt. Teadsin varem.
See, kuidas horo sündis see on veel.
Eriti keeruline, ma mäletan vist praegu Gazpromi endised tooteproduktsiooniettevõtjad projektivine töötamas. Saabus Clint siis õlid projekti ühistult näha, laual tanksaapad, mille sees olid projektijuht, kes magas. Oi, seal oli palju selle sõnaga seal kõige ilusamaid mälestusi veel, et kui Lähed kontorisse hommikul ja siis vaatad üks disainer poeb valgete linade vahelt välja ja siis selgub, et ta tõsteti juba kolm kuud tagasi ühikast välja. Aga mina, ta kogu aeg esimesel ööl ja hapupiimast keldrist elasse.
Aeg-ajalt tuli ette, 90.-te Tartu oli mõnevõrra teistsugune, kui näiteks. Aga sa rääkisid, et tekkis võimalus oma rongid. Kas sul oli siis sul oligi sihuke selge visioon, tuleb laine ja ma lähen sinna laine peale ja nüüd on seal oluline asi, mida teha. Või sa lihtsalt kuidagi tegid seda, mis nagu mõnus tundmus.
No ette mõtlemist selgelt liiga vähe, et seesama all oleks ikkagi nelja aastaga pankrotti kaaslastega mingi piirini, nagu on võimalik ehitada, ehitada asju intuitsiooni pealt, aga mingil hetkel peaks nagu jalutama läbi ka või mõtlema, et mida see et, et Seli see oli, noh, mingil määral oli selline.
Avastus, et, et see asi, mida inimestel vaja või nagu suurtel firmadel näiteks noh, et kes nagu halo klienditöid ja seal tõibus jumakas kleidiks amet nagu täiesti kes on kes kõik mingisugused suured pangad, mingisugused rahvusvahelised bändid, kes olid Eestisse jõudnud, ma mingi audi või, või ESS, mis tol hetkel tohutult kasvas, aga mingeid siukseid nagu kauba mänginud ja kõik teadsid ja siis nad on valmis nagu see internetivärk oli nende jaoks nii arusaamatu, et sinu jaoks on see intuitiivne ja lihtne ja nagu milles siis probleem on, teeme ära ja siis on nagu suured kauba kõigega nõus, nagu alla neelame selle, et nad ostavad mingite kaheksateistaastast üheksateistaastaste tartu kuttide käest teenust. Ja noh, ei osanud seda hinnastada, on ju, et noh, et iga kord mõtled, et kurat, see number kõlab nagu liiga suur, et ei tea, kas nad selle nagu alguses toimis nagu liiga hästi lihtsalt ehk siis. Ja siis ostis, üheksin kuus alustasime 97 kevadel ostis üks suureklaamingeid teedeebee halo juba ära või kontrolli seal ettevõttes. Ja, ja siis me sattusime ootamatult nagu päris ärikeskkonda, kus on päris inimesed, Soome juhid, kes olid nõukogus, tahtsid mingisugust eelarveid, näha mingisuguseid asju ja et ühest küljest oli see kõik nagu selline nagu vaeses nooruses nagu kokkupuutumine päis asjadega. Teistpidi ka need nagu reklaami ja, ja nagu investorid, siis tol hetkel ka nemad Pavesti läksid siis internetibuumi sisse, et ka nende kliendid loopisid raha vasakule või ma tea, rahvuslased, kes seal olid veisel siis kes võttis meid 100000 dollarit esimese kohtumise eest mingid sihuksed agentuurid. Ja, ja, ja et see tähendas seda, et nad ei, Nad ei suutnud. P näiteks palkasime selgelt kohe liiga palju inimesi, sellepärast kohe-kohe pidid nagu need lääne kliendid tulema Eestisse asju arendama ja meil nagu tõsised, mingisugused jutud seal teedeebeegeti sees, et siis kui see mull nagu 2000 nagu lõhkes, siis lõhkas kõigi jaoks kojaga, et lõpp, kes meil Eestis lõhkes seal ja, ja vot see oli asi, mida, nagu sellist nagu täiesti vaheliste tüüpi teame ja sul on see, et majanduses on tõusud ja langused. See, see oli nagu täielik müstika, kui see juhtus, oli ta ei olnud ühtegi orienteerunud. Ja see noh, mitte ühelgi hetkel ei olnud seal sellist tunnet, et oh, et me, Me oleme ettevõtjad, võimlemisstartup või, või, või ikkagi sa teed asju, mida sa oskad ja mida nagu mingi tõmme tujust või tahetakse, et seda teeks ja järgmine homme küsitakse multimeedia CD, homme teeme neid ka multiCD-ROM. Müstiline asi, et kas tõepoolest ühispank küsimuseks, kas ma võin seda teha küll ja, ja võis olla küll, jah. Ma arvan, et ühiskaks tükki tegime suut, üks oli ühispank, millele tegime ja selle lainelt müüsime ära Eesti Telekomi. Ja see idee oli selles, et kuulge, et varsti on aasta 2000, mis ta oma aastaraamatut paberitükkidele
Mäletan üks neist maksid mäleta kumb max 200000 Eesti krooni selle eest. Aga teisest küljest ära mõtled, 15000 euro eest ei saa ühtegi proget liigutama. Kostitada.
Mind, mis mind tagasi vaadates seda rohkem tõenäoliselt siit ka see võib olla siis sind ka, et.
Teadmata midagi virsiooneerimisest, testimisest ning süsteemselt. Kuidagi suutsime mingisuguse softi, mis enam-vähem töötas, meil häbematus, klientide klient maksja varbad ära. Seal oli üks asi, mis ma olen, nagu mõelnud paar korda, on see, et et nii nagu vaata see kahe tuhanded või see 90.-te internetilaine peal kogu aeg uuest majandusest ja tõesti palju neid kohti, kus, nagu hajuti, et uus majandus ei alluvale majandusreeglitele. Mõnes asjas ei allu ka, aga kui turu suuruse mõõtmine või füüsiline kaugus ja mingid asjad aga, aga mingites põhiasjades ikkagi vood tuludega töötada kulusid nagu alluvaid. Ja, ja siis samamoodi nagu selline vastandamine e-kommerts ja päris kaubandus ja kõik nagu kuidagi käsitleda, öeldi, et minu arust me tegime mitu aastat seda äri nagu selle koha pealt valesti, et me mõtlesime, et see veebiehitamine ei ole nagu tarkvaraarendus. Tegelikult meil ei olnud seal tiimis keegi, kes oleks nagu mõelnud ühegi veebisaidi ehitamisest nagu tarkvaraarendusprojektist või lugenud mõne raamatu selle koha pealt, nagu, et kuidas suust pätid kuule tunnust veebi oleks ju nagunii lihtne. Et siis, kui see kogemata ehitas sinna taha ka mingi sisuhaldussüsteemi, mida me üritasime, mingid asjad siis sellel hetkel oleks pidanud nagu üle minema. Ma arvan, et tagantjärgi see, mis Taavi, Kotka ja veel meedias tegi, on ju, et nad hakkaksid mõtlema oma tarka õenduse protsessi peale oli see, mis nagu hoidis neil nagu elus nad alustasid, vestlesin natuke hiljem aga, aga, aga ta nagu kiiremini hammustasid läbi, et kui nad lähevad suurt maksuameti infosüsteemi tegema, siis ei peaks, nagu seda veebilehena käsitleb. Ehitus on ikkagi veel ühtesid, mis midagi kogemata mingeid skripte taga. Ja siis kuna andmebaas tundus keeruline, siis võeti kõiki maailma asju, hoiti koos projekti failide näiteks mida reageeriti tellija puhul, et kuidas see oli tol ajal tiiv hüppega kuidagi seotud sinna otsa koperdas ja see oli ka üks kooliaegne asi, kus jällegi seesama Tatu ja füüsika instituut ja, ja füüsikute seas oli ka Jaak Aaviksoo, kellega mu isa käis koos ülikoolis kunagi samal ajal. Ja kes oli ka Miina härma vilistlane või teise keskkooli vilistlane, ehk siis mingisugused seosed jälle seal ja siis siis kui tiigrihüpet tegema hakati, siis oligi Toomas Hendrik Ilves ja Jaak Aaviksoo mõtlesid välja, nagu nad ise räägivad kolmekesi, et Ilves, Aaviksoo, Johnny Walker ja, ja mõtlesid selle oma rollides tol hetkel välisministrina ja haridusministrina välja ja moodustasid selle ümber tiigrite peakomiteed. Ja siis 96 koma jäin keskkooli viimases klassis, siis, siis kutsusid mind sinna õpilaste esindajaks. Sest nagu komiteesse, mis on iseenesest nagu žestile nagu suhteliselt ilusaid tehes nagu haridusprojektiks või õpilane kaitsega seotud. Aga kuna see oli selline kujutanud ette jällegi oli, et ma lähen Tartust bussiga Tallinnasse, haisus ministeeriumisse, kus toimub koosolek, kus on umbes nagu Peeter Marvet ja Marju Lauristini, Ants Sild ja mingid sellised korüfeed nagu laua ümber, siis ma sisuliselt istusin seal komitees lihtsalt vait ja kuulasid nagu et nüüd tundus lugudest müstiline. Et, et ma arvan, et minu kõige suurem nagu tiigri palus, oli see, kui me käisime ühe kaevama sel maailmal, mu esimene tele, ETV ka otsesaates seda täpselt ei mäleta, kas ma kunagi üldse kaameratele lisatud, et aga aga otsesaade, kus oli Tiigrihüppe teemani väitlus, kus oli Marju Lauristin versus Lauri Leesi ja mina. Ja siis kuidas sa oled otse-eetrisse kunagi televisioonis on see keskkoolis ja siis siis noh, jällegi kaks suhteliselt nagu lõbu tunnustatud inimestel raske, ja siis vaatad, et kui Lauri Leesi jahub sulle midagi sellest, et et arvutit kooli vaja. Kristel on aastasadu vastu pidanud ja siis sa saad aru, kui läinud on, nagu need selle tüübi jaoks oli, et sa ise juba tunnetada seda, et kuhu maailm liigub ja mis seal nagu juhtub. See oli ka tegelikult see koht, kus ma tegelikult sattusin sinna, et ma ei, ei läinud.
Ülikooli kui ma läksin, siis ma läinud õppima arvutiteadus siin, sest umbes sellest hetkest. Ma ei suutnud seda võib-olla nii hästi sõnastada, aga et ma juba juba nagu pagesin juba piirasin BBC hängisin internetis ja mul tekkis tunne, et see asi, kus on mul nagu tegelik lünk, on see, et miks need asjad toimivad nagu võrgustikke ja see sotsiaalne pool. Ja kuidas täpselt sel hetkel sealtsamast saatest välja tulles Lauristin küsis, et et kuule, et me teeme thats uute eriala, kus me hakkame kommunikatsiooniteooria, ütleb, et tema asju ei tule sinna. Et siis siis seal oli ka see, kuidas sattus üldse nagu sotsiaalsee kompromiss, mis ütleme paljude jaoks ka ma enda jaoks oli suhteliselt üllatav, et polügooni sihukestesse tegevused
Jah, võrreldes kõik need paketid, kes olid dialoogid, läks mul isegi hästi füüsikat. Millega sa praegu tegeled? Teekond sind on toonud tänaseks.
Lühidalt põhist ehitan ettevõtteid, et ehitan ettevõtteid. Pigem nende esimeses otsas aeglases faasis. Et mõnes mõttes võibki öelda täpselt sama, mis 90.-te tegime, aga lihtsalt iga kord nagu tead seda võib-olla natuke uue kogemusega või natuke juba tead ka, mis et ma mitte ettevõtteid siis sedapidi, et ma panen osade alla oma aega, et ma hakkasin tegema siukseid võivad nagu telefon viis aastat tagasi ja kaks aastat tagasi meid osteti ära, et ma tegutsen edasitud trupi ja kes meid ostis. Ja juhin nagu põhjust oli see niisugune asutajale see hetk, et kas nagu tahan suurt tüki väikest pisukest või väikest tüki suurest pirukast ja ja, ja trupi ja on. Või see telefonimüük ja õppija edasi tegutsemine jättis nagu võimalused on paar aastat nagu vahele jätta või hüpata trepist kõrgema öösse 12 inimesega, start upis. Nüüd meil on 100 170 inimese tiim ja, ja mul on kuskil 70 inimest tootevaldkonnas tootearenduse valdkonnas, kellega saab sama visiooni Giulia ehitatud. Ja siis, kui mul aega ülejäänud, siis ma investeering start-upidesse ja annan mõne nõu mulle tassi pudeli. Põhiliselt. Või noh, see sarnasus 90.-tega on see, et selleks, et et ma tean täpselt, mida ma suudan üksi ehitada. On see siis BBC oht või mitte, aga ma tean, mis väärtus, mis võlu on sellel, kui, kui proge ja disain ja äriinimene koos mingit asja teevad ja mida siis saab valmis teha alates siis trankad või, või, või mingitest veebiprojektidest halos. Ja ma tean, et ma ei taha kunagi elus müüa oma aega tonni põhiselt, sest neil on tunde lõplik hulk, järelikult tuleb ehitada tooteid, ehk siis see on noh, see läheb juba selle ajalisi aknaskoobist välja või hiljem Skype'is nähtud asi, et kui sul, kui Supaaviimiste ehitavad mõne kuu mingit asja, mida kuuega hiljem kasutab miljon inimest, et siis siis on asi, mida me kui see 90.-te selline rahmeldamine võib-olla õpetas, oli see, et et asi, mida ma alati otsin, nüüd on, on see, et kus sisse pandud töötundide hulk konverteeriks võimalikult suureks väärtuseks. Et tol hetkel kippus ikka väga palju olema, et kui sa nagu puhta õhuõhinapõhiselt üheksakümnendatel tegid noh tõesti, et oleks kõik neid samu asju teinud ka siis, kui oleks paka saanud. Et, et siis kui sa müüd oma töötunde, siis sulle lihtsalt nagu väga pikad päevad, aga lühikesed ööd ja, ja.
Eks neid tüüpe, kes seal üheksakümnendatel nagu väga läbiga põlesid, on juba juba kohe ostetavast hiljem, aga ja siis olid näiteks ma ei tea, kes olid seal legendaarsed mehed, nagu näiteks väga austan sellest ajast või Taavi Talvik näiteks on ju, et kui sa nagu samade tausta huvide pealt ehitanud nagu unineti, kus ongi, et sa võid ka magada nagu bitid, müüvad ennast ise. Et see seal on jälle nagu see oli küll selline teenuse infrastruktuuri, aga seal oli see nagu alge olemas, kuidas ehitad midagi, mis nagu saab hakkama ka siis, kui sina ei ole näppupidi juures kogu aeg?
                 
Kõneleja 1:
Ja see on väga õpetlik ja väga tore lõpp jutuajamisele, aitäh sulle, aitäh.


\chapter{Veiko Tamm}
\label{chptr:lucifer}
\index[ppl]{Tamm, Veiko}

\question{Hakkame peale sealt, kust asjad ikka pihta hakkavad. Kuidas arvutit 
sinu juurde said?}
                 
See on komplitseeritud küsimus. Ma ise olen ülikoolis keemia eriala\index{Tartu 
Ülikool!Keemia Instituut} lõpetanud. Aga ei oma neljanda kursuse kursusetööd ja 
diplomitööd ei ole ma ühegi kolvi ega katseklaasiga solberdanud, sest  mind 
kutsuti tollal üsna põnevasse asja. Nimelt juhendajaks oli mul Mati 
Karelson\index[ppl]{Karelson, Mati}, kes alustas arvuti ja kompuuterkeemiaga,  
ühesõnaga kvantkeemiaga.  Ja sealt siis olid minu esimesed kokkupuuted. Meie 
tööväljaks olid alguses perfolindid alguses, viieaugulised.

\question{Juba tol ajal ta üritas teha kvantkeemiat? Nonde arvutitega?}   

Jah, see asi hakkas pihta juba kaheksakümnendatel.
                 
\question{Ja sul enne seda üldse mingit kokkupuudet arvutitega ei olnud?}

No meil matemaatika kursuse käigus väga lühidalt näidati selliseid arvuteid 
nagu Nairi-2\index{Nairi!Nairi-2}, ja nende PA ja AP keel\sidenote{Vt. ka lk. 
\pageref{sisu:apkeel}.}. Sai endale \enquote{Hello World!} trükkida ja 
\emph{that's it}. 

Aga no, ütleme, need masinad, millega me hiljem töötasime\ldots Alguses oli 
Minsk-32\index{Minsk!Minsk-32} ja hiljem  Jessukesteks\index{Jessuke|see{ES 
EVM}} kutsutud ES-1022 arvutid\index{ES EVM!ES-1022}, mis olid Ülikooli 
Arvutuskeskuses. Arvutamine käiski nii et  perforeerisid oma programmi sisse, 
viisid sinna ja tema lahendas. 

\question{Mida need programmid tegid?}

Oli mitmesuguseid kvantkeemia meetodeid. Aatomorbitaalid, kuidas need 
keemilised sidemed moodustuvad, kuidas elektronpilved suhtuvad üksteisega ja 
kas see aitaks seletada neid keemia asju, kas  lähedaks arvutuslikke tulemusi 
reaalsetele. Ja need arvutused oli ikka nii, et võtsime lihtsad kahe aatomiga 
molekulid ja nendega oli nii,  nagu on. Aga keerukamate, näiteks seal metaan 
\ce{CH4},  molekuli välja arvutamine nõudis  sellelt suurelt arvutilt umbes 10 
korda rohkem tööd kui terve lihakombinaadi aastaaruande välja arvutamine. See 
oli päris kõva ja kallis arvutiaeg, mis sinna alla läks.
           
\question{Kuidas see käis? Juhendaja ütles \enquote{Veiko, hakka 
programmeerima} ja sa hakkasid programmeerima?}      

Ei, programmeerimisest jäi asi ikka kaugele seal. Tuli natukene keeltega 
pusserdada küll,  põhimõtteliselt tuli mõningaid asju Fortranis\index{Fortran} 
kirjutada ka. Et midagi ma nagu aru sain, aga seda programmeerimise asja ma ei 
ole selgeks saanudki. 

Siis katkes see asi tükiks ajaks ära. Taaskohtumine arvutitega oli, nüüd võiks 
öelda juba üle 31 aasta tagasi, kui Tartu Tähetornis\index{Tartu Tähetorn} ajas 
tol ajal suur arvuti-fänn Enn Kasak kokku arvutihuviliste ringi ja hankis sinna 
arvuteid.

\question{Mis aastal see oli?}

See oli 1988. aasta suvel. Sel ajal algas ju suur kooperatiivide ajastu. Sai 
tehtud kooperatiiv Tähetark\index{Tähetark}, mille liige ma olin ja mille 
eesmärk oli hankida planetaarium. See planetaarium sai isegi ostetud ja  
Füüsika Instituudi\index{Tartu Ülikool!Füüsikahoone} fuajeesse kuidagi üles 
säetud, aga ega temaga mingit tööd tegema ei hakatud. 

Aga samal oli see selline teadlik ja teaduslik keha,  mille varjus sai arvuteid 
osta-müüa. Arvutite müügiga oli üldse see asi, et hiljem keelati igasugustel 
väikestel asjadel see ära. Kui sa oled kunagi näinud vene seriaali 
\begin{russian}Бригада\end{russian}, segaste aegade maffia-elust, kus äritseti 
kõigega. Juhtuski niiviisi, et ma sattusin  olukorda, et mine ja too mingeid 
arvuteid. Käisin Peterburis, Moskvas, sain kätte otsekontaktid ja sidemed ja 
nii see arvutiärisse sattumine nagu järsult puhkeski. Sest arvutid olid tol 
ajal ikkagi hirmsalt kallid asjad. Minu esimene arvuti näiteks, Amiga 
500\index{Amiga!Amiga 500},  maksis sama palju, kui tutikas 08\sidenote{Nii 
kutsuti autosid VAZ-2108, tuntud ka kui Lada Samara. Tegu oli oma aja ja 
Nõukogude Liidu kohta innovatiivse autoga, Samara oli teine (esimene oli siiani 
populaarne Niva) ise arendatud AvtoVAZ-i mudel ja esimene, mis ei tuginenud 
Fiat 124 mehaanikale.}. No ikka mingi 40 000 rubla.
                 
\question{Miks sa sinna Tähetorni juurde läksid, arvutivärk ikka nagu tõmbas 
või?}

Sõber kutsus, et tule kaasa, põnev värk selline.

Ja siis ma mäletan jah, kuidas tuli algusest peale endale kõik see värk selgeks 
teha. Kuidas vaadata, et palju tal op mälu on ja testiprogramme kasutada. 
Alguses kuiva trennina, aga järjest hakkas see asi nagu liikuma. Ja kui endale 
sai arvuti koju ostetud,  oli jube põnev. Alguses ma rändasin  üle Amigate.  
500\index{Amiga!Amiga 500}, siis 500 Plus\index{Amiga!Amiga 500 Plus}, siis 
Amiga 1000\index{Amiga!Amiga 1000}, Amiga 2000\index{Amiga!Amiga 2000}. Ja siis 
sai sinna kõrvale esimene PC, sest tuli üks suurem tellimus. Arvuteid saada oli 
kuradi raske, sellepärast et igal pool olid embargod nende sisse vedamiseks ja  
põhimõtteliselt tõime arvuteid Moskvast ja 
Peterburist.\label{sisu!veiko_moskvas}

\question{Kuidas need arvutid Moskvasse ja Peterburi said?}

No näiteks see Moskva kanal, mis mul oli ja millega ma embargo-arvuteid tõin ja 
äritsesin, tuli läbi saatkondade. Singapuri saatkonnas toodi paljundusmadina 
kastis arvuti saatkonda ja siis Lumumba\sidenote{Moskvas asutati 1960. aastal 
Vene Rahvaste Sõpruse Ülikool (\begin{russian}Российский университет дружбы 
народов\end{russian}), mis 1961. aastal nimetati Kongo poliitiku ja 
vabadusvõitleja Patrice Émery Lumumba auks ümber Patrice Lumumba Ülikooliks. 
Kooli eesmärk oli toetada vastselt koloniaalsõltuvusest vabanenud riike 
koolitades sealset tulevast teadus-tehnilist eliiti. Praktikas oli tegu Moskva 
ühe vähese ülikooliga, kus õppis suurel määral välisüliõpilasi.} üliõpilaste 
kaudu läks see kohalike ärikate kätte ja sealt sain siis mina osta. 

\question{Sihuke tarneahel!}

Jaa, tarneahel oli päris võimas, kusjuures hinnad tarneahelas liikusid väga 
põnevalt. Iga ots pani omale ikka julgelt kuskilt seal 30 kuni 50 protsenti 
otsa, aga väga sageli ei viitsinud hakata liigutamagi, kui 100 protsenti 
kasumit ei olnud. 

\question{See oli ju riskantne äri?}

Oli jah, tean ikka väga palju tuttavaid, kes said kuuli ja nuga. Olen isegi 
kihutanud öises Peterburis punase tule alt läbi, sabad järel ja kõik. Kui sa 
ikka sõidad sinna kolm miljonit rubla sularahas seljakottidega kaasas, siis ta 
on riskantne.

Võtame ülekande rubla, vene ajal nimetati \begin{russian}песналик\end{russian}. 
Arvuti hind ülekande rublas oli kaks miljonit. Kui sa tõid sularahas selle 
raha, said arvuti kätte, ütleme, 1.4 miljoniga. Aga kui sa maksid valuutas, 
võisid üldse  ühe miljoniga kätte saada ümberarvutatult väärtustes. Ja teine 
lisaväärtus, mis tuligi  Moskvaga tuuritades oli see, et Venemaal, Moskvas, 
süva-venemaal, hinnati saksa marka ja  dollarit. Meil siin jälle, vastupidi, 
olid hinnas Rootsi kroonid, Soome margad, kellega ärikad äri ajasid. Ja siis 
oli nii, et ostsid siit kokku saksa margad, ostsid kokku dollarid, läksid 
Moskvasse, maksid nendega ära. Ja kõik need saksa margad ja dollarid, mis üle 
jäid,  vahetasid seal Soome markadeks ja Rootsi kroonideks. Ja tulid siia ja 
vahetasid ära ja üksinda selle ülejäänud nii-öelda valuutavahetuse eest võtsid 
ka rahulikult iga raksu pealt seal mingisugune sada tuhat vahelt. 

See oli hirmus aeg. Siis eriti enam ei kütitud, kuigi nõuka aja lõpuni kehtis 
ju  valuutaseadus. Et suurtes hulkades valuuta äritsemise eest võisid saada  
seitse aastat. Suureks hulgaks loeti juba seda, kui sul oli rohkem kui 100 
dollarit. Aga seal sai tuhandetega arvestatud.

\question{Kes neid arvuteid ostis?}

Oh jumal, kõik ajasid taga. Riiklikud ettevõtted, instituudid\ldots  Kui ma oma 
esimese PC arvuti ostsin, siis see läks Tartu Ülikooli 
Füüsika-Keemiateaduskonna pea serveriks. Ja see oli 286, 20 MhZ. Võimas masin, 
kahekümne megahertsine!. Tavalisel masinal oli ju ainult 12 MhZ. Tal oli kaks 
megabaiti op mälu ja 120-megane kõvaketas. Ja teise sellise arvuti ma ostsin 
endale. Muidugi käisid paljud tuttavad, kes ka arvutitega tegelesid, vaatamas, 
et \enquote{Mida sa, loll, tast endale ostsid, mis sa teed selle arvutiga, kuna 
sa selle 120 mega arvad täis saavat?}. Vot sellised ajad olid. Mulle siiamaani 
meeldib, Enn Kasaku\index[ppl]{Kasak, Enn}  üks selline paralleelne näide, mida 
ma olen pidevalt kasutanud. Et kui auto-teadus oleks samamoodi arenenud, nagu 
arvutiteadus, siis sõidaks Mercedes praegu valguse kiirusega, võtaks  10 000 
kilomeetri peale tilga bensiini ja maksaks pool senti.

\question{Tundub tõepärane. Mis tolle 286 serveri peal jooksis? Novell?}

Tead, ma isegi ei mäleta, mis nad sinna panid. Põhiliselt, mida jooksutati, 
olid ikka Unixid. V5\index{UNIX!System V} näiteks ja kõik sellised asjad. Siin 
ikka käis rahvast meie teadlastel välismaalt külas ja kõik väga imestasid seda, 
et nii palju tehakse Unixitega. Aga kõigi Eesti Unixite seerianumber oli üks. 
Ega peale piraatluse muud võimalust ei olnud. Tarkvara hindade juures, kes 
andis mingi tarkvara jaoks sellist raha!

\question{Ühesõnaga, kuna ei ühte ega viit ei jaksanud osta, piraaditi viis ja 
oligi edusamm!}

Et nagunii ei ühte ega teist ei olnud aga saadi vähemalt midagi teha!

\question{Mis aastal see PC-lugu oli?}

Mingi 1989? 
                 
\question{Sa ketrasid ennast Amigatest siis ikka väga ruttu läbi PC peale?}

Jah, kogu aeg vahetasin. 

Üheksakümnendate alguses, kui juba iseseisvus hakkas, oli mul kõige esimene 
kodu-386. Jälle imestati, milleks seda vaja on, kes sellise asjaga tegeleb, mis 
sa sellega teed. See oli tüüpiline.

\question{Sul pidi ikka siis mingi huvi olema, et sa neid arvuteid nii sageli 
vahetasid ja kooperatiivi ka sisse jäid?} 

Kooperatiivis meil väga kaua see asi ei kestnud,  sellepärast et väga kiiresti 
tuli peale see seadus, mis keelas mitteriiklikele ettevõtetele ja 
kooperatiividele arvutitega äritsemise. Ja kuna mul olid sidemed olemas, polnud 
kooperatiivi enam vaja. Me paari tuttavaga kliente leidsime ja nii edasi, 
tekkis küsimus, et kus kohta ja mida me teeme. Mida meil vaja on? Tootmisruume 
ei ole vaja! Meil ei ole vaja mingeid ladusid, mingeid tooraineid, midagi. Mis 
meil vaja on? Raha!

Mõtlesime, et Tartu Kommertspank\index{Tartu Kommertspank}\sidenote{1988. 
aastal tegevusloa saanud Tartu Kommertspank oli esimene aktsiaseltsina tegutsev 
ning ka välisvaluutatehingute litsentsi saanud kommertspank NSV Liidus. Panga 
tegevus lõppes pankrotiga 1994. aastal. See pank oli mingis mõttes oma aja 
tõeline sümbol põledes heledalt ja kiiresti. Ka Hansapank\index{Hansapank} 
alustas tegevust Tartu Kommertspanga filiaalina!}! Tore koht! Läksime 
Veetõusme\index[ppl]{Veetõusme, Ants}\sidenote{Ants Veetõusme, kes kuni 1990. 
aastani oli Tartu Kommertspanga juhatuse esimees.} jutule. Poisid ütlesid ka, 
et kui lähed, räägi, mis vaja on, et võiks olla nagu oma raha ka raha 
loksutada, küsi kuskil sada tuhat. Hinnad olid sellised, et sellega sai juba 
enam-vähem masina osta! Muidu äri käis ju kogu aeg ettemaksetega. Raha tuli ära 
ja sa võisid teda tükk aega pööritada, siis ostsid  masina ja andsid kliendile 
kätte. See, et kuu aega tuli oodata, oli tavaline nähtus. Tulid jälle suuremad 
summad, siis meil oli käsi üsna hästi sees Novgorodi elektroonikatehases, mis 
tegi  Panasonicu MV-25 pealt maha viksitud vene videomakke VM-12. Laadisid 
furgooni neid täis! Kui sinna läksid, et oleks hea suhe, kast vana Tallinnat 
paar kasti suitsusinki, meie oma suitsukana. Sellega, kõmm, Novgorodi, auto 
videomakke täis ja neid me ei viitsinud üksikult müüa, müüsime hulgi 
koperativšikutele maha, need siis oma kooperatiivipoodides parseldasid edasi. 

Sellist rahakeerutust sai tehtud kogu aja. Aga sai, jah, mõeldud, et võiks olla 
käibevahendeid. Algkapitali, nagu öeldakse. Tavaliselt tead, et kui  midagi 
küsid, siis nii kui nii tõmmatakse maha. Rääkisin 
Veetõusmele\index[ppl]{Veetõusme, Ants} ära, et vot selline arvuti-äri. Ta oli 
väga huvitatud, kõrvad liikusid, et kas neile ka saaks. Ikka saab! \enquote{Aga 
mis te meile pakute?} Noh, ütlesin, et 11\%. Meie näiteks teenime miljoni, teie 
saate 110 000. \enquote{Täitsa hea mõte! Ja palju te meie käest tahate?}. 
Mõtlesin, et küsin rohkem, niikuinii kaubeldakse alla. Ütlesin miljon. 
\enquote{Ahah. Avage arve, pange miljon peale!}.

Kusjuures meie firma oli selline, et kui see kommertspanga  pankrot pihta 
hakkas, siis meie olime üks väheseid, kes selle raha tagasi maksis. Oleks 
võinud põhimõtteliselt ka teha mingid varifirmad ja asjad ära kantida ja külma 
teha. Aga meie tasusime kogu selle raha ja kuskile võlgu ei jäänud.

\question{Kui sa neid amigasid ja PC-side keerutasid, sul pidi mingi huvi 
olema, mis sa tegid nendega?}

Oh, jumal! See oli ka omaette nuhtlus! Kui Tähetornis\index{Tartu Tähetorn} 
need esimesed MSX\index{Yamaha MSX} arvutid tulid, ma veel töötasin 
keemiainsenerina. Pärast tööd sõidad bussiga alla linna, lähed Tähetorni ja 
siis istud ja mängid seal täpselt nii kaua, et on aeg bussi peale minna ja 
tagasi tööle sõita. Vaatad hommikused ringid üle, keerad kabineti lukku, keerad 
magama. Mängud olid naiivsed, aga tead, ta  oli nii põnev aeg! Ja kui endale 
arvuti tuli, see oli košmaar! Järjekord oli pidevalt ukse taga, kõik tulid 
tasuta mängima. 

\question{Sa siis mängisid?}
Jah. Sai muidugi igasugu asju uuritud ja kui tuli internet, siis\ldots 
Tegelikult hakkas see maailmaga ringi käimine juba enne seda, BBS-i ajal.
                 
\question{Vot sinna ma tahtsin jõuda! Kust sul tuli mõte, et paneks endale 
BBS-i püsti? Ja millal see oli?}

See oli kuskil väga varastel üheksakümnendatel. Päris internet jõudis Eestisse 
kahe satelliiditaldrikuga, üks oli seal KBFI peal Tallinnas ja teine oli Tartus 
Tähetornis. Ja sealt siis üle Rootsi Kuningliku 
Tehnoloogiainstituudi\index{Rootsi Kuninglik Tehnikaülikool}, KTH,  käis meil 
side. Siis hakkas BBS-indus vaikselt juba ära vajuma, kuigi ta  töötas veel 
edasi, eks ole. Mäletan, et internet jõudis Tartusse,  ma elasin tollal  seal, 
märtsikuus 1992. Sain üle  EBC, Biokeskuse\index{Eesti Biokeskus}, endale oma 
isikliku \emph{account}-i juba aprillis, kuu aega hiljem.
                 
Tol ajal oli meie kontor Rüütli tänavas, kohe Treffneri kooli\index{Hugo 
Treffneri Gümnaasium} vastas. Muidugi ägedad trefneristid  käisid seal kõik 
hoolega arvuteid näppimas. Üks põhimehi, kes seda asja suunas ja üles pani ja 
majandas tarkvara poole pealt oli Einar Entsik\index[ppl]{Entsik, Einar}, 
praegu kõva kinnisvaraärimees. Tema oli nagu meie peamine \emph{sysop} ja mina 
olin siis \emph{co-sysop}. Hiljem, kui me kontori likvideerisime, siis Lucifer 
BBS\index{Lucifer BBS} tegutses mul kodus edasi, kuni peaaegu lõpuni, kui see 
BBS-i maailm ära hääbus. Siiamaani mäletan veel oma aadressi: 2:491.666.
           
\question{Millest selline nimi, Lucifer?}      

Ta tõi valgust maailma! 

Muidugi, meil oli väga palju igasugust sellist maagiat ja värki, kuna ma 
loomult olen anti-kristlane olnud eluaeg, nüüd olen ma veel suurem 
anti-islamist. Seda ma ei suuda üldse taluda,  selle kõrval kristlased on 
väikesed voonakesed. Vaimupimedust, mis siin on, keskaega tagasi pürgimist! 

Mäletan selgelt, et oli suur jama, kui KAPO käis meie neid materjale uurimas, 
kui mingisugused nõndanimetatud satanistid pussitasid Tartus Hando 
Runnelit\index[ppl]{Runnel, Hando}. Neid uuriti, et kust saadud ja kellegi 
kaudu tuli  välja, et meie BBS-is oli väga palju neid materjale, Lavey Saatana 
Piibel\sidenote{La Vey, Anton Szandor. The satanic bible. New York: Avon Books, 
1969.}. Käidi, uuriti ja vaadati. Ma mäletan üks mehike tutkis nii põhjalikult, 
et pööras täitsa ära, hakkas ise ka satanistiks!

Kusjuures kui sa küsid, kas ma olen satanist, ma ütlen, et ma ei ole. Kui ei 
usu kristlust, kuidas ma saan siis tema peegelpilti kummardada?

\question{Miks te BBS-i tegite? Äri sai ju muud moodi ka teha?}                 

See oli lihtsalt hobi, poisid tahtsid teha. Igasugused sidemed, materjalid üle 
maailma\ldots Tol ajal kaugekõned olid ju kõik tasulised aga no selle äri 
juures telefoni hinnad! See polnud tähtis, ma võisin tundide kaupa rippuda 
Ameerika või Iisraeli või kuskil\ldots Euroopa polnud üldse küsimus! Sai 
helistatud Jaapanisse, sealt igasugusi materjale tõmmatud, sai sealse skeenega 
suheldud ja.

\question{Kust sa numbrid said, kuhu helistada? Üheksakümnendate alguse Tartut 
meenutades, kust ma võisin saada Jaapani telefoninumbri, kus taga BBS vastas?}

BBS-idel olid ju kõigil suured \emph{listing}-ud olemas, kus oli maailma 
olulised BBS-id loetletud. See on nagu aadressiraamat,  telefoniraamat, seal on 
kõik riikide BBS-id sees! 

\question{Ja sa käisid seal infot lugemas?}

Jah, loed uudiseid, infot\ldots Nüüd  kõik istuvad Facebookis, aga siis seda ju 
ei olnud. Siis olidki BBS-id, mille kaudu käis info vahetamine, meili saatmine 
ja kõik. No hiljem, kui internet tuli, olid juba teised ajad. Et seal 
tegutseda, tuli endale UNIX-i alused selgeks teha ja käsureal töötada. Siis ei 
olnud ju veel Linux-itki olemas ega midagi. Põhiline oli just Santa Cruz 
Operation\index{Siit tuleb lühend SCO.} V5 UNIX\index{UNIX System V}, millega 
me kõik siin tegutsesime.

\question{Ohoh, too BBS käis teil UNIXi all?}

Ei BBS-il on oma tarkvara, UNIX tuli hiljem, kui hakkasime juba internetis 
käima. Kliendiga, lehvik kuskile terminali otsa\ldots Ega siis kellelgi kodus 
interneti polnud, ei olnud võimalik saadagi. Pidid teadma, kus istusid  sisse 
helistamise modemid, millega sa said ennast kaugelt kuskile interneti arvutisse 
sisse logida. Näiteks Toomemäele Tähetorni\index{Tartu Tähetorn} ja sealt siis 
juba edasi liikusid, siuh-säuh, internetiavarustes.
                 
\question{Ja kõik käis käsureal!}
                 
Jah. Hiljem hakkasid tulema Gopher ruumid ja muud sellised algelised 
otsingusüsteemid. Infopangad, kus oli erialaseid raamatuid. Siis tekkisid 
interneti BBS-id. Printa oli näiteks Euroopa üks suurimaid ja 
Iska\sidenote{Iowa Student Corporation Association} BBS\index{Iska BBS} oli 
maailma kõige suurem interneti BBS veel sellisel kujul. Nagu BBS, 
teadetetahvliga, lihtsalt sinna ligipääs oli Interneti kaudu.

\question{Kas sa teiste Eesti sysopidega ka suhtlesid?}

Ja, ikka, meil olid ju igasugused üritused, BBSummerid\index{BBSummer} ja 
BBWinterid\index{BBWinter}. Seal käisid nii sysopid kui ka  kasutajad, oli 
niisugune päris tihe seltskond, kes seal käis ja omavahel niisama suhtles, no 
nagu praegu Facebookis käivad suhted.  See seltskond ei olnud nii suur, et ka  
ei oleks võimalik hallata. Katsu sa teha näiteks Eesti Facebooki liikmete 
kokkutulekut, võib-olla  jääb tulemata  viis protsenti inimesi!
        
\question{Kui palju teil Luciferis\index{Lucifer BBS} liine oli ja, anna palun 
suurusjärku, kui palju kasutajaid küljes käis?}         

Üksainuke telefoni liin. Sellega oligi see, et kui kasutaja tuli  liini külge, 
siis ta pidi seal rippuma. Kõneaeg jooksis kogu aeg. Et kui sa näiteks 
sikutasid mingit tarkvara kuskilt Ameerikast, siis sa rippusid kogu aeg 
kaugekõnega liini peal, päris soolane kopikas tiksus! Eraldi üüriliinid tulid 
alles ISDN-i ajastul, kui tulid 64 ja 128 kilobitised asjad. Algselt, kõige 
esimene modem, mille ma sain, oli 2400 boodi.
                 
\question{See oli isegi juba kiire, sest 1200-sed olid ka levinud}

Isegi 600-sed! Finlandia  BBS\index{Finlandia BBS} oli 2400,  ülejäänud kõik 
olid aeglasemate peal. Ega see modem maksis ka kaks korda rohkem kui sõiduauto 
Žiguli, niisugune tavaline.

\question{Hobi jaoks tundub Žiguli nagu kallis investeerida?}

No võtame niiviisi, mõni rikas mees korjab hobi jaoks vanemaid autosid, 
uunikume, mis maksavad ka meie praeguses rahas seal sada ja kakassada tuhat. 
Teevad oma automuuseumi. See on samuti hobi, ega te sellega ka midagi muud ei 
teeni, piletit ka ei küsi!
       
\question{Ma ikka ei jäta. Mis sind just selle hobi juures paelus, mis hoidis 
sind arvutite juures?}          
Seesamane kübermaailm. 

Kui vaadata praegu Ghost in the Shell-i\sidenote{Masamune Shirow samanimelisel 
mangal põhinev frantsiis, millesse kuulub nii animesid kui ka 2017. aastal 
Hollywoodis linale tulnud film. Frantsiisi tegevus leiab aset post-küberpunk 
maailmas ja selle peategelane, Major Motoko Kusanagi, on küborg, kelle 
mehaanilises kehas (\emph{shell}) toimib inimese teadvus (\emph{ghost}). Tegu 
on kunagistes küberpunk ringkondades kultusliku teosega, mille mõju on 
võrreldav Willigam Gibsoni loominug omaga.} või midagi sellist, et oleks 
võimalik oma teadvus enne surma Internetti üle kanda. Läheks küll sinna 
virtuaalmaailma tondiks!  

\question{Kas sa tol ajal Gibsonit ka juba lugesid?}
Ja, ikka.

\question{Siis on selge!}
                 
Kõik see küberpunk ja see värk oli sisuliselt \emph{must be} kõigile, kes olid 
toll ajal arvuti-friigid. Siis muidugi virtuaalmaailmades\ldots Kuhu ma eriti 
sisse ei jõudnudki, olid mudad\index{Muda}. Mina sattusin virtuaalmängumaailma 
siis, kui tuli selline asi nagu EverQuest I\index{EverQuest}. Seal sai 
järk-järgult läbi gildide mindud, mängisin seda mängu neli pool aastat jutti. 
Mängus sees oli \emph{online counter}, mis luges, kui palju sa mänginud oled, 
mitu päeva, mitu tundi ja nii edasi. Summeeris kokku. Ja kui ma pärast sealt 
vaatasin, siis selle nelja poole aasta jooksul, ma oleksin pidanud iga jumala 
päev mängima neli ja pool tundi. Aga mõnikord oled välismaal kuskilt ära, ei 
mängi. No, polnud sagedased, aga polnud ka väga haruldased juhtumid, kui ikka 
kakskümmend tundi jutti näiteks suuri \emph{raid}-e peetud.
                 
\question{Kas Gibson ja muu küberpungi kraam levis BBS-ides või olid füüsilised 
raamatud ka?}

Olid füüsilised raamatud, suuremad fännid siin, Jack\index[ppl]{Lippmaa, 
Jaak}\sidenote{Ilmselt peab Veiko silmas Jaak Lippmaad ja mitte Jaak Loondet, 
keda sama nime all tunti.} ja nii edasi, tõlkisid neid Eesti keelde,  isegi 
avaldati. Ja eks olid ingliskeelsed suured raamatuarhiivid. Ei olnud ju midagi 
eriti saada, just ulmet ja ilukirjandust. Teadusraamatukogud  ostsid rohkem 
teaduskirjandust,  ilukirjandust oli ikka väga vähe, ja need hakkasid liikuma 
juba digitaalsel kujul. Digitaalsetest arhiividest sai raamatuid tõmmatud, mul 
endalgi BBS-is kõik, mis kätte tuli, läks sinna üles ja rahvas käis ja sai neid 
sikutada ja hoolega lugeda. 

\question{Ja lugeja vaim sai valgemaks! Mis rahvas sul seal BBS-is käis? Mingi 
aimdus sul ju oli, tudengid või\ldots?}

Olid õpilastest ja tudengitest kuni paremate arvuti-inimesteni välja. Sest meil 
oli seal väga palju igasugust põnevat tarkvara, põnevat arvutialast kirjandust 
ja materjale, mida kuskil ka eriti ei liikunud. 

Ja kuna Printas juhtisin ka üht tuba, olin moderaatoriks,  siis saime nende 
\emph{underground}-iga  tuttavaks, IC piraadigrupi\sidenote{Siin peab Veiko ise 
seletama, mis tolle \emph{warez} grupi nimi täpselt oli. Suuremate gruppide 
nimekirjast sarnase nimega seltskonda leida ei õnnestunud.} liige sai oldud, 
seal liikus vahvat materjali. Oli täitsa kurioosseid olukordasid. Tollel ajal 
ei olnud ju mingeid päris pira FTP-sid. Tehti seda nii, et kui mingi asi algas 
punktiga, siis see oli nähtamatu. Ja kuskil, kus oli firmal FTP server püsti, 
siis mingisugusesse huina-muina kataloogi, kus on mingid süsteemsed asjad ja 
kuhu tavaline inimene ei lähe, tehti punktiga algavaid katalooge. Kui seda 
hakati avastama, tulid kasutusele igasugused muud asjad, näiteks mingid 
kontrollsümbolid. Sümbol, mis tegi näiteks piiksu või mis tegi reavahetuse. 
Seesama \emph{enter}-i vajutamine, et sa sisestad asja, oli ka võimalik 
kontrollsümbolina kirja panna. Ja kui seal see sümbol oli ees, siis sa pidid 
teadma, et sa sinna ette panid just selle kontrollsümboli, vist oli CTRL+L, mis 
käskis lugeda järgnevaid asju kui lihtsalt tekstistringe.

\question{Ehk te munesite kellelegi FTP serverisse  oma piravara?}
                 
Jah, niiviisi oli terve maailm täis! Katoloogide kaupa oli kõikvõimalikke asju.
                 
\question{Suurde rahvusvahelisse piragruppi ligi saamine oli ju seotud, ütleme, 
raskustega. Päris avasüli ei võetud vastu?}

Suurte raskustega! Praeguses torrenti-ajastus\ldots

\question{Aga kuidas see sul õnnestus?}

Selle Printa BBS-i kaudu tulid, kutsusid.

Igasugu põnevaid asju sai uurida ja vaadata. Vaata, ega sa ei oska ju midagi 
soovitada, kui sa ei ole ise seda näppida saanud.

Vene aja lõpus müüdi igal pool mustal turul ja laatadel piraat-kogumikke. 
Igasugused Gamez ja nii edasi. Aga et sa piraat-tarkvaraga raha teenid, loeti 
tõsisemates gruppides väga halvaks märgiks, selle eest said kohe kangiga vasta 
pead. See oli patt. 

Kui tuli välja Windows 95, oli see alguses kohutavalt suur saladus. No nüüd on 
Microsoft ennast täiesti teisipidi pööranud. Tahad, võtke uusi versioone, 
uurige, tutvuge, vaadake! Nad on lõpuks aru saanud, et see, et sa midagi püüad 
kinni hoida, see ei takista. Aga kui sa tahad endale miljonit kuju, kes uut 
toorest tarkvara katsetab ja kakub oma juukseid, mis lähevad tänu sellele 
halliks, et kõik hunnikusse lendab? Sealt tuleb tagasiside! Kõik sõimavad 
\enquote{Parandage see ära, see on valesti!} Sa ei jõua endale nii palju 
töötajaid otsida, kes  kõik selle debugimise töö nii põhjalikult ära teevaed, 
kui see vabatahtlik jõuk. Aga siis oli see, jah, nii keelatud, et kui tuli 
Windows 95, tollal koodnimega \enquote{Chicago}, siis ta oli muidugi olemas, ja 
poisid tegid igavese pulli, nad panid selle Microsofti peamisse FTP serverisse. 
Tegid seal punktidega kataloogid, ja järgmine päev panid sinna Chicago üles. Ja 
kui see üle  maailma kulutulena levis, \enquote{Microsoftist saab pira Windowsi 
tõmmata, 95t!}, oi kuidas siis Microsoft marutas!. Ja kõik, kes seal käinud 
olid (mina käisin lihtsalt vaatamas ja irvitamas,  et \enquote{vaadake, mis 
seal seisab}), nad olid ära loginud. Muu hulgas ka ühe  aadressi, mille kaudu 
mina käisin. Mul oli neid palju, üle maailma, sest üks modem oli kinni, teine 
kinni, eks ikka leidus mingi auk. Tuligi teade, et \enquote{karistage, võtke 
\emph{account} ära, see on igavene vastik piraat, käis piilus meie juures 
Windowsi!}. Kuigi ma ei tõmmanud teda, ammu olemas! Sysadmin tuli minu juurde, 
et \enquote{Windowsi vaatasid, sellistele asjadele saate ligi? Kuule, aga mul 
on üks küsimus. Meil Soome kolleegid näitasid SPSS\sidenote{Algselt IBM-i poolt 
välja töötatud tarkvarapakett, mille nimi on lühend väljendist 
\emph{Statistical Package for the Social Sciences}.}  5.0-i, jube kallis, neil 
on ainult pooled moodulid ostetud. Kas seda oleks kuskilt saada?} Küsisin ICE 
põhimeeste käest. Sealt tuleb vastu, et \enquote{Äh, miks sa viit tahad? 
Poolteist kuud tagasi tuli kuus välja!} Ja oi kuidas siis meie onud-teadlased 
olid rõõmsad, et kui Soomlased tulid vaatama, et Eestis on täis pakett SPSS 
6.0-i, mida Soomes ei ole mitte kellelgi! 

\question{Ühel hetkel toimus ikkagi nihe, mingi hetk hakati tarkvara eest ju 
maksma?}
Oma riik tuli, kõik asju sai hakata ostma ja hinnad ka normaliseerusid. Vene 
aja lõpul olid need hinnad ju\ldots No kujuta ette, kahekümnemegase kõvaketta 
eest maksad sa 45 000 rubla! Mäletan, Eesti kõige esimene 486 läks Punase 
RET-i\sidenote{Asutatud 1935. aastal OÜ Raadio-Elektrotehnika Tehas nime all. 
Tehas tegutses 1993. aastani ja tootis raadioid ning mitmesugust 
audiotehnikat.} spets konstrueerimisbüroole. See toodi mingisuguse, ma ei tea, 
mis kuradi värgiga (tol ajal Ukraina, Valgevened ja teised hakkasid ka 
eralduma)  peidetud transpordiga Minskisse. 486-d olid ju totaalse embargo all, 
CoCom\sidenote{\emph{Coordinating Committee for Multilateral Export Controls 
(CoCom)} oli mitteformaalne multilateraalne organisatsioon, mille abil USA ja 
tema liitlased üritasid koordineerida erinevaid kommunistlikke riikide suhtes 
strateegilistele kaupadele rakendatud piiranguid.} ei lubanud neid sotsmaadesse 
viia. Ainult 386 kõige lahjemad versioonid olid need, mida juba võis ametlikult 
tuua. Ja selle masina hind oli neli pool miljonit rubla.

\question{Hoomamatu number toona rublades, isegi täna Eurodes!}

Tehti seesamanegi piir, majanduspiir, Narva jõe peale. Sõidame sinna, vastik 
ilm oli. Vahepeal oli see reegel, et ainult juht tohtis läbi sõita, teised 
pidid minema jalgsi läbi putka. Ei viitsinud minna. Olime kahekesi, 
mikrobussiga. Tuleb siis sõdurpoiss, lööb kulpi, \enquote{miks te kahekesi 
olete?} Ütlesin, et, saadan kaupa, seda ei tohi üksi viia. \enquote{Mis kaupa? 
Tehke lahti!} Kolm suurt matkajate seljakotti, näha, et mingid nurgelised 
klotsid on sees. \enquote{Mis te veate?} \enquote{Raha!} \enquote{Mida?} Teeb 
koti lahti, seal on sajaste klotsid, sada rahatähte oli pakk, mis panderolliga 
ümber tõmmati. Kümme sellist pakki oli üks \enquote{tellis}. \enquote{Palju 
siin on?} \enquote{Kuskil ligi viis miljonit\ldots} \enquote{Vabandust!} Edasi 
ei huvitanud, tema jaoks oli see ka hoomamatu. 

Aga meil olid niivõrd head sidemed vene poistega. Venemaaga äri ajades sa pead 
teadma, kuidas ja mismoodi on. Meil oli usaldus nii suur, et teinekord läksid,  
täpselt ei teagi, palju sul on. Võtad kaasa, valid selle ja teise arvuti ja 
jääb näiteks ütleme poolteist miljonit üle. Jätsime rahulikult tema juurde 
seifi. See edasi-tagasi sõidutamine oli kõige riskantsem, kui sul võis saba 
peale lennata, sind maha võtta, ära tulistada auto ja kõik, eks. Hiljem 
helistab \enquote{Kuule, meil tuli selline väga huvitav asi väga hea hinnaga. 
Huvitab?} \enquote{Huvitab!} \enquote{Okei, ma tõstan selle raha siis endale} 
Ja samamoodi, teinekord lähed sinna kahte masinat tooma ja ütleb \enquote{Tead, 
mul õnnestus kolm tükki saada. Tahad? No võta kaasa, järgmine kord tood raha 
ära!} 

\question{Kui ma sind kuulan, siis see ei ole mitte kümned ja sajad ja 
konteinerid, vaid kaks-kolm masinat?}

Jah. Suuremaid tehinguid oli vähe. Meil olid kontaktid arvutite juurde, mõnedel 
teistel meestel olid kontaktid raha üle. Olid meil näiteks niisugused 
sõbralikud suhted EVEA Panga\index{EVEA Pank} mitmete tegelinskitega, kes 
ajasid meile ikka tõsiseid ärisid välja. Tõime suure-pika reisi-Ikarus 
bussitäie arvuteid Poolast, näiteks. Sai need kõik viidud Peterburi, seal 
laaditud sõjaväe transport kopterite peale. Kõik on kuulipildujate ja värkide 
all, relvastatud eriväelased ümber. No aga see tehing oli ka seal, ma ei tea, 
kui palju miljoneid seal kokku läks. Kopterid sõitsid põrr-põrr-põrr, kogu 
Kuibõševi linnavalitsuse arvutipark tuli siitkaudu. Igaüks sai oma sellest! 
Tulevad poisid, istuvad, ajavad juttu, väga head konjakid kaasas, saab joodud, 
hakkavad ära minema. \enquote{Oot, kohvri unustasid maha} Ah, jah, miljon 
sularahas kohvriga näpu otsas kaasas\ldots See oli väga, ütleme, selline 
kauboikapitalismi aeg. 

\question{Kas õnneks või kahjuks see aeg ei kestnud väga kaua}

Jah, kuidas kellelegi. Mõned lõpetasid kuuliga kuklas, teised lõpetasid 
praeguste tippmiljardäride hulgas. Kes kuda mida jõudis kinni võtta.

\question{Tuleme tagasi Luciferi\index{Lucifer BBS} juurde. Ta oli Eestis 
ikkagi üks hetk kõige populaarsem koht, kus käidi. Vähemalt nii on räägitud ja 
endalgi on meeles. Mis ta nii populaarseks tegi?}

Info hulk, mis seal oli. Sest kui BBS-indus veel õitses, siis internetile, kust 
sai neid materjale sikutada, oli ligipääsu väga vähestel. Enamik interneti 
kasutajaid olid tõsised töötegijaid, kes tegid tõsist tööd, eksole. Oma 
kolleegidega seal vahetasid emaile ja \emph{that's it}. Nemad ju ei kaevanud 
ringi mingisuguseid suuri raamatute ja igasugu failide ladusid pidi. Nad ei 
toppinud oma nina igale poole ja tänu sellele kuskile mujale ei jõudnudki. Ja 
kellel olid BBS-id, ei olnud jälle sellist finantsvõimekust, et sikutada kogu 
seda materjali lihtsalt BBS-i kaudu ühest teise, see oli väga kallis.
                 
\question{Ja sinu juures said kokku huvi ja aru saam nende asjade väärtusest ja 
finantsvõimekus?}

Jah.

\question{Ja seetõttu sinu \emph{stash} oli populaarne, sinna oli popp külge 
tulla?}

No sealt igaüks leidis midagi põnevat! Seal oli kõike, alates igasugu maagiast 
ja okultismist ja satanimist üle arvutikirjanduse, ulme ja \emph{science 
fictioni} kokaraamatute ja retseptikogumikeni välja. Kõike.
                 
\question{Räägime korra nendest online-mängudest, mida sa mainisid. Mis aastal 
sul esimene mäng tuli?}
                 
Mina sattusin sinna kuskil kahetuhandendal. Enne ma suur mängur ei olnud, 
mõningaid üksikuid mänge  sai toksitud, aga põhimõtteliselt oli internet see 
maailm, kus ma ringi kolasin. Ja huvitaval kombel muda\index{Muda}, see 
mitte-graafiline \emph{dungeon}, jäi kõrvale. Muidugi sai kõvasti mängitud 
\emph{Dungeons \& Dragons}-it\sidenote{Dungeons \& Dragons, tihti lühendatud ka 
DnD või D\&D, on 1974. aastal esmakordselt ilmavalgust näinud rolli-lauamäng. 
Mäng oli esimene omataoline võimaldades suhteliselt vaba vooluga kuid siiski 
kindla struktuuriga mängu-karakterite ja lugude arendust. Pikad mängukampaaniad 
võivad kesta aastaid.}. Sinna vedas mind Jaanus 
Lillenberg\index[ppl]{Lillenberg, Jaanus} ta on nüüd ERR-is. Tema oli mul 
esimene DM\sidenote{\emph{Dungeon Master} on Dungeons \& Dragons mängu kohtunik 
ja jutustaja, kes täidab ka loo mitte-mängijatest tegelaste rolle. DM 
kontrollib ja organiseerib kogu mängu, temast sõltub mängukogemuse kvaliteet.}, 
see oli vist 1994 kui ma sinna mängu sattusin. Siiamaani saab seda mängitud. 
Käime siin vaikselt ja toksima korra nädalas täringuid ringi.

\question{Väga põnev, sest mina olin ka sel ajal Tartus aga minu jalg tolle 
maailma peale küll ei sattunud?}

No seda oli vähe. Põhimõtteliselt  tõid ta Tartusse sellised mehed nagu Arlis 
Narusberg\index[ppl]{Narusberg, Arlis} ja Uuk\sidenote{Ei ole selge, keda Veiko 
silmas peab}. Võiks öelda, et DnD üheks kõige esimeseks maaletoojaks on, vana 
hea tuttav Vormsi Enn\index[ppl]{Vormsi Enn|see{Mikker, 
Enn}}\index[ppl]{Mikker, Enn}. Ta sai neil segastel lõpu-aastatel Soome sõita. 
Tema käest telliti, et too ikka mõni mäng, arvutimäng. Tema tuttavad, vanemad 
inimesed, ega nemad ka täpselt ei teadnud, läksid poodi ja ostsid talle teise 
\emph{edition}-i Dungeons \& Dragonsit, \emph{dungeon master}-i raamatud, 
\emph{players handbook}-id ja kõik. Algul oli pettumus, polnudki nagu kuskile 
arvutisse panna seda asja, aga kui süvenesid, oli väga kõva. See oli 
Kunstiinstituudi\index{Eesti NSV Riiklik Kunstiinstituut} punt, kes seda 
mängis. Meelis Mikker\index[ppl]{Mikker, Meelis} näiteks oli väga kõva DM. Ja 
kui sealt lõpetanutest üks ports Tartusse kolis,  tuli nendega koos ka DnD ja 
seda sai ikka  mängitud. Vahepeal üheksakümnendate keskpaik oligi mul selline 
hullumeelne aeg, kus ise mängisid näiteks kahes mängus ja tegid ise kolme-nelja 
mängu. Nii et terve nädal otsa iga päev oli mingisugune seltskond.

\question{Kui sa \enquote{tegid mängu}, siis sa olid DM?}

Jah.
                 
\question{See tahab ju fantaasiat saada, ei ole niisama!}

Aga selleks interneti maailm ja ulmekirjandus ongi, et fantaasiat arendada!

\question{Ja fantaasiat sul on?}

Võiks öelda, et jagub. Siiamaani käivad ja painavad. Üks seltskond, 
filmi-inimesed, tahavad, et ma ingliskeelset mängu hakkaks tegema. On kõik 
põnevil, aga küsimus ongi DM-ide vähesus, kes viitsiks ja oskaks teha. Ma olen 
ka mõned korrad sattunud sellisesse mängu mängima, kus käib asi niiviisi, et 
\enquote{Lähete nädal aega, midagi ei juhtu. Nüüd tuleb kari lendavaid lõvisid, 
hakake lööma!}. Kõik veeretavad täringut, kolm tundi täring klõbiseb, kõik 
kaklevad, lõvid on surnud, \enquote{nüüd lähete veel kuu aega, midagi juhtu}.  
Sisulist mängu nagu ei olegi. 

Kui sa võtad kätte ja lased rahval mängida, tõmbad neile konkse ja igasugu asju 
üles\ldots Ühes mängus oli, kus kõik mängisid nii hästi oma osa. Loomulikult 
kõiki aeti asju nurga taga, et ülejäänud rahvas ei kuule. Tegelikult sellest 
moodulist, mida pidi me  liikusime, ei liigutud ühtegi sammu, kogu asi käis 
omavahel. Keegi ei tea, kõik kahtlustasid, et see või teine on mingisugune kuri 
koll. Mäletan, kuidas Mario Pizzolanti\index[ppl]{Pizzolanti, 
Mario}\sidenote{Tartus tuntud kuju, pidas legendaarset baari 
Zavood\index{Zavood}.}  kellegagi  igavesi lahingud lõid. Ja kui pärast 
sessiooni kokku tõmbasime ja asjad avalikuks tulid, said teise keretäie veel: 
\enquote{Mina arvasin seda!} \enquote{Aga mina arvasin nii!} Ja vahel kõik 
naersid nii, et püksid märjad. 

Siis tuligi seesamunegi EverQuest\index{EverQuest}, seesama \emph{dungeon}, aga 
arvutimaailmas. Et sa ei pea ise olema DM vaid arvuti teeb selle sinu eest ära. 
Ei olnud enam aega DnD-d teha ega midagi, aeg läks kõik sinna. Parimatel 
aegadel ma olin  EverQuest I-s maailma seitsmes \emph{warrior}. Eestist tuli 
neid veelgi, üks sõber on \emph{ranger}-ina veelgi kõrgemale tõusnud. Minul oli 
keskmine mänguaeg ööpäeva kohta neli ja pool tundi, temal oli kuus ja pool. 

\question{Oh jumal, see on ju investeering!}

Jah, ta sel ajal oligi.
                 
\question{Kui ma sind kuulan, siis kerkivad inimesed kuidagi esile. Sa tundud 
nendega hakkama saavat, neist aru saavat?}

Jah. Praegugi juhin gilde. 

\question{Selles mõttes ka, et ega Venemaal nende tõsiste inimestega jutu peale 
saada ei ole lihtne. See tahab pealehakkamist ja suhtlemise oskust?}

Nagu öeldakse, sa pead teadma \begin{russian}русская душа\end{russian}-d, vene 
hinge. 
See on hoopis teine, kui sa seda ei mõista\ldots Jätame poliitika kui sellise 
kõrvale. Aga  praeguse aja noortel ja lääne inimestel äri ajamisel ongi see, et 
nad ei saa aru. Et kui tema ütleb hinna ja selle peale öeldakse \enquote{Ahah, 
et selle tehingu väärtus on meil 20 miljonit? Aga teeme nii, et on 15 
miljonit!} Ei saa aru! Kuidas? Mis? Miks? \enquote{No me anname ühe Šveitsi 
panga arve, kuhu kolm miljonit panna.}
                 
\question{Sa saad ju sellest  eestlase hingest ka aru, sa saad aru, mida 
inimesed vajavad ja mis neid huvitab, kasvõi Luciferi püsti panekuga}

Jah, kindlasti.

\question{Kust see sul tuleb, oled lihtsalt sündinud sellega? Oled sa mõelnud?}

Võib-olla on see oskus. Ma ei oska öelda, ei ole nii palju analüüsinud. Aga 
nii-öelda juutimise asjaga ma olen tegelenud palju. Kui muu rahvas rüüpas EÜE-s 
oma elu, siis mul kõik suved ja talved olid talvel suusamatkadel ja suvel 
mägimatkadel alpilaagrites. Ma olen palju igasugu matkagruppe juhtinud.

Seal ongi see, et kuidas seda asja teha. Arvutifirmasid, erinevad, on saanud 
juhitud ja\ldots Ma mäletan, kui  sattus kätte raamat \enquote{Kuidas võita 
sõpru ja mõjutada inimesi}, Carnegie oma\sidenote{Carnegie, Dale. How to Win 
Friends and Influence People. Simon \& Schuster, 1936.}, siis paljusid asju 
sealt ma olen kuidagi instinktiivselt teinud. \enquote{Ma tahan sind midagi 
tegema panna}, isegi kui see on kasulik, tekitab trotsi. Vastumeelsust. Et kes 
sa selline oled? Kui sa tahad, et inimene midagi teeks, kujunda selline 
olukord, et inimene ise tahab niiviisi mõelda. Vot see on meie poliitikute üks 
suur puudus ka, et kõik tahavad kedagi juhtida, sundida. Öelda, et sa oled väga 
loll, sa mõtled valesti. Anna parem talle võimalus, et mina olen kuidagi rumal. 
Lase tal minu arvamust ümber pöörata selle arvamuse peale, mida ma tahan, et ta 
tegelikult teeks ja mõtleks. Ja hoopis rohkem tuleks  saavutusi. 

Nagu öeldakse, palju on inimesi, kes tahavad, et  keegi midagi ära otsustaks, 
keegi midagi ära teeks. Nad ei ole huvitatud sellest, et nad peavad oma peaga 
mõtlema ja, mis veel hullem, selle mõtlemise tagajärjel tehtud tegude eest 
vastutama. Jõle hea on näidata, et valitsus on loll, minister on loll, euroliit 
on loll, onu trump on loll, jumal taevas on ka loll. Ainult mitte mina!
                 
\question{Sellest järeldame, et sina oled ka loll?}

Loomulikult! See võtab päris kaua aega, enne kui võiks hakata vana kreeklase 
kombel ütlema, et ma tean, et ma midagi ei tea. 

Kasvõi kõik needsamad jumala teemad, alguse ja lõpu teemad. Teadus on jube 
võimsalt edasi. Kõik need kvandid ja mustad augud ja. Aga mis edasi, kuidas 
edasi? Kust see kõik tuli? Suur pauk? Kes paugu tegi? Mis enne suurt pauku oli? 
Ütleme niiviisi, et kui enne ei olnud midagi ja  nüüd korraga tuli maailm, siis 
see ongi nagu maailma loomine. Selles mõttes jumala mõiste, kui me ei hakka 
siin  mõtlema mingit halli habemega taati, kes karjasekepp käes pilve peal 
jalgu kõigutab, võib võtta loodus seaduste, loodusteaduste, looduse enda 
kompleksina. 

Kui sa oled juhuslikult lugenud Ijon Tichy kosmoselendude 
päevikuid\sidenote{Lem, Stanisław. Loomingu Raamatukogu 1962 Nr. 22. Ijon Tichy 
kosmoserändude päevikud. Ajalehtede-Ajakirjade Kirjastus, 1962}? Mäletad seda, 
kui ta  äikselise ilmaga maja ukse taga koputas, sisse ei tahetud lasta ja kui 
lõpuks lasti, siis hullunud teadlane näitas talle oma ülakorrusel neid 
plaadikaste, kus loeti, et \enquote{see on noor neiu ja see on keegi teine}. 
Aga äkki me olen ise ka plaadimängijad kellegi tolmunud pööningul? Ei tea! Need 
deja vu efektid ja  parapsühholoogia. Minu arust Ijon Tichy väga ilusti 
illustreeris selle ära. Aga ma ei tea , me ei saa seda kontrollida! 

\question{Sind huvitavad sihukesed asjad?}

Aga loomulikult! Kõik räägivad, et kes tõestab jumala olemasolu, kes tõestab 
selle mitte-olemasolu. Mina olen võrrelnud seda sellega, kui meil varbaküüne 
üks rakk hakkaks tõestama inimese olemasolu või mitte-olemasolu. Kuidas ta seda 
teeb? Peremees võib käärid kätte võtta ja küüned lühemaks lõigata\ldots

Meie orgaanilise keemia professor Viktor Palm\index[ppl]{Palm, Viktor} , kui ta 
luges  teadusliku maailmavaatele aluseid, siis ta ütles tol ajal väga julgelt,  
1981. aastal ikkagi, et tema arust on näiteks teaduslik kommunism ja teaduslik 
ateism täpselt samasugused pseudoteadused nagu teaduslik teism või teaduslik 
jumala-õpetus. Nendel asjadel pole teadusega midagi pistmist!
                 
\question{81. aastal öeldi auditooriumi ees selline asi välja?}

Jah. Nad olla ikka selle eest vasu pead ka saanud, sest usinad tegelased käisid 
ikka, käsi kõrva ääres, raporteerimas. Aga miks nimetada mingit asja 
teaduslikuks, kui seal ei ole teadusega mitte mingit pistmist?
                 
\question{No oli ju vaja kuidagi nimetada, et uhkem oleks!}

No eks  praegu on ka, kui me vaatame, igasugused majandusteadused ja nii edasi. 
Jube palju on seal soolapuhumist! Üks mees võtab need meetodid, tõestab ühe 
asja ära, ütleb \enquote{must}, teine ütleb \enquote{ei, ei, ei} ja tõestab 
ära, et kõik on valge. Kolmas räägib pallist ja neljas räägib üldse kokku 
sulanud spektrist. No võta siis kinni, mis on! Täpselt see, kellele mida vaja.
                 
\question{Meil hakkab tasapisi aeg otsa saama, sellepärast küsin selle kohta, 
mis sa praegu teed. Sa teed palju kirjatööd, kuidas sa selle juurde jõudsid? 
Üks asi on palju lugeda, teine asi on  palju kirjutada.}

No miski aeg tagasi, kahe tuhandete alguses, töötas mu naine sellise ajakirja 
nagu Arvutimaailm\index{Arvutimaailm} peatoimetajana. Oli selline tore aeg, kui 
ka IT-ajakirjanikke  mööda maailma lohistati mitte nagu nüüd, kus  keegi meie 
vastu huvi ei tunne ja veetakse ainult autoajakirjanikke, see teeb kohe suisa 
kadedaks. HP-l oli parasjagu mingisugune järjekordne suur konverents tulemas ja 
Arvutimaailma kaastööline, kes pidi sinna minema, tema pass oli ära aegunud, ta 
ei saanud üle piiri. Ja abikaasa küsis, et kas sul on pass korras, sa tunned 
seda värki, teed ära? No mis seal ikka! Läksin sinna, tegin ära. HP ütles, et 
nii põhjaliku ülevaate, nii sisukat, nad pole näinud. Ma olin HP masinatega ka 
juba aastaid aastaid kokku puutunud. 

Ma  ei ole kunagi arvutiteadust õppinud. Aga vaata, kuidas praegu nooremad 
põlvkonnad on hädas, kasvõi DOS-i käsureaga. Kui ikka hiirega lohistatavat 
värvilist ekraani ees ei ole, on kaks käppa püsti. Aga ma olen selle kõigega 
üles kasvanud, nende arvutitega, mis mul endale läbi on käinud, järjest 
arenenud. Ja loomulikult, kui sa nendega tegeled, siis sul tekib huvi. Vastasel 
juhul ei ole vahet, kas sa müüd kartuleid, arvuteid või kaalikad, eks ole. 

Tegin selle HP loo ära, pärast tuldi veel, \enquote{oi kuule, tead, siin on 
jälle üks asi teha} ja tegelikult ongi nii, et väga palju nüüd ütleme nooremast 
põlvest, on neid, kes ei tunne seda raudvara. Raudvara-ajakirjandusega on see, 
et sa pead tegema, sa pead teadma, oskama küsida ja eks ma selle pärast jäin 
neile nagu silma ka. Kukuti igale poole saatma. Teine asi on see, et mul ei 
ole, nagu mainisid, inimestega läbi saamine probleem. Mind ei  kohuta  näiteks, 
kui me konverentsil saame kokku näiteks Inteli viitsepresidentidega. Või kui 
Otellini\sidenote{Paul S. Otellini oli Inteli CEO 2005-2013.}  oli veel see 
kõige suurem füürer, lähed juurde, pistad viis pihku ja ajad  juttu, küsid ta 
käest kõike. Ja kuna mitmetel üritustel sai käidud, siis paljud mehed, näiteks 
põhimine tehnikaohvitser, viitsepresident, juba eemalt tundsid ära,  tuli kohe 
juttu rääkima. Ütlesid, et \enquote{sa oled ainuke, kes asjast aru saab!}

No eks see oli ka üks asi, miks kirjutama kutsuti, miks hoiti orbiidis. Praegu 
on see asi ära vaibunud. Ma olen vist neli korda USAs käinud Singapuris ja ma 
ei tea, mitu korda Koreas, Hiinas, igal pool. Euroopast ei räägigi, seal 
vahepeal oli pidevalt üks konverents teise järel. Aga nüüd on  IT-firmad 
kuidagi nii maha vaibunud. 
                 
\question{Ehk vist saabunud  ka selles vallas nii-öelda pudukaupmeeste ajastu?}

Tihti oli ka see, et nad tulid turgu, sõid ennast sisse, kes saab vedu, kes 
teeb, kellest kirjutatakse, kõik olid väga põnevil sellest asjast. Aga eks nüüd 
on turu stabiliseerumine  käes, turg enam ei kasva. Ütleme, kasvõi 
lauaarvutitega. Ära nad ei kao, nagu paljud ennustasid, sellepärast, mängurid 
tahavad ikka suure 4K ekraani taga korralikult mängida. Siin on küll läpakas 
päris kõvad asjas sees ja kõik, aga ikkagi nii võimas ta ei ole, ta on alla 
\emph{clockitud} võrreldes sellega, mis lauaarvutis \emph{power}-it on. Neil on 
oma nišš olemas ja aga samal ajal niisugust huvi ei ole konkurentsi mõttes, 
nagu on autofirmadel.
                 
\question{Viimane küsimus. Mis oli viimane mäng, mis tõstis heas mõttes karvad 
püsti, et \enquote{see on äge asi!}?}

Kui sa mõtled seda mingit nime, mis on tulemas, siis selleks on 
Pantheon\sidenote{Pantheon: Rise of the Fallen on MMORPG, mille ilmumist on 
mitu korda edasi lükatud ning 2019. aasta lõpus suri juba varem mainitud 
EverQuesti kaasautor ning Pantheoni tootjafirma loovdirektor Brad McQuaid. 
Juunis 2022 on Pantheon eel-alfa staatuses, ajasime Veikoga juttu 2019. aasta 
algul.}. Pantheoni teeb praegu selline mees nagu Brad McQuaid, kes oli ka 
esimese EverQuesti\index{EverQuest} taga peamiseks ajuks. Oma paljude 
kaaslastega, kes dragonistid olid kunagi, kutsume seda mängurite kuldajastuks, 
mida paljud moodsad mängurid sõimavad. Aga meie ei mängi jälle neid kiireid 
piu-pau mänge, ma nimetan neid selja-aju-mängudeks. Ega seal muud pole vaja, 
võta ahv, õpeta kiiresti punast nuppu vajutama, mängib paremini. Ootan ikka 
sellist mängu, kus on tõesti strateegiat, kombinatoorikat, gruppide juhtimist, 
suured AI-d. Selliseid mänge tehti mingil teisel ajastul, seda aega me igatseme 
ja seda lubab Brad McQuaid. 

Mõned moodsad mängud, mis on tulnud, osasid on saadetud, osasid olen ostnud, 
mõned on \emph{free-to-play}, need on kõige jubedamad, kus raha eest võid 
endale elu osta. Hiljuti mängisin seda Fallout 76-te, parasjagu huumoriga, 
jälle  levelid on kõrgel, \emph{quest}-id kõik viimseni tehtud, midagi teha ei 
ole. Käi ringi, kogu sodi, et see sodi maha müüa. Kuu aega mängitud ja kõik. 
Aga kui meenutada viimast tõsiselt head mängu viimasest ajast, jätame 
EverQuest-id sinna kaugustesse, siis üks hea nimi oli Fallen Earth. Ka 
niisugune hea  tuumasõja ja kataklüsmide järgne maailm, millel oli, ma 
ütleksin, kõige parem graafiline ja mängijate vahelise äri süsteem. Muidugi 
vana klassika LOTRO, Lord of the Rings Online, Tolkieni fännidele, kelle hulka 
ma ka ennast loen. Ja, kindlasti Secret World. Väga kõva ja paljulubav nimi, 
aga jälle, nagu on, raha sai otsa ja teiselt poolt oli kahjuks suure fännibaasi 
taga ajamine. See on nagu \emph{modern horror}-i tüüpi,  kõik need Lovecraft'id 
ja Poe'd. Selline maailm, kus mingisugune must kaasaja maagia üritab  maailma 
tungida ja salaühingud võitlevad selle vastu ja ka omavahel. Templirüütlid, 
illuminaadid, dragon'id Hiinas\ldots  Aga ma ei ole näinud nii mõttekaid ja 
põhjalikke ja keerulisi \emph{quest}-e ühelgi mängul. Enamikel on 
\emph{quest}-ide, või noh ülesannete, värk muutunud nii labaseks: mine 
keldrisse, tapa 10 rotti. Nüüd, suur kangelane, maailma päästja, mine uuesti 
keldrisse, tapa 10 rotti ja too sabad ka ära! Või vii pakikene kõrvalkülla onu 
Juliusele. Andke andeks, kas veel lollim saab olla, aga nii ta on.

EverQuestis oli kohti, kus sa pidid näiteks võtma midagi kiilkirjas, sa pidid 
teadma  hieroglüüfe ja muidugi on selge, et keegi neid asju nii täpselt ei tea. 
Selleks oli mängu sisse ehitatud Google'i brauser, sa ei pidanud mängust välja 
minema, said sealt abi otsida. Näiteks oli üks koht, kus oli mingi vihje 
nimega. Leidsid laiba, mille juures oli saatmata postkaart sellele nimele. Ja 
kui hakkasid otsima tuli välja, et see oli üks Saksa kõvemaid krüptograafia 
alusepanijaid, keda väga vähe teatakse. Otsid siis välja: tal on olemas 
spetsiaalne algoritm. Kes tahtsid, võisid seda algoritmi käsitsi kasutada. Aga 
sa võisid  programmi alla tõmmata ja read sinna sisse kopeerida. Ja kui laisk 
olid, pildistasid ekraanilt ära, OCR-isid tekstiks, lasid teksti sinna 
programmi ja tagasi tuli juba mõtestatud tekst, kuhu sa minema pidid.

\chapter{Anto Veldre}
\index[ppl]{Veldre, Anto}

\question{Alustame asjade algusest, nagu ikka. Kuidas sina arvutite juurde said?}

Minu ema töötas ülikooli arvutuskeskuses\index{Tartu Ülikool!Arvutuskeskus}.

\question{Millise ülikooli?}

Tartu ülikooli. Eestis ei ole rohkem ülikoole,  ülejäänud on mingisugused \enquote{techid}. Ema oli ülikoolis, õppis matemaatikat ja 1959. aastast läks arvutuskeskusse. Mina ei tea,  oli ta seal juba põhikohaga juba tööl, või niisama katsetas. Igatahes tegeles ta Ural-1\index{Ural!Ural-1}  juures mingisuguse programmeerimisega. 

\question{Arvutuskeskus asus Liivi tänaval?}
Ei. See oli ülikooli, kõrval, ma arvan. Seal, kus praegu see biofüüsika on, kohviku majas.  Ma usun, et see oli 1959. aastal seal. Mina ei tea, mis nad mu isaga  vahepeal tegid,  abiellusid ja midagi veel, ja 1961. aasta augustis sündisin mina. Ja seda ma ka enam ei mäleta, kas siis oli Ural-1 või Ural-2. Minust arvuti taustal, esimene pilt, mida ma näinud olen, on 62. aasta jõulud. No mis jõulud, siis olid näärid. Ei, ma usun, see oli ikka Ural-1. Ühesõnaga, see oli siis mingi imelik aparaat, mille taustal mind näidati ja mida ma tegelikult ise ei mäleta, ma olen ainult pilti näinud. Edasi  seda arvutuskeskust koliti, kus ta seal igal pool ei olnud küll Gagarinis, mis iganesselle tänava nimi praegu on\sidenote{Praegu on tegu Jaan Tõnissoni tänavaga}, Burdenkos, mis on praegu Aia\sidenote{Siiski Veski.}. Erinevad osad olid laiali. Kui juba hakkasin teadlikult masinatest aru saama, siis oli arvutuskeskus seal, kus praegu on see punane korporatsioonihoone\sidenote{Eesti Üliõpilaste Seltsi maja aadressiga  Jaan Tõnissoni tänav 1.}. Seal majas oli Ural neljaks nimetatud masin\index{Ural!Ural-4}, aasta pidi siis olema mingisugune 73 või umbes nii\sidenote{Ürikute järgi kolis Tartu Ülikooli arvutuskeskus Liivi tänava hoonesse 1972. aastal.}. 

\question{Kas su esivanemad olid programmeerijad? Mis nad tegid selle Ural-iga?}

Ema oli matemaatik. Ja kogu see arvutuskeskus oli niisugune kahtlane koht. Seal oli mingeid kõrgema haridusega naisterahvaid, kes puistasid varrukast mingeid korrelatsioonimaatrikseid ja arvutasid  lehmade poniteeti ja ma ei tea, mis jubedusi veel. Ema oli jah, põhikohaga arvutuskeskuses tööl. Isa oli bioloog, ta muidu töötas zooloogiamuuseumis, aga  käis haltuurat tegemas. Oli hobi-programmeerija, profid olla teda vihanud, sellepärast et ta leidis mingisuguse lokaalse optimumi. Üks näide. Kõvaketta \emph{interleav}-ingut siis veel ei tuntud, aga oli magnettrummel, mille peal see Ural-4 oma mälu pidas. Ja vanamnees, kurat, arvutas välja selle trumli pöörlemiskiiruse ja hakkas oma progesid tegema niimoodi, et täpselt seni, kuni tema muid asja teeb, jõuab see neetud trummel sama koha peale tagasi. Ja kuigi tema programm nii-öelda struktuurilt ei kõlvanud kodulooma istmikku ka, siis töökiiruselt olid vist kas 13 korda kiirem, kui profiprogejate oma.

Nii et see minu lapsepõlv oli jah, niisugune huvitav koht. Isal oli kapis kaks ülikonda. Üks, natuke kehvem, oli Vanemuises käimiseks. Teatris käidi tollal ülikonnaga, mitte nagu praegu. Teine, natuke parema ülikond, oli öösel \enquote{Masinasse minekuks}, kusjuures Masin kirjutati suure tähega, See oli siis Ural-4\index{Ural!Ural-4}. Mind sinna öösiti ei lastud, aga  päeval ma seal ikka kooserdasin. 

\question{Millal sai siis hakkasid seal nagu teadlikumalt käima hakkasid? Põhikooli ajal juba?} 

Nojah, kui teadlik ta oli. Tead, kuidas öelda, Vene ajal oli ju see päritolu, eks ole. Kui sa oled see neetud intelligent ja sellest kihist pärit,  siis sa pead tegelema, ma ei tea, kõige millega. Ma käisin muusikakoolis, noorte trummarite ringis ja, oi jumal, mida kõike! Aga  arvutuskeskuses, ma arvan, et see oli 1973. Ma ei suuda enam väga täpselt meenutada, aga see pidi ilmselt mingi 73 lõpp olema. Jälle ma ajan seal kaks arvutuskeskuse\index{Tartu Ülikool!Arvutuskeskus} juhatajat segamini, üks tuli, teine läks, aga ma arvan, et siis oli veel Tapfer\sidenote{Jüri Tapfer\index[ppl]{Tapfer, Jüri}, oli Tartu Ülikooli Arvutuskeskuse juhatajaks aastatel 1971 –- 1995}. Igatahes kutsuti mind arvutuskeskuse juhataja, ülemuse,  mis ta iganes oli, kabinetti ja anti pidulikult kätte kasutajatunnus. Minu arust oli mingi viiekohaline number, ma ei mäleta. Ja sellel polnud masinaga midagi pistmist, see oli lihtsalt aruandluse jaoks. Pidin mingisse paksus žurnaali allkirja andma, nagu nõuka ajal ikka. Ja see siis tähendas seda, et kui masinas mingi vaba hetk oli, siis tehniliselt oli lubatud minu programmi sealt ka läbi jooksutada. Ega ma nüüd väga edukas ei olnud, selles mõttes, et mingi paar programmi oli, mis seal enam-vähem tööle hakkasid. Eks isa aitas natukene siluda seda asja. Vot seda sõna ka enam ei ole. 

Aga kuidas see siis välja nägi. 

Esimene liin oli see, et selle paganama Ural-i küljes oli niisugune ese nagu lai-trükkal. Minu arust oli tal 128 märki reas, igatahes kole palju ja sealt tuli päris koleda kiirusega. Ühe sõnaga, kiirkirjutusmasin. 

Ma usun, et ma olin palju väiksem, mingi nelja-viieaastane siis, kui ma seal ema või isa juures käisin, ma istusin selle lai-printeri peal. See kurinahk oli soe, saad aru. See oli sihuke kõrge koht, kuhu, poisikese värk, päris ise ei saanudki, keegi pidi aitama. Istud seal, kõlgutad jalgu ja vaatad, et mis operaator seal natuke eespool teeb. Aga siis selle printeriga seoses siis esimene liin oli, et tahtsin ka trükkida mingisuguseid, kas mingeid tabeleid, mingeid pilte. Nad seal tähtede nii-öelda  trükitihedusega kodeerisid  igasuguseid Mona Lisa-side ja pilte. Need Mona Lisad olid suhteliselt alasti, ma mäletan. Mind see ei häirinud aga ma mäletan, et neid hoiti nurga taga. Muidugi Leninist ja milest kõigest, ühesõnaga tehti. Ja mina üritasin ka mingit pilti teha ja loomulikult seal olid igasugused vead sees ja ma ei mäleta, kas ta lõpuni sai või ei saanud. 

\question{See ju tähendab, et sa pidid ikka kuskilt programmeerimist õppima. Või korjasid sa selle lihtsalt õhust üles?}

Asjalikke õpikuid ei olnud. Urali kohta üks mingisugune oli. Aga põhimõtteliselt ta ei olnud isegi assembler, puhas masinkood. Käsukoodid, null-üks oli liitmine, null-kaks oli lahutamine, äkki äkki oli niimoodi, ja siis sinna midagi järele. Aga seal olid muidugi trikid,  nagu Asmiski, ja nende selgeks saamine oli ainult läbi vaeva. 

Algaski siis sellest, et mingisuguse lihtsama joonise tegid valmis, siis ta ei olnud muidugi õige, siis sa pidid saama, kas perfokaardi või perforlindi peale. Kaks võimalikku sisendit. Millegipärast lihtsam oli perfokaardiga. Siis tuli kuskil tagaruumis mingit telegrafistitädi painata, kes Kesktelegraafist käisid nii-öelda lisatööd tegemas. Neile maksti raha, tagusid sisse ma ei tea mitu  märki sekundis. See üks või poolteist perfokaarti perforeeriti ära, seda ei olnud palju. Nojah, programm tuli alguses ja kirjutada  plankide peale, rohelise värviga trükitud plangid, seal mingid  operandid ja kommentaarid ja, oh jeesus,  panna sinna oma kasutajatunnus veel, ja ma ei tea, mida kõike. Ja siis programm sai perfokaardi peale ja ega siis ei olnud nii, et läksid masinal ligi. Masinasaali ukse juures oli nagu pioneerilaagris hambarja kast, sihuke lahterdatud. Mingi viis korda kuus või kaheksa korda kaheksa, kes seda enam mäletab. Tühja lahtrisse panid oma programmi ja kui masinal kas mõni mõni perifeeria seade ei töötand, tähtsamaid programme ei saanud teha või, kuidas öelda, operaatoril öösel igav oli, siis ta võttis need nalja-asjad ja lasi läbi kuni esimese veani. Siis kirjutas sinna perfokaardile jõleda käekirjaga mingi jõleda kommentaari ja võib-olla pani välja trükitud paberi ka sinna juurde. 

\question{Kogu selle vaeva läbimiseks pidi sul ju mingisugune põhjus olema?}

Mina ei tea. Miks mõned poisid jalgpallis käivad või kuskil? Nojah, ma ei usu, et seal ratsionaalseid selgitusi on, mina lihtsalt selle  asutuse seinte vahel üles kasvasin. Ja, no muidugi seal sai ka muid lollusi tehtud. Aia tänaval oli vahepeal üks ruum, kus oli mingi paarkümmend tädi Robotroni ja  Rheinmetalli  mehaaniliste arvutitega. Mulle meeldis neid jagamistehtiga kinni lasta, aga pärast muidugi tuli mehaanik kutsuda ja siis ma sain sõimata ja eks sellele midagi eelnes. 

\question{Kuidas sa mehaanilist asja kinni jooksutad?} 

Jagamistehe ei lõpe kunagi. Ta üritab selle kelguga kogu aeg edasi jagada, kuni kelk jookseb ühele poole kinni ära ja siis kas midagi läheb\ldots

\question{Jagamistehe ju lõpeb millalgi ära?}

Ei lõpe, nulliga jagamine näiteks ei lõpe kunagi. 

\question{Miks sa, kulla mees, panid mehanilise arvuti nulliga jagama!?}

Loll masin, ta saab sellega kinni jooksutada, põnev lihtsalt. Kes piinab kasse, kes  paneb bensiinitünni põlema ja kes laseb Rheinmetalli kokku.

\question{Rheinmentall ei kiunu, muidugi}  

Ei, ta ragises ja logises, see ei olnud ikka mingi ettenähtud olukord. Et seda nalja oli nii, et vähe ei olnud. Minu arvates. Tädide arvates muidugi mitte. 

\question{Kas mehaanik, kes kutsuti, hammustas läbi, mis juhtunud oli?}

Muidugi, ega ma sain sõimata selle eest, ega ma ainuke olnud,  neid lapsi oli teisigi. See oli teada asi.

Nojah, vot ja teine programm, mille ma unustasin rääkimata. Keegi inseneridest tekitas Uralile\index{Ural}  külge heligeneraatori. Pärast ma olen kuulnud, et see oli praktiliselt igal tolleaegsel arvutil, see oli mingi Covox Speech Thing-i\sidenote{Väline audioseade, mis võimaldas arvutil läbi paralleelpordi heli väljastada. Seade oli väga lihtne koosnedes hulgast takistitest, mis moodustasid primitiivse digitaal-analoogmuunduri. Selliseid jootsid koolipoisid üheksakümnendatel ise kokku ja pusisid neile ka sobilikud draiverid. Mäletan, et suursaavutuseks oli Nethacki\index{Nethack} paaritamine Warcraft II heliklippidega. Tekstipõhise mängu ekraanipuhvrist loeti kindlast kohast tekst, sõnadele olid vastavusse pandud helifailid ja need mängiti Covoxi abil maha.} eellane, põhimõtteliselt. Andsid arvutile lolle käskusid mõttetute argumentidega, käsukood loeti välja ja kui ta 01 näiteks oli, tehti mingit madalat häält, null kaks oli  juba natuke kõrgem hääl ja siis sai niimoodi laulukesi teha. Ja kuna ma muusikakoolis käisin ka, siis ma üritasin. Aga minu mälu järgi see ei saanud ka kunagi valmis, alati oli mingi viga sees. 

\question{Midagi ta ju ometigi tegi, mingit piiksu ju sai?}

Muidugi, muidugi. Lihtsalt mingi noot oli jälle vale. Ega see ei olnud siis mingisugune Sibelius või Cuebase või Fruity Loops, eks ole, millega sa kohe kuuled! Oi ei! Sa pidid veel kuulama ka minema. Pidid operaatoriga kokku leppima, et sa lähed ja kuulad. Et vot selline raske elu oli lapsel, kes arvutuskeskusse oli sündinud. 

\question{Kui vana sa olid, kui sa neid programme tegid?}

No kindlasti see lõppes otsa 12-13, praegu viitsi ei viitsi arvutada, maksimum 14. Aga minu arust, kui ma 13 olin, sai Liivi\index{Tartu Ülikool!Liivi õppehoone} tänava arvutuskeskuss lõpuks valmis. Alguses uued masinad koliti üle, pärast see Ural\index{Ural} visati ju üldse välja. Nii et igatahes 1975. aastaks oli see kõik pidulikult läbi. 

\question{Mis tollest Uralist sai? Lihtsalt utiili?}

Paraku jah. 

Kui mina veel Tartus elasin, 76. aastani\ldots Algul Ural-1\index{Ural!Ural-1} läks Nõosse\sidenote{1965. aastal, sellest sai alguse Nõo Keskkooli\index{Nõo Keskkool} arvutiõpe.}, siis läks Ural-2\index{Ural!Ural-2} Nõosse. Ural-2 blokke vedelas veel Tartu vahel,  kui Ural-4\index{Ural!Ural-4} töötas. Jõe ääres, selle keskmise silla juures,  oli kunagi mingi füüsikamaja. Sealt onude käest sai kaubelda veel mõningaid. Nii et ma olen neid trigeri blokke ka näinud, mis nad olid, 6N9S või 6N8S lambi peal. 

\question{Nojah, toonane arvuti oli väga modulaarne asi.}

Jah. Aga neid praktiliselt ei ole järel, nii et see on kõik läinud metalliks. 

\question{Kuskil Venemaal kindlasti midagi on!}

Ma usun, et Nõos ka mõnel õpetajal kuskil tagatoas, on äkki on üks triger alles. 

\question{Mis sa edasi tegid, sebisid end Liivi tänavale?}

Ei. Vaata, mina olin juba siis kuulus isemõtleja, aga Nõukogude Liidus isemõtlemist ei sallitud. Nii et selle etapi võib ka põhimõtteliselt arvutite koha pealt vahele jätta. Nõuka ajal lihtsalt elasin nagu suutsin, lihtne ei olnud, ülikooli ei lastud, mõned muud probleemid veel. Aga nõukaaeg sai raha teenitud igasuguste aparaatide parandamisega. Põhimõtteliselt ma olin 16, kui tulin, tulin Tallinnasse. Läksime alguses raadiotehnikat õppima polütehnikumi\index{Tallinna Polütehnikum}. Ja sealt tuli kirg tinutus kolvi vastu, nii et vahepeal väga pikalt ei olnud mingisuguseid arvuteid kuskil. Lihtsalt sai ennast elektroonikaga lõbustatud. Mingitel segastel aegadel see muidugi tagas äraelamise,  kõik meenutavad, kui raske oli mingisugustel murdehetkedel, kui poes ei olnud midagi. Ja ma ütlen, et ma ei mäleta seda hetke selles mõttes, et mul lihtsalt härrased olid vorstiga ukse taga sellepärast et hommikul kell kuus pidi algama mingi jalgpalli MM kuskil ja telekas pidi mängima sel kellaajal. 

\question{See tähendab, et sa pidid kuskilt teadmised üles korjama?} 

See oli veel üks hobi muidugi, see oli veel üks hobi. Selle Ural-4\index{Ural!Ural-4} taga oli ju ka toatäis insenere, mingisugune tüki viis, kes seal istusid ja eks ostsillograaf oli kogu aeg arvutil ligi. Ei mina oska öelda, kust ma selle täpselt selle üles korjasin. Kusagilt sealt. 

\question{See on huvitav, et seda räägivad paljud, et sihukest süsteemset õpet on vähe olnud, aga kuskilt teadmine tuli?}

Nõuka ajal süsteemne algas ju sellest, et sa pidid olema kodumaale lojaalne ja igatpidi väga standartne ja siilisoenguga ja siis sind võib-olla lasti kuhugi õppima ja siis sa lõpetasid kuskil mingisuguses salajase töö instituudis, eks ole. Jah, see oli see ametlik \emph{track}. Aga mitteametlik oli\ldots Mäletan kui ma Tallinnas olin poisikesest peast,  16 või mis ma olin, sai Küberneetika Instituudi maja\index{Kübereetika Instituut} just valmis. Samamoodi, arvutuskeskus läks alati esimesena, see oli kõige kallim,  siin samamoodi. Ja  ma mäletan, et ma käisin  mikroskeemide käsiraamatuid nuiamas. Läksid kuskilt tagauksest, \emph{social engineer}-isid ennast sisse Kevin Mitnicku\index[ppl]{Mitnick, Kevin} moodi ja siis seletasid, et kuidas sul on ikka tähtis konstruktsioon pooleli, aga ainult kahe mikroskeemi parameetrid on veel puudu. Siis said kätte selle salajase käsiraamatu nii-öelda kohapeal kasutamiseks. Nii see asi käis. 

\question{Küberneetika instituudis, olid olemas need raamatud?} 

Jah, see oli üks koht, kus neid sai. Selliseid kohti oli Tallinnas veel,  mingid sõjatehased ja asjad, aga jah, see asi käis niimoodi. 

\question{Sa oli põhimõtteliselt vabakutseline, niukene vaba mees?} 

Ei, siis ma ikka veel käisin tehnikumis, õppisin raadiotehnikat. 

\question{Mingi hetk jõudsid ikkagi moodsad arvutid ka sinu juurde, millal see oli?}

Sinna vahepeale jääb veel mingi segane aeg. Ma mäletan, ma üritasin mingeid vene arvuteid ka parandada, mingi Iskra-555\index{Iskra!Iskra-555}.  Pärast tehti Iskraid, mis olid Inteli 8086 kloonide peale tehtud, aga see oli mingi ise leiutatud, mingisuguse magnetkaardi pealt töötav raamatupidamisarvuti. Ma olen ka remontinud neid mehaanilisi suuri Robotroni raamatupidamisarvuteid. Sama moodi, nagu see Rheinmetall. Aga kui sa nii-öelda ametlik mehaanik ei ole, kui sul ei ole kogu dokumentatsiooni, on see õnnemäng. Ütleme nii, et enamasti ei õnnestu teha, tõenäosusega 20 prossa. Aga kuna neid mehaanikuid telliti ka Moskvast ja ma ei tea kust, siis aeg-ajalt lasti ligi. Mingi kogemuse sain, aga head mälestust ei ole. 

Vist 1989 või millal see oli, sattus vend CeBIT-ile mingitel asjaoludel. 

\question{89? See oli ju puha nõukogude aeg!}

Oli jah, nõukaaja lõpus, igatahes, mul need aastaarvud natuke ujuvad, äkki oli 90? Ei mäleta, kusagil seal kandis. Igatahes pani ta kõik oma elusäästud kokku ja tõi endale sealt portatiivse Taiwani läpaka, kahe flopiga. Isegi Turbo Pascal-it\index{Turbo Pascal} sai sellega käivitada, selleks oli vaja millegipärast kahte flopit. Ühe peale ei mahtunud ära. 

Ega must väga Pascali\index{Pascal} progejat polnud, jälle  tegin paar näidet  ja keegi teine silus ta mul ära ja andis tulemuse. Ütleme nii, et idee oli õige, aga  näpud olid lühikesed. Õppinud ikka ei olnud seda asja. Ja ega ma progemises ei ole kunagi mingi kõva käsi olnud. Aga jah, too masin käis kahe 720-se flopiga,  kolm pool tolli küll. Bondwell mingi Taiwani, valge sihuke, XT, noh. Seda said näppida, eks muidugi, vend ise ka näppis teda, ta tegi tööd sellega, aga siis aeg-ajalt sai näppida, imelikke asju teha. 

Ja järgmine koht oli\ldots No vot, ma ei teagi, mida nüüd järgmiseks lugeda. 

93. alguses suht, kui oli ikka selge, et nüüd on Eesti riik, et nüüd ei ole enam Vene riik (seal vahepeal oli segane periood), sattusin ma kuidagi õpetajaks tööle. See oli selline õnnetu juhus, kuidas ma nüüd ütlen, Tallinnas seal, kus kus on vana 43. kool\index{Tallinna 43. Keskkool} praegune Tehnikagümnaasium\index{Tallinna Tehnikagümnaasium}\label{sisu:43kool}. Tegelikult oli ühes majas kaks asutust, juriidilist, üks oli kool ja  üleval neljandal korrusel oli vana kadunud Ants Reili\index[ppl]{Reili, Ants} poolt tehtud, mis ta nimi oli, ETEK. Eesti mingi teaduslik-tehniline ettevalmistuskeskus, nagu see nõuka ajal käis. Aga Reili Ants oli  sihuke sihuke kihvt vanamees, see õpetas ju tööõpetust, tal olid telekas mingisugused tööõpetuse saated ja. Igatahes  tagus ta kuskilt välja raha, mingi eksperimentaalse raha, tegi neljandale korrusele selle nii-öelda keskuse ja saavutas selle, et kooliõpilastele hakati seal mingeid tehnilisi aineid andma. Tipi-koolist vanad elektroonikud võttis sinna tööle ja nii edasi ja nii edasi. Nii et see koht, et olid arvutid, mingisugused, vana Elektronika\index{Elektronika} klass ja mida kõike, see oli tal hästi püsti pandud. Tal olid isegi mehed, kes neid parandada oskasid, mis oli tollel ajal täiesti kriitiline. Aga need vanad mehed ei saanud lastega suurt hakkama. Ja mina olin siis see nii-öelda päästerõngas, kes siis pidi hakkama mingit tundi andma. 

\question{Kuidas sa sinna sattusid? Lihtsalt tutvuste kaudu?}

Ema töötas seal kunagi psühholoogina, see on keeruline lugu. Selle taga on tegelikult see stiil, mis  Keevallik\index{Andres Keevallik\index[ppl]{Keevallik, Andres}, Tallinna Tehnikaülikooli rektor aastatel 2000–2005 ja 2010–2015.} pärast tegi, et miks TPI-st nii palju välja langeb. Selle taga on ju see, et ega elama ju keegi ei õpeta, eks ole. Et poisid lähevad  kooli ju et saada oma eriala kätte, aga kuidas õlut korralikult juua ja õhtul klubis käia? Ja siis tuleb ju veel tööandja, kes võtab esimese kursa pealt juba ära ja kui kogu seda asja kokku miksida, siis kukuvad välja. Ja, vot, veel enne Keevallikut Ants Reili\index[ppl]{Reili, Ants} oli üks, kes sai sellest põhimõttest aru. Nii et põhimõtteliselt 43. kooli\index{Tallinna 43. Keskkool} lõpueksamitele ta saavutas staatuse, et need ühtlasi olid Tehnikaülikooli\index{Tallinna Tehnikaülikool} sisseastumiseksamid. 

Seal on veel pikk-pikk jutt. Need, kes tehnika eriala valivad ei vali seda mitte sellepärast, et nad lollid ja jobud on. Tark inimene läheb ju arstiks ja advokaadiks, sa tead küll, eks ole. Aga tegelik põhjus on verbaalne võimekus. Ehk siis, kui sa ei suuda seda seletada kiiresti ja korralikult, mida sa tahad, siis arvatakse, et ah, mingi pagana tehnika-nohik. Aga mida Reili tegi oligi see, et ta ajas sinna kooli kokku inimesi, kes nende poistega kolme aasta vältel tegelesid ja õpetasid neid oma mõtteid inimese moodi väljendama. Ja minu ema siis mingil hetkel sattus sinna kooli psühholoogiks. 

\question{Ta oli ju programmeerija?}

No mis siis? Ta vahepeal tegeles kutsehariduses testidega, pikk lugu jälle. Aga see oli kuidagi nii naljakalt, et kas mina soovitan alguses Reilile\index[ppl]{Reili, Ants} oma ema ja minu ema soovitas pärast mind, ühesõnaga see oli kuidagi rekursiivselt tutvuste kaudu. 

Jaanuaris 1993 olin ma seal igatahes paigas ja mulle öeldi, et neljandast veerandist (see, mis suvega lõpeb) ma pean hakkama juba kellelegi midagi õpetama. Oli teisejärguline, kellele ja mida, aga noh, lihtsalt, et oleks projekti eesmärgid ära täidetud. Ma vist isegi hakkasin natuke varem. Põhimõtteliselt mind pandi olukorda, kus oli mingisugune Unix. See \enquote{mingisugune} osutus pärast SCO 3.2.2\index{Unix!SCO}, masinasse oli kusagilt Tõraverest kaks Urania\index{Urania} muxi kaarti siis ostetud, nii et põhimõtteliselt sellele masinale sai kaks korda kaheksa terminali taha võtta, pluss oma konsool.  

Mu käest küsiti \enquote{Tead, mis Unix on?}. Ma ütlesin \enquote{Jaa, ma olen vähemalt ühte raamatut lugenud ja saan aru ja umbes ja siis}. Ja siis tuli tegelda. Kaks kuud läks selleks, et ma ise aru sain, mis asi see on. Seejärel aeti lapsed ette ja tuli neile õpetada.

\question{Mis masin see oli?}

386, tal oli 8 mega mälu. 40 mega ketast ja SCO Unix 3.2.2. 

\question{Huvitav kombinatsioon! Kas selle organiseeris seesama koolidirektor?}

Ants Reili\index[ppl]{Reili, Ants} ja tema sõbrad ja sugulased. Kooli direktor oli proua Errit\index[ppl]{Lookherup, Errit} ja tema tegelase vene keele õpetamisega. Temast me täna rohkem ei räägi. 

Eks  ta oli niisugune nii-öelda mentaalne ülekanne vanast \emph{mainframe} ajastut, no need mehed mõtlesid lihtsalt niimoodi, nad olid sellega üles kasvanud. 

\question{386 vedamas kuuteteist terminali\ldots?}

Oi, väga hästi! Ega siis ei progetud, nii nagu praegu, et \emph{include}-takse kogu eelnev maailm. Siis ikkagi kirjutati asju ASM-is ja korralikult, aga see selleks. 

Mulle anti kaks seltskonda, anti kaheteistkümnendikud ja anti viiendikud. Ja kaheteistkümnendikega veel nii ja naa, aga mida  nendele viiendikele seletad aastal 1993? Ja  mingi Unix on ka veel, eks ole. Päris kole. Aga, nagu öeldud, seal olid vanakooli mehed, mina progemisest jälle ei jaga suurt mõhkugi, aga Sven Turnau\index[ppl]{Turnau, Sven} oli  süsteemiülem ja see proges ANSI C-s nagu issand jumal ja proges mulle seal paar abivahendit. Üks oli mingi mäng, millega sai terminali ekraanile tärne joonistada. No vot, mida mina arvutuskeskuses kunagi tegin, väga tuttav asi oli. Ja see proge töötas, ei kiilunud kinni kusagil, lapsed said aru, kuidas ta on, ja pärast sai printerist välja lasta. Kuskilt TPI laost saime seda vana murdekohtadega paberit kilode kaupa, selle eest maksma ei pidanud. Printer lindi eest küll pidi, aga Reili eelarve elas selle kuidagi  üle. Nii et põhimõtteliselt lastel oli praktiline väljund. Joonistas oma jubeduse  ekraani peal valmis ja trükkis välja, ühe tunniga tehtud. Ja teine põnev asi oli, et SCO Unixil\index{Unix!SCO Unix} on email sees, saab üksteise masina piires kirju saata. Ja eks siis Mari sai Jürile teatada, mis ta tast arvab ja tema emast ja nii edasi ja seda nad väga aktiivselt tegid. 

Järgmine õppeaasta see kõik lihtsalt jätkus. Aga olukord läks huvitavaks aprillis, kui mind veeti Nõkku\index{Nõo Keskkool}, kus oli mingisugune Unixi koolitus. Juhuks, kui ma veel millestki aru ei saanud, siis et ma nüüd ikka tõesti ise ka aru saaks, mis see on. Ja observatooriumi\index{Tõravere Observatoorium} all siis Urania\index{Urania} nimeline firma, nemad selle korraldasid. Margus Liiv\index[ppl]{Liiv, Margus} ja  Kaiti Kattai\index[ppl]{Kattai, Kaiti} ja kes nad seal olid. Muide sellest päevast hakkab mu digiarhiiv peale.  Minu  arust oli kas 2. või 5. aprill, ma ei mäleta. See on mul juba digitaalselt alles. Tegelikult neid Bondwelli programme on ka kuskil flopide peal, aga tühi nendega. 

Pärast seda Urania  koolitust,  ütlesin et lähme siis Nõkku ka juba, see siin lähedal, vaatame, mis need seal koolis teevad. Ja Nõos oli selline härrasmees nagu Kill Kask\index[ppl]{Kask, Kalju}\index[ppl]{Kask, Kill|see{Kask, Kalju}}. Räägib \enquote{Ah, me ripume mingi interneti küljes ja mingi trillallaa-trullallaa, mingi kirjade vahetamine} ja väga meeldiv, eks ole. Neil oli laual mingisugune karbike. Ma küsin, \enquote{mis see on}, \enquote{see on modem!}, \enquote{Ahah.} Rohkem ma ei julgenud küsida. Kui modem, siis modem. Nagu öeldud, netti ei olnud, ma ei tea, kust ma selle selgeks tegin aga mõne päevaga oli nagu kontekst selge, et mis aparaat see on ja mida sellega saab. Kirjutasin Avatud Eesti Fondi\index{Avatud Eesti Fond} taotluse, et \enquote{ma tahan nüüd ka seda saada}. Siis ma olin veel Sorosega sõber, siis nad tegelesid veel tõsiste asjadega, mitte nendega, millega praegu. Kool sai selle raha kätte, nii et põhimõtteliselt 1993 sügisel, kuupäeva ei mäleta (mingid projektid on kuskil trükitult alles, aga vahet pole), sai selle modemi. SCO Unix\index{Unix!SCO} võttis selle modemi ilusasti taha ja vot nüüd selgus, et meil on relv! Kuidas käis normaalsetes koolides tollel ajal emaili saatmine? No näiteks Tartus Treffneris\index{Hugo Treffneri Gümnaasium}, seal sai ka külas käidud,  oli palju arvuteid, üle 10. Ikka väga hästi varustatud kool. Aga modem oli neil taga ainult ühel ja selle arvuti taga oli järjekord. Üks õpilane siis kahe näpuga toksis oma kirja ära, saatis minema ja teised ootasid järjekorras. Oligi järjekord kogu aeg! 

\question{Aga sul olid ju terminalid, see on arhitektuurselt palju parem!} 

Absoluutselt! Ja see oli relv. 

Kuidas see välja tuli. Anne Villems\index[ppl]{Villems, Anne} Tartust korraldas õpilastele igasugu mängusid, et nad internetiga  ära harjuksid. Näiteks Gaia, kus olid mingid välja mõeldud riigid. Mäletan, meie kool nimetati Barbariaks, üks väga õige nimetus\sidenote{Gaia käivitus 1994. õppeaasta alguses ja sellest võttis osa 20 gruppi 17 keskkoolist või gümnaasiumist.}! 

Ja nagu öeldud, meil ei pidanud keegi järjekorras seisma. Meil 17 tükki (16 aga kui Turnaul\index[ppl]{Turnau, Sven} hea tuju oli, lubati keegi konsooli taha ka) võisid korraga oma kirju trükkida. Masin korjas nad kokku ja ära saatis siis, kui tal sellega tegelemiseks aega oli. Aga sellega asi ei piirdunud. SCO Unixis, siis sai modemiga sissehelistamise peale häälestada. Ma ei mäleta enam, mis pagana laos me käisime, me saime Eesti Energiast\index{Eesti Energia} paar vana mudelit, 1200 baudi (ilma vea korrektsioonita, mingi ürgaegne värk) ja mingisuguseid soome tähtedega terminale ja ühesõnaga mingeid koledaid asju. Soome tähed ju muudavad ära ASCII lõpu, kus nurksulud on, sinna tulevad nende ööd üüd. Aga selle tulemusena aktiivile, ehk siis kolm neli poissi, kes kõige aktiivsemad arvutiklassis käijad olid, õnnestus saada koju modemid ja terminalid, sest kes see kannatas siis endale arvutit osta. Nüüd läks asi hulluks kätte, sellepärast et poiss logis ennast öösel kooli SCO Unixisse ja trükkis kirja valmis.  Seepeale Anne Villems\index[ppl]{Villems, Anne} ütles, et te olete lihtsalt tehnoloogiliselt teistest nii palju üle, et see ei ole enam aus. 

\question{Huvitav on see, et sa ei olnud üle mitte tehnoloogiliselt selles mõttes, et sul oleks ägedam arvuti olnud aga just \emph{setup} oli äge!} 

Jah, organisatoorne pool, sest seal olid vana kooli mehed, kes teadsid, kuidas nad selle püsti panevad ja see sissehelistamine, vot see oli väärtuslik. See on omaette jutt, seda võib ka natuke rääkida, aga ütleme siis, et  Microsofti masinatega mingit sissehelistamist ei olnud kellelgi. Ja see, kui poisil öösel kell kolm und ei ole tuleb nii-öelda  inspiratsioon peale ja tahab seal Gaia mängus kaasvõitlejad teise planeedi peal saata, siis tal oli selleks täielik tehniline võimalus. 

\question{Legend räägib, et umbes sel ajal tehti Eestis esimesed veebmasterite kursused. Sina olla seal ka tembutanud?} 

Ei mäleta, see on minu jaoks praegu tühi koht. 

Aga sealt hakkas üks, teine rida. Nimelt ma sain kiiresti aru, et see 2400-ne  modem on nabanöör. Sest mina ise, selle asemel, et öösel koju magama minna (mul ei olnud kodus terminali tookord veel)  istusin 1993. aastal ja  1994. alguses öösel koolis ja rippusin, horos.kbfi.ee\index{horos.kbfi.ee} küljes, sinna sai sisse logida. Tallinna välisühendus oli tollel ajal 64 kilobitti sekundis. Ja kuna öösel normaalsed inimesed magasid, siis põhimõtteliselt viis pool kilobaiti sekundis oli maksimaalkiirus, mida ma sealt kätte sain. See masin muidu tõmbas uudiseid, \emph{news group}-e. Aga mina sain mööda netti ringi kolistada,  terminali ekraaniga asja tõmmata. Öö jooksul ma suutsin tavaliselt umbes viis flopitäit kohale tõmmata. Kui  järgmine päev tund ei olnud, magasin välja ja läksin Andres Baumani\index[ppl]{Bauman, Andres} juurde KBFIs\index{KBFI} ja anusin, et kas ma nüüd saaks tema masinast asjad flopi peale ära kopeerida. Teine variant oli, et järgmisel öösel, tõmbasid oma modemiga. 

\question{Mida sa tõmbasid?}

Tollel ajal oli tõmmata väga palju. Internetis paska veel ei olnud ja häid materjale oli palju. Ülikoolidel olid gopher-i saidid, FTP saidid. Veeb just hakkas tulema, ei olnud veel tulnud. Ja minu lemmikmeetod oli see, et ma läksin mingisse uudisgruppi nagu mingi alt.sex või mis iganes, noh, kus inimesed ikka käivad. Ja siis mingisuguse progejupiga, ma ei mäleta mingi sed ja awk, mida Turnau\index[ppl]{Turnau, Sven} mulle õpetas, otsisin välja  ülikoolide aadressid. \emph{Strip}-isin nimed eest ära ja läksin käsitsi mööda ülikooli FTP servereid kollama. Ja kuna tollel ajal mingit andmekaitset ega midagi ei olnud, siis absoluutselt kõik asjad olid ripakil. 

\question{Anonüümne FTP! Ütlesid \enquote{anonymous} ja \enquote{ftp}\sidenote{Levinud (ja kasutajate hulgas laialt teada) viis avalikke FTP teenuseid pakkuda oli kasutajale \enquote{anonymous} kas teadaolevat parooli \enquote{ftp} või aktsepteerida paroolina mis iganes sisendit.} ja saidki sisse}

Ja, aga tollel ajal USA-kad ei osanud seda veel karta, nii et põhimõtteliselt ma harvesteerisin vastuoluliste nimedega gruppidest, sest need olid kõige suuremad. Sealt sain maksimumkoguse nimesid, ja siis nende ülikooli nimede järgi (ega keegi pole mind õpetanud, mis ülikoolide USA on) käisin tõin FTP saidist kõik  ära, mis mulle tundus, et on lugemisväärne.

\question{Just lugemismaterjal, mitte programmid?}

Niipalju kui mul üldse mingisugust krüpto-teadmist on, sellest ikkagi tugevam osa on sealt pärit. Asjad, mis olid ripakil: küll õppematerjalid, küll igasugused teadustööd. Ja tollel ajal näiteks koolis arvuti õpetamine polnud teadus, see oli šamanism. Ja sealtkaudu ma leidsin esimesed teadustööd, mis seda USA-s käsitlesid, neetult huvitav oli lugeda! 

\question{Need ei olnud ju PDF?}

Enamasti olid tekstifailid, PDF-id hakkasid pigem ikka natuke hiljem tulema. ACII printeriga trükkisid välja, nii see elu käiski. 

\question{\LaTeX-i kraami ka leidus?}

Mina ei olnud tollel ajal selle ala inimene, hetkel juba kirjutan. Ilmselt oli. Ja Emacsit ma ei ole ka ära õppinud, ja ei õpi. Vi-ga kirjutan. 

Pisut hiljem ma sain tuttavaks mehega, kes rippus samamoodi öösiti Tartu satelliiditaldriku taga, tema nimi on Marek Tiits\index[ppl]{Tiits, Marek}. Tuli välja, et mina ekspluateerisin  seda Tallinna oma siin ja tema Tartu oma. 

Aga jah, siis läks ajad huvitavamaks. Ma sain kuidagi aru, et sellest ühest modemist ei jätku ja et mingisugune \enquote{otse internet} on ka olemas. Ma jätan praegu vahele selle jutu, kuidas Andres Bauman\index[ppl]{Bauman, Andres} KBFIst jälle mingi grandiraha eest kirjutas UUCP-d ja Pegasus maili siduva proge, mida koolid ja kõik kasutasid, aga see selleks, see on eraldi \emph{thread}. 

Nii koolis see salarelv käis, see tehnoloogiline üleolek nii-öelda. Loomulikult, Treffneri omad, ma usun, olid ikka mõnes mõttes haritumad, kui meie, vähemalt maailmavaate poolest, aga kahuri jämedus oli suurem. 

Ja mingi hetk siis sai suhtlema hakatud nendega, kes neid va õpetajaid koolitasid ja koolides seda internetti levitasid. Anne Villems\index[ppl]{Villems, Anne} ja tema  koolkond. Ja sealt kusagilt tuli siis see arusaamine, et ikka on vaja kuidagi püsiühendus sisse saada. Aga no 94. aastal ja püsiühendus! Alates rahast ja lõpetades kõige muuga, see oli ikka \emph{mission impossible} tollal. Põhimõtteliselt siis EENetiga\index{EENet} mingit koostööd teha, EENeti poisid aitasid mingeid projekte kirjutada. Siis tuli kirjutada mingi jube haridusprojekt, et paneme 117 kooli internetti ja tulevikus otse ka ja kui otse ka, siis tähendab, et kusagil on vaja nimeserverit. Ühesõnaga ütleme nii, et Andres Bauman\index[ppl]{Bauman, Andres} tegi viltuse näo pähe, et koole on nii palju saanud, et tema väike vaene MicroVAX-ist nimeserver ei jaksa neid enam pidada. Noh, sa saad aru, mis selle väite tõeväärtus on\sidenote{Nimeserver on üks väiksema ressursivajadusega Interneti tuumtehnoloogiaid, jutuks olnud riistvara oli selle teenuse pakkumiseks vähemalt suurusjärgu võrra üledimensioneeritud.}. Aga oli vaja teha eraldi koolide nimeserver. No ja kui vaja, siis vaja, eks ole. Kuskilt Haridusministeeriumi teadusrahadest eraldati mingisugune Sun selle jaoks. Kas see oli nüüd SPARCStation 10 või midagi taolist, issand, kes seda mäletab! 

Ja kusagilt kooli konkursilt õnnestus ka kolm arvutit saada. Rohkem ei antud, öeldi, et teistel on ka vaja. Arvuti all ma  mõtlen siis eraldiseisvat Microsofti masinat. Ja nendest  tegelikult üks masin läkski nimeserveri alla. Põhimõtteliselt tegime diili, et mina kooli konkursiga saadud masina panin nimeserveriks. EENeti SPARCStation-ist, oli ta nüüd viis või kaks või kümme, enam ei mäleta, sai hariduse veebiserver. 

Ma arvan, et see oli 1993 lõpp, kui see Sun oli juba olemas, peak.edu.ee\index{peak.edu.ee},  tuksus seal Bauman juures mingisuguse riiuli peal. Ma sain talle ise kaugelt ligi. Olime kord  Liivi tänava\index{Liivi tänava õppehoone} keldris saunas ja Toomas Soome\index[ppl]{Soome, Toomas} rääkis, et  mingi jubetumalt kihvt asi on olemas, veebiks nimetatakse. \enquote{Ahaa, rõõm kuulda!} \enquote{Tartu Ülikoolil olla ka} \enquote{Ei no tore on!}. Pärast istusime mingi terminali taha, Toomas Soome nõidus seal natuke, kompileeris. See on tegevus, millest ma siis absoluutselt aru saanud, poisid, õpilased, hiljem koolis õpetasid. Olgu. Aga igatahes mingi proge sai sinna kokku ja  mingisse faili kirjutati mingi üks rida mingi \enquote{killadi, kolladi \emph{coming soon}, siia varsti tuleb midagi}. No ja siis ma vaatasin, mingi programm oli, Mosaic, sellega sai täitsa vaadata, oli isegi mingi kiri, mis ütles, et midagi tuleb\ldots Põhimõtteliselt Toomas Soome selle laiendatud sauna vältel kompelleeris mulle veebiserveri ja tegi sinna esimese faili, ühe reaga, ja minu minu peal nüüd lasus kohustus. Ja mul ei jäänud muud üle, ma pidin selle selgeks õppima. Aga see väidetavalt oli Eestis seitsmes veebiserver. Ma arvan, et see saun oli 1993. detsember aga mälu võib siin petta. Igatahes niimoodi see www.edu.ee tekkis. Sisu tekkis alles palju hiljem.

\question{Selle peab ju keegi kirjutama!} 

Mis seal kirjutada, ma lihtsalt läksin nuiasin Haridusministeeriumist administraatori käest nende andmebaasi välja ja sai kõik avalikult netti pandud. Asi, mida praegu nagu üldse teha ei tohi. Kõik isikuandmed on ju saladus.

\question{Mis andmebaas see oli?}

No ega tol ajal inimesed ei teadnud, palju Eestis koole on. Selles andmebaasis oli isegi kooli direktori nimi olemas! Aadress ja, põhimõtteliselt seal mingi üheksa või kümme rida infi iga kooli kohta. Koole oli kokku mingisugune 600, aga sealt tuli mingi valik teha, päris mingeid erikoole ei hakanud panema ja see oligi veebiserveri kõige esimene versioon. 

Aga nüüd on see järgmine oluline \emph{thread}. Sai EENetiga\index{EENet} koostööd tehtud, põhimõtteliselt oli mingi rahastamise kokkulepe, milleks oli siis järgmine projekt: teeme koolide sissehelistamiskeskuse, teeme selle sinnasamma 43. kooli\index{Tallinna 43. Keskkool} juurde. Poisid olid selle teenusega hästi rahul, sest nad said aru, et neid modemeid saab muuks ka pruukida. Saavad ise öösel sisse helistada, kuhu vaja. Aga see trass, see vaskkaabel tuli ise välja ajada. Tuli ise käia keldris juhtmeid kokku mässimas ja takistusi mõõtmas. Ja siis oli küsimus, mis tehnoloogia saame. Vendomar\index{Vendomar}, kodanik Kingissepp\index[ppl]{Kingissepp, Meelis}, üritas meile RAD-i\sidenote{Tõenäoliselt peab Anto silmas Iisraeli samanimelise firma modemeid, mis toona uudse kontseptsioonina ei vajanud toimimiseks eraldi toiteallikat.} müüa See oleks muidugi olnud 64k kiirusega, aga see ei hakanud selle liini peal tööle. See, mis tööle hakkas, oli US Robotics\index{US Robotics} ja ka mitte päris 33,6 peal, aga ma ei mäleta, kas mingi 28 või 29, mingi sihuke kiirus. No vot. Ma võin kuupäevaga eksida, paar päeva siia sinna, minu arust juuli lõpp. Suveaeg, ta võis olla mingi 29. juuli või 28. või 27., mis on see päev, kui sai selle püsiliini  lõpuks tööle kooli ja KBFI\index{KBFI} vahel. Mõlemas otsas modemid ja võttiski \emph{carrier}-i üles! Nojah, ja nüüd on see õnnetu moment, kus saad oled otse internetti küljes kiirusega 28 kilobitti sekundis! 

Nüüd astus nurgast välja Antti Andreimann\index[ppl]{Andreimann, Antti} ja ütles, et ta on seda hetke oodanud sajandeid ja et meil koolis on kõik asjad tuksis, sest meil on ainult vana räpane ANSI kompilaator ja meil on vaja GNU C kompilaatorit, millega saaks maailma päästa. Ja viis pool tundi kõik ülejäänud ootasid, internetti kasutada ei saanud, sest Antti tõmbas GNU kompilaatorit. Kõik, kes olid pühitsemiseks kokku tulnud, ja arvasid, et nüüd saab midagi tõmmata\ldots {} See oli GNU kompilaator, mis sealt tuli! Aga ma usun, et Anttil oli lõigus, et  nii oligi ja elu läks oluliselt paremaks, sest nüüdsest õnnestus kompileerida asju, mida varem ei õnnestunud. See asi ei käinud nii, et läksid ja tõid funet.fi-ist\index{funet.fi} või kuskilt mingist pirasaidist, Unixi puhul oli teisiti. 

See kuu aega, mis koolini jäi, poisid ainult koolis elasid. Tolles augustis hakkasid ka esimesed nähtused tulema. Sellise kogemusega, kui sul on juba nihuke interneti kiirus,  siis sa paratamatult satud mõne kõrgkooli bugistesse veebidesse ja\ldots Kuidas ma nüüd ütlen. SCO Unixil\index{SCO Unix} oli üks knihv, kuidas  pidi modemit konfima ja kui sa seda knihvi ei teadnud, siis \emph{carrier detect} jäi püsti, kui sa meie kehva telefoniliini pealt maha kukkusid. Aga, ütleme niimoodi, et  Peda ja TTÜ adminnid kukkusid oma ruudu \emph{prompt}-i otsast päris sageli maha. Ja kuna meie poisid teadsid seda nippi, siis mõnikord sai niiviisi toimiva terminali otsa kätte. Noil aastatel juhtus igasugu artefakte. Seadusi hakati tegema alles hiljem, 1995 vist hakati ja 1996 kuulutati välja. See oli ka mingi töögrupp, ma mäletan, ma käisin Tartus Riigikohtu\index{Riigikohus} omadega midagi arutamas. No mis ma oskasin neile rääkida! Nii palju, kui ma olin netist lugenud, et kes läänes mille kust ära varastas, ega ma ka rohkem ei osanud. Oma õpilaste pealt ka midagi. 

\question{Siis sa puutusidki esimest korda infoturbega kokku?}

Nojah. Kas oli vist 1995\sidenote{Vastav lugu Äripäevas, kus ka Antot tsiteeritakse, on pärit 1996. aasta aprillist.}, oli see kuulus \emph{noname} lugu. Mingi pisipank oli meil Eestis, EVEA pank\index{EVEA Pank}? Igor Švets\index[ppl]{Švets, Igor} töötas seal süsadminina. Ta oli tehniliselt väga kvaliteetne häkker, kolistas kogu Eestit pidi ringi, kuhu aga sisse sai. Mina ei tea, kellele ta töötas. Iseendale või oma rahvuslikule päritolule või, nojah, hea küll, ei võta seda teemat ette. Ega mina mingisugune kõva käsi ei olnud, aga ma vähemalt olin võimeline aru saama, et keegi kurat elab mul masinas. Eks poisid natuke aitasid ja ega Sven Turnau\index[ppl]{Turnau, Sven}, ametlik masinaülem, polnud ka loll mees. Sai aru, mis toimub ja siis sai see skandaaliks keeratud. Ajalehes ära trükitud, kuidas \enquote{\emph{mister noname}},  Postimees tegi temast mingit noaga pildi. Press on press. Politseisse polnud üldse mõtet helistada, need ei saanud aru sellest. Tollal polnud ei paragrahve ega midagi. KAPOsse helistasin ka, need ei saanud ka aru. Kõvasti on ikka arenenud ja õppinud! 

Aga jah, mingi hetk me suutsime iseennast ära kaitsta, saada aru, et see et see värk on tegelikult olemas. Keerulisem, kui enda kaitsmine, oli tegelikult see, kuidas poisse seiklustest eemale hoida. Üks tüüpiline lugu oli see, poisid tulevad ja ütlevad, et \enquote{meil juhtus õnnetus}. Ma küsin siis, et mis õnnetustel teil siis täna juhtus. \enquote{Me saime kogemata Pedas ruuduks!} Pedas\index{Peda} oli Cadmus\index{Cadmus}\sidenote{Saksa arvutitootja Periphere Compter Systeme (PCS) toodetud tööjaam, mis käis toona kehtinud võimeka riistvara kolmandatesse riikidesse eksportimise embargo alla. Kuidagi aga õnnestus ühte Teaduste Akadeemia allasutusse selline masin hankida, kus tema nimeks sai konspiratsiooni huvides \enquote{MUSCAD}\index{Muscad}.}, mingi tumba  eks ole. Eks nad läksid netti ja küsisid, et mis bugid tal on, ja loomulikult tal olid ja  nad said ruuduks ja \enquote{mis me nüüd teeme?} Ma ütlesin, et \enquote{mis siis ikka, parandage bugid ära ja kinkige masin tagasi}. Parandasidki bugi ära, kompileerisid talle uue kerneli ja ma ei mäleta, mida veel (ma üldse imestan, et nad seda masinat õhku ei lasknud) ja saatsid süsadminile lõpuks kirja, kus nad seletasid, et nad on ta bugid ära paiganud ja et ta võib nüüd oma masina tagasi saada. See oli niisugune hästi tüüpiline juhtum.

\question{Need olid mingi 12. klassi poisid kes seal möllasid?}

Ei olnud. See aktiiv, kes üldse tunde ei saanud, olid mingid üheksandikud kümnendikud. See oli lihtsalt niisugune tore aastakäik. 

\question{Ehk, üheksakümnendate keskel üheksandikud võtavad ruutu?}

Jaa! Mõnest asjast ei julge tänaseni rääkida, inimesed on just \emph{scene} peal alles. Ikka juhtub õnnetusi ja, ütleme niimoodi, et oli ka üks poisslaps (ta on ka siin\sidenote{Vestlus toimus Cybernetica\index{Cybernetica} kontoris.} töötanud, vahet pole), kelle ema oli teadustöötaja.  Ma ei tea, kes ta isa oli, aga tundus, et igatahes seal peres kas oli raha rohkem või saadi aru, mis on oluline.  Ja temal oli 486 tollel ajal. Ta ei olnud meie koolipoiss, ta käis külaliseks. Aga näiteks tema, nähes SCO Unixit\index{SCO Unix} ja seda igavest muret, mis seal seerianumbritega oli, kirjutas programmi, mis neid seerianumbreid nagu küllusesarvest väljastas. Ta ei olnud veel keskkooli ära lõpetanud. Tema oli ka esimene,  kes Jack the Ripperiga\sidenote{John (mitte Jack) the Ripper on tuntud paroolide murdmise töövahend.} õppis ringi käima. Tallinnas Tehnikaülikoolis\index{TalTech} oli SUN-i klass. Mingid vanad hädaabi masinad, Rebane oli arvutiklassi pealik. Aga tollel ajal tsentraalne lahendus oli NIS-i peal, tänapäeval see sõna ei ütle mitte midagi. Aga,  ütleme niimoodi, paroolifail oli kättesaadav. No ja, eks ole, see räägitud kodanik, kellel oli natuke parem masin (tal oli kodus ka SCO Unix, kujutad ette!), lasi kõigepealt kodus Jack the Ripperi käima ja, kuidas ta ütles,  kolme päevaga sai vist 70 protsenti paroole lahti, mingi niisugune lugu. 

Olid ajad, olid ajad!

Ja siis oli ju veel see telefonide läbi helistamine. Modemeid oli maru palju, aga turvat mitte mingisugust. Sihuke proge oli nagu Tone Loc. Nagu sa praegu pingid läbi mingisuguseid IP aadresside plokke, tollel ajal sa selle progega nii-öelda pingisid läbi telefoninumbreid.  Kohalik kõne ei maksnud ju midagi. Nii et mina olen ka korra elus näinud olukorda, kui  ühest autost, kus on kolm laptopi, lähevad krokodillidega juhtmed öösel telefonikappi ja  lastakse kümnetuhandesi plokke läbi. See on juba laiemalt  Eesti infoturbest, sinna etappi jäävad ju Eesti Telefoni digikeskjaamade esimesed häkid. Mingid transpordikooli poisid said aru, kuidas digijaam töötab ja said aru, et paroolid on nirud ja detsembris,  kui kõik olid kas jõulupühal või välismaal komandeeringutes või kuskil mujal, siis võeti päris mitu jaama üle. Ka midagi hirmsat ei tehtud, kost oligi aru saada, kellega tegemist on. 

\question{Kuidas neid poisse siis kantseldati? Neil pidi ju olema mingisugune eetiline arusaam, et \enquote{pätti ei tehta ja puruks ei lasta}?}

Ma ei kujuta ette, kes need seal transpordikoolis olid. Ma mõnda õpetajat tunnen, aga see oli ka selline pikantne teema, sest, vaata, kui sa liiga targaks saad, kui sa saad lõpuks nagu täpselt teada, kes see oli, siis seal tekivad omal moraalsed kohustused. Mina lahendasin selle nii nagu mulle õige tundus. Ma püüdsin neid suurest jamast eemale hoida ja ega seal see tavatarkus, et \enquote{mis sa raisk nüüd tegid}, see ei aitand mitte midagi.  Sa pidid ta ära kuulama ja siis kuidagi väga ettevaatlikult ta sinna õige tee peale tagasi patsutama. 

\question{Sest tol hetkel sihukesel murdeealisel poisi nii-öelda võimekus on märksa suurem, kui mõistus või arusaama elust!} 

No võimekus on tal suur sellepärast et ta veel õlut ei joo, selle peale aega ja energiat ja raha ei kulu ja tütarlastega ka veel ei semmi ja selle peale ka raha ei kulu. Järelikult, kui parasjagu pole koolipäev, siis on 16 tundi aega, sest kaheksa tuleb magada. Ja kui sul on inimene, kellel on 16 tundi päevas aega istuda ja murda, siis see on väga toores jõud. Ja kui neid veel mitu on ja nad omavahel suhtlevad\ldots

\question{Kaua sa seal koolis olid?}

Vot nüüd ma jään nende aastanumbritega\ldots Kolm aastat olin ma seal kindlasti. Mingil hetkel, kui sai seda EENeti\index{EENet} keskust hakatud tegema, kujunes teiseks tööandjaks EENet. Ma võin praegu eksida, millal see oli, äkki oli juba 1994 lõpp, äkki oli 1995 algul. See on minu ülekandumine EENeti toimus kuidagi väga sujuvalt. Ametikoha nimi oli \enquote{insener}. Põhja regiooni oma, põhimõtteliselt ma vastutasin  tipphetkedel mingi 150 kooli UUCP side eest. Öösel kell 11 helistab mõni tädi,  kooliõpetaja, ja räägib, kuidas tal mitte miski ei tööta ja  siis sa pead peas looma mentaalse mudeli ja nagu helpdesk siis tema katkendite järgi tegutsema. \enquote{See sinine lätakas} --- ahah, Norton Commander, selge. Kui sa seda, Apollo lugu tead, kuidas neil seal midagi plahvatas ja maa peal tükikestest\ldots {} Vot samamoodi. 

Sel hetkel sai tehtud ka mõned koolitused,  olid mõned uued metoodikad. Käisime, see oli küll vist 1995, Anne Villemsi\index[ppl]{Villems, Anne} juures Tartu Ülikoolis\index{Tarttu Ülikool} koolitust tegemas. See oli siis niimoodi, et sissehelistamiskeskuseks oli meil minu läpakas. Telefoni keskjaamast tõime pikad otsad (kaks päeva oli ette valmistamist), mida sai klassis iga arvuti juurde viia. Ja  kooliõpetajatega koolitust alustasime niimoodi, et kõigepealt käskisime neil arvutit lüüa. Et neil tekiks\ldots {} Aga selle taga oli see, et tegelikult oli klassi valdajaga kokku lepitud, mis juhtub, kui mõni arvuti \enquote{ära lüüakse}. Aga ei nad nii kõvasti ei julgenud lüüa. Noh, et tekiks kohe selline õige tunne, et saaks hirmust üle, \enquote{kes on peremees?} Teine asi oli, kuidas hoida neid tädisid ülearuse info eest. Tollel ajal ju modemi konfigureerimine, eks ole, baasaadress oli mingisugune 0H370, ma ei mäleta, mida iganes. IRQ2 ja IRQ3. See on mõttetu! Ja mina nii ütlesingi, et \enquote{ära süvene sellesse, mis see on, siin on täpselt ainult nii palju valikuid. Vali neist üks ära!}. Ja tulemus oli siis see, et kõik kes seal koolitusel olid, said oma modemid konfitud. Läksid oma koolidesse laiali ja said ka. Aga selle ette valmistamine oli 50 ühele: ühe tunni hoolitsuse kohta, mis nad said, läks 50 tundi ettevalmistust. Et see on üks kõige jubedamaid asju olnud, aga metoodika mõttes toimis! 

Siis tekkis jah see, et kui Ants Reili\index[ppl]{Reili, Ants} suri ära, ma ei mäleta enam seda täpset surmapäeva, ma mäletan, et ma Keilast taksoga jõudsin täpselt selleks hetkeks kohale, kui ta seal Keila surnuaial maa sisse pandi. Aga siis lagunes ETEK ära, või noh, majanduslikult. Me kauplesime näiteks endale Tehnikaülikoolist\index{Tallinna Tehnikaülikool} Jüri Kaljundi\index[ppl]{Kaljundi, Jüri} käest vana Väksu\index{VAX}. Ainult et transaga midagi sai pihta, ei läinud enam tööle. Ja kooli direktor tundis väga huvi, et kust ma elektrit kavatsen võtta selle masina toiteks, nii et käima ta ei läinudki. Lõpuks ta läks ka kuhugi utiili. Ühesõnaga, see \emph{business} vajus seal sellisel kujul koost ära. Tipphetk oligi seal enne mind natuke aega, minu ajal ja mina kandusin rohkem nagu EENeti. 

Nii et enne, kui ma 1997. aastal järgmist korda mingi pöörde tegin, istusin  Tatari tänaval Microlinki kõrval üleval (seal oli üks tuba EENeti kontor), kus siis üks Fix'i kunagine helitehnik, Antti Andreimann\index[ppl]{Andreimann, Antti} ja  mina  kolme peale kokku tegime, mida me siis suutsime teha. Antti sai aru, mis tegelikult toimus ja suutis kernelit kompileerida ja meie tegime mingeid muid asju. Sellega see koolide saaga lõppes ära ja siis tekkisid juba mingid Tiigrihüpped\index{Tiigrihüpe} ja asjad, sellisel kujul lähenemist ei olnud enam vaja. 

\question{Ühesõnaga sa jõudsid infoturbe juurde kuidagi väga praktiliselt ja samas väga inimesekeskselt. Sa pidid tegelema nende kuramuse kaakide ohjeldamisega, kes 16 tundi päevas igale poole auke torkisid} 

Ei, mis nüüd kaagid, nad on kõik lugupeetud inimesed, nad on kõik skeene peal ja minu teada pole\ldots {} Vat ühe suhtes ei tea ja üks töötas siin aastaid ühe asutuse adminina  ja ma ei teagi täpselt, mille eest ta sealt lõpuks lahti tehti. Aga kellegi teise ajaus ei ole midagi paha kuulda, kõik on väga lugupeetud inimesed. Osad on siin üldse mingis sõjaväes kuskil x riigis ja nii edasi. 

\question{Mis sa praegu teed?}

Näed, me siin istume mingis Cybernetica-nimelises AS-is\index{Cybernetica} ja ma olen hetkel kirjaneitsi. Vot see on maru raske küsimus, et mida sa oskad. Alati, kui tööle võetakse küsisekse, et mida sa oskad. Ja mina mõnikord, mitte alati, aga mõnikord, oskan mõne asja suhteliselt selgelt kirja panna. Ja siin majas neid segaseid asju, mida on vaja pisut selgemini kirja panna on suurtes kogustes. 


\question{Vot seda sa oskasid küll väga selgesti öelda, ma sain kohe aru, mida sa teed} 

Sõltumata sellest, mis selle töökoha nimetus on, milline see järjekordne käimasolev projekt on, aga, jah, sedamoodi. Mingi natuke teine vaade, natuke parem sõnastus ja mõni järgmine seltskond saab sellele tuginedes juba järgmise lati ära võtta, mis iganes see siis oleks. Mõni asi kuhugi riiki maha müüa või midagi muud. 

\chapter{Vilve Vene}
\index[ppl]{Vene, Vilve}

\question{Kuidas sina said arvutite juurde ja arvutid 
sinu juurde?} 

See algas juba koolis. Käisin 1. keskkoolis\index{Tallinna 1. 
Keskkool}, mida täna tuntakse kui GAGi\index{Gustav Adolfi Gümnaasium}, 
matemaatika-füüsikaklassis. Meile õpetati ka programmeerimist. Mul oli väga 
hea matemaatikaõpetaja ja füüsikaõpetaja, mulle õudselt meeldis, aga ma 
ilmselt ei teadnud, kas see meeldib mulle piisavalt palju või meeldib mulle midagi 
muud rohkem. Mulle meeldis ka kirjutada ja tegelikult arvasin 
keskkooli lõpuni, et lähen pigem eesti keelt ja kirjandust 
õppima. Aga kuidagipidi arenes mõte sinnamaani, et õpiks 
füüsikat. 

Läksin Tartusse, vaatasin, et seal oli üldse vist neli tüdrukut, 
mõtlesin, et jube veider, ja andsin avalduse rakendusmatemaatikasse\index{Tartu 
Ülikool!Matemaatikateaduskond!Rakendusmatemaatika}. 
Kui avaldust sisse viisin, vaatas tädi laua taga mulle otsa ja 
ütles: \enquote{Teie oma tunnistuse ja kuldmedaliga saaksite ju 
arstiteaduskonda sisse!} Selgitasin talle, et ma ei taha, ja nii see 
läks. 

\question{Mille peal teid GAGis programmeerima 
õpetati?}

Jessukestel\index{Jessuke} ja perfokaartidel. 

\question{Kas koolis oli kõik see olemas?}

Ei, me käisime Teaduste Akadeemia Arvutuskeskuses\index{Teaduste 
Akadeemia!Arvutuskeskus}.

\question{Kooli poolt väga edasipüüdlik ettevõtmine! Kes seda asja koolis 
ajas?}

Ma arvan, et tollal oli direktoriks Helmi Viikholm.\index[ppl]{Viikholm, 
Helmi}\sidenote{Helmi Viikholm oli 1. keskkooli direktor aastatel 1962--1982.} 
Selline inimene, kes suutis ka nõukogude aja \emph{setup}'is 
organiseerida. Meil olid jube head õpetajad, avatud mõtlemine ja 
nii see tuligi.

\question{Kas Teaduste Akadeemia lasi õpilased oma arvutitele ligi?}

Jah! Aga eks kood läks perfokaardi peale -- ise toksisime 
perfokaardid, need läksid masinasse ja tulemus tuli välja. 

\question{Mis programme te kirjutasite?} 

Tegime lihtsaid asju. Ma sisu ja detaile ei mäleta, aga igatahes oli 
see Fortranis\index{Fortran}. 

\question{Seni on umbes igas teises loos läbi 
jooksnud ruutvõrrandi lahendamine \ldots} 

Vaat seda ma ei mäleta, minu meelest olid meil ikkagi natuke 
ärilisemad teemad.

\question{Nii et pärast keskkooli läksid Tartusse Vanemuise tänavale\index{Tartu 
Ülikool!Vanemuise tänava õppehoone} rakendusmatemaatikat õppima?}

Jah. Meid alustas 20. Minul, kellel 
koolis oli kõik jube lihtne, oli algus hästi raske. Ja alustati 
assemblerist\index{Assembler}. Pärast olen mõelnud, et see oli väga hea -- 
assembleris pead aru saama, mis seal 
sees toimub, ei saa lihtsalt kirjutada mingeid koodiridu, mõistmata, mis sügaval sees toimub. 

\question{Selle raskuse ületamise taga pidi olema mingisugune kihk. Miks sa ei 
läinud arstiks õppima?} 

Mind ei huvitanud!

\question{Ja arvutid huvitasid?} 

Mõtlen siiamaani, et kui edasi tulid intelligentsed 
programmeerimiskeeled, kus ei pea aru saama, mis seal sees toimub, siis 
tänapäeval on programmeerimine minu meelest 
käsitöö. Selles mõttes, et nii kampsunit kududes kui ka 
programme tehes tuleb osata võtteid ja kasutada tööriistu. Tollal
ei olnud sedasi. Kihvt oli alustuseks läbi mängida, mis juhtub: 
\enquote{Okei, see rida teeb nüüd seda, paneb selle sinna ja selle sinna. Kui 
mul on nüüd vaja võtta see sealt ja teha sellega midagi, siis mida ma selleks tegema 
pean?} Said sügavuti aru ja see oli hästi vahva. 

Viis, kuidas nad õpetasid, oli karm, aga tegelikult see 
selekteeris välja inimesed, kes tõesti tahtsid seda teha ja olid selleks ka
võimelised. Alguses tuli loogikat kõvasti, samuti mat-analüüsi. Ikka 
\emph{hard core}, mitte nagu tänapäeval õpetatakse. Rääkisin just
sõbrannaga, kes õppis teoreetilist matti ja kelle kursuselt on ülikoolis 
hästi palju õppejõude. Nad ütlevad, et ei saa enam sellel tasemel õpetada, vaid peavad oma 
programme lihtsustama. 

\question{Mulle on räägitud, et mõni oli keskkooli ajal juba kuskil tööl, sest 
kangesti oli vaja arvuti juurde pääseda. Kas sul ei olnud niisugust kihu?}

Ei, seda ei olnud. Ma ei ole kunagi arvutihull olnud, see on 
võibolla rohkem poiste teema. 

\question{Õige, mu valim on seni olnud väga kallutatud.} 

Aga Tartus oli küll nii, et asi hakkas tõsiselt meeldima ja sain aru, et see 
on õige, mida ma teen. 

\question{Millest sa aru said?}

Mulle lihtsalt meeldis! Ja see läks minu jaoks ajaga 
lihtsaks, kuigi ma ei saanud aru, kuidas ma saan 
aru. Kõlab väga filosoofiliselt ... Kuna see oli ikkagi rakendusmatemaatika, siis 
lisaks programmeerimisele oli palju teoreetilist matti. Vahel läheb matemaatiline analüüs nii abstraktseks, et sa 
tõesti enam ei saa aru, kuidas sa saad aru. Miskipärast tuli mul see jube hästi välja, kusjuures meid alustas 20 
ja pärast esimest aastat oli järel 14. 

\question{Väike grupp, aga väljalangevus ei olnudki nii suur, meie kursuselt 
läks rohkem. Aga mida sa mõtlesid
tegema hakata, kui kool läbi saab?}

Aeg oli ju selline, et valikuid loomulikult oli, aga enamik neist kuskil arvutuskeskuses. Tollal sai
kohti valida vastavalt sellele, kuidas lõpetasid ja milline oli 
pingerida. Hästi popp koht oli näiteks Raadiomaja 
arvutuskeskus\index{Raadiomaja Arvutuskeskus}, see oli number üks. 
Statistikaametisse ei tahtnud keegi minna.

\question{Miks?}

Ei tea, tagantjärele mõeldes on ju
statistikaametis töö palju huvitavam kui raadiomajas. Aga nii see 
kahjuks oli.

Minul oli selline õnnelik juhus, et sattusin Küberneetika 
Instituudi\index{Küberneetika Instituut} matemaatikaosakonda praktikale. Viiendal kursusel tegin 
diplomitööd, juhendaja oli Otto Vaarman\index[ppl]{Vaarman, 
Otto}, kes oli väga tunnustatud matemaatik. Uurisime Küberis Newtoni tüüpi meetodeid, mis kõige paremini 
lahendavad erinevaid võrrandeid. Otto oli vana meesteadlane, kes ise 
väga palju ei viitsinud teha, aga tal oli paar 
aastat varem lõpetanud noor jünger Maarika Lomp\index[ppl]{Lomp, Maarika}, kes oli 
mul sisuline juhendaja. Kollektiiv oli hästi tore ja töö 
huvitav. Toona tulid juba esimesed variandid, kus enam ei 
pidanud perfokaartide pealt toksima, vaid olid ikkagi 
Jessukese\index{Jessuke} taga ning said juba ise intelligentsel viisil oma 
koodi sisse viia.

Ja siis sattus mingi hetk täiesti ootamatult Otule\index[ppl]{Vaarman, Otto} ja 
Maarikale\index[ppl]{Lomp, Maarika} külla üks ääretult sümpaatne härrasmees 
sellisest huvitavast organisatsioonist nagu Algoritm. Tegelikult 
oli sellel asutusel hästi pikk nimi.\sidenote[][]{\phantomsection\label{sisu:algoritm} 1976. aastal asutatud Tallinna 
Teadus-Tootmiskeskus (TTTK), mis kuulus Üleliidulise Teadus-Tootmiserikoondise 
\enquote{Algoritm} koosseisu, Eestis tuntud lühendnime all Algoritm. Kuna 
asutus allus NSVLi tasemel kaitsetööstuse ministeeriumide gruppi, oli sellele
omistatud koodnimi Postkast A-3433 ja kehtestatud ka vastav tööre\v{z}iim. 
Üleliiduline alluvus seletab ka töökeelt. Asutus tegeles ES EVMi\index{ES EVM} 
hoolduse ja remondiga, IT-koolituse ja selle metoodikaga ning mitmesuguse 
tarkvara (nt matemaatika, arvutidiagnostika ja automatiseeritud 
projekteerimine) arendamise ja tira\v{z}eerimisega. Peale Tallinna olid 
asutusel filiaalid Tartus ja Kohtla-Järvel(!). 1980ndate keskel töötas 
Algoritmis ligi 1000 inimest, lisaks kaasati allhankijatena inimesi 
TPIst\index{Tallinna Tehnikaülikool} ja Tartust\index{Tartu Ülikool}. Algoritm 
lõpetas töö 1992. aastal.}. Küber oli Mustamäel. Kui sealt sõita edasi, kus on 
täna ARK, siis ühe risti peal olid koledad baraki tüüpi 
majad\sidenote{Toonase aadressiga Kadaka puiestee 165.}, kus asuski 
teadusuurimiskeskus Algoritm, millel oli eraldi matemaatikaosakond. 

Algoritm oli täielikult
venelaste sõjaline organisatsioon, kus tehti uurimistööd 
väga kummalistel aladel. Matemaatikaosakond oli ainuke eestlaste osakond ja Ants\sidenote{Ants Roose\index[ppl]{Roose, Ants}, kes hiljem töötas ka Algoritmi teadusala asedirektorina. 
Tegelikult oli osakonna nimi \enquote{matemaatika tarkvara 
osakond}. Rooselt on 
pärit kogu Algoritmi puudutav info.} selle
juhataja. Ühesõnaga tema rääkis mu ära ja läksin sinna tööle, ääretult tore kollektiiv oli. 

\question{Järelikult oli sul ikka akadeemiline siht silme ees?}

Jah, pidin tegelikult minema doktorantuuri Gennadi 
Vainiko\index[ppl]{Vainikko, Gennadi} juurde, aga kuidagipidi hakkasid asjad 
arenema ja huvitavaid teemasid tuli palju. 

Muide, Algoritmis hangiti tarkvara niimoodi, et 
keegi tundis näiteks Minskis kedagi, kes oli kuidagipidi saanud mingi softipaketi, ja meil oli seda vaja. Ülemus saatis 
minu, noore tüdruku, kes vene keelt väga hästi ei rääkinud, 
Minskisse ja ütles, et pean sellega tagasi tulema! Ja tulingi. 

\question{\emph{Millega} täpsemalt sa tagasi tulid?}

Suure lindikettaga! Nii me seda tarkvara hankisime ja testisime. Tegime tööd ja 
oli väga huvitav aeg.

Kui mul sai seal majas kolm aastat täis, hakkasid ajad muutuma. Ain 
Rasva\index[ppl]{Rasva, Ain} oli ka tollal seal ja tema soovitas mind 
ühele inimesele, kes juba toona tegi koostööd soomlastega. Läksingi aastal 1988 Tööstusprojekti\index{Tööstusprojekt}, kus olid juba tol ajal miniarvutid ja koostöö soomlastega. 

\question{Need on ju projekteerijad, mis seal programmeerida oli?}

Oi, palju! Ehitusprojekt koosneb paljudest asjadest, sealhulgas 
tugevusarvutustest.

\question{Sul oli võimalik minna akadeemilisse maailma või sealt ära. Miks sa just sellise valiku tegid?} 

Pakkumine oli tohutult ahvatlev -- võimalus teha tööd ka kuskil mujal, 
Soome projekte koos 
soomlastega. Ja ma sain õppida. Mulle anti C õpikud ja 
esimene proovitöö oli automaatsed konverterid, programmid, mis tõlkisid
Fortrani\index{Fortran} koodi C\index{C} koodiks. See kood, mis 
välja tuli, oli muidugi kohutav, aga töötas. 

1990. aastal olin natuke aega lapsega kodus. Tegin Soome väikseid töid, teenisin 
saja marga kaupa väga head raha. 

\question{Jõhker raha, sada marka!}

Jah! 1991. aasta talvel avaldati ajalehes 
kuulutus, et Rootsi-Eesti ühisfirma otsib programmeerijaid ja just naisterahvaid. Ma polnud elus niisugust asja näinud, aga tundus väga 
huvitav. Avaldusi oli üle kuuesaja. 

\question{Kuuesaja!? See tähendab, et 1991. aastal oli Eesti Vabariigis aktiivsel 
tööturul 600 naist, kes võisid enda kohta öelda \enquote{programmeerija}?}

Jah. Esimesel intervjuul selgus ka väga lihtne loogika, miks nad naisi 
otsisid. Firma taga oli üks rootslane, kes töötas Stockholmi linna ja 
lääni valitsuses (minu arust oli ta lausa IT osakonna juhataja), ja üks Eestist Rootsi läinud mees, keda mäletasin ülikooli ajast, 
Kalle Kullman\index[ppl]{Kullman, Kalle}. Neil tekkis idee, et kui Eesti saab 
vabaks, saab sealt odavalt head tööjõudu. Nad tegid firma, mis pidi hakkama 
Stockholmi lääni valitsusele teenust osutama, ja naisi otsiti sellepärast, et 
naised on korralikumad, leplikumad ja küsivad vähem raha. Meid see 
ei häirinud, sest, kujuta ette, saada tööle firmasse, mis maksab palka 
valuutas, ja teha Rootsi tööd! Fantastiline!

Välja valiti kolm naist: üks oli Maarika Lomp\index[ppl]{Lomp, 
Maarika}, teine mina ja keegi kolmas oli veel. Meile 
üüriti kontoriruumid vana ajakirjandusmaja taga. Pidime ise panema 
püsti kohtvõrgud ja kõik muu! 

\question{Kas tollal oli selline asi nagu kohtvõrk?}

Jah! Ise panime püsti. Suhtlesime üle modemite Rootsiga ja kõik toimis. Muidugi kasutasime tuttavate meesterahvaste abi, kes olid juba võibolla rohkem võrgu ja selle poole peal, aga saime hakkama. 

Programmeerisime sellises huvitavas keeles nagu Magic\index{Magic}. Oled 
kuulnud? 

\question{MUMPSist olen, aga Magicust mitte.}

See oli juba toona 4GL, Iisraeli päritolu.\sidenote{Platvormi tootnud 
Magic Software Enterprises oli tõesti asutatud 1983. aastal Iisraelis.} Üldse 
oli kasutada vist kaheksa erinevat käsku, neile panid 
parameetrid taha ja nendest koodi kokku. Koodi sai täita vastavalt 
vajadusele kas eest tahapoole või tagant ettepoole. Täiesti müstiline asi! Ja 
sellega sai teha kõike. Esimese projektina tegime nende varade ehk 
autopargi haldusprogrammi, kus olid kõik asjad alates muruniitjatest ja lõpetades 
autodega. 

\question{Kui sind kuulan, kerkib esile huvitav kontrast. See vahend, 
millega te tegite, kõlab palju keerulisem kui tänapäeval 
kasutatavad, aga ülesanne, mida lahendasite, palju 
lihtsam, kui tänapäeval tavaliselt lahendatakse. Kas teile ei tundunud see tegevus kahuriga 
kärbse tapmisena?}

Ma ei oska sulle öelda, miks nad valisid sellise. Igatahes ei kaldunud asi mitte mingil juhul FoxPro, 
Paradoksi või mõne muu sellise asja poole, mis toona juba olemas olid. 
Võibolla see oli see natuke päritoluga seotud, kuna Kalle on juut ja tal 
olid Iisraeliga tihedad sidemed. Ja kuna ta ise oli hästi kõva matemaatik ja 
keeruliste ülesannete lahendaja, siis talle ilmselt see 
süsteem sümpatiseeris. 

See oli tore aeg. Kasutajaliidesed olid rootsikeelsed ja mäletan siiamaani mingit 
sõnavara. Saime palka 800 Rootsi krooni kuus, mis oli 
tollal siin suur raha. Kui meid viidi restorani kliendiga 
kohtuma, siis meie igaühe restoraniarve oli suurem kui kuupalk. 
Muidugi tegime pikki päevi ja töö oli päris karm. 

\question{Huvitav, et selline programmeerijaamet oli olemas. Näiteks 
Henn Ruukel\index[ppl]{Ruukel, Henn} rääkis, et tema mäletamist mööda enamasti 
inimesed ei tegelenud igapäevatööna programmeerimisega. Leiva tõi lauale 
ikka kaabli vedamine või arvutite kokkupanek.}

No vot, meie tegelesime! Tegime algusest lõpuni: mõtlesime välja, disainisime 
mudelid, programmeerisime ja ka juurutasime Rootsis kliendi juures. 

\question{Kui teil olid arvuti ja modem\sidenote{Ja pidi olema ka võimalus 
välismaa numbritele helistada!}, kas teil ei tekkinud huvi, mida nendega 
veel teha saab?}

Mäletan päevi, mil alustasin 
kell 8 ja lõpetasin kell 12 öösel. Huvisid võis hästi palju 
olla, aga lihtsalt ei jõudnud nendeni. Tegelikult tol ajal tärkas mul huvi 
andmebaaside vastu, sest meile anti üks raamat (\enquote{Lugege läbi, tüdrukud!}), mis minu jaoks esimest korda kirjeldas relatsiooniliste andmebaaside 
teooriat. 

Rootslaste juures olin umbes aasta, siis nägin 
Hansapanga\index{Hansapank} kuulutust. Nad ei otsinud üldse IT inimesi, ma isegi täpselt ei mäleta, keda. 
Lõpetasin parasjagu kaheteisttunnist tööpäeva ja ütlesin Maarikale, et okei, 
ma kirjutan. Saatsin CV, see oli 1992. aasta novembris. Järgmisel päeval 
helistas mulle Tõnis Sildmäe\index[ppl]{Sildmäe, Tõnis} ja kutsus intervjuule. 

Mäletan, et kui ma seal vestlesin (Liivi\index[ppl]{Kompus, 
Liivi}\sidenote{Liivi Kompus, üks Hansapanga legendaarseid IT inimesi.} oli 
ka), rääkisin väga uhkelt oma Rootsi kogemusest ja teadsin juba 
relatsioonilistest andmebaasidest ning jätsin ikka tohutult muljet. Jüri 
Mõis\index[ppl]{Mõis, Jüri} läks vahepeal mööda ja ütles: \enquote{Tõnis, kui sina 
ei taha, ma võtan ise selle tüdruku!} Ja nii mind tööle võeti. 

\question{Kui suur Hansapank\index{Hansapank} toona oli?}

Umbes 40 inimest. ITs olin mina kolmeteistkümnes. See oli tol ajal veel
Crebit\index{Crebit}, tegelikult Spin Development\index{Spin 
Development|see{Crebit}}, mis oli pangast eraldi. 

\question{Kui organisatsioonis on kokku 40 inimest ja neist 13 on IT inimesed, 
siis see on ju päris suur protsent!} 

Pank oli tõesti väga väike, sest toona valdasid inimesed 
hästi laia teemaderingi -- alates klienditeenindusest kuni raamatupidamiseni 
olid samad inimesed. Sealt kasvas välja näiteks Agve 
Aasmaa\index[ppl]{Aasmaa, Agve}, kes tuli tööle telleriks, samuti Tea 
Trahov\index[ppl]{Trahov, Tea}.\sidenote{Mõlemad legendaarsed 
hansapankurid.} Kokku oli meid aga jah vähe. Kui 
olid tähistamised, siis mahtusime väiksesse ruumi ära. 

\question{Kust see suhteliselt suur IT osakaal ikkagi tuli? Praegu ei 
ole ju veerand Swedbanka IT.} 

See oli imetlusväärne visioon, millega omal ajal Hansapanka\index{Hansapank} 
tehti! Taheti teha midagi tõeliselt \emph{cool}'i. Tegelikult need mehed tegid täielikku \emph{start-up}'i. 
Nad tahtsid teha täiesti teistsugust panka, kus paberkataloogide asemel oli 
arvuti. 

\question{Kust neil tekkis arusaam, et sedasi on üldse võimalik?} 

Ma ei ole kunagi küsinud, aga arvan, et see tuli koostöös ja nad olid sõpruskond. 
Hästi palju oli ilmselt Tõnis Sildmäe\index[ppl]{Sildmäe, Tõnis} panust, kes
müüs seda ideed, et nii saab teha. Näiteks kust tuli mõte panna kõik kontorid juba sel ajal \emph{online}'i? 
Mõned aastad hiljem, kui käisime Inglismaal ja Iirimaal kohtumas nende 
suurte ja vanade pankadega, siis need imestasid: \enquote{Issand, lapsed, mis 
te räägite! Teil on mingi Paradoxi lahendus ja see töötab \emph{online}'is?}

See oli julgete mõtete maailm, samas ei tehtud 
lollusi. Keegi ei teadnud, keegi ei osanud, aga kogu aeg õpiti. Mõeldi, 
kuidas me nüüd sellest üle saame? Kuidas me hakkame oskama? Keegi saadeti
kuskile midagi õppima \dots 

Minu esimene ülesanne oli see, et 
Tõnis\index[ppl]{Sildmäe, Tõnis} ütles: \enquote{Näed, mul on siin neli 
programmeerijat kirjutanud. Kes on teinud laenu, kes kontosid. Vaata 
ja ütle, mis võiks olla teistmoodi.} Vaatasin ja stiili 
järgi oli kohe näha, et see on Liivi Kompus, see on Kadri Trahov, see Tõnis 
Argus -- igaühel oli täiesti oma stiil. Andmebaas oli selline, nagu oli, aga kõik
töötas. Kirjutasin ettepanekud ja loomulikult ei olnud need 
selle tehnoloogia peal teostatavad, aga sealt sai alguse mõte, et viime oma 
süsteemi Oracle peale -- teeme uue pangasüsteemi ja viime asja järgmisele 
tasemele.

Kogu eduloo \emph{point} oli ääretult avatud 
suhtlus, kõik soovisid midagi ära teha. 

\question{Mis seda takistas nurjumast? Kui hakata niisama nullist 
kõrgtehnoloogilist panka tegema, siis see võib ju vussi minna.}

Tead, mul on alati olnud ja on siiamaani usk, et inimesed on \emph{key}. Inimesed, kellega sa 
mingit asja teed, isegi kui asi võib minna ka totaalselt 
tuksi. Üks pool on muidugi oskused, aga isegi rohkem 
määrab ära suhtumine ja ambitsioon. 

\question{Milline suhtumine olema peaks?}

See, et tahan midagi ära teha, aga samas tean, et ma ei tee seda üksi, 
vaid me teeme koos. Oluline on ka arusaam, miks ma midagi teen. Palju on selliseid initsiatiive, et on 
hästi huvitav tehnoloogia, aga kas see lahendab mõnd probleemi, selle peale liiga palju ei mõelda. Tollast aega iseloomustabki koos ärategemine. Mida 
kindlasti ei olnud, oli see, et \enquote{kes on kõvem}.

Inimestel, kes seal toona töötasid, oli hästi suur ambitsioon --
kindlasti mitte isiklik karjäär, vaid meeskondlik ambitsioon ja 
saavutusvajadus. Tänapäeval räägitakse palju ja räägiti ka Swedbankis 
aastal 2000, kui rootslased tulid, protsessidest ja kvaliteedijuhtimisest. Peab 
olema tohutu hulk dokumente ja siis ma teen plaane, raporteerin ja 
kogu jõud lähebki selle peale. Hansapangas seda ei olnud, me ei mõelnud niimoodi. Näiteks kui pank tahtis 
välja tulla eraisiku pangakontoga, kutsus Jüri Mõis\index[ppl]{Mõis, Jüri} 
ühte ruumi kokku kõik, kellel võiks sellega mingit pistmist või 
arvamust olla. Tema rääkis, miks seda teha, ja meie mõtlesime, mis selleks tegema 
peab. Edasi hakkaski igaüks oma osa tegema. Tehti koos ja 
hästi ruttu. 

\question{Magicu moodi asjadega tegelemine annab 
ilmselt päris hea immuunsuse tehnoloogia järel jooksmise vastu. Kui oled korra 
pidanud tagurpidi käivat programmi kirjutama, siis ei ole miski enam väga 
uudne!}

Oluline on kasutada õiget asja õiges kohas. Populistlik jooksmine 
mingi asja järel\ldots{ }See on põhjus, miks ma ei poolda näiteks kunagist riiklikku initsiatiivi, 
et kõik programmid tuleb iga 13 aasta järel ümber 
kirjutada. \emph{Sorry}, aga võibolla ei peaks neid 13 aasta tagant ümber 
kirjutama, kui kogu aeg teeks nendega midagi? Aitaks järele ja muudaks? See 
oli see, mida me pangas tegime. Vaatasime täpselt seda, kust meil tulevikus 
pigistama hakkab, ja muutsime ning vahetasime välja. See oli pidev protsess. 

\question{Ometi pidid ka pangas uute tehnoloogiate lained tulema. Kuidas te otsustasite, mida üles korjata?}

Me tegime ka valeotsuseid, keegi ei ole ju selle suhtes immuunne. 
Kui tehnoloogia poolelt rääkida, siis esimene laine oli see, et meil oli 
Paradox\index{Paradox}, mis oli failipõhine süsteem ja millest me kindlasti 
nägime, et see hakkab meil takistuseks saama. Kas või näiteks see, et me ei 
suutnud olla 24h kättesaadavad, või päeva vahetuse teema.\sidenote[][]{Päeva vahetus on 
pangas oluline ja keeruline ning seetõttu arvutuslikus mõttes kaua aega 
võttev toiming, mille käigus tehakse konkreetse päeva seisuga raamatupidamiskanded, 
toimetatakse arveldused, esitatakse aruanded ja, lühidalt öeldes, üks pangapäev 
asendub teisega.} 

Minul oli relatsiooniliste baaside teoreetiline 
teadmine ja Tarmo Pajumets\index[ppl]{Pajumets, Tarmo} oli töötanud pool 
aastat Soomes ning arendanud Oracle-põhiseid süsteeme. Oracle\index{Oracle} oli 
sel ajal Eestis ikkagi number üks, seda kasutasid näiteks ka Elion ja EMT. Üks põhjus, miks 
Oracle nii tugevalt Eesti turule tuli, oli see, et meie tugi oli Soomes 
ja Soome Oracle oli väga tugev organisatsioon. Kui alustasin uue süsteemi 
arendamist, siis mul oli otseliin Soome ja tugi telefoni otsas -- võisin helistada Soome ja küsida, miks see või teine asi ei tööta. Võrdle sellega, mis on praegu!

Ja nii me konvertisimegi oma Paradoxi 
Oracle peale. Oli olemas automaatse konverteerimise võimalus, 
et kasutajaliides jäi endiselt Paradoxi ja baas taga oli Oracle. See andis 
meile vabaduse teha oma päevavahetuse protsessid kõik 
niimoodi, et see oli rohkem 24h. Saime hakata sinna külge kaarte panema ja 
tulevikus ka näiteks Telehansa\index{Telehansa}. Ja hakkasime Paradoxi rakendust ennast ümber 
kirjutama. 

\question{Kas Telehansa tuli enne kui Forexi modemipank või hiljem?}

Minu meelest umbes samal ajal.

\question{Mast, kes Forexi asja kirjutas, rääkis, et tolle 
käivitamis{\-}üritusel istunud Hansapanga tütarlapsed esireas ja teinud ohtralt 
märkmeid!}

Võis nii olla, see tuli enam-vähem sinna otsa, palju 
vahet ei olnud. 

\question{Miks te Telehansa\index{Telehansa} tegite?}

See algas sellest, et olid küll kontorid, aga firmade raamatupidajad ei 
tahtnud oma maksetega kontorisse tulla. Meil 
oli palju firmasid. Kui vaadata Hansapanga arengut, siis ta 
kõigepealt võttis ju sellised eesrindlikumad firmad ja seejärel tulid eraisikud 
pangakontodega järele. 

Ühelt poolt tahtsime elu mugavamaks teha ja see oli ka
rahaliselt kasulik. Kontori koormust vähendades hoiad kokku. Teine oluline asi oli
innovatsioon -- oleme teistmoodi pank, 
teeme asju teistmoodi. 

\question{Kui palju Telehansa tuumsüsteemi muutust eeldas? See ju vajab hoopis 
teistsugust interaktsiooni tuumaga.} 

Tegelikult ei olnud Telehansa tegemine kuigi keeruline, selle võimaluse lõi
andmebaasi vahetus. 

\question{Kas ajaliselt tekkis see umbes samal ajal?}

Jah. Andmebaasi vahetus oli vist 1994. aastal ja Telehansa tuli ka siis. See
oli väga kõva asi.

\question{Telehansa on siiamaani väga kõva asi!}

Kolm meest -- Toomas Lassmann\index[ppl]{Lassmann, 
Toomas}, Madis Tapupere\index[ppl]{Tapupere, Madis} ja Riho-Rene 
Ellermaa\index[ppl]{Ellermaa, Riho-Rene} -- selle tegid ja kirjutasid. 

\question{Kas tolleks hetkeks oli IT inimesi juba rohkem kui neliteist?} 

Jah. Toomas Rand ja mina kirjutasime töötlust, et kui 
maksed sisse tulid, siis mis nendega sai. Tiim kasvas väga kiiresti. Neliteist oli 
1992. aasta alguses ja see arv kahekordistus umbes 
aasta või pooleteisega. 

\question{Kuidas te kasvu kontrolli all hoidsite? Kui nii kähku kasvad, 
siis on ka suur tõenäosus mõni loll kogemata palgale võtta.}

Mäletan selgelt, et kui olime kasvanud kuskil 50 inimeseni, siis tegime 
oma esimesed kompromissid. Enne ikka väga valisime inimesi. Hansal kui Eesti majanduse lipulaeval oli võimalik valida! 
Esiteks, et nad oleksid professionaalsed tipptegijad. Teiseks, et nad inimestena sobiksid väga 
hästi tiimi. Aga siis tegime jah esimesed kompromissid\ldots

Hakkasid tekkima esimesed probleemid ja küsimused, kuidas asja hallata. Tekkis \emph{learning by doing} kogemus. Tänapäeval on ilmselt väga vähestel olnud 
võimalus kasvada koos organisatsiooniga väikesest suureks ja sealjuures väga kiiresti. 

Näiteks üks teema oli see, et kui keegi oli arendanud 
laenu- või väärtpaberisüsteemi, siis uue 
funktsionaalsuse tegemiseks tal enam aega ei jätkunud, kuna tal tuli teiselt poolt kogu aeg nii palju 
probleeme peale, mida pidi lahendama. Kes vajas 
raportit, kellel oli mõni \emph{case}, mis ei mahtunud sisse, 
ja seesama inimene tegeles mõlemaga. Siis tegimegi halduse poolel rakenduste halduse osakonna, kus olid 
inimesed, kes oskasid lihtsamaid probleeme lahendada, raporteid genereerida 
ja tundsid andmebaasi andmeid -- ühesõnaga olid arendaja 
kõrval, et arendajal oleks rohkem aega. 

\question{Nii et see otsus sündis praktilisest vajadusest, mitte te ei olnud kuulnud, et peaks olema 
rakendusadministraatorid? Teil oli vaja konkreetset asja teha, leidsite
inimesed, koolitasite neid ja panitegi tegema.}

Seal organisatsioonis sündis kõik praktilisest vajadusest. 

Mul on tohutu austus kadunud Tõnis Sildmäe\index[ppl]{Sildmäe, Tõnis} kui juhi 
vastu. Kuidas ta seda tiimi juhtis! Juhtkonna moodustasid inimesed, kes suutsid 
vedada teatud teemasid või olid juhipotentsiaaliga spetsialistid. Ta 
usaldas meid täielikult. Ta ei tulnud kunagi ütlema, mida keegi peab tegema. 
Ainuke negatiivne asi oli võibolla see, et ta kaitses liiga palju: kui 
keegi kallale tuli, siis läks ta kohe võitlusse ja teda pidi tagasi 
hoidma. Aga see lõigi kultuuri, et probleemi tekkides istusime koos
ja arutasime, kuidas seda oleks mõistlik lahendada. 

\question{Sinu jutust koorub põhiliselt välja 
inimeste ja juhtimise olulisus. Väga vähe on juttu sellest, et Oracle indekseid peaks 
tegema nii- või naamoodi.} 

Indeksite tegemine on lihtsalt tehnika. Loomulikult peavad olema 
oskused ja teadmised, aga see on õpitav. See, kui 
hästi ma indekseid teen või kui ilusat koodi kirjutan (kood peab ilus 
olema!), ei ole see, mis toob edu. 

\question{Milline on ilus kood?}

See kõlab nüüd hästi populistlikult, aga ilus on kood, millest saab aru. Kui ma isegi ei valda täielikult programmeerimiskeelt või 
tehnoloogiat, milles see on kirjutatud, siis vaatan koodile peale ja saan aru. 
Loomulikult, kui ei ole üldse kogemust sellel alal, siis 
võibolla ei saa aru. Aga kui oskan Cd kirjutada, siis vaatan Java koodile 
peale ja suudan seda lugeda. 

\question{Järelikult sõltub ilus kood sellest, keda sa 
arvad seda hiljem lugevat -- esimese kursuse tudengit või 20 aastat 
progenud inimest?} 

Ka esimese kursuse tudeng võiks aru saada!

Mul on just andmebaaside poolelt -- PL/SQL või PostgreSQL \mbox{baasi} protseduurid -- palju kogemust. Omal ajal vaatasin nelja inimese 
koodile peale siinsamas majas\sidenote{Meie jutuajamine toimus Tallinnas toonases Icefire 
kontoris aadressil Kauba 2a.} ja võin öelda, et kaks neist kirjutas väga ilusat koodi, kaks mitte. 
Kood töötas perfektselt ning kõik neli on tehniliselt väga head ja tugevad tegijad. 

\question{Aga mõnel on ilus kood ja mõnel ei ole. Kunst?}

Jah, see ongi kunst. See hakkab peale sellest, kuidas ma mõtlen ja 
oskan maailma abstraktselt kujutada. 

\question{See haakub Ahti jutuga, kuidas tema 
õppiski programmeerima just nimelt paberil ja programmist mõeldes. Temagi rõhutas võimekust asja oma peas ette kujutada.} 

Jah. Teine pool on see, et vahel kui lahendus liiga ära abstraheerida, tulevad sellised 
maailmamudelid, mis üldiselt eriti ei toeta \ldots 

\question{Kuidas seda tasakaalu hoida, et oleks piisavalt üldine ja ei paneks arendust kinni, 
aga ei üritaks ka maailma mudeldada?} 

Ilmselt see on 
mõtteviisis kinni ja tuleb loomulikult, kogemusega. Igaühel ei tulegi. 
Võibolla on asi analüütilises mõtteviisis \ldots 

\question{Sinu jutust jääb kõlama, et kui on 
hästi kokku pandud tiim, siis see tiim jõuab tasakaaluni loomulikult 
oma kogemuste, oskuste ja parasjagu käsil oleva ülesande pealt.} 

Just. Muidugi tiimitöö. Inimeste koosmõtlemist ei ole võimalik üle väärtustada, sest koos välja mõeldud väärtus on tükk maad 
suurem. Meil on väga hea näide -- Jan\sidenote{Vilve peab silmas legendaarset hansapankurit Jan 
Laksperet\index[ppl]{Lakspere, Jan}, kellega nad on 
intervjuu ajaks koos töötanud üle kahekümne aasta.} \emph{versus} mina. Jan on 
perfektsete lahenduste mees: tema lahendused katavad üldiselt ära kõik asjad, sealhulgas
ääre-\emph{case}'id, mis lõpptulemusena võib tekitada olukorra, et lahendus 
läheb liiga keeruliseks. Mina jällegi suudan läheneda selle poole pealt, kuidas 
oleks mõistlik -- me töötame koos jube hästi. 

\question{Tuleme korraks tagasi Hansapanga aja juurde. Kui Tõnis\index[ppl]{Sildmäe, 
Tõnis} juhtkonda kokku pani, oli see ju ka sinu jaoks valikukoht, kas 
hakata inimesi juhtima või jääda koodi kirjutama. Tihti öeldakse, et kui 
heast programmeerijast juht teha, saad omale kehva juhi ja heast 
programmeerijast lahti. Kas sul seda hirmu ei olnud?}

See oli balansseeritud protsess, see ei juhtunud ühe päevaga. 
Kirjutasin ju Hansas peaaegu lõpuni ka koodi. 

\question{Kuidas sa seda tasakaalu hoidsid? Inimesi juhtides on lihtne
sinna sisse ära uppuda, nii et ühel hetkel ei kirjuta 
enam üldse koodi ja minetad selle oskuse.} 

Praegu ma näiteks ei kirjuta.

\question{Aga tahaksid?}

Nojah, vahel on kurb ka, et ei saa. See on 
olnud nii viimased kolm aastat, lihtsalt juhtus sedasi. Aga ma annan endale aru, et keegi pidi 
selle ülesande võtma. Ma ei ütle, et oleksin selle pärast õnnetu, 
kuid vahel ikka kriibib natuke.

\question{Miks sa ikkagi otsustasid juhi rolli ka juurde võtta?} 

Võibolla oli mul iga asja kohta midagi öelda? Ja siis sind karistatakse selle eest. 

\question{Mäletades seda, kuidas Hansapank oli seesmiselt üles ehitatud, ja 
keskset Oracle baasi, kus kõik maailmaasjad sees olid, siis ühel hetkel läks
see süsteem ikkagi tehnoloogiliselt laiaks. Kuidas seda kontrolli 
all hoiti?} 

Alguses oli kõik väga lihtne ja koodi baasi kirjutamine oli õige otsus. Kood oli ilusti ja loogiliselt
struktureeritud, midagi ei tohtinud segamini ajada. Siis aga tulid igasugused tehnoloogilised \emph{switch}'id. Javat veel ei olnud 
nii palju, osaliselt kirjutati Cs. 

Kui tuli esimene internetipank, siis 
uurisime, kuidas seda teha, sest tehnoloogia ei olnud veel päris sealmaal. Ostsime sisse BroadVisioni platvormi.\sidenote{Varajane internetitehnoloogia, mis lubas igale kasutajale täielikult 
personaliseeritud kogemust. Paraku selgus, et kõigi nende kogemuste pakkumise 
vältimatuks eelduseks on nende väljamõtlemine. Küll aga võimaldas BroadVision 
üsna mõistlikult kombineerida HTMLi ja Cs\index{C} kirjutatud komponente 
(hiljem ka serveripoolset JavaScripti\index{JavaScript}.) ja sellest 
internetipanga, tellerirakenduse jms ehitamiseks juba piisas.} Kui oled kuskil liiga vara, siis teed otsuseid, mis 
ilmselt ei ole jätkusuutlikud, sest uus teletehnoloogia tuleb peale. 

\question{Kui mõelda, kuidas Sergei 
Anikin\index[ppl]{Anikin, Sergei} kirjutas Light Tellerit\sidenote{Lähemalt loe 
Sergei loost lk \pageref{sisu:teller}.}, siis see ongi see, kuidas praegu 
tehakse. Tehnoloogia võis ju jätkusuutmatu olla, aga lahendus oli vägagi 
jätkusuutlik!} 

Tegelikult meil juba tol ajal andmebaas pakkus teenuseid, loogika ei olnud 
segamini kliendis ja baasis. 

\question{Kes tegi otsuse niimoodi teha?} 

Meie tegime. Asi algas sellest, et mina ja Ruta Joost\index[ppl]{Joost, Ruta} 
panime esimesena need mustrid paika, kuna see tundus ainuvõimalik viis. 

\question{See on päris hea viis arhitektuuri teha, et võtad selle, mis tundub 
ainukesena võimalik!} 

Teatud asju ei saa põhjendada, kuidas need peas sünnivad. Ilmselt on see analüütiline mõtlemine, et vaatad, mida see tähendab, 
kui teen nii või naa. Mul oli kohe see mõte, et kui kirjutan 
mingi loogika klienti, siis ma ei saa seda ju korduvalt kasutada. Järelikult ma 
ei tee seda. Kirjutan nii vähe kui võimalik. Ja nii oligi. 
Sergei\index[ppl]{Anikin, Sergei} sai kiiresti teha Light Telleri, sest tal 
olid kõik teenused olemas. 

\question{Hansapank läks ju ka horisontaalselt suureks -- Markets,
kindlustused ja muu. Kuidas te seda pusa kontrolli all 
hoidsite, et keegi lollusi ei teeks?} 

See oligi hästi keeruline ja ega me ei kontrollinudki kõike lõpuni. 
Marketsil\index{Hansapank!Markets} oli väga suur eripära ja neil oli oma IT. 
Seal oli palju asju Excelis, lisaks analüütika, mis nad sealt pealt tegid. Neil olid sisseostetud lahendused, näiteks Condor, sest keegi ei hakanud 
\emph{trading}-lahendust ise tegema. Ja eks tehti
ka valesid otsuseid, näiteks kui osteti Marketsi platvorm, mis 
osutus liiga tooreks ja keeruliseks. Marketsit me tõesti keskselt palju 
ei hallanud, ainult nii palju, kui see haakus meie pakutavate 
süsteemidega. 

\question{Keegi pidi ju tegema otsuse, et \enquote{las nad toimetavad 
omaette}?} 

Just sel ajal, kui organisatsioon laienes, tekkisid meil ITs kindla
valdkonnaga tegelevad inimesed, kes meie poolelt olid sellel konkreetsel 
valdkonnal vastas. Paljuski oli tegu valdkonnajuhi ja Marketsi IT 
vahelise kompromissi, kokkuleppe ja ühise otsusega. 

\question{Kas põhimõtteliselt tõmmati neile organisatoorne kast ümber ja 
lasti neil seal sees vabalt toimetada?}

Jah, aga väljapoole kasti nad ei saanud, välismaailmaga suhtlemiseks olid väga 
selged liidesed. 

Teine näide oli Hansa Capital\index{Hansapank!Hansa Capital}, mis 
kasvas väga kiiresti, oli väga isepäine ja efektiivne ning äriliselt tegi superhead tulemust. Kuni üks hetk tuldi meie juurde ja öeldi: \enquote{Kuulge, 
meil läks Excel katki, tehke midagi!} Me siis vaatasime sellele loomaaiale 
peale, see oli peaaegu sama suur kui pank oma äridega! Panime projektitiimi kokku ja hakkasime tegema. 

\question{See haakub sellega, et üks asi on vedada inimesi, kes on 
\emph{hand picked}, aga teine asi on olla 
organisatsioonis kõrge taseme juht. Järelikult pidid sa aru saama, 
kuidas inimesed töötavad, nendega suhtlema ja panema neid vajalikke asju tegema. Kuidas sul see oskus tuli?}

Õppisin! Loomulikult oli ka palju koolitusi, mis tulid kindlasti kasuks ja panid 
teatud asjade peale mõtlema, aga tegelikult õpitakse läbi tegemise. 

Juhtimine on keeruline ja sellega toimetulemise määravad paljuski isikuomadused. Teine asi on kindlasti kogemus -- arvan, et 
olen praegu palju parem, kui ma olin siis, aga kui ma ei oleks seda 
protsessi läbi teinud, siis ma ei oleks seal, kus ma olen. Kõige keerulisem on 
hakkama saada inimestega, kes kõik sinult midagi tahavad, ja sa tead, et ei saa kõigile jah öelda. Kuidas 
teha seda niimoodi, et sünniksid õiged valikud, mitte anda
tähelepanu sellele, kes kõige kõvemini karjub. 

Ja vahel ei olegi keegi sinuga rahul. Sa pead sellega 
hakkama saama, et kõik sind ei armasta. Inimestega tuleb kindlasti rääkida, ega muu ei aita!

\question{Nii et kuskil sügaval peab lisaks arvutihuvile olema huvi inimese vastu?} 

Jah, muudmoodi ei saa! Huvi inimese vastu peab olema 
võibolla isegi natuke suurem kui huvi arvuti vastu! 

\question{Ometi läksid sa õppima rakendusmatemaatikat, mitte 
psühholoogiat. Kust sul huvi inimese vastu tuli?}

Mul on see huvi olnud vist kogu aeg, kui nüüd mõelda. Nagu ma rääkisin, siis valisin tegelikult kirjanduse ja 
matemaatika vahel. 

\question{Kui palju te suhtlesite tol ajal pangavälise kogukonnaga? 
Kuskil pulbitses BBSide kamp, toimis mingi võrgustik. Kui palju te selles osalesite?}

Meil oli inimesi, kes seal suhtlesid, aga need olid kindlasti tehnilisemad inimesed,
nagu Toomas Lasmann\index[ppl]{Lassmann, Toomas}. Meie mitte nii väga. Pigem suhtlesime natuke 
kõrgemal tasemel: kes kuhu liigub ja kes milliseid tehnoloogilisi valikuid teeb.

\question{Üks asi, mida mina tollest ajast igatsen, on see, kuidas sündis 
iPizza\sidenote{Hiljem tuntud kui Pangalink. Minu mäletamist mööda sai see kohapeal 
välja mõeldud, teised ütlevad, et tegu oli Soome lahenduse ülevõtmisega. 
Igatahes võimaldas see (lisaks algselt ideeks olnud maksete lahendusele 
eesmärgiga internetis pitsat tellida -- sealt ka nimi) anda panka sisse loginud 
inimese identiteeti edasi teistele osapooltele. ID-kaart ei olnud veel levinud, 
keegi maksuameti paroolikaarti endale ei võtnud, aga maksuametil oli vaja saada 
inimesed internetis tulu deklareerima. Nii sündiski koostöö, sest pangal oli 
vaja anda inimestele hea põhjus nende internetipanka kasutada.}. Kogu protsess 
hetkest, kui astuti uksest sisse, et nüüd hakkame tegema, kuni 
hetkeni, kui maksuametisse sai sisse logida, võttis aega kolm 
nädalat. Praegu võtaks isegi lepingu läbirääkimine tõenäoliselt kauem.} 

Me oleme jõudnud sinnasamasse, kus on kogu vana maailm. Tegelikult on 
see kurb, aga kõik algab inimestest. Miks on lepingu 
läbirääkimine nii pikk? Me oleme ise teinud endale kõik need protseduurid ja 
poliitikad. Meie ise, mitte keegi teine. 

On olemas selline organisatsioon nagu Nordic Finance Innovation Forum, mille eesmärk 
on panna Põhjamaade pankades, kus miski ei liigu (täpselt sama asi, neli 
kuud räägitakse lepingut läbi), mingilgi viisil liikuma innovatsioon. Panna nad 
omavahel koostööd tegema. Ka meie osaleme selles, sest see on huvitav. Nad korraldavad 
päevaseid seminare mõnes Põhjamaade pealinnas, kus 
erinevad inimesed räägivad erinevatest asjadest, mis on tehtud. 

Ükskord rääkis üks kutt väikesest 
Soome firmast, 15 inimest, kuidas nad tegid AliPay integratsiooni Soome 
maksesüsteemiga. Hiinlasena saad praegu minna Helsingis igasse poodi ja 
maksta oma AliPayga. Kujutad ette! Projekt sündis sellest, et inimesed 
ütlesid: \enquote{Nelja kuu pärast, novembri lõpus, lendab siia 
Rovaniemisse mitu suurt lennukitäit, palju-palju 
hiinlasi ja neil kõigil on vaja sellega maksta.} Ideest 
\emph{live}'ini kulus neli kuud, nad tegid selle ära! See kutt ütles: \enquote{Üldiselt on 
meil Soomes nüüd niimoodi, et selleks kulub neli kuud, et saada esimene kohtumine mõnes Soome pangas, et oma 
ideest rääkida.} Aga see projekt tehti nelja kuuga ära!

\question{Ometigi ei olnud Hansa tol ajal enam tilluke organisatsioon?!}

Aga toimis ikkagi! 

\question{Kuidas sai niimoodi, et läbirääkimised ei võtnud neli kuud?}

Selle asja nimi on kultuur! Kultuuri loovad inimesed, see ei sünni mitte 
millestki muust. Ja kui nüüd on organisatsioon, kelle juhil on ainult üks omaenda isiklik ambitsioon 
olla suur juht (ambitsioon rääkida ümber kõike, mida ta on raamatutest lugenud, 
hoolimata sellest, mis tegelikult tehtud saab), siis sünnibki tema ümber 
samasugune kultuur. Eesti häda täna ongi paljuski see. Mitte ainult 
riigis, vaid ka erasektoris. 

\question{Ja Hansas oli juhtkond, kellelt tuli teistsugune kultuur!}

Kui mõelda, mis seal hiljem toimus, siis kõik need inimesed, kes selle 
kultuuri olid ehitanud, ju lahkusid. Mingi põhjus pidi olema, palk ju kehv ei olnud. See on väga kurb.

Olen siin umbes aasta ühele pangale rääkinud, et 
kallikesed, kui tahate, et asjad hakkaksid liikuma, siis peate seda 
vana kultuuri kandvad inimesed lihtsalt ära saatma. Te ei saa enne üle ega 
ümber, kui te ei julge teha seda otsust. Inimesed ei julge tihtipeale 
otsustada, sest need on rasked otsused. 

\question{Kogu see teekond on sind kuhugi toonud. 
Millega sa praegu tegeled?} 

Viimased 16 aastat oleme ehitanud Icefire't. Me kogu aeg muutume ja areneme. Mõtleme, kuidas 
maailm muutub ja kes me võiksime olla ning kuidas sinna jõuda. Siin tuleb mängu jällegi seesama kultuuriküsimus -- me hoiame oma 
kultuuri, mida meie inimesed kannavad. Ja see võib olla ka põhjus, miks 
me ei ole näiteks läinud seda teed, et ostame ettevõtteid kokku, 
lihtsalt kasvatame väärtust, ajame asja suureks ja lõpuks saame väga rikkaks. Oleme 
pigem hoidnud organisatsiooni hoomatavana. Ja täna läheme platvormiärisse, mis tegelikult muudab täiesti seda paradigmat, kus me tegutseme. 

\question{Ja sina oled selle asja juht ja koodi ei kirjuta?} 

Praegu juba kolm aastat ei kirjuta. Mul on fantastiline tiim. Oleme viimase kolme aastaga teinud suurepärase tulemuse pärast seda, kui vahepeal augukeses 
käisime! See, kuidas oleme suutnud ennast kasvatada samal ajal efektiivsust säilitades, näitab, et teeme midagi väga õigesti. Kummalisel kombel olen viimasel aastal märganud, et see külvab 
ümberringi kadedust mingites inimestes, kellega olen kunagi koos 
asju teinud. Miks me tunneme kadedust? Mina tunnen küll 
rõõmu, kui teisel hästi läheb! Aga eks me peame sellega elama. 

\chapter{Anne Villems}
\index[ppl]{Villems, Anne}
Kõneleja 4
Tere, see siin on memm kopi. Meie tänase külalise üle on mul kohutavalt hea meel. Ja mitte ainult seetõttu, et tema sisuline panus Eesti IT-maailma on sõna tõsises mõttes hoomamatu vaid seepärast, et tegemist on äärmiselt sooja ja toreda inimesega kellega oli lihtsalt niivõrd tore juttu rääkida. Külasena Anne Villems, head kuulamist. 

Kõneleja 3
Tohib sina öelda? Tohib küll. Mis on sinu auväärt nimi? 

Kõneleja 1
Minu nimi on Anne Villems. 

Kõneleja 3
Tere tulemast, oleme jälle kord kogunenud kaamerate ja mikrofonidega kaamerad, olles jälle maha unustanud Tartu linna suurepärasesse uude delta hoonesse. Ja on tõeline rõõm. Olen sinuga istuda ja juttu rääkida, sellepärast et sa vist oled see inimene, kes kõige rohkematest lugudest nagu läbi käinud tegelasena. Et alates sellest, kuidas korraldati Eesti esimene veeemmasterite koolitus, kui, kui lõpetades kõikide muude asjadega. Aga kõige selleni me võib-olla jõuame, kui hästi läheb. Kui hästi ei lähe, siis räägime muudest huvitavatest asjadest. Et aga nagu Mika pihta hakkame, hakkame algusest. Kuidas sina arvutite juurde said? 

Kõneleja 1
Mina käisin mari ülikoolis. 

Mari ülikooli nime rahva seas kandis tolleaegne Tartu 10. algkool mis on kohe Vanemuise kõrval vana Vanemuise vastas. Ja nimi tuli tal tema karismaatilisest matemaatikaõpetajast. Marvetist. 

Sellel, kes oli seal lõpare juhatajaks ja kui me selle kooli aastal 1960 kevadel ära lõpetasime oma pinginaabriga siis meie vanemad olid väga huvitatud meid panemast, viiendasse kooli. No mis tol ajal oli siis Rostov gümnaasiumis, vabandust Rostovi ülikoolis, mis oli siis esimene ülikool, mis naisterahvaid vastu võttis. Aga, aga seal asus parajasti viienda keskkooli õpperuumid. Aga meie ei tahtnud viiendasse kooli minna, ma ei tea, miks, aga meie otsustasime, meie otsustame. No ikkagi 14 aastat isiklikku vanust. Loomulikult tuleb ise otsustada, kuhu kooli sa lähed ja meie otsustasime, et meie lähme hoopiski ikka vanasse endisesse Treffneri gümnaasiumisse mis tol ajal kandis siis nimetust Hansen Tammsaare-nimelises nimeline keskkool. Sinna aga minu meelest tol ajal ei olnud nagu üldse seda probleemi, eks ole, pidi mingeid katseid tegema ja midagi. Aga ega meil katsetega ka ollakse üsna hästi läinud. Me mõlemad õppisime üsna korralikult. Ja, ja siis, kui me juba aasta otsa olime seal ära olnud, siis avati seal matemaatika klass, minu jaoks oli see kohe otsest sirge tee, sellepärast et matemaatika oligi mu lemmikaine. Ja koolis ja kõiges on süüdi muidugi Marisin, tähendab õpetaja Marvet, sellepärast et nii kihvt matemaatikatunde nagu tegin meil viiendast seitsmenda klassini. Ma tean pärast ainult siis, kui Olaf previds tuli meid õpetama keskkoolis, aga see oli juba mata klassis niimoodi ja, ja siis nii et kui me üheksandasse klassi läksime, sest noh, tolleaegne algkool lõppes seitsmenda klassiga. Kui me üheksandasse klassi läksime, siis mina läksin kõigepealt ja kahe kuu pärast tuli mu pinginaaber ka järgi, sellepärast et siis tuli töö, mis, mis õpetus. 

See ei olnud tööõpetus, see oli, see oli mingisugune noh, ühesõnaga praktiline õpetus keskkoolis. Ja siis ta vaatas, et, et see, mida neid nende klassile pakuti. Et see nagu mingit erilist pinget ei pakkunud, tuli ära ka meile, kuigi tema oli Homantitaarsete huvidega. 

Ja siis olime alles viis tundi matemaatikat minu suureks rõõmuks nädalas ja muude ainete seas ka sellised toredad ained nagu elektrotehnika ligikaudne arvutamine ja programmeerimine. 

Kõneleja 3
Aga mille peale seda programmeerimist õpetati tol ajal 

Kõneleja 1
Selleks ajaks oli Eestis olemas parajasti üks elektronarvuti üks üks ja mis oli kandis nimetust Uural. Talle ei olnud veel numbrit külge pandud number üks. Ja see oli siis Tartu ülikooli arvutuskeskuses, mis oli 59. aastal moodustatud masin ise asus peahoone kõrval Majas. Ja seal treffnerist jõmmid üldse väga kaugel, üks kvartal. Ja, ja õpetasid meid seal alguses Ülo Kaasik ja pärast Mati Krull, Matti Krull oli sinna tööle läinud, sellepärast et tema oli see esimene kursus ülikoolis, kellele Ülo Kaasik programmeerimist õpetas. 

Kõneleja 3
Nii et ühesõnaga Ülo Kaasik kuidagimoodi ilma arvutita 

Kõneleja 1
Ei? Arvutiga. 

Kõneleja 3
Juba Ülo Ülo Kaasik 

Kõneleja 1
Arvutiga sellepärast, et meie olime üheksandas klassis 62. aastal seesamune arvuti oli juba kaks ja pool aastat kohal. Kaks aastat. 

Kõneleja 3
Aga kellelgi pidi seal keskkoolis olema siis kui parasjagu nagu visiooni, et noh, et aru saada, et elektrotehnika ligikaudne arvutamine ja arvutid ja et see on nagu oluline millegipärast, kellele. 

Kõneleja 1
Ei, ma arvan, et see visioon võib-olla, et ei olnud isegi keskkoolil kuigi meil oli ka väga karismaatiline direktor omal ajal Allan Liim, aga tema oli ajaloolane. Ja, ja ma arvan, et see initsiatiiv tuli tegelikult Olav Priinitsa ja Ülo Kaasiku poolt et nemad organiseerisid selle, see ei oleks olnud nõukogude liidus erakordne. Et selle arvuti nõkku panid ja näos oli esimene kooli arvutuskeskus, siis see oli kogu liidus esmakordne juhus, aga matemaatikaklassid olid tol ajal juba olemas ja nad olid Moskvas olemas. Nii et nemad võisid öelda, et nemad jälgivad Moskva mallija, sellega keeruv keerulisi asju enam ei tulnud, igal juhul. Ma usun, et Treffneri kool Sonnil valitud võib-olla sellepärast, et Treffneri kool oli südalinnas, ei olnud vaja kuskilgi kuskile kaugele minna, sest oli selge, et vähemasti esimestel aastatel ülikooliga Beud hakkavad õpetama nii et meid õpeta soolast prionits, matemaatikat. Täiesti ma ütlesin, mul ongi olnud siuksed täiesti fantastilised matemaatikaõpetajad algkoolis õpetaja Marvet ja keskkoolis Olaf Ernits. 

Kõneleja 3
Aga kust kohast see matemaatiku või nagu arvuti võiks üle läheb, et matemaatika niuke abstraktne ja kaunis kunstidega arvutid, need on mingid tolmavad ja blogistonid lendavad. 

Kõneleja 2
Tolm tuleb pärast on tal. 

Ikkagi messi. 

Kõneleja 1
Kuidas tähendab nad muidugi omavahel saavad seotud olema oluliselt kõrgemal tasemel see tähendab seda, et sa pead ikka natuke matemaatikat enne oskama, enne kui sa aru saad, kuidas matemaatika-informaatika või noh, programmeerimisele on. Aga, aga enne pead natukene programmeerida oskama enne kui hakkad aru saama, et mõnikord on matemaatikat ka vaja. Näiteks alustades sellest, eks ole, algoritmiliselt The lahenduvaid ülesandeid eristada algoritmiliselt lahenduvatest ja need neid lahenduvaid jagada ka sellistesse klassidesse, mille lahendid ei tule mitte 50 aasta pärast või 50000 aasta pärast vaid millel lahendit tulevad arvutist noh, kui mitte homme hommikuks, siis vähemasti ülehomme. 

Kõneleja 3
Aga seda enam tekib küsimus, et matemaatikahuvilisi on inimesel ei ole tingimata nagu arvutihuvi. Miks sul oli? 

Kõneleja 1
Ma ei tea, kas kas see huvi tekkis, tähendab, alguses oli muidugi lihtsalt jube põnev noh, kogu Eestis ainult üks arvuti ja meile õpetatakse ja tuled vilguvad ja mingisuguse perfolindilt, mis on filmilint, filmilindilt loetakse, mitte see telegraafilint loetakse andmeid siis no lihtsalt jube põnev oli. Aga, aga seda muidugi, eks ole, et, et kuskil mõnikümmend või natuke enam aastaid edasi igalühel, meil on siin taskus laua peal ja, ja meie epsilon ümbruses. 

Tuhandeid kordi võimsamad arvutid kui see, mis meil seal terves suures saalis noh, ikka mitukümmend ruutmeetrit enda alla võttis. Seda muidugi tol ajal, eks ole, 60 62, kolm et ei näinud, 64, ma lõpetasin keskkooli. Ja õppima läksin muidugi ka puhtalt matemaatikat, sellepärast et vabandage mind väga, ega kuskilgi, võib-olla tippis, vabandust, tehnikaülikoolis mingisuguseid siukseid tehnikaaineid, kus ka arvutid otsapidi sisse tulid, võib-olla õpetati, aga ma nagu ei tahtnud. Tähendab, tehnika pool mind väga ei tõmmanud. Ja, aga programmeerimine iseenesest on niisugune maagiline tegevus, eks ole, et enne seda arvuti midagi ei oska, siis sa kirjutad talle siukse kihvti programmi ja siis ta järsku oskab midagi, näeb välja juba peaaegu et nagu nagu saaks millestki aru. 

Kõneleja 3
Ja huvitav on see, et Meelis Roosi esimene programm oli ka vestlemise programm, mis on just täpselt see asi, et sul jääb mulje, et arvuti saab millestki arvab, millestki ära jäänud. Just see on, nagu on, on läbiv joon. 

Kõneleja 1
Ja meile anti ka, meil oli, ma usun, üheksandas või kümnendas klassis me tegime oma elu esimese programmi. Minu, minu programm ei pakkunud mulle nii palju pinget, aga elu lõpuni ja ilmselt on meelde jäänud ühe mu koolikaaslasi ja ma ei mäleta, kes selle programmi tegime Open klassikokkutulekul küsima. Kus ülesanne seisnes selles, et talle tuli anda, et kuupäev ja siis ta pidi välja trükkima, mis nädalapäev see on. Aga nüüd arvestage eurole ühega, tema ainuke väljundseade oli kitsas printer, kus sai trükkida ainult arve. Ja siis siis nohtunud. Kui õppejõud siis proovis neid meie programme, siis ta kõigepealt andiski mingi mõistliku kuupäeva ette ja sai vastuse ja siis andis ette 30. veebruari mille peale programm hakkas siis printeri peal siukest nullide joru ülevalt alla trükkima. Mille peale õppejõud ütles, et nojah, ilmselt trükib mingit jama, eks ole, katkestama ära ja õpilane väga vaikselt ja tagasihoidlikult lasta natukene veeldatud. Ja siis siis ta trükiski välja ühe nullide joru ülevalt alla, siis ühe nullide, kas ka täidetud rea siis ühe tühja rea, siis natuke null, siis natuke nulle äärtes, siis natuke nulle keskel ja siis veel kord selle nullide joru ülevalt alla ja ühe terve nulli, tere ja, ja veel kord sama. No ja kui ma seal paberis kätte saime, siis oli kõigile näha looli. 

Nii et see, see oli asi, mis mulle meie elu esimestest programmidest kõige paremini meelde jäi. 

Kõneleja 3
Ja programmeerimine on inimese inimese tegevus. Aga mis te nagu lahe siukseid, nagu nuputamisülesandeid tegid tegitegi või mis programmiga? 

Kõneleja 1
Me ei, me tähendab, see oli üks huvitavamaid seal igasuguseid asju tegime mingite jalgade keskmise leidsime ja ja ma kõikide ülesandeid ei teagi seal, kus nad olid individuaalselt antud antud ülesanded ja niuksed niuksed tavaliselt midagi, praegused programm, algajad programmeerijad. 

Kõneleja 3
Kas kas see? 

Kõneleja 1
Selle jaoks on valem olemas, muuseas kuidas seda kuupäevast nädalapäeva teha? Kindlasti. 

Kõneleja 3
Kas sel ajal mingisugust nihukest nõu kogukonda ka nagu oli, nagu liidu peale selle arvutiasjade ümber või mingid olümpiaad juba sel sel teemal peeti või? 

Kõneleja 1
Ma ei tea, kuna kuna programmeerimist koolides õpetati ka äärmiselt vähestes, ma ei teagi, näiteks kas moskva matemaatika klassis arvuteid ka õpetati, ma ei tea seda. Ja kui ma Ülo Kaasiku käest kunagi ühe intervjuu käigus küsisin, et aga kust ta metoodika võttis meile programmeerimise õpetamiseks, siis ta vaatas mulle suurte silmadega otsa ja ütles, et aga metoodikat ei olnud mitte mingisugust. Vähe sellest, et ei olnud metoodikat, ei olnud ka kirjandust. Nii et ise hakkas siis raamatuid kirjutama ja ja, ja, ja see metoodika oli tal, ta ütles, et noh, kui nii-öelda käigu pealt välja 

Kõneleja 3
Ühesõnaga, tegelikult kogu see eesti asi käib Ülo Kaasiku peal. 

Kõneleja 1
Jah, see see asi käis ikka kindlasti tähendab tema ja tema õpilased sellepärast et lihtsalt edasi, jah, sellepärast et noh, pärast seda hakkas neid matemaatikaklasse tekkima ikka kuigi kaks aastat hiljem või kolm aastat hiljem. Sest ma mäletan, et OV karu, kes oli omaaegne näodirektor, tema istus meie matemaatik, ka lõpueksamitel. Et nii-öelda vaadata, kuidas, kuidas, kuidas me siis oleme matemaatikas arenenud. Ja siis siis siis ma mäletan isegi omaenese vastust. Kuna mul oli keeruline joonist teha, siis sellel seletamisel mul läksime puntrasse, siis ma astusin kaks sammu tahvlist eemale ja alustasin otsast pihta. No see on see tõestus, et sirge on ristitasapinnaga, kui ta on ristikahesel lõikuma sirgega, noh, see on keskkooliprogrammi oli keskkooli programm praegu, ei milleski kindel olla. Kui me oma projekti kooriga meil nimelt on arvutiteaduse instituudis projektikoor. Meil tuleb kooli laul meelde poolteist aastat enne laulu võidu ja siis me võtame oma koori kokku ja siis käime, valmistume nii hästi, kui suudame mõndagi ettelaulmiseks laulupidudel. Ja kui me laulupeol küsiti meie käest, et kuidas meie koori iseloomustada, siis igaüks pakkus midagi välja ja siis siis mina pakkusin. Meie koori iga liiget ja detaagorusse teoreemile ise erinevat tõestus, neid on umbes 200, meie kooriliikmeid oli kuskilgi suurusjärk 40 mille peale matemaatika õpetamisega tegelevad õppejõud meie koorist tuletasid mulle meelde, et aga Pythagorase teoreemi enam ei tõestata koolis. Nii et ma nüüd ei tea enam mitte midagi. Paraku. 

Kõneleja 3
Modempro huumoris, aga ühel hetkel sai keskkool otsa. Ja siis tuli ülikool, Tartu Ülikool, eks ole? 

Kõneleja 1
Jah, Tartu ülikool ja noh, vaadake aasta oli siis, eks ole, 1964. See tähendab väga sügav nõukogude aeg, mis välistas minu jaoks absoluutselt kõik humanitaaralad. 

Järgi jäi suhteliselt vähe järgi jäi meditsiin, mu isa oli kirurg ja noh, ju ta siis vaikselt ikka lootis, et äkki ma aga, aga kirurgias mul on selline loogiliselt mõtlev mälu mul on, mul on väga hea igasuguseid tõestusi meelde jätta ja nii edasi. Aga kui ma pean pähe õppima 2000 kontide nimetust, siis ja sinna juurde veel, eks ole, lihaste ja veresoonte ja jumal teab veel ajus agarate, millede ladinakeelsed nimed. Siis ma ei arva, et ma ennast väga hästi tunneksin ja peale selle me juba viiendasse keskkooli ei tahtnud selle pärast minna, et vanematel olid seal liiga head suhted. Tahtsime ikka ise olla. No siis koolis ma ka tahtsin ülikoolis olla rohkem ise kui keegi muu. Ja siis jäigi järgi matemaatika ja matemaatikat. Ma armastasin koledasti. No pange tähele, mul olid, eks ole, viis, kuus, seitse head õppejõud, ma ei saa midagi paha öelda ka oma kaheksanda klassiõpetaja kohta, kes mind soovitas, mata klassi ja noh, siis oli nii-öelda eesti matemaatika koolis õpetamise korüfeed olla. Priinits oli siis mu küll tundmumale siiamaani mäletan funktsionaalse seose selgitust siiamaani, noh, see on siis, eks ole. Ligi 50 aastat pärast seda, kui ma seda õppisin. 

Kõneleja 3
No aga siis oli hästi-hästi tehtud. 

Pidigi olema seal Tartu Ülikoolis toona. 

Matemaatikul ikkagi nagu õpetati lõpuks ka programmeerimist. 

Kõneleja 1
Jah, meil oli kaks kursust programmeerimist, üks oli masinkoodis, programmeerimine, uuror nelja peale selle ma läksin ja tegin nii-öelda kohe septembrikuu jooksul mitte ära mille peale õppejõud võttis mu vastusse ja ütles, et lõhnab natuke Uural õe järgi. Aga programmeerida te oskate ja pani mulle arvestuse. Ja teine oli siis Ülo Kaasik kuu algul 60 õpetus ja vot seda ta õpetas küll pliiatsi ja paberiga. Sellepärast ühtki translaatoreid tol ajal algollist ei olnud. 

Kõneleja 3
Aga miks ta Algurit just õpetas? 

Kõneleja 1
Vaadakem, eks ole, aasta siis oli 66, seitse, mis keeled siis seal üldse olemas olid? 

Jumalale tänu, et keegi jaganud meile Koboliit õpetama, see oleks mind küll programmeerimisest viie kilomeetri kaugusele peletanud. Olete te proovinud kunagi Koboliit lugeda või? Ei ole? Ärge vaadake ka see on niisugune, eks ole, kupp, programmeerimine on selline kontsentreeritud väljendus, eks ole, sa saad valemeid kirjutada valemitena ja, ja siis sul on mõningad sellised kenad koodsõnad, foor ja too ja if-then else ja nii edasi. Siis pange nüüd sinna mingisugune filoloogide soust peale, kust te teate isegi, et palk pluss preemia välja kirjutada kolmesõnana? Vot niisugune programmeerimiskeel, noh. Ja ma tänan jumalat, et ma mitte kunagi ei ole pidanud programmeerima koolis oli vahepeal väga elujõuline. 

Kõneleja 3
Ja siiamaani on inimesed ju tegelevad sellega testi aktiivsed siiamaani ja mis on huvitav, selle jutu juures on see, et et siin on ikkagi nagu ekspetsiitne, nagu programmeerimisõpetus käis kuskil nagu keegi metoodiliselt õpetas nagu programmi kirjutama, siis on üsna haruldane. Sellest tol ajal kindlasti jah, aga, aga nendest lugudest nagu räägitakse üldist harva reegli inimesed ütlevad, et programmeerimine kuidagi jäi külge. Ja nad ei oska öelda, kust või kes õpetas. Aga, aga. 

Kõneleja 1
Ja ei, kindlasti on võimalik programmeerimist mu kui tahtmine on väga suur, siis kindlasti on võimalik programmeerimist iseseisvalt õppida. Aga no motivatsioon peab ikka niimoodi seinakõrgune olema. 

Kõneleja 3
Ja kas tol ajal oli, mis see nagu arvutite niisugune perspektiiv oli, milleks neid kasutati? Programmeerime, programmeerime, aga kas see on nagu teaduslik töövahend või mingi rahvamajandust aitab kaasa? 

Kõneleja 1
Ja tähendab, rahvamajanduses kasutus oli juba uurali ajal kuna mälu oli väga väike. Ärgem ära unustagem, et Uural ühe mälu, mida keegi ei usu, oli kaks kilo, aga noh, õnneks mitte baitisest baiti tol ajal ei tuntud, vaid sõna niimoodi kaks kilo sõna. Võtke oma telefoni, vaadake siis nende mega või gigabaitide arvu. Mis teil taskus on. Siis väga suur probleem oli see, et kuidas neid andmeid, mille pealt midagi pidi arvutama, tatama kuidas neid sinna arvutisse ära mahutada, välisseadmetega olid ka veel omad probleemid. Aga kas need pakiti niimodi? Ja, ja, ja ma mäletan, et kui ma ise programmeerimist õpetasin masinad voodis programmeerimist. Sellepärast et ma miskipärast läksin Mäele viimase kursuse viimasel semestril 70. aastal. Ja siis järsku nagu langes ära minu Tallinnasse tööleminek, sest mu mees teatas surmkindlalt, et tema küll Tallinnas ei kavatse minna. Noh ja siiamaani ei ole läinud. Ja, ja siis siis ma pidin ruttu Tartus endale töökoha otsima. Ja siis siis Ülo Kaasik, kui oli siis see kes ütles, et jah, ma usun küll, et ta võib tudengite ette saada, ta, tal jalad ei päris. 

Siis mind saadetigi niimoodi, pärast lõpetamist lõpetasin juunis vara ja siis septembris läksin siis. Tudengitele jah, põhiliselt programmeerimist õpetama. 

Kõneleja 3
Aga kas mingisugust teadustööd ei sirgunud sealt või poosist, õppejõu töö? 

Kõneleja 1
Teadustegevusega ma hakkasin tegelema hoopis palju hiljem sellepärast et kui meil praegu räägid, Öeldakse, et mis asi on õpetaja koolis norm töökoormuste teatubjastid ja 24 28 tundi siis algajate õppejõudude, nii-öelda jalul, seismise, auditooriumi ees seismise koormus. Vähemasti see, kus, kui mina õpetasin, oli 24 või 28 tundi nädalas noh, sealt kõrvalt vedama. Ja eriti kui ta esimesi aastaid õpetajatega ja, ja peale selle õpetajate kahes keeles selle pärast, et ega meil siis matemaatika poole peal neid vene keelt rääkivaid inimesi oli väga vähe. 

Kõneleja 3
Ja olidki kaks eraldi grupil eestlane. 

Kõneleja 1
Jah ja majandusteaduskonnas ma õpetasin elu esimese loengu, ma lugesin vene keeles majandusteaduskonna, ma ei tea, kas kaugõppijatele või vene keelt, ma oskasin väga nirult. Sellepärast et esiteks ei olnud seda vaja. Ja teiseks ma olin küll lugenud väga palju matemaatikaalaseid õpikuid, aga noh, mis ma võin neid tõestusi lugeda prantsuse keeles, mida ma ei oska. Sellepärast et seal on nagu neid vahesõnu on suhteliselt vähe. Aga, aga selleks, et õpetada, selleks on vaja palju sõnu. Aga siis mul õnneks jätkus taipu. Vaadake, meie venekeelsetes rühmades oli alati sees kakskeelseid inimesi. Ja siis mul jätkus teha nendega tudengitega, keda ma õpetasin nendega niisugused kokkulepped. Esiteks, kui ma ütlen midagi sellist, millest ei ole võimalik hea tahtmise juures aru saada, siis esimene rida ütleb mulle. No ja siis ma püüan ümber sõnastada niimoodi, et arvuga oleks võimalik saada. Ja KuMul sõna meelde ei tule, siis ma ütlen selle sõna eesti keeles ja need kakskeelsed ütlevad mulle vastava venekeelse termini. Minu vene keele mitteoskust iseloomustab võib-olla see, et Ma näiteks ei teadnud, kuidas on. 

Siis ma küll üritasin kasutada sõna Ati jäätis või noh, mis ei ole ka väga vale, aga aga Võõtšest sõna ma ei teadnud. 

Kõneleja 3
Ja see on ju pedagoogilise metoodika, mõtlesin õudselt hea kooli. 

Kõneleja 1
Jah igal juhul kuskilgi, umbes seitse või kaheksa aastat pärast seda, ma sattusin Moskvas, mind ei olnud Moskvas kuskilgi ääretul Nõukogude Kodumaal. Sattusin kuskilgi konverentsile, kus ma siis midagi seltskonnas ütlesin ja mille peale minu käest küsiti, et võismask. Noh, mille peale ma siis ütlesin, et kui ma teist korda veel suu lahti teen, siis teist, siis te saate kohe aru, et ma ei ole ei Moskvast ega Leningradist. 

Kõneleja 3
Ja selline õppejõu töö siis Tartu Ülikoolis läkski edasi. Kuni saabusid aastat 20. 

Kõneleja 1
Ei vist vist veel edasi. 

Kõneleja 3
Mis maa aga, aga lihtsalt aastast 80 algab see kuu, kus asjad lähevad, arvutid lähevad väiksemaks. 

Kõneleja 1
Arvutused lähevad väiksemaks ja noh, see aeg, kui meil ei olnud ohja sõjaarvutite saamine seal vahepeal ka 80.-te oli oli, oli ikka omaette tsirkus. 

Kõneleja 3
Nojaa vot Jaan, Tallinn on rääkinud, et tema tõi oma esimesed arvutid käsipagasis laevaga Rootsist. 

Kõneleja 1
Minul ei õnnestunud, väga rootsi, ai, ma käisin küll, aga siis siis siis manirikas seal Rootsis küll ei olnud, mul abikaasa oli Bostokis Uppsalas ja mina sealt arvud, et ei toonud, kuigi ma tean küll, eks ole, et kui oleks seal ostnud arvuti siis siin maha müünud, siis oleks selle eest väga palju muid asju saanud, aga ei, mul siukest ärivaimu ei olnud. 

Kõneleja 3
Aga Tartu Ülikoolis hakkas kuskil millal hakkas Tartu Ülikoolis arvutine tavaliseks asjaks muutuma, kus ta nagu noh, oli nagu selline igav. 

Kõneleja 1
Mul on üks selline murdepunkt, oligi 82 82 juhtus selline asi, et Tallinna sõbrad organiseerisid näituse välisnäituse, kust tulid vist 100 firmat, et sinna Tallinna näituseväljakule. Aga, aga miks minu nii-öelda tähelepanu köitis, sest noh, seal oli neid näitusi enne ka olnud. Sealt pidi tulema kohale kellegi sõber Suurbritanniast, kes oli hakanud Apple'i diileriks Apple'i diileri, kes oli ta hakanud sellepärast, et ta pidi ehitama ühe ühe seadme ülikõrgete rõhkude jaoks. Aga seda seadet oli vaja juhtida ja siis juhtimiseks tavalis välja Apple'i ja kõige odavam Apple'i saamise viis oli see, ta hakkas Apple'i diileriks niimoodi ja nüüd ta siis tuli siia näitusele, ma ei tea, mida ta muud siiani tõi, aga igal juhul Teaduste Akadeemia instituudid ja, ja pooled mulle head tuttavad inimesed hakkasid ette valmistama, et tema käest Apple'i arvuteid osta. No kuulge, kellele on, milleks näiteks üks kadunud professor Lippmaale arvud? Minul on arvutit vaja, eks ole, mitte ma arvan, et mul on arvutit rohkem vaja, kui tal on mingisugused oma magnetresonantsid, mida ta ka peab juhtima ja, ja aga, aga, aga meie õpetama tulevasi programmeerijaid välja ja ja, ja meie kasutame input, output, kapi, kui keegi teab, mis asi see on. Kuhu tudeng paneb oma perforeerimiseks oma Blaketile kirjutatud programmi ja siis kolme nelja päeva pärast saab sealt tagasi, süntaks vigadega. Tema väljakäsitluses esimeste süntaksi vigadega ja aga aga mitme eaga, kui ta neid mitu tükki tegi. Ja mul läks hammas koledasti verele, sõbrad saavad miskipärast arvuteid, aga tegelikult on neid mulle hooaja. Ja siis ma läksin ja rääkisin oma sõpradega. No ja siis siis tuli välja, et sõbrad on head sõbrad. Näiteks selleks, et üldse aru saada, millestki pill koosneb, sest noh, mul ei olnud teda vaja millegi juhtimiseks, mul vaja siukest alasti arvutit. Ja siis ma istusin. Siis ma istusin Tõraveres. 

Sest seal käisid baidid, baitidest sai, sai siis teada, mis on arvutil sees ja mis talle külge käib. Ajakiri balletiga, mõistusin pühapäevade kaupa seal ja panin siis oma konfiguratsiooni kokku ja sain aru, et mul ongi vaja alati arvutit. Ja siis, kui ma oma konfiguratsiooni siis siis Tallinna sõpradega siis Lippmaa instituudist siis pidasin aru, mis on mõistlik summa, mida plaanikomiteest küsida. Ja siis ma leidsin, et noh 10000, kuldrubla rubla ei toiminud välisturul ainult ainult kuldrubla 10000 kuldrubla oleks siis piir, mida võiks, on nii pisike summa, et keegi võib-olla et äkki annabki selle. 

Arvuti kolm arvutit saime selle eest, sest mul oli vaja paljaid arvuteid, ma isegi monitore ei ostnud, igaks juhuks, et mine sa isahane tea. Võib-olla meie nõukogud televiisorite äkki ei töötagi? Oota, ütles ja et Apple kahed. Ja siis Plaanikomitees käis minu eest Lippmaa Instituudi sõbrad. 

Tõnu Karu ja vahepeal oli välisministriks mul nimedega raskusi. 

Kõneleja 3
Lippmaa instituut on meil siis küberneetika, istud. 

Kõneleja 1
Meil ei ole ka PÖFF-i 

Seda nimetati kogu aeg Lippmaa instituudis ja tol ajal oli ta veel seal Teaduste Akadeemia raamatukogu all. Estonia puiesteel. 

Ja siis Sinijärv Riivo Sinijärv oli või need, kes minu meelest plaanikomitees käisid paberitega, mina ajasin kõik paberid korda, korjasin ülikoolist kõik allkirjad kuni rektori nii peale. Ja siis nemad käisid nendega ja antigi 10000 me saimegi oma. Ja siis me tegime nendest arvutiklassi kolmest arvutist ja panime programmeerimise õpetamise individuaalgraafikus praktikum idega. See oli juba väga tore aeg, sellepärast et inimesed olid harjunud input, output kapiga, eks ole, et annad sisse, siis unustad kõik ära, mis sa sinna kirjutasid, siis saad kolme või nelja päeva pärast siis mingi paberirulli tagasi oma oma süntaksivigade sisse, paratme, parandad neid süntaks vigu ja, ja nii edasi. Noh, selle pisikese elu esimese või teise või kolmanda programmi silumine võttis niimoodi ikka õige mitu nädalat aega. 

Kõneleja 3
Aga see, see tähendab siis seda, et tegelikult see nii-öelda tarkvara tehniline pool muutus nagu radikaalselt 

Kõneleja 1
Kui ma nimelt aegajalt see oli kevadsemester, eks ole, Me arvutit saime kätte kuskilgi jaanuaris veebruari algusest, panime siis algul programmeerimise algõpetuse, kuigi meil Peisiku keeli ei meeldinud ja aeg kell kaks on sündinud Peisikuga siis siis mõtlesime, et noh, me suudame need Peisiku hädad vast ehk kompenseerida oma hea õpetamisega. Et alg algesimese programmi jaoks ta kõlbab. Ja siis siis kevadel paistab Liivi tänaval paistis päike sinna klassi. Ja mina siis seisin, eks ole, nende kolme arvuti selja taga, et tudengid aidata. Ja siis tegelikult ma ei näinud, mis nad sinna ekraani peale kirjutasid, vaid ma nägin nende enda peegeldust. Ja see, kuidas tudeng sisestas oma programmi pani käima, sai sealt ise oma süntaks, vead kohe kätte. Noh, see miimika, eriti tütarlaste oma, see oli ikka ikka tasus vaatamist, et aru saada seda, et me oleme kuskil ligi suunas õige õige sammu astunud. 

Kõneleja 3
Suur tänu. Kas see metoodika ka pidi sisu muutuma, kuidas seda asja õpetati? 

Kõneleja 1
Ja koos keelega muutub alati sellepärast, et küsimus on alati võsa põhi. Põhikisma, mida algõpetuses arutatakse, on see, et missugust programmeerimiskeelt õpetada esimesena sest noh, sealt jäävad asjad külge. Ja pisiku häda on muidugi see, et kui ma olin valmis kirjutanud nende arvutitega on need, olid küll macho arvutid juba aga me kasutasime neid ka laialdasemalt arvutite tutvustamiseks. Ja siis kui ma olin Peisikus kirjutanud valmisprogrammi, mille väljatrükk oli umbes minu enda pikkune siis ma vandusin, et see on minu viimane programm, Peisikus, mina rohkem Peisikuse kirjutab esikus nimelt ei funktsioon ei alamprogramme. Ja, ja siiamaani ma olen seda vannet pidanud, ma ei ole Peisikus rohkem programme kirjutanud. Ei ole. 

Nii et tükk aega me muidugi, kuna mulle õpetati kõigepealt masinkoodi, vabandage väga uurali arvuti peal, mingit mingit kõrge taseme keelt ei olnud ja assembleriga ei olnud. Ja mulle tundus see nii, noh siis ma saan ju aru, mängin otse registritega ja ma saangi aru, mis, mis masinas toimub ja mida see aritmeetiline plokk seal teeb ja nii edasi ja sealt edasi on juba kõik väga lihtne. Siis siis üks muu Novosibirski nüüdseks juba kadunud sõber kunagi kumas võitlesin selle eest nii-öelda ühes üleliidulises seminaris, et ikkagi masinkoodist tuleb alustada. Siis ta minu seisukoha ilmestamiseks rääkis anekdoodi. Toot oli sihukene, et vot vaene mees läheb kirikuõpetaja juurde ja ütleb, et kole raske on elu naine ja lapsed ja ämm ja ja, ja, ja kõik peame elama, eks ole, oma onni ühes toas ja see on see, see on ikka väga raske. Ja siis kirikuõpetaja küsib, et aga kas sul kits on ja kits mul on? Ole hea, võta kits ka tuppa juurde. Ja. 

Milleks veel kits, aga kirikuõpetaja käskis, talumees võtab siis ka kitsetuppa ja, ja ütleb, et vot nüüd nädala pärast nüüd piigits välja. Ja, ja siis, ja siis siis tule räägi minuga. Nädala pärast viib mees kitse välja ja tuleb, räägib, et vot nüüd on küll juba väga hästi kitse ei ole ja oma naise ja ämma ja lasteaia kõikidega ma saan nüüd juba palju paremini hakkama. Must ja, ja siis sellest masinkoodist alustamine on nagu siis kitsetoomine sinna tuppa, et ja kui lõpuks siis saab hakata programmi kirjutama fooriaiitsia Helsiga. Et noh, siis on suur lõõgastus, et noh, enam ei pea, eks ole, tõstma midagi registrisse ja kontrollima ja andma suunamist ja nii edasi. 

Kõneleja 3
Kui ma nüüd seda juttu kuulama, siis linn on toimunud üks mingisugune kuskil jutu sees mingisugune nihe toimunud. Alustasime sellest, et meelisprogrammeerimine siis nüüd ühel hetkel oli, andis selle nii-öelda positiivse emotsiooni, see näoilme muutus selle ekraani peegelduse peale. Mis hetkest see nagu programmeerimise õpetamine muutus nagu huvitavamaks või nagu põnevamaks kui programmeerimine. Kui üldse niisugune hetk. 

Kõneleja 1
Jah, ei, ma usun küll. 

Selleks, et midagi väga tähelepanu väärset programmeerimises ära teha. Selleks on vaja head meeskonda. Ja ma arvan, et parim siis nii-öelda suur asi, mida ma programmeerimises olen teinud. Me Ain Isotammega sisuliselt kahekesi tegime omal ajal suure süsteemi villis, mis oli aruannete generaator ja millel olid ikka väga tähelepanuväärsed omadused ja minu elu kõige keerulisem programm, mis tegeles siis magnetlindi ja printeri juhtimisega aruannete väljatrükkimise ajal, kui aruanded on pandud segamini magnetlindi peale küll tekitamise järjekorras aga, aga 15 aruannet korraga niimoodi Juppé vaheldumisi ja siis printerist tulevad kõik aruanded õiges järjekorras ja, ja mul oli kaks katkestuste allikad ja siis tasakaalu hoidmine, printeri, puhvrite ja magnetlindi puhvrite vahel teisendamise töö seal vahel, et et andmetest teksti tekitada, see, see oli köömes. See on vast jah, mu kõige-kõige keerulisem programm, mille käigus muuseas on iga programmeerija unistus avastada arvutit Diviga. Teile kogu aeg tundub, eks ole, et arvuti teeb valesti, sellepärast teie teete kõik õigesti, aga arvuti ju eksi. Ja siis te lähete seda inseneridele rääkima. Ja kui te hakkate punainseneri, tea teie seda epsilon ümbrust seal või kontekstis ja Stakate inseneridele seletama seda asja algusest piir peale. Vot niimoodi niimoodi niimoodi niimoodi kuskil poole peal te saate aru, millal te ise olete pea teinud, haarata kõik oma väljatrükid ja ütlete insenerile, kes sinnamaani ei ole veel mitte millestki aru saanud. Ööd ütlete talle aitäh posti mängimise eest ja siis lähete oma viga parandama. Aga, aga selles programmis mul õnnestus leida arvutiga. Sellepärast et Sünkro impulss traatprinterile oli halvasti joodetud. Ja see tähendab seda, et kõik inimesed trükkisid sümbol haaval. Või siis rida, eks ole, äärmisel äärmisel juhul terve rea. Mina tahtsin, tegin lehekülje valmis ja tahtsin tervet lehekülge. Trükib mulle pool lehekülge, veerand lehekülge ja edasi lihtsalt ei trüki. Lähen kontrollin seda, printeri juht käsku, kõik on õige, kõik vaga ei trüki. Noh, ja siis insenerid avastas, see on minu ainus kord minu programmeerija karjääris, kui mul õnnestunud arvuti viga avastada. 

Kõneleja 2
Kusjuures mitte arvuti kui tüübi viga suur all nagu üldiselt, nagu teeb see konkreetne tükk spetsiifilise viisil. Lihtsalt konkreetne printer, jah, aga noh, see on ikkagi riist ikka inseneride pärusmaa, mitte mitte mitte programmeerija oma. Kuidas, kuidas õpetamise värk huvitavamaks läks? 

Kõneleja 1
Õpetamise värk on huvitav olnud kogu aeg sest ega ma siis ei oleks kaua seal selle koha peal, noh, Tallinnas oli palju keskusi, Tartus väga nagu arvutuskeskusi ei olnud väga Tartus väga vähestel asutustel olid, aga ega mind väga ei tõmmanud ka see, see tolleaegne noh, ma ei tea, ASC tegemine automatiseeritud juhtimissüsteemide tegemine, see, see, see nõukogude tehnika töökindlus oli ikka niivõrd vilets. Et isegi Nõukogude esimesed üle Kahaldatuse ameeriklastelt üle kavaldatud personaalarvutid ei kippunud hästi töötama. 

Nii et, aga õpetamine oli ju palju toredam. Seal õpetas isa ja, ja peale selle noh, kuni siiamaani eks ole, räägitakse meile, kuidas Eestis on veel ikka 8000. IT-spetsialisti puudu. 

Kõneleja 3
Need maagilised žürii kiri 8000. 

Kõneleja 1
Jah, ma usun küll ja see on enam-vähem konstantne suurus, see on seisnud juba niimoodi väga-väga palju aastaid. Noh, põhiline ei ole võib-olla see number ise 8000, vaid see Neid on puudu ja see, et neid on puudu. Seda me näeme kogu aeg oma teise kursuse peal. Nimelt inimesed omandavad esimese kursuse programmeerimise algõpetuse ja siis teevad oma esimese projekti hoopis tähendab objektorienteeritud programmeerimisest. Ja siis on ta kasulik firmale. Ja ülikool lükatakse teiseks plaaniks, noh ja kes siis peab välja, kes ei vea välja, enamus ei pea, see tähendab, ülikool jääb kõrvale. 

Kõneleja 3
Selle kohta on mul kaks küsimust, enne kui me jõuame selle küsimuse juurde, et kuidas arvuti siia jõudis. Aga mul on see küsimus, et. 

See õpetada välja arvutiõpetajaid. Kustkohast see idee tuli, sest see tahab ka ikka nagu ägedat visiooni saada, et just see, et et keskkoolis peaks nagu või üldse koolis peaks nagu arvutiõpetust õpetama, kust. 

Kõneleja 1
Sellel on nagu siuksed, kaks juurikad, üks üürikas on muidugi see, eks ole, et arvutid hakkasid jõudma ka koolidesse mitte ainult, eks ole matta klassidesse vaid ka. Sest tekkisid lihtsalt sellised arvutid, mida Uural ühte ei jõudnud keegi ükski kool osta, see on ka Eesti selge ja tal ei olnud peale omatud kuskil. Füüsiliselt ta võimlemissaali oleks mahtunud, aga aga seda ei saanud ka koolist arvuti alla ära võtta. Nii et noh, need õpetajad, kes läksid koolis arvutit õpetama, neid oli vaja välja töötada, välja õpetada. Teine oli muidugi arvutiside, sellepärast arvutiside hiilis koolides tagauksest. See ei olnud mingisugune niisugune riiklik programm Komil pärast tuli, eks ole, Tiigrihüppe ja igasugused asjad siis esimesed, ma usun, ligi 100 kooli. 

Arvutis sidega ühendatud noh ise küsimus, mismoodi ühendatud, nii et paljud ülemused ei teadnud üldse, et on. 

Osa nendest läks modemiga. 

Telefoniside kaudu. 

Alustame algusest. Tähendab, et, et aasta siis oli. Oh, anna mul mälu. 

80 alguses. Kõigepealt muidugi olid Fidu vennad. 

Ja ma arvan, et nende tegevusest ei teadnud mina suurt midagi, sest sest amatööride ridadesse ma ei kuulunud. Ja, ja seda ma kuulsin pärast. Fido vendadel oli modemside juba enne kui arvutis jõudsid. 

UDP-ga. 

Aga aga kui, kui siis arvutivõrgus juba midagi kuulda oli ja kaheksakümnendatel juba oli midagi kuulda siis siis ka meie, Soome sõbrad otsustasid Eesti Nõukogude Eesti järgi aidata. Ja noh, muidugi Tallinn on Helsingile palju lähemal kui. 

Tallinna tehnikaülikool sai siis soomlaste käest modemi, aga ja kuigi meil oli uuendatud telefoniside seoses 80. aasta olümpia, aga siis ei pidanud selle modemi kiirus. Ja ma nüüd ei nimeta numbreid, ma neid enam ei mäleta, ma usun, et 19000 millegiga bitt biti mitmes hakanud oli see mõõduühik see vist ei pidanud vastu. Siis pidid soomlased kinkima neile natukene aeglasema modemi, mis võttis ka madalamaid kiirused said Tallinnast uudsetes juuniks, juuniks. Me käisime protokolliga, side, meie tartus olime ka muidugi muljed muretsesime endale ka muude selle modemi üritasime Tartust Helsingisse helistada, aga Tartus ei olnud 80. aasta olümpiamängude jah, isegi mitte purjeregatti. Ja, ja meie telefoniliinid ei pidanud kaeglasse moode meid vastu. Siis me võtsime kätte ja üritasime Tallinnasse helistada. Ja kuna me ei jõudnud neid kiireid moslemeid osta, siis me ostsime mingi sisemise odava. Ja aitäh Mati kilbile, kes tol ajal oli matemaatikateaduskonna. 

Dekaan ja need mingid pisikesed valuutasummad, sest noh, kust sai kaheksakümnendatel ikka midagi, ostad valuutapoest ja muretses meile modemi. Ja siis siis meil käiski niimoodi, et meie helistasime Tallinnasse, Tallinn uudsopeega helistas, noh, alguses kaks korda nädalas, siis kolm korda nädalas, siis iga päev siis ma ei tea, mitu korda päevas, kuna mahud läksid järjest suuremaks. Helsingisse. Helsingist alates oli juba päris päris korralik internet olemas. 

Kõneleja 3
Aga mille külge need keskpolitseist tulid? 

Kõneleja 1
Tähendab, vahepeal toimus see Meil pandi ju satelliitside ülesse ja see on siis tänu Rootsi kuninglikule akadeemia suurusse fondil ja nad said nii palju siis raha kokku, et need otsa otsa seadmed kahes eksemplaris, sest noh, Tartu-Tallinna vahet ei jõua ju keegi ära kakelda. Nii et Tartus Tähetorni otsas oli siis Teleiks satelliidi, see vastuvõtu ja, ja Tallinnas konna Lippmaa instituut oli ikka endiselt veel Teaduste Akadeemia raamatukogu all sisuliselt raamatukogule teisel korrusel, nemad esimesel siis sinna katusele sai siis teine satelliitseade. 

Orientiiriti alguses vist valele satelliidile, meie seal kaasa ei mänginud, see oli puhtalt rootsipoolse otsa tegevus. Siis tekkis küsimus, et kes selle järgmise inseneri kinni maksab, kes tuleb ja õige peale seab. Aga siis tuli Rootsi kuninga visiit Tartusse. Ja loomulikult. Ma isegi ei mäleta, mis firma see oli, tahtis ta näidata, et nemad on kõikjal. Et Rootsi kuningas saab võtta telefonitoru rektori kabinetis Tartu Ülikooli rektori kabinetis ja ühenduda kohe oma koduga. 

Rootsis ja siis nad saatsid selle meheks õige satelliidi peale pani, nii et Rootsi kuninga visiidist alates on siis meil oli meil siis 64 kiloside satelliit üle satelliidi. Sel hetkel oli muidugi Tallinn-Tartu side väga huvitav. Tallinnast Tartusse, nagu me teame, on 285 kilomeetrit. Kui nüüd teistpidi on alati rohkem olnud. Aga kui nüüd vaadata, missuguse tee pidi läbima elektronkiri selleks et jõudu Tartust Tallinnasse siis ta kõigepealt pidin minema, eks ole, noh, see Tartusse jõnks kuni sinna tähetorni siis tähetorni, sellest satelliidi pealt Kotefoossete tähendab kuningliku Rootsi tehnikaülikoolist Stockholmis sealt olles avastanud siis eelnumbriga, kas meil tol ajal oli, see tuli natuke hiljem? Meil esimesed aadressi tulid suuga muuseas. Ja siis siis sealt Kothoost üle satelliidi Tallinna tehnikaülikooli ja siis natuke veel Tallinnas. Nii et see kõik oli vist mu, ma kunagi vist arvutasin kokku üle 70000 kilomeetri kaugemale ei ole Tartu Tallinnast kunagi asetanud. Aga see läbiti õnneks valguse kiirusel. 

Kõneleja 3
Visioon tuli, et siukest asja üldse nagu vaja on, et milleks internet hea oli. 

Kõneleja 1
No tolleks ajaks me olime selle selle uudsepe öös sibisidega juba juba imeasju teinud. Ma näiteks õpetasin tollel hetkel internet tudengitele tudengitega, vaadake e-kirja teel oli võimalik igasuguseid asju saada. ECB Sid RF Tseesid rikast fokument, mis on, eksole internetti, dokumentide alusdokumendid, Internetidokumentatsiooni alusdokumendid, neid oli võimalik tellida e-kirjaga, siis me tudengitega tudengite arvestusülesanne oli mõni FC kohale meelitada. Ja siis meil oli ketta peal peaaegu et kogu internetidokumentatsioon ja üksikud neist isegi üritasime välja trükkida. Näiteks FC kaheksa, kaks, kaks või kaheksa kaheksa kaks numbrit ei püsi hästi meeles. Mis oli elektronkirjade aluseks. See dokument on väga huvitav oli neid uurida peale, kui me olime juba terve suure hulga listide ülemaailmsete listide liikmed. Informatsioon levis. Ja üleüldse 

Üleüldse oli internet tore asi. 

Kõneleja 2
Kui suur ja siis ütleks? 

Kõneleja 1
Siis sealt muidugi jah, kõik said aru, et meil on päris internetti ka vaja. Kuigi pange tähele, veebi ei olnud veel sündinud. 

Kõneleja 3
Oot, aga kuidas meil oli kohver? Ja, ja just lugesin artiklit, kuidas? Kohver oli ikka see õige intervjuu. 

Kõneleja 2
Beebivärk see on vale ja mingisugune korporatsioonide välja mõeldud. 

Kõneleja 1
Ja see on tserni selles keskuses välja mõeldud, kuna neid hakkas uuesti ära tüütama, see CERN-is toodetakse palju artikleid ja need artiklid on, eks ole, viitavad üksteisele. Ja siis siis kuidas sa saad aru artiklist, eks ole, siis sa pead neid viidatavaid artikleid ka lugema ja see on üks igavene tüütus, käia neid kuskiltki otsimas. Ja siis mõeldi välja World vaid veed, kus on artikkel ja kus on kohe ka, eks ole, need pildid seal sees, see tähendab, see nõudis juba graafilist brauserit ja kus on viited niimoodi, et sa saad lõksida nende peale ja siis tuleb järgmine artikkel kohale. 

Kõneleja 3
Kas või kuidas? 

Kaks eraldi küsimust, kuidas see Mobeeb jõudis Eestisse Tartusse, täpsemalt ja b kust tuli mõte, et võiks hakata inimestele õpetama, kuidas seda veebi teha? 

Kõneleja 1
Need mõtted tulid peaaegu et korraga, aga kuidas Peeb jõudis Eestisse, minge küsige Morrektiitsugerest. Marek Tiits jälle ja tema töötas tol ajal. 

Tartu Ülikooli raamatukogus, kus oli jälle kuskiltki päranduseks kingitusena saadud mingisugune arvud teed ja ma ei mäleta, mis arvutis oli, aga sellel oli, sellel oli võimalik veebiserver peale panna, sellepärast vabandage mind väga, veebiserverit töötavad juuniks, siis. 

Ja Eesti esimese veebiserveri panin püsti Marek Tiits ja siis selles kursuses, mis vist kandis juba tol ajal pealkirja informaatika didaktika ja kus ma igasuguseid uusi asju, kaasa arvatud internet, üritasin lugeda ja ma usun, et selle kursuse nimi stega mulle internetikursust ei olnud. Mul oli informaatika didaktika kursus, kus me siis üritasime ka nüüd Errefftzeesid seal tagasi saada ja kätte saada ja nii edasi ja nii edasi. Ma kutsusin Marrektiitsu, et ta siis räägiks meile natukene nii kohvrist kui ka veebist. 

Marek Tiits, kes praegu kindlasti on üks parimaid lektoreid üldse tol ajal oli teise kursuse tudeng. Ja kui ma pärast küsisin oma tudengitelt, kellele ta siis sellest veebist rääkis et saite kõik aru, mis ta rääkis, sest tudengid vastasid ja vot ei tea, kui enne ei oleks midagi teadnud, siis vist ei oleks aru saanud. Aga kuna me enne ka midagi teadsime, siis me saime teda üsna hästi jälgida. Nii et nii et esimese serveri au on, on Marek Did sul aga siis kuskilgi, kunagi oli mingi sanspark täis ka Eesti biokeskuses. Ja seal peal ma võtsin oma tudengid üldse tudengitega igasuguseid. 

Kõneleja 3
Tudengitele ka ütlen oma mälestuste järgi. 

Kõneleja 1
Ja mingil hetkel ma saatsin oma, ma ei mäleta, mis kursuse võib-olla needsamad informaatika didaktika tudengid ülikooli peale, et hankige, ülikooli teadus, kui nii igaüks iseteaduskonna edasi hankige igasugust informatsiooni, mis teil õnnestub kätte saada selle samusegi teaduskonna allüksusi, loetavaid aineid ja mida iganes. Ja siis me nõelusime nendest sellise toreda ülikooli veebi kokku ja siis meil jätkus nahaalsust kutsuda kohale rektor, jagaks prorektorit ja paluda neil istuda arvuti taha ja siis vaadata, kuidas Tartu ülikool veebis välja näeb. 

Nägi välja niisugune, saad aru, niisugune, ma usun, nii nagu Eesti metsad praegu välja näevad, et noorendik ja lageraie ja siis mingisugune vana tükk ja nii edasi, see tähendab seda, et ta oli väga lapiline kuna keskust, mida kätte sai, kellel olid tuttavad, kus teaduskonnas ja nii edasi kindlasti oli asi tükati väga halvasti kajastatud. Aga muidugi kujunduse peale, noh, ikkagi Maida teaduskonna õeldi kujunduse peale väga palju auru ei raisanud, aga igal juhul said rektorid teada, mis asi on veeb, kuigi meil vist tol hetkel oli näppudel üles lugeda, mitu veebiserverit meil Eestis üldse tol hetkel oli ja siis ja siis, siis nad võtsid asja üle. Siis nad hakkasid päris päris päris ülikooli veebil. 

Kõneleja 3
Kiituseks öelda. 

Kõneleja 1
Ei, tead, rektorid on meil alati olnud suhteliselt taibukad. Nii palju kui mina neid näinud olen. 

Kõneleja 3
Sellist ideaalis peaks olema nagu nuga ometi eeldusi. 

Kõneleja 1
Ja igal juhul jah. Ma usun, et otse rumalusega ei ole hiilanud ükski Tartu Ülikooli rektor, nii et minu silmis olnud olnud kõik suhteliselt nutikad inimesed olnud. 

Kõneleja 3
Enne kui tõmbame joone alla, siis mul on üks niisugune abstraktne küsimus, et ma ei teagi, kas sellel ongi hea vastus, aga. 

Sinu käest saab selle, ühesõnaga, kui keegi kuskil nagu teab, siis tõenäoselt sina tead kõige täpsemalt seda vastust. 

Mis osa sellest kogu sellest asjast, mis on meid toonud sellest kuskil 70.-te juurest siiamaani välja? Kogu see IT-tööstus ja kõik see, et me sealt Liivi tänavalt oleme nüüd jõudnud siia, deltas kõik see nagu areng ja kogu see värk mis professor Kaasiku nihukesest bussimisest nagu pihta hakkas. Kui palju sellest on Niukest nagu mäetipul kaugusse vaadates nagu püsti pandud visiooni ja kui palju sellest on lihtsalt, et nagu teeme ägedaid asju nagu järgmise kahe nädala jooksul. 

Kõneleja 1
Osakaalud, eks ma seda jagada ei jõua, aga näiteks internetikoolitusõpetajat, et selle, see oli küll see teeme kahe nädalaga neidsamuseid, kihvte asju, sellepärast et me tegime infopäevi, et kuna Eestis internet arenes seal pärast seda, kui, kui, kui see päris internet siia kohale jõudis ikka ikka väga penikoormasaabastega siis siis me otsustasime, et me hoiame ka õpetajaid kursis ja igasuguste koolituste käigus tegime siis suures ringauditooriumis tol ajal muuseas ülikooli küsinud auditoorium, et ei tasu raha meil ei oleks olnud. Ja siis me rääkisime, eks ole, kuidas nüüd, mis asi on internet ja kuidas ta on arenenud ja nii edasi, siis mingil hetkel 

Tekkis Marek Tiits, sul üks, üks europrojekt, mille käigus ta sai 100 moodemit. 50 oli tal projekti jaoks vaja, aga 50 oli, võisime koolidele jagada, me ei hakanud neid niimodi loopima vaid me korraldasime vot selle. Nende. 

Sest saad minna või postmaasturite kursused, aga siis me tegime juba siukseid kombineeritud kursusi, et meil oli viis õpetajat rühma kellelegi, Me õpetasime hiire liigutamist kellelegi. Me õpetasime vist midagi programmeerimist, kellelegi õpetasime, kuidas modemit paika panna ja kuidas sinna teenused peale tõmmata. Kellelegi me õpetasime, mis on, veeb ja ja siis kuskilgi õpetasime, veel mingeid siis saad minna või jumal teab keda. Ja siis nendele postmaasturite kursusele me saime neid teha kaks korda niimoodi, et kõik kohale tulnud koolid, kes tahtsid edukalt kursuse, kahepäevane kursus, laupäev, pühapäev õpetajat koolituvad entusiasmist. Mingit tasu ei maksnud mingil korral, meil oli raha, me saime piletid kinni maksta, aga mitte alati. Ja siis me saime modemi kaasa anda. Meil puudus kontroll, mis nendest muuseumitest pärast sai. Aga kui 95. aastal meil nagu enam ei mahtunud nüüd õpetajat kuskilgi ära, isegi Vanemuise suurde auditooriumisse ei kippunud ära mahtuma ja meie võhm hakkas otsa saama. Ma käisin tuis pedagoogika teadusliku uurimise instituut tudis, kes organiseeris õpetajate koolitust. Käisin küsimus, eks ole, et kas nad meie kursustele ei taha raha anda? Ja rääkisin, et, et telekommunikatsioon tuleb kohe ja ja siis siis vastab ülemus, kelle nime ma kahjuks ei mäleta. Ütles mulle, mis asi teil telle kommunikatsioon, see asi ei tule eesti kooli mitte kunagi. No ma panin suu kinni ja jätsin ütlemata teate, veel 50 kooli on juba moodemiga ühendatud. Ma keerasin otsa ringi, tulin sealt välja, aga sain aru, et sealt august raha ei tule. Sorose fond oli see, mis meile pärast igasugusteks üritusteks natuke raha andis. 

Avatud Eesti fond. Ja siis, ja siis, siis me tegimegi ilma rahata, see tähendab keegi Eeennetist oli Enok Sein või keegi algus, oli Marrektid, seal siis mu tudengid. Ja Tartu ülikool, nagu ma ütlesin, oli aeg, kui Tartu ülikooli küsinud auditooriumide ja arvutiklasside kasutamise eest tasu, sest nagu ma ütlesin, raha meil ei olnud. Kõik, me tegime seda puhtast entusiasmist ja, ja ja ka esimesed e-kursused tegime puhtast entusiasmist. Me lihtsalt ei jõudnud neid suuri kahepäevaseid kursusi enam teha. Ja siis me istusime. 

Terje Tuisuga kahekesi kokku ja mõtlesime, et aga teeks nüüd seda õige. 

Muuseumid on ju olemas, e-kirju nad saavad, üks inimene on koolis, kes oskab modemi käima panna. Teeme siis korjame tema ümber, viis õpetajat, teeme neile koolituse. Esimesel koolitusel me unustasime piirarvu panemata. Andsime õpetajatele teada, et niisugune koolitus tuleb registreeruge ja registreeruge, noh, just see inimene, eks ole, kes modemiga hakkama saab. Registreerige oma kool. Ja me mõtlesime, et nagu viis kooli tuleb, on ikka jube hästi. Kui me jaole saime, panime mingisugused Aktiivse emaili aadressi koonal registreeruma peas ja siis meil oli nii kiirem, ei käinud seda vaatamas vaatamas käisime siis oli seal juba 20 üle 20 kooli ja siis me mõtlesime, et oot, aga mis vahet seal on? Mõningad asjad tuli ära muuta, Ta on sellest, kui me mõtlesime, eks ole, viis kooligast viis inimest, siis nad võivad kõik meile oma elu esimese kirja saata ja me saame kõigile individuaalselt vastata. Kui neid kool ja pärast oli 50 ja osavõtjaid 400, siis me saime aru, et Meie isiklikult igalühel individuaalselt kirju ei kirjuta. Aga siis, siis, siis me lasime neil omavahel, panime nad paar. Panime paari niimoodi, et nad ei oleks üksteise poole külla saanud, väga lihtsalt sõita. Et 50 kilti pidi vähemasti vahet olema ja nad pidid sama aineõpetajad olema. See oli üks igavene paaritaminema, mäletan. 

Terjega sõime. 

Kõneleja 3
Mulle nii meeldib, et see on täpselt see nagu programm, programmeerija lähenemine ülesandele, paneme baari ja ei tohi olla lähedal ja kuidas. 

Kõneleja 2
Sama aine siis ma ise on see üsna loogiliselt tuleb mõelda, eks ole, programmeerimisega. 

Kõneleja 1
Peab ka loogiliselt mõtlema. Loogiline mõtlemine tuleb elus õige mitme koha peal kasuks. 

Kõneleja 3
Vot, need on vaat-vaat, need on kuldsed sõnad, millega võikski lõpetada. Aga mul on üks küsimus veel ja see küsimus on see, millega professor Villems Anevilemist täidab oma aega. Kaunistage moega praegu. 

Kõneleja 1
Aeg esiteks, nagu te kuulsite, ma tulin just usast meie kahjuks minu need tuttavad, kellega ma valas vanasti sain tihedalt läbi käima käia ja kes olid Moskva nii-öelda erinevates instituutides ja ülikoolides need ei ole enam Moskvas. Need on Californias. Ja nüüd ma olen siis avastanud, et California külastamise kõige meeldivam aeg on jaanuari lõpp-veebruari algus mis just sobib mulle, sest siis on ülikoolivaheaeg. Ja nüüd ma olen neli aastat vist järjest käinud oma sõpradel külas. 

Üks parimaid sõbrannasid elab seal. Ja ma veedan tema juures ja nüüd nüüd mul õnnestus seal veeta kaks nädalat ja siis ta mind ära. Saatejuht ütles, et nüüd sa oled aru saanud kaks nädalat on õige aeg, mitte üks nädal. Nii et järgmine kord tuleb kaheks nädalaks. Nii et jaanuari lõpp-veebruari algusvõtjate mind alati leida. Californiast Palo Aaltost. 

California kevadest, kus nii need kohalikud kui ka sissetoodud taimed nagu näiteks lüpsipuud, õitsevad. 

Kõneleja 3
Hurmav lisaks sellele, et tegemist on Palo Alto 

Kõneleja 1
Ei hakka, mis on eksole, Stanfordi kodulinn ja kus Palo Alto ja ookeani vahel on toredad mäed. 

Ja seal on siis ka päris ookean teispool mägesid. Mäe otsas on ka tore käia. Üks sõber viis Sanhose juures mäe otsa, kust me oleksime pidanud nägema ühele poole? San Franciscosse. Sagaruse asub Sanhose juures, nüüd seal on väga pikk vahemaa vahepeal ja teist poolt siis oleks näinud. Sanhozeetsed on ikka saanud seest natuke põhja pool ja ja, ja, ja siis oli piimjas udu ja vihmapilved. Ja siis korraks tuli tuul ja lõuna poole nägime siis seda vaadet, mida pidin nägema. Nii et jah, väga tore on reisida. Aga ma siin töötan veel küll tunnitasu alusel ja loen oma armastatud andmebaaside kursust. Kolmes versioonis. 


\chapter{Kokkuvõte}
Kuna tegu on inimeste endi lugudega on neid raske kuidagi üheselt kokku võtta: kõik lood on unikaalsed, need põimuvad, segunevad kummalistel ja vahel ebaloogilistel viisidel, viivad kuskilt kuhugi ja igasugune katse midagi üldistada teeb lugudele ja inimestele ülekohut. 

Mõnda  torkab siiski silma. 

Kõigepealt see kummaline tõmme, mis arvutitel inimeste suhtes oli. Seejuures on huvitav, et tegu ei ole lihtsalt tehnikahuviliste noorte huviga tehnika vastu. Pigem vastupidi: mitmel juhul öeldakse, et arvutid üldiselt ja programmeerimine spetsiifiliselt olid pigem vahendid millegi muu saavutamiseks, kui eesmärk iseeneses. Ka täiesti teiste huvidega (Jaanuse ja Tarvi puhul näitehuviga) inimesi tõmbas miskipärast tugevalt arvuti poole. Jaanuse kasutatud metafoor \enquote{lendamise trennist}, millest on võimatu niisama mööda minna, kajab igalt poolt vastu. 

Seejuures on see miski, mida arvutite abil saavutada, tugevalt humanistlik, üldinimlik, ja ehk seletab mõnel määral meie toonase arvuti-kogukonna teket. Ahti sõnastab seda kui sarnaselt mõtlevate noorte inimeste püüet koos, üksteisele toetudes, inimeseks saada. Priit ja Jaan aga ütlevad, et nende jaoks oli võluv asjaolu, et kõik, mis on võimalik inimese peas, on võimalik ka arvutis. Eks iga teismeline on kogenud frustratsiooni oma võimetuse üle viia ellu oma suurepäraseid ideid. Ühtäkki aga asendus kontrolli puudumine  täiusliku kontrolliga arvuti üle koos piiramatu vabadusega suhelda teiste omasugustega.  Ja mis võiks olla veel paeluvam, kui võimalus oma unistusi koos teistega ellu viia?

Just koos. Võiks ju arvata, et arvuti-inimesed tegelevad pigem arvutite, kui inimestega, kuid koos tegutsemine ja võimalus suhelda teiste omasugustega on oluline teema pea kõigis lugudes. Üksteiselt õpitakse, saadakse abi. Koos tehakse suuri asju ja ühel või teisel moel jookseb rõõm headest kaasteelistest läbi enamusest lugudest. Kindlasti on ka rivaalitsemist, tülisid. Tallinna ja Tartu asetsesid mingil hetkel hea põhjusega üksteisest 70 000 kilomeetri kaugusel\sidenote{Vt. lk. \pageref{sisu!70k}}. Siiski domineerib arusaam, et oluline on olla osa kogukonnast ja et kogukond toimib vaid kõigi osapoolte heast tahtest. Skype lugugi on ju vaadeldav kui lugu sõprusest.

Tugev tõmme arvuti poole võib  aga ilma sobiva keskkonnata lihtsasti vaid platooniliseks igatsuseks jääda. Lugudude alusel  võis see keskkond võtta mitmeid vorme näiteks mitmel puhul maagilise paigana mainitud kooli raadioruumi või vanemate arvutitega seotud töökoha näol ning vahel oli kodus olemas elektroonika-huvi. Samas on ka näiteid, kui inimene ületab teel arvutiteni hoomamatuid takistusi ning jõuab kaugele. Ehk, keskkond kahtlemata toetab arvuti-huvi kuid ei ole ilmtingimata vajalik.

Lugedes lugusid arvuti juurde jõudmisest ja nende juurde jäämisest, torkab silma tänasest dramaatiliselt eriline suhe nendega. Juba ammu ei ole arvuti ja internet asjad, millele ligipääs on probleemiks. Kuid toona olid arvuti ja temal toimiv tarkvara väga lihtne, täna on nii arvuti kui tarkvara ühele inimesele terviklikuks mõistmiseks selgelt liiga keerulised. Nii näiteks Arne kui Meelis ütlevad aga, et nad said oma arvutist lõpuni aru: BASICu detailidest kuni riistvarani välja. Ühelt poolt andis see põhimõtteline erinevus kogemuse kontrollist ja teisalt saavutuselamuse. Tihti oli koolipoisil puht praktiliselt vaja luua olemasolevaga samaväärset või isegi paremat tarkvara. Nii Jaan kui Andres kirjutasid  toimiva ja kasuliku tekstideraktori\sidenote{Sama lugu on olnud mujalgi (\url{https://corecursive.com/058-brian-kernighan-unix-bell-labs/}), arvutite algusaegadel kulus väga palju auru võimaldamaks arvutisse teksti sisestada. Donald Knuth ja tema \LaTeX, tänu millele ka see raamat sünnib, lahendab samuti tekstiga seotud probleeme.}, sest seda oli vaja. Täna ei ole sellisteks ettevõtmisteks ei praktilist vajadust ega ka sisulist võimalust. 

Need lihtsamad masinad paigutusid mõnes mõttes märksa lihtsamasse sotsiaalsesse konteksti, kus segavaid faktoreid oli vähe ning võimalusi keskendumiseks palju. Jah, kindlasti on teatud vanuses noor inimene juba piisavalt nutikas huvitavateks programmeerimisülesanneteks kuid veel mitte takerdunud täiskasvanu-ellu. Kuid teadlik keskendumine on siiski nii mitmeski loos läbivaks teemaks. Ja on selge, et tänases kommunikatsioonile vaikimisi avatud keskkonnas nõuab keskendumine teistsuguseid ning kindlamaid oskusi kui toonases suletud kontekstis. 

Aruvtite ja tarkvara lihtsus võimaldas kindlasti luua väga kiiresti väga kasulikku tarkvara. Jaani tekstiredaktorit sai juba mainitud, aga Masti ja Marguse kahe kuuga kirjutatud modemipank oleks samuti tänapäeval küllalt ennekuulmatu asi. Teisalt aga jookseb juttudest läbi terviku tajumise teema, mis tänaste arvutite puhul on raskem. Tõnis ja Tõnu mainivad, kuidas nad ei saa keerulistest asjadest aru, ning kui oluline on võime taandada keeruline probleem lihtsamale kujule. Sellist hoomatavat tervikpilti arvuti ja arvutivõrgu toimimisest on lihtsamate arvutite puhul kindlasti suhteliselt lihtsam luua. Samas jääb loodud mudel aga adekvaatseks ka keerukamate süsteemide puhul: Tõnul ei ole probleem tegelda mikroelektroonikaga ja Vilve ehitab ülikeerulisi finantssüsteeme, sest neil on olemas lihtsate toimivate süsteemidest pärinev toimiv mõttemudel.

Kujutage endale ette, et teil on töö juures ülemuse kabinetis umbes pool miljonit eurot maksev aparaat. Ja teie varateismeline laps avaldab soovi selle aparaadiga veidi mängida. Kõlab hullumeelselt? Ometi toimiti kaheksakümnendatel täpselt nii kõikvõimalikes asutustes üle Eesti lubades kõikvõimalikke jõnglasi toonases mõistes hirmkalleid arvuteid näppima. Veelgi enam, sagedasti võeti rüblik lausa palgale, kuna osundus, et ta suudab arvutist üle käia (sest need olid suhteliselt lihtsad!) ning temast on kasu. Mõnda sellist motiivi sisaldab peaaegu iga ära toodud lugu.

Ma usun, et selline usaldus inimeste vahel, kes saavad aru probleemdiest ja nende vahel, kes saavad aru lahendustest on Eesti IT eduloos põhimõttelise tähtsusega. Mõlemad osapooled ju mõistavad, et nende huvides ei ole usaldust kuritarvitada: kui IT-kutti liiast nöökida, läheb ta mujale, ning kui öise mängimis-sessiooni tagajärjed päevatööd häirivad, võetakse võtmed käest. Sel samal vastastikusel usaldusel ja sellest tuleneval koostööl põhinevad nii ID-kaart kui X-Tee kui Hansapank kui kõik teised meie eduloo peatükid. Võib ju olla visioon teistmoodi pangast, aga tuleb uskuda, et IT-inimesed selle ka valmis ehitavad. Ükski riigiametnik ei ärka ühel hommikul mõttega XML-sõnumite liikumisest asutuste vahel. See on inseneri mõte ja vajab realiseerumiseks usku sedalaadi mõtete kasulikkusse. Omavahelised usalduslikud suhted olid kindlasti olulised ka kogukonna sees, kus suhteliselt väikesearvuline seltskond üksteist vähemalt nime pidi tundis ning \enquote{letihinnast ikka allahindlust tegi}. 

Usaldusel on kindlasti ka teine pool. Enamus siin raamatus toodud lugudest oleksid oluliselt lühemad, kui toona oleks rakendatud tänapäevases mõistes infoturvet. Kindlasti oleksid suured tükid meie IT-edulugu olemata, kui tarkvarapiraatlusele oleks vaadatud samamoodi, kui praegu. Ometi ei kosta lugudest usalduse kuritarvitamist, pigem räägitakse üle võetud masinate paikamisest ja omanikule tagastamisest. Samamoodi tekib ilmselt küsimus, et kui legaalne oleks toonane suhteliselt kinnise seltskonna \enquote{käsi peseb kätt} lähenemine riigi- ja erasektori piiril tänase hankeregulatsiooni kontekstis. Kuid ka siin kostab pigem lugusid riigi raha eest võimalikult hea tulemuse toomisest (Tarvi ja sidemastide lugu, näiteks), kui seitsme naha koorimisest. 

Hea küll. Maagiline kast tõmbab maabilise jõuga noore inimest enda juurde. Kuid mida, kohale jõudnuna, selle kastiga ette võtta? Kust tulevad selleks vajalikud oskused? Läbivaks jooneks on siin selgelt ise õppimine. Seejuures on tähelepanuväärne, et institusionaliseeritud õppimist meenutatakse sisu mõttes kasulikuna pigem harva kuid vaimsuse, seltskonna ja kultuuri mõttes valgustavana pigem sageli. On üksikuid erandeid, nagu Ahti ja Vilve, kuid reeglina inimesed ei oska vastata, kuidas nad programmeerima või elektroonikaga tegelema õppisid. Vastupidiselt tänasele, kus tundub suund olevat võimalikult paljude inimeste programmeerima õpetamisele, võtavad toonase suhtumise ehk kenasti kokku Tõnise ütlus, et \enquote{õppida tuleb raskeid asju, lihtsad tulevad iseenesest} ning Andruse oma, et  \enquote{programmeerimine sünnib vajadusest}. 

Õppimise meetodina räägitakse palju kas plokkskeemide abil või niisama paberil programmeerimisest ning ega perfokaartide abil programmi loomine sellest palju ei erinenud. Võib arvata, et ülimalt kõrge barjäär (arvutil kas puudus üldse interaktiivne konsool või oli ligipääs sellele väga piiratud) programmi sisestamisel sundis inimesi rohkem süvenema ning oma koodi läbi mõtlema viies programmeerimise kunsti metoodilisema ja sügavama mõistmiseni kui internetist koodijuppide kopeerimine annab.

Samas on roll tolles ebamäärases ja seletuseta õppeprotsessis väga selge ja suur roll kogukonnal. Reeglina puudus arvutite kohta ametlik kirjandus, teadmine levis folkloorina suust suhu, seda kasutati väärtusliku kaubana, seda jagati vaid valitutega, seda kirjutati märkmikesse. Kogukonnaks võis olla arvutiklassis kogunev poistekamp, mõnd arvutifirmat ümbritsev seltskond aga ka kooliklass, konkreetne institutsioon (KBFI) või lihtsalt füüsiline koht (Tartu Tähetorn). Anto ütleb mitmel puhul, et õppis üht või teist asja oma kooli poistelt. Siit koorub ehk ka võti mõistmaks, miks kujunes reeglina tugevalt introvertsest arvuti-rahvast Eestile hoo andnud tugev kogukond. Kuna suurem osa teadmisest tuli kellegi teise käest, muutus suur suhtevõrgustik isikliku arengu mõttes hädavajalikuks. Tippudel pidi olema väga hea suhtevõrgustik ja, kuna kõigil suhetel on vähemalt kaks otsa, aitasid nad arendada ka teiste kogukonna liikmete võrgustikku ning oskusi. Üllatavalt sageli näeme inimesi tegutsemas mingit sorti müügifunktsioonis, mis jällegi rõhutab sotsiaalsete oskuste olulisust. 

Kogukonnad võivad olla isetekkelised, kuid reeglina mainitakse mõnda konkreetset inimest, kelle ümber koonduti. Keegi ei meenuta, et nad oleksid Jaak Loonde käest midagi konkreetselt õppinud. Küll aga meenutatakse tema hindamatut rolli arvutiklasside tekitamisel ning, mis veelgi olulisem, sinna kogunenud seltskonna jaoks katalüsaatorina toimimisel. Lõvi, Antot, Annet, Tarmot ja teisi meenutatakse soojalt lisaks nende teadmistele ka kogukonna loojatena. 

Siin kaante vahel toodud lugudega tegeldes torkab silma tugev kallutatus eestlastest meesterahvaste poole. Kindlasti tuleneb see osalt ka autorist, kuid ka lugudes tegutsevad reeglina eesti keelt rääkivad mehed. Seejuures, kui mõni naisterahvas pildile ilmub, teeb ta seda võimsalt mõjutades paljusid ja liigutades metafoorseid mägesid (Vilve ja Anne) või olles peategelase oluliseks suunajaks (Anto ja Ahti emad). Eesti ja vene kogukondade omavaheline suhe on aga keerulisem. Ainsana loob nende vahele tõsisema silla Sergei, kelle jutust avaneb tõeline paralleelmaailm oma seltskondade ning õpetajatega a la Jaak Loonde. Vilve jutust läbi jooksev keerulise nimega Moskvale allunud asutus annab aimu, et eksisteeris ka terve eraldiseisev enamasti vene töökeelega arvutitega tegelevate organisatsioonide võrgustik. Mõlemal puhul tundub, et ühel või teisel põhjusel oleme jätnud suure hulga tarku inimesi tähelepanuta ja sellest on kahju.

Lisaks juba mainitud müügitööle on mõnevõrra üllatav meedia, sealhulgas trükimeedia, oluline roll inimeste lugudes. Pangandus kui Eesti tehnoloogia taimelava on teada-tuntud fenomen, meediast on selles kontekstis vähem räägitud. Ometi olid Kaspar, Peeter, Sten ja Taavi ja teised üht või teistpidi seotud pabermeediaga ning Kaspar toimetas teles. Ilmselt oli meedia valdkond, kuhu esimesel võimalusel liikus raha ning kus tehnoloogia abil oli võimalik saavutada oluline kvaliteedihüpe. Tehnoloogia aga tõmbas ligi teatud liiki inimesi.

Teiseks mõningaseks üllatuseks oli lugude tugev rahvusvaheline mõõde. Eesti NSV oli juba Nõukogude liidus teistest erinevas rollis Soome füüsilise läheduse ning telekomi infra suhtelise kvaliteedi tõttu. Meilt oli teatud tingimustel võimalik \enquote{päris} välismaale helistada ning too side oli tänu lühikesele distantsile isegi arvutisideks kasutatav! See võimaldas toetada side osas näiteks Leedut ning toimida teatud väravana kogu Nõukogude Liidu arvuti-rahva jaoks. Lugu Vladivostokist flopidega Tallinna tarkvara järele lennanud inimestest kõlab uskumatuna, kuid on ilmselt siiski tõsi. Seejuures saime ka meie olulist abi Soomest ja Rootsist. Rootsi loodi meie esimesed satelliitühendused, Soome aitas Tallinna Tehnikaülikoolil modemeid hankida, nii Soome kui Rootsi tehti tööd. Ja kindlasti tuleb ära märkida Ron Dwight, kelle rolli Eesti Fido kogukonna arengul ei saa kuidagi üle hinnata. 

Kuidas siis võtta kokku \verb|print(memcpy[])|? 

Kugi lugudest saab aimu, kuidas ja miks toonane arvuti-kogukond kujunes, ei saa me täit vastust küsimusele \enquote{miks just Eesti IT-edulugu?}. Kindlasti mängisid oma rolli suure visiooniga inimesed kuid palju oli ka pragmaatilist asjade ära tegemist ja ka lihtsat lustimist. Õpetajad olid olulised, kuid enamasti mitte teadmiste edastajatena. Akadeemilised asutused olid olulised, kuid pigem üksikute kogukonna-kollete võimaldajate kui institutsioonidena. Eraettevõtted olid olulised, kuid olles lugenud toonase kauboi-kapitalismi kohta, valdab aknast Eesti elu vaadates kergendustunne. 

Küll aga koondab see raamat 29 suurepärase inimese lood. Ja ehk on sellest praeguseks küllalt.



\chapter{Fido ja BBS 1990-1991}

Ajatõmmis Eesti Fidonetist, regiooni 2:49 \emph{nodelist} seisuga 28. september 1990:

\begin{table}
\label{sisu:nodelist}
\centering
\begin{tabular}{lrllrl}
Region & 49  & Estonia              & Andrus Suitsu\index[ppl]{Suitsu, Andrus} & 2400  & MNP  \\
Host   & 490 & NET Estonia           & Tarmo Ausing\index[ppl]{Ausing, Tarmo}  & 9600  & HST  \\
       & 1   & Hackers Night System  & Tarmo Ausing  & 9600  & HST  \\
       & 10  & P.O.Box Maximus       & Andrus Suitsu & 2400  & MNP  \\
       & 20  & Goodwin BBS           & Sulo Kallas\index[ppl]{Kallas, Sulo}   & 2400  & MNP  \\
       & 30  & Mail Shark            & Madis Kaal\index[ppl]{Kaal, Madis}    & 1200  & XX   \\
       & 40  & MamBox                & Tarmo Mamers\index[ppl]{Mamers, Tarmo}  & 19200 & PEP 
\end{tabular}
\end{table}

Ajatõmmis Eesti BBS-i maastikust seisuga 22. veebruar 1991. See konkreetne versioon (kuigi kindlasti leidub teisigi), tuli märkega:

\begin{verbatim}
Compiled   by   Serge   A.   Terekhov   --    with
participation of Yuri PQ  (2:5010/2),  Maxim Nikitin  &
Vitaly Klochko (2:5000/30)
\end{verbatim}

\begin{table}[ht]
\centering
\begin{tabular}{llp{2cm}p{3cm}p{4cm}}
%\toprule
BBS                            & Fido & Modem              & Saadavus (nädala sees/nädala lõpp või üldine) & SysOp                       \\
\midrule
Eesti \#1                      &      & 9600/MNP           & 24                                            & Lembit Pirn\index[ppl]{Pirn, Lembit}                 \\
Flying Disks BBS               & +    & 2400/MNP           &                                               & Margus Sutt\index[ppl]{Sutt, Margus}                 \\
Goodwin BBS                    & +    & 2400/MNP           & 24                                            & Sulo Kallas\index[ppl]{Kallas, Sulo}\index[ppl]{Marvet, Peeter}, Peeter Marvet  \\
Great White of Kopli           & +    & 2400               &                                               & Urmet Jänes\index[ppl]{Jänes, Urmet}                 \\
Hacker's Inn                   &      &                    &                                               &                             \\
Hacker's Night System          & +    & 9600/USR, 2400/MNP & 18-08/24                                      & Tarmo Ausing\index[ppl]{Ausing, Tarmo}, Tõnis Reimo\index[ppl]{Reimo, Tõnis}   \\
Kroon                          &      &                    &                                               &                             \\
Lion's Cave                    & +    & 9600/HST           &                                               & Andres Lepp\index[ppl]{Lepp, Andres}                 \\
Mailbox for citizens of galaxy & +    & 1200               & 21-0230                                       & Madis Kaal\index[ppl]{Kaal, Madis}                  \\
MamBox                         & +    & 19200/PEP          & 20-08/24                                      & Tarmo Mamers\index[ppl]{Mamers, Tarmo}                \\
Micro                          &      & 2400               & 20-08/24                                      & Jan Kuman\index[ppl]{Kuman, Jan}                   \\
New Age System                 & +    & 2400               & 18-09/24                                      & Tanel Raja\index[ppl]{Raja, Tanel}                  \\
New Barbarian                  &      & 2400               & 23-11/24                                      & Yura Zaitsev\index[ppl]{Zaitsev, Yura}                \\
P.O. Box Maximus               & +    & 2400/MNP           & 21-11/20-11                                   & Andres Suitsu\index[ppl]{Suitsu, Andrus},Tarmo Soodla\index[ppl]{Soodla, Tarmo}  \\
The MESO                       & +    & 2400/MNP           & 19-08/24                                      & Viljo Allik\index[ppl]{Allik, Viljo}                 \\
PaPer                          & +    & 1200               & 20-09/24                                      & Taavi Talvik\index[ppl]{Talvik, Taavi}               \\
\bottomrule
\end{tabular}
\end{table}

%%
% The back matter contains appendices, bibliographies, indices, glossaries, etc.



\backmatter

\addcontentsline{toc}{chapter}{Nimeloend}
\printindex[ppl]
\addcontentsline{toc}{chapter}{Indeks}
\printindex


\end{document}

